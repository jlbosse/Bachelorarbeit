%!TEX root = main.tex
%% !TEX spellcheck=de_DE
%%%%%%%%%%%%%%%%%%%%%%%%%%%%%%%%%%%%%%%%%%%%%%%%%%%%%%%%%%%%%%%%%%%%%%%%%%%%%%%
% % Berechnen der Wellenfrontmenge von Delta_m
%%%%%%%%%%%%%%%%%%%%%%%%%%%%%%%%%%%%%%%%%%%%%%%%%%%%%%%%%%%%%%%%%%%%%%%%%%%%%%%%

\section{\texorpdfstring{Die Wellenfrontmenge von $\Delta_m$}
        {Die Wellenfrontmenge von Delta m}} % (fold)
\label{sec:die_wellenfrontmenge_von_delta_m}

Die massive Zweipunktfunktion ist die Fouriertransformierte der 1$m$-Massenschale positiver Energie (vgl. \textcite{Schwartz2014}, 24.69):

\begin{equation}
    \Delta_m (t,x) = \int \delta (\omega^2-k^2-m^2)
                    \Theta(\omega)e^{-i\omega t + i k x} \d \omega \d k
\end{equation}

woraus sich $\rwhat \Delta_m$ direkt als $\delta (\omega^2-k^2-m^2)\Theta(\omega)$
ablesen lässt.

\begin{figure}[h]
\centering
%% Creator: Matplotlib, PGF backend
%%
%% To include the figure in your LaTeX document, write
%%   \input{<filename>.pgf}
%%
%% Make sure the required packages are loaded in your preamble
%%   \usepackage{pgf}
%%
%% Figures using additional raster images can only be included by \input if
%% they are in the same directory as the main LaTeX file. For loading figures
%% from other directories you can use the `import` package
%%   \usepackage{import}
%% and then include the figures with
%%   \import{<path to file>}{<filename>.pgf}
%%
%% Matplotlib used the following preamble
%%   \usepackage[utf8x]{inputenc}
%%   \usepackage[T1]{fontenc}
%%   \usepackage{amssymb}
%%
\begingroup%
\makeatletter%
\begin{pgfpicture}%
\pgfpathrectangle{\pgfpointorigin}{\pgfqpoint{4.000000in}{2.350000in}}%
\pgfusepath{use as bounding box, clip}%
\begin{pgfscope}%
\pgfsetbuttcap%
\pgfsetmiterjoin%
\definecolor{currentfill}{rgb}{1.000000,1.000000,1.000000}%
\pgfsetfillcolor{currentfill}%
\pgfsetlinewidth{0.000000pt}%
\definecolor{currentstroke}{rgb}{1.000000,1.000000,1.000000}%
\pgfsetstrokecolor{currentstroke}%
\pgfsetdash{}{0pt}%
\pgfpathmoveto{\pgfqpoint{0.000000in}{0.000000in}}%
\pgfpathlineto{\pgfqpoint{4.000000in}{0.000000in}}%
\pgfpathlineto{\pgfqpoint{4.000000in}{2.350000in}}%
\pgfpathlineto{\pgfqpoint{0.000000in}{2.350000in}}%
\pgfpathclose%
\pgfusepath{fill}%
\end{pgfscope}%
\begin{pgfscope}%
\pgfsetbuttcap%
\pgfsetmiterjoin%
\definecolor{currentfill}{rgb}{1.000000,1.000000,1.000000}%
\pgfsetfillcolor{currentfill}%
\pgfsetlinewidth{0.000000pt}%
\definecolor{currentstroke}{rgb}{0.000000,0.000000,0.000000}%
\pgfsetstrokecolor{currentstroke}%
\pgfsetstrokeopacity{0.000000}%
\pgfsetdash{}{0pt}%
\pgfpathmoveto{\pgfqpoint{0.198611in}{0.198611in}}%
\pgfpathlineto{\pgfqpoint{3.801389in}{0.198611in}}%
\pgfpathlineto{\pgfqpoint{3.801389in}{2.151389in}}%
\pgfpathlineto{\pgfqpoint{0.198611in}{2.151389in}}%
\pgfpathclose%
\pgfusepath{fill}%
\end{pgfscope}%
\begin{pgfscope}%
\pgfpathrectangle{\pgfqpoint{0.198611in}{0.198611in}}{\pgfqpoint{3.602778in}{1.952778in}} %
\pgfusepath{clip}%
\pgfsetbuttcap%
\pgfsetmiterjoin%
\definecolor{currentfill}{rgb}{0.501961,0.501961,0.501961}%
\pgfsetfillcolor{currentfill}%
\pgfsetfillopacity{0.500000}%
\pgfsetlinewidth{0.501875pt}%
\definecolor{currentstroke}{rgb}{0.501961,0.501961,0.501961}%
\pgfsetstrokecolor{currentstroke}%
\pgfsetstrokeopacity{0.500000}%
\pgfsetdash{}{0pt}%
\pgfpathmoveto{\pgfqpoint{1.837063in}{0.365312in}}%
\pgfpathlineto{\pgfqpoint{1.937764in}{0.365312in}}%
\pgfpathlineto{\pgfqpoint{1.751054in}{0.722527in}}%
\pgfpathlineto{\pgfqpoint{1.348251in}{0.722527in}}%
\pgfpathclose%
\pgfusepath{stroke,fill}%
\end{pgfscope}%
\begin{pgfscope}%
\pgfpathrectangle{\pgfqpoint{0.198611in}{0.198611in}}{\pgfqpoint{3.602778in}{1.952778in}} %
\pgfusepath{clip}%
\pgfsetbuttcap%
\pgfsetmiterjoin%
\definecolor{currentfill}{rgb}{0.501961,0.501961,0.501961}%
\pgfsetfillcolor{currentfill}%
\pgfsetfillopacity{0.500000}%
\pgfsetlinewidth{0.501875pt}%
\definecolor{currentstroke}{rgb}{0.501961,0.501961,0.501961}%
\pgfsetstrokecolor{currentstroke}%
\pgfsetstrokeopacity{0.500000}%
\pgfsetdash{}{0pt}%
\pgfpathmoveto{\pgfqpoint{1.324479in}{0.841599in}}%
\pgfpathlineto{\pgfqpoint{1.549653in}{0.841599in}}%
\pgfpathlineto{\pgfqpoint{0.198611in}{2.627676in}}%
\pgfpathlineto{\pgfqpoint{-0.702083in}{2.627676in}}%
\pgfpathclose%
\pgfusepath{stroke,fill}%
\end{pgfscope}%
\begin{pgfscope}%
\pgfpathrectangle{\pgfqpoint{0.198611in}{0.198611in}}{\pgfqpoint{3.602778in}{1.952778in}} %
\pgfusepath{clip}%
\pgfsetbuttcap%
\pgfsetmiterjoin%
\definecolor{currentfill}{rgb}{0.501961,0.501961,0.501961}%
\pgfsetfillcolor{currentfill}%
\pgfsetfillopacity{0.500000}%
\pgfsetlinewidth{0.501875pt}%
\definecolor{currentstroke}{rgb}{0.501961,0.501961,0.501961}%
\pgfsetstrokecolor{currentstroke}%
\pgfsetstrokeopacity{0.500000}%
\pgfsetdash{}{0pt}%
\pgfpathmoveto{\pgfqpoint{2.154174in}{0.612615in}}%
\pgfpathlineto{\pgfqpoint{2.330815in}{0.612615in}}%
\pgfpathlineto{\pgfqpoint{3.323260in}{1.711739in}}%
\pgfpathlineto{\pgfqpoint{2.616697in}{1.711739in}}%
\pgfpathclose%
\pgfusepath{stroke,fill}%
\end{pgfscope}%
\begin{pgfscope}%
\pgfpathrectangle{\pgfqpoint{0.198611in}{0.198611in}}{\pgfqpoint{3.602778in}{1.952778in}} %
\pgfusepath{clip}%
\pgfsetrectcap%
\pgfsetroundjoin%
\pgfsetlinewidth{0.501875pt}%
\definecolor{currentstroke}{rgb}{0.894118,0.101961,0.109804}%
\pgfsetstrokecolor{currentstroke}%
\pgfsetdash{}{0pt}%
\pgfpathmoveto{\pgfqpoint{0.199504in}{2.165278in}}%
\pgfpathlineto{\pgfqpoint{0.777952in}{1.560435in}}%
\pgfpathlineto{\pgfqpoint{1.121936in}{1.204929in}}%
\pgfpathlineto{\pgfqpoint{1.339189in}{0.984574in}}%
\pgfpathlineto{\pgfqpoint{1.484024in}{0.841636in}}%
\pgfpathlineto{\pgfqpoint{1.592651in}{0.738492in}}%
\pgfpathlineto{\pgfqpoint{1.665068in}{0.673073in}}%
\pgfpathlineto{\pgfqpoint{1.737486in}{0.612018in}}%
\pgfpathlineto{\pgfqpoint{1.791799in}{0.570581in}}%
\pgfpathlineto{\pgfqpoint{1.828008in}{0.545905in}}%
\pgfpathlineto{\pgfqpoint{1.864217in}{0.524331in}}%
\pgfpathlineto{\pgfqpoint{1.900426in}{0.506629in}}%
\pgfpathlineto{\pgfqpoint{1.936635in}{0.493633in}}%
\pgfpathlineto{\pgfqpoint{1.972843in}{0.486109in}}%
\pgfpathlineto{\pgfqpoint{2.009052in}{0.484576in}}%
\pgfpathlineto{\pgfqpoint{2.045261in}{0.489147in}}%
\pgfpathlineto{\pgfqpoint{2.081470in}{0.499491in}}%
\pgfpathlineto{\pgfqpoint{2.117679in}{0.514944in}}%
\pgfpathlineto{\pgfqpoint{2.153887in}{0.534685in}}%
\pgfpathlineto{\pgfqpoint{2.190096in}{0.557900in}}%
\pgfpathlineto{\pgfqpoint{2.244410in}{0.597706in}}%
\pgfpathlineto{\pgfqpoint{2.298723in}{0.641871in}}%
\pgfpathlineto{\pgfqpoint{2.371140in}{0.705351in}}%
\pgfpathlineto{\pgfqpoint{2.461662in}{0.789475in}}%
\pgfpathlineto{\pgfqpoint{2.570289in}{0.894690in}}%
\pgfpathlineto{\pgfqpoint{2.733229in}{1.057446in}}%
\pgfpathlineto{\pgfqpoint{2.950482in}{1.279293in}}%
\pgfpathlineto{\pgfqpoint{3.276361in}{1.616965in}}%
\pgfpathlineto{\pgfqpoint{3.783284in}{2.147217in}}%
\pgfpathlineto{\pgfqpoint{3.800496in}{2.165278in}}%
\pgfpathlineto{\pgfqpoint{3.800496in}{2.165278in}}%
\pgfusepath{stroke}%
\end{pgfscope}%
\begin{pgfscope}%
\pgfpathrectangle{\pgfqpoint{0.198611in}{0.198611in}}{\pgfqpoint{3.602778in}{1.952778in}} %
\pgfusepath{clip}%
\pgfsetbuttcap%
\pgfsetroundjoin%
\pgfsetlinewidth{0.501875pt}%
\definecolor{currentstroke}{rgb}{0.501961,0.501961,0.501961}%
\pgfsetstrokecolor{currentstroke}%
\pgfsetdash{{1.850000pt}{0.800000pt}}{0.000000pt}%
\pgfpathmoveto{\pgfqpoint{1.941833in}{0.184722in}}%
\pgfpathlineto{\pgfqpoint{3.801389in}{2.151389in}}%
\pgfpathlineto{\pgfqpoint{3.801389in}{2.151389in}}%
\pgfusepath{stroke}%
\end{pgfscope}%
\begin{pgfscope}%
\pgfpathrectangle{\pgfqpoint{0.198611in}{0.198611in}}{\pgfqpoint{3.602778in}{1.952778in}} %
\pgfusepath{clip}%
\pgfsetbuttcap%
\pgfsetroundjoin%
\pgfsetlinewidth{0.501875pt}%
\definecolor{currentstroke}{rgb}{0.501961,0.501961,0.501961}%
\pgfsetstrokecolor{currentstroke}%
\pgfsetdash{{1.850000pt}{0.800000pt}}{0.000000pt}%
\pgfpathmoveto{\pgfqpoint{0.198611in}{2.151389in}}%
\pgfpathlineto{\pgfqpoint{2.058167in}{0.184722in}}%
\pgfpathlineto{\pgfqpoint{2.058167in}{0.184722in}}%
\pgfusepath{stroke}%
\end{pgfscope}%
\begin{pgfscope}%
\pgfpathrectangle{\pgfqpoint{0.198611in}{0.198611in}}{\pgfqpoint{3.602778in}{1.952778in}} %
\pgfusepath{clip}%
\pgfsetbuttcap%
\pgfsetroundjoin%
\pgfsetlinewidth{0.501875pt}%
\definecolor{currentstroke}{rgb}{0.501961,0.501961,0.501961}%
\pgfsetstrokecolor{currentstroke}%
\pgfsetdash{{1.850000pt}{0.800000pt}}{0.000000pt}%
\pgfpathmoveto{\pgfqpoint{2.000000in}{0.484383in}}%
\pgfpathlineto{\pgfqpoint{2.396306in}{0.484383in}}%
\pgfusepath{stroke}%
\end{pgfscope}%
\begin{pgfscope}%
\pgfsetrectcap%
\pgfsetmiterjoin%
\pgfsetlinewidth{0.501875pt}%
\definecolor{currentstroke}{rgb}{0.000000,0.000000,0.000000}%
\pgfsetstrokecolor{currentstroke}%
\pgfsetdash{}{0pt}%
\pgfpathmoveto{\pgfqpoint{2.000000in}{0.198611in}}%
\pgfpathlineto{\pgfqpoint{2.000000in}{2.151389in}}%
\pgfusepath{stroke}%
\end{pgfscope}%
\begin{pgfscope}%
\pgfsetrectcap%
\pgfsetmiterjoin%
\pgfsetlinewidth{0.501875pt}%
\definecolor{currentstroke}{rgb}{0.000000,0.000000,0.000000}%
\pgfsetstrokecolor{currentstroke}%
\pgfsetdash{}{0pt}%
\pgfpathmoveto{\pgfqpoint{0.198611in}{0.246240in}}%
\pgfpathlineto{\pgfqpoint{3.801389in}{0.246240in}}%
\pgfusepath{stroke}%
\end{pgfscope}%
\begin{pgfscope}%
\pgfsetroundcap%
\pgfsetroundjoin%
\pgfsetlinewidth{0.501875pt}%
\definecolor{currentstroke}{rgb}{0.000000,0.000000,0.000000}%
\pgfsetstrokecolor{currentstroke}%
\pgfsetdash{}{0pt}%
\pgfpathmoveto{\pgfqpoint{1.382621in}{0.546431in}}%
\pgfpathquadraticcurveto{\pgfqpoint{1.527316in}{0.554339in}}{\pgfqpoint{1.664258in}{0.561824in}}%
\pgfusepath{stroke}%
\end{pgfscope}%
\begin{pgfscope}%
\pgfsetroundcap%
\pgfsetroundjoin%
\pgfsetlinewidth{0.501875pt}%
\definecolor{currentstroke}{rgb}{0.000000,0.000000,0.000000}%
\pgfsetstrokecolor{currentstroke}%
\pgfsetdash{}{0pt}%
\pgfpathmoveto{\pgfqpoint{1.607269in}{0.586528in}}%
\pgfpathlineto{\pgfqpoint{1.664258in}{0.561824in}}%
\pgfpathlineto{\pgfqpoint{1.610301in}{0.531056in}}%
\pgfusepath{stroke}%
\end{pgfscope}%
\begin{pgfscope}%
\pgftext[x=0.423785in,y=0.484383in,left,base]{\rmfamily\fontsize{10.000000}{12.000000}\selectfont a = 0.2, s = -1}%
\end{pgfscope}%
\begin{pgfscope}%
\pgfsetroundcap%
\pgfsetroundjoin%
\pgfsetlinewidth{0.501875pt}%
\definecolor{currentstroke}{rgb}{0.000000,0.000000,0.000000}%
\pgfsetstrokecolor{currentstroke}%
\pgfsetdash{}{0pt}%
\pgfpathmoveto{\pgfqpoint{0.818790in}{1.787660in}}%
\pgfpathquadraticcurveto{\pgfqpoint{0.672544in}{1.808780in}}{\pgfqpoint{0.533983in}{1.828789in}}%
\pgfusepath{stroke}%
\end{pgfscope}%
\begin{pgfscope}%
\pgfsetroundcap%
\pgfsetroundjoin%
\pgfsetlinewidth{0.501875pt}%
\definecolor{currentstroke}{rgb}{0.000000,0.000000,0.000000}%
\pgfsetstrokecolor{currentstroke}%
\pgfsetdash{}{0pt}%
\pgfpathmoveto{\pgfqpoint{0.584998in}{1.793356in}}%
\pgfpathlineto{\pgfqpoint{0.533983in}{1.828789in}}%
\pgfpathlineto{\pgfqpoint{0.592939in}{1.848341in}}%
\pgfusepath{stroke}%
\end{pgfscope}%
\begin{pgfscope}%
\pgftext[x=0.874132in,y=1.675102in,left,base]{\rmfamily\fontsize{10.000000}{12.000000}\selectfont a = 0.04, s = -1}%
\end{pgfscope}%
\begin{pgfscope}%
\pgftext[x=2.450347in,y=0.460569in,left,base]{\rmfamily\fontsize{10.000000}{12.000000}\selectfont \(\displaystyle \omega = m\)}%
\end{pgfscope}%
\begin{pgfscope}%
\pgftext[x=3.193420in,y=1.436958in,left,base]{\rmfamily\fontsize{10.000000}{12.000000}\selectfont \(\displaystyle supp~(\hat\Delta_m)\)}%
\end{pgfscope}%
\begin{pgfscope}%
\pgfsetroundcap%
\pgfsetroundjoin%
\pgfsetlinewidth{0.501875pt}%
\definecolor{currentstroke}{rgb}{0.000000,0.000000,0.000000}%
\pgfsetstrokecolor{currentstroke}%
\pgfsetdash{}{0pt}%
\pgfpathmoveto{\pgfqpoint{2.000000in}{2.157510in}}%
\pgfpathquadraticcurveto{\pgfqpoint{2.000000in}{2.158331in}}{\pgfqpoint{2.000000in}{2.151389in}}%
\pgfusepath{stroke}%
\end{pgfscope}%
\begin{pgfscope}%
\pgfsetroundcap%
\pgfsetroundjoin%
\pgfsetlinewidth{0.501875pt}%
\definecolor{currentstroke}{rgb}{0.000000,0.000000,0.000000}%
\pgfsetstrokecolor{currentstroke}%
\pgfsetdash{}{0pt}%
\pgfpathmoveto{\pgfqpoint{1.972222in}{2.101954in}}%
\pgfpathlineto{\pgfqpoint{2.000000in}{2.157510in}}%
\pgfpathlineto{\pgfqpoint{2.027778in}{2.101954in}}%
\pgfusepath{stroke}%
\end{pgfscope}%
\begin{pgfscope}%
\pgftext[x=2.000000in,y=2.220833in,,bottom]{\rmfamily\fontsize{10.000000}{12.000000}\selectfont \(\displaystyle \omega\)}%
\end{pgfscope}%
\begin{pgfscope}%
\pgfsetroundcap%
\pgfsetroundjoin%
\pgfsetlinewidth{0.501875pt}%
\definecolor{currentstroke}{rgb}{0.000000,0.000000,0.000000}%
\pgfsetstrokecolor{currentstroke}%
\pgfsetdash{}{0pt}%
\pgfpathmoveto{\pgfqpoint{3.807488in}{0.246240in}}%
\pgfpathquadraticcurveto{\pgfqpoint{3.808320in}{0.246240in}}{\pgfqpoint{3.801389in}{0.246240in}}%
\pgfusepath{stroke}%
\end{pgfscope}%
\begin{pgfscope}%
\pgfsetroundcap%
\pgfsetroundjoin%
\pgfsetlinewidth{0.501875pt}%
\definecolor{currentstroke}{rgb}{0.000000,0.000000,0.000000}%
\pgfsetstrokecolor{currentstroke}%
\pgfsetdash{}{0pt}%
\pgfpathmoveto{\pgfqpoint{3.751932in}{0.274018in}}%
\pgfpathlineto{\pgfqpoint{3.807488in}{0.246240in}}%
\pgfpathlineto{\pgfqpoint{3.751932in}{0.218462in}}%
\pgfusepath{stroke}%
\end{pgfscope}%
\begin{pgfscope}%
\pgftext[x=3.870833in,y=0.246240in,left,]{\rmfamily\fontsize{10.000000}{12.000000}\selectfont \(\displaystyle k\)}%
\end{pgfscope}%
\end{pgfpicture}%
\makeatother%
\endgroup%

\caption{Die Träger von $\hat\Delta_m$ und $\hat\psi_{ast}$. Es ist zu sehen, dass für $a \rightarrow 0$ und $s \neq \pm 1$ die Träger schließlich disjunkt sind}
\label{fig:delta_m}
\end{figure}

%%%%%%%%%%%%%%%%%%%%%%%%%%%%%%%%%%%%%%%%%%%%%%%%%%%%%%%%%%%%%%%%%%%%%%%%%%%%%%%
% % Teil 2
%%%%%%%%%%%%%%%%%%%%%%%%%%%%%%%%%%%%%%%%%%%%%%%%%%%%%%%%%%%%%%%%%%%%%%%%%%%%%%%
\subsubsection*{Fall $s \neq \pm 1$}
Hier gibt es nicht viel zu tun, denn für a klein genug gilt
$supp (\hat \Delta_m) \cap supp (\hat \psi_{ast}) = \varnothing$ wie man \cref{fig:delta_m} entnehmen kann.
Also gilt

\begin{dmath}
    \left\langle \psi_{ast}, \Delta_m\right\rangle
    = \left\langle \hat\psi_{ast}, \rwhat\Delta_m\right\rangle
    = 0 = O(a^k)~ \forall k  \condition{für $a$ klein genug}
    \label{eq:delta_m_s_neq1}
\end{dmath}

 Dies gilt für alle $(t', x') \in \mathbb{R}^2$

\todo{hier noch blöde Abschätzerei machen, warum das tatsächlich gilt, oder stehen lassen. Oder im Kapitel Shearlets ne Bemerkung machen, warum wir in immer engeren Kegeln landen?}


%%%%%%%%%%%%%%%%%%%%%%%%%%%%%%%%%%%%%%%%%%%%%%%%%%%%%%%%%%%%%%%%%%%%%%%%%%%%%%%
% % Teil3
%%%%%%%%%%%%%%%%%%%%%%%%%%%%%%%%%%%%%%%%%%%%%%%%%%%%%%%%%%%%%%%%%%%%%%%%%%%%%%%
\subsubsection*{Fall $s=1$}
\paragraph*{Intuition}
Für $s=1$ schneidet die Diagonale  $supp (\hat \psi_{ast})$ auf der ganzen Länge. Der Betrag von $\hat\psi_{ast}$ skaliert mit $a^{\frac{3}{4}}$ (vgl. \cref{eq:hat_psi_ast}) und die Länge von $supp (\hat \psi_{ast})$ entlang der Diagonalen mit $a^{-1}$ (vgl. \cref{eq:hat_psi_ast}). Also erwarten wir schlimmstenfalls $\left< \hat \psi_{a1t}, \hat\Delta_m\right> = O\left(a^-{\frac{1}{4}}\right)$. Aber nur wenn die Wellenfronten von $e^{-i\omega t'+i k x'}$ parallel zu der Singularität und damit der Diagonalen liegen. Andernfalls erwarten wir, dass die immer schneller werdenden Oszillationen der Phase sich gegenseitig auslöschen.

\paragraph*{Fleißige Analysis}
\begin{align}
    \left<\hat \psi_{a1t} ,\hat \Delta_m\right> &=
        a^{\frac{3}{4}} \int \hat \psi_1(a\omega)
        \hat\psi_2\left(a^{-\frac{1}{2}} \left(\tfrac{k}{\omega}-1\right)\right)
        \delta (\omega^2 - k^2 - m^2) \theta(\omega)
        e^{-i\omega t' + ikx'} d \omega \d k \nonumber \\[2ex]
        & \kern 2em \underline{\textrm{Nullstellen von $\delta$}:}
        \nonumber \\
        & \kern 2em \omega^2 - k^2 - m^2 = 0 \Leftrightarrow k = \pm \sqrt{\omega^2-m^2}
        \nonumber \\
        & \kern 2em \Rightarrow \frac{\d k}{\d \omega} = \frac{\omega}{\sqrt{\omega^2-m^2}}; \textrm{   wobei nur ,,+'' in $supp (\hat \psi_2)$ liegt}
        \nonumber \\[2ex]
        &= a^{\frac{3}{4}} \int \hat\psi_1(a \omega)
        \hat\psi_2\left(a^{-\frac{1}{2}} \left(\tfrac{\sqrt{\omega^2-m^2}}{\omega}-1\right)\right)
        e^{-i\omega t' + i \sqrt{\omega^2-m^2}x'}
        \d \omega \nonumber \\
        &= a^{\frac{3}{4}} a^{-1} \int \hat\psi_1(\omega)
        \hat\psi_2
        \underbrace{
        \left(
            a^{-\frac{1}{2}} \left(\tfrac{\sqrt{a\omega^2-m^2}}{\omega}-1\right)
        \right)}_{= \frac{a^{\frac{3}{2}}m^2}{2 \omega^2}
                  + O\left(a^{\frac{7}{2}}\right)}
        e^{-i\frac{\omega}{a} t' + i \sqrt{\frac{\omega^2}{a^2}-m^2}x'}
        \d \omega \nonumber
\end{align}
Der Integrand lässt sich nun durch $\hat \psi_1(\omega) \left\lVert \hat\psi_2\right\lVert_\infty$ majorisieren und wir dürfen Lebesgue verwenden um Integral und Grenzwert $a \rightarrow 0$ zu vertauschen

\begin{dmath}
\lim_{a \to o}
\left\langle\hat \psi_{a1t} ,\hat \Delta_m\right\rangle
= a^{-\frac{1}{4}} \int \hat \psi_1(\omega) \hat \psi_2 (0)
    e^{-i\omega \left(\frac{t'-x'}{a}\right)}
= a^{-\frac{1}{4}} \hat \psi_2 (0) \psi_1\left(\frac{t'-x'}{a}\right)
\\
 \sim O\left(a^{-\frac{1}{4}}\right) \condition{falls $x'=t'$}
\\
 \sim O\left(a^k\right) ~ \forall k  \condition{sonst}
\label{eq:delta_m_s=1}
\end{dmath}

Das analoge Ergebnis erhält man mit gleicher Rechnung auch für $s=-1$ und $t' = -x'$
Dies bestätigt das intuitiv erwartete Ergebnis. Fassen wir die Ergebnisse aus
 \cref{eq:delta_m_s_neq1,eq:delta_m_s=1} noch einmal tabellarisch zusammen:

\todo[color=green]{Statt $a^k$ einfach leer lassen, und nur nicht reguläre Punkte und Richtungen aufschreiben?}

\begin{table}[h]
\centering
\label{my-label}
\begin{tabular}{l|cccc}
        & \multicolumn{1}{l}{$(t', x') = (0, 0)$} & \multicolumn{1}{l}{$t' = x'$} & \multicolumn{1}{l}{$t' = -x'$} & \multicolumn{1}{l}{$t' \neq \pm x'$} \\ \hline
s = 1   & $a^{-\frac{1}{4}}$    & $a^{-\frac{1}{4}}$    & $a^k$  & $a^k$    \\
s = -1  & $a^{-\frac{1}{4}}$    & $a^k$    & $a^{-\frac{1}{4}}$  & $a^k$    \\
$s \neq \pm 1$  & $a^k$         & $a^k$    & $a^k$               & $a^k$    \\
\end{tabular}
\caption{Konvergenzordnung von $\mathcal{S}_{\Delta_m} (a,s,(t',x'))$ im Limit $a \rightarrow 0$ für alle interessanten Kombinationen von $s$ und $(t',x')$}
\end{table}

% section die_wellenfrontmenge_von_ (end)
