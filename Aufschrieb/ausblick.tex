%!TEX root = main.tex
%!TEX spellcheck=de_DE
%%%%%%%%%%%%%%%%%%%%%%%%%%%%%%%%%%%%%%%%%%%%%%%%%%%%%%%%%%%%%%%%%%%%%%%%%%%%%%%
% % Section 2
%%%%%%%%%%%%%%%%%%%%%%%%%%%%%%%%%%%%%%%%%%%%%%%%%%%%%%%%%%%%%%%%%%%%%%%%%%%%%%%

\section{Ausblick} % (fold)
\label{sec:ausblick}

\subsection{\texorpdfstring{Ausdehnen von \cref{thm:main_theorem} auf $\mathcal{S}'$}{Ausdehnen auf Distributionen}} % (fold)
\label{sec:ausdehnen_von_thm:main_theorem}
Wie in \cref{rem:shearlets_no_distributions} angesprochen, zeigt der Beweis von \textcite{Kutyniok2008} \cref{thm:main_theorem} nur für beschränkte Funtkionen und nicht für allgemeine temperierte Distributionen in $\mathcal{S}'$. So werden alle Hilfslemmata für \cref{proof:main_theorem} nur für solche Funktionen bewiesen. Wir glauben aber, dass sich der Beweis auf alle temperierten Distributionen ausdehnen lässt, dank der Tatsache dass "`temperierte Distributionen polynomiell beschränkt sind"':

\begin{theorem}[Struktursatz für temperierte Distributionen]
\label{thm:struktursatz}
    Sei $X \subset \mathbb{R}^n$ offen. Sei $f \in \mathcal{S}'(X)$. Dann gibt es ein $F \in C(X)$ und $C \in \mathbb{R}, N \in \mathbb{N}$ s.d. für alle $x \in X$
    \begin{equation*}
        |F(x)| \leq C (1+|x| )^N
    \end{equation*}
    (also F polynomiell beschränkt ist) und
    \begin{equation*}
        f = \partial^\alpha F
    \end{equation*}
    als distributionelle Ableitung

    \emph{Beweis} \\[.5em]
    Der Beweis findet sich in \textcite[S. 97]{Friedlander1998}.
\end{theorem}


Leider fehlt aufgrund des stetigen Studienfortschritts die Zeit, diesen Beweis komplett auszuarbeiten. Der Beweis der auf temperierte Distributionen ausgeweiteten \cref{prop:shearlets_decay_rapidly} und wie das polynomielle Wachstum der temperierten Distributionen soll hier aber beispielhaft skizziert werden.  Mit ähnlichen Tricks lassen sich hoffentlich auch alle anderen Hilfslemmata auf temperierte Distributionen ausweiten.

\begin{lemma}[Verfeinerung von \cref{prop:shearlets_decay_rapidly}]
    \label{lemm:shearlets_decay_quickly_schwartz}
    Sei $f \in \mathcal{S}'(\mathbb{R}^2)$ und $supp(f) \subset U$. Sei $t \notin U$. Dann gilt für alle $k \in \mathbb{N}$
    \begin{equation*}
        \left\langle f, \psi_{ast}\right\rangle \leq C_k \left(1+a^{-1} d(t,U)\right)^{-k}
    \end{equation*}
    ab einem hinreichend kleinen $a$.
\end{lemma}

Vor dem Beweis zwei Worte zur Bedeutung des Lemmas: Der hauptsächliche Nutzen des Lemmas ist die Aussage, dass wir mit Shearlets, also Schwartzfunktionen und \emph{nicht} kompakt getragenen Funktionen, die lokalen Eigenschaften von temperierten Distributionen untersuchen können, da wir alles was $\delta$-weit von $t$ entfernt exponentiell schnell (in $a$) nicht mehr sehen.

\begin{proof}
    Nach \cref{thm:struktursatz} gibt es ein polynomiell beschränktes $F \in C(\mathbb{R}^2)$ s.d. $f = \partial^\alpha F$. O.B.d.A können wir annehmen, dass $f$ auf $B(0,\delta)$ verschwindet (im distributionellen Sinne) und damit auch $F$. Dann zeigen wir die Aussage für $t=0$:

    \begin{dmath*}
        \left \langle f, \psi_{as0} \right\rangle
        \stackrel{\textrm{formal}}{=}
        \int f(x) \psi\left(\left(\begin{smallmatrix}
            a & -\sqrt{a}s \\ 0 & \sqrt a
        \end{smallmatrix}
        \right)^{-1}\left(x-0\right)\right)
        \d x
        =
        \int F(x) \partial^{\alpha}\left[
        \psi\left(\left(\begin{smallmatrix}
            a^{-1} & \frac{s}{\sqrt a} \\ 0 & a^{-\frac{1}{2}}
        \end{smallmatrix}
        \right)^{-1}x\right)\right]
        \d x
        \leq
        \int \limits_{|x| \geq \delta}
        C (1+|x|)^N
        a^{-|\alpha|}
        \underbrace{
        \left[
            \left\lvert \partial_{x_1}^{|\alpha|} \right\rvert
            + \left\lvert \partial^\alpha \psi \right\rvert
        \right]
        }_{=: \phi \in \mathcal{S}}
        \left(\begin{smallmatrix}
            a^{-1} & \frac{s}{\sqrt{a}} \\ 0 & a^{-\frac{1}{2}}
        \end{smallmatrix}\right) x \d x
        \leq
        \int \limits_{|x| \geq \delta}
        C (1+|x|)^N
        a^{-|\alpha|}
        C_k \left(
        1+ \left\lvert \left( \begin{smallmatrix}
            a^{-1} & \frac{s}{\sqrt{a}} \\ 0 & a^{-\frac{1}{2}}
        \end{smallmatrix}\right) x \right\rvert
        \right)^{-k}
        \d x
        \leq
        \int \limits_{|x| \geq \delta}
        C C_k a^{-|\alpha|} (1+|x|)^N (1+a^{-1}|x|)^{-k} \d x
        \leq
        C C_k a^{-|\alpha|}
        \int \limits_{|x| \geq \delta}
        (1+a^{-1}|x|)^{N-k} \d x
        =
        C C_k a^{-|\alpha|} 2 \pi
        \int\limits_0^\infty
        (1+a^{-1}r)^{N-k} \d r
        =
        C C_k a^{-|\alpha|} 2 \pi
        \frac{(a+\delta)\left(1+\frac{\delta}{a}\right)^{N-k}(a+(k-N-1)\delta)}{(k-N-1)(k-N-2)}
        \leq
        C'_k\left(1+\frac{\delta}{a}\right)^{N-k-|\alpha|}
    \end{dmath*}
    Was die Aussage für $N-k-|\alpha|$ und damit auch für alle $k$ zeigt.
\end{proof}

% Die entscheidende Aussage des Lemmas ist, dass die Shearlettransformation Eigenschaften von $f$, die mindestens $delta$-weit von $t$ entfernt sind "`exponentiell schnell nicht mehr sieht"'. Obwohl die Shearlets nicht getragen sind!

% section ausdehnen_von_thm:main_theorem (end)

\subsection{Hörmanders Kriterium abschwächen}

\subsection{Höherdimensionale Shearlets}
% section ausblick (end)
