%!TEX root = main.tex
%!TEX spellcheck=de_DE
%%%%%%%%%%%%%%%%%%%%%%%%%%%%%%%%%%%%%%%%%%%%%%%%%%%%%%%%%%%%%%%%%%%%%%%%%%%%%%%
% % Section 2
%%%%%%%%%%%%%%%%%%%%%%%%%%%%%%%%%%%%%%%%%%%%%%%%%%%%%%%%%%%%%%%%%%%%%%%%%%%%%%%

\section{Ausblick} % (fold)
\label{sec:ausblick}

\subsection{Höherdimensionale Shearlets}
Eine offensichtliche weitere Frage ist: Wie steht es denn damit, das ganze Geschäft der Shearlets mal auf höhere Dimensionen auszudehnen und auch dort eine Technik zum Berechnen von Wellenfrontmengen zu erhalten?

\textcite{Guo2006} diskutieren Verallgemeinerungen der Schergruppe in höheren Dimensionen und entwickeln daraus auch diskrete Shearlets. Aus \cref{fig:delta_m} wird auch deutlich, was die richtige Verallgemeinerung der parabolischen Skalierung ist. Nämlich

\begin{equation*}
\begin{pmatrix}
k_1 \\ k_2 \\ \vdots \\ k_n
\end{pmatrix}
\mapsto
\begin{pmatrix}
	a & 0 		& \cdots & 0\\
	0 & \sqrt a & 		 & 	\vdots\\
	0 & 			& \ddots & 0 \\
	0 & \cdots  & 	0    & \sqrt{a}
\end{pmatrix}
\begin{pmatrix}
k_1 \\ k_2 \\ \vdots \\ k_n
\end{pmatrix},
\end{equation*}

denn diese sorgt wieder dafür, dass der Träger von $\psi_{ast}$ im Fourierraum für $a \to 0$ wieder einer immer spitzer werdenden Nadel gleicht. Die Wahl $\sqrt{a}$ statt $a^\delta$ für irgendein anderes $a<1$ ist ziemlich willkürlich. \textcite{Kutyniok2008} schreiben auch, dass sie für $\delta \neq \frac{1}{2}$ die Wellenfrontmenge an Beispielen genau so gut bestimmen konnten, wie für $\delta = \frac{1}{2}$. Tatsächlich geht $\delta = \frac{1}{2}$ in den Beweis von \cref{thm:main_theorem} in einer Weise ein, dass es m.E. möglich ist auch mit anderen Exponenten zu arbeiten.\footnote{Stellt sich nur die Frage, warum man das überhaupt wollte. $\delta = \frac{1}{2}$ ist doch ein ziemlich schöne Wahl.}

\subsection{Berechnung des Skalengrads mittels Shearlets}
\label{sec:scaling_degree}
Eine weitere Größe der mikrolokalen Analysis, die eventuell durch die Shearlettransformation bestimmt werden kann ist der Skalengrad. Er ist definiert wie folgt:

\begin{definition}[Skalengrad]
\label{def:skalengrad}
    Sei $u \in \mathcal{D}'(\Omega),~ \Omega \subset \mathbb{R}^n ~$ offen. Dann ist der Skalengrad $sd(u)$ definiert als
    \begin{equation*}
        \inf_{\omega} \left\{\lambda^\omega \left\langle u_\lambda, \phi \right\rangle
        \stackrel{\lambda \to 0}{\to} 0 \condition{für alle $\phi$} \right\}
    \end{equation*}

    wobei $u_\lambda$ definiert ist über
    \begin{equation*}
        \left\langle u_\lambda, \phi \right\rangle
        =
        \lambda^{-n} \left\langle u, \phi\left(\tfrac{\cdot}{\lambda}\right)\right\rangle
    \end{equation*}

    also falls $u \in C^\infty$:
    \begin{equation*}
        u_\lambda (x) = u(\lambda x)
    \end{equation*}
\end{definition}

Eine einfache Rechnung zeigt z.B. für die $\delta$-Distribution und ihre Ableitungen, dass

\begin{equation*}
    sd(\delta^{(\alpha)}) = n + |\alpha|
    .
\end{equation*}

Mit der Shearlettransformation erhalten wir aber

\begin{dmath*}
    \left\langle \delta_{x_1}^{(\alpha)} \otimes \delta_{x_2}, \psi_{a00} \right\rangle
    =
    \left.\partial_{x_1}^\alpha \left(a^{-\frac{3}{4}}\psi\left(\frac{x_1}{a}, \frac{x_2}{\sqrt a}\right)\right)\right|_{x=0} \\
    =
    a^{-\frac{3}{4}} a^{-\alpha} \partial_{x_1}^\alpha \psi(0)
    \hiderel \sim a^{ - \alpha-\frac{3}{4}}
\end{dmath*}

und bei Ableitung in die andere Richtung


\begin{dmath*}
    \left\langle \delta_{x_1} \otimes \delta_{x_2}^{(\alpha)}, \psi_{a00} \right\rangle
    =
    \left.\partial_{x_2}^\alpha \left(a^{-\frac{3}{4}}\psi\left(\frac{x_1}{a}, \frac{x_2}{\sqrt a}\right)\right)\right|_{x=0} \\
    =
    a^{-\frac{3}{4}} a^{-\frac{\alpha}{2}} \partial_{x_2}^\alpha \psi(0)
    \hiderel \sim a^{-\frac{\alpha}{2}-\frac{3}{4}}
\end{dmath*}

Und falls wir $s \neq 0$ wählen wird das ganze nur noch unübersichtlicher, da wir Mischterme erhalten. Dieses Beispiel legt also nahe, dass es einen Zusammenhang zwischen dem Skalengrad einer Distribution und dem Abfallverhalten der Shearlettransformation bei $t=0$ gibt. Aber die parabolische Skalierung in $a$ und Scherung in $s$ sorgen dafür, dass sie sich nicht mehr ganz einfach ablesen lässt.
% section ausblick (end)
