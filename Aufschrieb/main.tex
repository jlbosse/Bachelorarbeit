
% !TEX encoding = UTF-8 Unicode
%%%%%%%%%%%%%%%%%%%%%%%%%%%%%%%%%%%%%%%%%%%%%%%%%%%%%%%
% % Lines starting with % are comments, which are ignored.
% % This is a handy way of indicating the date and version of
% % your document, to wit:
% %
% % LaTeX sample file
% % Modified March, 2002
% %
%%%%%%%%%%%%%%%%%%%%%%%%%%%%%%%%%%%%%%%%%%%%%%%%%%%%%%%

% \documentclass{article}
\documentclass[bachelor,       %% Typ der Arbeit: bachelor oder master
               twoside,        %% zweiseitiges Layout
               BCOR5mm,
               DIV=11,       %% Bindekorrektur 10 mm
%               liststotoc,nomtotoc,bibtotoc, %% Aufnahme der div. Verzeichnisse
                                              %% ins Inhaltsverzeichnis
               english,ngerman, %% Alternativspr. Englisch, Dokumentspr. Deutsch
%               ngerman,english  %% Alternativspr. Deutsch, Dokumentspr. Englisch
%               final,          %% Endversion; draft fuer schnelles Kompilieren
               ]{GAUBM}
\usepackage{thesisstyle}

\addbibresource{literature.bib}


%%%%%%%%%%%%%%%%%%%%%%%%%%%%%%%%%%%%%%%%%%%%%%%%%%%%%%%%%%%%%%%%%%%%%%%%%%%%%%%
% % Import plots
%%%%%%%%%%%%%%%%%%%%%%%%%%%%%%%%%%%%%%%%%%%%%%%%%%%%%%%%%%%%%%%%%%%%%%%%%%%%%%%
%%%%%%%%%%%%%%%%%%%%%%%%%%%%%%%%%%%%%%%%%%%%%%%%%%%%%%%


\begin{document}
\pagenumbering{roman}

\ThesisAuthor{Jan Lukas}{Bosse}
%% Hier den Geburtsort einsetzen
\PlaceOfBirth{Freiburg im Breisgau}
%% Titel Arbeit. Das erste Argument ist der deutsche, das zweite der
%% englische Titel.
\ThesisTitle{Zur Auflösung der Wellenfrontmenge mittels Shearlets}{Resolution of the wavefrontset using shearlets}
%% Erst- und Zweitgutacher/in
%% Ist der/die Betreuer/in nicht identisch mit dem/r Erstgutachter/in,
%% muss diese/r als optionales Argument angegeben werden.
\FirstReferee[Prof.\ Dr.\ Dorothea Bahns]{Prof.\ Dr.\ Dorothea Bahns}
\Institute{Institut f\"ur Mathematik}
\SecondReferee{Prof.\ Dr.\ Ingo Witt}
%% Beginn und Ende des Anfertigungszeitraumes
\ThesisBegin{15}{2}{2018}
\ThesisEnd{30}{7}{2018}
%% DO NOT TOUCH THESE LINES!!!!
% \frontmatter
\maketitle
\cleardoublepage
%% Zusammenfassung. Falls nicht gewuenscht, bitte auskommentieren.
\begin{abstract}
  Mein schnönes Abstrakt
%% Optional: Stichwoerter. Wenn nicht gewuenscht, koennen die beiden
%% folgenden Zeilen geloescht werden
  \bigskip\par
  \textbf{Stichwörter:} Physik, Bachelorarbeit
\end{abstract}
%% So laesst sich in die andere Sprache umschalten (Englisch bzw. Deutsch)
\begin{otherlanguage}{english}
\begin{abstract}
  Do we need an english abstract?
  \bigskip\par
  \textbf{Keywords:} Physics, Bachelor thesis
\end{abstract}
\end{otherlanguage}

%% Ende des Vorspanns
\cleardoublepage
%% Ab hier 1 1/2 facher Zeilenabstand (durch setspace-Paket)
\onehalfspacing
%% Erzeugt Inhaltsverzeichnis
\tableofcontents
\pagenumbering{arabic}
% \mainmatter

%!TEX root = main.tex

\section{Fouriertransformation, mikrolokale Analysis und all die Mathematik} % (fold)
\label{sec:fouriertransformation_mikrolokale_analysis_und_all_die_mathematik}

Hier entsteht mal ein Kapitel, in dem die notwendigen mathematischen Begriffe eingeführt und motiviert werden.

\begin{definition}[high frequency set]
\label{def:high_frequency_set}
Sei $f \in \mathcal{E}'(\Omega), \Omega \subset \mathcal{R}^n$ ein kompakt getragene Distribution. Dann definieren wir die Richtungen höher Frequenzen als
\begin{dmath*}
\Sigma (v) = \left\{k \hiderel \in \hat{\mathbb{R}}^n \big| k \textrm{ hat \emph{keine} kegelförmige Umgebung U s.d. } \\ |\hat v (k')| \leq C_N(1+|k|)^{-N} \forall k \hiderel\in V, \forall N  \hiderel\in \mathbb{N} \right\}
\end{dmath*}
und darauf basierend definieren wir noch eine lokalisierte Variante:

Sei $f \in \mathcal{D}'(\Omega), \Omega \subset \mathbb{R}^n$ eine Distribution.
Sei $\mathcal{D}_x$ die Menge der kompakt getragenen glatten Funktionen, die an $x$ nicht verschwinden.
Dann ist die singuläre Faser von $f$ an $x$ definiert als
\begin{dmath*}
\Sigma_x (f) = \bigcap \limits_{\phi \includegraphics[scale=•]{•} \mathcal{D}_x} \Sigma (\underbrace{\phi f}_{\in \mathcal{E}'(\mathbb{R}^n)}) 
\end{dmath*}
\end{definition}

Damit können wir uns dann die Wellenfrontmenge definieren:

\begin{definition}[Wellenfrontmenge]
\label{def:wavefrontset}
\end{definition}
Sei $f \in \mathcal{D}'(\Omega), \Omega \subset \mathbb{R}^n$ eine Distribution. Dann ist ihre Wellenfrontmenge definiert als

\begin{dmath*}
WF(f) \coloneqq \left\{
	(x,k) \in \Omega \times (\hat{\mathbb{R}}^n \setminus 0)
	\Big | k \in \Sigma_x(f)
	\right\}
\end{dmath*}
\end{definition}

Anschaulich sagt uns die Wellenfrontmenge wo und in welche Richtungen eine Distribution singulär ist. So ist z.B. die Wellenfrontmenge der $\delta$-Distribution $(0, \mathbb{R}^n \setminus 0)$ oder das der 2-dimensionalen Heaviside-Funktion $1(x)\cdot\Theta(y)$ ist $\{((x,0),(0,1)\cdot \mathbb{R}\setminus 0)\}$

\begin{figure}
\caption{so ne tolle Faltung}
\label{fig:faltung_strahlen}
\end{figure}

% section fouriertransformation_mikrolokale_analysis_und_all_das (end)


%!TEX root = main.tex

\section{Zweipunktfunktionen, Sternprodukte und all die Physik} % (fold)
\label{sec:zweipunktfunktionen_sternprodukte_und_all_die_physik}

\subsection{Zweipunktfunktionen und warum wir sie potenzieren wollen}
In der \emph{perturbativen Quantenfeldtheorie}

\begin{equation}
    G_F(t,x)
    =
    \Theta (t)\Delta_m(t,x) + \Theta(-t)\Delta_m(-t,-x)
\end{equation}

Deshalb ist es wichtig zu wissen, dass das Produkt der Distributionen $\Delta_m(t,x)$ und $\Theta(t)\otimes 1(x)$ und seine Potenzen wohldefiniert sind.

\subsection{Sternprodukte und getwistete Faltungen}
Die \emph{nicht kommutativen Quantenfeldtheorie} beschäftigt sich mit Quantenfeldtheorien in der Größenordnung der \emph{Planck-Skala}. Bei diesen Größenordnungen wird erwartet, dass die Geometrie der Raumzeit nicht mehr kommutativ ist; sich also Ort und Zeit nicht mehr mit beliebiger Präzision messen lassen. Diese Schranken in der Messgenauigkeit lassen sich als Unschärferelation verstehen, ganz analog zur klassischen Unschärferelation zwischen Ort und Impuls.

Bei der Deformationsquantisierung wird das kommutative punktweise Produkt von Funktionen auf dem Phasenraum ersetzt durch ein nicht-kommutatives Sternprodukt/Moyal-Produkt, um das nicht-kommutieren von Ort und Impuls zu implementieren.
Mehr Details finden sich in \textcite[Kap. 6]{Waldmann2007}.

Analog zu dieser Konstruktion ersetzen \textcite{Doplicher1995} das kommutative Produkt von Funktionen auf der Raumzeit durch ein Sternprodukt $\star$ auf den Funktionen auf der Raumzeit, und erhalten damit eine nicht-kommutative Geometrie. Gemäß dem Faltungssatz für die Fouriertransformierte gibt es auch eine \emph{getwistete Faltung} $\circledast$, s.d. der Faltungssatz erfüllt ist. Also
\begin{equation*}
    \rwhat{f \star g} (k) = \hat f \circledast \hat g (k)
\end{equation*}

In zwei Dimensionen ist die getwistete Faltung definiert durch

\begin{definition}[getwistete Faltung]
\label{def:twisted_convolution}
    Seien $f,g \in $ "`passender Funktionen/Distributionenraum"'. Sei $\Omega \in \mathbb{R}^{2 \times 2}$ eine symplektische Matrix. Dann ist die verdrehte Faltung $(f \circledast g) (x)$ definiert als

    \begin{equation}
        (f \circledast g) \,(x) \coloneqq
        \int f(y) g(x-y)e^{\frac{i}{2} \Omega(x,y)} \d y
    \end{equation}

    Die getwistete Faltung ist also einfach die gewöhnliche Faltung, die noch mit einem ortsabhängigen Phasenfaktor verziert wurde.

    Wir verwenden die kanonische symplektische Matrix, also
    $\Omega = \left(\begin{smallmatrix}
        0 & 1 \\ -1 & 0
    \end{smallmatrix}\right)$
\end{definition}

Durch formale Rechnung, Ausschreiben der $e$-Funktion als Potenzreihe und nutzen der Fourieridentitäten $\cdot x \leftrightarrow i \partial_k$ sieht man, dass das Sternprodukt die Form

\begin{equation*}
    f \star g = fg + \frac{i}{2} \sum_{i,j} \Pi^{ik}(\partial_if)(\partial_jg) - \frac{1}{8}\sum_{i,j,k,m} \Pi^{ij} \Pi^{km} (\partial_i \partial_k f)(\partial_j \partial_m g) + \dots
\end{equation*}

hat. Dabei ist $\Pi$ der zu $\Omega$ korrespondierende Poisson-Bivektor.


% section zweipunktfunktionen_sternprodukte_und_all_die_physik (end)


%!TEX root = main.tex
%%%%%%%%%%%%%%%%%%%%%%%%%%%%%%%%%%%%%%%%%%%%%%%%%%%%%%%%%%%%%%%%%%%%%%%%%%%%%%%
% % Section 1
%%%%%%%%%%%%%%%%%%%%%%%%%%%%%%%%%%%%%%%%%%%%%%%%%%%%%%%%%%%%%%%%%%%%%%%%%%%%%%%
\section{Shearlets} % (fold)
\label{sec:shearlets}

\todo[color=green]{In dieser Sektion nur die wichtigsten Ergebnisse des Papers angeben, oder auch Beweise oder zumindest Beweisskizzen, damit man sieht wie alles zusammen spielt?}

\begin{proposition}[$\psi_{ast}$ fällt schnell ab]
\label{prop:shearlets_decay_rapidly}
Sei $\psi \in L^2(\mathbb{R}^2)$ ein Shearlet wie definiert und $M$ so ne Trafomatrix. Dann gilt für alle $k \in  \mathbb{N}$, dass es eine konstante $C_k$ gibt s.d. für alle $x \in \mathbb{R}^2$ gilt

\begin{dmath*}
    \left| \psi_{ast}(x) \right|
    \leq
    C_k \left| \det M \right|^{-\frac{1}{2}}\left(1+|M^{-1}(x-t)|^2\right)^{-k}
    = C_k a^{-\frac{3}{4}}\left(1+a^{-2}\left(x_1-t_1\right)^2
        + 2 a^{-2}s\left(x_1-t_1\right)\left(x_2-t_2\right)
        + a^{-1}\left(1+a^{-1}s^2\right)\left(x_2-t_2\right)^2
    \right)^{-k}
\end{dmath*}

Und insbesondere ist $C_k = \frac{15}{2}\frac{\sqrt{a} + s}{a^2}\left(\Vert \hat\psi \Vert_\infty + \Vert \Laplace^k \hat\psi \Vert_\infty\right)$

\end{proposition}

\begin{theorem}[$\mathcal{S}_f(a,s,t)$ misst $WF(f)$]
\label{thm:main_theorem}
    Sei $\mathcal{D} = \mathcal{D}_1 \cup \mathcal{D}_2$ wobei
    $\mathcal{D}_1$ = \{
        $(t_0, s_0) \in \mathbb{R}^2 \times [-1,1] \big|$
        $|\mathcal{S}_f (a, s, t)| = O(a^k)$ gleichmäßig $\forall k \in \mathbb{N}
        , \forall t \in U$ Umgebung von $(t_0, s_0)$
    \}
    und $\mathcal{D}_2$ analog für $\psi^{(v)}$

    Dann gilt $WF(f)^c = \mathcal{D}$
\end{theorem}

\todo{diesen Satz richtig hin schreiben und ordentlich setzen}
\todo{Stil und Nummerierung für Sätze, Propositionen etc. anpassen}

\begin{corollary}[WF(f) misst $sing ~supp (\psi)$]
Sei $\mathcal{R} =$ \{
    $t_0 \in \mathcal{R}^2 \big|$ $|\mathcal{S}_f(a,s,t)| = O(a^k)$
    $\forall k \in \mathbb{N}, \forall t \in U$ Umgebung von $t_0$
    \}

    Dann gilt $sing ~supp (\psi)^c = \mathcal{R}$
\end{corollary}

\begin{remark}[Träger von $\psi$]

\begin{figure}[h]
\centering
%% Creator: Matplotlib, PGF backend
%%
%% To include the figure in your LaTeX document, write
%%   \input{<filename>.pgf}
%%
%% Make sure the required packages are loaded in your preamble
%%   \usepackage{pgf}
%%
%% Figures using additional raster images can only be included by \input if
%% they are in the same directory as the main LaTeX file. For loading figures
%% from other directories you can use the `import` package
%%   \usepackage{import}
%% and then include the figures with
%%   \import{<path to file>}{<filename>.pgf}
%%
%% Matplotlib used the following preamble
%%   \usepackage[utf8x]{inputenc}
%%   \usepackage[T1]{fontenc}
%%   \usepackage{amssymb}
%%
\begingroup%
\makeatletter%
\begin{pgfpicture}%
\pgfpathrectangle{\pgfpointorigin}{\pgfqpoint{4.000000in}{2.000000in}}%
\pgfusepath{use as bounding box, clip}%
\begin{pgfscope}%
\pgfsetbuttcap%
\pgfsetmiterjoin%
\definecolor{currentfill}{rgb}{1.000000,1.000000,1.000000}%
\pgfsetfillcolor{currentfill}%
\pgfsetlinewidth{0.000000pt}%
\definecolor{currentstroke}{rgb}{1.000000,1.000000,1.000000}%
\pgfsetstrokecolor{currentstroke}%
\pgfsetdash{}{0pt}%
\pgfpathmoveto{\pgfqpoint{0.000000in}{0.000000in}}%
\pgfpathlineto{\pgfqpoint{4.000000in}{0.000000in}}%
\pgfpathlineto{\pgfqpoint{4.000000in}{2.000000in}}%
\pgfpathlineto{\pgfqpoint{0.000000in}{2.000000in}}%
\pgfpathclose%
\pgfusepath{fill}%
\end{pgfscope}%
\begin{pgfscope}%
\pgfsetbuttcap%
\pgfsetmiterjoin%
\definecolor{currentfill}{rgb}{1.000000,1.000000,1.000000}%
\pgfsetfillcolor{currentfill}%
\pgfsetlinewidth{0.000000pt}%
\definecolor{currentstroke}{rgb}{0.000000,0.000000,0.000000}%
\pgfsetstrokecolor{currentstroke}%
\pgfsetstrokeopacity{0.000000}%
\pgfsetdash{}{0pt}%
\pgfpathmoveto{\pgfqpoint{0.198611in}{0.198611in}}%
\pgfpathlineto{\pgfqpoint{3.801389in}{0.198611in}}%
\pgfpathlineto{\pgfqpoint{3.801389in}{1.801389in}}%
\pgfpathlineto{\pgfqpoint{0.198611in}{1.801389in}}%
\pgfpathclose%
\pgfusepath{fill}%
\end{pgfscope}%
\begin{pgfscope}%
\pgfpathrectangle{\pgfqpoint{0.198611in}{0.198611in}}{\pgfqpoint{3.602778in}{1.602778in}}%
\pgfusepath{clip}%
\pgfsetbuttcap%
\pgfsetmiterjoin%
\definecolor{currentfill}{rgb}{0.500000,0.500000,0.500000}%
\pgfsetfillcolor{currentfill}%
\pgfsetfillopacity{0.500000}%
\pgfsetlinewidth{0.501875pt}%
\definecolor{currentstroke}{rgb}{0.000000,0.000000,0.000000}%
\pgfsetstrokecolor{currentstroke}%
\pgfsetdash{}{0pt}%
\pgfpathmoveto{\pgfqpoint{1.954965in}{0.398958in}}%
\pgfpathlineto{\pgfqpoint{2.045035in}{0.398958in}}%
\pgfpathlineto{\pgfqpoint{2.180139in}{0.519167in}}%
\pgfpathlineto{\pgfqpoint{1.819861in}{0.519167in}}%
\pgfpathclose%
\pgfusepath{stroke,fill}%
\end{pgfscope}%
\begin{pgfscope}%
\pgfpathrectangle{\pgfqpoint{0.198611in}{0.198611in}}{\pgfqpoint{3.602778in}{1.602778in}}%
\pgfusepath{clip}%
\pgfsetbuttcap%
\pgfsetmiterjoin%
\definecolor{currentfill}{rgb}{0.500000,0.500000,0.500000}%
\pgfsetfillcolor{currentfill}%
\pgfsetfillopacity{0.500000}%
\pgfsetlinewidth{0.501875pt}%
\definecolor{currentstroke}{rgb}{0.000000,0.000000,0.000000}%
\pgfsetstrokecolor{currentstroke}%
\pgfsetdash{}{0pt}%
\pgfpathmoveto{\pgfqpoint{1.883721in}{0.626019in}}%
\pgfpathlineto{\pgfqpoint{2.116279in}{0.626019in}}%
\pgfpathlineto{\pgfqpoint{2.465117in}{1.427407in}}%
\pgfpathlineto{\pgfqpoint{1.534883in}{1.427407in}}%
\pgfpathclose%
\pgfusepath{stroke,fill}%
\end{pgfscope}%
\begin{pgfscope}%
\pgfpathrectangle{\pgfqpoint{0.198611in}{0.198611in}}{\pgfqpoint{3.602778in}{1.602778in}}%
\pgfusepath{clip}%
\pgfsetbuttcap%
\pgfsetmiterjoin%
\definecolor{currentfill}{rgb}{0.500000,0.500000,0.500000}%
\pgfsetfillcolor{currentfill}%
\pgfsetfillopacity{0.500000}%
\pgfsetlinewidth{0.501875pt}%
\definecolor{currentstroke}{rgb}{0.000000,0.000000,0.000000}%
\pgfsetstrokecolor{currentstroke}%
\pgfsetdash{}{0pt}%
\pgfpathmoveto{\pgfqpoint{2.183952in}{0.626019in}}%
\pgfpathlineto{\pgfqpoint{2.416511in}{0.626019in}}%
\pgfpathlineto{\pgfqpoint{3.666043in}{1.427407in}}%
\pgfpathlineto{\pgfqpoint{2.735809in}{1.427407in}}%
\pgfpathclose%
\pgfusepath{stroke,fill}%
\end{pgfscope}%
\begin{pgfscope}%
\pgfpathrectangle{\pgfqpoint{0.198611in}{0.198611in}}{\pgfqpoint{3.602778in}{1.602778in}}%
\pgfusepath{clip}%
\pgfsetbuttcap%
\pgfsetroundjoin%
\pgfsetlinewidth{0.501875pt}%
\definecolor{currentstroke}{rgb}{0.501961,0.501961,0.501961}%
\pgfsetstrokecolor{currentstroke}%
\pgfsetdash{{1.850000pt}{0.800000pt}}{0.000000pt}%
\pgfpathmoveto{\pgfqpoint{1.804251in}{0.184722in}}%
\pgfpathlineto{\pgfqpoint{3.636860in}{1.815278in}}%
\pgfpathlineto{\pgfqpoint{3.636860in}{1.815278in}}%
\pgfusepath{stroke}%
\end{pgfscope}%
\begin{pgfscope}%
\pgfpathrectangle{\pgfqpoint{0.198611in}{0.198611in}}{\pgfqpoint{3.602778in}{1.602778in}}%
\pgfusepath{clip}%
\pgfsetbuttcap%
\pgfsetroundjoin%
\pgfsetlinewidth{0.501875pt}%
\definecolor{currentstroke}{rgb}{0.501961,0.501961,0.501961}%
\pgfsetstrokecolor{currentstroke}%
\pgfsetdash{{1.850000pt}{0.800000pt}}{0.000000pt}%
\pgfpathmoveto{\pgfqpoint{0.363140in}{1.815278in}}%
\pgfpathlineto{\pgfqpoint{2.195749in}{0.184722in}}%
\pgfpathlineto{\pgfqpoint{2.195749in}{0.184722in}}%
\pgfusepath{stroke}%
\end{pgfscope}%
\begin{pgfscope}%
\pgfsetrectcap%
\pgfsetmiterjoin%
\pgfsetlinewidth{0.501875pt}%
\definecolor{currentstroke}{rgb}{0.000000,0.000000,0.000000}%
\pgfsetstrokecolor{currentstroke}%
\pgfsetdash{}{0pt}%
\pgfpathmoveto{\pgfqpoint{2.000000in}{0.198611in}}%
\pgfpathlineto{\pgfqpoint{2.000000in}{1.801389in}}%
\pgfusepath{stroke}%
\end{pgfscope}%
\begin{pgfscope}%
\pgfsetrectcap%
\pgfsetmiterjoin%
\pgfsetlinewidth{0.501875pt}%
\definecolor{currentstroke}{rgb}{0.000000,0.000000,0.000000}%
\pgfsetstrokecolor{currentstroke}%
\pgfsetdash{}{0pt}%
\pgfpathmoveto{\pgfqpoint{0.198611in}{0.358889in}}%
\pgfpathlineto{\pgfqpoint{3.801389in}{0.358889in}}%
\pgfusepath{stroke}%
\end{pgfscope}%
\begin{pgfscope}%
\pgfsetroundcap%
\pgfsetroundjoin%
\pgfsetlinewidth{0.501875pt}%
\definecolor{currentstroke}{rgb}{0.000000,0.000000,0.000000}%
\pgfsetstrokecolor{currentstroke}%
\pgfsetdash{}{0pt}%
\pgfpathmoveto{\pgfqpoint{1.413964in}{0.497069in}}%
\pgfpathquadraticcurveto{\pgfqpoint{1.625531in}{0.488637in}}{\pgfqpoint{1.829340in}{0.480514in}}%
\pgfusepath{stroke}%
\end{pgfscope}%
\begin{pgfscope}%
\pgfsetroundcap%
\pgfsetroundjoin%
\pgfsetlinewidth{0.501875pt}%
\definecolor{currentstroke}{rgb}{0.000000,0.000000,0.000000}%
\pgfsetstrokecolor{currentstroke}%
\pgfsetdash{}{0pt}%
\pgfpathmoveto{\pgfqpoint{1.774935in}{0.510482in}}%
\pgfpathlineto{\pgfqpoint{1.829340in}{0.480514in}}%
\pgfpathlineto{\pgfqpoint{1.772722in}{0.454971in}}%
\pgfusepath{stroke}%
\end{pgfscope}%
\begin{pgfscope}%
\pgftext[x=0.648958in,y=0.479097in,left,base]{\rmfamily\fontsize{10.000000}{12.000000}\selectfont \(\displaystyle a = 1, s = 0\)}%
\end{pgfscope}%
\begin{pgfscope}%
\pgfsetroundcap%
\pgfsetroundjoin%
\pgfsetlinewidth{0.501875pt}%
\definecolor{currentstroke}{rgb}{0.000000,0.000000,0.000000}%
\pgfsetstrokecolor{currentstroke}%
\pgfsetdash{}{0pt}%
\pgfpathmoveto{\pgfqpoint{1.141127in}{1.096480in}}%
\pgfpathquadraticcurveto{\pgfqpoint{1.354038in}{1.088809in}}{\pgfqpoint{1.559190in}{1.081418in}}%
\pgfusepath{stroke}%
\end{pgfscope}%
\begin{pgfscope}%
\pgfsetroundcap%
\pgfsetroundjoin%
\pgfsetlinewidth{0.501875pt}%
\definecolor{currentstroke}{rgb}{0.000000,0.000000,0.000000}%
\pgfsetstrokecolor{currentstroke}%
\pgfsetdash{}{0pt}%
\pgfpathmoveto{\pgfqpoint{1.504671in}{1.111178in}}%
\pgfpathlineto{\pgfqpoint{1.559190in}{1.081418in}}%
\pgfpathlineto{\pgfqpoint{1.502670in}{1.055658in}}%
\pgfusepath{stroke}%
\end{pgfscope}%
\begin{pgfscope}%
\pgftext[x=0.198611in,y=1.080139in,left,base]{\rmfamily\fontsize{10.000000}{12.000000}\selectfont \(\displaystyle a = 0.15, s = 0\)}%
\end{pgfscope}%
\begin{pgfscope}%
\pgfsetroundcap%
\pgfsetroundjoin%
\pgfsetlinewidth{0.501875pt}%
\definecolor{currentstroke}{rgb}{0.000000,0.000000,0.000000}%
\pgfsetstrokecolor{currentstroke}%
\pgfsetdash{}{0pt}%
\pgfpathmoveto{\pgfqpoint{3.348730in}{0.656645in}}%
\pgfpathquadraticcurveto{\pgfqpoint{3.136663in}{0.781233in}}{\pgfqpoint{2.931289in}{0.901887in}}%
\pgfusepath{stroke}%
\end{pgfscope}%
\begin{pgfscope}%
\pgfsetroundcap%
\pgfsetroundjoin%
\pgfsetlinewidth{0.501875pt}%
\definecolor{currentstroke}{rgb}{0.000000,0.000000,0.000000}%
\pgfsetstrokecolor{currentstroke}%
\pgfsetdash{}{0pt}%
\pgfpathmoveto{\pgfqpoint{2.965119in}{0.849795in}}%
\pgfpathlineto{\pgfqpoint{2.931289in}{0.901887in}}%
\pgfpathlineto{\pgfqpoint{2.993261in}{0.897696in}}%
\pgfusepath{stroke}%
\end{pgfscope}%
\begin{pgfscope}%
\pgftext[x=3.080833in,y=0.519167in,left,base]{\rmfamily\fontsize{10.000000}{12.000000}\selectfont \(\displaystyle a = 0.15, s = 1\)}%
\end{pgfscope}%
\begin{pgfscope}%
\pgfsetroundcap%
\pgfsetroundjoin%
\pgfsetlinewidth{0.501875pt}%
\definecolor{currentstroke}{rgb}{0.000000,0.000000,0.000000}%
\pgfsetstrokecolor{currentstroke}%
\pgfsetdash{}{0pt}%
\pgfpathmoveto{\pgfqpoint{2.000000in}{1.807510in}}%
\pgfpathquadraticcurveto{\pgfqpoint{2.000000in}{1.808331in}}{\pgfqpoint{2.000000in}{1.801389in}}%
\pgfusepath{stroke}%
\end{pgfscope}%
\begin{pgfscope}%
\pgfsetroundcap%
\pgfsetroundjoin%
\pgfsetlinewidth{0.501875pt}%
\definecolor{currentstroke}{rgb}{0.000000,0.000000,0.000000}%
\pgfsetstrokecolor{currentstroke}%
\pgfsetdash{}{0pt}%
\pgfpathmoveto{\pgfqpoint{1.972222in}{1.751954in}}%
\pgfpathlineto{\pgfqpoint{2.000000in}{1.807510in}}%
\pgfpathlineto{\pgfqpoint{2.027778in}{1.751954in}}%
\pgfusepath{stroke}%
\end{pgfscope}%
\begin{pgfscope}%
\pgftext[x=2.000000in,y=1.870833in,,bottom]{\rmfamily\fontsize{10.000000}{12.000000}\selectfont \(\displaystyle \omega\)}%
\end{pgfscope}%
\begin{pgfscope}%
\pgfsetroundcap%
\pgfsetroundjoin%
\pgfsetlinewidth{0.501875pt}%
\definecolor{currentstroke}{rgb}{0.000000,0.000000,0.000000}%
\pgfsetstrokecolor{currentstroke}%
\pgfsetdash{}{0pt}%
\pgfpathmoveto{\pgfqpoint{3.807488in}{0.358889in}}%
\pgfpathquadraticcurveto{\pgfqpoint{3.808320in}{0.358889in}}{\pgfqpoint{3.801389in}{0.358889in}}%
\pgfusepath{stroke}%
\end{pgfscope}%
\begin{pgfscope}%
\pgfsetroundcap%
\pgfsetroundjoin%
\pgfsetlinewidth{0.501875pt}%
\definecolor{currentstroke}{rgb}{0.000000,0.000000,0.000000}%
\pgfsetstrokecolor{currentstroke}%
\pgfsetdash{}{0pt}%
\pgfpathmoveto{\pgfqpoint{3.751932in}{0.386667in}}%
\pgfpathlineto{\pgfqpoint{3.807488in}{0.358889in}}%
\pgfpathlineto{\pgfqpoint{3.751932in}{0.331111in}}%
\pgfusepath{stroke}%
\end{pgfscope}%
\begin{pgfscope}%
\pgftext[x=3.870833in,y=0.358889in,left,]{\rmfamily\fontsize{10.000000}{12.000000}\selectfont \(\displaystyle k\)}%
\end{pgfscope}%
\end{pgfpicture}%
\makeatother%
\endgroup%

\label{fig:supp_psi_hat}
\caption{Der Träger von $\hat \psi_{ast}$ für verschiedene $a, s$. Man sieht gut,
wie $supp (\hat \psi_{ast})$ für kleinere $a$ in immer spitzeren Kegeln liegt.}
\end{figure}

\label{cor:psi_hat}
Im Fourierraum ist $\hat{\psi}_{ast}$ gegeben durch

\begin{equation}
    \hat \psi_{ast}{(\xi_1, \xi_2)} = a^{\frac{3}{4}}e^{-i\xi \cdot t}\hat\psi_1(a \xi_1) \hat\psi_{2}\left(a^{-\frac{1}{2}}\left(\frac{\xi_2}{\xi_1}-s\right)\right)
\end{equation}

und es gilt

\begin{equation}
\label{eq:supp_psi}
    supp(\hat \psi) \subset \left\{\xi \in  \hat{\mathbb{R}}^2 ~\Big| ~|\xi_1| \in \left[\frac{1}{2 a} , \frac{2}{a}\right], \left|\frac{\xi_2}{\xi_1} - s\right| \leq \sqrt{a} \right\}
\end{equation}

\end{remark}



% section allgemeines_gelaber_über_shearlets (end)


%%%%%%%%%%%%%%%%%%%%%%%%%%%%%%%%%%%%%%%%%%%%%%%%%%%%%%%%%%%%%%%%%%%%%%%%%%%%%%%%
% % Section 2
%%%%%%%%%%%%%%%%%%%%%%%%%%%%%%%%%%%%%%%%%%%%%%%%%%%%%%%%%%%%%%%%%%%%%%%%%%%%%%%%
\section{\texorpdfstring{Zwei nützliche Substitionen für  $\left<\psi_{ast}, f\right>$}{zwei nützliche Substitutionen}}
\label{sec:substitutionen}

\todo[color=green]{mit $(\omega, k)$ als Variablennamen arbeiten, um zum Rest des Textes zu passen, oder mit $(\xi_1, \xi_2)$ um zu \textcite{Kutyniok2008} zu passen?}

Zunächst werden wir zwei verschiedene Ausdrücke für $\left<\psi_{ast}, f\right>$
im Fourierraum herleiten, welche fast immer Ausgangspunkt für unsere Abschätzungen sein werden.

Sei also $\psi$ ein Shearlet wie in \cref{cor:psi_hat}. Sei $f$ die zu
analysierende fouriertransformierbare Funktion (oder Distribution) in
$\mathcal{D}' (\mathbb{R}^2)$. Dann ist $\mathcal{S}_f (ast)$ gegeben durch

\begin{align*}
\left< \psi_{ast}, f \right> &= \left<\hat\psi_{ast}, \hat f\right> \\
 &= \int a^{\frac{3}{4}} e^{-i \xi \cdot t} \hat \psi_1(a \xi_1)
    \hat \psi_2 \left(a^{-\frac{1}{2}} \left(\frac{\xi_2}{\xi_1} - s\right)\right)
    \hat f (\xi) \d \xi
\end{align*}

\todo{entscheiden, was mit dem fehlenden Faktor $\frac{1}{(2 \pi)^n}$ geschieht}
und nach "`entscheren"' und "`deskalieren"', also der Substitution

\begin{equation*}
\begin{aligned}[c]
a \xi_1 &= k_1\\
a^{-\frac{1}{2}} \left(\frac{\xi_2}{\xi_1} - s\right) &=\frac{k_2}{k_1}\\
\end{aligned}
\qquad\Longleftrightarrow\qquad
\begin{aligned}[c]
\xi_1 &= \frac{k_1}{a}\\
\xi_2 &= \frac{k_1 s}{a} + a^{-\frac{1}{2}} k_2\\
\end{aligned}
\end{equation*}

\begin{equation*}
\Rightarrow
\d \xi_1 \d \xi_2 = a^{-\frac{3}{2}} \d k_1 \d k_2
\end{equation*}

ergibt sich folgendes für $\left<\psi_{ast}, f\right>$:

\todo{\texttt{owntag} fixen}

\begin{align}
    \left\langle\psi_{ast},f\right\rangle
    &=  \left\langle\hat\psi_{ast},\hat f\right\rangle \nonumber \\
    &=  \iint a^{-\frac{3}{4}}~\hat \psi_1(k_1) ~\hat \psi_2 \left(\tfrac{k_2}{k_1}\right)
    ~\hat f \left(\tfrac{k_1}{a}, \tfrac{k_1 s}{a} + \tfrac{k_2}{\sqrt{a}}\right)
    ~e^{-i\frac{k_1}{a}(t_1+t_2 s) - i \frac{k_2 t_2}{\sqrt a}}
    \d k_1 \d k_2
\owntag[substitution1]{Substitution 1}
\end{align}

\todo{herausfinden, wie die Gleichungen auch Kapitelnummern erhalten}

Alternativ kann auch folgende Substitution

\begin{equation*}
\begin{aligned}[c]
a \xi_1 &= k_1\\
a^{-\frac{1}{2}} \left(\frac{\xi_2}{\xi_1} - s\right) &= k_2\\
\end{aligned}
\qquad\Longleftrightarrow\qquad
\begin{aligned}[c]
\xi_1 &= \frac{k_1}{a}\\
\xi_2 &= \left( a^{\frac{1}{2}} k_2 +s \right) \frac{k_1}{a}\\
\end{aligned}
\end{equation*}

\begin{equation*}
\Rightarrow
\d \xi_1 \d \xi_2 = a^{-\frac{3}{2}} k_1 \d k_1 \d k_2
\end{equation*}

gewählt werden, wodurch alle Parameter $(a,s,t)$ aus den Argumenten von $\hat\psi_1, \hat\psi_2$
verschwinden und sich

\begin{align}
    \left<\psi_{ast},f\right>
    =  \iint a^{-\frac{3}{4}}~ k_1~ \hat \psi_1(k_1)~ \hat \psi_2 (k_2)~
    \hat f \left(\tfrac{k_1}{a}, k_1 \left(a^{-\frac{1}{2}}k_2 + s a^{-1}\right)\right)
    ~e^{-i k_1 \left(\frac{t_1+s t_2}{a} + \frac{k_2 t_2}{\sqrt{a}}\right)}
    \d k_1 \d k_2
\owntag[substitution2]{Substitution 2}
\end{align}

ergibt. Dabei ist zu beachten, dass diese Substitution zulässig ist, obwohl sie
die Orientierung \emph{nicht} erhält und \emph{keine} Bijektion ist. Aber
der kritische Bereich, nämlich $\xi_1 = 0$, liegt nicht im Träger von $\rwhat{\psi}$.

\todo{Grafik basteln, die $supp ~\psi$ vor und nach der Substitution zeigt.}


%!TEX root = main.tex
%%%%%%%%%%%%%%%%%%%%%%%%%%%%%%%%%%%%%%%%%%%%%%%%%%%%%%%%%%%%%%%%%%%%%%%%%%%%%%%
% % Section 1
%%%%%%%%%%%%%%%%%%%%%%%%%%%%%%%%%%%%%%%%%%%%%%%%%%%%%%%%%%%%%%%%%%%%%%%%%%%%%%%
\section{Beweis von \cref{thm:main_theorem}} % (fold)
\label{sec:beweis_von_thm:main_theorem}


Bevor wir \cref{thm:main_theorem} beweisen können, benötigen wir aber noch ein paar technische Lemmata. Beweise dafür finden sich in \textcite{Kutyniok2008}

\begin{lemma}
\label{lemm:ruecktrafo_fourier_faellt_schnell_ab}

Seien $B(s_0,\Delta) \subset [-1,1]$ und $V \subset \mathbb{R}^2$ beschränkt. Nehme an, dass $G(a,s,t)$ schnell abfällt für $a \to 0$ gleichmäßig für $(s,t) \in  B(s_0,\Delta) \times V$. Dann fällt

\begin{equation*}
    \hat h(k) = \int \limits_0^1 \int \limits_V \int \limits_{[-1,1]}
    G(a,s,t) \hat \psi_{ast} (k)
        \d s \d t \frac{\d a}{a^3}
\end{equation*}

schnell ab, für $\Vert k \Vert \to \infty$ und $\frac{k_2}{k_1}$ in einer Umgebung von $s_0$.
\end{lemma}

\begin{proof}
Es sei
\begin{equation*}
    \Gamma_k = \left\{a\in [0,1], s \in [-1,1] \big| \tfrac{1}{2} \leq a|k| \leq 2 , \left|s-\tfrac{k_2}{k_1} \right| \leq \sqrt a
                   \right\}
\end{equation*}

Dann können wir dank \cref{eq:supp_psi} abschätzen

\begin{equation*}
    | \hat \psi_{ast} (k)| \leq C' a^{\frac{3}{4}} \chi_{\Gamma_k}
\end{equation*}

und nach Vorraussetzung gilt auch

\begin{equation*}
    |G(a,s,t)| \leq C_{N} a^{N} \condition{$\forall N \in \mathbb{N}$}
\end{equation*}

Außerdem sei
\begin{equation*}
    S = B(s_0,\Delta/2)
\end{equation*}

Um $\hat h(k)$ abzuschätzen, teilen wir es in den Bereich auf, in dem G(a,s,t) schnell abfällt, und in dem es nicht schnell abfällt:

\begin{dmath*}
    \hat h(k) =
    \underbrace{
    \int \limits_0^1 \int \limits_V \int \limits_{S}
    G(a,s,t) \hat \psi_{ast} (k)
        \d s \d t \frac{\d a}{a^3}}_{i)}
     +
    \underbrace{
     \int \limits_0^1 \int \limits_V \int \limits_{[-1,1]\setminus S}
    G(a,s,t) \hat \psi_{ast} (k)
        \d s \d t \frac{\d a}{a^3}}_{ii)}
\end{dmath*}


\emph{zu $i)$}

\begin{dmath*}
    i) \leq \int \limits_0^1 \int \limits_V \int \limits_{S}
    \left\lvert G(a,s,t)\right\rvert
    \left\lvert \hat \psi_{ast} (k) \right\rvert
        \d s \d t \frac{\d a}{a^3}
    \leq
    C_N C'
    \int \limits_0^1 \int \limits_V \int \limits_{S}
    a^{\frac{3}{4}} a^N \chi_{\Gamma_k} \d s \d t \frac{\d a}{a^3}
    \leq
    C_N \int \limits_{\frac{1}{2|k|}}^{\frac{2}{|k|}}
    a^{N-\frac{9}{4}} \d a
    \le C_N |k|^{-N+\frac{7}{4}}
\end{dmath*}

$i)$ fällt also schnell ab für $a \to 0$.


\emph{zu $ii)$}

\begin{dmath*}
    ii) \leq
     \int \limits_0^1 \int \limits_V \int \limits_{[-1,1]\setminus S}
    |G(a,s,t)| |\hat \psi_{ast} (k)|
        \d s \d t \frac{\d a}{a^3}
    \leq
    C' \int \limits_{0}^{1} \int \limits_{V} \int \limits_{[-1,1]\setminus S}
    |G(a,s,t)| \chi_{\Gamma_k} a^{\frac{3}{4}}
    \d s \d t \frac{\d a}{a^3}
\end{dmath*}

Für alle hinreichend großen $k$ ist aber $\Gamma_k \subset S$, also $\Gamma_k \cap [-1,1]\setminus S = \varnothing$ und demnach das Integral 0. Also

\begin{equation*}
    ii) = 0 \condition{für alle k groß genug}
\end{equation*}
\end{proof}


\begin{corollary}
    [Abschätzungen für $\left<\phi \psi_{a_0st},\psi_{a_1s't'}\right>$]
Sei $\phi \in C_0^\infty(B(t,\delta))$. Dann gilt für alle $N>0$

\begin{enumerate}
    \item Falls $0 \leq \sqrt{a_0} \leq \sqrt{a_1}\leq \delta \leq 1$
    \begin{equation*}
        |\left<\phi \psi_{a_0st},\psi_{a_1s't'}\right>| \leq
        C_N \left(1+\frac{a_1}{a_0}\right)^{-N}
        \left(1+\frac{|s-s'|^2}{a_1}\right)^{-N}
        \left(1+\frac{\Vert t-t' \Vert^2}{a_1}\right)^{-N}
    \end{equation*}
    \item Falls $0 \leq \sqrt{a_0} \leq \delta \leq \sqrt{a_1} \leq 1$
    \begin{equation*}
        |\left<\phi \psi_{a_0st},\psi_{a_1s't'}\right>| \leq
        C_N \left(1+\frac{a_1}{a_0}\right)^{-N}
        \left(1+\frac{|s-s'|^2}{\delta^2}\right)^{-N}
        \left(1+\frac{\Vert t-t' \Vert^2}{a_1}\right)^{-N}
    \end{equation*}
\end{enumerate}
\end{corollary}

\begin{proof}[von \ref{thm:main_theorem}]
Zunächst die einfachere Richtung, nämlich $WF(f)^c \subseteq \mathcal{D}$.
Wir nehmen also einen gerichteten regulären Punkt $((t_0,x_0),s_0) \in WF(f)^c$ und zeigen, dass er auch in $\mathcal{D}$ liegt. Dazu zerlegen wir $f$ zunächst wie folgt:
 Da $f$ bei $(t_0, x_0)$ in Richtung $s_0$ regulär ist, gibt es per Definition der Wellenfrontmenge ein $\phi \in C_0^\infty(\mathcal{R}^2)$ s.d. $\phi = 1$ in einer Umgebung von $(t_0, x_0)$ und für alle $N \in \mathbb{N}$ $\rwhat{\phi f} = O(1+|(\omega,k)|)^{-N}$ für $\frac{k}{\omega}$ in einer Umgebung von $s_0$. Dementsprechend ist $(1-\phi)f = 0$ in einer Umgebung von $(t_0, x_0)$ und es gilt

 \begin{equation}
     \mathcal{S}_f (a,s,(t',x')) = \left\langle \psi_{ast},\phi f \right\rangle
                                + \left\langle \psi_{ast},(1-\phi) f \right\rangle
 \label{eq:schlaue sache}
 \end{equation}

\begin{figure}[h]
\centering
%% Creator: Matplotlib, PGF backend
%%
%% To include the figure in your LaTeX document, write
%%   \input{<filename>.pgf}
%%
%% Make sure the required packages are loaded in your preamble
%%   \usepackage{pgf}
%%
%% Figures using additional raster images can only be included by \input if
%% they are in the same directory as the main LaTeX file. For loading figures
%% from other directories you can use the `import` package
%%   \usepackage{import}
%% and then include the figures with
%%   \import{<path to file>}{<filename>.pgf}
%%
%% Matplotlib used the following preamble
%%   \usepackage[utf8x]{inputenc}
%%   \usepackage[T1]{fontenc}
%%   \usepackage{amssymb}
%%
\begingroup%
\makeatletter%
\begin{pgfpicture}%
\pgfpathrectangle{\pgfpointorigin}{\pgfqpoint{4.000000in}{2.200000in}}%
\pgfusepath{use as bounding box, clip}%
\begin{pgfscope}%
\pgfsetbuttcap%
\pgfsetmiterjoin%
\definecolor{currentfill}{rgb}{1.000000,1.000000,1.000000}%
\pgfsetfillcolor{currentfill}%
\pgfsetlinewidth{0.000000pt}%
\definecolor{currentstroke}{rgb}{1.000000,1.000000,1.000000}%
\pgfsetstrokecolor{currentstroke}%
\pgfsetdash{}{0pt}%
\pgfpathmoveto{\pgfqpoint{0.000000in}{0.000000in}}%
\pgfpathlineto{\pgfqpoint{4.000000in}{0.000000in}}%
\pgfpathlineto{\pgfqpoint{4.000000in}{2.200000in}}%
\pgfpathlineto{\pgfqpoint{0.000000in}{2.200000in}}%
\pgfpathclose%
\pgfusepath{fill}%
\end{pgfscope}%
\begin{pgfscope}%
\pgfsetbuttcap%
\pgfsetmiterjoin%
\definecolor{currentfill}{rgb}{1.000000,1.000000,1.000000}%
\pgfsetfillcolor{currentfill}%
\pgfsetlinewidth{0.000000pt}%
\definecolor{currentstroke}{rgb}{0.000000,0.000000,0.000000}%
\pgfsetstrokecolor{currentstroke}%
\pgfsetstrokeopacity{0.000000}%
\pgfsetdash{}{0pt}%
\pgfpathmoveto{\pgfqpoint{0.500000in}{0.275000in}}%
\pgfpathlineto{\pgfqpoint{3.600000in}{0.275000in}}%
\pgfpathlineto{\pgfqpoint{3.600000in}{1.936000in}}%
\pgfpathlineto{\pgfqpoint{0.500000in}{1.936000in}}%
\pgfpathclose%
\pgfusepath{fill}%
\end{pgfscope}%
\begin{pgfscope}%
\pgfsetbuttcap%
\pgfsetroundjoin%
\definecolor{currentfill}{rgb}{0.000000,0.000000,0.000000}%
\pgfsetfillcolor{currentfill}%
\pgfsetlinewidth{0.803000pt}%
\definecolor{currentstroke}{rgb}{0.000000,0.000000,0.000000}%
\pgfsetstrokecolor{currentstroke}%
\pgfsetdash{}{0pt}%
\pgfsys@defobject{currentmarker}{\pgfqpoint{0.000000in}{-0.048611in}}{\pgfqpoint{0.000000in}{0.000000in}}{%
\pgfpathmoveto{\pgfqpoint{0.000000in}{0.000000in}}%
\pgfpathlineto{\pgfqpoint{0.000000in}{-0.048611in}}%
\pgfusepath{stroke,fill}%
}%
\begin{pgfscope}%
\pgfsys@transformshift{1.828571in}{0.344208in}%
\pgfsys@useobject{currentmarker}{}%
\end{pgfscope}%
\end{pgfscope}%
\begin{pgfscope}%
\pgftext[x=1.828571in,y=0.246986in,,top]{\rmfamily\fontsize{10.000000}{12.000000}\selectfont \(\displaystyle t_0\)}%
\end{pgfscope}%
\begin{pgfscope}%
\pgfsetbuttcap%
\pgfsetroundjoin%
\definecolor{currentfill}{rgb}{0.000000,0.000000,0.000000}%
\pgfsetfillcolor{currentfill}%
\pgfsetlinewidth{0.803000pt}%
\definecolor{currentstroke}{rgb}{0.000000,0.000000,0.000000}%
\pgfsetstrokecolor{currentstroke}%
\pgfsetdash{}{0pt}%
\pgfsys@defobject{currentmarker}{\pgfqpoint{0.000000in}{-0.048611in}}{\pgfqpoint{0.000000in}{0.000000in}}{%
\pgfpathmoveto{\pgfqpoint{0.000000in}{0.000000in}}%
\pgfpathlineto{\pgfqpoint{0.000000in}{-0.048611in}}%
\pgfusepath{stroke,fill}%
}%
\begin{pgfscope}%
\pgfsys@transformshift{1.939286in}{0.344208in}%
\pgfsys@useobject{currentmarker}{}%
\end{pgfscope}%
\end{pgfscope}%
\begin{pgfscope}%
\pgftext[x=1.939286in,y=0.246986in,,top]{\rmfamily\fontsize{10.000000}{12.000000}\selectfont \(\displaystyle t\)}%
\end{pgfscope}%
\begin{pgfscope}%
\pgfsetbuttcap%
\pgfsetroundjoin%
\definecolor{currentfill}{rgb}{0.000000,0.000000,0.000000}%
\pgfsetfillcolor{currentfill}%
\pgfsetlinewidth{0.803000pt}%
\definecolor{currentstroke}{rgb}{0.000000,0.000000,0.000000}%
\pgfsetstrokecolor{currentstroke}%
\pgfsetdash{}{0pt}%
\pgfsys@defobject{currentmarker}{\pgfqpoint{-0.048611in}{0.000000in}}{\pgfqpoint{0.000000in}{0.000000in}}{%
\pgfpathmoveto{\pgfqpoint{0.000000in}{0.000000in}}%
\pgfpathlineto{\pgfqpoint{-0.048611in}{0.000000in}}%
\pgfusepath{stroke,fill}%
}%
\begin{pgfscope}%
\pgfsys@transformshift{0.721429in}{1.036292in}%
\pgfsys@useobject{currentmarker}{}%
\end{pgfscope}%
\end{pgfscope}%
\begin{pgfscope}%
\pgftext[x=0.554779in,y=0.988464in,left,base]{\rmfamily\fontsize{10.000000}{12.000000}\selectfont 1}%
\end{pgfscope}%
\begin{pgfscope}%
\pgfpathrectangle{\pgfqpoint{0.500000in}{0.275000in}}{\pgfqpoint{3.100000in}{1.661000in}}%
\pgfusepath{clip}%
\pgfsetbuttcap%
\pgfsetroundjoin%
\pgfsetlinewidth{0.501875pt}%
\definecolor{currentstroke}{rgb}{0.501961,0.501961,0.501961}%
\pgfsetstrokecolor{currentstroke}%
\pgfsetdash{{1.850000pt}{0.800000pt}}{0.000000pt}%
\pgfpathmoveto{\pgfqpoint{1.939286in}{0.261111in}}%
\pgfpathlineto{\pgfqpoint{1.939286in}{1.949889in}}%
\pgfusepath{stroke}%
\end{pgfscope}%
\begin{pgfscope}%
\pgfpathrectangle{\pgfqpoint{0.500000in}{0.275000in}}{\pgfqpoint{3.100000in}{1.661000in}}%
\pgfusepath{clip}%
\pgfsetrectcap%
\pgfsetroundjoin%
\pgfsetlinewidth{0.501875pt}%
\definecolor{currentstroke}{rgb}{0.894118,0.101961,0.109804}%
\pgfsetstrokecolor{currentstroke}%
\pgfsetdash{}{0pt}%
\pgfpathmoveto{\pgfqpoint{0.486111in}{0.344208in}}%
\pgfpathlineto{\pgfqpoint{1.551509in}{0.345603in}}%
\pgfpathlineto{\pgfqpoint{1.586973in}{0.348858in}}%
\pgfpathlineto{\pgfqpoint{1.609138in}{0.353517in}}%
\pgfpathlineto{\pgfqpoint{1.626870in}{0.359911in}}%
\pgfpathlineto{\pgfqpoint{1.644602in}{0.369943in}}%
\pgfpathlineto{\pgfqpoint{1.657901in}{0.380785in}}%
\pgfpathlineto{\pgfqpoint{1.671200in}{0.395358in}}%
\pgfpathlineto{\pgfqpoint{1.684499in}{0.414585in}}%
\pgfpathlineto{\pgfqpoint{1.697798in}{0.439480in}}%
\pgfpathlineto{\pgfqpoint{1.711097in}{0.471104in}}%
\pgfpathlineto{\pgfqpoint{1.724396in}{0.510505in}}%
\pgfpathlineto{\pgfqpoint{1.737695in}{0.558631in}}%
\pgfpathlineto{\pgfqpoint{1.750994in}{0.616232in}}%
\pgfpathlineto{\pgfqpoint{1.768726in}{0.708477in}}%
\pgfpathlineto{\pgfqpoint{1.786458in}{0.818128in}}%
\pgfpathlineto{\pgfqpoint{1.808623in}{0.976524in}}%
\pgfpathlineto{\pgfqpoint{1.852953in}{1.327417in}}%
\pgfpathlineto{\pgfqpoint{1.875118in}{1.490059in}}%
\pgfpathlineto{\pgfqpoint{1.892850in}{1.597969in}}%
\pgfpathlineto{\pgfqpoint{1.906149in}{1.660357in}}%
\pgfpathlineto{\pgfqpoint{1.915015in}{1.691459in}}%
\pgfpathlineto{\pgfqpoint{1.923881in}{1.713383in}}%
\pgfpathlineto{\pgfqpoint{1.932747in}{1.725662in}}%
\pgfpathlineto{\pgfqpoint{1.937180in}{1.728093in}}%
\pgfpathlineto{\pgfqpoint{1.941613in}{1.728031in}}%
\pgfpathlineto{\pgfqpoint{1.946046in}{1.725475in}}%
\pgfpathlineto{\pgfqpoint{1.950479in}{1.720439in}}%
\pgfpathlineto{\pgfqpoint{1.959345in}{1.703051in}}%
\pgfpathlineto{\pgfqpoint{1.968211in}{1.676237in}}%
\pgfpathlineto{\pgfqpoint{1.977077in}{1.640567in}}%
\pgfpathlineto{\pgfqpoint{1.990376in}{1.572121in}}%
\pgfpathlineto{\pgfqpoint{2.008108in}{1.457967in}}%
\pgfpathlineto{\pgfqpoint{2.030273in}{1.290874in}}%
\pgfpathlineto{\pgfqpoint{2.083469in}{0.877388in}}%
\pgfpathlineto{\pgfqpoint{2.105634in}{0.732984in}}%
\pgfpathlineto{\pgfqpoint{2.123366in}{0.636529in}}%
\pgfpathlineto{\pgfqpoint{2.141098in}{0.557753in}}%
\pgfpathlineto{\pgfqpoint{2.158830in}{0.495768in}}%
\pgfpathlineto{\pgfqpoint{2.172129in}{0.459204in}}%
\pgfpathlineto{\pgfqpoint{2.185428in}{0.430056in}}%
\pgfpathlineto{\pgfqpoint{2.198727in}{0.407264in}}%
\pgfpathlineto{\pgfqpoint{2.212026in}{0.389778in}}%
\pgfpathlineto{\pgfqpoint{2.225325in}{0.376610in}}%
\pgfpathlineto{\pgfqpoint{2.238624in}{0.366877in}}%
\pgfpathlineto{\pgfqpoint{2.256356in}{0.357936in}}%
\pgfpathlineto{\pgfqpoint{2.274088in}{0.352285in}}%
\pgfpathlineto{\pgfqpoint{2.296253in}{0.348204in}}%
\pgfpathlineto{\pgfqpoint{2.331717in}{0.345389in}}%
\pgfpathlineto{\pgfqpoint{2.398212in}{0.344296in}}%
\pgfpathlineto{\pgfqpoint{2.876977in}{0.344208in}}%
\pgfpathlineto{\pgfqpoint{3.613889in}{0.344208in}}%
\pgfpathlineto{\pgfqpoint{3.613889in}{0.344208in}}%
\pgfusepath{stroke}%
\end{pgfscope}%
\begin{pgfscope}%
\pgfpathrectangle{\pgfqpoint{0.500000in}{0.275000in}}{\pgfqpoint{3.100000in}{1.661000in}}%
\pgfusepath{clip}%
\pgfsetrectcap%
\pgfsetroundjoin%
\pgfsetlinewidth{0.501875pt}%
\definecolor{currentstroke}{rgb}{0.215686,0.494118,0.721569}%
\pgfsetstrokecolor{currentstroke}%
\pgfsetdash{}{0pt}%
\pgfpathmoveto{\pgfqpoint{0.486111in}{0.344208in}}%
\pgfpathlineto{\pgfqpoint{1.409653in}{0.345316in}}%
\pgfpathlineto{\pgfqpoint{1.418519in}{0.348387in}}%
\pgfpathlineto{\pgfqpoint{1.427385in}{0.355340in}}%
\pgfpathlineto{\pgfqpoint{1.436251in}{0.368220in}}%
\pgfpathlineto{\pgfqpoint{1.445117in}{0.389050in}}%
\pgfpathlineto{\pgfqpoint{1.453983in}{0.419471in}}%
\pgfpathlineto{\pgfqpoint{1.462849in}{0.460375in}}%
\pgfpathlineto{\pgfqpoint{1.476148in}{0.540690in}}%
\pgfpathlineto{\pgfqpoint{1.493880in}{0.672835in}}%
\pgfpathlineto{\pgfqpoint{1.516045in}{0.839810in}}%
\pgfpathlineto{\pgfqpoint{1.529344in}{0.920125in}}%
\pgfpathlineto{\pgfqpoint{1.538210in}{0.961029in}}%
\pgfpathlineto{\pgfqpoint{1.547076in}{0.991450in}}%
\pgfpathlineto{\pgfqpoint{1.555942in}{1.012280in}}%
\pgfpathlineto{\pgfqpoint{1.564808in}{1.025160in}}%
\pgfpathlineto{\pgfqpoint{1.573674in}{1.032113in}}%
\pgfpathlineto{\pgfqpoint{1.582540in}{1.035184in}}%
\pgfpathlineto{\pgfqpoint{1.600272in}{1.036290in}}%
\pgfpathlineto{\pgfqpoint{2.074603in}{1.035184in}}%
\pgfpathlineto{\pgfqpoint{2.083469in}{1.032113in}}%
\pgfpathlineto{\pgfqpoint{2.092335in}{1.025160in}}%
\pgfpathlineto{\pgfqpoint{2.101201in}{1.012280in}}%
\pgfpathlineto{\pgfqpoint{2.110067in}{0.991450in}}%
\pgfpathlineto{\pgfqpoint{2.118933in}{0.961029in}}%
\pgfpathlineto{\pgfqpoint{2.127799in}{0.920125in}}%
\pgfpathlineto{\pgfqpoint{2.141098in}{0.839810in}}%
\pgfpathlineto{\pgfqpoint{2.158830in}{0.707665in}}%
\pgfpathlineto{\pgfqpoint{2.180995in}{0.540690in}}%
\pgfpathlineto{\pgfqpoint{2.194294in}{0.460375in}}%
\pgfpathlineto{\pgfqpoint{2.203160in}{0.419471in}}%
\pgfpathlineto{\pgfqpoint{2.212026in}{0.389050in}}%
\pgfpathlineto{\pgfqpoint{2.220892in}{0.368220in}}%
\pgfpathlineto{\pgfqpoint{2.229758in}{0.355340in}}%
\pgfpathlineto{\pgfqpoint{2.238624in}{0.348387in}}%
\pgfpathlineto{\pgfqpoint{2.247490in}{0.345316in}}%
\pgfpathlineto{\pgfqpoint{2.265222in}{0.344210in}}%
\pgfpathlineto{\pgfqpoint{3.613889in}{0.344208in}}%
\pgfpathlineto{\pgfqpoint{3.613889in}{0.344208in}}%
\pgfusepath{stroke}%
\end{pgfscope}%
\begin{pgfscope}%
\pgfpathrectangle{\pgfqpoint{0.500000in}{0.275000in}}{\pgfqpoint{3.100000in}{1.661000in}}%
\pgfusepath{clip}%
\pgfsetrectcap%
\pgfsetroundjoin%
\pgfsetlinewidth{0.501875pt}%
\definecolor{currentstroke}{rgb}{0.301961,0.686275,0.290196}%
\pgfsetstrokecolor{currentstroke}%
\pgfsetdash{}{0pt}%
\pgfpathmoveto{\pgfqpoint{0.486111in}{1.036292in}}%
\pgfpathlineto{\pgfqpoint{1.409653in}{1.035184in}}%
\pgfpathlineto{\pgfqpoint{1.418519in}{1.032113in}}%
\pgfpathlineto{\pgfqpoint{1.427385in}{1.025160in}}%
\pgfpathlineto{\pgfqpoint{1.436251in}{1.012280in}}%
\pgfpathlineto{\pgfqpoint{1.445117in}{0.991450in}}%
\pgfpathlineto{\pgfqpoint{1.453983in}{0.961029in}}%
\pgfpathlineto{\pgfqpoint{1.462849in}{0.920125in}}%
\pgfpathlineto{\pgfqpoint{1.476148in}{0.839810in}}%
\pgfpathlineto{\pgfqpoint{1.493880in}{0.707665in}}%
\pgfpathlineto{\pgfqpoint{1.516045in}{0.540690in}}%
\pgfpathlineto{\pgfqpoint{1.529344in}{0.460375in}}%
\pgfpathlineto{\pgfqpoint{1.538210in}{0.419471in}}%
\pgfpathlineto{\pgfqpoint{1.547076in}{0.389050in}}%
\pgfpathlineto{\pgfqpoint{1.555942in}{0.368220in}}%
\pgfpathlineto{\pgfqpoint{1.564808in}{0.355340in}}%
\pgfpathlineto{\pgfqpoint{1.573674in}{0.348387in}}%
\pgfpathlineto{\pgfqpoint{1.582540in}{0.345316in}}%
\pgfpathlineto{\pgfqpoint{1.600272in}{0.344210in}}%
\pgfpathlineto{\pgfqpoint{2.074603in}{0.345316in}}%
\pgfpathlineto{\pgfqpoint{2.083469in}{0.348387in}}%
\pgfpathlineto{\pgfqpoint{2.092335in}{0.355340in}}%
\pgfpathlineto{\pgfqpoint{2.101201in}{0.368220in}}%
\pgfpathlineto{\pgfqpoint{2.110067in}{0.389050in}}%
\pgfpathlineto{\pgfqpoint{2.118933in}{0.419471in}}%
\pgfpathlineto{\pgfqpoint{2.127799in}{0.460375in}}%
\pgfpathlineto{\pgfqpoint{2.141098in}{0.540690in}}%
\pgfpathlineto{\pgfqpoint{2.158830in}{0.672835in}}%
\pgfpathlineto{\pgfqpoint{2.180995in}{0.839810in}}%
\pgfpathlineto{\pgfqpoint{2.194294in}{0.920125in}}%
\pgfpathlineto{\pgfqpoint{2.203160in}{0.961029in}}%
\pgfpathlineto{\pgfqpoint{2.212026in}{0.991450in}}%
\pgfpathlineto{\pgfqpoint{2.220892in}{1.012280in}}%
\pgfpathlineto{\pgfqpoint{2.229758in}{1.025160in}}%
\pgfpathlineto{\pgfqpoint{2.238624in}{1.032113in}}%
\pgfpathlineto{\pgfqpoint{2.247490in}{1.035184in}}%
\pgfpathlineto{\pgfqpoint{2.265222in}{1.036290in}}%
\pgfpathlineto{\pgfqpoint{3.613889in}{1.036292in}}%
\pgfpathlineto{\pgfqpoint{3.613889in}{1.036292in}}%
\pgfusepath{stroke}%
\end{pgfscope}%
\begin{pgfscope}%
\pgfsetrectcap%
\pgfsetmiterjoin%
\pgfsetlinewidth{0.501875pt}%
\definecolor{currentstroke}{rgb}{0.000000,0.000000,0.000000}%
\pgfsetstrokecolor{currentstroke}%
\pgfsetdash{}{0pt}%
\pgfpathmoveto{\pgfqpoint{0.721429in}{0.275000in}}%
\pgfpathlineto{\pgfqpoint{0.721429in}{1.936000in}}%
\pgfusepath{stroke}%
\end{pgfscope}%
\begin{pgfscope}%
\pgfsetrectcap%
\pgfsetmiterjoin%
\pgfsetlinewidth{0.501875pt}%
\definecolor{currentstroke}{rgb}{0.000000,0.000000,0.000000}%
\pgfsetstrokecolor{currentstroke}%
\pgfsetdash{}{0pt}%
\pgfpathmoveto{\pgfqpoint{0.500000in}{0.344208in}}%
\pgfpathlineto{\pgfqpoint{3.600000in}{0.344208in}}%
\pgfusepath{stroke}%
\end{pgfscope}%
\begin{pgfscope}%
\pgfsetroundcap%
\pgfsetroundjoin%
\pgfsetlinewidth{0.501875pt}%
\definecolor{currentstroke}{rgb}{0.000000,0.000000,0.000000}%
\pgfsetstrokecolor{currentstroke}%
\pgfsetdash{}{0pt}%
\pgfpathmoveto{\pgfqpoint{0.721429in}{1.942121in}}%
\pgfpathquadraticcurveto{\pgfqpoint{0.721429in}{1.942943in}}{\pgfqpoint{0.721429in}{1.936000in}}%
\pgfusepath{stroke}%
\end{pgfscope}%
\begin{pgfscope}%
\pgfsetroundcap%
\pgfsetroundjoin%
\pgfsetlinewidth{0.501875pt}%
\definecolor{currentstroke}{rgb}{0.000000,0.000000,0.000000}%
\pgfsetstrokecolor{currentstroke}%
\pgfsetdash{}{0pt}%
\pgfpathmoveto{\pgfqpoint{0.693651in}{1.886565in}}%
\pgfpathlineto{\pgfqpoint{0.721429in}{1.942121in}}%
\pgfpathlineto{\pgfqpoint{0.749206in}{1.886565in}}%
\pgfusepath{stroke}%
\end{pgfscope}%
\begin{pgfscope}%
\pgftext[x=0.721429in,y=2.005444in,,bottom]{\rmfamily\fontsize{10.000000}{12.000000}\selectfont \(\displaystyle  f\)}%
\end{pgfscope}%
\begin{pgfscope}%
\pgfsetroundcap%
\pgfsetroundjoin%
\pgfsetlinewidth{0.501875pt}%
\definecolor{currentstroke}{rgb}{0.000000,0.000000,0.000000}%
\pgfsetstrokecolor{currentstroke}%
\pgfsetdash{}{0pt}%
\pgfpathmoveto{\pgfqpoint{3.606114in}{0.344208in}}%
\pgfpathquadraticcurveto{\pgfqpoint{3.606939in}{0.344208in}}{\pgfqpoint{3.600000in}{0.344208in}}%
\pgfusepath{stroke}%
\end{pgfscope}%
\begin{pgfscope}%
\pgfsetroundcap%
\pgfsetroundjoin%
\pgfsetlinewidth{0.501875pt}%
\definecolor{currentstroke}{rgb}{0.000000,0.000000,0.000000}%
\pgfsetstrokecolor{currentstroke}%
\pgfsetdash{}{0pt}%
\pgfpathmoveto{\pgfqpoint{3.550559in}{0.371986in}}%
\pgfpathlineto{\pgfqpoint{3.606114in}{0.344208in}}%
\pgfpathlineto{\pgfqpoint{3.550559in}{0.316431in}}%
\pgfusepath{stroke}%
\end{pgfscope}%
\begin{pgfscope}%
\pgftext[x=3.669444in,y=0.344208in,left,]{\rmfamily\fontsize{10.000000}{12.000000}\selectfont \(\displaystyle t\)}%
\end{pgfscope}%
\begin{pgfscope}%
\pgfsetbuttcap%
\pgfsetmiterjoin%
\definecolor{currentfill}{rgb}{1.000000,1.000000,1.000000}%
\pgfsetfillcolor{currentfill}%
\pgfsetfillopacity{0.800000}%
\pgfsetlinewidth{0.501875pt}%
\definecolor{currentstroke}{rgb}{0.800000,0.800000,0.800000}%
\pgfsetstrokecolor{currentstroke}%
\pgfsetstrokeopacity{0.800000}%
\pgfsetdash{}{0pt}%
\pgfpathmoveto{\pgfqpoint{2.736382in}{1.243871in}}%
\pgfpathlineto{\pgfqpoint{3.502778in}{1.243871in}}%
\pgfpathquadraticcurveto{\pgfqpoint{3.530556in}{1.243871in}}{\pgfqpoint{3.530556in}{1.271648in}}%
\pgfpathlineto{\pgfqpoint{3.530556in}{1.838778in}}%
\pgfpathquadraticcurveto{\pgfqpoint{3.530556in}{1.866556in}}{\pgfqpoint{3.502778in}{1.866556in}}%
\pgfpathlineto{\pgfqpoint{2.736382in}{1.866556in}}%
\pgfpathquadraticcurveto{\pgfqpoint{2.708604in}{1.866556in}}{\pgfqpoint{2.708604in}{1.838778in}}%
\pgfpathlineto{\pgfqpoint{2.708604in}{1.271648in}}%
\pgfpathquadraticcurveto{\pgfqpoint{2.708604in}{1.243871in}}{\pgfqpoint{2.736382in}{1.243871in}}%
\pgfpathclose%
\pgfusepath{stroke,fill}%
\end{pgfscope}%
\begin{pgfscope}%
\pgfsetrectcap%
\pgfsetroundjoin%
\pgfsetlinewidth{0.501875pt}%
\definecolor{currentstroke}{rgb}{0.894118,0.101961,0.109804}%
\pgfsetstrokecolor{currentstroke}%
\pgfsetdash{}{0pt}%
\pgfpathmoveto{\pgfqpoint{2.764160in}{1.762389in}}%
\pgfpathlineto{\pgfqpoint{3.041937in}{1.762389in}}%
\pgfusepath{stroke}%
\end{pgfscope}%
\begin{pgfscope}%
\pgftext[x=3.153049in,y=1.713778in,left,base]{\rmfamily\fontsize{10.000000}{12.000000}\selectfont \(\displaystyle \psi_{ast}\)}%
\end{pgfscope}%
\begin{pgfscope}%
\pgfsetrectcap%
\pgfsetroundjoin%
\pgfsetlinewidth{0.501875pt}%
\definecolor{currentstroke}{rgb}{0.215686,0.494118,0.721569}%
\pgfsetstrokecolor{currentstroke}%
\pgfsetdash{}{0pt}%
\pgfpathmoveto{\pgfqpoint{2.764160in}{1.568716in}}%
\pgfpathlineto{\pgfqpoint{3.041937in}{1.568716in}}%
\pgfusepath{stroke}%
\end{pgfscope}%
\begin{pgfscope}%
\pgftext[x=3.153049in,y=1.520105in,left,base]{\rmfamily\fontsize{10.000000}{12.000000}\selectfont \(\displaystyle \phi\)}%
\end{pgfscope}%
\begin{pgfscope}%
\pgfsetrectcap%
\pgfsetroundjoin%
\pgfsetlinewidth{0.501875pt}%
\definecolor{currentstroke}{rgb}{0.301961,0.686275,0.290196}%
\pgfsetstrokecolor{currentstroke}%
\pgfsetdash{}{0pt}%
\pgfpathmoveto{\pgfqpoint{2.764160in}{1.375043in}}%
\pgfpathlineto{\pgfqpoint{3.041937in}{1.375043in}}%
\pgfusepath{stroke}%
\end{pgfscope}%
\begin{pgfscope}%
\pgftext[x=3.153049in,y=1.326432in,left,base]{\rmfamily\fontsize{10.000000}{12.000000}\selectfont \(\displaystyle 1-\phi\)}%
\end{pgfscope}%
\end{pgfpicture}%
\makeatother%
\endgroup%

\caption{Die Zerlegung von $f$ um $(t_0,x_0)$ herum visualisiert}
\label{fig:smart_decomposition}
\end{figure}

Da $(1-\phi)f$ in einer Umgebung von $(t_0, x_0)$ verschwindet und nach \cref{prop:shearlets_decay_rapidly} Shearlets außerhalb von $(t',x')$ schnell abfallen für $a \to 0$ fällt auch der zweite Term von \cref{eq:schlaue sache}
für $(t',x') = (t_0,x_0)$ schnell ab. Für den ersten Term überzeugen wir uns anhand von \cref{fig:supp_psi_hat,eq:supp_psi}, dass für $a$ klein genug $supp(\hat\psi_{ast})$ schließlich in jedem noch so kleinen Kegel um $s$ liegt. In einem solchen um $s_0$ fällt aber $\rwhat{\phi f}$ rapide ab nach Vorraussetzung und damit auch der erste Term in \cref{eq:schlaue sache}.

Die beiden entscheidenden Zutaten waren hier also die Tatsache, dass die Shearlets außerhalb von $(t',x')$ rapide Abfallen und damit bei immer feineren Skalen $a$ immer besser lokalisiert werden sowie die Tatsache, dass für $a \to 0$ der Träger im Frequenzbereich in immer engeren Kegeln liegt.

Deutlich schwieriger ist die umgekehrte Inklusion, nämlich dass die Shearlettransformation tatsächlich die ganze Wellenfrontmenge erkennt. Hier geht jetzt auch die Reproduktionseigenschaft der Transformation ein, eben genau dass sie alles sieht.

Für die umgekehrte Inklusion $\mathcal{D} \subseteq WF(f)^c$ haben wir zu zeigen, dass falls $\mathcal{S}_f (a,s,(t',x'))$ schnell abfällt für $a \to 0$ einer Umgebung $U$ von $(s_0, (t_0, x_0))$ dann auch $\rwhat{\phi f} (\omega,k)$ schnell abfällt für $\Vert(\omega,k)\Vert \to \infty$ für $\frac{k}{\omega}$ in einer Umgebung von $s_0$ und ein $\phi$ getragen in einer Umgebung von $(t_0, x_0)$.

Sei also $\phi \in C^\infty(\pi(U))$, wobei $\pi$ wieder die Projektion auf die Ortskomponente ist. Dann ist

\begin{dmath*}
    \rwhat{\phi f} (\omega, k)
    =
    \int \phi f e^{-i\omega t+ikx} \d t \d x \\
    \stackrel{\ref{thm:shearlets_reproduzieren}}{=}
    \iint \left\langle \psi_{as(t',x')},\phi f \right\rangle
        \psi_{as(t',x')} (t,x) \d \mu (as(t',x'))
        e^{\cdots} \d t \d x
    =
    \int \left\langle \psi_{as(t',x')},\phi f \right\rangle
    \hat\psi_{as(t',x')}(\omega,k) \d \mu(\cdots)
    = \kern -1em
    \underbrace{
        \int \limits_{U \times [0,1]} \kern -1em
        \left\langle \psi_{as(t',x')},\phi f \right\rangle
        \hat\psi_{as(t',x')}(\omega,k) \d \mu(\cdots)
    }_{i)}
    + \kern -1em
    \underbrace{
        \int \limits_{U^c \times [0,1]} \kern -1em
        \left\langle \psi_{as(t',x')},\phi f \right\rangle
        \hat\psi_{as(t',x')}(\omega,k) \d \mu(\cdots)
    }_{ii)}
\end{dmath*}

\emph{zu $ii)$}
Für $\Vert (\omega,k)\Vert \to \infty$ ist $\hat \psi$ nur für $a \to 0$

\end{proof}



% section beweis_von_thm:main_theorem (end)


%!TEX root = main.tex
%!TEX spellcheck=de_DE
%%%%%%%%%%%%%%%%%%%%%%%%%%%%%%%%%%%%%%%%%%%%%%%%%%%%%%%%%%%%%%%%%%%%%%%%%%%%%%%
% % Section 2
%%%%%%%%%%%%%%%%%%%%%%%%%%%%%%%%%%%%%%%%%%%%%%%%%%%%%%%%%%%%%%%%%%%%%%%%%%%%%%%
\section{\texorpdfstring{Zwei nützliche Substitionen für  $\left<\psi_{ast}, f\right>$}{zwei nützliche Substitutionen}}
\label{sec:substitutionen}

\todo[color=green]{mit $(\omega, k)$ als Variablennamen arbeiten, um zum Rest des Textes zu passen, oder mit $(\xi_1, \xi_2)$ um zu \textcite{Kutyniok2008} zu passen?}

Zunächst werden wir zwei verschiedene Ausdrücke für $\left<\psi_{ast}, f\right>$
im Fourierraum herleiten, welche fast immer Ausgangspunkt für unsere Abschätzungen sein werden.

Sei also $\psi$ ein Shearlet wie in \cref{cor:psi_hat}. Sei $f$ die zu
analysierende fouriertransformierbare Funktion (oder Distribution) in
$\mathcal{S}' (\mathbb{R}^2)$. Dann ist $\mathcal{S}_f (ast)$ gegeben durch

\begin{align*}
\left< \psi_{ast}, f \right> &= \left<\hat\psi_{ast}, \hat f\right> \\
 &= \int a^{\frac{3}{4}} e^{-i \xi \cdot t} \hat \psi_1(a \xi_1)
    \hat \psi_2 \left(a^{-\frac{1}{2}} \left(\frac{\xi_2}{\xi_1} - s\right)\right)
    \hat f (\xi) \d \xi
\end{align*}

\todo{entscheiden, was mit dem fehlenden Faktor $\frac{1}{(2 \pi)^n}$ geschieht}
und nach "`entscheren"' und "`deskalieren"', also der Substitution

\begin{equation}
\begin{aligned}[c]
a \xi_1 &= k_1\\
a^{-\frac{1}{2}} \left(\frac{\xi_2}{\xi_1} - s\right) &=\frac{k_2}{k_1}\\
\end{aligned}
\qquad\Longleftrightarrow\qquad
\begin{aligned}[c]
\xi_1 &= \frac{k_1}{a}\\
\xi_2 &= \frac{k_1 s}{a} + a^{-\frac{1}{2}} k_2\\
\end{aligned}
\label{eq:substitution1_coords}
\end{equation}

\begin{equation*}
\Rightarrow
\d \xi_1 \d \xi_2 = a^{-\frac{3}{2}} \d k_1 \d k_2
\end{equation*}

ergibt sich folgendes für $\left<\psi_{ast}, f\right>$:

\todo{\texttt{owntag} fixen}

\begin{align}
    \left\langle\psi_{ast},f\right\rangle
    &=  \left\langle\hat\psi_{ast},\hat f\right\rangle \nonumber \\
    &=  \iint a^{-\frac{3}{4}}~\hat \psi_1(k_1) ~\hat \psi_2 \left(\tfrac{k_2}{k_1}\right)
    ~\hat f \left(\tfrac{k_1}{a}, \tfrac{k_1 s}{a} + \tfrac{k_2}{\sqrt{a}}\right)
    ~e^{-i\frac{k_1}{a}(t_1+t_2 s) - i \frac{k_2 t_2}{\sqrt a}}
    \d k_1 \d k_2
\owntag[substitution1]{Substitution 1}
\end{align}

Wie man sieht, tauchen in den Argumente von $\hat\psi_1$ und $\hat\psi_2$ nun die Parameter $a,s,t$ gar nicht mehr auf, und wir können nun verwenden, was wir aus \ref{sec:shearlets} über deren Träger wissen.
Alternativ und mit ähnlichem Ergebniss kann auch folgende Substitution

\begin{equation}
\begin{aligned}[c]
a \xi_1 &= k_1\\
a^{-\frac{1}{2}} \left(\frac{\xi_2}{\xi_1} - s\right) &= k_2\\
\end{aligned}
\qquad\Longleftrightarrow\qquad
\begin{aligned}[c]
\xi_1 &= \frac{k_1}{a}\\
\xi_2 &= \left( a^{\frac{1}{2}} k_2 +s \right) \frac{k_1}{a}\\
\end{aligned}
\label{eq:substitution2_coords}
\end{equation}

\begin{equation*}
\Rightarrow
\d \xi_1 \d \xi_2 = a^{-\frac{3}{2}} k_1 \d k_1 \d k_2
\end{equation*}

gewählt werden, wodurch wieder alle Parameter $(a,s,t)$ aus den Argumenten von $\hat\psi_1, \hat\psi_2$
verschwinden und sich

\begin{align}
    \left<\psi_{ast},f\right>
    =  \iint a^{-\frac{3}{4}}~ k_1~ \hat \psi_1(k_1)~ \hat \psi_2 (k_2)~
    \hat f \left(\tfrac{k_1}{a}, k_1 \left(a^{-\frac{1}{2}}k_2 + s a^{-1}\right)\right)
    ~e^{-i k_1 \left(\frac{t_1+s t_2}{a} + \frac{k_2 t_2}{\sqrt{a}}\right)}
    \d k_1 \d k_2
\owntag[substitution2]{Substitution 2}
\end{align}

ergibt. Dabei ist zu beachten, dass diese Substitution zulässig ist, obwohl sie
die Orientierung \emph{nicht} erhält und \emph{keine} Bijektion ist. Aber
der kritische Bereich, nämlich $\xi_1 = 0$, liegt nicht im Träger von $\rwhat{\psi}$.

Beiden Substitution gemein ist aber, dass danach
$0=\omega \notin supp (\hat\psi)$ und dass $supp (\psi)$ sowohl in $k$ als auch in $\omega$ beschränkt ist. $\omega$ kann also sowohl nach unten als auch nach oben durch eine Konstante abgeschätzt werden, wannimmer dies der Sache dienlich ist. Auch $k$ kann zumindest nach oben immer durche eine Konstante abgeschätzt werden.

\begin{figure}[h]
    \centering
    \begin{minipage}{0.5\textwidth}
        \centering
        \resizebox{\textwidth}{!}{%% Creator: Matplotlib, PGF backend
%%
%% To include the figure in your LaTeX document, write
%%   \input{<filename>.pgf}
%%
%% Make sure the required packages are loaded in your preamble
%%   \usepackage{pgf}
%%
%% Figures using additional raster images can only be included by \input if
%% they are in the same directory as the main LaTeX file. For loading figures
%% from other directories you can use the `import` package
%%   \usepackage{import}
%% and then include the figures with
%%   \import{<path to file>}{<filename>.pgf}
%%
%% Matplotlib used the following preamble
%%   \usepackage[utf8x]{inputenc}
%%   \usepackage[T1]{fontenc}
%%   \usepackage{amssymb}
%%
\begingroup%
\makeatletter%
\begin{pgfpicture}%
\pgfpathrectangle{\pgfpointorigin}{\pgfqpoint{4.000000in}{2.800000in}}%
\pgfusepath{use as bounding box, clip}%
\begin{pgfscope}%
\pgfsetbuttcap%
\pgfsetmiterjoin%
\definecolor{currentfill}{rgb}{1.000000,1.000000,1.000000}%
\pgfsetfillcolor{currentfill}%
\pgfsetlinewidth{0.000000pt}%
\definecolor{currentstroke}{rgb}{1.000000,1.000000,1.000000}%
\pgfsetstrokecolor{currentstroke}%
\pgfsetdash{}{0pt}%
\pgfpathmoveto{\pgfqpoint{0.000000in}{0.000000in}}%
\pgfpathlineto{\pgfqpoint{4.000000in}{0.000000in}}%
\pgfpathlineto{\pgfqpoint{4.000000in}{2.800000in}}%
\pgfpathlineto{\pgfqpoint{0.000000in}{2.800000in}}%
\pgfpathclose%
\pgfusepath{fill}%
\end{pgfscope}%
\begin{pgfscope}%
\pgfsetbuttcap%
\pgfsetmiterjoin%
\definecolor{currentfill}{rgb}{1.000000,1.000000,1.000000}%
\pgfsetfillcolor{currentfill}%
\pgfsetlinewidth{0.000000pt}%
\definecolor{currentstroke}{rgb}{0.000000,0.000000,0.000000}%
\pgfsetstrokecolor{currentstroke}%
\pgfsetstrokeopacity{0.000000}%
\pgfsetdash{}{0pt}%
\pgfpathmoveto{\pgfqpoint{0.198611in}{0.198611in}}%
\pgfpathlineto{\pgfqpoint{3.801389in}{0.198611in}}%
\pgfpathlineto{\pgfqpoint{3.801389in}{2.601389in}}%
\pgfpathlineto{\pgfqpoint{0.198611in}{2.601389in}}%
\pgfpathclose%
\pgfusepath{fill}%
\end{pgfscope}%
\begin{pgfscope}%
\pgfpathrectangle{\pgfqpoint{0.198611in}{0.198611in}}{\pgfqpoint{3.602778in}{2.402778in}}%
\pgfusepath{clip}%
\pgfsetbuttcap%
\pgfsetmiterjoin%
\definecolor{currentfill}{rgb}{0.500000,0.500000,0.500000}%
\pgfsetfillcolor{currentfill}%
\pgfsetfillopacity{0.500000}%
\pgfsetlinewidth{0.501875pt}%
\definecolor{currentstroke}{rgb}{0.000000,0.000000,0.000000}%
\pgfsetstrokecolor{currentstroke}%
\pgfsetdash{}{0pt}%
\pgfpathmoveto{\pgfqpoint{1.963972in}{1.433372in}}%
\pgfpathlineto{\pgfqpoint{2.036028in}{1.433372in}}%
\pgfpathlineto{\pgfqpoint{2.144111in}{1.533488in}}%
\pgfpathlineto{\pgfqpoint{1.855889in}{1.533488in}}%
\pgfpathclose%
\pgfusepath{stroke,fill}%
\end{pgfscope}%
\begin{pgfscope}%
\pgfpathrectangle{\pgfqpoint{0.198611in}{0.198611in}}{\pgfqpoint{3.602778in}{2.402778in}}%
\pgfusepath{clip}%
\pgfsetbuttcap%
\pgfsetmiterjoin%
\definecolor{currentfill}{rgb}{0.500000,0.500000,0.500000}%
\pgfsetfillcolor{currentfill}%
\pgfsetfillopacity{0.500000}%
\pgfsetlinewidth{0.501875pt}%
\definecolor{currentstroke}{rgb}{0.000000,0.000000,0.000000}%
\pgfsetstrokecolor{currentstroke}%
\pgfsetdash{}{0pt}%
\pgfpathmoveto{\pgfqpoint{2.036028in}{1.366628in}}%
\pgfpathlineto{\pgfqpoint{1.963972in}{1.366628in}}%
\pgfpathlineto{\pgfqpoint{1.855889in}{1.266512in}}%
\pgfpathlineto{\pgfqpoint{2.144111in}{1.266512in}}%
\pgfpathclose%
\pgfusepath{stroke,fill}%
\end{pgfscope}%
\begin{pgfscope}%
\pgfpathrectangle{\pgfqpoint{0.198611in}{0.198611in}}{\pgfqpoint{3.602778in}{2.402778in}}%
\pgfusepath{clip}%
\pgfsetbuttcap%
\pgfsetmiterjoin%
\definecolor{currentfill}{rgb}{0.500000,0.500000,0.500000}%
\pgfsetfillcolor{currentfill}%
\pgfsetfillopacity{0.500000}%
\pgfsetlinewidth{0.501875pt}%
\definecolor{currentstroke}{rgb}{0.000000,0.000000,0.000000}%
\pgfsetstrokecolor{currentstroke}%
\pgfsetdash{}{0pt}%
\pgfpathmoveto{\pgfqpoint{2.246348in}{1.733719in}}%
\pgfpathlineto{\pgfqpoint{2.474208in}{1.733719in}}%
\pgfpathlineto{\pgfqpoint{3.896830in}{2.734877in}}%
\pgfpathlineto{\pgfqpoint{2.985392in}{2.734877in}}%
\pgfpathclose%
\pgfusepath{stroke,fill}%
\end{pgfscope}%
\begin{pgfscope}%
\pgfpathrectangle{\pgfqpoint{0.198611in}{0.198611in}}{\pgfqpoint{3.602778in}{2.402778in}}%
\pgfusepath{clip}%
\pgfsetbuttcap%
\pgfsetmiterjoin%
\definecolor{currentfill}{rgb}{0.500000,0.500000,0.500000}%
\pgfsetfillcolor{currentfill}%
\pgfsetfillopacity{0.500000}%
\pgfsetlinewidth{0.501875pt}%
\definecolor{currentstroke}{rgb}{0.000000,0.000000,0.000000}%
\pgfsetstrokecolor{currentstroke}%
\pgfsetdash{}{0pt}%
\pgfpathmoveto{\pgfqpoint{1.753652in}{1.066281in}}%
\pgfpathlineto{\pgfqpoint{1.525792in}{1.066281in}}%
\pgfpathlineto{\pgfqpoint{0.103170in}{0.065123in}}%
\pgfpathlineto{\pgfqpoint{1.014608in}{0.065123in}}%
\pgfpathclose%
\pgfusepath{stroke,fill}%
\end{pgfscope}%
\begin{pgfscope}%
\pgfpathrectangle{\pgfqpoint{0.198611in}{0.198611in}}{\pgfqpoint{3.602778in}{2.402778in}}%
\pgfusepath{clip}%
\pgfsetbuttcap%
\pgfsetroundjoin%
\pgfsetlinewidth{0.501875pt}%
\definecolor{currentstroke}{rgb}{0.501961,0.501961,0.501961}%
\pgfsetstrokecolor{currentstroke}%
\pgfsetdash{{1.850000pt}{0.800000pt}}{0.000000pt}%
\pgfpathmoveto{\pgfqpoint{0.688006in}{0.184722in}}%
\pgfpathlineto{\pgfqpoint{3.311994in}{2.615278in}}%
\pgfpathlineto{\pgfqpoint{3.311994in}{2.615278in}}%
\pgfusepath{stroke}%
\end{pgfscope}%
\begin{pgfscope}%
\pgfpathrectangle{\pgfqpoint{0.198611in}{0.198611in}}{\pgfqpoint{3.602778in}{2.402778in}}%
\pgfusepath{clip}%
\pgfsetbuttcap%
\pgfsetroundjoin%
\pgfsetlinewidth{0.501875pt}%
\definecolor{currentstroke}{rgb}{0.501961,0.501961,0.501961}%
\pgfsetstrokecolor{currentstroke}%
\pgfsetdash{{1.850000pt}{0.800000pt}}{0.000000pt}%
\pgfpathmoveto{\pgfqpoint{0.688006in}{2.615278in}}%
\pgfpathlineto{\pgfqpoint{3.311994in}{0.184722in}}%
\pgfpathlineto{\pgfqpoint{3.311994in}{0.184722in}}%
\pgfusepath{stroke}%
\end{pgfscope}%
\begin{pgfscope}%
\pgfsetrectcap%
\pgfsetmiterjoin%
\pgfsetlinewidth{0.501875pt}%
\definecolor{currentstroke}{rgb}{0.000000,0.000000,0.000000}%
\pgfsetstrokecolor{currentstroke}%
\pgfsetdash{}{0pt}%
\pgfpathmoveto{\pgfqpoint{2.000000in}{0.198611in}}%
\pgfpathlineto{\pgfqpoint{2.000000in}{2.601389in}}%
\pgfusepath{stroke}%
\end{pgfscope}%
\begin{pgfscope}%
\pgfsetrectcap%
\pgfsetmiterjoin%
\pgfsetlinewidth{0.501875pt}%
\definecolor{currentstroke}{rgb}{0.000000,0.000000,0.000000}%
\pgfsetstrokecolor{currentstroke}%
\pgfsetdash{}{0pt}%
\pgfpathmoveto{\pgfqpoint{0.198611in}{1.400000in}}%
\pgfpathlineto{\pgfqpoint{3.801389in}{1.400000in}}%
\pgfusepath{stroke}%
\end{pgfscope}%
\begin{pgfscope}%
\pgfsetroundcap%
\pgfsetroundjoin%
\pgfsetlinewidth{0.501875pt}%
\definecolor{currentstroke}{rgb}{0.000000,0.000000,0.000000}%
\pgfsetstrokecolor{currentstroke}%
\pgfsetdash{}{0pt}%
\pgfpathmoveto{\pgfqpoint{3.032230in}{2.375700in}}%
\pgfpathquadraticcurveto{\pgfqpoint{2.526526in}{1.930390in}}{\pgfqpoint{2.026649in}{1.490210in}}%
\pgfusepath{stroke}%
\end{pgfscope}%
\begin{pgfscope}%
\pgfsetroundcap%
\pgfsetroundjoin%
\pgfsetlinewidth{0.501875pt}%
\definecolor{currentstroke}{rgb}{0.000000,0.000000,0.000000}%
\pgfsetstrokecolor{currentstroke}%
\pgfsetdash{}{0pt}%
\pgfpathmoveto{\pgfqpoint{2.086701in}{1.506078in}}%
\pgfpathlineto{\pgfqpoint{2.026649in}{1.490210in}}%
\pgfpathlineto{\pgfqpoint{2.049986in}{1.547773in}}%
\pgfusepath{stroke}%
\end{pgfscope}%
\begin{pgfscope}%
\pgftext[x=3.080833in,y=2.401157in,left,base]{\rmfamily\fontsize{10.000000}{12.000000}\selectfont \(\displaystyle {\cdot}\)}%
\end{pgfscope}%
\begin{pgfscope}%
\pgftext[x=2.288222in,y=1.600231in,left,base]{\rmfamily\fontsize{10.000000}{12.000000}\selectfont Substitution 1}%
\end{pgfscope}%
\begin{pgfscope}%
\pgfsetroundcap%
\pgfsetroundjoin%
\pgfsetlinewidth{0.501875pt}%
\definecolor{currentstroke}{rgb}{0.000000,0.000000,0.000000}%
\pgfsetstrokecolor{currentstroke}%
\pgfsetdash{}{0pt}%
\pgfpathmoveto{\pgfqpoint{2.000000in}{2.607510in}}%
\pgfpathquadraticcurveto{\pgfqpoint{2.000000in}{2.608331in}}{\pgfqpoint{2.000000in}{2.601389in}}%
\pgfusepath{stroke}%
\end{pgfscope}%
\begin{pgfscope}%
\pgfsetroundcap%
\pgfsetroundjoin%
\pgfsetlinewidth{0.501875pt}%
\definecolor{currentstroke}{rgb}{0.000000,0.000000,0.000000}%
\pgfsetstrokecolor{currentstroke}%
\pgfsetdash{}{0pt}%
\pgfpathmoveto{\pgfqpoint{1.972222in}{2.551954in}}%
\pgfpathlineto{\pgfqpoint{2.000000in}{2.607510in}}%
\pgfpathlineto{\pgfqpoint{2.027778in}{2.551954in}}%
\pgfusepath{stroke}%
\end{pgfscope}%
\begin{pgfscope}%
\pgftext[x=2.000000in,y=2.670833in,,bottom]{\rmfamily\fontsize{10.000000}{12.000000}\selectfont \(\displaystyle \omega\)}%
\end{pgfscope}%
\begin{pgfscope}%
\pgfsetroundcap%
\pgfsetroundjoin%
\pgfsetlinewidth{0.501875pt}%
\definecolor{currentstroke}{rgb}{0.000000,0.000000,0.000000}%
\pgfsetstrokecolor{currentstroke}%
\pgfsetdash{}{0pt}%
\pgfpathmoveto{\pgfqpoint{3.807488in}{1.400000in}}%
\pgfpathquadraticcurveto{\pgfqpoint{3.808320in}{1.400000in}}{\pgfqpoint{3.801389in}{1.400000in}}%
\pgfusepath{stroke}%
\end{pgfscope}%
\begin{pgfscope}%
\pgfsetroundcap%
\pgfsetroundjoin%
\pgfsetlinewidth{0.501875pt}%
\definecolor{currentstroke}{rgb}{0.000000,0.000000,0.000000}%
\pgfsetstrokecolor{currentstroke}%
\pgfsetdash{}{0pt}%
\pgfpathmoveto{\pgfqpoint{3.751932in}{1.427778in}}%
\pgfpathlineto{\pgfqpoint{3.807488in}{1.400000in}}%
\pgfpathlineto{\pgfqpoint{3.751932in}{1.372222in}}%
\pgfusepath{stroke}%
\end{pgfscope}%
\begin{pgfscope}%
\pgftext[x=3.870833in,y=1.400000in,left,]{\rmfamily\fontsize{10.000000}{12.000000}\selectfont \(\displaystyle k\)}%
\end{pgfscope}%
\end{pgfpicture}%
\makeatother%
\endgroup%
} %
        \caption{Der Träger von $\hat\psi$ vor und nach der Substitution aus \cref{eq:substitution1_coords}}
        \label{fig:supp_psi_substitution1}
    \end{minipage}\hfill
    \begin{minipage}{0.5\textwidth}
        \centering
        \resizebox{\textwidth}{!}{%% Creator: Matplotlib, PGF backend
%%
%% To include the figure in your LaTeX document, write
%%   \input{<filename>.pgf}
%%
%% Make sure the required packages are loaded in your preamble
%%   \usepackage{pgf}
%%
%% Figures using additional raster images can only be included by \input if
%% they are in the same directory as the main LaTeX file. For loading figures
%% from other directories you can use the `import` package
%%   \usepackage{import}
%% and then include the figures with
%%   \import{<path to file>}{<filename>.pgf}
%%
%% Matplotlib used the following preamble
%%   \usepackage[utf8x]{inputenc}
%%   \usepackage[T1]{fontenc}
%%   \usepackage{amssymb}
%%
\begingroup%
\makeatletter%
\begin{pgfpicture}%
\pgfpathrectangle{\pgfpointorigin}{\pgfqpoint{4.000000in}{2.000000in}}%
\pgfusepath{use as bounding box, clip}%
\begin{pgfscope}%
\pgfsetbuttcap%
\pgfsetmiterjoin%
\definecolor{currentfill}{rgb}{1.000000,1.000000,1.000000}%
\pgfsetfillcolor{currentfill}%
\pgfsetlinewidth{0.000000pt}%
\definecolor{currentstroke}{rgb}{1.000000,1.000000,1.000000}%
\pgfsetstrokecolor{currentstroke}%
\pgfsetdash{}{0pt}%
\pgfpathmoveto{\pgfqpoint{0.000000in}{0.000000in}}%
\pgfpathlineto{\pgfqpoint{4.000000in}{0.000000in}}%
\pgfpathlineto{\pgfqpoint{4.000000in}{2.000000in}}%
\pgfpathlineto{\pgfqpoint{0.000000in}{2.000000in}}%
\pgfpathclose%
\pgfusepath{fill}%
\end{pgfscope}%
\begin{pgfscope}%
\pgfsetbuttcap%
\pgfsetmiterjoin%
\definecolor{currentfill}{rgb}{1.000000,1.000000,1.000000}%
\pgfsetfillcolor{currentfill}%
\pgfsetlinewidth{0.000000pt}%
\definecolor{currentstroke}{rgb}{0.000000,0.000000,0.000000}%
\pgfsetstrokecolor{currentstroke}%
\pgfsetstrokeopacity{0.000000}%
\pgfsetdash{}{0pt}%
\pgfpathmoveto{\pgfqpoint{0.198611in}{0.198611in}}%
\pgfpathlineto{\pgfqpoint{3.801389in}{0.198611in}}%
\pgfpathlineto{\pgfqpoint{3.801389in}{1.801389in}}%
\pgfpathlineto{\pgfqpoint{0.198611in}{1.801389in}}%
\pgfpathclose%
\pgfusepath{fill}%
\end{pgfscope}%
\begin{pgfscope}%
\pgfpathrectangle{\pgfqpoint{0.198611in}{0.198611in}}{\pgfqpoint{3.602778in}{1.602778in}} %
\pgfusepath{clip}%
\pgfsetbuttcap%
\pgfsetmiterjoin%
\definecolor{currentfill}{rgb}{0.500000,0.500000,0.500000}%
\pgfsetfillcolor{currentfill}%
\pgfsetfillopacity{0.500000}%
\pgfsetlinewidth{0.501875pt}%
\definecolor{currentstroke}{rgb}{0.000000,0.000000,0.000000}%
\pgfsetstrokecolor{currentstroke}%
\pgfsetdash{}{0pt}%
\pgfpathmoveto{\pgfqpoint{1.909931in}{0.398958in}}%
\pgfpathlineto{\pgfqpoint{1.909931in}{0.519167in}}%
\pgfpathlineto{\pgfqpoint{2.090069in}{0.519167in}}%
\pgfpathlineto{\pgfqpoint{2.090069in}{0.398958in}}%
\pgfpathclose%
\pgfusepath{stroke,fill}%
\end{pgfscope}%
\begin{pgfscope}%
\pgfpathrectangle{\pgfqpoint{0.198611in}{0.198611in}}{\pgfqpoint{3.602778in}{1.602778in}} %
\pgfusepath{clip}%
\pgfsetbuttcap%
\pgfsetmiterjoin%
\definecolor{currentfill}{rgb}{0.500000,0.500000,0.500000}%
\pgfsetfillcolor{currentfill}%
\pgfsetfillopacity{0.500000}%
\pgfsetlinewidth{0.501875pt}%
\definecolor{currentstroke}{rgb}{0.000000,0.000000,0.000000}%
\pgfsetstrokecolor{currentstroke}%
\pgfsetdash{}{0pt}%
\pgfpathmoveto{\pgfqpoint{2.307935in}{0.759583in}}%
\pgfpathlineto{\pgfqpoint{2.592760in}{0.759583in}}%
\pgfpathlineto{\pgfqpoint{4.371038in}{1.961667in}}%
\pgfpathlineto{\pgfqpoint{3.231740in}{1.961667in}}%
\pgfpathclose%
\pgfusepath{stroke,fill}%
\end{pgfscope}%
\begin{pgfscope}%
\pgfpathrectangle{\pgfqpoint{0.198611in}{0.198611in}}{\pgfqpoint{3.602778in}{1.602778in}} %
\pgfusepath{clip}%
\pgfsetbuttcap%
\pgfsetroundjoin%
\pgfsetlinewidth{0.501875pt}%
\definecolor{currentstroke}{rgb}{0.501961,0.501961,0.501961}%
\pgfsetstrokecolor{currentstroke}%
\pgfsetdash{{1.850000pt}{0.800000pt}}{0.000000pt}%
\pgfpathmoveto{\pgfqpoint{1.804251in}{0.184722in}}%
\pgfpathlineto{\pgfqpoint{3.636860in}{1.815278in}}%
\pgfpathlineto{\pgfqpoint{3.636860in}{1.815278in}}%
\pgfusepath{stroke}%
\end{pgfscope}%
\begin{pgfscope}%
\pgfpathrectangle{\pgfqpoint{0.198611in}{0.198611in}}{\pgfqpoint{3.602778in}{1.602778in}} %
\pgfusepath{clip}%
\pgfsetbuttcap%
\pgfsetroundjoin%
\pgfsetlinewidth{0.501875pt}%
\definecolor{currentstroke}{rgb}{0.501961,0.501961,0.501961}%
\pgfsetstrokecolor{currentstroke}%
\pgfsetdash{{1.850000pt}{0.800000pt}}{0.000000pt}%
\pgfpathmoveto{\pgfqpoint{0.363140in}{1.815278in}}%
\pgfpathlineto{\pgfqpoint{2.195749in}{0.184722in}}%
\pgfpathlineto{\pgfqpoint{2.195749in}{0.184722in}}%
\pgfusepath{stroke}%
\end{pgfscope}%
\begin{pgfscope}%
\pgfsetrectcap%
\pgfsetmiterjoin%
\pgfsetlinewidth{0.501875pt}%
\definecolor{currentstroke}{rgb}{0.000000,0.000000,0.000000}%
\pgfsetstrokecolor{currentstroke}%
\pgfsetdash{}{0pt}%
\pgfpathmoveto{\pgfqpoint{2.000000in}{0.198611in}}%
\pgfpathlineto{\pgfqpoint{2.000000in}{1.801389in}}%
\pgfusepath{stroke}%
\end{pgfscope}%
\begin{pgfscope}%
\pgfsetrectcap%
\pgfsetmiterjoin%
\pgfsetlinewidth{0.501875pt}%
\definecolor{currentstroke}{rgb}{0.000000,0.000000,0.000000}%
\pgfsetstrokecolor{currentstroke}%
\pgfsetdash{}{0pt}%
\pgfpathmoveto{\pgfqpoint{0.198611in}{0.358889in}}%
\pgfpathlineto{\pgfqpoint{3.801389in}{0.358889in}}%
\pgfusepath{stroke}%
\end{pgfscope}%
\begin{pgfscope}%
\pgfsetroundcap%
\pgfsetroundjoin%
\pgfsetlinewidth{0.501875pt}%
\definecolor{currentstroke}{rgb}{0.000000,0.000000,0.000000}%
\pgfsetstrokecolor{currentstroke}%
\pgfsetdash{}{0pt}%
\pgfpathmoveto{\pgfqpoint{3.302084in}{1.537713in}}%
\pgfpathquadraticcurveto{\pgfqpoint{2.661671in}{0.997340in}}{\pgfqpoint{2.027193in}{0.461973in}}%
\pgfusepath{stroke}%
\end{pgfscope}%
\begin{pgfscope}%
\pgfsetroundcap%
\pgfsetroundjoin%
\pgfsetlinewidth{0.501875pt}%
\definecolor{currentstroke}{rgb}{0.000000,0.000000,0.000000}%
\pgfsetstrokecolor{currentstroke}%
\pgfsetdash{}{0pt}%
\pgfpathmoveto{\pgfqpoint{2.087566in}{0.476570in}}%
\pgfpathlineto{\pgfqpoint{2.027193in}{0.461973in}}%
\pgfpathlineto{\pgfqpoint{2.051739in}{0.519030in}}%
\pgfusepath{stroke}%
\end{pgfscope}%
\begin{pgfscope}%
\pgftext[x=3.351042in,y=1.560972in,left,base]{\rmfamily\fontsize{10.000000}{12.000000}\selectfont \(\displaystyle {\cdot}\)}%
\end{pgfscope}%
\begin{pgfscope}%
\pgftext[x=2.360278in,y=0.599306in,left,base]{\rmfamily\fontsize{10.000000}{12.000000}\selectfont Substitution 1}%
\end{pgfscope}%
\begin{pgfscope}%
\pgfsetroundcap%
\pgfsetroundjoin%
\pgfsetlinewidth{0.501875pt}%
\definecolor{currentstroke}{rgb}{0.000000,0.000000,0.000000}%
\pgfsetstrokecolor{currentstroke}%
\pgfsetdash{}{0pt}%
\pgfpathmoveto{\pgfqpoint{2.000000in}{1.807510in}}%
\pgfpathquadraticcurveto{\pgfqpoint{2.000000in}{1.808331in}}{\pgfqpoint{2.000000in}{1.801389in}}%
\pgfusepath{stroke}%
\end{pgfscope}%
\begin{pgfscope}%
\pgfsetroundcap%
\pgfsetroundjoin%
\pgfsetlinewidth{0.501875pt}%
\definecolor{currentstroke}{rgb}{0.000000,0.000000,0.000000}%
\pgfsetstrokecolor{currentstroke}%
\pgfsetdash{}{0pt}%
\pgfpathmoveto{\pgfqpoint{1.972222in}{1.751954in}}%
\pgfpathlineto{\pgfqpoint{2.000000in}{1.807510in}}%
\pgfpathlineto{\pgfqpoint{2.027778in}{1.751954in}}%
\pgfusepath{stroke}%
\end{pgfscope}%
\begin{pgfscope}%
\pgftext[x=2.000000in,y=1.870833in,,bottom]{\rmfamily\fontsize{10.000000}{12.000000}\selectfont \(\displaystyle \omega\)}%
\end{pgfscope}%
\begin{pgfscope}%
\pgfsetroundcap%
\pgfsetroundjoin%
\pgfsetlinewidth{0.501875pt}%
\definecolor{currentstroke}{rgb}{0.000000,0.000000,0.000000}%
\pgfsetstrokecolor{currentstroke}%
\pgfsetdash{}{0pt}%
\pgfpathmoveto{\pgfqpoint{3.807488in}{0.358889in}}%
\pgfpathquadraticcurveto{\pgfqpoint{3.808320in}{0.358889in}}{\pgfqpoint{3.801389in}{0.358889in}}%
\pgfusepath{stroke}%
\end{pgfscope}%
\begin{pgfscope}%
\pgfsetroundcap%
\pgfsetroundjoin%
\pgfsetlinewidth{0.501875pt}%
\definecolor{currentstroke}{rgb}{0.000000,0.000000,0.000000}%
\pgfsetstrokecolor{currentstroke}%
\pgfsetdash{}{0pt}%
\pgfpathmoveto{\pgfqpoint{3.751932in}{0.386667in}}%
\pgfpathlineto{\pgfqpoint{3.807488in}{0.358889in}}%
\pgfpathlineto{\pgfqpoint{3.751932in}{0.331111in}}%
\pgfusepath{stroke}%
\end{pgfscope}%
\begin{pgfscope}%
\pgftext[x=3.870833in,y=0.358889in,left,]{\rmfamily\fontsize{10.000000}{12.000000}\selectfont \(\displaystyle k\)}%
\end{pgfscope}%
\end{pgfpicture}%
\makeatother%
\endgroup%
}
        \caption{Der Träger von $\hat\psi$ vor und nach der Substitution aus \cref{eq:substitution2_coords}}
        \label{fig:supp_psi_substitution2}
    \end{minipage}
\end{figure}


%%%%%%%%%%%%%%%%%%%%%%%%%%%%%%%%%%%%%%%%%%%%%%%%%%%%%%%%%%%%%%%%%%%%%%%%%%%%%%%
% % Berechnen der Wellenfrontmenge von Delta_m
%%%%%%%%%%%%%%%%%%%%%%%%%%%%%%%%%%%%%%%%%%%%%%%%%%%%%%%%%%%%%%%%%%%%%%%%%%%%%%%%

\section{Die Wellenfrontmenge von $\Delta_m$} % (fold)
\label{sec:die_wellenfrontmenge_von_delta_m}

Die massive Zweipunktfunktion ist die Fouriertransformierte der 1$m$-Massenschale positiver Energie:
\todo{Quelle dafür zitiesfdren...}

\begin{equation}
    \Delta_m (t,x) = \int \delta (\omega^2-k^2-m^2)
                    \Theta(\omega)e^{-i\omega t + i k x} \d \omega \d k
\end{equation}

woraus sich $\hat \Delta_m$ direkt zu $\delta (\omega^2-k^2-m^2)\Theta(\omega)$
ablesen lässt.

\begin{figure}[h]
\centering
%% Creator: Matplotlib, PGF backend
%%
%% To include the figure in your LaTeX document, write
%%   \input{<filename>.pgf}
%%
%% Make sure the required packages are loaded in your preamble
%%   \usepackage{pgf}
%%
%% Figures using additional raster images can only be included by \input if
%% they are in the same directory as the main LaTeX file. For loading figures
%% from other directories you can use the `import` package
%%   \usepackage{import}
%% and then include the figures with
%%   \import{<path to file>}{<filename>.pgf}
%%
%% Matplotlib used the following preamble
%%   \usepackage[utf8x]{inputenc}
%%   \usepackage[T1]{fontenc}
%%   \usepackage{amssymb}
%%
\begingroup%
\makeatletter%
\begin{pgfpicture}%
\pgfpathrectangle{\pgfpointorigin}{\pgfqpoint{5.000000in}{2.750000in}}%
\pgfusepath{use as bounding box, clip}%
\begin{pgfscope}%
\pgfsetbuttcap%
\pgfsetmiterjoin%
\definecolor{currentfill}{rgb}{1.000000,1.000000,1.000000}%
\pgfsetfillcolor{currentfill}%
\pgfsetlinewidth{0.000000pt}%
\definecolor{currentstroke}{rgb}{1.000000,1.000000,1.000000}%
\pgfsetstrokecolor{currentstroke}%
\pgfsetdash{}{0pt}%
\pgfpathmoveto{\pgfqpoint{0.000000in}{0.000000in}}%
\pgfpathlineto{\pgfqpoint{5.000000in}{0.000000in}}%
\pgfpathlineto{\pgfqpoint{5.000000in}{2.750000in}}%
\pgfpathlineto{\pgfqpoint{0.000000in}{2.750000in}}%
\pgfpathclose%
\pgfusepath{fill}%
\end{pgfscope}%
\begin{pgfscope}%
\pgfsetbuttcap%
\pgfsetmiterjoin%
\definecolor{currentfill}{rgb}{1.000000,1.000000,1.000000}%
\pgfsetfillcolor{currentfill}%
\pgfsetlinewidth{0.000000pt}%
\definecolor{currentstroke}{rgb}{0.000000,0.000000,0.000000}%
\pgfsetstrokecolor{currentstroke}%
\pgfsetstrokeopacity{0.000000}%
\pgfsetdash{}{0pt}%
\pgfpathmoveto{\pgfqpoint{0.198611in}{0.198611in}}%
\pgfpathlineto{\pgfqpoint{4.801389in}{0.198611in}}%
\pgfpathlineto{\pgfqpoint{4.801389in}{2.551389in}}%
\pgfpathlineto{\pgfqpoint{0.198611in}{2.551389in}}%
\pgfpathclose%
\pgfusepath{fill}%
\end{pgfscope}%
\begin{pgfscope}%
\pgfpathrectangle{\pgfqpoint{0.198611in}{0.198611in}}{\pgfqpoint{4.602778in}{2.352778in}} %
\pgfusepath{clip}%
\pgfsetbuttcap%
\pgfsetmiterjoin%
\definecolor{currentfill}{rgb}{0.500000,0.500000,0.500000}%
\pgfsetfillcolor{currentfill}%
\pgfsetfillopacity{0.500000}%
\pgfsetlinewidth{0.501875pt}%
\definecolor{currentstroke}{rgb}{0.000000,0.000000,0.000000}%
\pgfsetstrokecolor{currentstroke}%
\pgfsetdash{}{0pt}%
\pgfpathmoveto{\pgfqpoint{2.291837in}{0.399458in}}%
\pgfpathlineto{\pgfqpoint{2.420489in}{0.399458in}}%
\pgfpathlineto{\pgfqpoint{2.181956in}{0.829844in}}%
\pgfpathlineto{\pgfqpoint{1.667350in}{0.829844in}}%
\pgfpathclose%
\pgfusepath{stroke,fill}%
\end{pgfscope}%
\begin{pgfscope}%
\pgfpathrectangle{\pgfqpoint{0.198611in}{0.198611in}}{\pgfqpoint{4.602778in}{2.352778in}} %
\pgfusepath{clip}%
\pgfsetbuttcap%
\pgfsetmiterjoin%
\definecolor{currentfill}{rgb}{0.500000,0.500000,0.500000}%
\pgfsetfillcolor{currentfill}%
\pgfsetfillopacity{0.500000}%
\pgfsetlinewidth{0.501875pt}%
\definecolor{currentstroke}{rgb}{0.000000,0.000000,0.000000}%
\pgfsetstrokecolor{currentstroke}%
\pgfsetdash{}{0pt}%
\pgfpathmoveto{\pgfqpoint{2.708163in}{0.112534in}}%
\pgfpathlineto{\pgfqpoint{2.579511in}{0.112534in}}%
\pgfpathlineto{\pgfqpoint{2.818044in}{-0.317852in}}%
\pgfpathlineto{\pgfqpoint{3.332650in}{-0.317852in}}%
\pgfpathclose%
\pgfusepath{stroke,fill}%
\end{pgfscope}%
\begin{pgfscope}%
\pgfpathrectangle{\pgfqpoint{0.198611in}{0.198611in}}{\pgfqpoint{4.602778in}{2.352778in}} %
\pgfusepath{clip}%
\pgfsetbuttcap%
\pgfsetmiterjoin%
\definecolor{currentfill}{rgb}{0.500000,0.500000,0.500000}%
\pgfsetfillcolor{currentfill}%
\pgfsetfillopacity{0.500000}%
\pgfsetlinewidth{0.501875pt}%
\definecolor{currentstroke}{rgb}{0.000000,0.000000,0.000000}%
\pgfsetstrokecolor{currentstroke}%
\pgfsetdash{}{0pt}%
\pgfpathmoveto{\pgfqpoint{1.636979in}{0.973306in}}%
\pgfpathlineto{\pgfqpoint{1.924653in}{0.973306in}}%
\pgfpathlineto{\pgfqpoint{0.198611in}{3.125237in}}%
\pgfpathlineto{\pgfqpoint{-0.952083in}{3.125237in}}%
\pgfpathclose%
\pgfusepath{stroke,fill}%
\end{pgfscope}%
\begin{pgfscope}%
\pgfpathrectangle{\pgfqpoint{0.198611in}{0.198611in}}{\pgfqpoint{4.602778in}{2.352778in}} %
\pgfusepath{clip}%
\pgfsetbuttcap%
\pgfsetmiterjoin%
\definecolor{currentfill}{rgb}{0.500000,0.500000,0.500000}%
\pgfsetfillcolor{currentfill}%
\pgfsetfillopacity{0.500000}%
\pgfsetlinewidth{0.501875pt}%
\definecolor{currentstroke}{rgb}{0.000000,0.000000,0.000000}%
\pgfsetstrokecolor{currentstroke}%
\pgfsetdash{}{0pt}%
\pgfpathmoveto{\pgfqpoint{3.363021in}{-0.461314in}}%
\pgfpathlineto{\pgfqpoint{3.075347in}{-0.461314in}}%
\pgfpathlineto{\pgfqpoint{4.801389in}{-2.613245in}}%
\pgfpathlineto{\pgfqpoint{5.952083in}{-2.613245in}}%
\pgfpathclose%
\pgfusepath{stroke,fill}%
\end{pgfscope}%
\begin{pgfscope}%
\pgfpathrectangle{\pgfqpoint{0.198611in}{0.198611in}}{\pgfqpoint{4.602778in}{2.352778in}} %
\pgfusepath{clip}%
\pgfsetbuttcap%
\pgfsetmiterjoin%
\definecolor{currentfill}{rgb}{0.500000,0.500000,0.500000}%
\pgfsetfillcolor{currentfill}%
\pgfsetfillopacity{0.500000}%
\pgfsetlinewidth{0.501875pt}%
\definecolor{currentstroke}{rgb}{0.000000,0.000000,0.000000}%
\pgfsetstrokecolor{currentstroke}%
\pgfsetdash{}{0pt}%
\pgfpathmoveto{\pgfqpoint{2.696967in}{0.697418in}}%
\pgfpathlineto{\pgfqpoint{2.922637in}{0.697418in}}%
\pgfpathlineto{\pgfqpoint{4.190549in}{2.021683in}}%
\pgfpathlineto{\pgfqpoint{3.287870in}{2.021683in}}%
\pgfpathclose%
\pgfusepath{stroke,fill}%
\end{pgfscope}%
\begin{pgfscope}%
\pgfpathrectangle{\pgfqpoint{0.198611in}{0.198611in}}{\pgfqpoint{4.602778in}{2.352778in}} %
\pgfusepath{clip}%
\pgfsetbuttcap%
\pgfsetmiterjoin%
\definecolor{currentfill}{rgb}{0.500000,0.500000,0.500000}%
\pgfsetfillcolor{currentfill}%
\pgfsetfillopacity{0.500000}%
\pgfsetlinewidth{0.501875pt}%
\definecolor{currentstroke}{rgb}{0.000000,0.000000,0.000000}%
\pgfsetstrokecolor{currentstroke}%
\pgfsetdash{}{0pt}%
\pgfpathmoveto{\pgfqpoint{2.303033in}{-0.185426in}}%
\pgfpathlineto{\pgfqpoint{2.077363in}{-0.185426in}}%
\pgfpathlineto{\pgfqpoint{0.809451in}{-1.509691in}}%
\pgfpathlineto{\pgfqpoint{1.712130in}{-1.509691in}}%
\pgfpathclose%
\pgfusepath{stroke,fill}%
\end{pgfscope}%
\begin{pgfscope}%
\pgfpathrectangle{\pgfqpoint{0.198611in}{0.198611in}}{\pgfqpoint{4.602778in}{2.352778in}} %
\pgfusepath{clip}%
\pgfsetrectcap%
\pgfsetroundjoin%
\pgfsetlinewidth{0.501875pt}%
\definecolor{currentstroke}{rgb}{0.894118,0.101961,0.109804}%
\pgfsetstrokecolor{currentstroke}%
\pgfsetdash{}{0pt}%
\pgfpathmoveto{\pgfqpoint{0.202627in}{2.565278in}}%
\pgfpathlineto{\pgfqpoint{0.869368in}{1.907495in}}%
\pgfpathlineto{\pgfqpoint{1.285699in}{1.500656in}}%
\pgfpathlineto{\pgfqpoint{1.563254in}{1.233366in}}%
\pgfpathlineto{\pgfqpoint{1.748290in}{1.058774in}}%
\pgfpathlineto{\pgfqpoint{1.887067in}{0.931316in}}%
\pgfpathlineto{\pgfqpoint{2.002715in}{0.828998in}}%
\pgfpathlineto{\pgfqpoint{2.095233in}{0.751283in}}%
\pgfpathlineto{\pgfqpoint{2.164622in}{0.696698in}}%
\pgfpathlineto{\pgfqpoint{2.234010in}{0.646774in}}%
\pgfpathlineto{\pgfqpoint{2.280269in}{0.617044in}}%
\pgfpathlineto{\pgfqpoint{2.326528in}{0.591050in}}%
\pgfpathlineto{\pgfqpoint{2.372788in}{0.569722in}}%
\pgfpathlineto{\pgfqpoint{2.419047in}{0.554064in}}%
\pgfpathlineto{\pgfqpoint{2.442176in}{0.548659in}}%
\pgfpathlineto{\pgfqpoint{2.465306in}{0.544999in}}%
\pgfpathlineto{\pgfqpoint{2.488435in}{0.543152in}}%
\pgfpathlineto{\pgfqpoint{2.511565in}{0.543152in}}%
\pgfpathlineto{\pgfqpoint{2.534694in}{0.544999in}}%
\pgfpathlineto{\pgfqpoint{2.557824in}{0.548659in}}%
\pgfpathlineto{\pgfqpoint{2.580953in}{0.554064in}}%
\pgfpathlineto{\pgfqpoint{2.627212in}{0.569722in}}%
\pgfpathlineto{\pgfqpoint{2.673472in}{0.591050in}}%
\pgfpathlineto{\pgfqpoint{2.719731in}{0.617044in}}%
\pgfpathlineto{\pgfqpoint{2.765990in}{0.646774in}}%
\pgfpathlineto{\pgfqpoint{2.812249in}{0.679455in}}%
\pgfpathlineto{\pgfqpoint{2.881637in}{0.732666in}}%
\pgfpathlineto{\pgfqpoint{2.951026in}{0.789561in}}%
\pgfpathlineto{\pgfqpoint{3.043544in}{0.869370in}}%
\pgfpathlineto{\pgfqpoint{3.159192in}{0.973351in}}%
\pgfpathlineto{\pgfqpoint{3.321099in}{1.123763in}}%
\pgfpathlineto{\pgfqpoint{3.529264in}{1.321922in}}%
\pgfpathlineto{\pgfqpoint{3.806819in}{1.590617in}}%
\pgfpathlineto{\pgfqpoint{4.223150in}{1.998443in}}%
\pgfpathlineto{\pgfqpoint{4.797373in}{2.565278in}}%
\pgfpathlineto{\pgfqpoint{4.797373in}{2.565278in}}%
\pgfusepath{stroke}%
\end{pgfscope}%
\begin{pgfscope}%
\pgfpathrectangle{\pgfqpoint{0.198611in}{0.198611in}}{\pgfqpoint{4.602778in}{2.352778in}} %
\pgfusepath{clip}%
\pgfsetbuttcap%
\pgfsetroundjoin%
\pgfsetlinewidth{0.501875pt}%
\definecolor{currentstroke}{rgb}{0.501961,0.501961,0.501961}%
\pgfsetstrokecolor{currentstroke}%
\pgfsetdash{{1.850000pt}{0.800000pt}}{0.000000pt}%
\pgfpathmoveto{\pgfqpoint{2.428540in}{0.184722in}}%
\pgfpathlineto{\pgfqpoint{4.801389in}{2.551389in}}%
\pgfpathlineto{\pgfqpoint{4.801389in}{2.551389in}}%
\pgfusepath{stroke}%
\end{pgfscope}%
\begin{pgfscope}%
\pgfpathrectangle{\pgfqpoint{0.198611in}{0.198611in}}{\pgfqpoint{4.602778in}{2.352778in}} %
\pgfusepath{clip}%
\pgfsetbuttcap%
\pgfsetroundjoin%
\pgfsetlinewidth{0.501875pt}%
\definecolor{currentstroke}{rgb}{0.501961,0.501961,0.501961}%
\pgfsetstrokecolor{currentstroke}%
\pgfsetdash{{1.850000pt}{0.800000pt}}{0.000000pt}%
\pgfpathmoveto{\pgfqpoint{0.198611in}{2.551389in}}%
\pgfpathlineto{\pgfqpoint{2.571460in}{0.184722in}}%
\pgfpathlineto{\pgfqpoint{2.571460in}{0.184722in}}%
\pgfusepath{stroke}%
\end{pgfscope}%
\begin{pgfscope}%
\pgfpathrectangle{\pgfqpoint{0.198611in}{0.198611in}}{\pgfqpoint{4.602778in}{2.352778in}} %
\pgfusepath{clip}%
\pgfsetbuttcap%
\pgfsetroundjoin%
\pgfsetlinewidth{0.501875pt}%
\definecolor{currentstroke}{rgb}{0.501961,0.501961,0.501961}%
\pgfsetstrokecolor{currentstroke}%
\pgfsetdash{{1.850000pt}{0.800000pt}}{0.000000pt}%
\pgfpathmoveto{\pgfqpoint{2.500000in}{0.542920in}}%
\pgfpathlineto{\pgfqpoint{3.006306in}{0.542920in}}%
\pgfusepath{stroke}%
\end{pgfscope}%
\begin{pgfscope}%
\pgfsetrectcap%
\pgfsetmiterjoin%
\pgfsetlinewidth{0.501875pt}%
\definecolor{currentstroke}{rgb}{0.000000,0.000000,0.000000}%
\pgfsetstrokecolor{currentstroke}%
\pgfsetdash{}{0pt}%
\pgfpathmoveto{\pgfqpoint{2.500000in}{0.198611in}}%
\pgfpathlineto{\pgfqpoint{2.500000in}{2.551389in}}%
\pgfusepath{stroke}%
\end{pgfscope}%
\begin{pgfscope}%
\pgfsetrectcap%
\pgfsetmiterjoin%
\pgfsetlinewidth{0.501875pt}%
\definecolor{currentstroke}{rgb}{0.000000,0.000000,0.000000}%
\pgfsetstrokecolor{currentstroke}%
\pgfsetdash{}{0pt}%
\pgfpathmoveto{\pgfqpoint{0.198611in}{0.255996in}}%
\pgfpathlineto{\pgfqpoint{4.801389in}{0.255996in}}%
\pgfusepath{stroke}%
\end{pgfscope}%
\begin{pgfscope}%
\pgfsetroundcap%
\pgfsetroundjoin%
\pgfsetlinewidth{0.501875pt}%
\definecolor{currentstroke}{rgb}{0.000000,0.000000,0.000000}%
\pgfsetstrokecolor{currentstroke}%
\pgfsetdash{}{0pt}%
\pgfpathmoveto{\pgfqpoint{1.323403in}{0.614481in}}%
\pgfpathquadraticcurveto{\pgfqpoint{1.610175in}{0.635090in}}{\pgfqpoint{1.889202in}{0.655142in}}%
\pgfusepath{stroke}%
\end{pgfscope}%
\begin{pgfscope}%
\pgfsetroundcap%
\pgfsetroundjoin%
\pgfsetlinewidth{0.501875pt}%
\definecolor{currentstroke}{rgb}{0.000000,0.000000,0.000000}%
\pgfsetstrokecolor{currentstroke}%
\pgfsetdash{}{0pt}%
\pgfpathmoveto{\pgfqpoint{1.831798in}{0.678866in}}%
\pgfpathlineto{\pgfqpoint{1.889202in}{0.655142in}}%
\pgfpathlineto{\pgfqpoint{1.835781in}{0.623453in}}%
\pgfusepath{stroke}%
\end{pgfscope}%
\begin{pgfscope}%
\pgftext[x=0.342448in,y=0.542920in,left,base]{\rmfamily\fontsize{10.000000}{12.000000}\selectfont \(\displaystyle a = 0.2, s = -1\)}%
\end{pgfscope}%
\begin{pgfscope}%
\pgfsetroundcap%
\pgfsetroundjoin%
\pgfsetlinewidth{0.501875pt}%
\definecolor{currentstroke}{rgb}{0.000000,0.000000,0.000000}%
\pgfsetstrokecolor{currentstroke}%
\pgfsetdash{}{0pt}%
\pgfpathmoveto{\pgfqpoint{0.850569in}{1.153085in}}%
\pgfpathquadraticcurveto{\pgfqpoint{0.868538in}{1.350610in}}{\pgfqpoint{0.885803in}{1.540402in}}%
\pgfusepath{stroke}%
\end{pgfscope}%
\begin{pgfscope}%
\pgfsetroundcap%
\pgfsetroundjoin%
\pgfsetlinewidth{0.501875pt}%
\definecolor{currentstroke}{rgb}{0.000000,0.000000,0.000000}%
\pgfsetstrokecolor{currentstroke}%
\pgfsetdash{}{0pt}%
\pgfpathmoveto{\pgfqpoint{0.853107in}{1.487591in}}%
\pgfpathlineto{\pgfqpoint{0.885803in}{1.540402in}}%
\pgfpathlineto{\pgfqpoint{0.908434in}{1.482558in}}%
\pgfusepath{stroke}%
\end{pgfscope}%
\begin{pgfscope}%
\pgftext[x=0.342448in,y=1.001999in,left,base]{\rmfamily\fontsize{10.000000}{12.000000}\selectfont \(\displaystyle a = 0.04, s = -1\)}%
\end{pgfscope}%
\begin{pgfscope}%
\pgfsetroundcap%
\pgfsetroundjoin%
\pgfsetlinewidth{0.501875pt}%
\definecolor{currentstroke}{rgb}{0.000000,0.000000,0.000000}%
\pgfsetstrokecolor{currentstroke}%
\pgfsetdash{}{0pt}%
\pgfpathmoveto{\pgfqpoint{3.623224in}{2.239553in}}%
\pgfpathquadraticcurveto{\pgfqpoint{3.635112in}{2.151002in}}{\pgfqpoint{3.645966in}{2.070147in}}%
\pgfusepath{stroke}%
\end{pgfscope}%
\begin{pgfscope}%
\pgfsetroundcap%
\pgfsetroundjoin%
\pgfsetlinewidth{0.501875pt}%
\definecolor{currentstroke}{rgb}{0.000000,0.000000,0.000000}%
\pgfsetstrokecolor{currentstroke}%
\pgfsetdash{}{0pt}%
\pgfpathmoveto{\pgfqpoint{3.666105in}{2.128905in}}%
\pgfpathlineto{\pgfqpoint{3.645966in}{2.070147in}}%
\pgfpathlineto{\pgfqpoint{3.611044in}{2.121513in}}%
\pgfusepath{stroke}%
\end{pgfscope}%
\begin{pgfscope}%
\pgftext[x=3.075347in,y=2.321850in,left,base]{\rmfamily\fontsize{10.000000}{12.000000}\selectfont \(\displaystyle a = 0.065, s = 0.7\)}%
\end{pgfscope}%
\begin{pgfscope}%
\pgftext[x=3.075347in,y=0.514228in,left,base]{\rmfamily\fontsize{10.000000}{12.000000}\selectfont \(\displaystyle \omega = m\)}%
\end{pgfscope}%
\begin{pgfscope}%
\pgftext[x=4.053437in,y=1.690617in,left,base]{\rmfamily\fontsize{10.000000}{12.000000}\selectfont \(\displaystyle supp~(\hat\Delta_m)\)}%
\end{pgfscope}%
\begin{pgfscope}%
\pgfsetroundcap%
\pgfsetroundjoin%
\pgfsetlinewidth{0.501875pt}%
\definecolor{currentstroke}{rgb}{0.000000,0.000000,0.000000}%
\pgfsetstrokecolor{currentstroke}%
\pgfsetdash{}{0pt}%
\pgfpathmoveto{\pgfqpoint{2.500000in}{2.557510in}}%
\pgfpathquadraticcurveto{\pgfqpoint{2.500000in}{2.558331in}}{\pgfqpoint{2.500000in}{2.551389in}}%
\pgfusepath{stroke}%
\end{pgfscope}%
\begin{pgfscope}%
\pgfsetroundcap%
\pgfsetroundjoin%
\pgfsetlinewidth{0.501875pt}%
\definecolor{currentstroke}{rgb}{0.000000,0.000000,0.000000}%
\pgfsetstrokecolor{currentstroke}%
\pgfsetdash{}{0pt}%
\pgfpathmoveto{\pgfqpoint{2.472222in}{2.501954in}}%
\pgfpathlineto{\pgfqpoint{2.500000in}{2.557510in}}%
\pgfpathlineto{\pgfqpoint{2.527778in}{2.501954in}}%
\pgfusepath{stroke}%
\end{pgfscope}%
\begin{pgfscope}%
\pgftext[x=2.500000in,y=2.620833in,,bottom]{\rmfamily\fontsize{10.000000}{12.000000}\selectfont \(\displaystyle \omega\)}%
\end{pgfscope}%
\begin{pgfscope}%
\pgfsetroundcap%
\pgfsetroundjoin%
\pgfsetlinewidth{0.501875pt}%
\definecolor{currentstroke}{rgb}{0.000000,0.000000,0.000000}%
\pgfsetstrokecolor{currentstroke}%
\pgfsetdash{}{0pt}%
\pgfpathmoveto{\pgfqpoint{4.807488in}{0.255996in}}%
\pgfpathquadraticcurveto{\pgfqpoint{4.808320in}{0.255996in}}{\pgfqpoint{4.801389in}{0.255996in}}%
\pgfusepath{stroke}%
\end{pgfscope}%
\begin{pgfscope}%
\pgfsetroundcap%
\pgfsetroundjoin%
\pgfsetlinewidth{0.501875pt}%
\definecolor{currentstroke}{rgb}{0.000000,0.000000,0.000000}%
\pgfsetstrokecolor{currentstroke}%
\pgfsetdash{}{0pt}%
\pgfpathmoveto{\pgfqpoint{4.751932in}{0.283774in}}%
\pgfpathlineto{\pgfqpoint{4.807488in}{0.255996in}}%
\pgfpathlineto{\pgfqpoint{4.751932in}{0.228218in}}%
\pgfusepath{stroke}%
\end{pgfscope}%
\begin{pgfscope}%
\pgftext[x=4.870833in,y=0.255996in,left,]{\rmfamily\fontsize{10.000000}{12.000000}\selectfont \(\displaystyle k\)}%
\end{pgfscope}%
\end{pgfpicture}%
\makeatother%
\endgroup%

\caption{Die Träger von $\hat\Delta_m$ und $\hat\psi_{ast}$. Es ist zu sehen, dass für $a \rightarrow 0$ und $s \neq \pm 1$ die Träger schließlich disjunkt sind}
\label{fig:delta_m}
\end{figure}

%%%%%%%%%%%%%%%%%%%%%%%%%%%%%%%%%%%%%%%%%%%%%%%%%%%%%%%%%%%%%%%%%%%%%%%%%%%%%%%
% % Teil 2
%%%%%%%%%%%%%%%%%%%%%%%%%%%%%%%%%%%%%%%%%%%%%%%%%%%%%%%%%%%%%%%%%%%%%%%%%%%%%%%
\subsubsection*{Fall $s \neq \pm 1$}
Hier gibt es nicht viel zu tun, denn für a klein genug gilt
$supp (\hat \Delta_m) \cap supp (\hat \psi_{ast}) = \varnothing$ wie man Abb. \ref{fig:delta_m} entnehmen kann.
Also gilt $\left< \psi_{ast}, \Delta_m\right> = 0 = O(a^k)~ \forall k$ für $a$ klein genug. Dies gilt für alle $(t', x') \in \mathbb{R}^2$

\todo{hier noch blöde Abschätzerei machen, warum das tatsächlich gilt, oder stehen lassen. Oder im Kapitel Shearlets ne Bemerkung machen, warum wir in immer engeren Kegeln landen?}

%%%%%%%%%%%%%%%%%%%%%%%%%%%%%%%%%%%%%%%%%%%%%%%%%%%%%%%%%%%%%%%%%%%%%%%%%%%%%%%
% % Teil3
%%%%%%%%%%%%%%%%%%%%%%%%%%%%%%%%%%%%%%%%%%%%%%%%%%%%%%%%%%%%%%%%%%%%%%%%%%%%%%%
\subsubsection*{Fall $s=1$}
\paragraph*{Intuition}
Für $s=-1$ schneidet die Diagonale  $supp (\hat \psi_{ast})$ auf der ganzen Länge. Der Betrag von $\hat\psi_{ast}$ skaliert mit $a^{\frac{3}{4}}$ und die Länge von $supp (\hat \psi_{ast})$ mit $a^{-1}$. Also erwarten wir schlimmstenfalls $\left< \hat \psi_{a1t}, \hat\Delta_m\right> = O(a^-{\frac{1}{4}})$. Aber nur wenn die Wellenfronten von $e^{-i\omega t'+i k x'}$ parallel zu der Singularität und damit der Diagonalen liegen. Andernfalls erwarten wir, dass die immer schneller werdenden Oszillationen der Phase sich gegenseitig auslöschen/wegheben.

\paragraph*{Fleißige Analysis}
\begin{align}
    \left<\hat \psi_{a1t} ,\hat \Delta_m\right> &=
        a^{\frac{3}{4}} \int \hat \psi_1(a\omega)
        \hat\psi_2\left(a^{-\frac{1}{2}} \left(\tfrac{k}{\omega}-1\right)\right)
        \delta (\omega^2 - k^2 - m^2) \theta(\omega)
        e^{-i\omega t' + ikx'} d \omega \d k \nonumber \\[2ex]
        & \kern 2em \underline{\textrm{Nullstellen von $\delta$}:}
        \nonumber \\
        & \kern 2em \omega^2 - k^2 - m^2 = 0 \Leftrightarrow k = \pm \sqrt{\omega^2-m^2}
        \nonumber \\
        & \kern 2em \Rightarrow \frac{\d k}{\d \omega} = \frac{\omega}{\sqrt{\omega^2-m^2}}; \textrm{   wobei nur ,,+'' in $supp (\hat \psi_2)$ liegt}
        \nonumber \\[2ex]
        &= a^{\frac{3}{4}} \int \hat\psi_1(a \omega)
        \hat\psi_2\left(a^{-\frac{1}{2}} \left(\tfrac{\sqrt{\omega^2-m^2}}{\omega}-1\right)\right)
        e^{-i\omega t' + i \sqrt{\omega^2-m^2}x'}
        \d \omega \nonumber \\
        &= a^{\frac{3}{4}} a^{-1} \int \hat\psi_1(\omega)
        \hat\psi_2
        \underbrace{
        \left(
            a^{-\frac{1}{2}} \left(\tfrac{\sqrt{a\omega^2-m^2}}{\omega}-1\right)
        \right)}_{= \frac{a^{\frac{3}{2}}m^2}{2 \omega^2}
                  + O\left(a^{\frac{7}{2}}\right)}
        e^{-i\frac{\omega}{a} t' + i \sqrt{\frac{\omega^2}{a^2}-m^2}x'}
        \d \omega \nonumber
\end{align}
Der Integrand lässt sich nun durch $\hat \psi_1(\omega) \left\lVert \hat\psi_2\right\lVert_\infty$ majorisieren und wir dürfen Lebesgue verwenden um Integral und Grenzwert $a \rightarrow 0$ zu vertauschen

\begin{dmath}
 = a^{-\frac{1}{4}} \int \hat \psi_1(\omega) \hat \psi_2 (0)
    e^{-i\omega \left(\frac{t'-x'}{a}\right)}
= a^{-\frac{1}{4}} \hat \psi_2 (0) \psi_1\left(\frac{t'-x'}{a}\right)
\\
 \sim O\left(a^{-\frac{1}{4}}\right) \condition{falls $x'=t'$}
\\
 \sim O\left(a^k\right) ~ \forall k  \condition{sonst}
\end{dmath}

Das analoge Ergebnis erhält man mit gleicher Rechnung auch für $s=-1$ und $t' = -x'$
Dies bestätigt das intuitiv erwartete Ergebnis. Insgesamt erhalten wir also für die Wellenfrontmenge:

\begin{table}[h]
\centering
\label{my-label}
\begin{tabular}{l|cccc}
        & \multicolumn{1}{l}{$(t', x') = (0, 0)$} & \multicolumn{1}{l}{$t' = x'$} & \multicolumn{1}{l}{$t' = -x'$} & \multicolumn{1}{l}{$t' \neq \pm x'$} \\ \hline
s = 1   & $a^{-\frac{1}{4}}$    & $a^{-\frac{1}{4}}$    & $a^k$  & $a^k$    \\
s = -1  & $a^{-\frac{1}{4}}$    & $a^k$    & $a^{-\frac{1}{4}}$  & $a^k$    \\
$s \neq \pm 1$  & $a^k$         & $a^k$    & $a^k$               & $a^k$    \\
\end{tabular}
\caption{Konvergenzordnung von $\mathcal{S}_{\Delta_m} (a,s,(t',x'))$ im Limit $a \rightarrow 0$ für alle interessanten Kombinationen von $s$ und $(t',x')$}
\end{table}

% section die_wellenfrontmenge_von_ (end)


%%%%%%%%%%%%%%%%%%%%%%%%%%%%%%%%%%%%%%%%%%%%%%%%%%%%%%%%%%%%%%%%%%%%%%%%%%%%%%%
% % Berechnen der Wellenfrontmenge von Delta_m
%%%%%%%%%%%%%%%%%%%%%%%%%%%%%%%%%%%%%%%%%%%%%%%%%%%%%%%%%%%%%%%%%%%%%%%%%%%%%%%
%
\section{\texorpdfstring{Die Wellenfrontmenge von $\Delta_m^2$}
    {Wellenfrontmenge von delta m}} % (fold)
\label{sec:die_wellenfrontmenge_von_delta_m_2_}

\subsection{\texorpdfstring{$\hat\Delta^{\ast 2}$ berechnen}
    {delta-ast-hat berechnen}}
Bevor wir die Wellenfrontmenge von $\Delta_m^2$ berechnen können, benötigen wir einen Ausdruck dafür, oder besser noch einen für die Fouriertransformierte davon. Gemäß dem Faltungssatz gilt $\rwhat{\Delta_m^2} = \rwhat \Delta_m * \rwhat \Delta_m = \rwhat\Delta_m^{*2}$. Wir müssen also die Faltung von $\rwhat \Delta_m$ mit sich selber ausrechnen.

\begin{figure}
    \centering
    \begin{minipage}{0.5\textwidth}
        \centering
        \resizebox{\textwidth}{!}{%% Creator: Matplotlib, PGF backend
%%
%% To include the figure in your LaTeX document, write
%%   \input{<filename>.pgf}
%%
%% Make sure the required packages are loaded in your preamble
%%   \usepackage{pgf}
%%
%% Figures using additional raster images can only be included by \input if
%% they are in the same directory as the main LaTeX file. For loading figures
%% from other directories you can use the `import` package
%%   \usepackage{import}
%% and then include the figures with
%%   \import{<path to file>}{<filename>.pgf}
%%
%% Matplotlib used the following preamble
%%   \usepackage[utf8x]{inputenc}
%%   \usepackage[T1]{fontenc}
%%   \usepackage{amssymb}
%%
\begingroup%
\makeatletter%
\begin{pgfpicture}%
\pgfpathrectangle{\pgfpointorigin}{\pgfqpoint{4.000000in}{3.000000in}}%
\pgfusepath{use as bounding box, clip}%
\begin{pgfscope}%
\pgfsetbuttcap%
\pgfsetmiterjoin%
\definecolor{currentfill}{rgb}{1.000000,1.000000,1.000000}%
\pgfsetfillcolor{currentfill}%
\pgfsetlinewidth{0.000000pt}%
\definecolor{currentstroke}{rgb}{1.000000,1.000000,1.000000}%
\pgfsetstrokecolor{currentstroke}%
\pgfsetdash{}{0pt}%
\pgfpathmoveto{\pgfqpoint{0.000000in}{0.000000in}}%
\pgfpathlineto{\pgfqpoint{4.000000in}{0.000000in}}%
\pgfpathlineto{\pgfqpoint{4.000000in}{3.000000in}}%
\pgfpathlineto{\pgfqpoint{0.000000in}{3.000000in}}%
\pgfpathclose%
\pgfusepath{fill}%
\end{pgfscope}%
\begin{pgfscope}%
\pgfsetbuttcap%
\pgfsetmiterjoin%
\definecolor{currentfill}{rgb}{1.000000,1.000000,1.000000}%
\pgfsetfillcolor{currentfill}%
\pgfsetlinewidth{0.000000pt}%
\definecolor{currentstroke}{rgb}{0.000000,0.000000,0.000000}%
\pgfsetstrokecolor{currentstroke}%
\pgfsetstrokeopacity{0.000000}%
\pgfsetdash{}{0pt}%
\pgfpathmoveto{\pgfqpoint{0.198611in}{0.198611in}}%
\pgfpathlineto{\pgfqpoint{3.801389in}{0.198611in}}%
\pgfpathlineto{\pgfqpoint{3.801389in}{2.801389in}}%
\pgfpathlineto{\pgfqpoint{0.198611in}{2.801389in}}%
\pgfpathclose%
\pgfusepath{fill}%
\end{pgfscope}%
\begin{pgfscope}%
\pgfsetbuttcap%
\pgfsetroundjoin%
\definecolor{currentfill}{rgb}{0.000000,0.000000,0.000000}%
\pgfsetfillcolor{currentfill}%
\pgfsetlinewidth{0.803000pt}%
\definecolor{currentstroke}{rgb}{0.000000,0.000000,0.000000}%
\pgfsetstrokecolor{currentstroke}%
\pgfsetdash{}{0pt}%
\pgfsys@defobject{currentmarker}{\pgfqpoint{0.000000in}{-0.048611in}}{\pgfqpoint{0.000000in}{0.000000in}}{%
\pgfpathmoveto{\pgfqpoint{0.000000in}{0.000000in}}%
\pgfpathlineto{\pgfqpoint{0.000000in}{-0.048611in}}%
\pgfusepath{stroke,fill}%
}%
\begin{pgfscope}%
\pgfsys@transformshift{1.219976in}{0.632407in}%
\pgfsys@useobject{currentmarker}{}%
\end{pgfscope}%
\end{pgfscope}%
\begin{pgfscope}%
\pgftext[x=1.219976in,y=0.535185in,,top]{\rmfamily\fontsize{10.000000}{12.000000}\selectfont \(\displaystyle k^\prime_{0-} = -\sqrt{(\frac{\omega}{2})^2-m^2}\)}%
\end{pgfscope}%
\begin{pgfscope}%
\pgfsetbuttcap%
\pgfsetroundjoin%
\definecolor{currentfill}{rgb}{0.000000,0.000000,0.000000}%
\pgfsetfillcolor{currentfill}%
\pgfsetlinewidth{0.803000pt}%
\definecolor{currentstroke}{rgb}{0.000000,0.000000,0.000000}%
\pgfsetstrokecolor{currentstroke}%
\pgfsetdash{}{0pt}%
\pgfsys@defobject{currentmarker}{\pgfqpoint{0.000000in}{-0.048611in}}{\pgfqpoint{0.000000in}{0.000000in}}{%
\pgfpathmoveto{\pgfqpoint{0.000000in}{0.000000in}}%
\pgfpathlineto{\pgfqpoint{0.000000in}{-0.048611in}}%
\pgfusepath{stroke,fill}%
}%
\begin{pgfscope}%
\pgfsys@transformshift{2.780024in}{0.632407in}%
\pgfsys@useobject{currentmarker}{}%
\end{pgfscope}%
\end{pgfscope}%
\begin{pgfscope}%
\pgftext[x=2.780024in,y=0.535185in,,top]{\rmfamily\fontsize{10.000000}{12.000000}\selectfont \(\displaystyle k^\prime_{0+} = \sqrt{(\frac{\omega}{2})^2-m^2}\)}%
\end{pgfscope}%
\begin{pgfscope}%
\pgfsetbuttcap%
\pgfsetroundjoin%
\definecolor{currentfill}{rgb}{0.000000,0.000000,0.000000}%
\pgfsetfillcolor{currentfill}%
\pgfsetlinewidth{0.803000pt}%
\definecolor{currentstroke}{rgb}{0.000000,0.000000,0.000000}%
\pgfsetstrokecolor{currentstroke}%
\pgfsetdash{}{0pt}%
\pgfsys@defobject{currentmarker}{\pgfqpoint{-0.048611in}{0.000000in}}{\pgfqpoint{0.000000in}{0.000000in}}{%
\pgfpathmoveto{\pgfqpoint{0.000000in}{0.000000in}}%
\pgfpathlineto{\pgfqpoint{-0.048611in}{0.000000in}}%
\pgfusepath{stroke,fill}%
}%
\begin{pgfscope}%
\pgfsys@transformshift{2.000000in}{1.066204in}%
\pgfsys@useobject{currentmarker}{}%
\end{pgfscope}%
\end{pgfscope}%
\begin{pgfscope}%
\pgftext[x=1.780831in,y=1.018376in,left,base]{\rmfamily\fontsize{10.000000}{12.000000}\selectfont \(\displaystyle m\)}%
\end{pgfscope}%
\begin{pgfscope}%
\pgfsetbuttcap%
\pgfsetroundjoin%
\definecolor{currentfill}{rgb}{0.000000,0.000000,0.000000}%
\pgfsetfillcolor{currentfill}%
\pgfsetlinewidth{0.803000pt}%
\definecolor{currentstroke}{rgb}{0.000000,0.000000,0.000000}%
\pgfsetstrokecolor{currentstroke}%
\pgfsetdash{}{0pt}%
\pgfsys@defobject{currentmarker}{\pgfqpoint{-0.048611in}{0.000000in}}{\pgfqpoint{0.000000in}{0.000000in}}{%
\pgfpathmoveto{\pgfqpoint{0.000000in}{0.000000in}}%
\pgfpathlineto{\pgfqpoint{-0.048611in}{0.000000in}}%
\pgfusepath{stroke,fill}%
}%
\begin{pgfscope}%
\pgfsys@transformshift{2.000000in}{1.500000in}%
\pgfsys@useobject{currentmarker}{}%
\end{pgfscope}%
\end{pgfscope}%
\begin{pgfscope}%
\pgftext[x=1.338511in,y=1.444321in,left,base]{\rmfamily\fontsize{10.000000}{12.000000}\selectfont \(\displaystyle \omega/2 = \omega^\prime_0\)}%
\end{pgfscope}%
\begin{pgfscope}%
\pgfsetbuttcap%
\pgfsetroundjoin%
\definecolor{currentfill}{rgb}{0.000000,0.000000,0.000000}%
\pgfsetfillcolor{currentfill}%
\pgfsetlinewidth{0.803000pt}%
\definecolor{currentstroke}{rgb}{0.000000,0.000000,0.000000}%
\pgfsetstrokecolor{currentstroke}%
\pgfsetdash{}{0pt}%
\pgfsys@defobject{currentmarker}{\pgfqpoint{-0.048611in}{0.000000in}}{\pgfqpoint{0.000000in}{0.000000in}}{%
\pgfpathmoveto{\pgfqpoint{0.000000in}{0.000000in}}%
\pgfpathlineto{\pgfqpoint{-0.048611in}{0.000000in}}%
\pgfusepath{stroke,fill}%
}%
\begin{pgfscope}%
\pgfsys@transformshift{2.000000in}{1.933796in}%
\pgfsys@useobject{currentmarker}{}%
\end{pgfscope}%
\end{pgfscope}%
\begin{pgfscope}%
\pgftext[x=1.519645in,y=1.885969in,left,base]{\rmfamily\fontsize{10.000000}{12.000000}\selectfont \(\displaystyle \omega - m\)}%
\end{pgfscope}%
\begin{pgfscope}%
\pgfsetbuttcap%
\pgfsetroundjoin%
\definecolor{currentfill}{rgb}{0.000000,0.000000,0.000000}%
\pgfsetfillcolor{currentfill}%
\pgfsetlinewidth{0.803000pt}%
\definecolor{currentstroke}{rgb}{0.000000,0.000000,0.000000}%
\pgfsetstrokecolor{currentstroke}%
\pgfsetdash{}{0pt}%
\pgfsys@defobject{currentmarker}{\pgfqpoint{-0.048611in}{0.000000in}}{\pgfqpoint{0.000000in}{0.000000in}}{%
\pgfpathmoveto{\pgfqpoint{0.000000in}{0.000000in}}%
\pgfpathlineto{\pgfqpoint{-0.048611in}{0.000000in}}%
\pgfusepath{stroke,fill}%
}%
\begin{pgfscope}%
\pgfsys@transformshift{2.000000in}{2.367593in}%
\pgfsys@useobject{currentmarker}{}%
\end{pgfscope}%
\end{pgfscope}%
\begin{pgfscope}%
\pgftext[x=1.811343in,y=2.319765in,left,base]{\rmfamily\fontsize{10.000000}{12.000000}\selectfont \(\displaystyle \omega\)}%
\end{pgfscope}%
\begin{pgfscope}%
\pgfpathrectangle{\pgfqpoint{0.198611in}{0.198611in}}{\pgfqpoint{3.602778in}{2.602778in}} %
\pgfusepath{clip}%
\pgfsetbuttcap%
\pgfsetroundjoin%
\pgfsetlinewidth{0.501875pt}%
\definecolor{currentstroke}{rgb}{0.501961,0.501961,0.501961}%
\pgfsetstrokecolor{currentstroke}%
\pgfsetdash{{1.850000pt}{0.800000pt}}{0.000000pt}%
\pgfpathmoveto{\pgfqpoint{1.219976in}{0.632407in}}%
\pgfpathlineto{\pgfqpoint{1.219976in}{1.500000in}}%
\pgfusepath{stroke}%
\end{pgfscope}%
\begin{pgfscope}%
\pgfpathrectangle{\pgfqpoint{0.198611in}{0.198611in}}{\pgfqpoint{3.602778in}{2.602778in}} %
\pgfusepath{clip}%
\pgfsetbuttcap%
\pgfsetroundjoin%
\pgfsetlinewidth{0.501875pt}%
\definecolor{currentstroke}{rgb}{0.501961,0.501961,0.501961}%
\pgfsetstrokecolor{currentstroke}%
\pgfsetdash{{1.850000pt}{0.800000pt}}{0.000000pt}%
\pgfpathmoveto{\pgfqpoint{2.780024in}{0.632407in}}%
\pgfpathlineto{\pgfqpoint{2.780024in}{1.500000in}}%
\pgfusepath{stroke}%
\end{pgfscope}%
\begin{pgfscope}%
\pgfpathrectangle{\pgfqpoint{0.198611in}{0.198611in}}{\pgfqpoint{3.602778in}{2.602778in}} %
\pgfusepath{clip}%
\pgfsetrectcap%
\pgfsetroundjoin%
\pgfsetlinewidth{0.501875pt}%
\definecolor{currentstroke}{rgb}{0.894118,0.101961,0.109804}%
\pgfsetstrokecolor{currentstroke}%
\pgfsetdash{}{0pt}%
\pgfpathmoveto{\pgfqpoint{0.184722in}{0.566020in}}%
\pgfpathlineto{\pgfqpoint{0.524413in}{0.881513in}}%
\pgfpathlineto{\pgfqpoint{0.768088in}{1.104150in}}%
\pgfpathlineto{\pgfqpoint{0.957613in}{1.273814in}}%
\pgfpathlineto{\pgfqpoint{1.120063in}{1.415436in}}%
\pgfpathlineto{\pgfqpoint{1.255438in}{1.529408in}}%
\pgfpathlineto{\pgfqpoint{1.363738in}{1.616727in}}%
\pgfpathlineto{\pgfqpoint{1.444963in}{1.679103in}}%
\pgfpathlineto{\pgfqpoint{1.526188in}{1.737927in}}%
\pgfpathlineto{\pgfqpoint{1.607413in}{1.792107in}}%
\pgfpathlineto{\pgfqpoint{1.661563in}{1.824956in}}%
\pgfpathlineto{\pgfqpoint{1.715713in}{1.854594in}}%
\pgfpathlineto{\pgfqpoint{1.769863in}{1.880437in}}%
\pgfpathlineto{\pgfqpoint{1.824013in}{1.901850in}}%
\pgfpathlineto{\pgfqpoint{1.878163in}{1.918201in}}%
\pgfpathlineto{\pgfqpoint{1.932313in}{1.928924in}}%
\pgfpathlineto{\pgfqpoint{1.986463in}{1.933600in}}%
\pgfpathlineto{\pgfqpoint{2.013537in}{1.933600in}}%
\pgfpathlineto{\pgfqpoint{2.040612in}{1.932036in}}%
\pgfpathlineto{\pgfqpoint{2.094762in}{1.924297in}}%
\pgfpathlineto{\pgfqpoint{2.148912in}{1.910696in}}%
\pgfpathlineto{\pgfqpoint{2.203062in}{1.891737in}}%
\pgfpathlineto{\pgfqpoint{2.257212in}{1.868029in}}%
\pgfpathlineto{\pgfqpoint{2.311362in}{1.840212in}}%
\pgfpathlineto{\pgfqpoint{2.365512in}{1.808899in}}%
\pgfpathlineto{\pgfqpoint{2.419662in}{1.774644in}}%
\pgfpathlineto{\pgfqpoint{2.500887in}{1.718774in}}%
\pgfpathlineto{\pgfqpoint{2.582112in}{1.658661in}}%
\pgfpathlineto{\pgfqpoint{2.690412in}{1.573580in}}%
\pgfpathlineto{\pgfqpoint{2.798712in}{1.484365in}}%
\pgfpathlineto{\pgfqpoint{2.934087in}{1.368722in}}%
\pgfpathlineto{\pgfqpoint{3.096537in}{1.225745in}}%
\pgfpathlineto{\pgfqpoint{3.313137in}{1.030397in}}%
\pgfpathlineto{\pgfqpoint{3.583887in}{0.781444in}}%
\pgfpathlineto{\pgfqpoint{3.815278in}{0.566020in}}%
\pgfpathlineto{\pgfqpoint{3.815278in}{0.566020in}}%
\pgfusepath{stroke}%
\end{pgfscope}%
\begin{pgfscope}%
\pgfpathrectangle{\pgfqpoint{0.198611in}{0.198611in}}{\pgfqpoint{3.602778in}{2.602778in}} %
\pgfusepath{clip}%
\pgfsetrectcap%
\pgfsetroundjoin%
\pgfsetlinewidth{0.501875pt}%
\definecolor{currentstroke}{rgb}{0.215686,0.494118,0.721569}%
\pgfsetstrokecolor{currentstroke}%
\pgfsetdash{}{0pt}%
\pgfpathmoveto{\pgfqpoint{0.184722in}{2.433980in}}%
\pgfpathlineto{\pgfqpoint{0.524413in}{2.118487in}}%
\pgfpathlineto{\pgfqpoint{0.768088in}{1.895850in}}%
\pgfpathlineto{\pgfqpoint{0.957613in}{1.726186in}}%
\pgfpathlineto{\pgfqpoint{1.120063in}{1.584564in}}%
\pgfpathlineto{\pgfqpoint{1.255438in}{1.470592in}}%
\pgfpathlineto{\pgfqpoint{1.363738in}{1.383273in}}%
\pgfpathlineto{\pgfqpoint{1.444963in}{1.320897in}}%
\pgfpathlineto{\pgfqpoint{1.526188in}{1.262073in}}%
\pgfpathlineto{\pgfqpoint{1.607413in}{1.207893in}}%
\pgfpathlineto{\pgfqpoint{1.661563in}{1.175044in}}%
\pgfpathlineto{\pgfqpoint{1.715713in}{1.145406in}}%
\pgfpathlineto{\pgfqpoint{1.769863in}{1.119563in}}%
\pgfpathlineto{\pgfqpoint{1.824013in}{1.098150in}}%
\pgfpathlineto{\pgfqpoint{1.878163in}{1.081799in}}%
\pgfpathlineto{\pgfqpoint{1.932313in}{1.071076in}}%
\pgfpathlineto{\pgfqpoint{1.986463in}{1.066400in}}%
\pgfpathlineto{\pgfqpoint{2.013537in}{1.066400in}}%
\pgfpathlineto{\pgfqpoint{2.040612in}{1.067964in}}%
\pgfpathlineto{\pgfqpoint{2.094762in}{1.075703in}}%
\pgfpathlineto{\pgfqpoint{2.148912in}{1.089304in}}%
\pgfpathlineto{\pgfqpoint{2.203062in}{1.108263in}}%
\pgfpathlineto{\pgfqpoint{2.257212in}{1.131971in}}%
\pgfpathlineto{\pgfqpoint{2.311362in}{1.159788in}}%
\pgfpathlineto{\pgfqpoint{2.365512in}{1.191101in}}%
\pgfpathlineto{\pgfqpoint{2.419662in}{1.225356in}}%
\pgfpathlineto{\pgfqpoint{2.500887in}{1.281226in}}%
\pgfpathlineto{\pgfqpoint{2.582112in}{1.341339in}}%
\pgfpathlineto{\pgfqpoint{2.690412in}{1.426420in}}%
\pgfpathlineto{\pgfqpoint{2.798712in}{1.515635in}}%
\pgfpathlineto{\pgfqpoint{2.934087in}{1.631278in}}%
\pgfpathlineto{\pgfqpoint{3.096537in}{1.774255in}}%
\pgfpathlineto{\pgfqpoint{3.313137in}{1.969603in}}%
\pgfpathlineto{\pgfqpoint{3.583887in}{2.218556in}}%
\pgfpathlineto{\pgfqpoint{3.815278in}{2.433980in}}%
\pgfpathlineto{\pgfqpoint{3.815278in}{2.433980in}}%
\pgfusepath{stroke}%
\end{pgfscope}%
\begin{pgfscope}%
\pgfpathrectangle{\pgfqpoint{0.198611in}{0.198611in}}{\pgfqpoint{3.602778in}{2.602778in}} %
\pgfusepath{clip}%
\pgfsetbuttcap%
\pgfsetroundjoin%
\pgfsetlinewidth{0.501875pt}%
\definecolor{currentstroke}{rgb}{0.501961,0.501961,0.501961}%
\pgfsetstrokecolor{currentstroke}%
\pgfsetdash{{1.850000pt}{0.800000pt}}{0.000000pt}%
\pgfpathmoveto{\pgfqpoint{1.535234in}{0.184722in}}%
\pgfpathlineto{\pgfqpoint{3.815278in}{2.380971in}}%
\pgfpathlineto{\pgfqpoint{3.815278in}{2.380971in}}%
\pgfusepath{stroke}%
\end{pgfscope}%
\begin{pgfscope}%
\pgfpathrectangle{\pgfqpoint{0.198611in}{0.198611in}}{\pgfqpoint{3.602778in}{2.602778in}} %
\pgfusepath{clip}%
\pgfsetbuttcap%
\pgfsetroundjoin%
\pgfsetlinewidth{0.501875pt}%
\definecolor{currentstroke}{rgb}{0.501961,0.501961,0.501961}%
\pgfsetstrokecolor{currentstroke}%
\pgfsetdash{{1.850000pt}{0.800000pt}}{0.000000pt}%
\pgfpathmoveto{\pgfqpoint{0.184722in}{2.380971in}}%
\pgfpathlineto{\pgfqpoint{2.464766in}{0.184722in}}%
\pgfpathlineto{\pgfqpoint{2.464766in}{0.184722in}}%
\pgfusepath{stroke}%
\end{pgfscope}%
\begin{pgfscope}%
\pgfpathrectangle{\pgfqpoint{0.198611in}{0.198611in}}{\pgfqpoint{3.602778in}{2.602778in}} %
\pgfusepath{clip}%
\pgfsetbuttcap%
\pgfsetroundjoin%
\pgfsetlinewidth{0.501875pt}%
\definecolor{currentstroke}{rgb}{0.501961,0.501961,0.501961}%
\pgfsetstrokecolor{currentstroke}%
\pgfsetdash{{1.850000pt}{0.800000pt}}{0.000000pt}%
\pgfpathmoveto{\pgfqpoint{2.464766in}{2.815278in}}%
\pgfpathlineto{\pgfqpoint{0.184722in}{0.619029in}}%
\pgfpathlineto{\pgfqpoint{0.184722in}{0.619029in}}%
\pgfusepath{stroke}%
\end{pgfscope}%
\begin{pgfscope}%
\pgfpathrectangle{\pgfqpoint{0.198611in}{0.198611in}}{\pgfqpoint{3.602778in}{2.602778in}} %
\pgfusepath{clip}%
\pgfsetbuttcap%
\pgfsetroundjoin%
\pgfsetlinewidth{0.501875pt}%
\definecolor{currentstroke}{rgb}{0.501961,0.501961,0.501961}%
\pgfsetstrokecolor{currentstroke}%
\pgfsetdash{{1.850000pt}{0.800000pt}}{0.000000pt}%
\pgfpathmoveto{\pgfqpoint{3.815278in}{0.619029in}}%
\pgfpathlineto{\pgfqpoint{1.535234in}{2.815278in}}%
\pgfpathlineto{\pgfqpoint{1.535234in}{2.815278in}}%
\pgfusepath{stroke}%
\end{pgfscope}%
\begin{pgfscope}%
\pgfsetrectcap%
\pgfsetmiterjoin%
\pgfsetlinewidth{0.501875pt}%
\definecolor{currentstroke}{rgb}{0.000000,0.000000,0.000000}%
\pgfsetstrokecolor{currentstroke}%
\pgfsetdash{}{0pt}%
\pgfpathmoveto{\pgfqpoint{2.000000in}{0.198611in}}%
\pgfpathlineto{\pgfqpoint{2.000000in}{2.801389in}}%
\pgfusepath{stroke}%
\end{pgfscope}%
\begin{pgfscope}%
\pgfsetrectcap%
\pgfsetmiterjoin%
\pgfsetlinewidth{0.501875pt}%
\definecolor{currentstroke}{rgb}{0.000000,0.000000,0.000000}%
\pgfsetstrokecolor{currentstroke}%
\pgfsetdash{}{0pt}%
\pgfpathmoveto{\pgfqpoint{0.198611in}{0.632407in}}%
\pgfpathlineto{\pgfqpoint{3.801389in}{0.632407in}}%
\pgfusepath{stroke}%
\end{pgfscope}%
\begin{pgfscope}%
\pgfsetroundcap%
\pgfsetroundjoin%
\pgfsetlinewidth{0.501875pt}%
\definecolor{currentstroke}{rgb}{0.000000,0.000000,0.000000}%
\pgfsetstrokecolor{currentstroke}%
\pgfsetdash{}{0pt}%
\pgfpathmoveto{\pgfqpoint{2.000000in}{2.807510in}}%
\pgfpathquadraticcurveto{\pgfqpoint{2.000000in}{2.808331in}}{\pgfqpoint{2.000000in}{2.801389in}}%
\pgfusepath{stroke}%
\end{pgfscope}%
\begin{pgfscope}%
\pgfsetroundcap%
\pgfsetroundjoin%
\pgfsetlinewidth{0.501875pt}%
\definecolor{currentstroke}{rgb}{0.000000,0.000000,0.000000}%
\pgfsetstrokecolor{currentstroke}%
\pgfsetdash{}{0pt}%
\pgfpathmoveto{\pgfqpoint{1.972222in}{2.751954in}}%
\pgfpathlineto{\pgfqpoint{2.000000in}{2.807510in}}%
\pgfpathlineto{\pgfqpoint{2.027778in}{2.751954in}}%
\pgfusepath{stroke}%
\end{pgfscope}%
\begin{pgfscope}%
\pgftext[x=2.000000in,y=2.870833in,,bottom]{\rmfamily\fontsize{10.000000}{12.000000}\selectfont \(\displaystyle \omega^{\prime}\)}%
\end{pgfscope}%
\begin{pgfscope}%
\pgfsetroundcap%
\pgfsetroundjoin%
\pgfsetlinewidth{0.501875pt}%
\definecolor{currentstroke}{rgb}{0.000000,0.000000,0.000000}%
\pgfsetstrokecolor{currentstroke}%
\pgfsetdash{}{0pt}%
\pgfpathmoveto{\pgfqpoint{3.807510in}{0.632407in}}%
\pgfpathquadraticcurveto{\pgfqpoint{3.808332in}{0.632407in}}{\pgfqpoint{3.801389in}{0.632407in}}%
\pgfusepath{stroke}%
\end{pgfscope}%
\begin{pgfscope}%
\pgfsetroundcap%
\pgfsetroundjoin%
\pgfsetlinewidth{0.501875pt}%
\definecolor{currentstroke}{rgb}{0.000000,0.000000,0.000000}%
\pgfsetstrokecolor{currentstroke}%
\pgfsetdash{}{0pt}%
\pgfpathmoveto{\pgfqpoint{3.751955in}{0.660185in}}%
\pgfpathlineto{\pgfqpoint{3.807510in}{0.632407in}}%
\pgfpathlineto{\pgfqpoint{3.751955in}{0.604630in}}%
\pgfusepath{stroke}%
\end{pgfscope}%
\begin{pgfscope}%
\pgftext[x=3.870833in,y=0.632407in,left,]{\rmfamily\fontsize{10.000000}{12.000000}\selectfont \(\displaystyle k^{\prime}\)}%
\end{pgfscope}%
\end{pgfpicture}%
\makeatother%
\endgroup%
} %
        \caption{Das zu berechnende Integral aus \eqref{eq:mass_shell_convolution} visualisiert}
        \label{fig:mass_shell_convolution}
    \end{minipage}\hfill
    \begin{minipage}{0.5\textwidth}
        \centering
        \resizebox{\textwidth}{!}{%% Creator: Matplotlib, PGF backend
%%
%% To include the figure in your LaTeX document, write
%%   \input{<filename>.pgf}
%%
%% Make sure the required packages are loaded in your preamble
%%   \usepackage{pgf}
%%
%% Figures using additional raster images can only be included by \input if
%% they are in the same directory as the main LaTeX file. For loading figures
%% from other directories you can use the `import` package
%%   \usepackage{import}
%% and then include the figures with
%%   \import{<path to file>}{<filename>.pgf}
%%
%% Matplotlib used the following preamble
%%   \usepackage[utf8x]{inputenc}
%%   \usepackage[T1]{fontenc}
%%   \usepackage{amssymb}
%%
\begingroup%
\makeatletter%
\begin{pgfpicture}%
\pgfpathrectangle{\pgfpointorigin}{\pgfqpoint{4.000000in}{3.000000in}}%
\pgfusepath{use as bounding box, clip}%
\begin{pgfscope}%
\pgfsetbuttcap%
\pgfsetmiterjoin%
\definecolor{currentfill}{rgb}{1.000000,1.000000,1.000000}%
\pgfsetfillcolor{currentfill}%
\pgfsetlinewidth{0.000000pt}%
\definecolor{currentstroke}{rgb}{1.000000,1.000000,1.000000}%
\pgfsetstrokecolor{currentstroke}%
\pgfsetdash{}{0pt}%
\pgfpathmoveto{\pgfqpoint{0.000000in}{0.000000in}}%
\pgfpathlineto{\pgfqpoint{4.000000in}{0.000000in}}%
\pgfpathlineto{\pgfqpoint{4.000000in}{3.000000in}}%
\pgfpathlineto{\pgfqpoint{0.000000in}{3.000000in}}%
\pgfpathclose%
\pgfusepath{fill}%
\end{pgfscope}%
\begin{pgfscope}%
\pgfsetbuttcap%
\pgfsetmiterjoin%
\definecolor{currentfill}{rgb}{1.000000,1.000000,1.000000}%
\pgfsetfillcolor{currentfill}%
\pgfsetlinewidth{0.000000pt}%
\definecolor{currentstroke}{rgb}{0.000000,0.000000,0.000000}%
\pgfsetstrokecolor{currentstroke}%
\pgfsetstrokeopacity{0.000000}%
\pgfsetdash{}{0pt}%
\pgfpathmoveto{\pgfqpoint{0.198611in}{0.198611in}}%
\pgfpathlineto{\pgfqpoint{3.801389in}{0.198611in}}%
\pgfpathlineto{\pgfqpoint{3.801389in}{2.801389in}}%
\pgfpathlineto{\pgfqpoint{0.198611in}{2.801389in}}%
\pgfpathclose%
\pgfusepath{fill}%
\end{pgfscope}%
\begin{pgfscope}%
\pgfpathrectangle{\pgfqpoint{0.198611in}{0.198611in}}{\pgfqpoint{3.602778in}{2.602778in}} %
\pgfusepath{clip}%
\pgfsetrectcap%
\pgfsetroundjoin%
\pgfsetlinewidth{1.003750pt}%
\definecolor{currentstroke}{rgb}{0.215686,0.494118,0.721569}%
\pgfsetstrokecolor{currentstroke}%
\pgfsetdash{}{0pt}%
\pgfpathmoveto{\pgfqpoint{0.198611in}{1.239722in}}%
\pgfpathlineto{\pgfqpoint{0.235003in}{1.266013in}}%
\pgfpathlineto{\pgfqpoint{0.271395in}{1.292304in}}%
\pgfpathlineto{\pgfqpoint{0.307786in}{1.318594in}}%
\pgfpathlineto{\pgfqpoint{0.344178in}{1.344885in}}%
\pgfpathlineto{\pgfqpoint{0.380570in}{1.371176in}}%
\pgfpathlineto{\pgfqpoint{0.416961in}{1.397466in}}%
\pgfpathlineto{\pgfqpoint{0.453353in}{1.423757in}}%
\pgfpathlineto{\pgfqpoint{0.489745in}{1.450048in}}%
\pgfpathlineto{\pgfqpoint{0.526136in}{1.476338in}}%
\pgfpathlineto{\pgfqpoint{0.562528in}{1.502629in}}%
\pgfpathlineto{\pgfqpoint{0.598920in}{1.528920in}}%
\pgfpathlineto{\pgfqpoint{0.635311in}{1.555210in}}%
\pgfpathlineto{\pgfqpoint{0.671703in}{1.581501in}}%
\pgfpathlineto{\pgfqpoint{0.708095in}{1.607792in}}%
\pgfpathlineto{\pgfqpoint{0.744487in}{1.634082in}}%
\pgfpathlineto{\pgfqpoint{0.780878in}{1.660373in}}%
\pgfpathlineto{\pgfqpoint{0.817270in}{1.686664in}}%
\pgfpathlineto{\pgfqpoint{0.853662in}{1.712955in}}%
\pgfpathlineto{\pgfqpoint{0.890053in}{1.739245in}}%
\pgfpathlineto{\pgfqpoint{0.926445in}{1.765536in}}%
\pgfpathlineto{\pgfqpoint{0.962837in}{1.791827in}}%
\pgfpathlineto{\pgfqpoint{0.999228in}{1.818117in}}%
\pgfpathlineto{\pgfqpoint{1.035620in}{1.844408in}}%
\pgfpathlineto{\pgfqpoint{1.072012in}{1.870699in}}%
\pgfpathlineto{\pgfqpoint{1.108403in}{1.896989in}}%
\pgfpathlineto{\pgfqpoint{1.144795in}{1.923280in}}%
\pgfpathlineto{\pgfqpoint{1.181187in}{1.949571in}}%
\pgfpathlineto{\pgfqpoint{1.217579in}{1.975861in}}%
\pgfpathlineto{\pgfqpoint{1.253970in}{2.002152in}}%
\pgfpathlineto{\pgfqpoint{1.290362in}{2.028443in}}%
\pgfpathlineto{\pgfqpoint{1.326754in}{2.054733in}}%
\pgfpathlineto{\pgfqpoint{1.363145in}{2.081024in}}%
\pgfpathlineto{\pgfqpoint{1.399537in}{2.107315in}}%
\pgfpathlineto{\pgfqpoint{1.435929in}{2.133605in}}%
\pgfpathlineto{\pgfqpoint{1.472320in}{2.159896in}}%
\pgfpathlineto{\pgfqpoint{1.508712in}{2.186187in}}%
\pgfpathlineto{\pgfqpoint{1.545104in}{2.212478in}}%
\pgfpathlineto{\pgfqpoint{1.581496in}{2.238768in}}%
\pgfpathlineto{\pgfqpoint{1.617887in}{2.265059in}}%
\pgfpathlineto{\pgfqpoint{1.654279in}{2.291350in}}%
\pgfpathlineto{\pgfqpoint{1.690671in}{2.317640in}}%
\pgfpathlineto{\pgfqpoint{1.727062in}{2.343931in}}%
\pgfpathlineto{\pgfqpoint{1.763454in}{2.370222in}}%
\pgfpathlineto{\pgfqpoint{1.799846in}{2.396512in}}%
\pgfpathlineto{\pgfqpoint{1.836237in}{2.422803in}}%
\pgfpathlineto{\pgfqpoint{1.872629in}{2.449094in}}%
\pgfpathlineto{\pgfqpoint{1.909021in}{2.475384in}}%
\pgfpathlineto{\pgfqpoint{1.945412in}{2.501675in}}%
\pgfpathlineto{\pgfqpoint{1.981804in}{2.527966in}}%
\pgfpathlineto{\pgfqpoint{2.018196in}{2.554256in}}%
\pgfpathlineto{\pgfqpoint{2.054588in}{2.580547in}}%
\pgfpathlineto{\pgfqpoint{2.090979in}{2.606838in}}%
\pgfpathlineto{\pgfqpoint{2.127371in}{2.633129in}}%
\pgfpathlineto{\pgfqpoint{2.163763in}{2.659419in}}%
\pgfpathlineto{\pgfqpoint{2.200154in}{2.685710in}}%
\pgfpathlineto{\pgfqpoint{2.236546in}{2.712001in}}%
\pgfpathlineto{\pgfqpoint{2.272938in}{2.738291in}}%
\pgfpathlineto{\pgfqpoint{2.309329in}{2.764582in}}%
\pgfpathlineto{\pgfqpoint{2.345721in}{2.790873in}}%
\pgfpathlineto{\pgfqpoint{2.379503in}{2.815278in}}%
\pgfusepath{stroke}%
\end{pgfscope}%
\begin{pgfscope}%
\pgfpathrectangle{\pgfqpoint{0.198611in}{0.198611in}}{\pgfqpoint{3.602778in}{2.602778in}} %
\pgfusepath{clip}%
\pgfsetrectcap%
\pgfsetroundjoin%
\pgfsetlinewidth{1.003750pt}%
\definecolor{currentstroke}{rgb}{0.215686,0.494118,0.721569}%
\pgfsetstrokecolor{currentstroke}%
\pgfsetdash{}{0pt}%
\pgfpathmoveto{\pgfqpoint{1.620497in}{0.184722in}}%
\pgfpathlineto{\pgfqpoint{1.654279in}{0.209127in}}%
\pgfpathlineto{\pgfqpoint{1.690671in}{0.235418in}}%
\pgfpathlineto{\pgfqpoint{1.727062in}{0.261709in}}%
\pgfpathlineto{\pgfqpoint{1.763454in}{0.287999in}}%
\pgfpathlineto{\pgfqpoint{1.799846in}{0.314290in}}%
\pgfpathlineto{\pgfqpoint{1.836237in}{0.340581in}}%
\pgfpathlineto{\pgfqpoint{1.872629in}{0.366871in}}%
\pgfpathlineto{\pgfqpoint{1.909021in}{0.393162in}}%
\pgfpathlineto{\pgfqpoint{1.945412in}{0.419453in}}%
\pgfpathlineto{\pgfqpoint{1.981804in}{0.445744in}}%
\pgfpathlineto{\pgfqpoint{2.018196in}{0.472034in}}%
\pgfpathlineto{\pgfqpoint{2.054588in}{0.498325in}}%
\pgfpathlineto{\pgfqpoint{2.090979in}{0.524616in}}%
\pgfpathlineto{\pgfqpoint{2.127371in}{0.550906in}}%
\pgfpathlineto{\pgfqpoint{2.163763in}{0.577197in}}%
\pgfpathlineto{\pgfqpoint{2.200154in}{0.603488in}}%
\pgfpathlineto{\pgfqpoint{2.236546in}{0.629778in}}%
\pgfpathlineto{\pgfqpoint{2.272938in}{0.656069in}}%
\pgfpathlineto{\pgfqpoint{2.309329in}{0.682360in}}%
\pgfpathlineto{\pgfqpoint{2.345721in}{0.708650in}}%
\pgfpathlineto{\pgfqpoint{2.382113in}{0.734941in}}%
\pgfpathlineto{\pgfqpoint{2.418504in}{0.761232in}}%
\pgfpathlineto{\pgfqpoint{2.454896in}{0.787522in}}%
\pgfpathlineto{\pgfqpoint{2.491288in}{0.813813in}}%
\pgfpathlineto{\pgfqpoint{2.527680in}{0.840104in}}%
\pgfpathlineto{\pgfqpoint{2.564071in}{0.866395in}}%
\pgfpathlineto{\pgfqpoint{2.600463in}{0.892685in}}%
\pgfpathlineto{\pgfqpoint{2.636855in}{0.918976in}}%
\pgfpathlineto{\pgfqpoint{2.673246in}{0.945267in}}%
\pgfpathlineto{\pgfqpoint{2.709638in}{0.971557in}}%
\pgfpathlineto{\pgfqpoint{2.746030in}{0.997848in}}%
\pgfpathlineto{\pgfqpoint{2.782421in}{1.024139in}}%
\pgfpathlineto{\pgfqpoint{2.818813in}{1.050429in}}%
\pgfpathlineto{\pgfqpoint{2.855205in}{1.076720in}}%
\pgfpathlineto{\pgfqpoint{2.891597in}{1.103011in}}%
\pgfpathlineto{\pgfqpoint{2.927988in}{1.129301in}}%
\pgfpathlineto{\pgfqpoint{2.964380in}{1.155592in}}%
\pgfpathlineto{\pgfqpoint{3.000772in}{1.181883in}}%
\pgfpathlineto{\pgfqpoint{3.037163in}{1.208173in}}%
\pgfpathlineto{\pgfqpoint{3.073555in}{1.234464in}}%
\pgfpathlineto{\pgfqpoint{3.109947in}{1.260755in}}%
\pgfpathlineto{\pgfqpoint{3.146338in}{1.287045in}}%
\pgfpathlineto{\pgfqpoint{3.182730in}{1.313336in}}%
\pgfpathlineto{\pgfqpoint{3.219122in}{1.339627in}}%
\pgfpathlineto{\pgfqpoint{3.255513in}{1.365918in}}%
\pgfpathlineto{\pgfqpoint{3.291905in}{1.392208in}}%
\pgfpathlineto{\pgfqpoint{3.328297in}{1.418499in}}%
\pgfpathlineto{\pgfqpoint{3.364689in}{1.444790in}}%
\pgfpathlineto{\pgfqpoint{3.401080in}{1.471080in}}%
\pgfpathlineto{\pgfqpoint{3.437472in}{1.497371in}}%
\pgfpathlineto{\pgfqpoint{3.473864in}{1.523662in}}%
\pgfpathlineto{\pgfqpoint{3.510255in}{1.549952in}}%
\pgfpathlineto{\pgfqpoint{3.546647in}{1.576243in}}%
\pgfpathlineto{\pgfqpoint{3.583039in}{1.602534in}}%
\pgfpathlineto{\pgfqpoint{3.619430in}{1.628824in}}%
\pgfpathlineto{\pgfqpoint{3.655822in}{1.655115in}}%
\pgfpathlineto{\pgfqpoint{3.692214in}{1.681406in}}%
\pgfpathlineto{\pgfqpoint{3.728605in}{1.707696in}}%
\pgfpathlineto{\pgfqpoint{3.764997in}{1.733987in}}%
\pgfpathlineto{\pgfqpoint{3.801389in}{1.760278in}}%
\pgfusepath{stroke}%
\end{pgfscope}%
\begin{pgfscope}%
\pgfpathrectangle{\pgfqpoint{0.198611in}{0.198611in}}{\pgfqpoint{3.602778in}{2.602778in}} %
\pgfusepath{clip}%
\pgfsetrectcap%
\pgfsetroundjoin%
\pgfsetlinewidth{1.003750pt}%
\definecolor{currentstroke}{rgb}{0.894118,0.101961,0.109804}%
\pgfsetstrokecolor{currentstroke}%
\pgfsetdash{}{0pt}%
\pgfpathmoveto{\pgfqpoint{1.620497in}{2.815278in}}%
\pgfpathlineto{\pgfqpoint{1.654279in}{2.790873in}}%
\pgfpathlineto{\pgfqpoint{1.690671in}{2.764582in}}%
\pgfpathlineto{\pgfqpoint{1.727062in}{2.738291in}}%
\pgfpathlineto{\pgfqpoint{1.763454in}{2.712001in}}%
\pgfpathlineto{\pgfqpoint{1.799846in}{2.685710in}}%
\pgfpathlineto{\pgfqpoint{1.836237in}{2.659419in}}%
\pgfpathlineto{\pgfqpoint{1.872629in}{2.633129in}}%
\pgfpathlineto{\pgfqpoint{1.909021in}{2.606838in}}%
\pgfpathlineto{\pgfqpoint{1.945412in}{2.580547in}}%
\pgfpathlineto{\pgfqpoint{1.981804in}{2.554256in}}%
\pgfpathlineto{\pgfqpoint{2.018196in}{2.527966in}}%
\pgfpathlineto{\pgfqpoint{2.054588in}{2.501675in}}%
\pgfpathlineto{\pgfqpoint{2.090979in}{2.475384in}}%
\pgfpathlineto{\pgfqpoint{2.127371in}{2.449094in}}%
\pgfpathlineto{\pgfqpoint{2.163763in}{2.422803in}}%
\pgfpathlineto{\pgfqpoint{2.200154in}{2.396512in}}%
\pgfpathlineto{\pgfqpoint{2.236546in}{2.370222in}}%
\pgfpathlineto{\pgfqpoint{2.272938in}{2.343931in}}%
\pgfpathlineto{\pgfqpoint{2.309329in}{2.317640in}}%
\pgfpathlineto{\pgfqpoint{2.345721in}{2.291350in}}%
\pgfpathlineto{\pgfqpoint{2.382113in}{2.265059in}}%
\pgfpathlineto{\pgfqpoint{2.418504in}{2.238768in}}%
\pgfpathlineto{\pgfqpoint{2.454896in}{2.212478in}}%
\pgfpathlineto{\pgfqpoint{2.491288in}{2.186187in}}%
\pgfpathlineto{\pgfqpoint{2.527680in}{2.159896in}}%
\pgfpathlineto{\pgfqpoint{2.564071in}{2.133605in}}%
\pgfpathlineto{\pgfqpoint{2.600463in}{2.107315in}}%
\pgfpathlineto{\pgfqpoint{2.636855in}{2.081024in}}%
\pgfpathlineto{\pgfqpoint{2.673246in}{2.054733in}}%
\pgfpathlineto{\pgfqpoint{2.709638in}{2.028443in}}%
\pgfpathlineto{\pgfqpoint{2.746030in}{2.002152in}}%
\pgfpathlineto{\pgfqpoint{2.782421in}{1.975861in}}%
\pgfpathlineto{\pgfqpoint{2.818813in}{1.949571in}}%
\pgfpathlineto{\pgfqpoint{2.855205in}{1.923280in}}%
\pgfpathlineto{\pgfqpoint{2.891597in}{1.896989in}}%
\pgfpathlineto{\pgfqpoint{2.927988in}{1.870699in}}%
\pgfpathlineto{\pgfqpoint{2.964380in}{1.844408in}}%
\pgfpathlineto{\pgfqpoint{3.000772in}{1.818117in}}%
\pgfpathlineto{\pgfqpoint{3.037163in}{1.791827in}}%
\pgfpathlineto{\pgfqpoint{3.073555in}{1.765536in}}%
\pgfpathlineto{\pgfqpoint{3.109947in}{1.739245in}}%
\pgfpathlineto{\pgfqpoint{3.146338in}{1.712955in}}%
\pgfpathlineto{\pgfqpoint{3.182730in}{1.686664in}}%
\pgfpathlineto{\pgfqpoint{3.219122in}{1.660373in}}%
\pgfpathlineto{\pgfqpoint{3.255513in}{1.634082in}}%
\pgfpathlineto{\pgfqpoint{3.291905in}{1.607792in}}%
\pgfpathlineto{\pgfqpoint{3.328297in}{1.581501in}}%
\pgfpathlineto{\pgfqpoint{3.364689in}{1.555210in}}%
\pgfpathlineto{\pgfqpoint{3.401080in}{1.528920in}}%
\pgfpathlineto{\pgfqpoint{3.437472in}{1.502629in}}%
\pgfpathlineto{\pgfqpoint{3.473864in}{1.476338in}}%
\pgfpathlineto{\pgfqpoint{3.510255in}{1.450048in}}%
\pgfpathlineto{\pgfqpoint{3.546647in}{1.423757in}}%
\pgfpathlineto{\pgfqpoint{3.583039in}{1.397466in}}%
\pgfpathlineto{\pgfqpoint{3.619430in}{1.371176in}}%
\pgfpathlineto{\pgfqpoint{3.655822in}{1.344885in}}%
\pgfpathlineto{\pgfqpoint{3.692214in}{1.318594in}}%
\pgfpathlineto{\pgfqpoint{3.728605in}{1.292304in}}%
\pgfpathlineto{\pgfqpoint{3.764997in}{1.266013in}}%
\pgfpathlineto{\pgfqpoint{3.801389in}{1.239722in}}%
\pgfusepath{stroke}%
\end{pgfscope}%
\begin{pgfscope}%
\pgfpathrectangle{\pgfqpoint{0.198611in}{0.198611in}}{\pgfqpoint{3.602778in}{2.602778in}} %
\pgfusepath{clip}%
\pgfsetrectcap%
\pgfsetroundjoin%
\pgfsetlinewidth{1.003750pt}%
\definecolor{currentstroke}{rgb}{0.894118,0.101961,0.109804}%
\pgfsetstrokecolor{currentstroke}%
\pgfsetdash{}{0pt}%
\pgfpathmoveto{\pgfqpoint{0.198611in}{1.760278in}}%
\pgfpathlineto{\pgfqpoint{0.235003in}{1.733987in}}%
\pgfpathlineto{\pgfqpoint{0.271395in}{1.707696in}}%
\pgfpathlineto{\pgfqpoint{0.307786in}{1.681406in}}%
\pgfpathlineto{\pgfqpoint{0.344178in}{1.655115in}}%
\pgfpathlineto{\pgfqpoint{0.380570in}{1.628824in}}%
\pgfpathlineto{\pgfqpoint{0.416961in}{1.602534in}}%
\pgfpathlineto{\pgfqpoint{0.453353in}{1.576243in}}%
\pgfpathlineto{\pgfqpoint{0.489745in}{1.549952in}}%
\pgfpathlineto{\pgfqpoint{0.526136in}{1.523662in}}%
\pgfpathlineto{\pgfqpoint{0.562528in}{1.497371in}}%
\pgfpathlineto{\pgfqpoint{0.598920in}{1.471080in}}%
\pgfpathlineto{\pgfqpoint{0.635311in}{1.444790in}}%
\pgfpathlineto{\pgfqpoint{0.671703in}{1.418499in}}%
\pgfpathlineto{\pgfqpoint{0.708095in}{1.392208in}}%
\pgfpathlineto{\pgfqpoint{0.744487in}{1.365918in}}%
\pgfpathlineto{\pgfqpoint{0.780878in}{1.339627in}}%
\pgfpathlineto{\pgfqpoint{0.817270in}{1.313336in}}%
\pgfpathlineto{\pgfqpoint{0.853662in}{1.287045in}}%
\pgfpathlineto{\pgfqpoint{0.890053in}{1.260755in}}%
\pgfpathlineto{\pgfqpoint{0.926445in}{1.234464in}}%
\pgfpathlineto{\pgfqpoint{0.962837in}{1.208173in}}%
\pgfpathlineto{\pgfqpoint{0.999228in}{1.181883in}}%
\pgfpathlineto{\pgfqpoint{1.035620in}{1.155592in}}%
\pgfpathlineto{\pgfqpoint{1.072012in}{1.129301in}}%
\pgfpathlineto{\pgfqpoint{1.108403in}{1.103011in}}%
\pgfpathlineto{\pgfqpoint{1.144795in}{1.076720in}}%
\pgfpathlineto{\pgfqpoint{1.181187in}{1.050429in}}%
\pgfpathlineto{\pgfqpoint{1.217579in}{1.024139in}}%
\pgfpathlineto{\pgfqpoint{1.253970in}{0.997848in}}%
\pgfpathlineto{\pgfqpoint{1.290362in}{0.971557in}}%
\pgfpathlineto{\pgfqpoint{1.326754in}{0.945267in}}%
\pgfpathlineto{\pgfqpoint{1.363145in}{0.918976in}}%
\pgfpathlineto{\pgfqpoint{1.399537in}{0.892685in}}%
\pgfpathlineto{\pgfqpoint{1.435929in}{0.866395in}}%
\pgfpathlineto{\pgfqpoint{1.472320in}{0.840104in}}%
\pgfpathlineto{\pgfqpoint{1.508712in}{0.813813in}}%
\pgfpathlineto{\pgfqpoint{1.545104in}{0.787522in}}%
\pgfpathlineto{\pgfqpoint{1.581496in}{0.761232in}}%
\pgfpathlineto{\pgfqpoint{1.617887in}{0.734941in}}%
\pgfpathlineto{\pgfqpoint{1.654279in}{0.708650in}}%
\pgfpathlineto{\pgfqpoint{1.690671in}{0.682360in}}%
\pgfpathlineto{\pgfqpoint{1.727062in}{0.656069in}}%
\pgfpathlineto{\pgfqpoint{1.763454in}{0.629778in}}%
\pgfpathlineto{\pgfqpoint{1.799846in}{0.603488in}}%
\pgfpathlineto{\pgfqpoint{1.836237in}{0.577197in}}%
\pgfpathlineto{\pgfqpoint{1.872629in}{0.550906in}}%
\pgfpathlineto{\pgfqpoint{1.909021in}{0.524616in}}%
\pgfpathlineto{\pgfqpoint{1.945412in}{0.498325in}}%
\pgfpathlineto{\pgfqpoint{1.981804in}{0.472034in}}%
\pgfpathlineto{\pgfqpoint{2.018196in}{0.445744in}}%
\pgfpathlineto{\pgfqpoint{2.054588in}{0.419453in}}%
\pgfpathlineto{\pgfqpoint{2.090979in}{0.393162in}}%
\pgfpathlineto{\pgfqpoint{2.127371in}{0.366871in}}%
\pgfpathlineto{\pgfqpoint{2.163763in}{0.340581in}}%
\pgfpathlineto{\pgfqpoint{2.200154in}{0.314290in}}%
\pgfpathlineto{\pgfqpoint{2.236546in}{0.287999in}}%
\pgfpathlineto{\pgfqpoint{2.272938in}{0.261709in}}%
\pgfpathlineto{\pgfqpoint{2.309329in}{0.235418in}}%
\pgfpathlineto{\pgfqpoint{2.345721in}{0.209127in}}%
\pgfpathlineto{\pgfqpoint{2.379503in}{0.184722in}}%
\pgfusepath{stroke}%
\end{pgfscope}%
\begin{pgfscope}%
\pgfpathrectangle{\pgfqpoint{0.198611in}{0.198611in}}{\pgfqpoint{3.602778in}{2.602778in}} %
\pgfusepath{clip}%
\pgfsetrectcap%
\pgfsetroundjoin%
\pgfsetlinewidth{0.501875pt}%
\definecolor{currentstroke}{rgb}{0.501961,0.501961,0.501961}%
\pgfsetstrokecolor{currentstroke}%
\pgfsetdash{}{0pt}%
\pgfpathmoveto{\pgfqpoint{0.955194in}{1.213694in}}%
\pgfpathlineto{\pgfqpoint{2.000000in}{2.541111in}}%
\pgfusepath{stroke}%
\end{pgfscope}%
\begin{pgfscope}%
\pgfpathrectangle{\pgfqpoint{0.198611in}{0.198611in}}{\pgfqpoint{3.602778in}{2.602778in}} %
\pgfusepath{clip}%
\pgfsetrectcap%
\pgfsetroundjoin%
\pgfsetlinewidth{0.501875pt}%
\definecolor{currentstroke}{rgb}{0.501961,0.501961,0.501961}%
\pgfsetstrokecolor{currentstroke}%
\pgfsetdash{}{0pt}%
\pgfpathmoveto{\pgfqpoint{2.000000in}{0.458889in}}%
\pgfpathlineto{\pgfqpoint{3.441111in}{0.458889in}}%
\pgfusepath{stroke}%
\end{pgfscope}%
\begin{pgfscope}%
\pgfpathrectangle{\pgfqpoint{0.198611in}{0.198611in}}{\pgfqpoint{3.602778in}{2.602778in}} %
\pgfusepath{clip}%
\pgfsetrectcap%
\pgfsetroundjoin%
\pgfsetlinewidth{0.501875pt}%
\definecolor{currentstroke}{rgb}{0.501961,0.501961,0.501961}%
\pgfsetstrokecolor{currentstroke}%
\pgfsetdash{}{0pt}%
\pgfpathmoveto{\pgfqpoint{3.441111in}{0.458889in}}%
\pgfpathlineto{\pgfqpoint{3.441111in}{1.500000in}}%
\pgfusepath{stroke}%
\end{pgfscope}%
\begin{pgfscope}%
\pgftext[x=2.720556in,y=0.289708in,left,base]{\rmfamily\fontsize{10.000000}{12.000000}\selectfont d\(\displaystyle k^\prime\)}%
\end{pgfscope}%
\begin{pgfscope}%
\pgftext[x=3.477139in,y=0.914375in,left,base]{\rmfamily\fontsize{10.000000}{12.000000}\selectfont d\(\displaystyle \omega^\prime\)}%
\end{pgfscope}%
\begin{pgfscope}%
\pgftext[x=2.180139in,y=0.497931in,left,base]{\rmfamily\fontsize{10.000000}{12.000000}\selectfont \(\displaystyle \alpha\)}%
\end{pgfscope}%
\begin{pgfscope}%
\pgftext[x=0.703000in,y=1.460958in,left,base]{\rmfamily\fontsize{10.000000}{12.000000}\selectfont \(\displaystyle 2 \alpha\)}%
\end{pgfscope}%
\begin{pgfscope}%
\pgftext[x=1.189375in,y=2.085625in,left,base]{\rmfamily\fontsize{10.000000}{12.000000}\selectfont \(\displaystyle l\)}%
\end{pgfscope}%
\begin{pgfscope}%
\pgftext[x=1.549653in,y=1.825347in,left,base]{\rmfamily\fontsize{10.000000}{12.000000}\selectfont \(\displaystyle h\)}%
\end{pgfscope}%
\end{pgfpicture}%
\makeatother%
\endgroup%
}
        \caption{Die Kreuzungstelle bei $k_{0+}'$ von ganz nah angeschaut}
        \label{fig:schulgeometrie}
    \end{minipage}
\end{figure}

\begin{equation}
    \rwhat{\Delta_m^{* 2}} (\omega, k)
    = \int \theta(\omega') \delta(\omega'^2-k'^2-m^2)\theta(\omega-\omega')
      \delta((\omega-\omega')^2-(k-k')^2-m^2) \d \omega' \d k'
\label{eq:mass_shell_convolution}
\end{equation}

An Abbildung \ref{fig:mass_shell_convolution} sehen wir schon, dass das Faltungsintegral nur dann ungleich null ist, wenn $(\omega, k)$ in der 2$m$-Massenschale liegen. Es ist also insbesondere $\omega > 0$.
Da $\Delta_m$ Poincare-invariant ist, sind $\Delta_m^2$ und $\rwhat{\Delta_m^{*2}}$ es auch. Es genügt also $\rwhat{\Delta_m^{*2}}$ für $k=0$ und positive $\omega$ zu berechnen. Alle anderen Werte holen wir uns dann aus der Poincare-Invarianz.

\todo{Wie erklärt man das besser, ohne an Anschaulichkeit oder Rigorosität zu verlieren}
Um nun das Integral über zwei sich schneidende lineare\footnote{Linear in dem Sinne, dass die Distribution entlang einer Linie getragen ist. Nicht das es eine lineare Distribution ist} $\delta$-Distributionen zu berechnen bedienen wir uns eines Physikertricks und stellen uns die $\delta$-Distribution als Grenzwert einer $\frac{1}{h}$-hohen und $h$ breiten Rechtecksfunktion vor.
% section die_wellenfrontmenge_von_delta_m_2_ (end)


% !TEX root = main.tex
% !TEX spellcheck=de_DE
%%%%%%%%%%%%%%%%%%%%%%%%%%%%%%%%%%%%%%%%%%%%%%%%%%%%%%%%%%%%%%%%%%%%%%%%%%%%%%%
% % Berechnen der Wellenfrontmenge von Delta_m_twisted
%%%%%%%%%%%%%%%%%%%%%%%%%%%%%%%%%%%%%%%%%%%%%%%%%%%%%%%%%%%%%%%%%%%%%%%%%%%%%%%

\section{\texorpdfstring{Die Wellenfrontmenge von $\Delta_m^{\star 2}$}
         {Die Wellenfrontmenge der getwisteten Zweipunktfunktion}} % (fold)
\label{sec:die_wellenfrontmenge_von_delta_m2_twisted}

Bevor wir uns aber der Wellenfrontmenge widmen können, brauchen wir einen Ausdruck für die Fouriertransformierte $\rwhat{\Delta}_m^{\circledast 2}$ von $\Delta_m^{\star 2}$.

\subsection{\texorpdfstring{$\hat\Delta_m^{\circledast 2}$ berechnen}
            {Die getwistete Zweipunktfunktion berechnen}} % (fold)
\label{sec:delta_m2_twisted_berechnen}

Sammeln wir zunächst einmal die Zutaten, die wir für die getwistete Faltung der massiven Zweipunktfunktion mit sich selber brauchen:

\begin{dgroup}
    \begin{dmath}
        \rwhat{\Delta}_m = \delta(\omega^2-k^w-m^2)\Theta(\omega)\\
        \textrm{die Fouriertransformierte der massiven Zweipunktfunktion}
    \label{eq:material_fuer_delta_m2_twisted_a}
    \end{dmath}
    \begin{dmath}
        \Omega = \begin{pmatrix}
            0 & 1 \\ -1 & 1
        \end{pmatrix}
        \\ \textrm{die kanonische symplektische Matrix auf } \mathbb{R}^n
    \label{eq:material_fuer_delta_m2_twisted_b}
    \end{dmath}
\end{dgroup}

mit \cref{def:twisted_convolution,eq:material_fuer_delta_m2_twisted_a,eq:material_fuer_delta_m2_twisted_b} erhalten wir also

\begin{dmath}
    \rwhat{\Delta}_m^{\circledast 2} (\omega, k)
    = \int
    \delta(\omega^{\prime 2}-k^{\prime 2}-m^2)
    \delta((\omega' - \omega)^2 - (k-k')^2 -m^2)
    \cdot
    \Theta(\omega') \Theta(\omega - \omega')
    e^{\frac{i}{2}(\omega'k-\omega k')}
    \d \omega' \d k'
\end{dmath}

und damit das selbe Integral wie in \cref{eq:mass_shell_convolution} bis auf einen zusätzlichen Phasenfaktor. Nachdem wir gezeigt haben, dass auch dieser Lorenz-Invariant ist, können wir das Integral mit dem selben Trick wie in \cref{sec:delta_m2_berechnen} berechnen.

\begin{proposition}[$\Omega_{std}$ ist Lorenz-invariant für $n=2$]
\label{prop:omega_ist_lorenz_invariant}
    $\Omega_{std}$ ist Lorenz-invariant für $n=2$
\\[1em]
\emph{Beweis}\\
    Eine einfache Rechnung zeigt
    \begin{dmath*}
        \begin{pmatrix}
            \cosh \beta & \sinh \beta \\ -\sinh \beta & \cosh \beta
        \end{pmatrix}
        \begin{pmatrix}
            0 & 1 \\ -1 & 0
        \end{pmatrix}
        \begin{pmatrix}
            \cosh \beta & -\sinh \beta \\ \sinh \beta & \cosh \beta
        \end{pmatrix}
        =
        \begin{pmatrix}
            0 & 1 \\ -1 & 0
        \end{pmatrix}
    \end{dmath*}
    für alle $\beta \in \mathbb{R}$
\end{proposition}

Mit \cref{prop:omega_ist_lorenz_invariant} ist $\rwhat{\Delta}_m^{\circledast 2}$ Lorenz-Invariant und es reicht aus $\rwhat{\Delta}_m^{\circledast 2} (\omega, 0)$ zu berechnen.

\todo{herausfinden, warum hier Satz statt Proposition geschrieben wird... Warum ist cleverref nicht so clever?}

Die beiden Kreuzungspunkte der $\delta$-Distributionen liegen bei (vgl. \cref{fig:mass_shell_convolution})

\begin{equation*}
    \left(\omega'_0,k'_{0\pm}\right) = \left(\frac{\omega}{2}, \pm \sqrt{\left(\frac{\omega}{2}\right)^2-m^2}\right)
\end{equation*}


Die "`Fläche"' der Kreuzungspunkte der $\delta$-Distributionen wurde in
 \cref{sec:delta_m2_berechnen} berechnet und ist

\begin{equation*}
A = \frac{\omega^2-3m^2}{\omega \sqrt{\omega^2-4m^2}}
\end{equation*}

Der Phasenfaktor nimmt bei den Kreuzungspunkten folgende Werte an:
\begin{dmath*}
    e^{\frac{i}{2}\Omega \left((\omega, k),(\omega'_0,k'_{0\pm})\right)}
    =
    e^{\pm \frac{i}{2}\left(-\omega^2\sqrt{\frac{1}{4}-\frac{m^2}{\omega^2}}\right)}
\end{dmath*}


Kombinieren wir also die vorhergehenden Resultate erhalten wir

\begin{align*}
    \rwhat{\Delta}_m^{\circledast 2} (\omega, 0)
    &=
    A e^{\frac{i}{2}\Omega \left((\omega, k),(\omega'_0,k'_{0+})\right)}
    + A e^{\frac{i}{2}\Omega \left((\omega, k),(\omega'_0,k'_{0-})\right)}
    \\&=
    \frac{\omega^2-3m^2}{\omega \sqrt{\omega^2-4m^2}}
    \left\{
        e^{-\frac{i}{2}\omega^2\sqrt{\frac{1}{4}-\frac{m^2}{\omega^2}}}
      + e^{\frac{i}{2}\omega^2\sqrt{\frac{1}{4}-\frac{m^2}{\omega^2}}}
    \right\}
    \Theta\left(\omega^2-4m^2\right)
    \\&=
    2 \frac{\omega^2 -3m^2}{\omega \sqrt{\omega^2-4m^2}}
    \cos \left(\varphi(\omega^2)\right) \Theta\left(\omega^2-4m^2\right)
\end{align*}
% \begin{dmath*}
%     \rwhat{\Delta}_m^{\circledast 2} (\omega, 0)
%     =
%     A e^{\frac{i}{2}\Omega \left((\omega, k),(\omega'_0,k'_{0+})\right)}
%     + A e^{\frac{i}{2}\Omega \left((\omega, k),(\omega'_0,k'_{0-})\right)}
%     =
%     \frac{\omega^2-3m^2}{\omega \sqrt{\omega^2-4m^2}}
%     \left\{
%         e^{-\frac{i}{2}\omega^2\sqrt{\frac{1}{4}-\frac{m^2}{\omega^2}}}
%       + e^{\frac{i}{2}\omega^2\sqrt{\frac{1}{4}-\frac{m^2}{\omega^2}}}
%     \right\}
%     \Theta\left(\omega^2-4m^2\right)
%     =
%     2 \frac{\omega^2 -3m^2}{\omega \sqrt{\omega^2-4m^2}}
%     \cos \left(\varphi(\omega^2)\right) \Theta\left(\omega^2-4m^2\right)
% \end{dmath*}

wobei im letzten Schritt noch implizit $\varphi(\omega^2)$ definiert wurde.
Und mit Lorenz-Invarianz erhalten wir schließlich

\begin{align}
    \rwhat{\Delta}_m^{\circledast 2} (\omega, k)
    &=
    \rwhat{\Delta}_m^{\circledast 2} (\sqrt{\omega^2-k^2}, 0)
    \nonumber \\ &=
    2\frac{\omega^2-k^2-3m^2}{\sqrt{\omega^2-k^2} \sqrt{\omega^2-k^2-4m^2}}
    \cos \left(\frac{k^2-\omega^2}{2}
    \sqrt{\frac{1}{4}+\frac{m^2}{k^2-\omega^2}}
    \right)
    \nonumber \\ & \kern 12em\cdot
    \Theta \left(\omega^2-k^2-4m^2\right)
    \nonumber \\ &=
    \rwhat{\Delta}_m^{* 2}(\omega, k) \cos (\varphi(\omega^2-k^2))
    \Theta\left(\omega^2-k^2-4m^2\right)
\end{align}
% \begin{dmath}
%     \rwhat{\Delta}_m^{\circledast 2} (\omega, k)
%     =
%     \rwhat{\Delta}_m^{\circledast 2} (\sqrt{\omega^2-k^2}, 0)
%     =
%     2\frac{\omega^2-k^2-3m^2}{\sqrt{\omega^2-k^2} \sqrt{\omega^2-k^2-4m^2}}
%     \cos \left(\frac{k^2-\omega^2}{2}
%     \sqrt{\frac{1}{4}+\frac{m^2}{k^2-\omega^2}}
%     \right)
%     \Theta \left(\omega^2-k^2-4m^2\right)
%     =
%     \rwhat{\Delta}_m^{* 2}(\omega, k) \cos (\varphi(\omega^2-k^2))
% \end{dmath}

\subsection{
\texorpdfstring{\dots und nun zur Wellenfrontmenge von $\hat{\Delta}_m^{\circledast 2}$}{... und nun zur Wellenfrontmenge der getwisteten Zweipunktfunktion}} % (fold)
\label{sec:dots_und_nun_zur_wellenfrontmenge_von_delta_m2_twisted}

\begin{figure}
    \centering
    \begin{minipage}{0.55\textwidth}
        \centering
        \resizebox{\textwidth}{!}{%% Creator: Matplotlib, PGF backend
%%
%% To include the figure in your LaTeX document, write
%%   \input{<filename>.pgf}
%%
%% Make sure the required packages are loaded in your preamble
%%   \usepackage{pgf}
%%
%% Figures using additional raster images can only be included by \input if
%% they are in the same directory as the main LaTeX file. For loading figures
%% from other directories you can use the `import` package
%%   \usepackage{import}
%% and then include the figures with
%%   \import{<path to file>}{<filename>.pgf}
%%
%% Matplotlib used the following preamble
%%   \usepackage[utf8x]{inputenc}
%%   \usepackage[T1]{fontenc}
%%   \usepackage{amssymb}
%%
\begingroup%
\makeatletter%
\begin{pgfpicture}%
\pgfpathrectangle{\pgfpointorigin}{\pgfqpoint{10.000000in}{5.500000in}}%
\pgfusepath{use as bounding box, clip}%
\begin{pgfscope}%
\pgfsetbuttcap%
\pgfsetmiterjoin%
\definecolor{currentfill}{rgb}{1.000000,1.000000,1.000000}%
\pgfsetfillcolor{currentfill}%
\pgfsetlinewidth{0.000000pt}%
\definecolor{currentstroke}{rgb}{1.000000,1.000000,1.000000}%
\pgfsetstrokecolor{currentstroke}%
\pgfsetdash{}{0pt}%
\pgfpathmoveto{\pgfqpoint{0.000000in}{0.000000in}}%
\pgfpathlineto{\pgfqpoint{10.000000in}{0.000000in}}%
\pgfpathlineto{\pgfqpoint{10.000000in}{5.500000in}}%
\pgfpathlineto{\pgfqpoint{0.000000in}{5.500000in}}%
\pgfpathclose%
\pgfusepath{fill}%
\end{pgfscope}%
\begin{pgfscope}%
\pgfsetbuttcap%
\pgfsetmiterjoin%
\definecolor{currentfill}{rgb}{1.000000,1.000000,1.000000}%
\pgfsetfillcolor{currentfill}%
\pgfsetlinewidth{0.000000pt}%
\definecolor{currentstroke}{rgb}{0.000000,0.000000,0.000000}%
\pgfsetstrokecolor{currentstroke}%
\pgfsetstrokeopacity{0.000000}%
\pgfsetdash{}{0pt}%
\pgfpathmoveto{\pgfqpoint{0.198611in}{0.723208in}}%
\pgfpathlineto{\pgfqpoint{7.919722in}{0.723208in}}%
\pgfpathlineto{\pgfqpoint{7.919722in}{4.776792in}}%
\pgfpathlineto{\pgfqpoint{0.198611in}{4.776792in}}%
\pgfpathclose%
\pgfusepath{fill}%
\end{pgfscope}%
\begin{pgfscope}%
\pgfpathrectangle{\pgfqpoint{0.198611in}{0.723208in}}{\pgfqpoint{7.721111in}{4.053583in}} %
\pgfusepath{clip}%
\pgfsys@transformshift{0.198611in}{0.723208in}%
\pgftext[left,bottom]{\pgfimage[interpolate=true,width=7.722222in,height=4.055556in]{delta_m2_twisted-img0.png}}%
\end{pgfscope}%
\begin{pgfscope}%
\pgfpathrectangle{\pgfqpoint{0.198611in}{0.723208in}}{\pgfqpoint{7.721111in}{4.053583in}} %
\pgfusepath{clip}%
\pgfsetrectcap%
\pgfsetroundjoin%
\pgfsetlinewidth{0.501875pt}%
\definecolor{currentstroke}{rgb}{0.894118,0.101961,0.109804}%
\pgfsetstrokecolor{currentstroke}%
\pgfsetdash{}{0pt}%
\pgfpathmoveto{\pgfqpoint{0.204004in}{4.790681in}}%
\pgfpathlineto{\pgfqpoint{1.303835in}{3.698482in}}%
\pgfpathlineto{\pgfqpoint{1.991702in}{3.019436in}}%
\pgfpathlineto{\pgfqpoint{2.447704in}{2.573297in}}%
\pgfpathlineto{\pgfqpoint{2.772315in}{2.259749in}}%
\pgfpathlineto{\pgfqpoint{3.011909in}{2.032385in}}%
\pgfpathlineto{\pgfqpoint{3.197401in}{1.860524in}}%
\pgfpathlineto{\pgfqpoint{3.336520in}{1.735539in}}%
\pgfpathlineto{\pgfqpoint{3.452453in}{1.635361in}}%
\pgfpathlineto{\pgfqpoint{3.545199in}{1.559044in}}%
\pgfpathlineto{\pgfqpoint{3.622487in}{1.499098in}}%
\pgfpathlineto{\pgfqpoint{3.692047in}{1.448980in}}%
\pgfpathlineto{\pgfqpoint{3.753877in}{1.408415in}}%
\pgfpathlineto{\pgfqpoint{3.807979in}{1.376816in}}%
\pgfpathlineto{\pgfqpoint{3.854352in}{1.353258in}}%
\pgfpathlineto{\pgfqpoint{3.900725in}{1.333540in}}%
\pgfpathlineto{\pgfqpoint{3.939370in}{1.320452in}}%
\pgfpathlineto{\pgfqpoint{3.978014in}{1.310729in}}%
\pgfpathlineto{\pgfqpoint{4.016658in}{1.304625in}}%
\pgfpathlineto{\pgfqpoint{4.055302in}{1.302311in}}%
\pgfpathlineto{\pgfqpoint{4.086218in}{1.303238in}}%
\pgfpathlineto{\pgfqpoint{4.124862in}{1.307841in}}%
\pgfpathlineto{\pgfqpoint{4.163506in}{1.316143in}}%
\pgfpathlineto{\pgfqpoint{4.202150in}{1.327919in}}%
\pgfpathlineto{\pgfqpoint{4.240794in}{1.342883in}}%
\pgfpathlineto{\pgfqpoint{4.287167in}{1.364592in}}%
\pgfpathlineto{\pgfqpoint{4.333540in}{1.389860in}}%
\pgfpathlineto{\pgfqpoint{4.387642in}{1.423124in}}%
\pgfpathlineto{\pgfqpoint{4.449473in}{1.465215in}}%
\pgfpathlineto{\pgfqpoint{4.519033in}{1.516666in}}%
\pgfpathlineto{\pgfqpoint{4.596321in}{1.577730in}}%
\pgfpathlineto{\pgfqpoint{4.689067in}{1.655028in}}%
\pgfpathlineto{\pgfqpoint{4.805000in}{1.756061in}}%
\pgfpathlineto{\pgfqpoint{4.944119in}{1.881730in}}%
\pgfpathlineto{\pgfqpoint{5.114153in}{2.039640in}}%
\pgfpathlineto{\pgfqpoint{5.330561in}{2.244951in}}%
\pgfpathlineto{\pgfqpoint{5.616528in}{2.520734in}}%
\pgfpathlineto{\pgfqpoint{6.002970in}{2.898006in}}%
\pgfpathlineto{\pgfqpoint{6.536260in}{3.423232in}}%
\pgfpathlineto{\pgfqpoint{7.309144in}{4.189062in}}%
\pgfpathlineto{\pgfqpoint{7.914330in}{4.790681in}}%
\pgfpathlineto{\pgfqpoint{7.914330in}{4.790681in}}%
\pgfusepath{stroke}%
\end{pgfscope}%
\begin{pgfscope}%
\pgfpathrectangle{\pgfqpoint{0.198611in}{0.723208in}}{\pgfqpoint{7.721111in}{4.053583in}} %
\pgfusepath{clip}%
\pgfsetbuttcap%
\pgfsetroundjoin%
\pgfsetlinewidth{0.501875pt}%
\definecolor{currentstroke}{rgb}{0.501961,0.501961,0.501961}%
\pgfsetstrokecolor{currentstroke}%
\pgfsetdash{{1.850000pt}{0.800000pt}}{0.000000pt}%
\pgfpathmoveto{\pgfqpoint{3.852250in}{0.709319in}}%
\pgfpathlineto{\pgfqpoint{7.919722in}{4.776792in}}%
\pgfpathlineto{\pgfqpoint{7.919722in}{4.776792in}}%
\pgfusepath{stroke}%
\end{pgfscope}%
\begin{pgfscope}%
\pgfpathrectangle{\pgfqpoint{0.198611in}{0.723208in}}{\pgfqpoint{7.721111in}{4.053583in}} %
\pgfusepath{clip}%
\pgfsetbuttcap%
\pgfsetroundjoin%
\pgfsetlinewidth{0.501875pt}%
\definecolor{currentstroke}{rgb}{0.501961,0.501961,0.501961}%
\pgfsetstrokecolor{currentstroke}%
\pgfsetdash{{1.850000pt}{0.800000pt}}{0.000000pt}%
\pgfpathmoveto{\pgfqpoint{0.198611in}{4.776792in}}%
\pgfpathlineto{\pgfqpoint{4.266083in}{0.709319in}}%
\pgfpathlineto{\pgfqpoint{4.266083in}{0.709319in}}%
\pgfusepath{stroke}%
\end{pgfscope}%
\begin{pgfscope}%
\pgfsetrectcap%
\pgfsetmiterjoin%
\pgfsetlinewidth{0.501875pt}%
\definecolor{currentstroke}{rgb}{0.000000,0.000000,0.000000}%
\pgfsetstrokecolor{currentstroke}%
\pgfsetdash{}{0pt}%
\pgfpathmoveto{\pgfqpoint{4.059167in}{0.723208in}}%
\pgfpathlineto{\pgfqpoint{4.059167in}{4.776792in}}%
\pgfusepath{stroke}%
\end{pgfscope}%
\begin{pgfscope}%
\pgfsetrectcap%
\pgfsetmiterjoin%
\pgfsetlinewidth{0.501875pt}%
\definecolor{currentstroke}{rgb}{0.000000,0.000000,0.000000}%
\pgfsetstrokecolor{currentstroke}%
\pgfsetdash{}{0pt}%
\pgfpathmoveto{\pgfqpoint{0.198611in}{0.916236in}}%
\pgfpathlineto{\pgfqpoint{7.919722in}{0.916236in}}%
\pgfusepath{stroke}%
\end{pgfscope}%
\begin{pgfscope}%
\pgfsetroundcap%
\pgfsetroundjoin%
\pgfsetlinewidth{0.501875pt}%
\definecolor{currentstroke}{rgb}{0.000000,0.000000,0.000000}%
\pgfsetstrokecolor{currentstroke}%
\pgfsetdash{}{0pt}%
\pgfpathmoveto{\pgfqpoint{4.059167in}{4.782913in}}%
\pgfpathquadraticcurveto{\pgfqpoint{4.059167in}{4.783734in}}{\pgfqpoint{4.059167in}{4.776792in}}%
\pgfusepath{stroke}%
\end{pgfscope}%
\begin{pgfscope}%
\pgfsetroundcap%
\pgfsetroundjoin%
\pgfsetlinewidth{0.501875pt}%
\definecolor{currentstroke}{rgb}{0.000000,0.000000,0.000000}%
\pgfsetstrokecolor{currentstroke}%
\pgfsetdash{}{0pt}%
\pgfpathmoveto{\pgfqpoint{4.031389in}{4.727357in}}%
\pgfpathlineto{\pgfqpoint{4.059167in}{4.782913in}}%
\pgfpathlineto{\pgfqpoint{4.086944in}{4.727357in}}%
\pgfusepath{stroke}%
\end{pgfscope}%
\begin{pgfscope}%
\pgftext[x=4.059167in,y=4.846236in,,bottom]{\rmfamily\fontsize{10.000000}{12.000000}\selectfont \(\displaystyle \omega\)}%
\end{pgfscope}%
\begin{pgfscope}%
\pgfsetroundcap%
\pgfsetroundjoin%
\pgfsetlinewidth{0.501875pt}%
\definecolor{currentstroke}{rgb}{0.000000,0.000000,0.000000}%
\pgfsetstrokecolor{currentstroke}%
\pgfsetdash{}{0pt}%
\pgfpathmoveto{\pgfqpoint{7.925821in}{0.916236in}}%
\pgfpathquadraticcurveto{\pgfqpoint{7.926654in}{0.916236in}}{\pgfqpoint{7.919722in}{0.916236in}}%
\pgfusepath{stroke}%
\end{pgfscope}%
\begin{pgfscope}%
\pgfsetroundcap%
\pgfsetroundjoin%
\pgfsetlinewidth{0.501875pt}%
\definecolor{currentstroke}{rgb}{0.000000,0.000000,0.000000}%
\pgfsetstrokecolor{currentstroke}%
\pgfsetdash{}{0pt}%
\pgfpathmoveto{\pgfqpoint{7.870265in}{0.944014in}}%
\pgfpathlineto{\pgfqpoint{7.925821in}{0.916236in}}%
\pgfpathlineto{\pgfqpoint{7.870265in}{0.888458in}}%
\pgfusepath{stroke}%
\end{pgfscope}%
\begin{pgfscope}%
\pgftext[x=7.989167in,y=0.916236in,left,]{\rmfamily\fontsize{10.000000}{12.000000}\selectfont \(\displaystyle k\)}%
\end{pgfscope}%
\begin{pgfscope}%
\pgfpathrectangle{\pgfqpoint{8.402292in}{0.930000in}}{\pgfqpoint{0.173333in}{3.640000in}} %
\pgfusepath{clip}%
\pgfsetbuttcap%
\pgfsetmiterjoin%
\definecolor{currentfill}{rgb}{1.000000,1.000000,1.000000}%
\pgfsetfillcolor{currentfill}%
\pgfsetlinewidth{0.010037pt}%
\definecolor{currentstroke}{rgb}{1.000000,1.000000,1.000000}%
\pgfsetstrokecolor{currentstroke}%
\pgfsetdash{}{0pt}%
\pgfpathmoveto{\pgfqpoint{8.402292in}{0.930000in}}%
\pgfpathlineto{\pgfqpoint{8.402292in}{0.943542in}}%
\pgfpathlineto{\pgfqpoint{8.402292in}{4.396667in}}%
\pgfpathlineto{\pgfqpoint{8.488958in}{4.570000in}}%
\pgfpathlineto{\pgfqpoint{8.488958in}{4.570000in}}%
\pgfpathlineto{\pgfqpoint{8.575625in}{4.396667in}}%
\pgfpathlineto{\pgfqpoint{8.575625in}{0.943542in}}%
\pgfpathlineto{\pgfqpoint{8.575625in}{0.930000in}}%
\pgfpathclose%
\pgfusepath{stroke,fill}%
\end{pgfscope}%
\begin{pgfscope}%
\pgfsys@transformshift{8.402778in}{0.930556in}%
\pgftext[left,bottom]{\pgfimage[interpolate=true,width=0.166667in,height=3.638889in]{delta_m2_twisted-img1.png}}%
\end{pgfscope}%
\begin{pgfscope}%
\pgfsetbuttcap%
\pgfsetroundjoin%
\definecolor{currentfill}{rgb}{0.000000,0.000000,0.000000}%
\pgfsetfillcolor{currentfill}%
\pgfsetlinewidth{0.803000pt}%
\definecolor{currentstroke}{rgb}{0.000000,0.000000,0.000000}%
\pgfsetstrokecolor{currentstroke}%
\pgfsetdash{}{0pt}%
\pgfsys@defobject{currentmarker}{\pgfqpoint{0.000000in}{0.000000in}}{\pgfqpoint{0.048611in}{0.000000in}}{%
\pgfpathmoveto{\pgfqpoint{0.000000in}{0.000000in}}%
\pgfpathlineto{\pgfqpoint{0.048611in}{0.000000in}}%
\pgfusepath{stroke,fill}%
}%
\begin{pgfscope}%
\pgfsys@transformshift{8.575625in}{0.930000in}%
\pgfsys@useobject{currentmarker}{}%
\end{pgfscope}%
\end{pgfscope}%
\begin{pgfscope}%
\pgftext[x=8.672847in,y=0.882172in,left,base]{\rmfamily\fontsize{10.000000}{12.000000}\selectfont \(\displaystyle -4\)}%
\end{pgfscope}%
\begin{pgfscope}%
\pgfsetbuttcap%
\pgfsetroundjoin%
\definecolor{currentfill}{rgb}{0.000000,0.000000,0.000000}%
\pgfsetfillcolor{currentfill}%
\pgfsetlinewidth{0.803000pt}%
\definecolor{currentstroke}{rgb}{0.000000,0.000000,0.000000}%
\pgfsetstrokecolor{currentstroke}%
\pgfsetdash{}{0pt}%
\pgfsys@defobject{currentmarker}{\pgfqpoint{0.000000in}{0.000000in}}{\pgfqpoint{0.048611in}{0.000000in}}{%
\pgfpathmoveto{\pgfqpoint{0.000000in}{0.000000in}}%
\pgfpathlineto{\pgfqpoint{0.048611in}{0.000000in}}%
\pgfusepath{stroke,fill}%
}%
\begin{pgfscope}%
\pgfsys@transformshift{8.575625in}{1.363333in}%
\pgfsys@useobject{currentmarker}{}%
\end{pgfscope}%
\end{pgfscope}%
\begin{pgfscope}%
\pgftext[x=8.672847in,y=1.315506in,left,base]{\rmfamily\fontsize{10.000000}{12.000000}\selectfont \(\displaystyle -3\)}%
\end{pgfscope}%
\begin{pgfscope}%
\pgfsetbuttcap%
\pgfsetroundjoin%
\definecolor{currentfill}{rgb}{0.000000,0.000000,0.000000}%
\pgfsetfillcolor{currentfill}%
\pgfsetlinewidth{0.803000pt}%
\definecolor{currentstroke}{rgb}{0.000000,0.000000,0.000000}%
\pgfsetstrokecolor{currentstroke}%
\pgfsetdash{}{0pt}%
\pgfsys@defobject{currentmarker}{\pgfqpoint{0.000000in}{0.000000in}}{\pgfqpoint{0.048611in}{0.000000in}}{%
\pgfpathmoveto{\pgfqpoint{0.000000in}{0.000000in}}%
\pgfpathlineto{\pgfqpoint{0.048611in}{0.000000in}}%
\pgfusepath{stroke,fill}%
}%
\begin{pgfscope}%
\pgfsys@transformshift{8.575625in}{1.796667in}%
\pgfsys@useobject{currentmarker}{}%
\end{pgfscope}%
\end{pgfscope}%
\begin{pgfscope}%
\pgftext[x=8.672847in,y=1.748839in,left,base]{\rmfamily\fontsize{10.000000}{12.000000}\selectfont \(\displaystyle -2\)}%
\end{pgfscope}%
\begin{pgfscope}%
\pgfsetbuttcap%
\pgfsetroundjoin%
\definecolor{currentfill}{rgb}{0.000000,0.000000,0.000000}%
\pgfsetfillcolor{currentfill}%
\pgfsetlinewidth{0.803000pt}%
\definecolor{currentstroke}{rgb}{0.000000,0.000000,0.000000}%
\pgfsetstrokecolor{currentstroke}%
\pgfsetdash{}{0pt}%
\pgfsys@defobject{currentmarker}{\pgfqpoint{0.000000in}{0.000000in}}{\pgfqpoint{0.048611in}{0.000000in}}{%
\pgfpathmoveto{\pgfqpoint{0.000000in}{0.000000in}}%
\pgfpathlineto{\pgfqpoint{0.048611in}{0.000000in}}%
\pgfusepath{stroke,fill}%
}%
\begin{pgfscope}%
\pgfsys@transformshift{8.575625in}{2.230000in}%
\pgfsys@useobject{currentmarker}{}%
\end{pgfscope}%
\end{pgfscope}%
\begin{pgfscope}%
\pgftext[x=8.672847in,y=2.182172in,left,base]{\rmfamily\fontsize{10.000000}{12.000000}\selectfont \(\displaystyle -1\)}%
\end{pgfscope}%
\begin{pgfscope}%
\pgfsetbuttcap%
\pgfsetroundjoin%
\definecolor{currentfill}{rgb}{0.000000,0.000000,0.000000}%
\pgfsetfillcolor{currentfill}%
\pgfsetlinewidth{0.803000pt}%
\definecolor{currentstroke}{rgb}{0.000000,0.000000,0.000000}%
\pgfsetstrokecolor{currentstroke}%
\pgfsetdash{}{0pt}%
\pgfsys@defobject{currentmarker}{\pgfqpoint{0.000000in}{0.000000in}}{\pgfqpoint{0.048611in}{0.000000in}}{%
\pgfpathmoveto{\pgfqpoint{0.000000in}{0.000000in}}%
\pgfpathlineto{\pgfqpoint{0.048611in}{0.000000in}}%
\pgfusepath{stroke,fill}%
}%
\begin{pgfscope}%
\pgfsys@transformshift{8.575625in}{2.663333in}%
\pgfsys@useobject{currentmarker}{}%
\end{pgfscope}%
\end{pgfscope}%
\begin{pgfscope}%
\pgftext[x=8.672847in,y=2.615506in,left,base]{\rmfamily\fontsize{10.000000}{12.000000}\selectfont \(\displaystyle 0\)}%
\end{pgfscope}%
\begin{pgfscope}%
\pgfsetbuttcap%
\pgfsetroundjoin%
\definecolor{currentfill}{rgb}{0.000000,0.000000,0.000000}%
\pgfsetfillcolor{currentfill}%
\pgfsetlinewidth{0.803000pt}%
\definecolor{currentstroke}{rgb}{0.000000,0.000000,0.000000}%
\pgfsetstrokecolor{currentstroke}%
\pgfsetdash{}{0pt}%
\pgfsys@defobject{currentmarker}{\pgfqpoint{0.000000in}{0.000000in}}{\pgfqpoint{0.048611in}{0.000000in}}{%
\pgfpathmoveto{\pgfqpoint{0.000000in}{0.000000in}}%
\pgfpathlineto{\pgfqpoint{0.048611in}{0.000000in}}%
\pgfusepath{stroke,fill}%
}%
\begin{pgfscope}%
\pgfsys@transformshift{8.575625in}{3.096667in}%
\pgfsys@useobject{currentmarker}{}%
\end{pgfscope}%
\end{pgfscope}%
\begin{pgfscope}%
\pgftext[x=8.672847in,y=3.048839in,left,base]{\rmfamily\fontsize{10.000000}{12.000000}\selectfont \(\displaystyle 1\)}%
\end{pgfscope}%
\begin{pgfscope}%
\pgfsetbuttcap%
\pgfsetroundjoin%
\definecolor{currentfill}{rgb}{0.000000,0.000000,0.000000}%
\pgfsetfillcolor{currentfill}%
\pgfsetlinewidth{0.803000pt}%
\definecolor{currentstroke}{rgb}{0.000000,0.000000,0.000000}%
\pgfsetstrokecolor{currentstroke}%
\pgfsetdash{}{0pt}%
\pgfsys@defobject{currentmarker}{\pgfqpoint{0.000000in}{0.000000in}}{\pgfqpoint{0.048611in}{0.000000in}}{%
\pgfpathmoveto{\pgfqpoint{0.000000in}{0.000000in}}%
\pgfpathlineto{\pgfqpoint{0.048611in}{0.000000in}}%
\pgfusepath{stroke,fill}%
}%
\begin{pgfscope}%
\pgfsys@transformshift{8.575625in}{3.530000in}%
\pgfsys@useobject{currentmarker}{}%
\end{pgfscope}%
\end{pgfscope}%
\begin{pgfscope}%
\pgftext[x=8.672847in,y=3.482172in,left,base]{\rmfamily\fontsize{10.000000}{12.000000}\selectfont \(\displaystyle 2\)}%
\end{pgfscope}%
\begin{pgfscope}%
\pgfsetbuttcap%
\pgfsetroundjoin%
\definecolor{currentfill}{rgb}{0.000000,0.000000,0.000000}%
\pgfsetfillcolor{currentfill}%
\pgfsetlinewidth{0.803000pt}%
\definecolor{currentstroke}{rgb}{0.000000,0.000000,0.000000}%
\pgfsetstrokecolor{currentstroke}%
\pgfsetdash{}{0pt}%
\pgfsys@defobject{currentmarker}{\pgfqpoint{0.000000in}{0.000000in}}{\pgfqpoint{0.048611in}{0.000000in}}{%
\pgfpathmoveto{\pgfqpoint{0.000000in}{0.000000in}}%
\pgfpathlineto{\pgfqpoint{0.048611in}{0.000000in}}%
\pgfusepath{stroke,fill}%
}%
\begin{pgfscope}%
\pgfsys@transformshift{8.575625in}{3.963333in}%
\pgfsys@useobject{currentmarker}{}%
\end{pgfscope}%
\end{pgfscope}%
\begin{pgfscope}%
\pgftext[x=8.672847in,y=3.915506in,left,base]{\rmfamily\fontsize{10.000000}{12.000000}\selectfont \(\displaystyle 3\)}%
\end{pgfscope}%
\begin{pgfscope}%
\pgfsetbuttcap%
\pgfsetroundjoin%
\definecolor{currentfill}{rgb}{0.000000,0.000000,0.000000}%
\pgfsetfillcolor{currentfill}%
\pgfsetlinewidth{0.803000pt}%
\definecolor{currentstroke}{rgb}{0.000000,0.000000,0.000000}%
\pgfsetstrokecolor{currentstroke}%
\pgfsetdash{}{0pt}%
\pgfsys@defobject{currentmarker}{\pgfqpoint{0.000000in}{0.000000in}}{\pgfqpoint{0.048611in}{0.000000in}}{%
\pgfpathmoveto{\pgfqpoint{0.000000in}{0.000000in}}%
\pgfpathlineto{\pgfqpoint{0.048611in}{0.000000in}}%
\pgfusepath{stroke,fill}%
}%
\begin{pgfscope}%
\pgfsys@transformshift{8.575625in}{4.396667in}%
\pgfsys@useobject{currentmarker}{}%
\end{pgfscope}%
\end{pgfscope}%
\begin{pgfscope}%
\pgftext[x=8.672847in,y=4.348839in,left,base]{\rmfamily\fontsize{10.000000}{12.000000}\selectfont \(\displaystyle 4\)}%
\end{pgfscope}%
\begin{pgfscope}%
\pgfsetbuttcap%
\pgfsetmiterjoin%
\pgfsetlinewidth{0.501875pt}%
\definecolor{currentstroke}{rgb}{0.000000,0.000000,0.000000}%
\pgfsetstrokecolor{currentstroke}%
\pgfsetdash{}{0pt}%
\pgfpathmoveto{\pgfqpoint{8.402292in}{0.930000in}}%
\pgfpathlineto{\pgfqpoint{8.402292in}{0.943542in}}%
\pgfpathlineto{\pgfqpoint{8.402292in}{4.396667in}}%
\pgfpathlineto{\pgfqpoint{8.488958in}{4.570000in}}%
\pgfpathlineto{\pgfqpoint{8.488958in}{4.570000in}}%
\pgfpathlineto{\pgfqpoint{8.575625in}{4.396667in}}%
\pgfpathlineto{\pgfqpoint{8.575625in}{0.943542in}}%
\pgfpathlineto{\pgfqpoint{8.575625in}{0.930000in}}%
\pgfpathclose%
\pgfusepath{stroke}%
\end{pgfscope}%
\end{pgfpicture}%
\makeatother%
\endgroup%
} %
        \caption{Plot von $\hat\Delta_m^{\circledast 2}$ und $\hat\Delta_m$. Wieder liegt der Träger von $\hat\Delta_m^{\circledast 2}$ in der kausalen Zukunft.
        }
        \label{fig:delta_2m_twisted}
    \end{minipage}\hfill
    \begin{minipage}{0.45\textwidth}
        \centering
        \resizebox{\textwidth}{!}{%% Creator: Matplotlib, PGF backend
%%
%% To include the figure in your LaTeX document, write
%%   \input{<filename>.pgf}
%%
%% Make sure the required packages are loaded in your preamble
%%   \usepackage{pgf}
%%
%% Figures using additional raster images can only be included by \input if
%% they are in the same directory as the main LaTeX file. For loading figures
%% from other directories you can use the `import` package
%%   \usepackage{import}
%% and then include the figures with
%%   \import{<path to file>}{<filename>.pgf}
%%
%% Matplotlib used the following preamble
%%   \usepackage[utf8x]{inputenc}
%%   \usepackage[T1]{fontenc}
%%   \usepackage{amssymb}
%%
\begingroup%
\makeatletter%
\begin{pgfpicture}%
\pgfpathrectangle{\pgfpointorigin}{\pgfqpoint{4.000000in}{2.200000in}}%
\pgfusepath{use as bounding box, clip}%
\begin{pgfscope}%
\pgfsetbuttcap%
\pgfsetmiterjoin%
\definecolor{currentfill}{rgb}{1.000000,1.000000,1.000000}%
\pgfsetfillcolor{currentfill}%
\pgfsetlinewidth{0.000000pt}%
\definecolor{currentstroke}{rgb}{1.000000,1.000000,1.000000}%
\pgfsetstrokecolor{currentstroke}%
\pgfsetdash{}{0pt}%
\pgfpathmoveto{\pgfqpoint{0.000000in}{0.000000in}}%
\pgfpathlineto{\pgfqpoint{4.000000in}{0.000000in}}%
\pgfpathlineto{\pgfqpoint{4.000000in}{2.200000in}}%
\pgfpathlineto{\pgfqpoint{0.000000in}{2.200000in}}%
\pgfpathclose%
\pgfusepath{fill}%
\end{pgfscope}%
\begin{pgfscope}%
\pgfsetbuttcap%
\pgfsetmiterjoin%
\definecolor{currentfill}{rgb}{1.000000,1.000000,1.000000}%
\pgfsetfillcolor{currentfill}%
\pgfsetlinewidth{0.000000pt}%
\definecolor{currentstroke}{rgb}{0.000000,0.000000,0.000000}%
\pgfsetstrokecolor{currentstroke}%
\pgfsetstrokeopacity{0.000000}%
\pgfsetdash{}{0pt}%
\pgfpathmoveto{\pgfqpoint{0.198611in}{0.198611in}}%
\pgfpathlineto{\pgfqpoint{3.801389in}{0.198611in}}%
\pgfpathlineto{\pgfqpoint{3.801389in}{2.001389in}}%
\pgfpathlineto{\pgfqpoint{0.198611in}{2.001389in}}%
\pgfpathclose%
\pgfusepath{fill}%
\end{pgfscope}%
\begin{pgfscope}%
\pgfpathrectangle{\pgfqpoint{0.198611in}{0.198611in}}{\pgfqpoint{3.602778in}{1.802778in}} %
\pgfusepath{clip}%
\pgfsetbuttcap%
\pgfsetroundjoin%
\pgfsetlinewidth{0.501875pt}%
\definecolor{currentstroke}{rgb}{0.501961,0.501961,0.501961}%
\pgfsetstrokecolor{currentstroke}%
\pgfsetdash{{1.850000pt}{0.800000pt}}{0.000000pt}%
\pgfpathmoveto{\pgfqpoint{0.507421in}{0.198611in}}%
\pgfpathlineto{\pgfqpoint{0.507421in}{2.001389in}}%
\pgfusepath{stroke}%
\end{pgfscope}%
\begin{pgfscope}%
\pgfpathrectangle{\pgfqpoint{0.198611in}{0.198611in}}{\pgfqpoint{3.602778in}{1.802778in}} %
\pgfusepath{clip}%
\pgfsetrectcap%
\pgfsetroundjoin%
\pgfsetlinewidth{1.003750pt}%
\definecolor{currentstroke}{rgb}{0.894118,0.101961,0.109804}%
\pgfsetstrokecolor{currentstroke}%
\pgfsetdash{}{0pt}%
\pgfpathmoveto{\pgfqpoint{0.512092in}{2.015278in}}%
\pgfpathlineto{\pgfqpoint{0.512367in}{1.888581in}}%
\pgfpathlineto{\pgfqpoint{0.516488in}{1.548954in}}%
\pgfpathlineto{\pgfqpoint{0.520610in}{1.386168in}}%
\pgfpathlineto{\pgfqpoint{0.528853in}{1.217162in}}%
\pgfpathlineto{\pgfqpoint{0.537096in}{1.125446in}}%
\pgfpathlineto{\pgfqpoint{0.545339in}{1.065712in}}%
\pgfpathlineto{\pgfqpoint{0.557704in}{1.005174in}}%
\pgfpathlineto{\pgfqpoint{0.570069in}{0.962974in}}%
\pgfpathlineto{\pgfqpoint{0.582433in}{0.930877in}}%
\pgfpathlineto{\pgfqpoint{0.598920in}{0.897252in}}%
\pgfpathlineto{\pgfqpoint{0.619528in}{0.863755in}}%
\pgfpathlineto{\pgfqpoint{0.644257in}{0.830408in}}%
\pgfpathlineto{\pgfqpoint{0.677230in}{0.791728in}}%
\pgfpathlineto{\pgfqpoint{0.739054in}{0.725344in}}%
\pgfpathlineto{\pgfqpoint{0.837972in}{0.618605in}}%
\pgfpathlineto{\pgfqpoint{1.044051in}{0.391956in}}%
\pgfpathlineto{\pgfqpoint{1.085266in}{0.353376in}}%
\pgfpathlineto{\pgfqpoint{1.118239in}{0.326322in}}%
\pgfpathlineto{\pgfqpoint{1.147090in}{0.306084in}}%
\pgfpathlineto{\pgfqpoint{1.175941in}{0.289580in}}%
\pgfpathlineto{\pgfqpoint{1.200671in}{0.278774in}}%
\pgfpathlineto{\pgfqpoint{1.225400in}{0.271346in}}%
\pgfpathlineto{\pgfqpoint{1.250130in}{0.267550in}}%
\pgfpathlineto{\pgfqpoint{1.270738in}{0.267317in}}%
\pgfpathlineto{\pgfqpoint{1.291346in}{0.269857in}}%
\pgfpathlineto{\pgfqpoint{1.311953in}{0.275245in}}%
\pgfpathlineto{\pgfqpoint{1.332561in}{0.283528in}}%
\pgfpathlineto{\pgfqpoint{1.353169in}{0.294724in}}%
\pgfpathlineto{\pgfqpoint{1.377899in}{0.311977in}}%
\pgfpathlineto{\pgfqpoint{1.402628in}{0.333297in}}%
\pgfpathlineto{\pgfqpoint{1.427358in}{0.358504in}}%
\pgfpathlineto{\pgfqpoint{1.456209in}{0.392466in}}%
\pgfpathlineto{\pgfqpoint{1.489181in}{0.436571in}}%
\pgfpathlineto{\pgfqpoint{1.526276in}{0.491590in}}%
\pgfpathlineto{\pgfqpoint{1.579856in}{0.577165in}}%
\pgfpathlineto{\pgfqpoint{1.649923in}{0.688630in}}%
\pgfpathlineto{\pgfqpoint{1.687017in}{0.741589in}}%
\pgfpathlineto{\pgfqpoint{1.715868in}{0.777427in}}%
\pgfpathlineto{\pgfqpoint{1.740598in}{0.803323in}}%
\pgfpathlineto{\pgfqpoint{1.761206in}{0.820916in}}%
\pgfpathlineto{\pgfqpoint{1.781814in}{0.834460in}}%
\pgfpathlineto{\pgfqpoint{1.798300in}{0.842148in}}%
\pgfpathlineto{\pgfqpoint{1.814786in}{0.846876in}}%
\pgfpathlineto{\pgfqpoint{1.831273in}{0.848522in}}%
\pgfpathlineto{\pgfqpoint{1.847759in}{0.846999in}}%
\pgfpathlineto{\pgfqpoint{1.864245in}{0.842253in}}%
\pgfpathlineto{\pgfqpoint{1.880732in}{0.834264in}}%
\pgfpathlineto{\pgfqpoint{1.897218in}{0.823053in}}%
\pgfpathlineto{\pgfqpoint{1.917826in}{0.804607in}}%
\pgfpathlineto{\pgfqpoint{1.938434in}{0.781437in}}%
\pgfpathlineto{\pgfqpoint{1.959042in}{0.753857in}}%
\pgfpathlineto{\pgfqpoint{1.983771in}{0.715536in}}%
\pgfpathlineto{\pgfqpoint{2.012622in}{0.664800in}}%
\pgfpathlineto{\pgfqpoint{2.049717in}{0.592828in}}%
\pgfpathlineto{\pgfqpoint{2.140391in}{0.413262in}}%
\pgfpathlineto{\pgfqpoint{2.169242in}{0.364361in}}%
\pgfpathlineto{\pgfqpoint{2.193972in}{0.328891in}}%
\pgfpathlineto{\pgfqpoint{2.214580in}{0.304945in}}%
\pgfpathlineto{\pgfqpoint{2.231066in}{0.289973in}}%
\pgfpathlineto{\pgfqpoint{2.247553in}{0.279056in}}%
\pgfpathlineto{\pgfqpoint{2.264039in}{0.272442in}}%
\pgfpathlineto{\pgfqpoint{2.276404in}{0.270421in}}%
\pgfpathlineto{\pgfqpoint{2.288768in}{0.270983in}}%
\pgfpathlineto{\pgfqpoint{2.301133in}{0.274159in}}%
\pgfpathlineto{\pgfqpoint{2.313498in}{0.279957in}}%
\pgfpathlineto{\pgfqpoint{2.329984in}{0.291732in}}%
\pgfpathlineto{\pgfqpoint{2.346470in}{0.308025in}}%
\pgfpathlineto{\pgfqpoint{2.362957in}{0.328643in}}%
\pgfpathlineto{\pgfqpoint{2.383565in}{0.360074in}}%
\pgfpathlineto{\pgfqpoint{2.404173in}{0.397119in}}%
\pgfpathlineto{\pgfqpoint{2.428902in}{0.447714in}}%
\pgfpathlineto{\pgfqpoint{2.461875in}{0.522473in}}%
\pgfpathlineto{\pgfqpoint{2.540185in}{0.703588in}}%
\pgfpathlineto{\pgfqpoint{2.564914in}{0.752398in}}%
\pgfpathlineto{\pgfqpoint{2.585522in}{0.786944in}}%
\pgfpathlineto{\pgfqpoint{2.602009in}{0.809694in}}%
\pgfpathlineto{\pgfqpoint{2.618495in}{0.827512in}}%
\pgfpathlineto{\pgfqpoint{2.630860in}{0.837367in}}%
\pgfpathlineto{\pgfqpoint{2.643224in}{0.844041in}}%
\pgfpathlineto{\pgfqpoint{2.655589in}{0.847416in}}%
\pgfpathlineto{\pgfqpoint{2.667954in}{0.847411in}}%
\pgfpathlineto{\pgfqpoint{2.680319in}{0.843984in}}%
\pgfpathlineto{\pgfqpoint{2.692683in}{0.837133in}}%
\pgfpathlineto{\pgfqpoint{2.705048in}{0.826900in}}%
\pgfpathlineto{\pgfqpoint{2.721534in}{0.808145in}}%
\pgfpathlineto{\pgfqpoint{2.738021in}{0.783846in}}%
\pgfpathlineto{\pgfqpoint{2.754507in}{0.754452in}}%
\pgfpathlineto{\pgfqpoint{2.775115in}{0.711435in}}%
\pgfpathlineto{\pgfqpoint{2.799844in}{0.652477in}}%
\pgfpathlineto{\pgfqpoint{2.841060in}{0.544076in}}%
\pgfpathlineto{\pgfqpoint{2.886398in}{0.426470in}}%
\pgfpathlineto{\pgfqpoint{2.911127in}{0.370530in}}%
\pgfpathlineto{\pgfqpoint{2.931735in}{0.331451in}}%
\pgfpathlineto{\pgfqpoint{2.948221in}{0.306336in}}%
\pgfpathlineto{\pgfqpoint{2.964708in}{0.287474in}}%
\pgfpathlineto{\pgfqpoint{2.977073in}{0.277785in}}%
\pgfpathlineto{\pgfqpoint{2.989437in}{0.272123in}}%
\pgfpathlineto{\pgfqpoint{3.001802in}{0.270619in}}%
\pgfpathlineto{\pgfqpoint{3.014167in}{0.273340in}}%
\pgfpathlineto{\pgfqpoint{3.026531in}{0.280293in}}%
\pgfpathlineto{\pgfqpoint{3.038896in}{0.291421in}}%
\pgfpathlineto{\pgfqpoint{3.051261in}{0.306604in}}%
\pgfpathlineto{\pgfqpoint{3.067747in}{0.332826in}}%
\pgfpathlineto{\pgfqpoint{3.084234in}{0.365307in}}%
\pgfpathlineto{\pgfqpoint{3.104842in}{0.413477in}}%
\pgfpathlineto{\pgfqpoint{3.129571in}{0.479776in}}%
\pgfpathlineto{\pgfqpoint{3.224367in}{0.744676in}}%
\pgfpathlineto{\pgfqpoint{3.244975in}{0.787714in}}%
\pgfpathlineto{\pgfqpoint{3.261462in}{0.814705in}}%
\pgfpathlineto{\pgfqpoint{3.273826in}{0.829994in}}%
\pgfpathlineto{\pgfqpoint{3.286191in}{0.840690in}}%
\pgfpathlineto{\pgfqpoint{3.298556in}{0.846559in}}%
\pgfpathlineto{\pgfqpoint{3.306799in}{0.847712in}}%
\pgfpathlineto{\pgfqpoint{3.315042in}{0.846627in}}%
\pgfpathlineto{\pgfqpoint{3.323285in}{0.843296in}}%
\pgfpathlineto{\pgfqpoint{3.335650in}{0.834120in}}%
\pgfpathlineto{\pgfqpoint{3.348015in}{0.820032in}}%
\pgfpathlineto{\pgfqpoint{3.360380in}{0.801236in}}%
\pgfpathlineto{\pgfqpoint{3.376866in}{0.769366in}}%
\pgfpathlineto{\pgfqpoint{3.393352in}{0.730577in}}%
\pgfpathlineto{\pgfqpoint{3.413960in}{0.674169in}}%
\pgfpathlineto{\pgfqpoint{3.442811in}{0.585459in}}%
\pgfpathlineto{\pgfqpoint{3.496392in}{0.419034in}}%
\pgfpathlineto{\pgfqpoint{3.517000in}{0.364713in}}%
\pgfpathlineto{\pgfqpoint{3.533486in}{0.328507in}}%
\pgfpathlineto{\pgfqpoint{3.549972in}{0.300226in}}%
\pgfpathlineto{\pgfqpoint{3.562337in}{0.284891in}}%
\pgfpathlineto{\pgfqpoint{3.574702in}{0.274996in}}%
\pgfpathlineto{\pgfqpoint{3.582945in}{0.271550in}}%
\pgfpathlineto{\pgfqpoint{3.591188in}{0.270684in}}%
\pgfpathlineto{\pgfqpoint{3.599431in}{0.272420in}}%
\pgfpathlineto{\pgfqpoint{3.607675in}{0.276761in}}%
\pgfpathlineto{\pgfqpoint{3.620039in}{0.288098in}}%
\pgfpathlineto{\pgfqpoint{3.632404in}{0.305056in}}%
\pgfpathlineto{\pgfqpoint{3.644769in}{0.327337in}}%
\pgfpathlineto{\pgfqpoint{3.661255in}{0.364590in}}%
\pgfpathlineto{\pgfqpoint{3.681863in}{0.421417in}}%
\pgfpathlineto{\pgfqpoint{3.706593in}{0.500380in}}%
\pgfpathlineto{\pgfqpoint{3.772538in}{0.718180in}}%
\pgfpathlineto{\pgfqpoint{3.793146in}{0.772633in}}%
\pgfpathlineto{\pgfqpoint{3.801389in}{0.790946in}}%
\pgfpathlineto{\pgfqpoint{3.801389in}{0.790946in}}%
\pgfusepath{stroke}%
\end{pgfscope}%
\begin{pgfscope}%
\pgfpathrectangle{\pgfqpoint{0.198611in}{0.198611in}}{\pgfqpoint{3.602778in}{1.802778in}} %
\pgfusepath{clip}%
\pgfsetrectcap%
\pgfsetroundjoin%
\pgfsetlinewidth{1.003750pt}%
\definecolor{currentstroke}{rgb}{0.894118,0.101961,0.109804}%
\pgfsetstrokecolor{currentstroke}%
\pgfsetdash{}{0pt}%
\pgfpathmoveto{\pgfqpoint{0.184722in}{0.559167in}}%
\pgfpathlineto{\pgfqpoint{0.324422in}{0.559167in}}%
\pgfpathlineto{\pgfqpoint{0.507421in}{0.559167in}}%
\pgfusepath{stroke}%
\end{pgfscope}%
\begin{pgfscope}%
\pgfpathrectangle{\pgfqpoint{0.198611in}{0.198611in}}{\pgfqpoint{3.602778in}{1.802778in}} %
\pgfusepath{clip}%
\pgfsetbuttcap%
\pgfsetroundjoin%
\pgfsetlinewidth{0.501875pt}%
\definecolor{currentstroke}{rgb}{0.501961,0.501961,0.501961}%
\pgfsetstrokecolor{currentstroke}%
\pgfsetdash{{1.850000pt}{0.800000pt}}{0.000000pt}%
\pgfpathmoveto{\pgfqpoint{0.184722in}{0.847611in}}%
\pgfpathlineto{\pgfqpoint{3.815278in}{0.847611in}}%
\pgfusepath{stroke}%
\end{pgfscope}%
\begin{pgfscope}%
\pgfsetrectcap%
\pgfsetmiterjoin%
\pgfsetlinewidth{0.501875pt}%
\definecolor{currentstroke}{rgb}{0.000000,0.000000,0.000000}%
\pgfsetstrokecolor{currentstroke}%
\pgfsetdash{}{0pt}%
\pgfpathmoveto{\pgfqpoint{0.301548in}{0.198611in}}%
\pgfpathlineto{\pgfqpoint{0.301548in}{2.001389in}}%
\pgfusepath{stroke}%
\end{pgfscope}%
\begin{pgfscope}%
\pgfsetrectcap%
\pgfsetmiterjoin%
\pgfsetlinewidth{0.501875pt}%
\definecolor{currentstroke}{rgb}{0.000000,0.000000,0.000000}%
\pgfsetstrokecolor{currentstroke}%
\pgfsetdash{}{0pt}%
\pgfpathmoveto{\pgfqpoint{0.198611in}{0.559167in}}%
\pgfpathlineto{\pgfqpoint{3.801389in}{0.559167in}}%
\pgfusepath{stroke}%
\end{pgfscope}%
\begin{pgfscope}%
\pgftext[x=0.548595in,y=0.342833in,left,base]{\rmfamily\fontsize{10.000000}{12.000000}\selectfont \(\displaystyle \omega = 2 m\)}%
\end{pgfscope}%
\begin{pgfscope}%
\pgftext[x=0.548595in,y=1.136056in,left,base]{\rmfamily\fontsize{10.000000}{12.000000}\selectfont \(\displaystyle \approx \frac{1}{\sqrt{\omega}}\)}%
\end{pgfscope}%
\begin{pgfscope}%
\pgftext[x=2.566151in,y=0.876456in,left,base]{\rmfamily\fontsize{10.000000}{12.000000}\selectfont \(\displaystyle \approx 2 \cos\left(\frac{\omega^2}{2}\right)\)}%
\end{pgfscope}%
\begin{pgfscope}%
\pgfsetroundcap%
\pgfsetroundjoin%
\pgfsetlinewidth{0.501875pt}%
\definecolor{currentstroke}{rgb}{0.000000,0.000000,0.000000}%
\pgfsetstrokecolor{currentstroke}%
\pgfsetdash{}{0pt}%
\pgfpathmoveto{\pgfqpoint{0.301548in}{2.007506in}}%
\pgfpathquadraticcurveto{\pgfqpoint{0.301548in}{2.008330in}}{\pgfqpoint{0.301548in}{2.001389in}}%
\pgfusepath{stroke}%
\end{pgfscope}%
\begin{pgfscope}%
\pgfsetroundcap%
\pgfsetroundjoin%
\pgfsetlinewidth{0.501875pt}%
\definecolor{currentstroke}{rgb}{0.000000,0.000000,0.000000}%
\pgfsetstrokecolor{currentstroke}%
\pgfsetdash{}{0pt}%
\pgfpathmoveto{\pgfqpoint{0.273770in}{1.951951in}}%
\pgfpathlineto{\pgfqpoint{0.301548in}{2.007506in}}%
\pgfpathlineto{\pgfqpoint{0.329325in}{1.951951in}}%
\pgfusepath{stroke}%
\end{pgfscope}%
\begin{pgfscope}%
\pgftext[x=0.301548in,y=2.070833in,,bottom]{\rmfamily\fontsize{10.000000}{12.000000}\selectfont \(\displaystyle \hat\Delta^{\circledast 2} ~(-\omega, 0)\)}%
\end{pgfscope}%
\begin{pgfscope}%
\pgfsetroundcap%
\pgfsetroundjoin%
\pgfsetlinewidth{0.501875pt}%
\definecolor{currentstroke}{rgb}{0.000000,0.000000,0.000000}%
\pgfsetstrokecolor{currentstroke}%
\pgfsetdash{}{0pt}%
\pgfpathmoveto{\pgfqpoint{3.807500in}{0.559167in}}%
\pgfpathquadraticcurveto{\pgfqpoint{3.808327in}{0.559167in}}{\pgfqpoint{3.801389in}{0.559167in}}%
\pgfusepath{stroke}%
\end{pgfscope}%
\begin{pgfscope}%
\pgfsetroundcap%
\pgfsetroundjoin%
\pgfsetlinewidth{0.501875pt}%
\definecolor{currentstroke}{rgb}{0.000000,0.000000,0.000000}%
\pgfsetstrokecolor{currentstroke}%
\pgfsetdash{}{0pt}%
\pgfpathmoveto{\pgfqpoint{3.751945in}{0.586944in}}%
\pgfpathlineto{\pgfqpoint{3.807500in}{0.559167in}}%
\pgfpathlineto{\pgfqpoint{3.751945in}{0.531389in}}%
\pgfusepath{stroke}%
\end{pgfscope}%
\begin{pgfscope}%
\pgftext[x=3.870833in,y=0.559167in,left,]{\rmfamily\fontsize{10.000000}{12.000000}\selectfont \(\displaystyle \omega\)}%
\end{pgfscope}%
\end{pgfpicture}%
\makeatother%
\endgroup%
}
        \caption{Plot von $\left.\hat{\Delta}_m^{\circledast 2}\right|_{k=0}$ um das asymptotische Verhalten für $\omega \rightarrow 0$ und $\omega \rightarrow \infty$ zu verdeutlichen}
        \label{fig:delta_2m_twisted_k0}
    \end{minipage}
\end{figure}

\subsubsection*{Fall $|s| > 1$}
Wir bedienen uns wieder genau des selben Arguments, wie in \cref{eq:delta_m2_s>1} und dürfen direkt schreiben:

\begin{equation}
    \left\langle \rwhat{\Delta}_m^{\circledast 2}, \hat\psi_{ast}\right\rangle
    = 0 \condition{für alle $a$ klein genug}
\label{eq:delta_m2_twisted_s>1}
\end{equation}


\subsubsection*{Fall $|s| < 1, (x,t) \neq 0$}
Da
$\rwhat\Delta_m^{\circledast 2} = \rwhat\Delta_m^{* 2} \cos(\dots)$ können wir direkt mit dem Ausdruck (\ref{eq:psi_ast_delta_m2_s<1}) $\cdot \cos$ weiter arbeiten und genau die selben Abschätzungen machen. $\cos(\phi)$ ist bekanntermaßen beschränkt.

\begin{dmath}
\label{eq:delta_m2_twisted_s<1}
    \left\langle \rwhat{\Delta}_m^{\circledast 2}, \rwhat{\psi}_{ast}
    \right\rangle
    =
     2 a^{-\frac{3}{4}} \int \frac{
    \hat\psi_1(\omega)~ \hat\psi_2(k) \left(
    \omega^2 \left(\Delta s - 2 a^{\frac{1}{2}} k s - ak^2
            \right) - 3a^2m^2
    \right)
     }
     {
        \sqrt{\Delta s -2a^{\frac{1}{2}}ks - ak^2}
            \sqrt{\Delta s \omega^2 -2a^{\frac{1}{2}} \omega^2 k s
                    - a\omega^2k^2-4 a^2 m^2}
     }
     \cdot
     \Theta(\cdots)
     \cos(\varphi(\omega^2-k^2))
     e^{-i \omega \left(\frac{t'-sx'}{a}+k \frac{x'}{\sqrt{a}}\right)}
     \d \omega \d k
     \leq
     2 a^{-\frac{3}{4}} \int
     \omega \hat\psi_1(\omega)\, \hat\psi_2(k)
     e^{-i \omega \left(\frac{t'-sx'}{a}+k \frac{x'}{\sqrt{a}}\right)}
     \d \omega \d k
     \sim O(a^k) ~~ \forall k \hiderel \in \mathbb{N}
\label{eq:delta_m2_twisted_s<1_x_neq_0}
\end{dmath}


\subsubsection*{Fall $|s| < 1, (x,t) = 0$}
In diesem Fall lassen wir den $\cos$-Faktor in \cref{eq:delta_m2_twisted_s<1} in der ersten Ungleichung nicht heraus fallen, dafür wird der $e^\cdots$-Faktor 1. Den $\cos$-Faktor schreiben wir als Summe von $e$-Funktionen und erhalten

% \begin{equation}
% \begin{aligned}
%     \left\langle \rwhat{\Delta}_m^{\circledast 2}, \rwhat{\psi}_{ast}
%     \right\rangle
%     \\&=
%     2 a^{-\frac{3}{4}} \int
%     \omega \hat\psi_1(\omega) \hat\psi_2(k)
%     \left\{
%         \exp\left(i a^{-2} \frac{
%         \omega^2 (\Delta s - 2 a^{\frac{1}{2}} k s - a k^2)
%         }{2}
%         \sqrt{\frac{1}{4} - \frac{a^2 m^2}{\omega^2(
%             \Delta s - 2 a^{\frac{1}{2}} k s - ak^2
%         )}}
%         \right)
%         \\ &+
%         \exp (-i \cdots)
%     \right\}
%     \d \omega \d k
%     \\ &=
%     2 a^{-\frac{3}{4}} \int
%     \cancel{\sqrt{\omega}} \hat\psi_1(\sqrt{\omega}) \hat\psi_2(k)
%     \left\{
%         \exp\left(i a^{-2} \frac{
%         \omega (\Delta s - 2 a^{\frac{1}{2}} k s - a k^2)
%         }{2}
%         \sqrt{\frac{1}{4} - \frac{a^2 m^2}{\omega(
%             \Delta s - 2 a^{\frac{1}{2}} k s - ak^2
%         )}}
%         \right)
%         \\ &+ \mathrm{c.c.}
%     \right\}
%     \frac{{\d \omega \d k}}{\cancel{\sqrt{\omega}}}
%     \\ &=
%     2 a^{-\frac{3}{4}} \int \left\{
%         \int
%         \hat\psi_1(\sqrt\omega)
%         \left\{
%             \exp
%             \left(ia^{-2} \left(\frac{\omega \Delta s}{4}
%                                 + O\left(a^{\frac{1}{2}}\right)\right)
%             \right)
%             \\ &+
%             \mathrm{c.c.}
%         \right\}
%         \d \omega
%     \right\}
%     \hat \psi_2(k) \d k
%     \\ &=
%     2 a^{-\frac{3}{4}} \int
%     \underbrace{
%     \left\{
%     (\hat\psi_1 \circ \sqrt{\cdot })^\vee
%     \left(\frac{\Delta s}{4a^2}\right)
%      + (\hat\psi_1 \circ \sqrt{\cdot })^\vee
%     \left(-\frac{\Delta s}{4a^2}\right)
%     + \mathrm{c.c.}
%     \right\}}_{
%     \sim O(a^k) ~\forall k \in \mathbb{N}
%     }
%     \psi_2(k) \d k
%     \\ &
%     \sim O(a^k) ~~ \forall k \hiderel \in \mathbb{N}
% \label{eq:delta_m2_twisted_s<1_x=0}
% \end{aligned}
% \end{equation}
\begin{dmath}
    \left\langle \rwhat{\Delta}_m^{\circledast 2}, \rwhat{\psi}_{ast}
    \right\rangle
    =
    2 a^{-\frac{3}{4}} \int
    \omega \hat\psi_1(\omega) \hat\psi_2(k)
    \left\{
        \exp\left(i a^{-2} \frac{
        \omega^2 (\Delta s - 2 a^{\frac{1}{2}} k s - a k^2)
        }{2}
        \sqrt{\frac{1}{4} - \frac{a^2 m^2}{\omega^2(
            \Delta s - 2 a^{\frac{1}{2}} k s - ak^2
        )}}
        \right)
        +\exp (-i \cdots)
    \right\}
    \d \omega \d k
    =
    2 a^{-\frac{3}{4}} \int
    \cancel{\sqrt{\omega}} \hat\psi_1(\sqrt{\omega}) \hat\psi_2(k)
    \left\{
        \exp\left(i a^{-2} \frac{
        \omega (\Delta s - 2 a^{\frac{1}{2}} k s - a k^2)
        }{2}
        \sqrt{\frac{1}{4} - \frac{a^2 m^2}{\omega(
            \Delta s - 2 a^{\frac{1}{2}} k s - ak^2
        )}}
        \right)
        + \mathrm{c.c.}
    \right\}
    \frac{{\d \omega \d k}}{\cancel{\sqrt{\omega}}}
    =
    2 a^{-\frac{3}{4}} \int \left\{
        \int
        \hat\psi_1(\sqrt\omega)
        \left\{
            \exp
            \left(ia^{-2} \left(\frac{\omega \Delta s}{4}
                                + O\left(a^{\frac{1}{2}}\right)\right)
            \right)
            + \mathrm{c.c.}
        \right\}
        \d \omega
    \right\}
    \hat \psi_2(k) \d k
    =
    2 a^{-\frac{3}{4}} \int
    \underbrace{
    \left\{
    (\hat\psi_1 \circ \sqrt{\cdot })^\vee
    \left(\frac{\Delta s}{4a^2}\right)
     + (\hat\psi_1 \circ \sqrt{\cdot })^\vee
    \left(-\frac{\Delta s}{4a^2}\right)
    + \mathrm{c.c.}
    \right\}}_{
    \sim O(a^k) ~\forall k \in \mathbb{N}
    }
    \psi_2(k) \d k
    \sim O(a^k) ~~ \forall k \hiderel \in \mathbb{N}
\label{eq:delta_m2_twisted_s<1_x=0}
\end{dmath}

\todo[color=red]{Schritt von dritte in die vierte Zeile rechtfertigen. Ist wieder die Geschichte vom "`halben Lebesgue"'}

wobei bei der Substition $\omega \to \sqrt{\omega}$ in der zweiten Zeile wichtig ist, dass $0 \notin supp (\hat\psi_1)$, also auch nach der Substitution noch $\hat\psi_1 \in C_c^\infty (\mathbb{R})$ ist.


\subsubsection*{Fall $s = -1$}

Da $\rwhat\Delta_m^{\circledast 2} = \rwhat\Delta_m^{* 2} \cos(\dots)$ ist, haben wir bis auf den $\cos$-Faktor die selben Analysis zu betreiben, wie für $\rwhat\Delta_m^{*2}$.

\todo{ordentliche Formulierung und ordentliche Referenzen einfügen}

\begin{dmath}
    \left\langle \rwhat{\Delta}_m^{\circledast 2}, \rwhat{\psi}_{a-1t}
    \right\rangle
    =
    2 a^{-\frac{3}{4}} \int
    \underbrace{\frac{
        \hat\psi_1(\omega) \hat\psi_2(k'+k_0)
        \left(
        2\omega^2(k'+k_0)-a^{\frac{1}{2}}\omega^2(k'+k_0)^2-a^{\frac{3}{2}}3m^2
        \right)
        \Theta(k')
    }
    {
        \sqrt{k'} \sqrt{k'+k_0}
        \sqrt{2-a^{\frac{1}{2}}(k'+k_0)}
        \sqrt{-a^{\frac{1}{2}}\omega^2\left(k'-\tfrac{2\sqrt{\omega^2-4a^2m^2}}
                    {\sqrt a \omega}\right)}
    }}_{i)}
    \cdot
    \cos
    \underbrace{\left(
        \frac{2\omega^2(k'+k_0)-a^{\frac{1}{2}}\omega^2(k'+k_0)^2}
             {2 a^{\frac{3}{2}}}
        \sqrt{
            \frac{1}{4}
            + \frac{a^{\frac{3}{2}} m^2}
                   {2\omega^2(k'+k_0)-a^{\frac{1}{2}}\omega^2(k'+k_0)^2}
        }
    \right)}_{ii)}
    \cdot
    e^{-i\omega\left(\frac{t'+x'}{a}+\frac{(k'+k_0)x'}{\sqrt a}\right)}
    \d \omega \d k'
\label{eq:psi_a-1t_delta_m2_twisted}
\end{dmath}

Genau wie in \cref{eq:lange_1/sqrt_abschaetzerei} können wir für $i)$ wieder Abschätzen\footnote{Da cos beschränkt ist, spielt er bei den Abschätzungen keine Rolle}

\begin{dmath*}
    % \frac{
    %     \hat\psi_1(\omega) \hat\psi_2(k'+k_0)
    %     \left(
    %     2\omega^2(k'+k_0)-a^{\frac{1}{2}}\omega^2(k'+k_0)^2-a^{\frac{3}{2}}3m^2
    %     \right)
    %     \Theta(k')
    % }
    % {
    %     \sqrt{k'} \sqrt{k'+k_0}
    %     \sqrt{2-a^{\frac{1}{2}}(k'+k_0)}
    %     \sqrt{-a^{\frac{1}{2}}\omega^2\left(k'-\tfrac{2\sqrt{\omega^2-4a^2m^2}}
    %                 {\sqrt a \omega}\right)}
    % }
    i)
    \leq
    \frac{\textrm{const}}{\sqrt{k'}} \Theta(k')
\end{dmath*}

Damit dürfen wir wieder Lebesgue anwenden, um den Grenzwert $a \to 0$ des Integrals zu berechnen.
Des weiteren ist analog zu \cref{eq:langer_sqrt_bruch_punktweise_konvergenz}

\begin{dmath}
    i)
    \stackrel{\textrm{\scriptsize punktweise f.ü.}}{\longrightarrow}
    \omega\,\hat\psi_1(\omega) \,\hat\psi_2(k') \Theta(k')
\label{eq:delta_m2_twisted_s=-1_material_a}
\end{dmath}

Widmen wir uns also dem Argument des Kosinus $ii)$:

\begin{dmath*}
    \frac{2\omega^2(k'+k_0)-a^{\frac{1}{2}}\omega^2(k'+k_0)^2}
         {2 a^{\frac{3}{2}}}
    \sqrt{
        \frac{1}{4}
        + \frac{a^{\frac{3}{2}} m^2}
               {2\omega^2(k'+k_0)-a^{\frac{1}{2}}\omega^2(k'+k_0)^2}
    }
    =
    \frac{\omega^2 (k'+k'0)(2-a^{\frac{1}{2}}(k_+k_0))}
         {2 a^{\frac{3}{2}}}
    \sqrt{
        \frac{1}{4}
        + \frac{a^{\frac{3}{2}}m^2}
               {\omega^2(k'+k_0)(a^{\frac{1}{2}}(k'+k_0)-2)}
    }\\
    \stackrel{\textrm{\scriptsize punktweise, außer } k'=0}{\longrightarrow}
    \frac{\omega^2 k' a^{-\frac{3}{2}}}{2}
\end{dmath*}

also

\begin{dmath}
    \cos\big(ii)\big)
    \stackrel{\textrm{\scriptsize punktweise, außer } k'=0}{\longrightarrow}
    \cos\left(\frac{\omega^2 k' a^{-\frac{3}{2}}}{2}\right)
\label{eq:delta_m2_twisted_s=-1_material_b}
\end{dmath}

Einsetzen von \cref{eq:delta_m2_twisted_s=-1_material_a,eq:delta_m2_twisted_s=-1_material_b} in \cref{eq:psi_a-1t_delta_m2_twisted} ergibt mit Lebesgue

\todo{den Ausdruck hübscher machen. So ist das ja grauenhaft}

\begin{dmath*}
    \lim_{a \to 0}
    \left\langle \rwhat{\Delta}_m^{\circledast 2}, \rwhat{\psi}_{a-1t}
    \right\rangle
    =
    2 a^{-\frac{3}{4}}
    \int \omega \,\hat\psi_1(\omega) \,\hat\psi_2(k')
         \cos\left(a^{-\frac{3}{2}}\frac{\omega^2 k'}{2}\right)
         e^{-i\omega k' \frac{x'}{\sqrt a}}
         e^{-i \omega \frac{t'+x'}{a}}
         \d \omega \d k'
    =
    a^{-\frac{3}{4}} \int
    \underbrace{\left\{
            \int \hat\psi_2(k')\Theta(k')
            \left(e^{i a^{-\frac{3}{2}}\frac{\omega^2 k'}{2}} + e^{-i\cdots}\right)
            e^{-i\omega k' \frac{x'}{\sqrt a}}
            \d k'
        \right\}}_{=: \hat f_a(\omega)}
    \cdot
    \omega \hat\psi_1(\omega)e^{-i\omega \frac{t'+x'}{a}} \d \omega
\end{dmath*}

Nun betrachten wir $\hat f_a(\omega)$ und erhalten analog zu
\cref{eq:faltung_mit_1/x_rechnung}

\begin{align*}
    \hat f_a(\omega)
    &=
    \int \hat\psi_2(k')\Theta(k')
    \left(
        e^{i a^{-\frac{3}{2}}\left(\frac{\omega^2 k'}{2} + O(a^1)\right)}
        + e^{-i \cdots}
    \right) \d k
    \\ &\stackrel{a \to 0}{\longrightarrow}
    \int \hat\psi_2(k')\Theta(k')
    \left(
        e^{i a^{-\frac{3}{2}}\left(\frac{\omega^2 k'}{2}\right)}
        + \mathrm{c.c}
    \right) \d k'
    \\ &=
    \bigg[
        \underbrace{
            \psi_2\left(-\frac{\omega^2}{2 a^{\frac{3}{2}}}\right)
        }_{O(a^k) \; \forall k \in \mathbb{N}}
        + \underbrace{i
            \underbrace{\left(\psi_2 * \mathcal{P}(1/x)\right)}_{O(x^{-1})}
            \left(-\frac{\omega^2}{2 a^{\frac{3}{2}}}\right)
        }_{
            O\left(\left(-\frac{\omega^2}{2 a^{\frac{3}{2}}}\right)^{-1}\right)
            = O\left(a^{\frac{3}{2}}\right)
           }
     + \mathrm{c.c}
     \bigg]
     \\ &\sim
    O\left(a^{\frac{3}{2}}\right)
\end{align*}
% \begin{dmath*}
%     \hat f_a(\omega)
%     =
%     \int \hat\psi_2(k')\Theta(k')
%     \left(
%         e^{i a^{-\frac{3}{2}}\left(\frac{\omega^2 k'}{2} + O(a^1)\right)}
%         + e^{-i \cdots}
%     \right) \d k
%     \\
%     \stackrel{a \to 0}{\longrightarrow}
%     \int \hat\psi_2(k')\Theta(k')
%     \left(
%         e^{i a^{-\frac{3}{2}}\left(\frac{\omega^2 k'}{2}\right)}
%         + e^{-i \cdots}
%     \right) \d k'
%     =
%     \bigg[
%         \underbrace{
%             \psi_2\left(-\frac{\omega^2}{2 a^{\frac{3}{2}}}\right)
%         }_{O(a^k) \; \forall k \in \mathbb{N}}
%         + \underbrace{i
%             \underbrace{\left(\psi_2 * \mathcal{P}(1/x)\right)}_{O(x^{-1})}
%             \left(-\frac{\omega^2}{2 a^{\frac{3}{2}}}\right)
%         }_{
%             O\left(\left(-\frac{\omega^2}{2 a^{\frac{3}{2}}}\right)^{-1}\right)
%             = O\left(a^{\frac{3}{2}}\right)
%            }
%      + (\textrm{anderer Term})
%      \bigg]
%      \sim O\left(a^{\frac{3}{2}}\right)
% \end{dmath*}

Also dürfen wir für $a \to 0$ schreiben $\hat f_a(\omega) = C a^{\frac{3}{2}} + o\left(a^{\frac{3}{2}}\right)$ und landen bei

\begin{dmath}
    \lim_{a \to 0}
    \left\langle \rwhat{\Delta}_m^{\circledast 2}, \rwhat{\psi}_{a-1t}
    \right\rangle
    =
    a^{-\frac{3}{4}} \int C a^{\frac{3}{2}} \omega \hat\psi_1(\omega)
    e^{-i\omega \frac{t'+x'}{a}}
    \d \omega
    \sim O\left(a^{\frac{3}{4}}\right) \condition{falls $t'=-x'$}
    \sim O\left(a^k\right) ~~ \forall k \hiderel \in \mathbb{N}
                              \condition{sonst}
\label{eq:delta_m2_twisted_s=-1}
\end{dmath}

\subsection{Zusammenfassung und Vergleich der Ergebnisse}
Fassen wir wie bisher schon die Ergebnisse aus \cref{eq:delta_m2_twisted_s>1,eq:delta_m2_twisted_s<1_x_neq_0,eq:delta_m2_twisted_s<1_x=0,eq:delta_m2_twisted_s=-1} wieder in einer Übersichtstabelle zusammen:

\begin{table}[h]
\centering
\begin{tabular}{l|cccc}
        & $(t',x') = 0$     & $t'=x' \neq 0$    & $t'=-x' \neq 0$   & $t' \neq \pm x'$ \\ \hline
$s=1$   & $a^{\frac{3}{4}}$ & $a^{\frac{3}{4}}$ & $a^k$             & $a^k$            \\
$s=-1$  & $a^{\frac{3}{4}}$ & $a^k$             & $a^{\frac{3}{4}}$ & $a^k$            \\
$|s|<1$ & $a^k$             & $a^k$             & $a^k$             & $a^k$            \\
$|s|>1$ & $a^k$             & $a^k$             & $a^k$             & $a^k$
\end{tabular}
\caption{Konvergenzordnung von $\mathcal{S}_{\Delta_m^{\star 2}}(a,s,(t',x'))$ im Limit $a \to 0$ für alle interessanten Kombinationen von $s$ und $(t',x')$}
\label{tab:wavefrontset_delta_m2_twisted}
\end{table}

Auch diesmal stimmen die Ergebnisse mit denen von \textcite[Prop. 3.72]{Schulz2014}\footnote{So weit sie gegeben wurden} überein, welcher für alle Potenzen des getwisteten Produkts $\Delta_+^{\star k}$ erhält:

\begin{equation*}
\left\langle t,x; \omega, k \right\rangle \hiderel\in WF(\Delta_+^{\star k})
\Rightarrow
- \omega \geq |k|
\end{equation*}
% section dots_und_nun_zur_wellenfrontmenge_von_ (end)


% section die_wellenfrontmenge_von_ (end)


%!TEX root = main.tex
% !TEX spellcheck=de_DE
%%%%%%%%%%%%%%%%%%%%%%%%%%%%%%%%%%%%%%%%%%%%%%%%%%%%%%%%%%%%%%%%%%%%%%%%%%%%%%%
% % Section 3
%%%%%%%%%%%%%%%%%%%%%%%%%%%%%%%%%%%%%%%%%%%%%%%%%%%%%%%%%%%%%%%%%%%%%%%%%%%%%%%
\section{\texorpdfstring{Berechnen von $WF(G_F)$}
        {Berechnen von WF(Gf)}} % (fold)
\label{sec:berechnen_von_}

\subsection{\texorpdfstring{Ausdrücke für $\left< \psi_{ast}, G_F\right>$}
        {Ausdrücke für psi ast, gf}} % (fold)
\label{sec:psiast_gf}

Ab jetzt werden wir der Namenskonvention der Physiker in der SRT folgen und unsere
Ortsraumvariablen mit $x = (t, x)$ und unsere Impulsraumvariablen mit $\xi = (\omega, k)$
bezeichnen sowie konsequenterweise das Minkowskiskalarprodukt $x \cdot \xi = \omega t - k x$
verwenden. Des weiteren wird der Verschiebungsparameter mit  $t = (t', x')$ bezeichnet.

Die massive skalare Zweipunktfunktion bzw. der Feynmanpropagator im Impulsraum ist dann gegeben durch (\textcite{Schwartz2014}, (6.34))

\begin{equation}
\label{eq:gf}
    \hat G_F(\omega, k) = \frac{1}{m^2 - \omega^2 + k^2 - i 0^+}
\end{equation}

Setzen wir dies in unsere Ausdrücke für $\left< \psi_{ast}, f\right>$ aus \eqref{Eq:substitution1}
bzw. \eqref{Eq:substitution1} ergibt sich, unter Verwendung des Minkowskiskalaprodukts,

\begin{align}
\left< \hat\psi_{ast}, \hat G_F \right> &=
    \int \hat \psi_{ast}(\omega, t) ~\hat G_F(\omega, t) ~\d \omega \d k
    \nonumber \\
    &=
    a^{\frac{3}{4}} \iint \frac{
        \hat \psi_1 (a \omega)
        ~\hat \psi_2 \left(a^{-\frac{1}{2}}\frac{k}{\omega} - s\right)
        ~ e^{-i\omega t' + i k x'}
    }
    {
        m^2 - \omega^2 + k^2 - i 0^+
    }
    \d \omega \d k
    \nonumber \\
    &=
    a^{-\frac{3}{4}} \iint \frac{
        \hat\psi_1(\omega)
        ~\hat \psi_2\left(\tfrac{k}{w}\right)
        ~e^{-i \omega \frac{t' - sx'}{a} + ik \frac{x'}{\sqrt{a}}}
    }
    {
        m^2 - \left(\frac{\omega}{a}\right)^2
        + \left(\frac{\omega s}{a} + \frac{k}{\sqrt{a}}\right)^2 - i0^+
    }
    \d \omega \d k \nonumber \\
    &=
    a^{-\frac{3}{4}}
    \kern -2em \iint
    \limits_{
    \substack{
        \omega \in [-2, -\frac{1}{2}]\cup[\frac{1}{2},2] \\
        \left|\frac{k}{2}-s\right| \leq \sqrt{ax}
        }
    }
    \kern -1.5em
    \frac{
        \hat\psi_1(\omega)
        ~\hat \psi_2\left(\tfrac{k}{\omega}\right)
        ~e^{-i \omega \frac{t' - sx'}{a} + ik \frac{x'}{\sqrt{a}}}
    }
    {
        m^2 + a^{-2} \omega^2 (s^2 - 1) + a^{-\frac{3}{2}} 2 s \omega k + a^{-1} k  - i0^+
    }
    \d \omega \d k
    \label{eq:psi_ast_gf_1}
\end{align}

\todo{Integral hübsch machen. Größeres Integralzeichen?}

und mit der anderen Substitution analog

\begin{align}
    \left< \hat\psi_{ast}, \hat G_F \right>
    &=
    a^{-\frac{3}{4}}
    \kern -1em
    \iint \limits_{\substack{
        |\omega|~ \in~ [\frac{1}{2},2] \\
        k ~\in~ [-1,1]
        }
    }
    \kern -1em
    \frac{
        \omega ~\hat \psi_1(\omega) ~\hat \psi_2(k)
        e^{-i \omega \left(\frac{t' - sx'}{a} + \frac{kx'}{\sqrt{a}}\right)}
    }
    {
        m^2 - \omega^2(a^{-2}(1-s^2)-a^{-1}k^2 - 2 k s a^{-\frac{3}{2}})
    }
    \d \omega \d k
    \label{eq:psi_ast_gf_2}
\end{align}

wobei sich die Integrationsbereiche aus den Forderungen an den Träger von $\psi$
(vgl. \eqref{eq:supp_psi}) ergeben.




Nach Satz \eqref{thm:main_theorem} genügt es zu bestimmen, an welchen Punkten
$(t', x')$ und in welche Richtungen $s$ $\mathcal{S}_f(a,s,(t',x'))$ nicht schnell-fallend in $a^{-1}$ ist, um die Wellenfrontmenge zu bestimmen. Da wir keine explizite erzeugende Funktion $\psi$ angegeben haben, werden wir uns dabei Argumente bedienen, die alleine auf den allgemeinen Eigenschaften von $\psi_{ast}$ beruhen, aber nicht einer expliziten Form.

\todo{In Textform beschreiben, was die grobe Strategie ist, also wie der Integrand
vernünfitg vereinfacht wird und welche Eigenschaften von¸$\psi$ wie eingehen.}

\todo{Hier schon die Ergebnisse als Satz angeben, und dann Beweis hinschreiben?}
\todo{Bemerkung einfügen, warum dass auch ziemlich unmöglich ist}

Das allgemeine Vorgehen wird dabei folgendes sein: Die Ausdrücke in \eqref{eq:psi_ast_gf_1}
und \eqref{eq:psi_ast_gf_2} genau anstarren, um zu sehen für welche Werte von
$(t',x')$ und $s$ potentiell interessante Dinge geschehen, also z.B. Terme im Nenner
weg fallen, oder die Phase konstant wird. Dann werden diese Werte von $(t',x')$ und
$s$ eingesetzt und alles so weit vereinfacht und genähert -- im Rahmen des Erlaubten, ohne
das Verhalten für $a \rightarrow 0$ zu ändern --, bis die $a$-Abhängigkeit abgelesen
werden kann. Entscheidende Zutaten sind dabei der beschränkte Träger von $\hat \psi$
und der schnelle Abfall von $\psi$.


\subsubsection*{Fall $s=1, t' = 0 = x'$}
Nach \eqref{eq:psi_ast_gf_2} erhalten wir mit $s=1, t' = 0 = x'$

\begin{align*}
    \left< \hat\psi_{a10}, \hat G_F \right>
    &=
    \int a^{-\frac{3}{4}} \frac{
        \omega ~\hat \psi_1(\omega) ~\hat \psi_2(k)
    }
    {
        m^2+\omega^2 (a^{-1}k^2 + a^{-\frac{3}{2}}2 k )
    }
    \d \omega \d k \\
    &=
    \int a^{\frac{3}{4}} \frac{
        \omega ~\hat \psi_1(\omega) ~\hat \psi_2(k)
    }
    {
        a^{\frac{3}{2}} m^2+\omega^2 (a^{\frac{1}{2}}k^2 + 2 k )
    }
    \d \omega \d k
\end{align*}

Da aber $|\omega| \in [\frac{1}{2},2]$ und $k \in [-1,1]$ ist, ist für hinreichend
kleine $a$ (und für genau die interessieren wir uns ja)

\begin{equation*}
    \left|
        \frac{\omega ~\hat \psi_1(\omega) ~\hat \psi_2(k)}{k \omega^2}
    \right|
    \geq
    \left|
        \frac{\omega ~\hat \psi_1(\omega) ~\hat \psi_2(k)}
        {a^{\frac{3}{2}}m^2+a^{\frac{1}{2}}\omega^2 k+2k \omega^2}
    \right|
\end{equation*}

eine integrierbare (im Sinne des Cauchy-Hauptwertes) Majorante für den Integranden.

\todo{Warum ist Cauchy-Hauptwert hier erlaubt? Weiter ausführe, warum es diese Majorante tut?}

Wir dürfen uns also des Lebesgueschen Konvergenzsatzes bedienen und schreiben

\begin{equation}
    \lim_{a \rightarrow 0} \left< \hat\psi_{a10}, \hat G_F \right> =
    a^{\frac{3}{4}} \int \frac{
    \omega ~\hat \psi_1(\omega) ~\hat \psi_2(k)
    }
    {
    2k \omega^2
    }
    \d \omega \d k
    \sim O(a^{\frac{3}{4}})
\end{equation}

Für $s = -1$ erhalten wir genau das selbe Ergebniss, da ja der $\omega^2 (1-s^2)$-Term
im Nenner genauso wieder verschwindet.

\subsubsection*{Fall $s \neq \pm 1, t' = 0 = x'$}
In diesem Fall verschwindet der $\omega^2 (1-s^2)$-Term im Nenner nicht und
dementsprechend folgt

\begin{align*}
    \left< \hat\psi_{as0}, \hat G_F \right>
    &=
    \int a^{-\frac{3}{4}} \frac{
        \omega ~\hat \psi_1(\omega) ~\hat \psi_2(k)
    }
    {
        m^2-\omega^2 ((1-s^2) - a^{-1}k^2 - a^{-\frac{3}{2}}2 k )
    }
    \d \omega \d k \\
    &=
    \int a^{\frac{5}{4}} \frac{
        \omega ~\hat \psi_1(\omega) ~\hat \psi_2(k)
    }
    {
        a^2 m^2+\omega^2 (s^2-1) + a \omega^2 k^2 + a^{\frac{1}{2}}2 \omega^2 k s
    }
    \d \omega \d k
\end{align*}

Analog zum vorigen Teil ist, diesmal sogar ohne den Cauchy-Hauptwert bemühen zu
müssen,

\todo{Überall wo es sein muss $\lim_{a \rightarrow 0}$ dazu schreiben, oder sagen
dass der Limit überall impliziert ist}

\begin{equation*}
    \left|
        \frac{2 \omega ~\hat \psi_1(\omega) ~\hat \psi_2(k)}{\omega^2 (1-s^2)}
    \right|
    \geq
    \left|
        \frac{
        \omega ~\hat \psi_1(\omega) ~\hat \psi_2(k)
    }
    {
        a^2 m^2+\omega^2 (s^2-1) + a \omega^2 k^2 + a^{\frac{1}{2}}2 \omega^2 k s
    }
    \right|
\end{equation*}

dass eine integrierbare Majorante ist (in der Tat ja sogar in $C_c^\infty (\mathbb{R}^2)$)
Damit können wir folgende Abschätzung treffen:

\begin{equation*}
    \lim_{a \rightarrow 0} \left< \hat\psi_{as0}, \hat G_F \right> =
    a^{\frac{5}{4}} \int \frac{2 \omega ~\hat \psi_1(\omega) ~\hat \psi_2(k)}
    {\omega^2 (1-s^2)}
    \d \omega \d k
    \sim O(a^{\frac{5}{4}})
\end{equation*}


\subsubsection*{Fall $s \neq \pm 1, (t', s') \neq 0$}
In diesem Fall benutzen wir wieder die erste Substitution \eqref{eq:psi_ast_gf_1}
und klammern wie schon in den beiden vorigen Teilen die höchste negative
Potenz von $a$ im Nenner aus.

\begin{align}
\Rightarrow ~
    \left< \hat\psi_{ast}, \hat G_F \right>
    &=
    a^{\frac{5}{4}} \int \frac{
        \hat \psi_1 (\omega) ~\hat \psi_2 \left(\frac{k}{\omega}\right)
        ~ e^{-i \omega \left(\frac{t'-sx'}{a}\right) + i k \frac{x'}{\sqrt{a}}}
    }
    {
        a^2 m^2 - \omega^2 (1-s^2) + a^{\frac{1}{2}} s \omega k +a k^2
    }
    \d \omega \d k
\end{align}

und da immer noch $0 \notin supp(\psi_1)$ gilt ist ein weiteres mal eine integrierbare Majorante gegeben durch

\begin{equation}
    2\frac{\hat \psi_1 (\omega)~\hat\psi_2 \left(\frac{k}{\omega}\right)}
    {\omega^2(s^2-1)}
\end{equation}

In der Tat ist sogar

\begin{equation}
    \hat f(\omega, k) := \frac{\hat \psi_1 (\omega)~\hat\psi_2 \left(\frac{k}{\omega}\right)}
    {\omega^2(s^2-1)}
    \in C_c^\infty (\hat{\mathbb{R}}^2)
\end{equation}

da $\psi_1$ und $\psi_2$ getragen sind. Demnach ist die Fourierinverse von
$\hat f$, $f := \mathcal{F}^{-1}(\hat f) \in \mathcal{S}(\mathbb{R}^2)$, also schnell
fallend. Damit können wir schließlich abschätzen

\begin{align}
    \left| \left< \hat\psi_{ast}, \hat G_F \right> \right|
    &=
    a^{\frac{5}{4}} \left|  \int \hat f(\omega, k)
    ~e^{-i \omega \left(\frac{t'-sx'}{a}\right) + ik \frac{x'}{\sqrt{a}}}
    \d \omega \d k
    \right|
    \nonumber \\
    &=
    a^{\frac{5}{4}} \left| f \left(\frac{t'-sx}{a}, \frac{x'}{\sqrt{a}}\right) \right|
    \leq
    a^{\frac{5}{4}} C_k\left(
    1 + \left\lVert \substack{(t'-sx')/a \\ x'/\sqrt{a}} \right\rVert
    \right)^{-k}
    \nonumber \\
    &\leq
    a^{\frac{5}{4}} \frac{C_k}{2} a^{\frac{k}{2}} \left\lVert
    \substack{(t'-sx') \\ x'} \right\rVert^{-k}
    \sim O\left(a^{\frac{5/2+k}{2}}\right) ~~ \forall k \in \mathbb{N}
    \nonumber \\[1em]
    \Rightarrow
     \left| \left< \hat\psi_{ast}, \hat G_F \right> \right|
     &\sim
     O\left(a^k\right) ~~ \forall k \in \mathbb{N}
\end{align}


\subsubsection*{Fall $s = 1, (t', x') \neq 0$}
Auch in diesem Fall nutzen wir wieder den ersten Ausdruck für
$\left< \hat\psi_{a1t}, \hat G_F \right>$ aus \eqref{eq:psi_ast_gf_1} und sorgen
wir auch bisher jedes Mal dafür, dass wir im Nenner nur noch positive Potenzen von
$a$ und einen von $a$ unabhängigen Term haben. Dann sieht das ganze so aus:

\begin{equation*}
    \left< \hat\psi_{a1t}, \hat G_F \right>
    =
    a^{\frac{3}{4}} \int \frac{\hat \psi_1(\omega)
    ~\hat\psi_2 \left(\frac{k}{\omega}\right)
    ~ e^{-i \omega \left(\frac{t'-x'}{a}\right) + i k \frac{x'}{\sqrt{a}}}
    }
    {
        a^{\frac{3}{2}} m^2 + a^{\frac{1}{2}} k^2 + 2 \omega k
    }
    \d \omega \d k
\end{equation*}

wo wir im $\lim_{a \rightarrow 0}$ wieder doe $a$-Potenzen im Nenner weg fallen lassen
und auch dieses Mal dafür wieder den Cauchy-Hauptwert bemühen müssen, um den
Lebesgueschen Konvergenzsatz benutzen zu dürfen.
Weiter geht's:

\begin{align}
    &=
    a^{\frac{3}{4}} \int \frac{
    \hat \psi_1(\omega)
    ~\hat\psi_2 \left(\frac{k}{\omega}\right)
    ~ e^{-i \omega \left(\frac{t'-x'}{a}\right) + i k \frac{x'}{\sqrt{a}}}
    }
    {
    2\omega k
    } \d \omega \d k \nonumber \\
    &= a^{\frac{3}{4}} \int
    \underbrace{\left\{ \int \frac{\hat \psi_2\left(\frac{k}{\omega}\right)
        ~e^{ik\frac{x'}{\sqrt{a}}}
        }
        {
            2 k \omega
        }
        \d k
        \right\}}_{=: \hat f_a (\omega)}
    \hat \psi_1(\omega) e^{-i \omega \left(\frac{t'-x'}{a}\right)}
    \d \omega
\end{align}

und um hier weiter zu kommen, schauen wir uns $\hat f_a$ genauer an. Sei dazu
$\Psi_2(\omega) := \int_{-\infty}^\omega \psi_2(\omega ') \d \omega '
    -  \int_{\omega}^{+ \infty} \psi_2(\omega ') \d \omega '$ eine
Stammfunktion von $\psi_2$. Dies ist offenbar $C^\infty$ und beschränkt, da
 $\hat \psi_2 \in C^\infty_c$. Mithilfe von Fourieridentitäten und Substitution können wir nun weiter rechnen:

\begin{align*}
    \hat f_a (\omega) &=
    \int \frac{\hat \psi_2\left(\frac{k}{\omega}\right)}{2k\omega}
    e^{i k \frac{x'}{\sqrt{a}}}
    \d \omega \\
    &\stackrel{i)}{=}
    \int \frac{\hat \psi_2 (k)}{2k}e^{ik\frac{x' \omega}{\sqrt{a}}}
    \d \omega \\
    &\stackrel{ii)}{=} \frac{i}{2}  \Psi_2\left(\frac{x' \omega}{\sqrt{a}}\right)
\end{align*}

Hier wurde in $i)$ einfach $k \rightarrow \omega k$ substituiert und im Schritt $ii)$
wurde genutzt, dass $f(x) = \mathrm{sgn}(x) \leftrightarrow \hat f(k) \sim \frac{1}{k}$.
Nun stecken wir diese Erkenntnisse in unseren vorigen Ausdruck und erhalten

\begin{align}
 \left< \hat\psi_{a1t}, \hat G_F \right>
    &=
    \frac{i a^{\frac{3}{4}}}{2} \int \Psi_2\left(\frac{x' \omega}{\sqrt{a}}\right)
    ~\hat \psi_1(\omega)
    ~ e^{-i \omega \left(\frac{t'-x'}{a}\right)}
    \d \omega \d k
    \nonumber \\
    &\sim O\left(a^{\frac{3}{4}}\right) \kern 1em \textrm{ ; für } t'=x'
    \nonumber \\
    &\sim O\left(a^k\right) ~ \forall k \in \mathbb{N} \kern 1em \textrm{;   andernfalls}
\end{align}

Im letzten Schritt wurde wieder genutzt, dass
$\Psi_2\left(\frac{x' \omega}{\sqrt{a}}\right) ~\hat \psi_1(\omega) \in \mathcal{S}(\mathbb{R})$
ist, und demnach eine schnell fallende Fouriertransformierte hat.

% Es gilt $\psi_2(0) = 1$, da nach Konstruktion
% $\Vert \psi_2 \Vert_1 = 1$. Außerdem können wir $\psi_2$ so wählen, dass es
% auf einer ganzen offenen Umgebung von 0 konstant 1 ist. Durch geschicktes addieren
% einer 0 können wir nun schreiben

% \begin{equation*}
%     \hat f_a (\omega) =
%     \int \frac{\hat \psi_2 \left(\frac{k}{\omega}\right) - 1}{2k\omega}
%     e^{ik\frac{x'}{\sqrt{a}}} \d k
%     + \int \frac{1}{2k\omega}
%     e^{ik\frac{x'}{\sqrt{a}}} \d k
% \end{equation*}

% wobei das Symbol des ersten Terms glatt ist, da $\psi_2 (0) = 0$. Der zweite Term
% wird also für $a^{-\frac{1}{2}} \rightarrow \infty$ dominieren. Für diesen gilt:
% \todo{Diese Argument verfeinern, oder mindestens raus finden, warum es denn
% zulässig ist.}

% \begin{equation*}
%     \int \frac{e^{ik\frac{x'}{\sqrt{a}}}}{2k\omega} \d k
%     = \frac{2 \pi i}{2 \omega} \mathrm{sgn}\left(\frac{x'}{\sqrt{a}}\right)
% \end{equation*}

% als Hauptwertintegral. Bedenkend dass $\psi_1 \in \mathcal{S} (\mathbb{R})$ und
% $\hat \psi_1 = 0$ in einer Umgebung von $0$ sowie
% $\mathrm{sgn}\left(\frac{x'}{\sqrt{a}}\right) = \mathrm{sgn}\left(x'\right)$  für $a>0$
% können wir also schließlich abschätzen

% \begin{align}
%     \lim_{a \rightarrow 0}\left< \hat\psi_{a1t}, \hat G_F \right>
%     &= \lim_{a \rightarrow 0} a^\frac{3}{4} C \int \frac{\mathrm{sgn}(x')
%     ~\hat \psi_1(\omega)}{\omega}
%     e^{-i\omega \frac{t'-x}{a}} \d \omega
%     \nonumber \\
%     &\sim O\left(a^\frac{3}{4}\right) \kern 1em \textrm{ ; für } t'=x'
%     \nonumber \\
%     &\sim O\left(a^k\right) ~ \forall k \in \mathbb{N} \kern 1em \textrm{;   andernfalls}
% \end{align}

% wobei in $C$ alle irrelevanten Vorfaktoren gesammelt wurden.

Das analoge Ergebnis erhält man auch für $s=-1$ und $t' = -x'$
% section berechnen_von_ (end)


%!TEX root = main.tex
%!TEX spellcheck=de_DE
%%%%%%%%%%%%%%%%%%%%%%%%%%%%%%%%%%%%%%%%%%%%%%%%%%%%%%%%%%%%%%%%%%%%%%%%%%%%%%%
% % Section 2
%%%%%%%%%%%%%%%%%%%%%%%%%%%%%%%%%%%%%%%%%%%%%%%%%%%%%%%%%%%%%%%%%%%%%%%%%%%%%%%

\section{Ausblick} % (fold)
\label{sec:ausblick}

\subsection{Den Beweis des Hauptsatzes auf $f \in \mathcal{S}'$ ausweiten}

\subsection{Hörmanders Kriterium abschwächen}

\subsection{Höherdimensionale Shearlets}
% section ausblick (end)

\printbibliography

\Declaration

\end{document}
