
% !TEX encoding = UTF-8 Unicode
%%%%%%%%%%%%%%%%%%%%%%%%%%%%%%%%%%%%%%%%%%%%%%%%%%%%%%%
% % Lines starting with % are comments, which are ignored.
% % This is a handy way of indicating the date and version of
% % your document, to wit:
% %
% % LaTeX sample file
% % Modified March, 2002
% %
%%%%%%%%%%%%%%%%%%%%%%%%%%%%%%%%%%%%%%%%%%%%%%%%%%%%%%%

% \documentclass{article}
\documentclass[bachelor,       %% Typ der Arbeit: bachelor oder master
               twoside,        %% zweiseitiges Layout
               BCOR=8mm,
               DIV=11,       %% Bindekorrektur 10 mm
%               liststotoc,nomtotoc,bibtotoc, %% Aufnahme der div. Verzeichnisse
                                              %% ins Inhaltsverzeichnis
               english,ngerman, %% Alternativspr. Englisch, Dokumentspr. Deutsch
%               ngerman,english  %% Alternativspr. Deutsch, Dokumentspr. Englisch
%               final,          %% Endversion; draft fuer schnelles Kompilieren
               ]{GAUBM}
\usepackage{thesisstyle}

\addbibresource{literature.bib}

\begin{document}

\makeatletter
\everymath{%
  \ifodd\value{page}\allowdisplaybreaks[0]%
    \else \allowdisplaybreaks[4]%
  \fi
}
\makeatother

\pagenumbering{roman}

\ThesisAuthor{Jan Lukas}{Bosse}
%% Hier den Geburtsort einsetzen
\PlaceOfBirth{Freiburg im Breisgau}
%% Titel Arbeit. Das erste Argument ist der deutsche, das zweite der
%% englische Titel.
\ThesisTitle{Zur Auflösung der Wellenfrontmenge mittels Shearlets}{Resolution of the wavefrontset using shearlets}
%% Erst- und Zweitgutacher/in
%% Ist der/die Betreuer/in nicht identisch mit dem/r Erstgutachter/in,
%% muss diese/r als optionales Argument angegeben werden.
\FirstReferee{Prof.\ Dr.\ Dorothea Bahns}
\Institute{Institut f\"ur Mathematik}
\SecondReferee{Prof.\ Dr.\ Ingo Witt}
%% Beginn und Ende des Anfertigungszeitraumes
\ThesisBegin{15}{2}{2018}
\ThesisEnd{30}{7}{2018}
%% DO NOT TOUCH THESE LINES!!!!
% \frontmatter
\maketitle
\cleardoublepage

%%%%%%%%%%%%%%%%%%%%%%%%%%%%%%%%%%%%%%%%%%%%%%%%%%%%%%%%%%%%%%%%%%%%%%%%%%%%%%%
% % Kein Abstract
%%%%%%%%%%%%%%%%%%%%%%%%%%%%%%%%%%%%%%%%%%%%%%%%%%%%%%%%%%%%%%%%%%%%%%%%%%%%%%%
% \begin{abstract}
%   %!TEX root = main.tex

Einer der Zugänge zur Renormierung in der Quantenfeldtheorie ist die Erweiterung der auftretenden Produkte von Distributionen auf ganz $\mathbb{R}^{1+d}$. Um zu bestimmen, wo und wie diese erweitert werden können, muss die Wellenfrontmenge der Faktoren bestimmt werden. Leider ist es notorisch schwierig Wellenfrontmengen für Distributionen komplizierter als die $\delta$-Distribution und Ableitungen direkt zu bestimmen.
Ursprünglich in der Bildbearbeitung und Kompression wurde erkannt und zur Kompression genutzt, dass Wavelettransformationen in der Lage sind, die Singularitätsstruktur von Bildern zu erkennen.
Wie \textcite {Kutyniok2008} sowie \textcite
{Candes2005} aber gezeigt haben, lässt sich diese Erkenntnis auf Distributionen ausweiten und mit anisotropen und gerichteten Wavelets Wellenfrontmengen ausrichten. In der vorliegenden Arbeit wollen wir am Beispiel von \emph{Shearlets} untersuchen, wie praktikabel für die direkte Berechnung diese Methoden für Distributionen abseits der $\delta$-Distribution sind um Wellenfrontmengen zu bestimmen. Des weiteren wird noch eine kurze Diskussion gegeben, welche weiteren Begriffe der mikrolokalen Analysis mithilfe von Shearlets

%   % \bigskip\par
%   % \textbf{Stichwörter:} Physik, Bachelorarbeit
% \end{abstract}
% %% So laesst sich in die andere Sprache umschalten (Englisch bzw. Deutsch)
% \begin{otherlanguage}{english}
% \begin{abstract}
%   Do we need an english abstract?
%   \bigskip\par
%   \textbf{Keywords:} Physics, Bachelor thesis
% \end{abstract}
% \end{otherlanguage}

%% Ende des Vorspanns
\cleardoublepage
%% Ab hier 1 1/2 facher Zeilenabstand (durch setspace-Paket)
\onehalfspacing
%% Erzeugt Inhaltsverzeichnis

\pagenumbering{arabic}
\tableofcontents
% \mainmatter

%!TEX root = main.tex
\section{Einleitung für Mathematiker} % (fold)
\label{sec:einleitung_mathematik}

Ursprünglich in der Bildbearbeitung und -kompression wurde erkannt und genutzt, dass die (stetige) Wavelettransformationen einer Funktion $f$ schnell abfällt an Punkten, an denen $f$ glatt ist und langsam an den Singularitäten. Bekanntestes Beispiel dafür ist die JPEG-Kompression, welche auf der Wavelettransformation basiert.

Allerdings ist die klassische Wavelettransformation mit gleichmäßiger Skalierung in alle Richtungen nicht in der Lage, die Orientierung der Singularitäten zu erkennen. Deshalb wurden verschiedene Verallgemeinerungen von Wavelets mit anisotroper Skalierung entwickelt (\cite{Guo2006} \cite{Kutyniok2008} \cite{Candes2005}), die in der Lage sind auch die Orientierung der Singularitäten zu erkennen. Im Fourierraum bedeutet anisotrope Skalierung, dass der Träger der Wavelets mit feiner werdendem Skalenparameter in immer engeren Kegeln liegt. Im Realraum entspricht dies immer flacher werdenden `"Wellenpaketen"' als Testfunktionen. Bei isotroper Skalierung hingegen wird der Träger im Realraum in alle Richtungen gleichmäßig kleiner.

Dies ist eng verwandt mit dem Konzept der \emph{Wellenfrontmenge} aus der \emph{mikrolokalen Analysis}. Die Wellenfrontmenge misst, vereinfacht gesagt, die Lage und Orientierung der Singularitäten von nicht nur Funktionen, sondern auch Distributionen. Das Versprechen in \cite{Kutyniok2008} ist, dass es mit anisotrop skalierenden Wavelettransformationen möglich ist Wellenfrontmengen zu berechnen.

Ziel der vorliegenden Arbeit ist es, genauer zu untersuchen inwiefern die Shearletttransformation von \textcite{Kutyniok2008} geeignet ist, um Wellenfrontmengen zu bestimmen. Dazu werden die Wellenfrontmengen von physikalisch motivierten Distributionen berechnet. Außerdem füllen wir eine kleine Lücke in \cite{Kutyniok2008}, geben einen Ansatz, wie die Ergebnisse von \textcite{Kutyniok2008} auf temperierte Distributionen ausgedehnt werden können und erklären, warum sie nicht auf alle Distributionen ausgedehnt werden können.
Des weiteren wird noch eine kurze Diskussion gegeben, welche weiteren Größen der mikrolokalen Analysis mithilfe von Shearlets berechnet werden können und welche Möglichkeiten es für höherdimensionale Shearlets gibt.


% section einleitung (end)



%!TEX root = main.tex
\chapter{Einleitung für Physiker} % (fold)
\label{sec:einleitung_physics}
Einer der Zugänge zur Renormierung in der Quantenfeldtheorie ist die Fortsetzung der auftretenden Produkte von Distributionen\footnote{Wir erinnern uns, dass allgemeine Produkte von Distributionen nicht unbedingt immer definiert sind. Was zum Beispiel soll \(\delta^2\) sein?} auf ganz $\mathbb{R}^{1+d}$. Um zu bestimmen, wo und mit welchen Freiheiten diese fortgesetzt werden können, müssen die Wellenfrontmengen der Faktoren bestimmt werden. Leider ist es notorisch schwierig Wellenfrontmengen für Distributionen, die komplizierter sind als die $\delta$-Distribution und Ableitungen, direkt zu bestimmen. Unter anderem in Modellen der \emph{nichtkommutativen Quantenfeldtheorie} \cite{kappaMinkowski,Doplicher1995,StringLocalized} treten Distributionen auf, deren Wellenfrontmengen mit den bisherigen Methoden nicht bestimmt werden konnten.

Ursprünglich in der Bildbearbeitung und -kompression wurde erkannt und zur Kompression genutzt, dass Wavelettransformationen in der Lage sind, die Singularitätsstruktur von Bildern zu erkennen. D.h. dass die Wavelettransformation mit feiner werdendem Skalenparameter an Singularitäten nicht schnell abfällt, überall sonst aber schon.
Wie \textcite{Kutyniok2008,Candes2005,Contourlets} in ihren respektiven Arbeiten gezeigt haben, lassen sich diese Erkenntnisse auf Distributionen ausweiten und mit anisotropen und gerichteten Wavelets Wellenfrontmengen ausrechnen.

In der vorliegenden Arbeit wollen wir am Beispiel der \emph{Shearlets} untersuchen, wie praktikabel diese Methoden sind, um Wellenfrontmengen komplizierterer Distributionen auszurechnen. Dazu ziehen wir als Beispiele die massive Zweipunktfunktion \(\Delta_m\), ihre (getwisteten) Quadrate \(\Delta_m^{(\star) 2}\) und die Heaviside-Funktion \(\Theta\) heran. Diese sind von besonderem Interesse, da der Feynmanpropagator und Potenzen davon als zentrale zu renormierende Distribution der QFT ein Produkt von Zweipunktfunktion und Heaviside-Funktion ist.

Daneben gibt es noch eine kurze Diskussion, ob und wie es möglich ist, die Ergebnisse auf mehr als nur zwei Dimensionen auszuweiten.
Des Weiteren wird skizziert, welche weiteren Größen der \emph{mikrolokalen Analysis}, wie z.B. der Skalengrad, mithilfe von Shearlets berechnet werden können. Der Skalengrad einer Distribution ist eng verwandt mit dem Abzählen der Potenzen (engl. "`power counting"') in der QFT.

Wir kommen zu dem Ergebnis, dass die Shearlettransformation in zwei Dimensionen zwar eine theoretische Möglichkeit ist, Wellefrontmengen zu berechnen, aber deutlich mehr Arbeit als weniger direkte Methoden. In höheren -- und damit physikalisch relevanteren -- Dimensionen sind noch keine Verallgemeinerung bekannt, aber die konkreten Rechnungen werden sicher nicht übersichtlicher als in zwei Dimensionen.


% section einleitung (end)


%!TEX root = main.tex

\section{Fouriertransformation, mikrolokale Analysis und all die Mathematik} % (fold)
\label{sec:fouriertransformation_mikrolokale_analysis_und_all_die_mathematik}

Hier entsteht mal ein Kapitel, in dem die notwendigen mathematischen Begriffe eingeführt und motiviert werden.

% section fouriertransformation_mikrolokale_analysis_und_all_das (end)


%!TEX root = main.tex

\section{Zweipunktfunktionen, Sternprodukte und all die Physik} % (fold)
\label{sec:zweipunktfunktionen_sternprodukte_und_all_die_physik}

Da als konkrete Beispiele für Distributionen, deren Wellenfrontmengen wir berechnen Distributionen aus der Physik verwendet werden, wollen wir diese kurz erklären.

\subsection{Die Zweipunktfunktionen und warum wir sie potenzieren wollen}
\label{sec:die_zweipunktfunktionen_und_warum_wir_sie_potenzieren_wollen}

In der störungstheoretischen Quantenfeldtheorie entsprechen schon einfache Feynmandiagramme, wie z.B. das in \cref{fig:feynman-diagramm} (formal) Integralen über Produkte von Distributionen, in diesem Fall dem Feynman-Propagator.

% \begin{tikzpicture}
% \begin{feynman}
% \vertex (a) {\(\phi)};
% \vertex [right=of a] (b);
% \vertex [right=of b] (c);
% \vertex [right=of c] (d) {\(\phi)};
% \diagram*{
% (a) -- [boson, edge label'=\(p\)] (b) %[dot, label=right:\(x_1\)]
% -- [boson, half left, edge label'=\(k\)] (c) %[dot, label=left:\(x_2\)]
% -- [boson, half left, edge label'=\(k-p\)] (b),
% (c) -- [boson, edge label=\(p\)] (d),
% };
% \end{feynman}
% \end{tikzpicture}

\begin{figure}[h]
\begin{equation*}
\feynmandiagram [layered layout, horizontal=b to c] {
a [particle=\(\phi\)] -- [fermion, edge label=\(p\)] b [dot, label=right:\(x_1\)]
-- [fermion, half left, edge label=\(k\)] c [dot, label=left:\(x_2\)]
-- [fermion, half left, edge label=\(k-p\)] b,
c -- [fermion, edge label=\(p\)] d [particle=\(\phi\)],
};
 =  \int G_F(x_1,x_2) \,G_F(x_2,x_1) e^{ik(x_1-x_2)} e^{i(p-k)(x_2-x_1)}
 \d x_1 \d x_2 \d k
\end{equation*}
\caption{Ein einfaches Feynman-Diagramm aus der skalaren $\phi^3$-Theorie und das entsprechende Integral über Feynman-Propagatoren}
\label{fig:feynman-diagramm}
\end{figure}

Der Feynman-Propagator in zwei Dimensionen kann geschrieben werden als zeitgeordnete Zweipunktfunktion (vgl. \textcite{ReedSimon}), also

\begin{equation}
    G_F(t,x)
    =
    \Theta (t)\Delta_m(t,x) + \Theta(-t)\Delta_m(-t,-x)
    \label{eq:feynman_propgator_as_product}
\end{equation}

Wobei $\Theta$ die Heaviside-Funktion bezeichnet und als $\Theta(t) \cdot 1(x)$ zu verstehen ist. Also sind Potenzen des Feynman-Propagators gegeben durch Potenzen der Zweipunktfunktion und der Heaviside-Funktion. Um zu wissen, wo diese Produkte definiert werden können, muss man deren Wellenfrontmengen kennen; dann liefert Hörmanders Kriterium \ref{thm:hoermanders_criterion} ein Kriterium für die Wohldefiniertheit.

In all dem kann die Zweipunktfunktion $\Delta_m$ geschrieben werden als Fouriertransformierte eines positiven Maßes auf der ngativen Massenschale $H_m$ (vgl. \textcite{Schwartz2014}, 24.69):

\begin{equation}
    \Delta_m (t,x) = \int \delta (\omega^2-k^2-m^2)
                    \Theta(-\omega)e^{-i\omega t + i k x} \d \omega \d k
\label{eq:delta_m}
\end{equation}

\subsection{Sternprodukte und getwistete Faltungen}
Die \emph{nicht kommutative Quantenfeldtheorie} beschäftigt sich mit Quantenfeldtheorien in der Größenordnung der \emph{Planck-Skala}. Bei diesen Größenordnungen wird erwartet, dass die Geometrie der Raumzeit nicht mehr kommutativ ist, sich also Ort und Zeit nicht mehr mit beliebiger Präzision messen lassen. Das physikalische Argument (nach \textcite{Doplicher1995}) für diese \emph{Raumzeitunschärferelation} basiert auf Einsteins allgemeiner Relativitätstheorie und Heisenbergs Unschärferelation: Wenn wir ein Raumzeit-Ereignis mit Genauigkeit $a$ messen, haben wir eine Impuls-Unschärfe von der Größenordnung $\frac{1}{a}$. Also wurde Energie der Größenordnung $\frac{1}{a}$ auf das System übertragen und zu einem Zeitpunkt in der gemessenen Ortsregion konzentriert. Diese Energie erzeugt ein Gravitationsfeld, welches um so stärker ist, je kleiner die Region in der die Energie konzentriert ist. Sobald dieses so stark ist, dass kein Licht mehr die Region verlassen kann (wir also ein schwarzes Loch erzeugt haben), erhalten wir keine Information aus der Raumzeitregion, eine Messung ist also nicht möglich. Das bedeutet, dass die Genauigkeit mit der wir die Lokalisation eines Ereignisses in der Raumzeit messen können beschränkt ist durch die Energiedichte, ab der wir ein schwarzes Loch erzeugen. Diese Schranken in der Messgenauigkeit lassen sich als Unschärferelation zwischen Zeit und Ort verstehen, ganz analog zur klassischen Unschärferelation zwischen Ort und Impuls.

Es gibt verschiedene Möglichkeiten, solche nicht-kommutativen Raumzeiten zu konstruieren. Ihnen allen ist gemein, dass das kommutative punktweise Produkt von Funktionen ersetzt wird durch ein nicht-kommutatives \emph{Sternrprodukt}.
Mehr Details finden sich in \textcite[Kap. 6]{Waldmann2007}.

Analog zu dieser Konstruktion ersetzen \textcite{Doplicher1995} das kommutative Produkt von Funktionen auf der Raumzeit durch ein Moyalprodukt $\star_M$ auf den Funktionen auf der Raumzeit und erhalten damit eine nicht-kommutative Geometrie. Gemäß dem Faltungssatz für die Fouriertransformierte gibt es auch eine \emph{getwistete Faltung} $\circledast$, s.d. der Faltungssatz erfüllt ist. Also
Dieses Produkt ist äquivalent zu den Vertauschungsrelationen
    \left[t,x_i\right] = -\frac{i}{\kappa} x_i.
Ein anderer Ansatz verwendet das Moyal-Produkt \cite{MoyalProduct} aus der Deformationsquantisierung. Hier wird das kommutative Produkt von Funktionen auf dem Phasenraum so deformiert, dass es danach die kanonischen Vertauschungsrelationen aus der Quantenmechanik erfüllt, also
\begin{equation*}
    \left[x_k, p_l\right] = i\delta_{kl}.
\end{equation*}

In zwei Dimensionen ist die getwistete Faltung definiert wie folgt:

Dabei ist $\Omega_{can}$ die kanonische symplektische Form auf $\mathbb{R}^{2n}$ und $x \in \mathbb{R}^{2n}$, also ein Punkt im Phasenraum und \emph{nicht} nur die Ortskoordinate.
Da das Moyalprodukt als Fourier-Multiplikator geschrieben werden kann, korrespondiert es auch zu einer getwisteten Faltung $\ast_\Omega$ für die Fouriertransformierten, s.d. der Faltungssatz $$\rwhat{f \star_M g} (k) = \rwhat{f} \ast_\Omega \rwhat{g} (k)$$ gilt.
\begin{definition}[getwistete Faltung]
\label{def:twisted_convolution}
    Seien $\hat f,\hat g \in $ "`passender Funktionen-/Distributionenraum"'. Sei $\Omega \in \mathbb{R}^{2n \times 2n}$ die kanonische symplektische Matrix. Dann ist die getwistete Faltung $(\hat f \ast_\Omega \hat g) (k)$ definiert als

    \begin{equation}
        (f \ast_\Omega g) \,(k) \coloneqq
        \int f(y) g(x-y)e^{\frac{i}{2} \Omega(x,y)} \d y
    \end{equation}

    Die getwistete Faltung ist also einfach die gewöhnliche Faltung, die noch mit einem ortsabhängigen Phasenfaktor verziert wurde.

    Wir verwenden die kanonische symplektische Matrix, also
    $\Omega = \left(\begin{smallmatrix}
        0 & 1 \\ -1 & 0
    \end{smallmatrix}\right)$
\end{definition}

Durch formale Rechnung, Ausschreiben der $e$-Funktion als Potenzreihe und nutzen der Fourieridentitäten $x \cdot \leftrightarrow i \partial_k$ sieht man, dass das Sternprodukt die Form

\begin{equation*}
    f \star_M g = fg + \frac{i}{2} \sum_{i,j} \Pi^{ij}(\partial_if)(\partial_jg) - \frac{1}{8}\sum_{i,j,k,m} \Pi^{ij} \Pi^{km} (\partial_i \partial_k f)(\partial_j \partial_m g) + \dots
\end{equation*}

hat. Dabei ist $\Pi$ der zu $\Omega$ korrespondierende Poisson-Bivektor.


% section zweipunktfunktionen_sternprodukte_und_all_die_physik (end)


%!TEX root = main.tex
%%%%%%%%%%%%%%%%%%%%%%%%%%%%%%%%%%%%%%%%%%%%%%%%%%%%%%%%%%%%%%%%%%%%%%%%%%%%%%%
% % Section 1
%%%%%%%%%%%%%%%%%%%%%%%%%%%%%%%%%%%%%%%%%%%%%%%%%%%%%%%%%%%%%%%%%%%%%%%%%%%%%%%
\section{Wavelettransformation und die Wellenfrontmenge} % (fold)
\label{sec:shearlets}

Die klassische Fouriertransformation $f(x) \mapsto \int f(x) e^{-ik x} \d x$
zerlegt eine Funktion in ihre verschiedenen Frequenzanteile und misst nach dem Satz von Payley-Wiener dabei auch die Regularität der Funktion. Es gilt nämlich $f \in C^N(\mathbb{R}^n) \cap L^1(\mathbb{R}^n) \Rightarrow \hat f(k) = O(k^N) $ für $|k| \to \infty$. Leider "`sieht"' die Fouriertransformation aber nicht, an welchen Punkten $x$ die Funktion $f$ singulär (= nicht glatt) ist. Das hängt damit zusammen, dass die "`Basisfunktionen"', die ebenen Wellen, nicht lokalisiert sind. Das Argument der Fouriertransformation $k$ kontrolliert die Richtung und Skala, die von der Basisfunktion $e^{-ikx}$ aufgelöst werden. Zusätzliche Ortsauflösung der Singularitäten gibt uns die


%%%%%%%%%%%%%%%%%%%%%%%%%%%%%%%%%%%%%%%%%%%%%%%%%%%%%%%%%%%%%%%%%%%%%%%%%%%%%%%
% % Wavelets
%%%%%%%%%%%%%%%%%%%%%%%%%%%%%%%%%%%%%%%%%%%%%%%%%%%%%%%%%%%%%%%%%%%%%%%%%%%%%%%
\subsection{Wavelettransformation} % (fold)
\label{sec:wavelettransformation}

% section wavelettransformation (end)

Einen Schritt in die richtige Richtung, nämlich die Ortsauflösung der Singularitäten, macht die Wavelettransformation. Hier wird eine Familie von Basisfunktionen für $L^2(\mathbb{R}^n)$ erzeugt von einem \textit{Mutterwavelet} $\psi$. Anders als die ebenen Wellen ist $\psi$ aber lokalisiert -- häufig sogar kompakt getragen -- und die Basis wird erzeugt durch Verschieben \emph{und} Skalieren des Mutterwavelets.

Eine Hamel-Basis für $L^2(\mathbb{R}^n)$ die aus Funktionen der Form

\begin{equation*}
    \left\{\psi_{at}(x) = a^{-\frac{n}{2}}\psi\left(a^{-1}(x-t)\right)  |~ t \in \mathbb{R}^n,  ~a \in \mathbb{R}\right\}
\end{equation*}

 mit einem \textit{Mutterwavelet} $\psi \in C^\infty (\mathbb{R}^n)$ besteht heißt \textit{stetige Waveletbasis} für $L^2(\mathbb{R}^n)$. Der Parameter $t$ heißt \textit{Verschiebungsparameter} und verschiebt das Wavelet an alle Orte des $\mathbb{R}^n$ während der \textit{Skalierungsparameter} $a$ für $a \to 0$ $\psi$ immer genauer lokalisiert. Der Faktor $a^{-\frac{n}{2}}$ sorgt dafür, dass die $L^2$-Norm aller $\psi_{at}$ gleich ist. In der Fourierdomäne wird die Verschiebung zum Phasenfaktor und der Träger mit verschwindendem $a$ immer \emph{größer}.
% Noch einmal als Formel:

% \begin{align*}
%     supp (\psi_{at}) &= a \cdot supp(\psi)+t \\
%     \Longleftrightarrow ~~
%     supp (\hat\psi_{at}) &= a^{-1} \cdot supp(\hat\psi_{at})
% \end{align*}

% Ist $t \in \mathbb{Z}^n$ (oder auf einem anderen diskreten Gitter in $\mathbb{R}^n$) und $a = 2^{-j}$, so spricht man von der \textit{diskreten Wavelettransformation}. Ist $t \in \mathbb{R}^n$ und $G \subseteq \mathrm{GL}(n,\mathbb{R})$ so ist es eine \textit{stetige Wavelettransformation} (engl. \textit{continuous wavelet transform}). Die Wavelettransformation einer Funktion (später auch temperierten Distribution) $f$ ist dann definiert als

Analog zur Definition der Fouriertransformation ist die stetige Wavelettransformation von $f$ die Projektion auf die Basisfunktionen:
\begin{equation}
    \mathcal{W}_f (a,t) = \left\langle f, \psi_{at} \right\rangle
    = a^{-\frac{n}{2}} \int f(x) \psi \left(a^{-1}(x-t)\right) \d x
\end{equation}

Ist $\psi$ eine glatte Funktion und $f$ bei $t$ glatt, so fällt $\mathcal{W}_f (a,t)$ schnell ab für $a \to 0$.
% Dies ist schnell einzusehen, wenn man sich überlegt, dass $\psi \in C^k$ impliziert, dass $\hat\psi$ eine $k$-fache Nullstelle bei $0$ hat und $\int x^l \psi(x) \d x = (-i)^{l} \partial_k^l \hat\psi(0) = 0 ~~\textrm{falls}~ l<k $.
\todo[color=green]{Kurze Plausibel machen, warum dem so ist?}
Umgekehrt fällt auch $\mathcal{W}_f(a,t)$ \emph{genau dann} nicht schnell ab, wenn $f$ bei $t$ \emph{nicht} glatt ist. Also löst die Wavelettransformation die Lage der Singularitäten von $f$ auf. Allerdings sind die klassischen  Wavelettransformationen mit isotroper Skalierung nicht in der Lage die  Orientierung der Singularitäten aufzulösen. Sie besitzen ja gar keinen Orientierungsparameter.

Um auch die Orientierung aufzulösen, muss einerseits ein Richtungsparameter eingeführt werden und andererseits dafür gesorgt werden, dass die Basisfunktionen mit immer feinerer Skala immer orientierter werden. Deshalb gibt es


%%%%%%%%%%%%%%%%%%%%%%%%%%%%%%%%%%%%%%%%%%%%%%%%%%%%%%%%%%%%%%%%%%%%%%%%%%%%%%%
% % gerichtete Wavelets
%%%%%%%%%%%%%%%%%%%%%%%%%%%%%%%%%%%%%%%%%%%%%%%%%%%%%%%%%%%%%%%%%%%%%%%%%%%%%%%
\subsection{Verallgemeinerte, gerichtete Wavelets} % (fold)
\label{sec:verallgemeinerte_gerichtete_wavelets}

% section verallgemeinerte_gerichtete_wavelets (end)

 Beispiele solcher gerichteter Wavelets sind die \textit{Curvelets} von \textcite{Candes2005}, die \textit{Shearlets} von \textcite{Kutyniok2008} sowie \textit{Contourlets} von \textcite{Contourlets}. Die mit feiner werdender Skala schärfer werdende Orientierung wird in den ersten beiden Fällen durch parabolische Skalierung implementiert. D.h. in Richtung der Orientierung im Fourierraum wird mit $a$ skaliert, während in den Richtungen senkrecht dazu mit $\sqrt a$ skaliert wird. In zwei Dimensionen gibt es nur eine weitere senkrechte Richtung, aber später wird deutlich werden, dass dies in mehr Dimensionen die richtige Verallgemeinerung der parabolischen Skalierung sein muss.

 Die Richtung der Curvelets wird durch Drehmatrizen implementiert, die auf die Variablen $(x_1,x_2)$ wirken, während bei den Shearlets die Variablen $(x_1,x_2)$ geschert werden.

%  Beide Ansätze sind bisher nur in zwei Dimensionen im Detail untersucht und arbeiten mit parabolischer Skalierung. Im Fall der \textit{Curvelets} wird die Richtungsabhängigkeit durch Drehmatrizen implementiert, während die \textit{Shearlets} mit Scherungen arbeiten. Konkret sieht dass dann so aus:

% \todo[color=green]{Hier die Formeln hin schreiben, oder ganz drauf verzichten}

% \begin{align*}
%     \psi_{a\theta t}(x) &= \det (AR)^{-\frac{1}{2}}
%                     \psi \left((AR)^{-1}(x-t) \right)
%     \condition{für Curvelets}\\
%     \psi_{ast}(x) &= \det (AS)^{-\frac{1}{2}}
%                     \psi \left((AS)^{-1}(x-t)\right)
%     \condition{für Shearlets}
% \end{align*}

% \begin{dgroup*}
% \begin{dsuspend}
%     mit der parabolischen Skalierungsmatrix
% \end{dsuspend}
% \begin{dmath*}
%     A = \begin{pmatrix}  a & 0 \\ 0 & \sqrt a\end{pmatrix}
% \end{dmath*}
% \begin{dsuspend}
%     der Drehmatrix
% \end{dsuspend}
% \begin{dmath*}
%     R = \begin{pmatrix}  \cos \theta & \sin \theta \\ - \sin \theta & \cos \theta \end{pmatrix}
% \end{dmath*}
% \begin{dsuspend}
%     und der Scherungsmatrix
% \end{dsuspend}
% \begin{dmath*}
%     S = \begin{pmatrix}  1 & -s \\ 0 & 1\end{pmatrix}
% \end{dmath*}
% \end{dgroup*}

Beide Ansätze sind in der Lage, die Wellenfrontmenge einer Distribution zu identifizieren. Allerdings sind die Rechnungen bei den \textit{Shearlets} in der praktischen Umsetzung einfacher, wenn auch von einem ästhetischen Standpunkt nicht ganz so befriedigend, da sie nicht inhärent rotationsinvariant sind, also nicht alle Symmetrien unseres Raumes abbilden. Aber nach allzu viel Ästhetik sollte man in dieser Arbeit, mit Hinblick auf die Rechnungen ab \cref{sec:die_wellenfrontmenge_von_delta_m}, ohnehin nicht fragen.

Bevor wir die konkrete Konstruktion der Shearlets widmen, brauchen wir noch ein kleines bisschen Theorie, welche Möglichkeiten wir überhaupt haben, um die Konstruktion der Wavelets zu verallgemeinern. Die weitestgehende Verallgemeinerung von "`verschiebe und skaliere ein Mutterwavelet"' um ein reproduzierendes System zu erhalten ist "`verschiebe es und lasse eine beliebge invertierbare Matrix auf die Koordinaten wirken"'. Wir definieren also eine Wirkung der affinen Gruppe $\mathbb{A}^n$ auf Funktionen $\psi \in L^2(\mathbb{R}^n)$ via

\begin{align*}
    \mathbb{A}^n \times L^2(\mathbb{R}^n) &\to L^2(\mathbb{R}^n)\\
   ((M,t) ,\psi (x)) &\mapsto |\det M ^{-\frac{1}{2}}|  \psi\left(M^{-1}(x-t)\right) \eqqcolon \psi_{M,t} (x)\\
\textrm{mit}\kern 8em&\\
    (M,t) &\in \mathbb{A}^n = \mathrm{GL}(n,\mathbb{R}) \ltimes \mathbb{R}^n
\end{align*}

Im Allgemeinen wird man aber nicht die ganze affine Gruppe benötigen, um ein reproduzierendes System zu erhalten, sondern nur alle Verschiebungen und eine Untermenge\footnote{Auch nicht notwendigerweise Untergruppe} der $\mathrm{GL}(n,\mathbb{R})$. Wann ein Mutterwavelet und die Wirkung einer solchen Untermenge ein reproduzierendes System erzeugen, sagt uns der nächste Satz:

\begin{theorem}[Zulässigkeitskriterium]
\label{thm:admissibility_criterion}
    Sei $\psi \in L^2(\mathbb{R}^n)$.
    Sei $G \subset \mathrm{GL}(n,\mathbb{R})$, $\d \mu(M)$ ein Maß auf $G$, im Falle einer Untergruppe z.B. das Haarmaß, und es gelte
    \begin{equation}
        \Delta(\psi)(k) = \int_G |\hat\psi(M^t k)|^2 |\det M| \d \mu(M) = 1
    \label{eq:admissibility criterion}
    \end{equation}

    für fast alle $k \in \hat{\mathbb{R}}^2$.
    Dann ist $(\psi, G\ltimes \mathbb{R}^n)$ ein reproduzierendes System für $L^2(\mathbb{R}^n)$, also gilt für alle $f \in L^2(\mathbb{R}^n)$

    \begin{equation}
        f = \int_{\mathbb{R}^n} \int_G \left\langle \psi_{M,t},f\right\rangle
            \psi_{M,t} \d \mu (M) \d t
    \end{equation}
\end{theorem}
In einer Dimension entspricht das Zulässigkeitskriterium genau Calderons Kriterium:
\begin{remark}[\ref{thm:admissibility_criterion} ist Calderon]
    Für $n=1$ ist $\mathrm{GL}(1,\mathbb{R}) = (\mathbb{R}^*, \cdot)$ mit dem Haarmaß $\d \mu(a) = \frac{\d \lambda(a)}{a}$ und \cref{eq:admissibility criterion} wird zu

    \begin{equation}
        \int_0^\infty \left\lvert \hat\psi(a k) \right\rvert^2 \frac{\d \lambda(a)}{a} = 1 \condition{für fast alle $k \in \hat{\mathbb{R}}$}
        \label{eq:calderon}
    \end{equation}
    \cref{eq:admissibility criterion} ist also das mehrdimensionale Analogon zu Calderons Kriterium \cite[S. 105]{Mallat2008}.
\end{remark}

Jetzt aber mehr Details zur Konstruktion der Shearlets und deren Eigenschaften:


%%%%%%%%%%%%%%%%%%%%%%%%%%%%%%%%%%%%%%%%%%%%%%%%%%%%%%%%%%%%%%%%%%%%%%%%%%%%%%%
% % Die Shearlets
%%%%%%%%%%%%%%%%%%%%%%%%%%%%%%%%%%%%%%%%%%%%%%%%%%%%%%%%%%%%%%%%%%%%%%%%%%%%%%%
\subsection{Konstruktion, Eigenschaften der Shearlets und ein wichtiger Satz} % (fold)
\label{sec:konstruktion_und_eigenschaften_der_shearlets}

Der folgende Abschnitt basiert größtenteils auf der Arbeit von \textcite{Kutyniok2008}.
 Da wir später auch komplexwertige Distributionen analysieren wollen, deren Wellenfrontmenge nicht zwingend punktsymmetrisch um den Ursprung (in der Richtung, nicht im Ort) sind, werden wir Shearlets verwenden, deren Fouriertransformierte asymetrischen Träger hat, indem wir die Shearlets aus \cite{Kutyniok2008} jeweils in zwei Shearlets aufteilen, eines mit Träger im Frequenzbereich "`nach vorne"', und eines mit Träger "`nach hinten"'. \todo{Grafik mit der "Parzellierung" des Frequenzbereiches}

\begin{definition}[Shearlettransformation]
\label{def:shearletttransformationI}
Seien
% \begin{dgroup*}
\begin{equation}
    \psi_1 \in L^2(\mathbb{R})
            \textrm{ mit }supp(\hat\psi_1) \subseteq \left[\frac{1}{2},2\right]
            \textrm{ und } \psi_1 \textrm{ erfülle \cref{eq:calderon}}
            \footnote{mit $G = (\mathbb{R}^*, \cdot), |\det M| d \lambda(M) = \frac{\d a}{a}$}
\label{eq:psi_1}
\end{equation}
\begin{equation}
    \psi_2 \in L^2(\mathbb{R})
            \textrm{ mit } supp(\hat\psi_1) \subseteq \left[-1,1\right]
            \textrm{ und } \left\Vert\psi_2 \right\Vert_2 = 1
\label{eq:psi_2}
\end{equation}
% \end{dgroup*}
und $\psi \in C^\infty(\mathbb{R}^2)$ implizit definiert durch

\begin{equation}
    \hat \psi(k_1,k_2) = \hat\psi_1(k_1) \hat \psi_2 \left(\frac{k_2}{k_1}\right).
\end{equation}

Sei des weiteren

\begin{equation}
    G = \left\{M_{as} \in \mathrm{GL}(2,\mathbb{R}) ~\Big|~ M_{as} = \left(\begin{smallmatrix}
        a & -\sqrt a s \\ 0 & \sqrt a
    \end{smallmatrix}\right)
    , a \in [0,1], s \in [-2,2]
    \right\}
    \label{eq:schermatrizen}
\end{equation}

Dann ist für $t \in \mathbb{R}^n, M_{as} \in G$ die Shearlettransformation von $f$ bezüglich $\psi$ definiert als

\begin{equation}
    \mathcal{S}_f(a,s,t)) =
    \left\langle D_{M_{as}}T_{t}\psi, f
    \right\rangle
    = a^{-\frac{3}{4}}\int f(x) \psi\left(
    \left(\begin{smallmatrix}
        a & -\sqrt a s \\ 0 & \sqrt a
    \end{smallmatrix}\right)^{-1}
    \left(\begin{smallmatrix}
        x-t
    \end{smallmatrix}\right)
    \right) \d x
\end{equation}

\end{definition}

Bevor es mit dem Text weiter geht, noch eine kurze Bemerkung zu Vereinfachung der Notation:

\begin{remark}[Notation]
    Der Kompaktheit halber schreiben wir auch
    \begin{equation*}
        \psi_{ast} (x_1,x_2) \coloneqq
        a^{-\frac{3}{4}} \psi\left(
    \left(\begin{smallmatrix}
        a & -\sqrt a s \\ 0 & \sqrt a
    \end{smallmatrix}\right)^{-1}
    \left(\begin{smallmatrix}
        x-t
    \end{smallmatrix}\right)
    \right)
    \end{equation*}
\end{remark}

Offenbar können $\psi_1$ und $\psi_2$ in \cref{def:shearletttransformationI} problemlos auch so gewählt werden, dass $\hat\psi_1$ und $\hat\psi_2$ glatt sind; wir stellen in \cref{eq:psi_1,eq:psi_2} ja keine allzu restriktiven Anforderungen an sie. Dann ist $\psi_{ast}$ eine Schwartz-Funktion für alle $(a,s,t)$ und die Shearlettransformation temperierter Distributionen ist wohldefiniert.  Die Anforderungen aus \cref{eq:psi_1,eq:psi_2} sind genau so gewählt, dass \cref{eq:admissibility criterion} von $\psi$ erfüllt wird, und gleichzeitig die konkreten Rechnungen zur Bestimmung der Wellenfrontmenge auch analytisch möglich sind.

Der kompakte Träger von $\hat\psi$ in der Frequenzdomäne erlaubt einfachere
Abschätzungen von Ausdrücken der Form $\left<\hat f, \hat \psi_{ast}\right>$, ist aber m.E. \emph{nicht} zwingend notwendig, um mit diesem Shearlet die Wellenfrontmenge zu bestimmen.


Die Wirkung der Scher- und Skalierungsmatrizen aus \cref{eq:schermatrizen} versteht man am besten in der Frequenzdomäne. Mit $a \to 0$ wird $\hat \psi$ immer weiter "`weg vom Ursprung"' geschoben in der Frequenzdomäne und der Träger liegt gleichzeitig in immer engeren Kegeln. Dies ist genau die Anisotropie, die uns erlaubt, nicht nur die Position der Singularitäten, sondern auch ihre Orientierung zu erkennen. Der Parameter $s$ bestimmt die Scherung des Trägers von $\psi$. Für $s=0$ ist der Träger um $k=0$ herum lokalisiert, für $s = \pm 1$ um die Diagonale bzw Antidiagonale. Das ist dargestellt in \cref{fig:supp_psi_hat} und \cref{rem:psi_hat}.


\begin{remark}[Eigenschaften von $\hat\psi_{ast}$]
\label{rem:psi_hat}
\begin{figure}[h]
\centering
%% Creator: Matplotlib, PGF backend
%%
%% To include the figure in your LaTeX document, write
%%   \input{<filename>.pgf}
%%
%% Make sure the required packages are loaded in your preamble
%%   \usepackage{pgf}
%%
%% Figures using additional raster images can only be included by \input if
%% they are in the same directory as the main LaTeX file. For loading figures
%% from other directories you can use the `import` package
%%   \usepackage{import}
%% and then include the figures with
%%   \import{<path to file>}{<filename>.pgf}
%%
%% Matplotlib used the following preamble
%%   \usepackage[utf8x]{inputenc}
%%   \usepackage[T1]{fontenc}
%%   \usepackage{amssymb}
%%
\begingroup%
\makeatletter%
\begin{pgfpicture}%
\pgfpathrectangle{\pgfpointorigin}{\pgfqpoint{4.000000in}{2.000000in}}%
\pgfusepath{use as bounding box, clip}%
\begin{pgfscope}%
\pgfsetbuttcap%
\pgfsetmiterjoin%
\definecolor{currentfill}{rgb}{1.000000,1.000000,1.000000}%
\pgfsetfillcolor{currentfill}%
\pgfsetlinewidth{0.000000pt}%
\definecolor{currentstroke}{rgb}{1.000000,1.000000,1.000000}%
\pgfsetstrokecolor{currentstroke}%
\pgfsetdash{}{0pt}%
\pgfpathmoveto{\pgfqpoint{0.000000in}{0.000000in}}%
\pgfpathlineto{\pgfqpoint{4.000000in}{0.000000in}}%
\pgfpathlineto{\pgfqpoint{4.000000in}{2.000000in}}%
\pgfpathlineto{\pgfqpoint{0.000000in}{2.000000in}}%
\pgfpathclose%
\pgfusepath{fill}%
\end{pgfscope}%
\begin{pgfscope}%
\pgfsetbuttcap%
\pgfsetmiterjoin%
\definecolor{currentfill}{rgb}{1.000000,1.000000,1.000000}%
\pgfsetfillcolor{currentfill}%
\pgfsetlinewidth{0.000000pt}%
\definecolor{currentstroke}{rgb}{0.000000,0.000000,0.000000}%
\pgfsetstrokecolor{currentstroke}%
\pgfsetstrokeopacity{0.000000}%
\pgfsetdash{}{0pt}%
\pgfpathmoveto{\pgfqpoint{0.500000in}{0.250000in}}%
\pgfpathlineto{\pgfqpoint{3.600000in}{0.250000in}}%
\pgfpathlineto{\pgfqpoint{3.600000in}{1.760000in}}%
\pgfpathlineto{\pgfqpoint{0.500000in}{1.760000in}}%
\pgfpathclose%
\pgfusepath{fill}%
\end{pgfscope}%
\begin{pgfscope}%
\pgfpathrectangle{\pgfqpoint{0.500000in}{0.250000in}}{\pgfqpoint{3.100000in}{1.510000in}} %
\pgfusepath{clip}%
\pgfsetbuttcap%
\pgfsetmiterjoin%
\definecolor{currentfill}{rgb}{0.500000,0.500000,0.500000}%
\pgfsetfillcolor{currentfill}%
\pgfsetfillopacity{0.500000}%
\pgfsetlinewidth{0.501875pt}%
\definecolor{currentstroke}{rgb}{0.000000,0.000000,0.000000}%
\pgfsetstrokecolor{currentstroke}%
\pgfsetdash{}{0pt}%
\pgfpathmoveto{\pgfqpoint{2.011250in}{0.438750in}}%
\pgfpathlineto{\pgfqpoint{2.088750in}{0.438750in}}%
\pgfpathlineto{\pgfqpoint{2.205000in}{0.552000in}}%
\pgfpathlineto{\pgfqpoint{1.895000in}{0.552000in}}%
\pgfpathclose%
\pgfusepath{stroke,fill}%
\end{pgfscope}%
\begin{pgfscope}%
\pgfpathrectangle{\pgfqpoint{0.500000in}{0.250000in}}{\pgfqpoint{3.100000in}{1.510000in}} %
\pgfusepath{clip}%
\pgfsetbuttcap%
\pgfsetmiterjoin%
\definecolor{currentfill}{rgb}{0.500000,0.500000,0.500000}%
\pgfsetfillcolor{currentfill}%
\pgfsetfillopacity{0.500000}%
\pgfsetlinewidth{0.501875pt}%
\definecolor{currentstroke}{rgb}{0.000000,0.000000,0.000000}%
\pgfsetstrokecolor{currentstroke}%
\pgfsetdash{}{0pt}%
\pgfpathmoveto{\pgfqpoint{1.949948in}{0.652667in}}%
\pgfpathlineto{\pgfqpoint{2.150052in}{0.652667in}}%
\pgfpathlineto{\pgfqpoint{2.450208in}{1.407667in}}%
\pgfpathlineto{\pgfqpoint{1.649792in}{1.407667in}}%
\pgfpathclose%
\pgfusepath{stroke,fill}%
\end{pgfscope}%
\begin{pgfscope}%
\pgfpathrectangle{\pgfqpoint{0.500000in}{0.250000in}}{\pgfqpoint{3.100000in}{1.510000in}} %
\pgfusepath{clip}%
\pgfsetbuttcap%
\pgfsetmiterjoin%
\definecolor{currentfill}{rgb}{0.500000,0.500000,0.500000}%
\pgfsetfillcolor{currentfill}%
\pgfsetfillopacity{0.500000}%
\pgfsetlinewidth{0.501875pt}%
\definecolor{currentstroke}{rgb}{0.000000,0.000000,0.000000}%
\pgfsetstrokecolor{currentstroke}%
\pgfsetdash{}{0pt}%
\pgfpathmoveto{\pgfqpoint{2.208281in}{0.652667in}}%
\pgfpathlineto{\pgfqpoint{2.408385in}{0.652667in}}%
\pgfpathlineto{\pgfqpoint{3.483542in}{1.407667in}}%
\pgfpathlineto{\pgfqpoint{2.683125in}{1.407667in}}%
\pgfpathclose%
\pgfusepath{stroke,fill}%
\end{pgfscope}%
\begin{pgfscope}%
\pgfpathrectangle{\pgfqpoint{0.500000in}{0.250000in}}{\pgfqpoint{3.100000in}{1.510000in}} %
\pgfusepath{clip}%
\pgfsetbuttcap%
\pgfsetroundjoin%
\pgfsetlinewidth{0.501875pt}%
\definecolor{currentstroke}{rgb}{0.501961,0.501961,0.501961}%
\pgfsetstrokecolor{currentstroke}%
\pgfsetdash{{1.850000pt}{0.800000pt}}{0.000000pt}%
\pgfpathmoveto{\pgfqpoint{1.880743in}{0.236111in}}%
\pgfpathlineto{\pgfqpoint{3.459257in}{1.773889in}}%
\pgfpathlineto{\pgfqpoint{3.459257in}{1.773889in}}%
\pgfusepath{stroke}%
\end{pgfscope}%
\begin{pgfscope}%
\pgfpathrectangle{\pgfqpoint{0.500000in}{0.250000in}}{\pgfqpoint{3.100000in}{1.510000in}} %
\pgfusepath{clip}%
\pgfsetbuttcap%
\pgfsetroundjoin%
\pgfsetlinewidth{0.501875pt}%
\definecolor{currentstroke}{rgb}{0.501961,0.501961,0.501961}%
\pgfsetstrokecolor{currentstroke}%
\pgfsetdash{{1.850000pt}{0.800000pt}}{0.000000pt}%
\pgfpathmoveto{\pgfqpoint{0.640743in}{1.773889in}}%
\pgfpathlineto{\pgfqpoint{2.219257in}{0.236111in}}%
\pgfpathlineto{\pgfqpoint{2.219257in}{0.236111in}}%
\pgfusepath{stroke}%
\end{pgfscope}%
\begin{pgfscope}%
\pgfsetrectcap%
\pgfsetmiterjoin%
\pgfsetlinewidth{0.501875pt}%
\definecolor{currentstroke}{rgb}{0.000000,0.000000,0.000000}%
\pgfsetstrokecolor{currentstroke}%
\pgfsetdash{}{0pt}%
\pgfpathmoveto{\pgfqpoint{2.050000in}{0.250000in}}%
\pgfpathlineto{\pgfqpoint{2.050000in}{1.760000in}}%
\pgfusepath{stroke}%
\end{pgfscope}%
\begin{pgfscope}%
\pgfsetrectcap%
\pgfsetmiterjoin%
\pgfsetlinewidth{0.501875pt}%
\definecolor{currentstroke}{rgb}{0.000000,0.000000,0.000000}%
\pgfsetstrokecolor{currentstroke}%
\pgfsetdash{}{0pt}%
\pgfpathmoveto{\pgfqpoint{0.500000in}{0.401000in}}%
\pgfpathlineto{\pgfqpoint{3.600000in}{0.401000in}}%
\pgfusepath{stroke}%
\end{pgfscope}%
\begin{pgfscope}%
\pgfsetroundcap%
\pgfsetroundjoin%
\pgfsetlinewidth{0.501875pt}%
\definecolor{currentstroke}{rgb}{0.000000,0.000000,0.000000}%
\pgfsetstrokecolor{currentstroke}%
\pgfsetdash{}{0pt}%
\pgfpathmoveto{\pgfqpoint{1.652485in}{0.528210in}}%
\pgfpathquadraticcurveto{\pgfqpoint{1.779212in}{0.521920in}}{\pgfqpoint{1.898185in}{0.516015in}}%
\pgfusepath{stroke}%
\end{pgfscope}%
\begin{pgfscope}%
\pgfsetroundcap%
\pgfsetroundjoin%
\pgfsetlinewidth{0.501875pt}%
\definecolor{currentstroke}{rgb}{0.000000,0.000000,0.000000}%
\pgfsetstrokecolor{currentstroke}%
\pgfsetdash{}{0pt}%
\pgfpathmoveto{\pgfqpoint{1.844075in}{0.546513in}}%
\pgfpathlineto{\pgfqpoint{1.898185in}{0.516015in}}%
\pgfpathlineto{\pgfqpoint{1.841321in}{0.491026in}}%
\pgfusepath{stroke}%
\end{pgfscope}%
\begin{pgfscope}%
\pgftext[x=0.887500in,y=0.514250in,left,base]{\rmfamily\fontsize{10.000000}{12.000000}\selectfont \(\displaystyle a = 1, s = 0\)}%
\end{pgfscope}%
\begin{pgfscope}%
\pgfsetroundcap%
\pgfsetroundjoin%
\pgfsetlinewidth{0.501875pt}%
\definecolor{currentstroke}{rgb}{0.000000,0.000000,0.000000}%
\pgfsetstrokecolor{currentstroke}%
\pgfsetdash{}{0pt}%
\pgfpathmoveto{\pgfqpoint{1.442466in}{1.092221in}}%
\pgfpathquadraticcurveto{\pgfqpoint{1.557990in}{1.086989in}}{\pgfqpoint{1.665758in}{1.082108in}}%
\pgfusepath{stroke}%
\end{pgfscope}%
\begin{pgfscope}%
\pgfsetroundcap%
\pgfsetroundjoin%
\pgfsetlinewidth{0.501875pt}%
\definecolor{currentstroke}{rgb}{0.000000,0.000000,0.000000}%
\pgfsetstrokecolor{currentstroke}%
\pgfsetdash{}{0pt}%
\pgfpathmoveto{\pgfqpoint{1.611516in}{1.112371in}}%
\pgfpathlineto{\pgfqpoint{1.665758in}{1.082108in}}%
\pgfpathlineto{\pgfqpoint{1.609002in}{1.056872in}}%
\pgfusepath{stroke}%
\end{pgfscope}%
\begin{pgfscope}%
\pgftext[x=0.500000in,y=1.080500in,left,base]{\rmfamily\fontsize{10.000000}{12.000000}\selectfont \(\displaystyle a = 0.15, s = 0\)}%
\end{pgfscope}%
\begin{pgfscope}%
\pgfsetroundcap%
\pgfsetroundjoin%
\pgfsetlinewidth{0.501875pt}%
\definecolor{currentstroke}{rgb}{0.000000,0.000000,0.000000}%
\pgfsetstrokecolor{currentstroke}%
\pgfsetdash{}{0pt}%
\pgfpathmoveto{\pgfqpoint{3.243990in}{0.689247in}}%
\pgfpathquadraticcurveto{\pgfqpoint{3.046540in}{0.802467in}}{\pgfqpoint{2.855825in}{0.911824in}}%
\pgfusepath{stroke}%
\end{pgfscope}%
\begin{pgfscope}%
\pgfsetroundcap%
\pgfsetroundjoin%
\pgfsetlinewidth{0.501875pt}%
\definecolor{currentstroke}{rgb}{0.000000,0.000000,0.000000}%
\pgfsetstrokecolor{currentstroke}%
\pgfsetdash{}{0pt}%
\pgfpathmoveto{\pgfqpoint{2.890202in}{0.860092in}}%
\pgfpathlineto{\pgfqpoint{2.855825in}{0.911824in}}%
\pgfpathlineto{\pgfqpoint{2.917837in}{0.908287in}}%
\pgfusepath{stroke}%
\end{pgfscope}%
\begin{pgfscope}%
\pgftext[x=2.980000in,y=0.552000in,left,base]{\rmfamily\fontsize{10.000000}{12.000000}\selectfont \(\displaystyle a = 0.15, s = 1\)}%
\end{pgfscope}%
\begin{pgfscope}%
\pgfsetroundcap%
\pgfsetroundjoin%
\pgfsetlinewidth{0.501875pt}%
\definecolor{currentstroke}{rgb}{0.000000,0.000000,0.000000}%
\pgfsetstrokecolor{currentstroke}%
\pgfsetdash{}{0pt}%
\pgfpathmoveto{\pgfqpoint{2.050000in}{1.766125in}}%
\pgfpathquadraticcurveto{\pgfqpoint{2.050000in}{1.766944in}}{\pgfqpoint{2.050000in}{1.760000in}}%
\pgfusepath{stroke}%
\end{pgfscope}%
\begin{pgfscope}%
\pgfsetroundcap%
\pgfsetroundjoin%
\pgfsetlinewidth{0.501875pt}%
\definecolor{currentstroke}{rgb}{0.000000,0.000000,0.000000}%
\pgfsetstrokecolor{currentstroke}%
\pgfsetdash{}{0pt}%
\pgfpathmoveto{\pgfqpoint{2.022222in}{1.710569in}}%
\pgfpathlineto{\pgfqpoint{2.050000in}{1.766125in}}%
\pgfpathlineto{\pgfqpoint{2.077778in}{1.710569in}}%
\pgfusepath{stroke}%
\end{pgfscope}%
\begin{pgfscope}%
\pgftext[x=2.050000in,y=1.829444in,,bottom]{\rmfamily\fontsize{10.000000}{12.000000}\selectfont \(\displaystyle k_1 / \omega\)}%
\end{pgfscope}%
\begin{pgfscope}%
\pgfsetroundcap%
\pgfsetroundjoin%
\pgfsetlinewidth{0.501875pt}%
\definecolor{currentstroke}{rgb}{0.000000,0.000000,0.000000}%
\pgfsetstrokecolor{currentstroke}%
\pgfsetdash{}{0pt}%
\pgfpathmoveto{\pgfqpoint{3.606104in}{0.401000in}}%
\pgfpathquadraticcurveto{\pgfqpoint{3.606934in}{0.401000in}}{\pgfqpoint{3.600000in}{0.401000in}}%
\pgfusepath{stroke}%
\end{pgfscope}%
\begin{pgfscope}%
\pgfsetroundcap%
\pgfsetroundjoin%
\pgfsetlinewidth{0.501875pt}%
\definecolor{currentstroke}{rgb}{0.000000,0.000000,0.000000}%
\pgfsetstrokecolor{currentstroke}%
\pgfsetdash{}{0pt}%
\pgfpathmoveto{\pgfqpoint{3.550548in}{0.428778in}}%
\pgfpathlineto{\pgfqpoint{3.606104in}{0.401000in}}%
\pgfpathlineto{\pgfqpoint{3.550548in}{0.373222in}}%
\pgfusepath{stroke}%
\end{pgfscope}%
\begin{pgfscope}%
\pgftext[x=3.669444in,y=0.401000in,left,]{\rmfamily\fontsize{10.000000}{12.000000}\selectfont \(\displaystyle k_2 / k\)}%
\end{pgfscope}%
\end{pgfpicture}%
\makeatother%
\endgroup%

\caption{Der Träger von $\hat \psi_{ast}$ für verschiedene $a, s$. Man sieht gut,
wie $supp (\hat \psi_{ast})$ für kleinere $a$ in immer spitzeren Kegeln liegt. Da wir in den konkreten Rechnungen später $(k_1, k_2) = (\omega, k)$ nennen werden und Minkowski-Diagramme üblicherweise mit $\omega$ auf der $y$-Achse dargestellt werden, haben wir hier beide Namen eingetragen und alles an der Diagonale gespiegelt.}
\label{fig:supp_psi_hat}
\end{figure}

\label{cor:psi_hat}
Im Fourierraum ist $\hat{\psi}_{ast}$ gegeben durch

\begin{equation}
    \hat \psi_{ast}{(k_1, k_2)} = a^{\frac{3}{4}}e^{-ikx}\hat\psi_1(a k_1) \hat\psi_{2}\left(a^{-\frac{1}{2}}\left(\frac{k_2}{k_1}-s\right)\right)
\label{eq:hat_psi_ast}
\end{equation}

und es gilt

\begin{equation}
\label{eq:supp_psi}
    supp(\hat \psi) \subset \left\{k \in  \hat{\mathbb{R}}^2 ~\Big| ~k_1 \in \left[\frac{1}{2 a} , \frac{2}{a}\right], \left|\frac{k_2}{k_1} - s\right| \leq \sqrt{a} \right\}
\end{equation}
\end{remark}

Eine weitere Eigenschaft, die aus dieser Definition der Shearlets folgt, ist der schnelle Abfall der Shearlets abseits von $t$:

\begin{proposition}[$\psi_{ast}$ fällt schnell ab]
\label{prop:shearlets_decay_rapidly}
Sei $\psi \in L^2(\mathbb{R}^2)$ ein Shearlet wie in \cref{def:shearletttransformationI} und $M_{as}$ wie in \cref{eq:schermatrizen}. Dann gilt für alle $N \in  \mathbb{N}$, dass es eine konstante $C_N$ gibt s.d. für alle $t \in \mathbb{R}^2$ gilt

\begin{dmath*}
    \left| \psi_{ast}(x_1,x_2) \right|
    \leq
    C_k \left| \det M_{as} \right|^{-\frac{1}{2}}\left(1+\left|M_{as}^{-1}
    \left(
    \begin{smallmatrix}
        x_1-t_1 \\ x_2-t_2
    \end{smallmatrix}\right)
    \right|^2\right)^{-N}
    = C_k a^{-\frac{3}{4}}\left(1+a^{-2}\left(x_1-t_1\right)^2
        + 2 a^{-2}s\left(x_1-t_1\right)\left(x_2-t_2\right)
        + a^{-1}\left(1+a^{-1}s^2\right)\left(x_2-t_2\right)^2
    \right)^{-N}
\end{dmath*}

Und insbesondere ist $C_N = \frac{15}{2}\frac{\sqrt{a} + s}{a^2}\left(\Vert \hat\psi \Vert_\infty + \Vert \Laplace^N \hat\psi \Vert_\infty\right)$

\end{proposition}


Wer bis hier aufmerksam mitgelesen hat und sich \cref{fig:supp_psi_hat} genau angeschaut hat, wird bemerkt haben, dass $supp(\hat\psi_{ast})$ für alle $a$ und $s$ quasi nur im Quadranten $x_1^2 \geq x_2^2$ und $x_2 \geq 0$ liegt.
Mit den Namen der Physik $(k_1, k_2) = (\omega, k)$ entspricht das dem "`Vorwärtslichtkegel"'. Glücklicherweise liegen alle analysierten Distributionen im gleich definierten $L(C)^\vee$
Wie der folgende Satz zeigt, erzeugt $\psi$ auch nur für solche $f$ ein reproduzierendes System.
\todo[color=green]{Die konkrete Konstruktion für die volle Transformation angeben, oder reicht der Hinweis, dass ich mit Spiegeln und Drehen alles sehen kann?}

\begin{theorem}[$\psi$ reproduziert $L^2(C)^\vee$]
\label{thm:shearlets_reproduzieren}
    Sei
    \begin{equation*}
        C \coloneqq \left\{(k_1,k_2)\in \hat{\mathbb{R}}^2
        \Big| k_1 \geq 2 \textrm{ und } \left\lvert\tfrac{k_2}{k_1}\right\rvert \leq 1
        \right\}
    \end{equation*}
    und
    \begin{equation}
        L^2(C)^\vee \coloneqq \{f \in L^2(\mathbb{R}^2) ~|~ supp (\hat f) \subset C\}
        \label{eq:L2_cone}
    \end{equation}

    Dann ist $\psi$ aus \cref{def:shearletttransformationI} ein reproduzierendes System für $L^2(C)^\vee$, also gilt für alle
    $f \in L^2(C)^\vee$:
    \begin{equation}
        f(x) = \int_{\mathbb{R}^2} \int_{-2}^2 \int_0^1
                \left\langle f,\psi_{ast}\right\rangle \psi_{ast}(x)
                \frac{\d a}{a^3} \d s \d t.
    \end{equation}
\end{theorem}

Um nun nicht nur ein reproduzierendes System für $L^2(C)^\vee$, sondern für ganz $L^2(\mathbb{R}^2)$ zu erhalten, muss $\hat \psi_{ast}$ noch in den rechten, linken und rückwärts liegenden Kegel gedreht und geschoben werden. Zusätzlich muss noch eine weitere Funktion $W$ gefunden werden, welche die groben Skalen (also $|k_1|, |k_2| \leq 2$) auflöst.

\begin{figure}[h]
\centering
%% Creator: Matplotlib, PGF backend
%%
%% To include the figure in your LaTeX document, write
%%   \input{<filename>.pgf}
%%
%% Make sure the required packages are loaded in your preamble
%%   \usepackage{pgf}
%%
%% Figures using additional raster images can only be included by \input if
%% they are in the same directory as the main LaTeX file. For loading figures
%% from other directories you can use the `import` package
%%   \usepackage{import}
%% and then include the figures with
%%   \import{<path to file>}{<filename>.pgf}
%%
%% Matplotlib used the following preamble
%%   \usepackage[utf8x]{inputenc}
%%   \usepackage[T1]{fontenc}
%%   \usepackage{amssymb}
%%
\begingroup%
\makeatletter%
\begin{pgfpicture}%
\pgfpathrectangle{\pgfpointorigin}{\pgfqpoint{3.000000in}{3.000000in}}%
\pgfusepath{use as bounding box, clip}%
\begin{pgfscope}%
\pgfsetbuttcap%
\pgfsetmiterjoin%
\definecolor{currentfill}{rgb}{1.000000,1.000000,1.000000}%
\pgfsetfillcolor{currentfill}%
\pgfsetlinewidth{0.000000pt}%
\definecolor{currentstroke}{rgb}{1.000000,1.000000,1.000000}%
\pgfsetstrokecolor{currentstroke}%
\pgfsetdash{}{0pt}%
\pgfpathmoveto{\pgfqpoint{0.000000in}{0.000000in}}%
\pgfpathlineto{\pgfqpoint{3.000000in}{0.000000in}}%
\pgfpathlineto{\pgfqpoint{3.000000in}{3.000000in}}%
\pgfpathlineto{\pgfqpoint{0.000000in}{3.000000in}}%
\pgfpathclose%
\pgfusepath{fill}%
\end{pgfscope}%
\begin{pgfscope}%
\pgfsetbuttcap%
\pgfsetmiterjoin%
\definecolor{currentfill}{rgb}{1.000000,1.000000,1.000000}%
\pgfsetfillcolor{currentfill}%
\pgfsetlinewidth{0.000000pt}%
\definecolor{currentstroke}{rgb}{0.000000,0.000000,0.000000}%
\pgfsetstrokecolor{currentstroke}%
\pgfsetstrokeopacity{0.000000}%
\pgfsetdash{}{0pt}%
\pgfpathmoveto{\pgfqpoint{0.198611in}{0.198611in}}%
\pgfpathlineto{\pgfqpoint{2.801389in}{0.198611in}}%
\pgfpathlineto{\pgfqpoint{2.801389in}{2.801389in}}%
\pgfpathlineto{\pgfqpoint{0.198611in}{2.801389in}}%
\pgfpathclose%
\pgfusepath{fill}%
\end{pgfscope}%
\begin{pgfscope}%
\pgfpathrectangle{\pgfqpoint{0.198611in}{0.198611in}}{\pgfqpoint{2.602778in}{2.602778in}} %
\pgfusepath{clip}%
\pgfsetbuttcap%
\pgfsetroundjoin%
\pgfsetlinewidth{0.501875pt}%
\definecolor{currentstroke}{rgb}{0.501961,0.501961,0.501961}%
\pgfsetstrokecolor{currentstroke}%
\pgfsetdash{{1.850000pt}{0.800000pt}}{0.000000pt}%
\pgfpathmoveto{\pgfqpoint{1.066204in}{1.066204in}}%
\pgfpathlineto{\pgfqpoint{1.066204in}{1.933796in}}%
\pgfusepath{stroke}%
\end{pgfscope}%
\begin{pgfscope}%
\pgfpathrectangle{\pgfqpoint{0.198611in}{0.198611in}}{\pgfqpoint{2.602778in}{2.602778in}} %
\pgfusepath{clip}%
\pgfsetbuttcap%
\pgfsetroundjoin%
\pgfsetlinewidth{0.501875pt}%
\definecolor{currentstroke}{rgb}{0.501961,0.501961,0.501961}%
\pgfsetstrokecolor{currentstroke}%
\pgfsetdash{{1.850000pt}{0.800000pt}}{0.000000pt}%
\pgfpathmoveto{\pgfqpoint{1.933796in}{1.066204in}}%
\pgfpathlineto{\pgfqpoint{1.933796in}{1.933796in}}%
\pgfusepath{stroke}%
\end{pgfscope}%
\begin{pgfscope}%
\pgfpathrectangle{\pgfqpoint{0.198611in}{0.198611in}}{\pgfqpoint{2.602778in}{2.602778in}} %
\pgfusepath{clip}%
\pgfsetbuttcap%
\pgfsetroundjoin%
\pgfsetlinewidth{0.501875pt}%
\definecolor{currentstroke}{rgb}{0.501961,0.501961,0.501961}%
\pgfsetstrokecolor{currentstroke}%
\pgfsetdash{{1.850000pt}{0.800000pt}}{0.000000pt}%
\pgfpathmoveto{\pgfqpoint{1.933796in}{1.933796in}}%
\pgfpathlineto{\pgfqpoint{1.979459in}{1.979459in}}%
\pgfpathlineto{\pgfqpoint{2.025122in}{2.025122in}}%
\pgfpathlineto{\pgfqpoint{2.070785in}{2.070785in}}%
\pgfpathlineto{\pgfqpoint{2.116447in}{2.116447in}}%
\pgfpathlineto{\pgfqpoint{2.162110in}{2.162110in}}%
\pgfpathlineto{\pgfqpoint{2.207773in}{2.207773in}}%
\pgfpathlineto{\pgfqpoint{2.253436in}{2.253436in}}%
\pgfpathlineto{\pgfqpoint{2.299098in}{2.299098in}}%
\pgfpathlineto{\pgfqpoint{2.344761in}{2.344761in}}%
\pgfpathlineto{\pgfqpoint{2.390424in}{2.390424in}}%
\pgfpathlineto{\pgfqpoint{2.436087in}{2.436087in}}%
\pgfpathlineto{\pgfqpoint{2.481750in}{2.481750in}}%
\pgfpathlineto{\pgfqpoint{2.527412in}{2.527412in}}%
\pgfpathlineto{\pgfqpoint{2.573075in}{2.573075in}}%
\pgfpathlineto{\pgfqpoint{2.618738in}{2.618738in}}%
\pgfpathlineto{\pgfqpoint{2.664401in}{2.664401in}}%
\pgfpathlineto{\pgfqpoint{2.710063in}{2.710063in}}%
\pgfpathlineto{\pgfqpoint{2.755726in}{2.755726in}}%
\pgfpathlineto{\pgfqpoint{2.801389in}{2.801389in}}%
\pgfusepath{stroke}%
\end{pgfscope}%
\begin{pgfscope}%
\pgfpathrectangle{\pgfqpoint{0.198611in}{0.198611in}}{\pgfqpoint{2.602778in}{2.602778in}} %
\pgfusepath{clip}%
\pgfsetbuttcap%
\pgfsetroundjoin%
\pgfsetlinewidth{0.501875pt}%
\definecolor{currentstroke}{rgb}{0.501961,0.501961,0.501961}%
\pgfsetstrokecolor{currentstroke}%
\pgfsetdash{{1.850000pt}{0.800000pt}}{0.000000pt}%
\pgfpathmoveto{\pgfqpoint{1.933796in}{1.066204in}}%
\pgfpathlineto{\pgfqpoint{1.979459in}{1.020541in}}%
\pgfpathlineto{\pgfqpoint{2.025122in}{0.974878in}}%
\pgfpathlineto{\pgfqpoint{2.070785in}{0.929215in}}%
\pgfpathlineto{\pgfqpoint{2.116447in}{0.883553in}}%
\pgfpathlineto{\pgfqpoint{2.162110in}{0.837890in}}%
\pgfpathlineto{\pgfqpoint{2.207773in}{0.792227in}}%
\pgfpathlineto{\pgfqpoint{2.253436in}{0.746564in}}%
\pgfpathlineto{\pgfqpoint{2.299098in}{0.700902in}}%
\pgfpathlineto{\pgfqpoint{2.344761in}{0.655239in}}%
\pgfpathlineto{\pgfqpoint{2.390424in}{0.609576in}}%
\pgfpathlineto{\pgfqpoint{2.436087in}{0.563913in}}%
\pgfpathlineto{\pgfqpoint{2.481750in}{0.518250in}}%
\pgfpathlineto{\pgfqpoint{2.527412in}{0.472588in}}%
\pgfpathlineto{\pgfqpoint{2.573075in}{0.426925in}}%
\pgfpathlineto{\pgfqpoint{2.618738in}{0.381262in}}%
\pgfpathlineto{\pgfqpoint{2.664401in}{0.335599in}}%
\pgfpathlineto{\pgfqpoint{2.710063in}{0.289937in}}%
\pgfpathlineto{\pgfqpoint{2.755726in}{0.244274in}}%
\pgfpathlineto{\pgfqpoint{2.801389in}{0.198611in}}%
\pgfusepath{stroke}%
\end{pgfscope}%
\begin{pgfscope}%
\pgfpathrectangle{\pgfqpoint{0.198611in}{0.198611in}}{\pgfqpoint{2.602778in}{2.602778in}} %
\pgfusepath{clip}%
\pgfsetbuttcap%
\pgfsetroundjoin%
\pgfsetlinewidth{0.501875pt}%
\definecolor{currentstroke}{rgb}{0.501961,0.501961,0.501961}%
\pgfsetstrokecolor{currentstroke}%
\pgfsetdash{{1.850000pt}{0.800000pt}}{0.000000pt}%
\pgfpathmoveto{\pgfqpoint{0.198611in}{0.198611in}}%
\pgfpathlineto{\pgfqpoint{0.244274in}{0.244274in}}%
\pgfpathlineto{\pgfqpoint{0.289937in}{0.289937in}}%
\pgfpathlineto{\pgfqpoint{0.335599in}{0.335599in}}%
\pgfpathlineto{\pgfqpoint{0.381262in}{0.381262in}}%
\pgfpathlineto{\pgfqpoint{0.426925in}{0.426925in}}%
\pgfpathlineto{\pgfqpoint{0.472588in}{0.472588in}}%
\pgfpathlineto{\pgfqpoint{0.518250in}{0.518250in}}%
\pgfpathlineto{\pgfqpoint{0.563913in}{0.563913in}}%
\pgfpathlineto{\pgfqpoint{0.609576in}{0.609576in}}%
\pgfpathlineto{\pgfqpoint{0.655239in}{0.655239in}}%
\pgfpathlineto{\pgfqpoint{0.700902in}{0.700902in}}%
\pgfpathlineto{\pgfqpoint{0.746564in}{0.746564in}}%
\pgfpathlineto{\pgfqpoint{0.792227in}{0.792227in}}%
\pgfpathlineto{\pgfqpoint{0.837890in}{0.837890in}}%
\pgfpathlineto{\pgfqpoint{0.883553in}{0.883553in}}%
\pgfpathlineto{\pgfqpoint{0.929215in}{0.929215in}}%
\pgfpathlineto{\pgfqpoint{0.974878in}{0.974878in}}%
\pgfpathlineto{\pgfqpoint{1.020541in}{1.020541in}}%
\pgfpathlineto{\pgfqpoint{1.066204in}{1.066204in}}%
\pgfusepath{stroke}%
\end{pgfscope}%
\begin{pgfscope}%
\pgfpathrectangle{\pgfqpoint{0.198611in}{0.198611in}}{\pgfqpoint{2.602778in}{2.602778in}} %
\pgfusepath{clip}%
\pgfsetbuttcap%
\pgfsetroundjoin%
\pgfsetlinewidth{0.501875pt}%
\definecolor{currentstroke}{rgb}{0.501961,0.501961,0.501961}%
\pgfsetstrokecolor{currentstroke}%
\pgfsetdash{{1.850000pt}{0.800000pt}}{0.000000pt}%
\pgfpathmoveto{\pgfqpoint{0.198611in}{2.801389in}}%
\pgfpathlineto{\pgfqpoint{0.244274in}{2.755726in}}%
\pgfpathlineto{\pgfqpoint{0.289937in}{2.710063in}}%
\pgfpathlineto{\pgfqpoint{0.335599in}{2.664401in}}%
\pgfpathlineto{\pgfqpoint{0.381262in}{2.618738in}}%
\pgfpathlineto{\pgfqpoint{0.426925in}{2.573075in}}%
\pgfpathlineto{\pgfqpoint{0.472588in}{2.527412in}}%
\pgfpathlineto{\pgfqpoint{0.518250in}{2.481750in}}%
\pgfpathlineto{\pgfqpoint{0.563913in}{2.436087in}}%
\pgfpathlineto{\pgfqpoint{0.609576in}{2.390424in}}%
\pgfpathlineto{\pgfqpoint{0.655239in}{2.344761in}}%
\pgfpathlineto{\pgfqpoint{0.700902in}{2.299098in}}%
\pgfpathlineto{\pgfqpoint{0.746564in}{2.253436in}}%
\pgfpathlineto{\pgfqpoint{0.792227in}{2.207773in}}%
\pgfpathlineto{\pgfqpoint{0.837890in}{2.162110in}}%
\pgfpathlineto{\pgfqpoint{0.883553in}{2.116447in}}%
\pgfpathlineto{\pgfqpoint{0.929215in}{2.070785in}}%
\pgfpathlineto{\pgfqpoint{0.974878in}{2.025122in}}%
\pgfpathlineto{\pgfqpoint{1.020541in}{1.979459in}}%
\pgfpathlineto{\pgfqpoint{1.066204in}{1.933796in}}%
\pgfusepath{stroke}%
\end{pgfscope}%
\begin{pgfscope}%
\pgfpathrectangle{\pgfqpoint{0.198611in}{0.198611in}}{\pgfqpoint{2.602778in}{2.602778in}} %
\pgfusepath{clip}%
\pgfsetbuttcap%
\pgfsetroundjoin%
\pgfsetlinewidth{0.501875pt}%
\definecolor{currentstroke}{rgb}{0.501961,0.501961,0.501961}%
\pgfsetstrokecolor{currentstroke}%
\pgfsetdash{{1.850000pt}{0.800000pt}}{0.000000pt}%
\pgfpathmoveto{\pgfqpoint{1.066204in}{1.933796in}}%
\pgfpathlineto{\pgfqpoint{1.933796in}{1.933796in}}%
\pgfusepath{stroke}%
\end{pgfscope}%
\begin{pgfscope}%
\pgfpathrectangle{\pgfqpoint{0.198611in}{0.198611in}}{\pgfqpoint{2.602778in}{2.602778in}} %
\pgfusepath{clip}%
\pgfsetbuttcap%
\pgfsetroundjoin%
\pgfsetlinewidth{0.501875pt}%
\definecolor{currentstroke}{rgb}{0.501961,0.501961,0.501961}%
\pgfsetstrokecolor{currentstroke}%
\pgfsetdash{{1.850000pt}{0.800000pt}}{0.000000pt}%
\pgfpathmoveto{\pgfqpoint{1.066204in}{1.066204in}}%
\pgfpathlineto{\pgfqpoint{1.933796in}{1.066204in}}%
\pgfusepath{stroke}%
\end{pgfscope}%
\begin{pgfscope}%
\pgfsetrectcap%
\pgfsetmiterjoin%
\pgfsetlinewidth{0.501875pt}%
\definecolor{currentstroke}{rgb}{0.000000,0.000000,0.000000}%
\pgfsetstrokecolor{currentstroke}%
\pgfsetdash{}{0pt}%
\pgfpathmoveto{\pgfqpoint{1.500000in}{0.198611in}}%
\pgfpathlineto{\pgfqpoint{1.500000in}{2.801389in}}%
\pgfusepath{stroke}%
\end{pgfscope}%
\begin{pgfscope}%
\pgfsetrectcap%
\pgfsetmiterjoin%
\pgfsetlinewidth{0.501875pt}%
\definecolor{currentstroke}{rgb}{0.000000,0.000000,0.000000}%
\pgfsetstrokecolor{currentstroke}%
\pgfsetdash{}{0pt}%
\pgfpathmoveto{\pgfqpoint{0.198611in}{1.500000in}}%
\pgfpathlineto{\pgfqpoint{2.801389in}{1.500000in}}%
\pgfusepath{stroke}%
\end{pgfscope}%
\begin{pgfscope}%
\pgftext[x=1.565069in,y=2.367593in,left,base]{\rmfamily\fontsize{10.000000}{12.000000}\selectfont \(\displaystyle C = C^{(1)}\)}%
\end{pgfscope}%
\begin{pgfscope}%
\pgftext[x=0.632407in,y=1.565069in,left,base]{\rmfamily\fontsize{10.000000}{12.000000}\selectfont \(\displaystyle C^{(2)}\)}%
\end{pgfscope}%
\begin{pgfscope}%
\pgftext[x=1.565069in,y=0.632407in,left,base]{\rmfamily\fontsize{10.000000}{12.000000}\selectfont \(\displaystyle C^{(3)}\)}%
\end{pgfscope}%
\begin{pgfscope}%
\pgftext[x=2.259144in,y=1.565069in,left,base]{\rmfamily\fontsize{10.000000}{12.000000}\selectfont \(\displaystyle C^{(4)}\)}%
\end{pgfscope}%
\begin{pgfscope}%
\pgftext[x=1.152963in,y=1.565069in,left,base]{\rmfamily\fontsize{10.000000}{12.000000}\selectfont grobe Skalen}%
\end{pgfscope}%
\begin{pgfscope}%
\pgfsetroundcap%
\pgfsetroundjoin%
\pgfsetlinewidth{0.501875pt}%
\definecolor{currentstroke}{rgb}{0.000000,0.000000,0.000000}%
\pgfsetstrokecolor{currentstroke}%
\pgfsetdash{}{0pt}%
\pgfpathmoveto{\pgfqpoint{1.500000in}{2.807510in}}%
\pgfpathquadraticcurveto{\pgfqpoint{1.500000in}{2.808331in}}{\pgfqpoint{1.500000in}{2.801389in}}%
\pgfusepath{stroke}%
\end{pgfscope}%
\begin{pgfscope}%
\pgfsetroundcap%
\pgfsetroundjoin%
\pgfsetlinewidth{0.501875pt}%
\definecolor{currentstroke}{rgb}{0.000000,0.000000,0.000000}%
\pgfsetstrokecolor{currentstroke}%
\pgfsetdash{}{0pt}%
\pgfpathmoveto{\pgfqpoint{1.472222in}{2.751954in}}%
\pgfpathlineto{\pgfqpoint{1.500000in}{2.807510in}}%
\pgfpathlineto{\pgfqpoint{1.527778in}{2.751954in}}%
\pgfusepath{stroke}%
\end{pgfscope}%
\begin{pgfscope}%
\pgftext[x=1.500000in,y=2.870833in,,bottom]{\rmfamily\fontsize{10.000000}{12.000000}\selectfont \(\displaystyle k_1\)}%
\end{pgfscope}%
\begin{pgfscope}%
\pgfsetroundcap%
\pgfsetroundjoin%
\pgfsetlinewidth{0.501875pt}%
\definecolor{currentstroke}{rgb}{0.000000,0.000000,0.000000}%
\pgfsetstrokecolor{currentstroke}%
\pgfsetdash{}{0pt}%
\pgfpathmoveto{\pgfqpoint{2.807604in}{1.500000in}}%
\pgfpathquadraticcurveto{\pgfqpoint{2.808378in}{1.500000in}}{\pgfqpoint{2.801389in}{1.500000in}}%
\pgfusepath{stroke}%
\end{pgfscope}%
\begin{pgfscope}%
\pgfsetroundcap%
\pgfsetroundjoin%
\pgfsetlinewidth{0.501875pt}%
\definecolor{currentstroke}{rgb}{0.000000,0.000000,0.000000}%
\pgfsetstrokecolor{currentstroke}%
\pgfsetdash{}{0pt}%
\pgfpathmoveto{\pgfqpoint{2.752048in}{1.527778in}}%
\pgfpathlineto{\pgfqpoint{2.807604in}{1.500000in}}%
\pgfpathlineto{\pgfqpoint{2.752048in}{1.472222in}}%
\pgfusepath{stroke}%
\end{pgfscope}%
\begin{pgfscope}%
\pgftext[x=2.870833in,y=1.500000in,left,]{\rmfamily\fontsize{10.000000}{12.000000}\selectfont \(\displaystyle k_2\)}%
\end{pgfscope}%
\end{pgfpicture}%
\makeatother%
\endgroup%

\caption{Aufteilung des Fourierraums in vier Quadranten plus einen Teil für die groben Skalen. Die Quadranten $C^{(i)}$ entsprechen den Unterräumen von $L^2(\hat{\mathbb{R}}^2)$ die von $\hat \psi_{ast}^{(i)}$ reproduziert werden.}
\label{fig:quadrants}
\end{figure}

Wir definieren also
\begin{align}
    \hat \psi_{ast}^{(1)} (k_1, k_2) &\coloneqq \hat\psi_{ast} (k_1,k_2),
    \qquad
    \hat\psi_{ast}^{(3)}(k_1,k_2) \coloneqq \hat\psi_{ast} (-k_1,-k_2),
    \nonumber\\
    \hat \psi_{ast}^{(2)} (k_1, k_2) &\coloneqq \hat\psi_{ast} (k_2,k_1),
    \qquad
    \hat\psi_{ast}^{(4)}(k_1, k_2) \coloneqq \hat\psi_{ast} (-k_2,-k_1)
    \label{eq:psi(i)}
\end{align}

Des Weiteren gibt es ein $W(x)$ s.d. $\hat W(k) \in C^\infty (\hat{\mathbb{R}}^2)$ und

\begin{align*}
    \Vert f \Vert^2 =
    \int_{\mathbb{R}^2} |\left\langle f, T_t W \right\rangle|^2 \d t
    + \sum_{i=1}^4 \int |\left\langle f, \psi_{ast} \right\rangle|
    \, \d \mu (a,s,t)
\end{align*}

und somit erhalten wir (dank Parseval) ein reproduzierendes System für ganz $L^2$, da ja
$L^2 (\mathbb{R}^2) = L^2 (\textrm{grobe Skalen})^\vee \oplus \sum_{i=1}^4 L^2 \left(C^{(i)}\right)^\vee$ ist. Mehr Details dazu finden sich in \cite[S. 28 ff]{Kutyniok2008}.


% Wer bisher aufmerksam mitgelesen hat und sich \cref{fig:supp_psi_hat} genau angeschaut hat, wird bemerkt haben, dass $supp(\hat\psi_{ast})$ für alle $a,s$ quasi nur im Quadranten positiver Energie und Masse (also $\omega \geq 0, \omega^2 \geq k^2$) liegt. Also wird $\mathcal{S}_f(a,s,t)$ auch nur für solche $f$ von Nutzen sein. Diesen Mangel wollen wir nun durch "`Drehen und Spiegeln"' von $\hat \psi$ und Einführen einer weiteren Analysefunktion $W$ für die groben Skalen beheben. Mehr der Vollständigkeit halber, als aus wirklicher Notwendigkeit, denn in den späteren Kapiteln betrachten wir doch nur Distributionen deren Träger komplett im Vorwärtsquadranten(?) liegt. Und die groben Skalen interessieren beim Bestimmen der Wellenfrontmenge ohnehin nicht.

% \begin{definition}[Shearlettransformation, Fortsetzung]
% Seien $\psi_1$ und $\psi_2$ wie in \cref{def:shearletttransformationI} und

% \begin{equation}
%     G^{(v)} = \left\{M_{as} \in \mathrm{GL}(2,\mathbb{R}) \Big | M_{as} = \left(\begin{smallmatrix}
%         \sqrt a & 0 \\ -\sqrt a s &  a
%     \end{smallmatrix}\right)
%     , a \in [0,1], s \in [-2,2]
%     \right\}
% \end{equation}


% \end{definition}

% \begin{theorem}[$\{\psi(i), W\}$ sind ein reproduzierendes System]

% \end{theorem}

% section konstruktion_und_eigenschaften_der_shearlets (end)


Das große Versprechen der Shearlettransformation war ja, dass sie in der Lage ist nicht nur Position, sondern auch Orientierung der Singularitäten aufzulösen. Dass dem auch so ist, ist Aussage des folgenden Satzes:

\begin{theorem}[$\mathcal{S}_f(a,s,t)$ misst $WF(f)$]
\label{thm:main_theorem}
    Sei $f \in \mathcal{S}'(C)^\vee \cap B(\mathbb{R}^2)$ (wobei $\mathcal{S}(C)^\vee$ analog zu $L^2(C)^\vee$ definiert ist).
    Sei $\mathcal{D}$ die Menge der $(s, t)$ s.d. $\mathcal{S}(a,s,t)$ schnell verschwindet. Genauer

    \begin{dmath*}
        \mathcal{D} = \left\{
        (s_0, t_0) \hiderel \in \mathbb{R}^2 \times [-1,1]~ \Big| ~\textrm{ für  $(s,t)$ in einer Umgebung $U$ von } (t_0,s_0) \hiderel :\\
        |S_f (a,s,t)| = O(a^k) \textrm{ für } a \hiderel \to 0, \forall k \hiderel \in \mathbb{N}  \textrm{ mit $O(\cdot)$ gleichmäßig über } (s,t) \hiderel \in U
        \right\}
    \end{dmath*}

    Dann gilt $WF(f)^c = \mathcal{D}$
\end{theorem}


\begin{remark}
\label{rem:shearlets_no_distributions}
    Die Einschränkung $f \in B(\mathbb{R}^2)$ ist gravierend und bedeutet zunächst, dass die Shearlettransformation nur bedingt geeignet ist, um Wellenfrontmengen auszurechnen. In \cref{sec:ausblick} soll aber ein Ansatz gegeben werden, wie der Beweis von \cref{thm:main_theorem} hoffentlich auf ganz $\mathcal{S}'(C)^\vee$ ausgedehnt werden kann. Des Weiteren zeigen ja auch die konkreten Rechnungen an Distributionen mit bereits bekannter Wellefrontmenge, dass die Shearlettransformation diese korrekt erkennt.
\end{remark}
Wenn wir die Wellenfrontmenge einer Distribution kennen, kennen wir auch ihren singulären Träger:

\begin{corollary}[$\mathcal{S}_f(a,s,t)$ misst $sing ~supp (\psi)$]
\label{cor:psi_ast_misst_sing_supp}
Sei $f$ wie eben und $\mathcal{R}$ die Projektion von $\mathcal{D}$ auf die  Ortskomponente. Also

\begin{equation*}
    \mathcal{R} = \pi (\mathcal{D})~~;~~~~
    \pi : (t,s) \mapsto t
\end{equation*}

Dann gilt $sing ~supp (f)^c = \mathcal{R}$
\end{corollary}


%!TEX root = main.tex
%%%%%%%%%%%%%%%%%%%%%%%%%%%%%%%%%%%%%%%%%%%%%%%%%%%%%%%%%%%%%%%%%%%%%%%%%%%%%%%
% % Section 1
%%%%%%%%%%%%%%%%%%%%%%%%%%%%%%%%%%%%%%%%%%%%%%%%%%%%%%%%%%%%%%%%%%%%%%%%%%%%%%%
\section{\texorpdfstring{Beweis von \cref{thm:main_theorem}}{beweis des hauptsatzes}} % (fold)
\label{sec:beweis_von_thm:main_theorem}


Bevor wir \cref{thm:main_theorem} beweisen können, benötigen wir aber noch ein paar technische Lemmata. Beweise für diese finden sich, wenn nicht gegeben, in \cite{Kutyniok2008}

\begin{lemma}
\label{lemm:lemma54}
    Sei $g \in L^2(\mathbb{R}^2)$ mit $\Vert g\Vert_\infty < \infty$. Nehme an, dass $supp (g) \subset U$ für ein $U \subset \mathbb{R}^2$ und setze
    $(U^\delta)^c \coloneqq \left\{x \in \mathbb{R}^2 | d(x,U) > \delta\right\}$.
    Dann fällt

    \begin{equation*}
        \hat h(k) \coloneqq \int \limits_0^1 \int \limits_{(B^\delta)^c} \int \limits_{-2}^{2} \left<\psi_{ast},f\right> \hat \psi_{ast}(k) \d \mu (a,s,t)
    \end{equation*}

    schnell ab für $\Vert k \Vert \to \infty$.
\end{lemma}

Der Beweist findet sich in \cite{Kutyniok2008} und ist unserem für \cref{lemm:ruecktrafo_fourier_faellt_schnell_ab} sehr ähnlich.

Das nächste Lemma ist eine Verfeinerung von Lemma 5.5 in \textcite{Kutyniok2008}, der Beweis wird deshalb erbracht.
\begin{lemma}
\label{lemm:ruecktrafo_fourier_faellt_schnell_ab}

Seien $B(s_0,\Delta) \subset [-2,2]$ und $U \subset \mathbb{R}^2$ beschränkt. Nehme an, dass $G(a,s,t)$ schnell abfällt für $a \to 0$ gleichmäßig für $(s,t) \in  B(s_0,\Delta) \times U$. Dann fällt

\begin{equation*}
    \hat h(k) = \int \limits_0^1 \int \limits_V \int \limits_{-2}^2
    G(a,s,t) \hat \psi_{ast} (k)
        \d s \d t \frac{\d a}{a^3}
\end{equation*}

schnell ab, für $\Vert k \Vert \to \infty$ und $\frac{k_2}{k_1}$ in einer Umgebung von $s_0$.
\end{lemma}

\begin{proof}
Es sei
\begin{equation*}
    \Gamma_k = \left\{a\in [0,1], s \in [-2,2] ~\Big|~ \tfrac{1}{2} \leq a|k| \leq 2 , \left|s-\tfrac{k_2}{k_1} \right| \leq \sqrt a
                   \right\}
\end{equation*}

Dann können wir dank \cref{eq:supp_psi} abschätzen

\begin{equation*}
    | \hat \psi_{ast} (k)| \leq C' a^{\frac{3}{4}} \chi_{\Gamma_k}
\end{equation*}

und nach Vorraussetzung gilt auch

\begin{equation*}
    |G(a,s,t)| \leq C_{N} a^{N} \condition{$\forall N \in \mathbb{N}$}
\end{equation*}

Außerdem sei
\begin{equation*}
    S = B(s_0,\Delta/2)
\end{equation*}

Um $\hat h(k)$ abzuschätzen, teilen wir es in den Bereich auf, in dem G(a,s,t) schnell abfällt, und in dem es nicht schnell abfällt:

\begin{dmath*}
    \hat h(k) =
    \underbrace{
    \int \limits_0^1 \int \limits_V \int \limits_{S}
    G(a,s,t) \hat \psi_{ast} (k)
        \d s \d t \frac{\d a}{a^3}}_{i)}
     +
    \underbrace{
     \int \limits_0^1 \int \limits_V \int \limits_{[-2,2]\setminus S}
    G(a,s,t) \hat \psi_{ast} (k)
        \d s \d t \frac{\d a}{a^3}}_{ii)}
\end{dmath*}


\emph{zu $i)$}

\begin{dmath*}
    i) \leq \int \limits_0^1 \int \limits_V \int \limits_{S}
    \left\lvert G(a,s,t)\right\rvert
    \left\lvert \hat \psi_{ast} (k) \right\rvert
        \d s \d t \frac{\d a}{a^3}
    \leq
    C_N C'
    \int \limits_0^1 \int \limits_V \int \limits_{S}
    a^{\frac{3}{4}} a^N \chi_{\Gamma_k} \d s \d t \frac{\d a}{a^3}
    \leq
    C_N \int \limits_{\frac{1}{2|k|}}^{\frac{2}{|k|}}
    a^{N-\frac{9}{4}} \d a
    \le C_N |k|^{-N+\frac{7}{4}}
\end{dmath*}

$i)$ fällt also schnell ab für $a \to 0$.


\emph{zu $ii)$}

\begin{dmath*}
    ii) \leq
     \int \limits_0^1 \int \limits_V \int \limits_{[-2,2]\setminus S}
    |G(a,s,t)| |\hat \psi_{ast} (k)|
        \d s \d t \frac{\d a}{a^3}
    \leq
    C' \int \limits_{0}^{1} \int \limits_{V} \int \limits_{[-2,2]\setminus S}
    |G(a,s,t)| \chi_{\Gamma_k} a^{\frac{3}{4}}
    \d s \d t \frac{\d a}{a^3}
\end{dmath*}

Für alle hinreichend großen $k$ ist aber $\Gamma_k \subset S$, also $\Gamma_k \cap [-1,1]\setminus S = \varnothing$ und demnach das Integral 0. Also

\begin{equation*}
    ii) = 0 \condition{für alle k groß genug}
\end{equation*}
\end{proof}


\begin{lemma}
    [Abschätzungen für $\left<\phi \psi_{a_0st},\psi_{a_1s't'}\right>$]
\label{lemm:lemma57}
Sei $\phi \in C_0^\infty(B(t,\delta))$. Dann gilt für alle $N>0$

\begin{enumerate}
    \item Falls $0 \leq \sqrt{a_0} \leq \sqrt{a_1}\leq \delta \leq 1$
    \begin{equation*}
        |\left<\phi \psi_{a_0st},\psi_{a_1s't'}\right>| \leq
        C_N \left(1+\frac{a_1}{a_0}\right)^{-N}
        \left(1+\frac{|s-s'|^2}{a_1}\right)^{-N}
        \left(1+\frac{\Vert t-t' \Vert^2}{a_1}\right)^{-N}
    \end{equation*}
    \item Falls $0 \leq \sqrt{a_0} \leq \delta \leq \sqrt{a_1} \leq 1$
    \begin{equation*}
        |\left<\phi \psi_{a_0st},\psi_{a_1s't'}\right>| \leq
        C_N \left(1+\frac{a_1}{a_0}\right)^{-N}
        \left(1+\frac{|s-s'|^2}{\delta^2}\right)^{-N}
        \left(1+\frac{\Vert t-t' \Vert^2}{a_1}\right)^{-N}
    \end{equation*}
\end{enumerate}
\end{lemma}

% Das jetzt folgende Lemma fehlt in \textcite{Kutyniok2008}, weshalb auf Seite 26 fälschlicherweise benutzt werden muss, dass $f$ beschränkt ist. Es reicht aber, dass $\mathcal{S}_f(a,s,t)$ für $\Vert t \Vert \to \infty$ polynomiell beschränkt ist. Mehr geht auch gar nicht, da $f$ i.A. eine temperierte Distribution ist.

% \begin{lemma}[$|\left<f,\psi_{ast}\right>|$ ist langsam wachsend]
% Sei $f \in \mathcal{S}'(\mathbb{R}^2)$. Dann gibt es ein $N \in \mathbb{N}$ s.d. für alle $a \in [0,1]$ und $s \in [-1,1]$
% \begin{equation*}
%     \left|\left\langle f, \psi_{ast}\right\rangle\right| \leq C_N (1+\Vert t\Vert^2)^{N}
% \end{equation*}
% \end{lemma}

% \begin{proof}
% $\psi_{ast} \in \mathcal{S}$ für alle $a,s,t$, da $\hat \psi_{ast}$ kompakt getragen ist für alle $a,s,t$. Also gibt es per Definition von $\mathcal{S}'$ $C \in \mathbb{R}, M,N \in \mathbb{N}$ so dass

% \begin{equation*}
%     \left|\left\langle f, \psi_{ast} \right\rangle \right|
%     \leq
%     C \sum_{|\alpha| \leq N, |\beta|\leq M} \sup_{x \in \mathbb{R}^2} \left| x^\alpha \partial^\beta \psi_{ast}(x)\right|
% \end{equation*}
% \todo{hier den Beweis fertig machen}
% \end{proof}

Kommen wir nun endlich zu dem Beweis unseres Hauptsatzes:

\begin{proof}[von \ref{thm:main_theorem}]
\label{proof:main_theorem}
Zunächst die einfachere Richtung, nämlich $WF(f)^c \subseteq \mathcal{D}$.
Wir nehmen also einen gerichteten regulären Punkt $(t_0,s_0) \in WF(f)^c$ und zeigen, dass er auch in $\mathcal{D}$ liegt. Dazu zerlegen wir $f$ zunächst wie folgt:
 Da $f$ bei $t_0$ in Richtung $s_0$ regulär ist, gibt es per Definition der Wellenfrontmenge ein $\phi \in C_0^\infty(\mathcal{R}^2)$ s.d. $\phi = 1$ in einer Umgebung von $t_0$ und für alle $N \in \mathbb{N}$ $\rwhat{\phi f} = O(1+|k|)^{-N}$ für $\frac{k_2}{k_1}$ in einer Umgebung von $s_0$. Dementsprechend ist $(1-\phi)f = 0$ in einer Umgebung von $t_0$ und es gilt

 \begin{equation}
     \mathcal{S}_f (a,s,t = \left\langle \psi_{ast},\phi f \right\rangle
                                + \left\langle \psi_{ast},(1-\phi) f \right\rangle
 \label{eq:schlaue sache}
 \end{equation}

\begin{figure}[h]
\centering
%% Creator: Matplotlib, PGF backend
%%
%% To include the figure in your LaTeX document, write
%%   \input{<filename>.pgf}
%%
%% Make sure the required packages are loaded in your preamble
%%   \usepackage{pgf}
%%
%% Figures using additional raster images can only be included by \input if
%% they are in the same directory as the main LaTeX file. For loading figures
%% from other directories you can use the `import` package
%%   \usepackage{import}
%% and then include the figures with
%%   \import{<path to file>}{<filename>.pgf}
%%
%% Matplotlib used the following preamble
%%   \usepackage[utf8x]{inputenc}
%%   \usepackage[T1]{fontenc}
%%   \usepackage{amssymb}
%%
\begingroup%
\makeatletter%
\begin{pgfpicture}%
\pgfpathrectangle{\pgfpointorigin}{\pgfqpoint{4.000000in}{2.200000in}}%
\pgfusepath{use as bounding box, clip}%
\begin{pgfscope}%
\pgfsetbuttcap%
\pgfsetmiterjoin%
\definecolor{currentfill}{rgb}{1.000000,1.000000,1.000000}%
\pgfsetfillcolor{currentfill}%
\pgfsetlinewidth{0.000000pt}%
\definecolor{currentstroke}{rgb}{1.000000,1.000000,1.000000}%
\pgfsetstrokecolor{currentstroke}%
\pgfsetdash{}{0pt}%
\pgfpathmoveto{\pgfqpoint{0.000000in}{0.000000in}}%
\pgfpathlineto{\pgfqpoint{4.000000in}{0.000000in}}%
\pgfpathlineto{\pgfqpoint{4.000000in}{2.200000in}}%
\pgfpathlineto{\pgfqpoint{0.000000in}{2.200000in}}%
\pgfpathclose%
\pgfusepath{fill}%
\end{pgfscope}%
\begin{pgfscope}%
\pgfsetbuttcap%
\pgfsetmiterjoin%
\definecolor{currentfill}{rgb}{1.000000,1.000000,1.000000}%
\pgfsetfillcolor{currentfill}%
\pgfsetlinewidth{0.000000pt}%
\definecolor{currentstroke}{rgb}{0.000000,0.000000,0.000000}%
\pgfsetstrokecolor{currentstroke}%
\pgfsetstrokeopacity{0.000000}%
\pgfsetdash{}{0pt}%
\pgfpathmoveto{\pgfqpoint{0.500000in}{0.275000in}}%
\pgfpathlineto{\pgfqpoint{3.600000in}{0.275000in}}%
\pgfpathlineto{\pgfqpoint{3.600000in}{1.936000in}}%
\pgfpathlineto{\pgfqpoint{0.500000in}{1.936000in}}%
\pgfpathclose%
\pgfusepath{fill}%
\end{pgfscope}%
\begin{pgfscope}%
\pgfsetbuttcap%
\pgfsetroundjoin%
\definecolor{currentfill}{rgb}{0.000000,0.000000,0.000000}%
\pgfsetfillcolor{currentfill}%
\pgfsetlinewidth{0.803000pt}%
\definecolor{currentstroke}{rgb}{0.000000,0.000000,0.000000}%
\pgfsetstrokecolor{currentstroke}%
\pgfsetdash{}{0pt}%
\pgfsys@defobject{currentmarker}{\pgfqpoint{0.000000in}{-0.048611in}}{\pgfqpoint{0.000000in}{0.000000in}}{%
\pgfpathmoveto{\pgfqpoint{0.000000in}{0.000000in}}%
\pgfpathlineto{\pgfqpoint{0.000000in}{-0.048611in}}%
\pgfusepath{stroke,fill}%
}%
\begin{pgfscope}%
\pgfsys@transformshift{2.050000in}{0.344208in}%
\pgfsys@useobject{currentmarker}{}%
\end{pgfscope}%
\end{pgfscope}%
\begin{pgfscope}%
\pgftext[x=2.050000in,y=0.246986in,,top]{\rmfamily\fontsize{10.000000}{12.000000}\selectfont \(\displaystyle (t_0, x_0)\)}%
\end{pgfscope}%
\begin{pgfscope}%
\pgfpathrectangle{\pgfqpoint{0.500000in}{0.275000in}}{\pgfqpoint{3.100000in}{1.661000in}}%
\pgfusepath{clip}%
\pgfsetrectcap%
\pgfsetroundjoin%
\pgfsetlinewidth{0.501875pt}%
\definecolor{currentstroke}{rgb}{0.894118,0.101961,0.109804}%
\pgfsetstrokecolor{currentstroke}%
\pgfsetdash{}{0pt}%
\pgfpathmoveto{\pgfqpoint{0.500000in}{0.344208in}}%
\pgfpathlineto{\pgfqpoint{1.778478in}{0.345597in}}%
\pgfpathlineto{\pgfqpoint{1.803303in}{0.348841in}}%
\pgfpathlineto{\pgfqpoint{1.818819in}{0.353486in}}%
\pgfpathlineto{\pgfqpoint{1.831231in}{0.359861in}}%
\pgfpathlineto{\pgfqpoint{1.843644in}{0.369866in}}%
\pgfpathlineto{\pgfqpoint{1.856056in}{0.385070in}}%
\pgfpathlineto{\pgfqpoint{1.865365in}{0.401049in}}%
\pgfpathlineto{\pgfqpoint{1.874675in}{0.422003in}}%
\pgfpathlineto{\pgfqpoint{1.883984in}{0.448968in}}%
\pgfpathlineto{\pgfqpoint{1.896396in}{0.496107in}}%
\pgfpathlineto{\pgfqpoint{1.908809in}{0.558191in}}%
\pgfpathlineto{\pgfqpoint{1.921221in}{0.637077in}}%
\pgfpathlineto{\pgfqpoint{1.936737in}{0.760517in}}%
\pgfpathlineto{\pgfqpoint{1.952252in}{0.909895in}}%
\pgfpathlineto{\pgfqpoint{1.977077in}{1.185411in}}%
\pgfpathlineto{\pgfqpoint{2.005005in}{1.489311in}}%
\pgfpathlineto{\pgfqpoint{2.017417in}{1.597376in}}%
\pgfpathlineto{\pgfqpoint{2.026727in}{1.659912in}}%
\pgfpathlineto{\pgfqpoint{2.036036in}{1.703327in}}%
\pgfpathlineto{\pgfqpoint{2.042242in}{1.720595in}}%
\pgfpathlineto{\pgfqpoint{2.048448in}{1.728063in}}%
\pgfpathlineto{\pgfqpoint{2.051552in}{1.728063in}}%
\pgfpathlineto{\pgfqpoint{2.054655in}{1.725569in}}%
\pgfpathlineto{\pgfqpoint{2.060861in}{1.713168in}}%
\pgfpathlineto{\pgfqpoint{2.067067in}{1.691126in}}%
\pgfpathlineto{\pgfqpoint{2.076376in}{1.641065in}}%
\pgfpathlineto{\pgfqpoint{2.085686in}{1.572759in}}%
\pgfpathlineto{\pgfqpoint{2.098098in}{1.458747in}}%
\pgfpathlineto{\pgfqpoint{2.119820in}{1.221041in}}%
\pgfpathlineto{\pgfqpoint{2.150851in}{0.878170in}}%
\pgfpathlineto{\pgfqpoint{2.169469in}{0.707845in}}%
\pgfpathlineto{\pgfqpoint{2.184985in}{0.595451in}}%
\pgfpathlineto{\pgfqpoint{2.197397in}{0.525148in}}%
\pgfpathlineto{\pgfqpoint{2.209810in}{0.470810in}}%
\pgfpathlineto{\pgfqpoint{2.222222in}{0.430271in}}%
\pgfpathlineto{\pgfqpoint{2.234635in}{0.401049in}}%
\pgfpathlineto{\pgfqpoint{2.247047in}{0.380681in}}%
\pgfpathlineto{\pgfqpoint{2.259459in}{0.366946in}}%
\pgfpathlineto{\pgfqpoint{2.271872in}{0.357980in}}%
\pgfpathlineto{\pgfqpoint{2.287387in}{0.351274in}}%
\pgfpathlineto{\pgfqpoint{2.306006in}{0.347197in}}%
\pgfpathlineto{\pgfqpoint{2.333934in}{0.344936in}}%
\pgfpathlineto{\pgfqpoint{2.395996in}{0.344227in}}%
\pgfpathlineto{\pgfqpoint{3.600000in}{0.344208in}}%
\pgfpathlineto{\pgfqpoint{3.600000in}{0.344208in}}%
\pgfusepath{stroke}%
\end{pgfscope}%
\begin{pgfscope}%
\pgfpathrectangle{\pgfqpoint{0.500000in}{0.275000in}}{\pgfqpoint{3.100000in}{1.661000in}}%
\pgfusepath{clip}%
\pgfsetrectcap%
\pgfsetroundjoin%
\pgfsetlinewidth{0.501875pt}%
\definecolor{currentstroke}{rgb}{0.215686,0.494118,0.721569}%
\pgfsetstrokecolor{currentstroke}%
\pgfsetdash{}{0pt}%
\pgfpathmoveto{\pgfqpoint{0.500000in}{0.344208in}}%
\pgfpathlineto{\pgfqpoint{1.744344in}{0.344919in}}%
\pgfpathlineto{\pgfqpoint{1.750551in}{0.350590in}}%
\pgfpathlineto{\pgfqpoint{1.756757in}{0.361837in}}%
\pgfpathlineto{\pgfqpoint{1.766066in}{0.388737in}}%
\pgfpathlineto{\pgfqpoint{1.775375in}{0.426758in}}%
\pgfpathlineto{\pgfqpoint{1.787788in}{0.492274in}}%
\pgfpathlineto{\pgfqpoint{1.803303in}{0.591792in}}%
\pgfpathlineto{\pgfqpoint{1.849850in}{0.906003in}}%
\pgfpathlineto{\pgfqpoint{1.862262in}{0.967573in}}%
\pgfpathlineto{\pgfqpoint{1.871572in}{1.002023in}}%
\pgfpathlineto{\pgfqpoint{1.880881in}{1.024974in}}%
\pgfpathlineto{\pgfqpoint{1.887087in}{1.033451in}}%
\pgfpathlineto{\pgfqpoint{1.893293in}{1.036292in}}%
\pgfpathlineto{\pgfqpoint{2.209810in}{1.035581in}}%
\pgfpathlineto{\pgfqpoint{2.216016in}{1.029910in}}%
\pgfpathlineto{\pgfqpoint{2.222222in}{1.018663in}}%
\pgfpathlineto{\pgfqpoint{2.231532in}{0.991763in}}%
\pgfpathlineto{\pgfqpoint{2.240841in}{0.953742in}}%
\pgfpathlineto{\pgfqpoint{2.253253in}{0.888226in}}%
\pgfpathlineto{\pgfqpoint{2.268769in}{0.788708in}}%
\pgfpathlineto{\pgfqpoint{2.315315in}{0.474497in}}%
\pgfpathlineto{\pgfqpoint{2.327728in}{0.412927in}}%
\pgfpathlineto{\pgfqpoint{2.337037in}{0.378477in}}%
\pgfpathlineto{\pgfqpoint{2.346346in}{0.355526in}}%
\pgfpathlineto{\pgfqpoint{2.352553in}{0.347049in}}%
\pgfpathlineto{\pgfqpoint{2.358759in}{0.344208in}}%
\pgfpathlineto{\pgfqpoint{3.600000in}{0.344208in}}%
\pgfpathlineto{\pgfqpoint{3.600000in}{0.344208in}}%
\pgfusepath{stroke}%
\end{pgfscope}%
\begin{pgfscope}%
\pgfpathrectangle{\pgfqpoint{0.500000in}{0.275000in}}{\pgfqpoint{3.100000in}{1.661000in}}%
\pgfusepath{clip}%
\pgfsetrectcap%
\pgfsetroundjoin%
\pgfsetlinewidth{0.501875pt}%
\definecolor{currentstroke}{rgb}{0.301961,0.686275,0.290196}%
\pgfsetstrokecolor{currentstroke}%
\pgfsetdash{}{0pt}%
\pgfpathmoveto{\pgfqpoint{0.500000in}{1.036292in}}%
\pgfpathlineto{\pgfqpoint{1.744344in}{1.035581in}}%
\pgfpathlineto{\pgfqpoint{1.750551in}{1.029910in}}%
\pgfpathlineto{\pgfqpoint{1.756757in}{1.018663in}}%
\pgfpathlineto{\pgfqpoint{1.766066in}{0.991763in}}%
\pgfpathlineto{\pgfqpoint{1.775375in}{0.953742in}}%
\pgfpathlineto{\pgfqpoint{1.787788in}{0.888226in}}%
\pgfpathlineto{\pgfqpoint{1.803303in}{0.788708in}}%
\pgfpathlineto{\pgfqpoint{1.849850in}{0.474497in}}%
\pgfpathlineto{\pgfqpoint{1.862262in}{0.412927in}}%
\pgfpathlineto{\pgfqpoint{1.871572in}{0.378477in}}%
\pgfpathlineto{\pgfqpoint{1.880881in}{0.355526in}}%
\pgfpathlineto{\pgfqpoint{1.887087in}{0.347049in}}%
\pgfpathlineto{\pgfqpoint{1.893293in}{0.344208in}}%
\pgfpathlineto{\pgfqpoint{2.209810in}{0.344919in}}%
\pgfpathlineto{\pgfqpoint{2.216016in}{0.350590in}}%
\pgfpathlineto{\pgfqpoint{2.222222in}{0.361837in}}%
\pgfpathlineto{\pgfqpoint{2.231532in}{0.388737in}}%
\pgfpathlineto{\pgfqpoint{2.240841in}{0.426758in}}%
\pgfpathlineto{\pgfqpoint{2.253253in}{0.492274in}}%
\pgfpathlineto{\pgfqpoint{2.268769in}{0.591792in}}%
\pgfpathlineto{\pgfqpoint{2.315315in}{0.906003in}}%
\pgfpathlineto{\pgfqpoint{2.327728in}{0.967573in}}%
\pgfpathlineto{\pgfqpoint{2.337037in}{1.002023in}}%
\pgfpathlineto{\pgfqpoint{2.346346in}{1.024974in}}%
\pgfpathlineto{\pgfqpoint{2.352553in}{1.033451in}}%
\pgfpathlineto{\pgfqpoint{2.358759in}{1.036292in}}%
\pgfpathlineto{\pgfqpoint{3.600000in}{1.036292in}}%
\pgfpathlineto{\pgfqpoint{3.600000in}{1.036292in}}%
\pgfusepath{stroke}%
\end{pgfscope}%
\begin{pgfscope}%
\pgfsetrectcap%
\pgfsetmiterjoin%
\pgfsetlinewidth{0.501875pt}%
\definecolor{currentstroke}{rgb}{0.000000,0.000000,0.000000}%
\pgfsetstrokecolor{currentstroke}%
\pgfsetdash{}{0pt}%
\pgfpathmoveto{\pgfqpoint{2.050000in}{0.275000in}}%
\pgfpathlineto{\pgfqpoint{2.050000in}{1.936000in}}%
\pgfusepath{stroke}%
\end{pgfscope}%
\begin{pgfscope}%
\pgfsetrectcap%
\pgfsetmiterjoin%
\pgfsetlinewidth{0.501875pt}%
\definecolor{currentstroke}{rgb}{0.000000,0.000000,0.000000}%
\pgfsetstrokecolor{currentstroke}%
\pgfsetdash{}{0pt}%
\pgfpathmoveto{\pgfqpoint{0.500000in}{0.344208in}}%
\pgfpathlineto{\pgfqpoint{3.600000in}{0.344208in}}%
\pgfusepath{stroke}%
\end{pgfscope}%
\begin{pgfscope}%
\pgfsetroundcap%
\pgfsetroundjoin%
\pgfsetlinewidth{0.501875pt}%
\definecolor{currentstroke}{rgb}{0.000000,0.000000,0.000000}%
\pgfsetstrokecolor{currentstroke}%
\pgfsetdash{}{0pt}%
\pgfpathmoveto{\pgfqpoint{2.050000in}{1.942121in}}%
\pgfpathquadraticcurveto{\pgfqpoint{2.050000in}{1.942943in}}{\pgfqpoint{2.050000in}{1.936000in}}%
\pgfusepath{stroke}%
\end{pgfscope}%
\begin{pgfscope}%
\pgfsetroundcap%
\pgfsetroundjoin%
\pgfsetlinewidth{0.501875pt}%
\definecolor{currentstroke}{rgb}{0.000000,0.000000,0.000000}%
\pgfsetstrokecolor{currentstroke}%
\pgfsetdash{}{0pt}%
\pgfpathmoveto{\pgfqpoint{2.022222in}{1.886565in}}%
\pgfpathlineto{\pgfqpoint{2.050000in}{1.942121in}}%
\pgfpathlineto{\pgfqpoint{2.077778in}{1.886565in}}%
\pgfusepath{stroke}%
\end{pgfscope}%
\begin{pgfscope}%
\pgftext[x=2.050000in,y=2.005444in,,bottom]{\rmfamily\fontsize{10.000000}{12.000000}\selectfont \(\displaystyle  f\)}%
\end{pgfscope}%
\begin{pgfscope}%
\pgfsetroundcap%
\pgfsetroundjoin%
\pgfsetlinewidth{0.501875pt}%
\definecolor{currentstroke}{rgb}{0.000000,0.000000,0.000000}%
\pgfsetstrokecolor{currentstroke}%
\pgfsetdash{}{0pt}%
\pgfpathmoveto{\pgfqpoint{3.606111in}{0.344208in}}%
\pgfpathquadraticcurveto{\pgfqpoint{3.606938in}{0.344208in}}{\pgfqpoint{3.600000in}{0.344208in}}%
\pgfusepath{stroke}%
\end{pgfscope}%
\begin{pgfscope}%
\pgfsetroundcap%
\pgfsetroundjoin%
\pgfsetlinewidth{0.501875pt}%
\definecolor{currentstroke}{rgb}{0.000000,0.000000,0.000000}%
\pgfsetstrokecolor{currentstroke}%
\pgfsetdash{}{0pt}%
\pgfpathmoveto{\pgfqpoint{3.550556in}{0.371986in}}%
\pgfpathlineto{\pgfqpoint{3.606111in}{0.344208in}}%
\pgfpathlineto{\pgfqpoint{3.550556in}{0.316431in}}%
\pgfusepath{stroke}%
\end{pgfscope}%
\begin{pgfscope}%
\pgftext[x=3.669444in,y=0.344208in,left,]{\rmfamily\fontsize{10.000000}{12.000000}\selectfont \(\displaystyle (x,t)\)}%
\end{pgfscope}%
\begin{pgfscope}%
\pgfsetbuttcap%
\pgfsetmiterjoin%
\definecolor{currentfill}{rgb}{1.000000,1.000000,1.000000}%
\pgfsetfillcolor{currentfill}%
\pgfsetfillopacity{0.800000}%
\pgfsetlinewidth{0.501875pt}%
\definecolor{currentstroke}{rgb}{0.800000,0.800000,0.800000}%
\pgfsetstrokecolor{currentstroke}%
\pgfsetstrokeopacity{0.800000}%
\pgfsetdash{}{0pt}%
\pgfpathmoveto{\pgfqpoint{2.736382in}{1.243871in}}%
\pgfpathlineto{\pgfqpoint{3.502778in}{1.243871in}}%
\pgfpathquadraticcurveto{\pgfqpoint{3.530556in}{1.243871in}}{\pgfqpoint{3.530556in}{1.271648in}}%
\pgfpathlineto{\pgfqpoint{3.530556in}{1.838778in}}%
\pgfpathquadraticcurveto{\pgfqpoint{3.530556in}{1.866556in}}{\pgfqpoint{3.502778in}{1.866556in}}%
\pgfpathlineto{\pgfqpoint{2.736382in}{1.866556in}}%
\pgfpathquadraticcurveto{\pgfqpoint{2.708604in}{1.866556in}}{\pgfqpoint{2.708604in}{1.838778in}}%
\pgfpathlineto{\pgfqpoint{2.708604in}{1.271648in}}%
\pgfpathquadraticcurveto{\pgfqpoint{2.708604in}{1.243871in}}{\pgfqpoint{2.736382in}{1.243871in}}%
\pgfpathclose%
\pgfusepath{stroke,fill}%
\end{pgfscope}%
\begin{pgfscope}%
\pgfsetrectcap%
\pgfsetroundjoin%
\pgfsetlinewidth{0.501875pt}%
\definecolor{currentstroke}{rgb}{0.894118,0.101961,0.109804}%
\pgfsetstrokecolor{currentstroke}%
\pgfsetdash{}{0pt}%
\pgfpathmoveto{\pgfqpoint{2.764160in}{1.762389in}}%
\pgfpathlineto{\pgfqpoint{3.041937in}{1.762389in}}%
\pgfusepath{stroke}%
\end{pgfscope}%
\begin{pgfscope}%
\pgftext[x=3.153049in,y=1.713778in,left,base]{\rmfamily\fontsize{10.000000}{12.000000}\selectfont \(\displaystyle \psi\)}%
\end{pgfscope}%
\begin{pgfscope}%
\pgfsetrectcap%
\pgfsetroundjoin%
\pgfsetlinewidth{0.501875pt}%
\definecolor{currentstroke}{rgb}{0.215686,0.494118,0.721569}%
\pgfsetstrokecolor{currentstroke}%
\pgfsetdash{}{0pt}%
\pgfpathmoveto{\pgfqpoint{2.764160in}{1.568716in}}%
\pgfpathlineto{\pgfqpoint{3.041937in}{1.568716in}}%
\pgfusepath{stroke}%
\end{pgfscope}%
\begin{pgfscope}%
\pgftext[x=3.153049in,y=1.520105in,left,base]{\rmfamily\fontsize{10.000000}{12.000000}\selectfont \(\displaystyle \phi\)}%
\end{pgfscope}%
\begin{pgfscope}%
\pgfsetrectcap%
\pgfsetroundjoin%
\pgfsetlinewidth{0.501875pt}%
\definecolor{currentstroke}{rgb}{0.301961,0.686275,0.290196}%
\pgfsetstrokecolor{currentstroke}%
\pgfsetdash{}{0pt}%
\pgfpathmoveto{\pgfqpoint{2.764160in}{1.375043in}}%
\pgfpathlineto{\pgfqpoint{3.041937in}{1.375043in}}%
\pgfusepath{stroke}%
\end{pgfscope}%
\begin{pgfscope}%
\pgftext[x=3.153049in,y=1.326432in,left,base]{\rmfamily\fontsize{10.000000}{12.000000}\selectfont \(\displaystyle 1-\phi\)}%
\end{pgfscope}%
\end{pgfpicture}%
\makeatother%
\endgroup%

\caption{Die Zerlegung von $f$ um $(t_0,x_0)$ herum visualisiert}
\label{fig:smart_decomposition}
\end{figure}

Da $(1-\phi)f$ in einer Umgebung von $t_0$ verschwindet und nach \cref{prop:shearlets_decay_rapidly} Shearlets außerhalb von $t$ schnell abfallen für $a \to 0$ fällt auch der zweite Term von \cref{eq:schlaue sache}
für $t \neq t_0$ schnell ab. Für den ersten Term überzeugen wir uns anhand von \cref{fig:supp_psi_hat,eq:supp_psi}, dass für $a$ klein genug $supp(\hat\psi_{ast})$ schließlich in jedem noch so kleinen Kegel um $s$ liegt. In einem solchen um $s_0$ fällt aber $\rwhat{\phi f}$ rapide ab nach Vorraussetzung und damit auch der erste Term in \cref{eq:schlaue sache}.

Die beiden entscheidenden Zutaten waren hier also die Tatsache, dass die Shearlets außerhalb von $(t',x')$ rapide abfallen und damit bei immer feineren Skalen $a$ immer besser lokalisiert werden sowie die Tatsache, dass für $a \to 0$ der Träger im Frequenzbereich in immer engeren Kegeln liegt.

Deutlich schwieriger ist die umgekehrte Inklusion, nämlich dass die Shearlettransformation tatsächlich die ganze Wellenfrontmenge erkennt. Hier geht jetzt auch die Reproduktionseigenschaft der Transformation ein, eben genau dass sie alles sieht.

Für die umgekehrte Inklusion $\mathcal{D} \subseteq WF(f)^c$ haben wir zu zeigen, dass falls $\mathcal{S}_f (a,s,t)$ schnell abfällt in einer Umgebung $U$ von $(s_0, t_0)$ für $a \to 0$ dann auch $\rwhat{\phi f} (k)$ schnell abfällt für $\Vert(k)\Vert \to \infty$ für $\frac{k_2}{k_1}$ in einer Umgebung von $s_0$ und ein $\phi$ getragen in einer Umgebung von $t_0$.

Sei also $\mathcal{S}_f(a,s,t) \xrightarrow[a \to 0]{\text{\tiny schnell}} 0$
für $(s,t) \in B(s_0, \Delta) \times B(t_0, 2 \delta) \eqqcolon S \times U$\footnote{Falls das ganze gilt für $(s,t)$ in einer offenen Umgebung von $(s_0,t_0)$, dann auch in solchen Bällen mit passenden $\Delta, \delta$}.
Sei $\phi \in C_0^\infty(B(t_0,\delta))$ und $\phi \equiv 1$ in einer Umgebung von $t_0$. Dann müssen wir zeigen, dass $\rwhat{\phi f}(k) \xrightarrow[a \to 0]{\text{\tiny schnell}} 0$ für $\frac{k_2}{k_1} \in S$

\todo{Wie schreibt an das ganze auf, im Sinne von Distributionen, also $f$ keien Funktion?}
\begin{dmath*}
    \rwhat{\phi f} (k)
    =
    \int \phi(x) f(x) e^{-ikx} \d x \\
    \stackrel{\ref{thm:shearlets_reproduzieren}}{=}
    \iint \left\langle \psi_{ast},\phi f \right\rangle
        \psi_{ast} (x) \d \mu (ast)
        e^{\cdots} \d x
    =
    \int \left\langle \psi_{ast'},\phi f \right\rangle
    \hat\psi_{ast}(k) \d \mu(\cdots)
    = \kern -1em
    \underbrace{
        \int \limits_{U \times [-2,2] \times [0,1]} \kern -1em
        \left\langle \psi_{ast},\phi f \right\rangle
        \hat\psi_{ast}(k) \d \mu(\cdots)
    }_{i)}
    + \kern -1em
    \underbrace{
        \int \limits_{U^c \times [-2,2] \times [0,1]} \kern -1em
        \left\langle \psi_{ast},\phi f \right\rangle
        \hat\psi_{ast}(k) \d \mu(\cdots)
    }_{ii)}
\end{dmath*}

\emph{zu $ii)$}\\[.5em]
Per Konstruktion von $\phi$ gilt $d(supp(\phi f, U^c)) = \delta > 0$. Also fällt nach \cref{lemm:lemma54} $ii)$ schnell ab.

\emph{zu $i)$}\\[.5em]
Falls $\left<\psi_{ast},\phi f\right> \xrightarrow[a \to 0]{\text{\tiny schnell}} 0$ für $(s,t) \in S \times U$, dann fällt $ii)$ schnell ab $\frac{k_2}{k_2} \in B(s_0,\Delta/2)$ nach \cref{lemm:ruecktrafo_fourier_faellt_schnell_ab}. Wir zeigen also, dass $\left<\psi_{ast},\phi f\right> \xrightarrow[a \to 0]{\text{\tiny schnell}} 0$ für $(s,t) \in S \times U$.

\begin{dmath*}
    \left\langle \phi f, \psi_{ast} \right\rangle
    =
    \int \phi f \psi_{ast} \d x
    =
    \int \phi(x) \int \left\langle f, \psi_{a' s' t'} \right\rangle
    \psi_{a's't'} (x) \d \mu(a',s',t') \psi_{ast}(x) \d x
    =
    \int\limits_{0}^{1} \int\limits_{\mathbb{R}^2} \int\limits_{-2}^2
    \left\langle \phi \psi_{ast}, \psi_{a's't'} \right\rangle
    \left\langle f, \psi_{a's't'} \right\rangle
    \d s' \d t' \frac{\d a'}{a^{\prime 3}}
\end{dmath*}


Für $a < \delta$\footnote{Und da wir uns für kleine $a$ interessieren, dürfen wir das auch direkt annehmen} teilen wir die Integration über $s'$ auf in die drei Fälle $a)$ $0<a'<a<\delta$, $b)$ $a < a' < \delta$ und $c)$ $\delta < a'$ und nutzen \cref{lemm:lemma57}. Des weiteren Teilen wir die Integration über $t'$ auf in die Fälle $t' \in U$ und $t' \in U^c$

\emph{Zu $a)$, $t' \in U^c$}\\[.5em]
Hier gilt im Integrationsbereich $\Vert t' - t \Vert \geq \delta$. Mit \cref{lemm:lemma57} können wir für alle $N\in \mathbb{N}$ abschätzen wie folgt:
\begin{dmath*}
 \int\limits_{0}^{a} \int\limits_{U^c} \int\limits_{-2}^2
 \left\langle \phi \psi_{ast}, \psi_{a's't'} \right\rangle
 \left\langle f, \psi_{a's't'} \right\rangle
    \d s' \d t' \frac{\d a'}{a^{\prime 3}}
\leq
\int\limits_{0}^{a} \int\limits_{U^c} \int\limits_{-2}^2
C_N
\left| \left\langle f, \psi_{a's't'} \right\rangle \right|
\left(1+\frac{\Vert t-t' \Vert^2}{a}\right)^{-N} \d \mu(a',s',t')
\leq
\int\limits_{0}^{a} \int\limits_{U^c} \int\limits_{-2}^2
C_N
\left| \left\langle f, \psi_{a's't'} \right\rangle \right|
a^N \Vert t-t' \Vert^{-N} \d \mu(a',s',t')
\leq
C_N a^N
\int\limits_{0}^{a} \int\limits_{U^c} \int\limits_{-2}^2
\left| \left\langle f, \psi_{a's't'} \right\rangle \right|
\Vert t-t' \Vert^{-N} \d \mu(a',s',t')
\end{dmath*}

Wobei das letzte Integral endlich ist, da $\Vert t-t'\Vert \geq \delta$ im Integrationsbereich und $\left| \left\langle f, \psi_{a's't'} \right\rangle \right|$ beschränkt ist, da $f$ beschränkt ist. Für kleinere $a$ kann es nur kleiner werden.


\emph{Zu $a)$, $t' \in U$}\\[.5em]
Eine kurze Erinnerung: Falls $(s',t') \in S \times U$ gilt nach Vorraussetzung für alle $N \in \mathbb{N}$:
$\left| \left\langle f, \psi_{a's't'} \right\rangle \right| \leq C_N a^N$.
Außerdem gilt für $s' \notin S : ~$ $|s-s'|  \geq \Delta$ für alle $s$ die wir hier betrachten. Also:

\begin{dmath*}
 \int\limits_{0}^{a} \int\limits_{U} \int\limits_{-2}^2
 \cdots
    \d \mu(a',s',t')
=
 \int\limits_{0}^{a} \int\limits_{U} \int\limits_{S}
 \cdots
    \d \mu(a',s',t')
    +
 \int\limits_{0}^{a} \int\limits_{U} \int\limits_{S^c}
 \cdots
    \d \mu(a',s',t')
=
 \int\limits_{0}^{a} \int\limits_{U} \int\limits_{S} C_N a^N
 \d \mu(a',s',t')
 +
 \int\limits_{0}^{a} \int\limits_{U} \int\limits_{S^c}
 |\left\langle f, \psi_{a's't'} \right\rangle|
 \left(1+\frac{|s-s'|^2}{a}\right)^{-N}
 \d \mu(a',s',t')
 \leq
 C_N a^N +
  \int\limits_{0}^{a} \int\limits_{U} \int\limits_{S^c}
 \left\langle f, \psi_{a's't'} \right\rangle
 a^N \Delta^{-2N}
 \d \mu(a',s',t')
 \leq
 C_N a^N
\end{dmath*}

Mit analogen Abschätzungen und \cref{lemm:lemma57} erhalten wir auch noch, dass auch die Integral zu $b)$ und $c)$ schnell abfallen in $a$.

\raggedleft{$QED$}
\end{proof}



% section beweis_von_thm:main_theorem (end)


%!TEX root = main.tex
%!TEX spellcheck=de_DE
%%%%%%%%%%%%%%%%%%%%%%%%%%%%%%%%%%%%%%%%%%%%%%%%%%%%%%%%%%%%%%%%%%%%%%%%%%%%%%%
% % Section 2
%%%%%%%%%%%%%%%%%%%%%%%%%%%%%%%%%%%%%%%%%%%%%%%%%%%%%%%%%%%%%%%%%%%%%%%%%%%%%%%

\begin{remark}[Notation]
    Da wir ab jetzt Distributionen aus der Physik betrachten, für die es üblich ist als Variablen $(t, x)$ und als Variablen im Fourierraum $(\omega, k)$ zu verwenden, schreiben wir statt $(x_1, x_2)$ ab jetzt $(t,x)$ und statt $(k_1, k_2)$ schreiben wir $(\omega, k)$. Außerdem verwenden wir das Minkowski-Skalarprodukt für die Fouriertransformation d.h.
    \begin{equation*}
        \hat f (\omega, k) \coloneqq \int f(t,x) e^{-i\omega t + i k x}
        \d t \d x
        ,
    \end{equation*}
    wieder um den Konventionen in der Physik gerecht zu werden.
\end{remark}

\section{\texorpdfstring{Zwei nützliche Substitionen für  $\left<f, \psi_{ast}\right>$ und ein Lemma}{Zwei nützliche Substitutionen und ein Lemma}}
\label{sec:substitutionen}

Um die Shearlettransformation und damit Wellenfrontmengen auszurechnen, müssen wir Ausdrücke der Form \(\lim_{a \to 0} \int f (t,x) \psi_{ast}^ (t,x) \d t \d x\) abschätzen. Da kein expliziter Ausdruck für $\psi_{ast}(x)$ gegeben ist, aber zumindest der Träger der Fouriertransformierten \(\hat \psi_{ast} (\omega, k)\) bekannt ist, ist es einfacher \(\left<\hat f, \hat \psi_{ast}\right>\) statt \(\left< f, \psi_{ast} \right>\) zu berechnen. Das hat dann die Form
\begin{equation}
    \left< \hat f, \hat\psi_{ast}\right> \\
    = \int a^{\frac{3}{4}} \hat \psi_1(a \omega)
    \hat \psi_2 \left(a^{-\frac{1}{2}} \left(\frac{k}{\omega} - s\right)\right)
    \hat f (\omega,k) e^{-i \omega t + ikx} \d \omega \d k.
\end{equation}

Punktweise konvergiert der Integrand gegen 0, da $\hat \psi$ kompakt getragen ist. Für $a \to 0$ wird aber der Träger des Integranden immer weiter "`nach außen"' verschoben (vgl. \cref{fig:supp_psi_hat}), ohne aber (notwendigerweise) im Betrag abzunehmen. Deshalb existiert auch keine integrierbare Majorante, die unabhängig von $a$ ist. Also verschieben wir den Integrationsbereich mittels Substitution so, dass der Integrationsbereich für $a \to 0$ immer der selbe bleibt und wir $\hat f$ "`immer weiter draußen"' anschauen. Diese Substituionen werden was immer Ausgangspunkt unserer Abschätzungen sein.

\begin{figure}[h]
    \centering
    \begin{minipage}{0.5\textwidth}
        \centering
        \resizebox{\textwidth}{!}{%% Creator: Matplotlib, PGF backend
%%
%% To include the figure in your LaTeX document, write
%%   \input{<filename>.pgf}
%%
%% Make sure the required packages are loaded in your preamble
%%   \usepackage{pgf}
%%
%% Figures using additional raster images can only be included by \input if
%% they are in the same directory as the main LaTeX file. For loading figures
%% from other directories you can use the `import` package
%%   \usepackage{import}
%% and then include the figures with
%%   \import{<path to file>}{<filename>.pgf}
%%
%% Matplotlib used the following preamble
%%   \usepackage[utf8x]{inputenc}
%%   \usepackage[T1]{fontenc}
%%   \usepackage{amssymb}
%%
\begingroup%
\makeatletter%
\begin{pgfpicture}%
\pgfpathrectangle{\pgfpointorigin}{\pgfqpoint{4.000000in}{2.800000in}}%
\pgfusepath{use as bounding box, clip}%
\begin{pgfscope}%
\pgfsetbuttcap%
\pgfsetmiterjoin%
\definecolor{currentfill}{rgb}{1.000000,1.000000,1.000000}%
\pgfsetfillcolor{currentfill}%
\pgfsetlinewidth{0.000000pt}%
\definecolor{currentstroke}{rgb}{1.000000,1.000000,1.000000}%
\pgfsetstrokecolor{currentstroke}%
\pgfsetdash{}{0pt}%
\pgfpathmoveto{\pgfqpoint{0.000000in}{0.000000in}}%
\pgfpathlineto{\pgfqpoint{4.000000in}{0.000000in}}%
\pgfpathlineto{\pgfqpoint{4.000000in}{2.800000in}}%
\pgfpathlineto{\pgfqpoint{0.000000in}{2.800000in}}%
\pgfpathclose%
\pgfusepath{fill}%
\end{pgfscope}%
\begin{pgfscope}%
\pgfsetbuttcap%
\pgfsetmiterjoin%
\definecolor{currentfill}{rgb}{1.000000,1.000000,1.000000}%
\pgfsetfillcolor{currentfill}%
\pgfsetlinewidth{0.000000pt}%
\definecolor{currentstroke}{rgb}{0.000000,0.000000,0.000000}%
\pgfsetstrokecolor{currentstroke}%
\pgfsetstrokeopacity{0.000000}%
\pgfsetdash{}{0pt}%
\pgfpathmoveto{\pgfqpoint{0.198611in}{0.198611in}}%
\pgfpathlineto{\pgfqpoint{3.801389in}{0.198611in}}%
\pgfpathlineto{\pgfqpoint{3.801389in}{2.601389in}}%
\pgfpathlineto{\pgfqpoint{0.198611in}{2.601389in}}%
\pgfpathclose%
\pgfusepath{fill}%
\end{pgfscope}%
\begin{pgfscope}%
\pgfpathrectangle{\pgfqpoint{0.198611in}{0.198611in}}{\pgfqpoint{3.602778in}{2.402778in}}%
\pgfusepath{clip}%
\pgfsetbuttcap%
\pgfsetmiterjoin%
\definecolor{currentfill}{rgb}{0.500000,0.500000,0.500000}%
\pgfsetfillcolor{currentfill}%
\pgfsetfillopacity{0.500000}%
\pgfsetlinewidth{0.501875pt}%
\definecolor{currentstroke}{rgb}{0.000000,0.000000,0.000000}%
\pgfsetstrokecolor{currentstroke}%
\pgfsetdash{}{0pt}%
\pgfpathmoveto{\pgfqpoint{1.963972in}{1.433372in}}%
\pgfpathlineto{\pgfqpoint{2.036028in}{1.433372in}}%
\pgfpathlineto{\pgfqpoint{2.144111in}{1.533488in}}%
\pgfpathlineto{\pgfqpoint{1.855889in}{1.533488in}}%
\pgfpathclose%
\pgfusepath{stroke,fill}%
\end{pgfscope}%
\begin{pgfscope}%
\pgfpathrectangle{\pgfqpoint{0.198611in}{0.198611in}}{\pgfqpoint{3.602778in}{2.402778in}}%
\pgfusepath{clip}%
\pgfsetbuttcap%
\pgfsetmiterjoin%
\definecolor{currentfill}{rgb}{0.500000,0.500000,0.500000}%
\pgfsetfillcolor{currentfill}%
\pgfsetfillopacity{0.500000}%
\pgfsetlinewidth{0.501875pt}%
\definecolor{currentstroke}{rgb}{0.000000,0.000000,0.000000}%
\pgfsetstrokecolor{currentstroke}%
\pgfsetdash{}{0pt}%
\pgfpathmoveto{\pgfqpoint{2.036028in}{1.366628in}}%
\pgfpathlineto{\pgfqpoint{1.963972in}{1.366628in}}%
\pgfpathlineto{\pgfqpoint{1.855889in}{1.266512in}}%
\pgfpathlineto{\pgfqpoint{2.144111in}{1.266512in}}%
\pgfpathclose%
\pgfusepath{stroke,fill}%
\end{pgfscope}%
\begin{pgfscope}%
\pgfpathrectangle{\pgfqpoint{0.198611in}{0.198611in}}{\pgfqpoint{3.602778in}{2.402778in}}%
\pgfusepath{clip}%
\pgfsetbuttcap%
\pgfsetmiterjoin%
\definecolor{currentfill}{rgb}{0.500000,0.500000,0.500000}%
\pgfsetfillcolor{currentfill}%
\pgfsetfillopacity{0.500000}%
\pgfsetlinewidth{0.501875pt}%
\definecolor{currentstroke}{rgb}{0.000000,0.000000,0.000000}%
\pgfsetstrokecolor{currentstroke}%
\pgfsetdash{}{0pt}%
\pgfpathmoveto{\pgfqpoint{2.246348in}{1.733719in}}%
\pgfpathlineto{\pgfqpoint{2.474208in}{1.733719in}}%
\pgfpathlineto{\pgfqpoint{3.896830in}{2.734877in}}%
\pgfpathlineto{\pgfqpoint{2.985392in}{2.734877in}}%
\pgfpathclose%
\pgfusepath{stroke,fill}%
\end{pgfscope}%
\begin{pgfscope}%
\pgfpathrectangle{\pgfqpoint{0.198611in}{0.198611in}}{\pgfqpoint{3.602778in}{2.402778in}}%
\pgfusepath{clip}%
\pgfsetbuttcap%
\pgfsetmiterjoin%
\definecolor{currentfill}{rgb}{0.500000,0.500000,0.500000}%
\pgfsetfillcolor{currentfill}%
\pgfsetfillopacity{0.500000}%
\pgfsetlinewidth{0.501875pt}%
\definecolor{currentstroke}{rgb}{0.000000,0.000000,0.000000}%
\pgfsetstrokecolor{currentstroke}%
\pgfsetdash{}{0pt}%
\pgfpathmoveto{\pgfqpoint{1.753652in}{1.066281in}}%
\pgfpathlineto{\pgfqpoint{1.525792in}{1.066281in}}%
\pgfpathlineto{\pgfqpoint{0.103170in}{0.065123in}}%
\pgfpathlineto{\pgfqpoint{1.014608in}{0.065123in}}%
\pgfpathclose%
\pgfusepath{stroke,fill}%
\end{pgfscope}%
\begin{pgfscope}%
\pgfpathrectangle{\pgfqpoint{0.198611in}{0.198611in}}{\pgfqpoint{3.602778in}{2.402778in}}%
\pgfusepath{clip}%
\pgfsetbuttcap%
\pgfsetroundjoin%
\pgfsetlinewidth{0.501875pt}%
\definecolor{currentstroke}{rgb}{0.501961,0.501961,0.501961}%
\pgfsetstrokecolor{currentstroke}%
\pgfsetdash{{1.850000pt}{0.800000pt}}{0.000000pt}%
\pgfpathmoveto{\pgfqpoint{0.688006in}{0.184722in}}%
\pgfpathlineto{\pgfqpoint{3.311994in}{2.615278in}}%
\pgfpathlineto{\pgfqpoint{3.311994in}{2.615278in}}%
\pgfusepath{stroke}%
\end{pgfscope}%
\begin{pgfscope}%
\pgfpathrectangle{\pgfqpoint{0.198611in}{0.198611in}}{\pgfqpoint{3.602778in}{2.402778in}}%
\pgfusepath{clip}%
\pgfsetbuttcap%
\pgfsetroundjoin%
\pgfsetlinewidth{0.501875pt}%
\definecolor{currentstroke}{rgb}{0.501961,0.501961,0.501961}%
\pgfsetstrokecolor{currentstroke}%
\pgfsetdash{{1.850000pt}{0.800000pt}}{0.000000pt}%
\pgfpathmoveto{\pgfqpoint{0.688006in}{2.615278in}}%
\pgfpathlineto{\pgfqpoint{3.311994in}{0.184722in}}%
\pgfpathlineto{\pgfqpoint{3.311994in}{0.184722in}}%
\pgfusepath{stroke}%
\end{pgfscope}%
\begin{pgfscope}%
\pgfsetrectcap%
\pgfsetmiterjoin%
\pgfsetlinewidth{0.501875pt}%
\definecolor{currentstroke}{rgb}{0.000000,0.000000,0.000000}%
\pgfsetstrokecolor{currentstroke}%
\pgfsetdash{}{0pt}%
\pgfpathmoveto{\pgfqpoint{2.000000in}{0.198611in}}%
\pgfpathlineto{\pgfqpoint{2.000000in}{2.601389in}}%
\pgfusepath{stroke}%
\end{pgfscope}%
\begin{pgfscope}%
\pgfsetrectcap%
\pgfsetmiterjoin%
\pgfsetlinewidth{0.501875pt}%
\definecolor{currentstroke}{rgb}{0.000000,0.000000,0.000000}%
\pgfsetstrokecolor{currentstroke}%
\pgfsetdash{}{0pt}%
\pgfpathmoveto{\pgfqpoint{0.198611in}{1.400000in}}%
\pgfpathlineto{\pgfqpoint{3.801389in}{1.400000in}}%
\pgfusepath{stroke}%
\end{pgfscope}%
\begin{pgfscope}%
\pgfsetroundcap%
\pgfsetroundjoin%
\pgfsetlinewidth{0.501875pt}%
\definecolor{currentstroke}{rgb}{0.000000,0.000000,0.000000}%
\pgfsetstrokecolor{currentstroke}%
\pgfsetdash{}{0pt}%
\pgfpathmoveto{\pgfqpoint{3.032230in}{2.375700in}}%
\pgfpathquadraticcurveto{\pgfqpoint{2.526526in}{1.930390in}}{\pgfqpoint{2.026649in}{1.490210in}}%
\pgfusepath{stroke}%
\end{pgfscope}%
\begin{pgfscope}%
\pgfsetroundcap%
\pgfsetroundjoin%
\pgfsetlinewidth{0.501875pt}%
\definecolor{currentstroke}{rgb}{0.000000,0.000000,0.000000}%
\pgfsetstrokecolor{currentstroke}%
\pgfsetdash{}{0pt}%
\pgfpathmoveto{\pgfqpoint{2.086701in}{1.506078in}}%
\pgfpathlineto{\pgfqpoint{2.026649in}{1.490210in}}%
\pgfpathlineto{\pgfqpoint{2.049986in}{1.547773in}}%
\pgfusepath{stroke}%
\end{pgfscope}%
\begin{pgfscope}%
\pgftext[x=3.080833in,y=2.401157in,left,base]{\rmfamily\fontsize{10.000000}{12.000000}\selectfont \(\displaystyle {\cdot}\)}%
\end{pgfscope}%
\begin{pgfscope}%
\pgftext[x=2.288222in,y=1.600231in,left,base]{\rmfamily\fontsize{10.000000}{12.000000}\selectfont Substitution 1}%
\end{pgfscope}%
\begin{pgfscope}%
\pgfsetroundcap%
\pgfsetroundjoin%
\pgfsetlinewidth{0.501875pt}%
\definecolor{currentstroke}{rgb}{0.000000,0.000000,0.000000}%
\pgfsetstrokecolor{currentstroke}%
\pgfsetdash{}{0pt}%
\pgfpathmoveto{\pgfqpoint{2.000000in}{2.607510in}}%
\pgfpathquadraticcurveto{\pgfqpoint{2.000000in}{2.608331in}}{\pgfqpoint{2.000000in}{2.601389in}}%
\pgfusepath{stroke}%
\end{pgfscope}%
\begin{pgfscope}%
\pgfsetroundcap%
\pgfsetroundjoin%
\pgfsetlinewidth{0.501875pt}%
\definecolor{currentstroke}{rgb}{0.000000,0.000000,0.000000}%
\pgfsetstrokecolor{currentstroke}%
\pgfsetdash{}{0pt}%
\pgfpathmoveto{\pgfqpoint{1.972222in}{2.551954in}}%
\pgfpathlineto{\pgfqpoint{2.000000in}{2.607510in}}%
\pgfpathlineto{\pgfqpoint{2.027778in}{2.551954in}}%
\pgfusepath{stroke}%
\end{pgfscope}%
\begin{pgfscope}%
\pgftext[x=2.000000in,y=2.670833in,,bottom]{\rmfamily\fontsize{10.000000}{12.000000}\selectfont \(\displaystyle \omega\)}%
\end{pgfscope}%
\begin{pgfscope}%
\pgfsetroundcap%
\pgfsetroundjoin%
\pgfsetlinewidth{0.501875pt}%
\definecolor{currentstroke}{rgb}{0.000000,0.000000,0.000000}%
\pgfsetstrokecolor{currentstroke}%
\pgfsetdash{}{0pt}%
\pgfpathmoveto{\pgfqpoint{3.807488in}{1.400000in}}%
\pgfpathquadraticcurveto{\pgfqpoint{3.808320in}{1.400000in}}{\pgfqpoint{3.801389in}{1.400000in}}%
\pgfusepath{stroke}%
\end{pgfscope}%
\begin{pgfscope}%
\pgfsetroundcap%
\pgfsetroundjoin%
\pgfsetlinewidth{0.501875pt}%
\definecolor{currentstroke}{rgb}{0.000000,0.000000,0.000000}%
\pgfsetstrokecolor{currentstroke}%
\pgfsetdash{}{0pt}%
\pgfpathmoveto{\pgfqpoint{3.751932in}{1.427778in}}%
\pgfpathlineto{\pgfqpoint{3.807488in}{1.400000in}}%
\pgfpathlineto{\pgfqpoint{3.751932in}{1.372222in}}%
\pgfusepath{stroke}%
\end{pgfscope}%
\begin{pgfscope}%
\pgftext[x=3.870833in,y=1.400000in,left,]{\rmfamily\fontsize{10.000000}{12.000000}\selectfont \(\displaystyle k\)}%
\end{pgfscope}%
\end{pgfpicture}%
\makeatother%
\endgroup%
} %
        \caption{Der Träger von $\hat\psi$ vor und nach der Substitution aus \cref{eq:substitution1_coords}}
        \label{fig:supp_psi_substitution1}
    \end{minipage}\hfill
    \begin{minipage}{0.5\textwidth}
        \centering
        \resizebox{\textwidth}{!}{%% Creator: Matplotlib, PGF backend
%%
%% To include the figure in your LaTeX document, write
%%   \input{<filename>.pgf}
%%
%% Make sure the required packages are loaded in your preamble
%%   \usepackage{pgf}
%%
%% Figures using additional raster images can only be included by \input if
%% they are in the same directory as the main LaTeX file. For loading figures
%% from other directories you can use the `import` package
%%   \usepackage{import}
%% and then include the figures with
%%   \import{<path to file>}{<filename>.pgf}
%%
%% Matplotlib used the following preamble
%%   \usepackage[utf8x]{inputenc}
%%   \usepackage[T1]{fontenc}
%%   \usepackage{amssymb}
%%
\begingroup%
\makeatletter%
\begin{pgfpicture}%
\pgfpathrectangle{\pgfpointorigin}{\pgfqpoint{4.000000in}{2.000000in}}%
\pgfusepath{use as bounding box, clip}%
\begin{pgfscope}%
\pgfsetbuttcap%
\pgfsetmiterjoin%
\definecolor{currentfill}{rgb}{1.000000,1.000000,1.000000}%
\pgfsetfillcolor{currentfill}%
\pgfsetlinewidth{0.000000pt}%
\definecolor{currentstroke}{rgb}{1.000000,1.000000,1.000000}%
\pgfsetstrokecolor{currentstroke}%
\pgfsetdash{}{0pt}%
\pgfpathmoveto{\pgfqpoint{0.000000in}{0.000000in}}%
\pgfpathlineto{\pgfqpoint{4.000000in}{0.000000in}}%
\pgfpathlineto{\pgfqpoint{4.000000in}{2.000000in}}%
\pgfpathlineto{\pgfqpoint{0.000000in}{2.000000in}}%
\pgfpathclose%
\pgfusepath{fill}%
\end{pgfscope}%
\begin{pgfscope}%
\pgfsetbuttcap%
\pgfsetmiterjoin%
\definecolor{currentfill}{rgb}{1.000000,1.000000,1.000000}%
\pgfsetfillcolor{currentfill}%
\pgfsetlinewidth{0.000000pt}%
\definecolor{currentstroke}{rgb}{0.000000,0.000000,0.000000}%
\pgfsetstrokecolor{currentstroke}%
\pgfsetstrokeopacity{0.000000}%
\pgfsetdash{}{0pt}%
\pgfpathmoveto{\pgfqpoint{0.198611in}{0.198611in}}%
\pgfpathlineto{\pgfqpoint{3.801389in}{0.198611in}}%
\pgfpathlineto{\pgfqpoint{3.801389in}{1.801389in}}%
\pgfpathlineto{\pgfqpoint{0.198611in}{1.801389in}}%
\pgfpathclose%
\pgfusepath{fill}%
\end{pgfscope}%
\begin{pgfscope}%
\pgfpathrectangle{\pgfqpoint{0.198611in}{0.198611in}}{\pgfqpoint{3.602778in}{1.602778in}} %
\pgfusepath{clip}%
\pgfsetbuttcap%
\pgfsetmiterjoin%
\definecolor{currentfill}{rgb}{0.500000,0.500000,0.500000}%
\pgfsetfillcolor{currentfill}%
\pgfsetfillopacity{0.500000}%
\pgfsetlinewidth{0.501875pt}%
\definecolor{currentstroke}{rgb}{0.000000,0.000000,0.000000}%
\pgfsetstrokecolor{currentstroke}%
\pgfsetdash{}{0pt}%
\pgfpathmoveto{\pgfqpoint{1.909931in}{0.398958in}}%
\pgfpathlineto{\pgfqpoint{1.909931in}{0.519167in}}%
\pgfpathlineto{\pgfqpoint{2.090069in}{0.519167in}}%
\pgfpathlineto{\pgfqpoint{2.090069in}{0.398958in}}%
\pgfpathclose%
\pgfusepath{stroke,fill}%
\end{pgfscope}%
\begin{pgfscope}%
\pgfpathrectangle{\pgfqpoint{0.198611in}{0.198611in}}{\pgfqpoint{3.602778in}{1.602778in}} %
\pgfusepath{clip}%
\pgfsetbuttcap%
\pgfsetmiterjoin%
\definecolor{currentfill}{rgb}{0.500000,0.500000,0.500000}%
\pgfsetfillcolor{currentfill}%
\pgfsetfillopacity{0.500000}%
\pgfsetlinewidth{0.501875pt}%
\definecolor{currentstroke}{rgb}{0.000000,0.000000,0.000000}%
\pgfsetstrokecolor{currentstroke}%
\pgfsetdash{}{0pt}%
\pgfpathmoveto{\pgfqpoint{2.307935in}{0.759583in}}%
\pgfpathlineto{\pgfqpoint{2.592760in}{0.759583in}}%
\pgfpathlineto{\pgfqpoint{4.371038in}{1.961667in}}%
\pgfpathlineto{\pgfqpoint{3.231740in}{1.961667in}}%
\pgfpathclose%
\pgfusepath{stroke,fill}%
\end{pgfscope}%
\begin{pgfscope}%
\pgfpathrectangle{\pgfqpoint{0.198611in}{0.198611in}}{\pgfqpoint{3.602778in}{1.602778in}} %
\pgfusepath{clip}%
\pgfsetbuttcap%
\pgfsetroundjoin%
\pgfsetlinewidth{0.501875pt}%
\definecolor{currentstroke}{rgb}{0.501961,0.501961,0.501961}%
\pgfsetstrokecolor{currentstroke}%
\pgfsetdash{{1.850000pt}{0.800000pt}}{0.000000pt}%
\pgfpathmoveto{\pgfqpoint{1.804251in}{0.184722in}}%
\pgfpathlineto{\pgfqpoint{3.636860in}{1.815278in}}%
\pgfpathlineto{\pgfqpoint{3.636860in}{1.815278in}}%
\pgfusepath{stroke}%
\end{pgfscope}%
\begin{pgfscope}%
\pgfpathrectangle{\pgfqpoint{0.198611in}{0.198611in}}{\pgfqpoint{3.602778in}{1.602778in}} %
\pgfusepath{clip}%
\pgfsetbuttcap%
\pgfsetroundjoin%
\pgfsetlinewidth{0.501875pt}%
\definecolor{currentstroke}{rgb}{0.501961,0.501961,0.501961}%
\pgfsetstrokecolor{currentstroke}%
\pgfsetdash{{1.850000pt}{0.800000pt}}{0.000000pt}%
\pgfpathmoveto{\pgfqpoint{0.363140in}{1.815278in}}%
\pgfpathlineto{\pgfqpoint{2.195749in}{0.184722in}}%
\pgfpathlineto{\pgfqpoint{2.195749in}{0.184722in}}%
\pgfusepath{stroke}%
\end{pgfscope}%
\begin{pgfscope}%
\pgfsetrectcap%
\pgfsetmiterjoin%
\pgfsetlinewidth{0.501875pt}%
\definecolor{currentstroke}{rgb}{0.000000,0.000000,0.000000}%
\pgfsetstrokecolor{currentstroke}%
\pgfsetdash{}{0pt}%
\pgfpathmoveto{\pgfqpoint{2.000000in}{0.198611in}}%
\pgfpathlineto{\pgfqpoint{2.000000in}{1.801389in}}%
\pgfusepath{stroke}%
\end{pgfscope}%
\begin{pgfscope}%
\pgfsetrectcap%
\pgfsetmiterjoin%
\pgfsetlinewidth{0.501875pt}%
\definecolor{currentstroke}{rgb}{0.000000,0.000000,0.000000}%
\pgfsetstrokecolor{currentstroke}%
\pgfsetdash{}{0pt}%
\pgfpathmoveto{\pgfqpoint{0.198611in}{0.358889in}}%
\pgfpathlineto{\pgfqpoint{3.801389in}{0.358889in}}%
\pgfusepath{stroke}%
\end{pgfscope}%
\begin{pgfscope}%
\pgfsetroundcap%
\pgfsetroundjoin%
\pgfsetlinewidth{0.501875pt}%
\definecolor{currentstroke}{rgb}{0.000000,0.000000,0.000000}%
\pgfsetstrokecolor{currentstroke}%
\pgfsetdash{}{0pt}%
\pgfpathmoveto{\pgfqpoint{3.302084in}{1.537713in}}%
\pgfpathquadraticcurveto{\pgfqpoint{2.661671in}{0.997340in}}{\pgfqpoint{2.027193in}{0.461973in}}%
\pgfusepath{stroke}%
\end{pgfscope}%
\begin{pgfscope}%
\pgfsetroundcap%
\pgfsetroundjoin%
\pgfsetlinewidth{0.501875pt}%
\definecolor{currentstroke}{rgb}{0.000000,0.000000,0.000000}%
\pgfsetstrokecolor{currentstroke}%
\pgfsetdash{}{0pt}%
\pgfpathmoveto{\pgfqpoint{2.087566in}{0.476570in}}%
\pgfpathlineto{\pgfqpoint{2.027193in}{0.461973in}}%
\pgfpathlineto{\pgfqpoint{2.051739in}{0.519030in}}%
\pgfusepath{stroke}%
\end{pgfscope}%
\begin{pgfscope}%
\pgftext[x=3.351042in,y=1.560972in,left,base]{\rmfamily\fontsize{10.000000}{12.000000}\selectfont \(\displaystyle {\cdot}\)}%
\end{pgfscope}%
\begin{pgfscope}%
\pgftext[x=2.360278in,y=0.599306in,left,base]{\rmfamily\fontsize{10.000000}{12.000000}\selectfont Substitution 1}%
\end{pgfscope}%
\begin{pgfscope}%
\pgfsetroundcap%
\pgfsetroundjoin%
\pgfsetlinewidth{0.501875pt}%
\definecolor{currentstroke}{rgb}{0.000000,0.000000,0.000000}%
\pgfsetstrokecolor{currentstroke}%
\pgfsetdash{}{0pt}%
\pgfpathmoveto{\pgfqpoint{2.000000in}{1.807510in}}%
\pgfpathquadraticcurveto{\pgfqpoint{2.000000in}{1.808331in}}{\pgfqpoint{2.000000in}{1.801389in}}%
\pgfusepath{stroke}%
\end{pgfscope}%
\begin{pgfscope}%
\pgfsetroundcap%
\pgfsetroundjoin%
\pgfsetlinewidth{0.501875pt}%
\definecolor{currentstroke}{rgb}{0.000000,0.000000,0.000000}%
\pgfsetstrokecolor{currentstroke}%
\pgfsetdash{}{0pt}%
\pgfpathmoveto{\pgfqpoint{1.972222in}{1.751954in}}%
\pgfpathlineto{\pgfqpoint{2.000000in}{1.807510in}}%
\pgfpathlineto{\pgfqpoint{2.027778in}{1.751954in}}%
\pgfusepath{stroke}%
\end{pgfscope}%
\begin{pgfscope}%
\pgftext[x=2.000000in,y=1.870833in,,bottom]{\rmfamily\fontsize{10.000000}{12.000000}\selectfont \(\displaystyle \omega\)}%
\end{pgfscope}%
\begin{pgfscope}%
\pgfsetroundcap%
\pgfsetroundjoin%
\pgfsetlinewidth{0.501875pt}%
\definecolor{currentstroke}{rgb}{0.000000,0.000000,0.000000}%
\pgfsetstrokecolor{currentstroke}%
\pgfsetdash{}{0pt}%
\pgfpathmoveto{\pgfqpoint{3.807488in}{0.358889in}}%
\pgfpathquadraticcurveto{\pgfqpoint{3.808320in}{0.358889in}}{\pgfqpoint{3.801389in}{0.358889in}}%
\pgfusepath{stroke}%
\end{pgfscope}%
\begin{pgfscope}%
\pgfsetroundcap%
\pgfsetroundjoin%
\pgfsetlinewidth{0.501875pt}%
\definecolor{currentstroke}{rgb}{0.000000,0.000000,0.000000}%
\pgfsetstrokecolor{currentstroke}%
\pgfsetdash{}{0pt}%
\pgfpathmoveto{\pgfqpoint{3.751932in}{0.386667in}}%
\pgfpathlineto{\pgfqpoint{3.807488in}{0.358889in}}%
\pgfpathlineto{\pgfqpoint{3.751932in}{0.331111in}}%
\pgfusepath{stroke}%
\end{pgfscope}%
\begin{pgfscope}%
\pgftext[x=3.870833in,y=0.358889in,left,]{\rmfamily\fontsize{10.000000}{12.000000}\selectfont \(\displaystyle k\)}%
\end{pgfscope}%
\end{pgfpicture}%
\makeatother%
\endgroup%
}
        \caption{Der Träger von $\hat\psi$ vor und nach der Substitution aus \cref{eq:substitution2_coords}}
        \label{fig:supp_psi_substitution2}
    \end{minipage}
\end{figure}

Sei also $\psi$ ein Shearlet wie in \cref{rem:psi_hat}. Sei $f$ die zu
analysierende fouriertransformierbare Funktion (oder Distribution) in
$\mathcal{S}' (\mathbb{R}^2)$. Dann ist $\mathcal{S}_f (ast)$ gegeben durch

\begin{align*}
\left<f, \psi_{ast}\right> &= \left< \hat f, \hat\psi_{ast}\right> \\
 &= \int a^{\frac{3}{4}} e^{-i \omega t + ikx} \hat \psi_1(a \omega)
    \hat \psi_2 \left(a^{-\frac{1}{2}} \left(\frac{k}{\omega} - s\right)\right)
    \hat f (\omega,k) \d \omega \d k
\end{align*}

und nach "`entscheren"' und "`deskalieren"', also der Substitution

\begin{equation}
\begin{aligned}[c]
a \omega_1 &= \omega'\\
a^{-\frac{1}{2}} \left(\frac{k}{\omega} - s\right) &=\frac{k'}{\omega'}\\
\end{aligned}
\qquad\Longleftrightarrow\qquad
\begin{aligned}[c]
\omega &= \frac{\omega'}{a}\\
k &= \frac{\omega' s}{a} + a^{-\frac{1}{2}} k'\\
\end{aligned}
\label{eq:substitution1_coords}
\end{equation}

\begin{equation*}
\Rightarrow
\d \omega \d k = a^{-\frac{3}{2}} \d \omega' \d k'
\end{equation*}

ergibt sich folgendes für $\left<\psi_{ast}, f\right>$:

\begin{align}
    \left\langle f , \psi_{ast}\right\rangle
    &=  \left\langle \hat f, \hat\psi_{ast}\right\rangle \nonumber \\
    &=  \iint a^{-\frac{3}{4}}~\hat \psi_1(\omega') ~\hat \psi_2 \left(\tfrac{k'}{\omega'}\right)
    ~\hat f \left(\tfrac{\omega'}{a}, \tfrac{\omega' s}{a} + \tfrac{k'}{\sqrt{a}}\right)
    ~e^{-i\frac{\omega'}{a}(t'+sx') - i \frac{k' x'}{\sqrt a}}
    \d \omega' \d k'
\refstepcounter{equation}
\tag{Substitution 1, (\theequation)}
\label{eq:substitution1}
\end{align}

Wie man sieht, tauchen in den Argumente von $\hat\psi_1$ und $\hat\psi_2$ nun die Parameter $a,s,t$ gar nicht mehr auf, und wir können nun verwenden, was wir aus \eqref{sec:shearlets} über deren Träger wissen.
Alternativ und mit ähnlichem Ergebnis kann auch folgende Substitution

\begin{equation}
\begin{aligned}[c]
a \omega &= \omega'\\
a^{-\frac{1}{2}} \left(\frac{k}{\omega} - s\right) &= k'\\
\end{aligned}
\qquad\Longleftrightarrow\qquad
\begin{aligned}[c]
\omega &= \frac{\omega'}{a}\\
k &= \left( a^{\frac{1}{2}} k' +s \right) \frac{\omega'}{a}\\
\end{aligned}
\label{eq:substitution2_coords}
\end{equation}

\begin{equation*}
\Rightarrow
\d \omega \d k = a^{-\frac{3}{2}} \omega \d \omega' \d k'
\end{equation*}

gewählt werden, wodurch wieder alle Parameter $(a,s,t)$ aus den Argumenten von $\hat\psi_1, \hat\psi_2$
verschwinden und sich

\begin{align}
    \left<f, \psi_{ast}\right>
    =  \iint a^{-\frac{3}{4}}~ k_1~ \hat \psi_1(\omega')~ \hat \psi_2 (k')~
    \hat f \left(\tfrac{\omega'}{a}, \omega' \left(a^{-\frac{1}{2}}k' + s a^{-1}\right)\right)
    ~e^{-i \omega' \left(\frac{t'+s x'}{a} + \frac{k' x'}{\sqrt{a}}\right)}
    \d \omega' \d k'
\refstepcounter{equation}
\tag{Substitution 2, (\theequation)}
\label{eq:substitution2}
\end{align}

ergibt. Dabei ist zu beachten, dass diese Substitution zulässig ist, obwohl sie
die Orientierung \emph{nicht} erhält und \emph{keine} Bijektion ist. Aber
der kritische Bereich, nämlich $\omega = 0$, liegt nicht im Träger von $\hat{\psi}$.

Beiden Substitution gemein ist aber, dass danach
$0=\omega \notin supp (\hat\psi)$ und dass $supp (\psi)$ sowohl in $k$ als auch in $\omega$ beschränkt ist. $\omega$ kann also sowohl nach unten als auch nach oben durch eine Konstante abgeschätzt werden, wann immer dies der Sache dienlich ist. Auch $k$ kann, zumindest nach oben, immer durch eine Konstante abgeschätzt werden.

Wie man \cref{eq:substitution1,eq:substitution2} ansieht, haben wir schließlich einen Ausdruck der Form \(\lim_{a \to 0} \int \hat f(a,k) e^{ik \frac{x}{a}} \d k\) abzuschätzen. Eine Möglichkeit für diese Abschätzungen liefert das folgende Lemma:

\begin{lemma}[$\int \hat f(a,k) e^{ik \frac{x}{a}} \d k$ abschätzen]
\label{lemm:f_a_abschaetzen}
 Sei \(\hat f:\mathbb{R} \times \hat{\mathbb{R}}^m \to \mathbb{C}; \quad (a,k) \mapsto \hat f(a,k)\) kompakt getragen in $k$ und s.d. \[\hat f(a,k) = \sum_{n=0}^\infty a^n \hat f_n(k) \] für alle hinreichend kleinen $a$ und $\hat f_n \in C^{N_n}_c (\hat{\mathbb{R}}^n)$. Die Potenzreihe sei punktweise absolut konvergent\footnote{Wie es z.B. innerhalb des Konvergenzradius von Taylorreihen gegeben ist.}. Sei $\delta > 0$ und $\mathbb{R}^n \ni x \neq 0$. Sei des weiteren \(p = \sup \{n+N_n \, \delta | n \in \mathbb{N}\}\). Dann gilt:
 \begin{equation*}
     \int \hat f(a,k) e^{ik\frac{x}{a^\delta}} \d k = O(a^p) \condition{für \(a \to 0\)}
 \end{equation*}
\end{lemma}

\begin{proof}
Da \(\hat f_n \in C^{N_n}_c (\hat{\mathbb{R}^n})\) ist, gilt auch \(f_n(x) = O(x^{-N_n})\) für $|x| \to \infty$, wobei wie erwartet \(f(x) = \hat f^\vee (x)\) ist. Dann können wir abschätzen:
\begin{align*}
    \lim_{a \to 0}
    &= \int \hat f(a,k) e^{ik\frac{x}{a^\delta}} \d k
    = \sum_n a^n \int \hat f_n(k)e^{ik \frac{x}{a^\delta}} \d k
    \\ &=
    \sum_n a^n\, \underbrace{\hat f_n\left(\frac{x}{a^\delta}\right)}_{O\left(\left(\frac{x}{a^\delta}\right)^{-N_n}\right)}
    = \sum_n O(a^n a^{N_n \,\delta}) = O(a^p)
 \end{align*}
\end{proof}

Bevor wir dann mit den konkreten Rechnungen beginnnen können, brauchen wir noch ein letztes Lemma über das Verhalten der Wellenfrontmengen Lorentz-invarianter Distributionen unter Lorentz-Transformationen.

\begin{lemma}[Wellenfrontmengen Lorentz-invarianter Distributionen]
\label{lemm:wavefrontset_and_lorentz}
Sei \(f \in \mathcal{D}'(\mathbb{R}^{1+d})\) lorentz-invariant und \(\Lambda \in \mathrm{SO}(d,1)\). Sei des Weiteren \(\phi_{x_0}:\mathbb{R}^{1+d} \to \mathbb{R}\) eine bei \(x_0 \in \mathbb{R}^{1+d}\) lokalisierte Testfunktion im Sinne von \cref{def:high_frequency_set}.

Dann ist \(\phi_{\Lambda x_0} (\cdot) \coloneqq\phi_{x_0} (\Lambda^{-1} \cdot)\) eine bei \(\Lambda x_o\) lokalisierte Testfunktion und es gilt für alle \(k \in \hat{\mathbb{R}}^{1+d}\)
\begin{equation*}
    \rwhat{f \phi_{x_0}} (k)  = \rwhat{f \phi_{\Lambda x_0}} (\Lambda k).
\end{equation*}
\end{lemma}

\begin{corollary}
\label{cor:wavefrontset_lorentz}
Seien \(f, \Lambda, x_0, k\) wie eben. Dann gilt
\begin{equation*}
    (x_0, k) \in WF(f)
    \quad \Longleftrightarrow \quad
    (\Lambda x_0, \Lambda k) \in WF(f)
\end{equation*}
\end{corollary}

\begin{proof}
    Mit formaler Rechnung folgt:
    \begin{align*}
        \rwhat{f \phi_{x_0}} (k) &=
        \int f(x) \, \phi_{x_0}(x)\, e^{-ik\cdot x} \d x
        = \int f(\Lambda^{-1}x ) \, \phi_{x_0}(\Lambda^{-1} x) \, e^{-ik\cdot\Lambda^{-1}x} \d x
        \\ &=
        \int f(x)\, \phi_{\Lambda x_0}(x) \, e^{-i\Lambda k \cdot x} \d x
        = \rwhat{f \phi_{\Lambda x_0}} (\Lambda k)
    \end{align*}
\end{proof}












%!TEX root = main.tex
%%%%%%%%%%%%%%%%%%%%%%%%%%%%%%%%%%%%%%%%%%%%%%%%%%%%%%%%%%%%%%%%%%%%%%%%%%%%%%%
% % Berechnen der Wellenfrontmenge von Delta_m
%%%%%%%%%%%%%%%%%%%%%%%%%%%%%%%%%%%%%%%%%%%%%%%%%%%%%%%%%%%%%%%%%%%%%%%%%%%%%%%%

\section{\texorpdfstring{Die Wellenfrontmenge von $\Delta_m$}
        {Die Wellenfrontmenge von Delta m}} % (fold)
\label{sec:die_wellenfrontmenge_von_delta_m}

Die massive Zweipunktfunktion ist die Fouriertransformierte der 1$m$-Massenschale positiver Energie:
\todo{Quelle dafür zitiesfdren...}

\begin{equation}
    \Delta_m (t,x) = \int \delta (\omega^2-k^2-m^2)
                    \Theta(\omega)e^{-i\omega t + i k x} \d \omega \d k
\end{equation}

woraus sich $\hat \Delta_m$ direkt zu $\delta (\omega^2-k^2-m^2)\Theta(\omega)$
ablesen lässt.

\begin{figure}[h]
\centering
%% Creator: Matplotlib, PGF backend
%%
%% To include the figure in your LaTeX document, write
%%   \input{<filename>.pgf}
%%
%% Make sure the required packages are loaded in your preamble
%%   \usepackage{pgf}
%%
%% Figures using additional raster images can only be included by \input if
%% they are in the same directory as the main LaTeX file. For loading figures
%% from other directories you can use the `import` package
%%   \usepackage{import}
%% and then include the figures with
%%   \import{<path to file>}{<filename>.pgf}
%%
%% Matplotlib used the following preamble
%%   \usepackage[utf8x]{inputenc}
%%   \usepackage[T1]{fontenc}
%%   \usepackage{amssymb}
%%
\begingroup%
\makeatletter%
\begin{pgfpicture}%
\pgfpathrectangle{\pgfpointorigin}{\pgfqpoint{4.000000in}{2.350000in}}%
\pgfusepath{use as bounding box, clip}%
\begin{pgfscope}%
\pgfsetbuttcap%
\pgfsetmiterjoin%
\definecolor{currentfill}{rgb}{1.000000,1.000000,1.000000}%
\pgfsetfillcolor{currentfill}%
\pgfsetlinewidth{0.000000pt}%
\definecolor{currentstroke}{rgb}{1.000000,1.000000,1.000000}%
\pgfsetstrokecolor{currentstroke}%
\pgfsetdash{}{0pt}%
\pgfpathmoveto{\pgfqpoint{0.000000in}{0.000000in}}%
\pgfpathlineto{\pgfqpoint{4.000000in}{0.000000in}}%
\pgfpathlineto{\pgfqpoint{4.000000in}{2.350000in}}%
\pgfpathlineto{\pgfqpoint{0.000000in}{2.350000in}}%
\pgfpathclose%
\pgfusepath{fill}%
\end{pgfscope}%
\begin{pgfscope}%
\pgfsetbuttcap%
\pgfsetmiterjoin%
\definecolor{currentfill}{rgb}{1.000000,1.000000,1.000000}%
\pgfsetfillcolor{currentfill}%
\pgfsetlinewidth{0.000000pt}%
\definecolor{currentstroke}{rgb}{0.000000,0.000000,0.000000}%
\pgfsetstrokecolor{currentstroke}%
\pgfsetstrokeopacity{0.000000}%
\pgfsetdash{}{0pt}%
\pgfpathmoveto{\pgfqpoint{0.198611in}{0.198611in}}%
\pgfpathlineto{\pgfqpoint{3.801389in}{0.198611in}}%
\pgfpathlineto{\pgfqpoint{3.801389in}{2.151389in}}%
\pgfpathlineto{\pgfqpoint{0.198611in}{2.151389in}}%
\pgfpathclose%
\pgfusepath{fill}%
\end{pgfscope}%
\begin{pgfscope}%
\pgfpathrectangle{\pgfqpoint{0.198611in}{0.198611in}}{\pgfqpoint{3.602778in}{1.952778in}} %
\pgfusepath{clip}%
\pgfsetbuttcap%
\pgfsetmiterjoin%
\definecolor{currentfill}{rgb}{0.501961,0.501961,0.501961}%
\pgfsetfillcolor{currentfill}%
\pgfsetfillopacity{0.500000}%
\pgfsetlinewidth{0.501875pt}%
\definecolor{currentstroke}{rgb}{0.501961,0.501961,0.501961}%
\pgfsetstrokecolor{currentstroke}%
\pgfsetstrokeopacity{0.500000}%
\pgfsetdash{}{0pt}%
\pgfpathmoveto{\pgfqpoint{1.837063in}{0.365312in}}%
\pgfpathlineto{\pgfqpoint{1.937764in}{0.365312in}}%
\pgfpathlineto{\pgfqpoint{1.751054in}{0.722527in}}%
\pgfpathlineto{\pgfqpoint{1.348251in}{0.722527in}}%
\pgfpathclose%
\pgfusepath{stroke,fill}%
\end{pgfscope}%
\begin{pgfscope}%
\pgfpathrectangle{\pgfqpoint{0.198611in}{0.198611in}}{\pgfqpoint{3.602778in}{1.952778in}} %
\pgfusepath{clip}%
\pgfsetbuttcap%
\pgfsetmiterjoin%
\definecolor{currentfill}{rgb}{0.501961,0.501961,0.501961}%
\pgfsetfillcolor{currentfill}%
\pgfsetfillopacity{0.500000}%
\pgfsetlinewidth{0.501875pt}%
\definecolor{currentstroke}{rgb}{0.501961,0.501961,0.501961}%
\pgfsetstrokecolor{currentstroke}%
\pgfsetstrokeopacity{0.500000}%
\pgfsetdash{}{0pt}%
\pgfpathmoveto{\pgfqpoint{1.324479in}{0.841599in}}%
\pgfpathlineto{\pgfqpoint{1.549653in}{0.841599in}}%
\pgfpathlineto{\pgfqpoint{0.198611in}{2.627676in}}%
\pgfpathlineto{\pgfqpoint{-0.702083in}{2.627676in}}%
\pgfpathclose%
\pgfusepath{stroke,fill}%
\end{pgfscope}%
\begin{pgfscope}%
\pgfpathrectangle{\pgfqpoint{0.198611in}{0.198611in}}{\pgfqpoint{3.602778in}{1.952778in}} %
\pgfusepath{clip}%
\pgfsetbuttcap%
\pgfsetmiterjoin%
\definecolor{currentfill}{rgb}{0.501961,0.501961,0.501961}%
\pgfsetfillcolor{currentfill}%
\pgfsetfillopacity{0.500000}%
\pgfsetlinewidth{0.501875pt}%
\definecolor{currentstroke}{rgb}{0.501961,0.501961,0.501961}%
\pgfsetstrokecolor{currentstroke}%
\pgfsetstrokeopacity{0.500000}%
\pgfsetdash{}{0pt}%
\pgfpathmoveto{\pgfqpoint{2.154174in}{0.612615in}}%
\pgfpathlineto{\pgfqpoint{2.330815in}{0.612615in}}%
\pgfpathlineto{\pgfqpoint{3.323260in}{1.711739in}}%
\pgfpathlineto{\pgfqpoint{2.616697in}{1.711739in}}%
\pgfpathclose%
\pgfusepath{stroke,fill}%
\end{pgfscope}%
\begin{pgfscope}%
\pgfpathrectangle{\pgfqpoint{0.198611in}{0.198611in}}{\pgfqpoint{3.602778in}{1.952778in}} %
\pgfusepath{clip}%
\pgfsetrectcap%
\pgfsetroundjoin%
\pgfsetlinewidth{0.501875pt}%
\definecolor{currentstroke}{rgb}{0.894118,0.101961,0.109804}%
\pgfsetstrokecolor{currentstroke}%
\pgfsetdash{}{0pt}%
\pgfpathmoveto{\pgfqpoint{0.199504in}{2.165278in}}%
\pgfpathlineto{\pgfqpoint{0.777952in}{1.560435in}}%
\pgfpathlineto{\pgfqpoint{1.121936in}{1.204929in}}%
\pgfpathlineto{\pgfqpoint{1.339189in}{0.984574in}}%
\pgfpathlineto{\pgfqpoint{1.484024in}{0.841636in}}%
\pgfpathlineto{\pgfqpoint{1.592651in}{0.738492in}}%
\pgfpathlineto{\pgfqpoint{1.665068in}{0.673073in}}%
\pgfpathlineto{\pgfqpoint{1.737486in}{0.612018in}}%
\pgfpathlineto{\pgfqpoint{1.791799in}{0.570581in}}%
\pgfpathlineto{\pgfqpoint{1.828008in}{0.545905in}}%
\pgfpathlineto{\pgfqpoint{1.864217in}{0.524331in}}%
\pgfpathlineto{\pgfqpoint{1.900426in}{0.506629in}}%
\pgfpathlineto{\pgfqpoint{1.936635in}{0.493633in}}%
\pgfpathlineto{\pgfqpoint{1.972843in}{0.486109in}}%
\pgfpathlineto{\pgfqpoint{2.009052in}{0.484576in}}%
\pgfpathlineto{\pgfqpoint{2.045261in}{0.489147in}}%
\pgfpathlineto{\pgfqpoint{2.081470in}{0.499491in}}%
\pgfpathlineto{\pgfqpoint{2.117679in}{0.514944in}}%
\pgfpathlineto{\pgfqpoint{2.153887in}{0.534685in}}%
\pgfpathlineto{\pgfqpoint{2.190096in}{0.557900in}}%
\pgfpathlineto{\pgfqpoint{2.244410in}{0.597706in}}%
\pgfpathlineto{\pgfqpoint{2.298723in}{0.641871in}}%
\pgfpathlineto{\pgfqpoint{2.371140in}{0.705351in}}%
\pgfpathlineto{\pgfqpoint{2.461662in}{0.789475in}}%
\pgfpathlineto{\pgfqpoint{2.570289in}{0.894690in}}%
\pgfpathlineto{\pgfqpoint{2.733229in}{1.057446in}}%
\pgfpathlineto{\pgfqpoint{2.950482in}{1.279293in}}%
\pgfpathlineto{\pgfqpoint{3.276361in}{1.616965in}}%
\pgfpathlineto{\pgfqpoint{3.783284in}{2.147217in}}%
\pgfpathlineto{\pgfqpoint{3.800496in}{2.165278in}}%
\pgfpathlineto{\pgfqpoint{3.800496in}{2.165278in}}%
\pgfusepath{stroke}%
\end{pgfscope}%
\begin{pgfscope}%
\pgfpathrectangle{\pgfqpoint{0.198611in}{0.198611in}}{\pgfqpoint{3.602778in}{1.952778in}} %
\pgfusepath{clip}%
\pgfsetbuttcap%
\pgfsetroundjoin%
\pgfsetlinewidth{0.501875pt}%
\definecolor{currentstroke}{rgb}{0.501961,0.501961,0.501961}%
\pgfsetstrokecolor{currentstroke}%
\pgfsetdash{{1.850000pt}{0.800000pt}}{0.000000pt}%
\pgfpathmoveto{\pgfqpoint{1.941833in}{0.184722in}}%
\pgfpathlineto{\pgfqpoint{3.801389in}{2.151389in}}%
\pgfpathlineto{\pgfqpoint{3.801389in}{2.151389in}}%
\pgfusepath{stroke}%
\end{pgfscope}%
\begin{pgfscope}%
\pgfpathrectangle{\pgfqpoint{0.198611in}{0.198611in}}{\pgfqpoint{3.602778in}{1.952778in}} %
\pgfusepath{clip}%
\pgfsetbuttcap%
\pgfsetroundjoin%
\pgfsetlinewidth{0.501875pt}%
\definecolor{currentstroke}{rgb}{0.501961,0.501961,0.501961}%
\pgfsetstrokecolor{currentstroke}%
\pgfsetdash{{1.850000pt}{0.800000pt}}{0.000000pt}%
\pgfpathmoveto{\pgfqpoint{0.198611in}{2.151389in}}%
\pgfpathlineto{\pgfqpoint{2.058167in}{0.184722in}}%
\pgfpathlineto{\pgfqpoint{2.058167in}{0.184722in}}%
\pgfusepath{stroke}%
\end{pgfscope}%
\begin{pgfscope}%
\pgfpathrectangle{\pgfqpoint{0.198611in}{0.198611in}}{\pgfqpoint{3.602778in}{1.952778in}} %
\pgfusepath{clip}%
\pgfsetbuttcap%
\pgfsetroundjoin%
\pgfsetlinewidth{0.501875pt}%
\definecolor{currentstroke}{rgb}{0.501961,0.501961,0.501961}%
\pgfsetstrokecolor{currentstroke}%
\pgfsetdash{{1.850000pt}{0.800000pt}}{0.000000pt}%
\pgfpathmoveto{\pgfqpoint{2.000000in}{0.484383in}}%
\pgfpathlineto{\pgfqpoint{2.396306in}{0.484383in}}%
\pgfusepath{stroke}%
\end{pgfscope}%
\begin{pgfscope}%
\pgfsetrectcap%
\pgfsetmiterjoin%
\pgfsetlinewidth{0.501875pt}%
\definecolor{currentstroke}{rgb}{0.000000,0.000000,0.000000}%
\pgfsetstrokecolor{currentstroke}%
\pgfsetdash{}{0pt}%
\pgfpathmoveto{\pgfqpoint{2.000000in}{0.198611in}}%
\pgfpathlineto{\pgfqpoint{2.000000in}{2.151389in}}%
\pgfusepath{stroke}%
\end{pgfscope}%
\begin{pgfscope}%
\pgfsetrectcap%
\pgfsetmiterjoin%
\pgfsetlinewidth{0.501875pt}%
\definecolor{currentstroke}{rgb}{0.000000,0.000000,0.000000}%
\pgfsetstrokecolor{currentstroke}%
\pgfsetdash{}{0pt}%
\pgfpathmoveto{\pgfqpoint{0.198611in}{0.246240in}}%
\pgfpathlineto{\pgfqpoint{3.801389in}{0.246240in}}%
\pgfusepath{stroke}%
\end{pgfscope}%
\begin{pgfscope}%
\pgfsetroundcap%
\pgfsetroundjoin%
\pgfsetlinewidth{0.501875pt}%
\definecolor{currentstroke}{rgb}{0.000000,0.000000,0.000000}%
\pgfsetstrokecolor{currentstroke}%
\pgfsetdash{}{0pt}%
\pgfpathmoveto{\pgfqpoint{1.382621in}{0.546431in}}%
\pgfpathquadraticcurveto{\pgfqpoint{1.527316in}{0.554339in}}{\pgfqpoint{1.664258in}{0.561824in}}%
\pgfusepath{stroke}%
\end{pgfscope}%
\begin{pgfscope}%
\pgfsetroundcap%
\pgfsetroundjoin%
\pgfsetlinewidth{0.501875pt}%
\definecolor{currentstroke}{rgb}{0.000000,0.000000,0.000000}%
\pgfsetstrokecolor{currentstroke}%
\pgfsetdash{}{0pt}%
\pgfpathmoveto{\pgfqpoint{1.607269in}{0.586528in}}%
\pgfpathlineto{\pgfqpoint{1.664258in}{0.561824in}}%
\pgfpathlineto{\pgfqpoint{1.610301in}{0.531056in}}%
\pgfusepath{stroke}%
\end{pgfscope}%
\begin{pgfscope}%
\pgftext[x=0.423785in,y=0.484383in,left,base]{\rmfamily\fontsize{10.000000}{12.000000}\selectfont a = 0.2, s = -1}%
\end{pgfscope}%
\begin{pgfscope}%
\pgfsetroundcap%
\pgfsetroundjoin%
\pgfsetlinewidth{0.501875pt}%
\definecolor{currentstroke}{rgb}{0.000000,0.000000,0.000000}%
\pgfsetstrokecolor{currentstroke}%
\pgfsetdash{}{0pt}%
\pgfpathmoveto{\pgfqpoint{0.818790in}{1.787660in}}%
\pgfpathquadraticcurveto{\pgfqpoint{0.672544in}{1.808780in}}{\pgfqpoint{0.533983in}{1.828789in}}%
\pgfusepath{stroke}%
\end{pgfscope}%
\begin{pgfscope}%
\pgfsetroundcap%
\pgfsetroundjoin%
\pgfsetlinewidth{0.501875pt}%
\definecolor{currentstroke}{rgb}{0.000000,0.000000,0.000000}%
\pgfsetstrokecolor{currentstroke}%
\pgfsetdash{}{0pt}%
\pgfpathmoveto{\pgfqpoint{0.584998in}{1.793356in}}%
\pgfpathlineto{\pgfqpoint{0.533983in}{1.828789in}}%
\pgfpathlineto{\pgfqpoint{0.592939in}{1.848341in}}%
\pgfusepath{stroke}%
\end{pgfscope}%
\begin{pgfscope}%
\pgftext[x=0.874132in,y=1.675102in,left,base]{\rmfamily\fontsize{10.000000}{12.000000}\selectfont a = 0.04, s = -1}%
\end{pgfscope}%
\begin{pgfscope}%
\pgftext[x=2.450347in,y=0.460569in,left,base]{\rmfamily\fontsize{10.000000}{12.000000}\selectfont \(\displaystyle \omega = m\)}%
\end{pgfscope}%
\begin{pgfscope}%
\pgftext[x=3.193420in,y=1.436958in,left,base]{\rmfamily\fontsize{10.000000}{12.000000}\selectfont \(\displaystyle supp~(\hat\Delta_m)\)}%
\end{pgfscope}%
\begin{pgfscope}%
\pgfsetroundcap%
\pgfsetroundjoin%
\pgfsetlinewidth{0.501875pt}%
\definecolor{currentstroke}{rgb}{0.000000,0.000000,0.000000}%
\pgfsetstrokecolor{currentstroke}%
\pgfsetdash{}{0pt}%
\pgfpathmoveto{\pgfqpoint{2.000000in}{2.157510in}}%
\pgfpathquadraticcurveto{\pgfqpoint{2.000000in}{2.158331in}}{\pgfqpoint{2.000000in}{2.151389in}}%
\pgfusepath{stroke}%
\end{pgfscope}%
\begin{pgfscope}%
\pgfsetroundcap%
\pgfsetroundjoin%
\pgfsetlinewidth{0.501875pt}%
\definecolor{currentstroke}{rgb}{0.000000,0.000000,0.000000}%
\pgfsetstrokecolor{currentstroke}%
\pgfsetdash{}{0pt}%
\pgfpathmoveto{\pgfqpoint{1.972222in}{2.101954in}}%
\pgfpathlineto{\pgfqpoint{2.000000in}{2.157510in}}%
\pgfpathlineto{\pgfqpoint{2.027778in}{2.101954in}}%
\pgfusepath{stroke}%
\end{pgfscope}%
\begin{pgfscope}%
\pgftext[x=2.000000in,y=2.220833in,,bottom]{\rmfamily\fontsize{10.000000}{12.000000}\selectfont \(\displaystyle \omega\)}%
\end{pgfscope}%
\begin{pgfscope}%
\pgfsetroundcap%
\pgfsetroundjoin%
\pgfsetlinewidth{0.501875pt}%
\definecolor{currentstroke}{rgb}{0.000000,0.000000,0.000000}%
\pgfsetstrokecolor{currentstroke}%
\pgfsetdash{}{0pt}%
\pgfpathmoveto{\pgfqpoint{3.807488in}{0.246240in}}%
\pgfpathquadraticcurveto{\pgfqpoint{3.808320in}{0.246240in}}{\pgfqpoint{3.801389in}{0.246240in}}%
\pgfusepath{stroke}%
\end{pgfscope}%
\begin{pgfscope}%
\pgfsetroundcap%
\pgfsetroundjoin%
\pgfsetlinewidth{0.501875pt}%
\definecolor{currentstroke}{rgb}{0.000000,0.000000,0.000000}%
\pgfsetstrokecolor{currentstroke}%
\pgfsetdash{}{0pt}%
\pgfpathmoveto{\pgfqpoint{3.751932in}{0.274018in}}%
\pgfpathlineto{\pgfqpoint{3.807488in}{0.246240in}}%
\pgfpathlineto{\pgfqpoint{3.751932in}{0.218462in}}%
\pgfusepath{stroke}%
\end{pgfscope}%
\begin{pgfscope}%
\pgftext[x=3.870833in,y=0.246240in,left,]{\rmfamily\fontsize{10.000000}{12.000000}\selectfont \(\displaystyle k\)}%
\end{pgfscope}%
\end{pgfpicture}%
\makeatother%
\endgroup%

\caption{Die Träger von $\hat\Delta_m$ und $\hat\psi_{ast}$. Es ist zu sehen, dass für $a \rightarrow 0$ und $s \neq \pm 1$ die Träger schließlich disjunkt sind}
\label{fig:delta_m}
\end{figure}

%%%%%%%%%%%%%%%%%%%%%%%%%%%%%%%%%%%%%%%%%%%%%%%%%%%%%%%%%%%%%%%%%%%%%%%%%%%%%%%
% % Teil 2
%%%%%%%%%%%%%%%%%%%%%%%%%%%%%%%%%%%%%%%%%%%%%%%%%%%%%%%%%%%%%%%%%%%%%%%%%%%%%%%
\subsubsection*{Fall $s \neq \pm 1$}
Hier gibt es nicht viel zu tun, denn für a klein genug gilt
$supp (\hat \Delta_m) \cap supp (\hat \psi_{ast}) = \varnothing$ wie man Abb. \ref{fig:delta_m} entnehmen kann.
Also gilt $\left< \psi_{ast}, \Delta_m\right> = 0 = O(a^k)~ \forall k$ für $a$ klein genug. Dies gilt für alle $(t', x') \in \mathbb{R}^2$

\todo{hier noch blöde Abschätzerei machen, warum das tatsächlich gilt, oder stehen lassen. Oder im Kapitel Shearlets ne Bemerkung machen, warum wir in immer engeren Kegeln landen?}

%%%%%%%%%%%%%%%%%%%%%%%%%%%%%%%%%%%%%%%%%%%%%%%%%%%%%%%%%%%%%%%%%%%%%%%%%%%%%%%
% % Teil3
%%%%%%%%%%%%%%%%%%%%%%%%%%%%%%%%%%%%%%%%%%%%%%%%%%%%%%%%%%%%%%%%%%%%%%%%%%%%%%%
\subsubsection*{Fall $s=1$}
\paragraph*{Intuition}
Für $s=-1$ schneidet die Diagonale  $supp (\hat \psi_{ast})$ auf der ganzen Länge. Der Betrag von $\hat\psi_{ast}$ skaliert mit $a^{\frac{3}{4}}$ und die Länge von $supp (\hat \psi_{ast})$ mit $a^{-1}$. Also erwarten wir schlimmstenfalls $\left< \hat \psi_{a1t}, \hat\Delta_m\right> = O(a^-{\frac{1}{4}})$. Aber nur wenn die Wellenfronten von $e^{-i\omega t'+i k x'}$ parallel zu der Singularität und damit der Diagonalen liegen. Andernfalls erwarten wir, dass die immer schneller werdenden Oszillationen der Phase sich gegenseitig auslöschen/wegheben.

\paragraph*{Fleißige Analysis}
\begin{align}
    \left<\hat \psi_{a1t} ,\hat \Delta_m\right> &=
        a^{\frac{3}{4}} \int \hat \psi_1(a\omega)
        \hat\psi_2\left(a^{-\frac{1}{2}} \left(\tfrac{k}{\omega}-1\right)\right)
        \delta (\omega^2 - k^2 - m^2) \theta(\omega)
        e^{-i\omega t' + ikx'} d \omega \d k \nonumber \\[2ex]
        & \kern 2em \underline{\textrm{Nullstellen von $\delta$}:}
        \nonumber \\
        & \kern 2em \omega^2 - k^2 - m^2 = 0 \Leftrightarrow k = \pm \sqrt{\omega^2-m^2}
        \nonumber \\
        & \kern 2em \Rightarrow \frac{\d k}{\d \omega} = \frac{\omega}{\sqrt{\omega^2-m^2}}; \textrm{   wobei nur ,,+'' in $supp (\hat \psi_2)$ liegt}
        \nonumber \\[2ex]
        &= a^{\frac{3}{4}} \int \hat\psi_1(a \omega)
        \hat\psi_2\left(a^{-\frac{1}{2}} \left(\tfrac{\sqrt{\omega^2-m^2}}{\omega}-1\right)\right)
        e^{-i\omega t' + i \sqrt{\omega^2-m^2}x'}
        \d \omega \nonumber \\
        &= a^{\frac{3}{4}} a^{-1} \int \hat\psi_1(\omega)
        \hat\psi_2
        \underbrace{
        \left(
            a^{-\frac{1}{2}} \left(\tfrac{\sqrt{a\omega^2-m^2}}{\omega}-1\right)
        \right)}_{= \frac{a^{\frac{3}{2}}m^2}{2 \omega^2}
                  + O\left(a^{\frac{7}{2}}\right)}
        e^{-i\frac{\omega}{a} t' + i \sqrt{\frac{\omega^2}{a^2}-m^2}x'}
        \d \omega \nonumber
\end{align}
Der Integrand lässt sich nun durch $\hat \psi_1(\omega) \left\lVert \hat\psi_2\right\lVert_\infty$ majorisieren und wir dürfen Lebesgue verwenden um Integral und Grenzwert $a \rightarrow 0$ zu vertauschen

\begin{dmath}
 = a^{-\frac{1}{4}} \int \hat \psi_1(\omega) \hat \psi_2 (0)
    e^{-i\omega \left(\frac{t'-x'}{a}\right)}
= a^{-\frac{1}{4}} \hat \psi_2 (0) \psi_1\left(\frac{t'-x'}{a}\right)
\\
 \sim O\left(a^{-\frac{1}{4}}\right) \condition{falls $x'=t'$}
\\
 \sim O\left(a^k\right) ~ \forall k  \condition{sonst}
\end{dmath}

Das analoge Ergebnis erhält man mit gleicher Rechnung auch für $s=-1$ und $t' = -x'$
Dies bestätigt das intuitiv erwartete Ergebnis. Insgesamt erhalten wir also für die Wellenfrontmenge:

\begin{table}[h]
\centering
\label{my-label}
\begin{tabular}{l|cccc}
        & \multicolumn{1}{l}{$(t', x') = (0, 0)$} & \multicolumn{1}{l}{$t' = x'$} & \multicolumn{1}{l}{$t' = -x'$} & \multicolumn{1}{l}{$t' \neq \pm x'$} \\ \hline
s = 1   & $a^{-\frac{1}{4}}$    & $a^{-\frac{1}{4}}$    & $a^k$  & $a^k$    \\
s = -1  & $a^{-\frac{1}{4}}$    & $a^k$    & $a^{-\frac{1}{4}}$  & $a^k$    \\
$s \neq \pm 1$  & $a^k$         & $a^k$    & $a^k$               & $a^k$    \\
\end{tabular}
\caption{Konvergenzordnung von $\mathcal{S}_{\Delta_m} (a,s,(t',x'))$ im Limit $a \rightarrow 0$ für alle interessanten Kombinationen von $s$ und $(t',x')$}
\end{table}

% section die_wellenfrontmenge_von_ (end)


%!TEX root = main.tex
% !TEX spellcheck=de_DE
%%%%%%%%%%%%%%%%%%%%%%%%%%%%%%%%%%%%%%%%%%%%%%%%%%%%%%%%%%%%%%%%%%%%%%%%%%%%%%%
% % Der Heaviside-Function
%%%%%%%%%%%%%%%%%%%%%%%%%%%%%%%%%%%%%%%%%%%%%%%%%%%%%%%%%%%%%%%%%%%%%%%%%%%%%%%

\section{\texorpdfstring{Die Wellenfrontmenge von $\Theta$}
        {Die Wellenfrontmenge der Heaviside-Funktion}} % (fold)
\label{sec:die_wellenfrontmenge_der_heaviside_function}

Wie in \cref{sec:die_zweipunktfunktionen_und_warum_wir_sie_potenzieren_wollen} erklärt, sind Potenzen des Feynmanpropagators gegeben durch Potenzen der Zweipunktfunktion $\Delta_m$ und der Heaviside-Funktion $\Theta$. Dementsprechend, muss auch die Wellenfrontmenge von $\Theta$ berechnet werden, aber dies ist glücklicherweise auch mit unseren Shearlet-Methoden relativ einfach.

Da $\Theta$ die Stammfunktion von $\delta$ (im distributionellen Sinne) ist, können wir die Fouriertransformierte dank der üblichen Fourierrechenregeln direkt hinschreiben:\footnote{Wieder nur korrekt bis auf Vorfaktoren von $2\pi$}

\begin{equation}
    \rwhat{\Theta(t)\otimes 1(x)}(\omega,k) = \rwhat{\Theta}(\omega) \otimes \rwhat{\delta}(k) = \left(\delta(\omega) + \frac{i}\omega{}\right) \delta (k)
    \label{eq:heaviside_fourier_transform}
\end{equation}.

\subsubsection*{Fall $s \neq 0$}

\begin{equation*}
supp (\rwhat{\Theta(t)\otimes 1(x)}) = \{(\omega, k) \in \hat{\mathbb{R}} | k = 0\}
\end{equation*}

 und nach \cref{eq:supp_psi}

\begin{equation*}
    supp(\hat \psi) \subset \left\{k \in  \hat{\mathbb{R}}^2 ~\Big| ~k_1 \in \left[\frac{1}{2 a} , \frac{2}{a}\right], \left|\frac{k_2}{k_1} - s\right| \leq \sqrt{a} \right\}
\end{equation*}.

Also gilt für hinreichend große $a$:

\begin{equation}
 supp (\hat\psi_{ast}) \cap supp (\rwhat{\Theta(t)\otimes 1(x)}) = \varnothing
 ~~\Longrightarrow~~
\left\langle \rwhat{\Theta(t)\otimes 1(x)},  \hat\psi_{ast}\right\rangle = 0
\label{eq:heaviside_s_neq_0}
\end{equation}

\subsubsection*{Fall $s = 0$}
Mit \cref{eq:heaviside_fourier_transform} können $\left<\hat \Theta \otimes \hat 1, \hat \psi_{ast}\right>$ direkt berechnen mit dem Ausdruck für $\hat \psi_{ast}$ aus \cref{rem:psi_hat}:

\begin{align}
    \left\langle \hat\Theta \otimes \hat 1, \hat \psi_{ast} \right\rangle
    &=
    a^{\frac{3}{4}} \int \hat \psi_1(a \omega) \hat \psi_2\left(
                a^{-\frac{1}{2}}\left(\tfrac{k}{\omega}\right)\right)
    \left(\delta(\omega) + \frac{i}{\omega}\right) \delta(k)
    e^{-i\omega t' + i kx'}
    \d \omega \d k
    \nonumber \\ &=
    \underbrace{a^{\frac{3}{4}} \hat\psi_1(0) \hat\psi_2(0)
    }_{=0, \textrm{ da } \hat\psi_1(0) = 0}
    +
    i a^{\frac{3}{4}}
    \int \frac{\hat\psi_1(\omega) \hat\psi_2(0)}
        {\omega} e^{-i \omega t'} \d \omega
    \nonumber \\ &=
    i a^{\frac{3}{4}} \hat\psi_2(0) \int
    \underbrace{\frac{\psi_1(\omega)}{\omega}}_{\in C_c^\infty}
    e^{-i\omega\frac{t'}{a}} \d \omega
    \nonumber \\ &=
    O\left(a^{\frac{3}{4}}\right) \condition{falls $t=0$}
    \nonumber \\ &=
    O\left(a^{k}\right) ~ \forall k \in \mathbb{N} \condition{falls $t\neq0$}.
    \label{eq:heaviside_s=0}
\end{align}

Wobei im letzten Schritt genutzt wurde, dass $\hat\psi_1(0) = 0$, $\frac{\hat\psi_1(\omega)}{\omega}$ also glatt ist und somit eine schnell fallende Fouriertransformierte hat.

Genau die selben Ergebnisse erhält man mit beinahe genau den selben Rechnungen für $\left<\Theta \otimes 1, \psi_{ast}^{(3)}\right>$.

\subsection{Zusammenfassung und Vergleich der Ergebnisse}
Und einmal der vollständig halber, die Ergebnisse aus \cref{eq:heaviside_s=0,eq:heaviside_s_neq_0} tabellarisch dargestellt:

\begin{table}[h]
\centering
\begin{tabular}{l|cc}
           & \multicolumn{1}{l}{$t'=0$} & \multicolumn{1}{l}{$t' \neq 0$} \\ \hline
$s = 0$    & $a^{\frac{3}{4}}$          & $a^k$                           \\
$s \neq 0$ & $a^k$                      & $a^k$
\end{tabular}
\caption{Konvergenzordnung von $\left<\Theta \otimes 1,\psi_{as(t',x')}\right>$ im Limit $a \rightarrow 0$ für alle interessanten Kombinationen von $s$ und $(t',x')$}
\label{tab:wavefront_set_heaviside}
\end{table}



% section die_wellenfrontmenge_von_ (end)


%!TEX root = main.tex
% !TEX spellcheck=de_DE
%%%%%%%%%%%%%%%%%%%%%%%%%%%%%%%%%%%%%%%%%%%%%%%%%%%%%%%%%%%%%%%%%%%%%%%%%%%%%%%
% % Berechnen der Wellenfrontmenge von Delta_m
%%%%%%%%%%%%%%%%%%%%%%%%%%%%%%%%%%%%%%%%%%%%%%%%%%%%%%%%%%%%%%%%%%%%%%%%%%%%%%%

\section{\texorpdfstring{Die Wellenfrontmenge von $\Delta_m^2$}
    {Wellenfrontmenge von delta m}} % (fold)
\label{sec:die_wellenfrontmenge_von_delta_m_2_}

Bevor wir die Wellenfrontmenge von $\Delta_m^2$ berechnen können benötigen wir einen Ausdruck dafür, oder besser noch für die Fouriertransformierte davon.

\subsection{\texorpdfstring{$\hat\Delta^{\ast 2}$ berechnen}
    {Die massive Zweipunktfunktion quadrieren}}
\label{sec:delta_m2_berechnen}

 Gemäß dem Faltungssatz gilt $\rwhat{\Delta_m^2} = \rwhat \Delta_m * \rwhat \Delta_m = \rwhat\Delta_m^{*2}$. Wir müssen also die Faltung von $\rwhat \Delta_m$ mit sich selber ausrechnen.


Das Faltungsintegral ist

\begin{equation}
    \rwhat{\Delta}_m^{* 2} (\omega, k)
    = \int \Theta(\omega') \delta(\omega^{\prime 2}-k^{\prime2}-m^2)\Theta(\omega-\omega')
      \delta((\omega-\omega')^2-(k-k')^2-m^2) \d \omega' \d k'
\label{eq:mass_shell_convolution}
\end{equation}

\begin{figure}[h]
    \centering
    \begin{minipage}{0.5\textwidth}
        \centering
        \resizebox{\textwidth}{!}{%% Creator: Matplotlib, PGF backend
%%
%% To include the figure in your LaTeX document, write
%%   \input{<filename>.pgf}
%%
%% Make sure the required packages are loaded in your preamble
%%   \usepackage{pgf}
%%
%% Figures using additional raster images can only be included by \input if
%% they are in the same directory as the main LaTeX file. For loading figures
%% from other directories you can use the `import` package
%%   \usepackage{import}
%% and then include the figures with
%%   \import{<path to file>}{<filename>.pgf}
%%
%% Matplotlib used the following preamble
%%   \usepackage[utf8x]{inputenc}
%%   \usepackage[T1]{fontenc}
%%   \usepackage{amssymb}
%%
\begingroup%
\makeatletter%
\begin{pgfpicture}%
\pgfpathrectangle{\pgfpointorigin}{\pgfqpoint{4.000000in}{3.000000in}}%
\pgfusepath{use as bounding box, clip}%
\begin{pgfscope}%
\pgfsetbuttcap%
\pgfsetmiterjoin%
\definecolor{currentfill}{rgb}{1.000000,1.000000,1.000000}%
\pgfsetfillcolor{currentfill}%
\pgfsetlinewidth{0.000000pt}%
\definecolor{currentstroke}{rgb}{1.000000,1.000000,1.000000}%
\pgfsetstrokecolor{currentstroke}%
\pgfsetdash{}{0pt}%
\pgfpathmoveto{\pgfqpoint{0.000000in}{0.000000in}}%
\pgfpathlineto{\pgfqpoint{4.000000in}{0.000000in}}%
\pgfpathlineto{\pgfqpoint{4.000000in}{3.000000in}}%
\pgfpathlineto{\pgfqpoint{0.000000in}{3.000000in}}%
\pgfpathclose%
\pgfusepath{fill}%
\end{pgfscope}%
\begin{pgfscope}%
\pgfsetbuttcap%
\pgfsetmiterjoin%
\definecolor{currentfill}{rgb}{1.000000,1.000000,1.000000}%
\pgfsetfillcolor{currentfill}%
\pgfsetlinewidth{0.000000pt}%
\definecolor{currentstroke}{rgb}{0.000000,0.000000,0.000000}%
\pgfsetstrokecolor{currentstroke}%
\pgfsetstrokeopacity{0.000000}%
\pgfsetdash{}{0pt}%
\pgfpathmoveto{\pgfqpoint{0.198611in}{0.198611in}}%
\pgfpathlineto{\pgfqpoint{3.801389in}{0.198611in}}%
\pgfpathlineto{\pgfqpoint{3.801389in}{2.801389in}}%
\pgfpathlineto{\pgfqpoint{0.198611in}{2.801389in}}%
\pgfpathclose%
\pgfusepath{fill}%
\end{pgfscope}%
\begin{pgfscope}%
\pgfsetbuttcap%
\pgfsetroundjoin%
\definecolor{currentfill}{rgb}{0.000000,0.000000,0.000000}%
\pgfsetfillcolor{currentfill}%
\pgfsetlinewidth{0.803000pt}%
\definecolor{currentstroke}{rgb}{0.000000,0.000000,0.000000}%
\pgfsetstrokecolor{currentstroke}%
\pgfsetdash{}{0pt}%
\pgfsys@defobject{currentmarker}{\pgfqpoint{0.000000in}{-0.048611in}}{\pgfqpoint{0.000000in}{0.000000in}}{%
\pgfpathmoveto{\pgfqpoint{0.000000in}{0.000000in}}%
\pgfpathlineto{\pgfqpoint{0.000000in}{-0.048611in}}%
\pgfusepath{stroke,fill}%
}%
\begin{pgfscope}%
\pgfsys@transformshift{1.219976in}{0.632407in}%
\pgfsys@useobject{currentmarker}{}%
\end{pgfscope}%
\end{pgfscope}%
\begin{pgfscope}%
\pgftext[x=1.219976in,y=0.535185in,,top]{\rmfamily\fontsize{10.000000}{12.000000}\selectfont \(\displaystyle k^\prime_{0-} = -\sqrt{\left(\frac{\omega}{2}\right)^2-m^2}\)}%
\end{pgfscope}%
\begin{pgfscope}%
\pgfsetbuttcap%
\pgfsetroundjoin%
\definecolor{currentfill}{rgb}{0.000000,0.000000,0.000000}%
\pgfsetfillcolor{currentfill}%
\pgfsetlinewidth{0.803000pt}%
\definecolor{currentstroke}{rgb}{0.000000,0.000000,0.000000}%
\pgfsetstrokecolor{currentstroke}%
\pgfsetdash{}{0pt}%
\pgfsys@defobject{currentmarker}{\pgfqpoint{0.000000in}{-0.048611in}}{\pgfqpoint{0.000000in}{0.000000in}}{%
\pgfpathmoveto{\pgfqpoint{0.000000in}{0.000000in}}%
\pgfpathlineto{\pgfqpoint{0.000000in}{-0.048611in}}%
\pgfusepath{stroke,fill}%
}%
\begin{pgfscope}%
\pgfsys@transformshift{2.780024in}{0.632407in}%
\pgfsys@useobject{currentmarker}{}%
\end{pgfscope}%
\end{pgfscope}%
\begin{pgfscope}%
\pgftext[x=2.780024in,y=0.535185in,,top]{\rmfamily\fontsize{10.000000}{12.000000}\selectfont \(\displaystyle k^\prime_{0+} = \sqrt{\left(\frac{\omega}{2}\right)^2-m^2}\)}%
\end{pgfscope}%
\begin{pgfscope}%
\pgfsetbuttcap%
\pgfsetroundjoin%
\definecolor{currentfill}{rgb}{0.000000,0.000000,0.000000}%
\pgfsetfillcolor{currentfill}%
\pgfsetlinewidth{0.803000pt}%
\definecolor{currentstroke}{rgb}{0.000000,0.000000,0.000000}%
\pgfsetstrokecolor{currentstroke}%
\pgfsetdash{}{0pt}%
\pgfsys@defobject{currentmarker}{\pgfqpoint{-0.048611in}{0.000000in}}{\pgfqpoint{0.000000in}{0.000000in}}{%
\pgfpathmoveto{\pgfqpoint{0.000000in}{0.000000in}}%
\pgfpathlineto{\pgfqpoint{-0.048611in}{0.000000in}}%
\pgfusepath{stroke,fill}%
}%
\begin{pgfscope}%
\pgfsys@transformshift{2.000000in}{1.066204in}%
\pgfsys@useobject{currentmarker}{}%
\end{pgfscope}%
\end{pgfscope}%
\begin{pgfscope}%
\pgftext[x=1.780831in,y=1.018376in,left,base]{\rmfamily\fontsize{10.000000}{12.000000}\selectfont \(\displaystyle m\)}%
\end{pgfscope}%
\begin{pgfscope}%
\pgfsetbuttcap%
\pgfsetroundjoin%
\definecolor{currentfill}{rgb}{0.000000,0.000000,0.000000}%
\pgfsetfillcolor{currentfill}%
\pgfsetlinewidth{0.803000pt}%
\definecolor{currentstroke}{rgb}{0.000000,0.000000,0.000000}%
\pgfsetstrokecolor{currentstroke}%
\pgfsetdash{}{0pt}%
\pgfsys@defobject{currentmarker}{\pgfqpoint{-0.048611in}{0.000000in}}{\pgfqpoint{0.000000in}{0.000000in}}{%
\pgfpathmoveto{\pgfqpoint{0.000000in}{0.000000in}}%
\pgfpathlineto{\pgfqpoint{-0.048611in}{0.000000in}}%
\pgfusepath{stroke,fill}%
}%
\begin{pgfscope}%
\pgfsys@transformshift{2.000000in}{1.500000in}%
\pgfsys@useobject{currentmarker}{}%
\end{pgfscope}%
\end{pgfscope}%
\begin{pgfscope}%
\pgftext[x=1.338511in,y=1.444321in,left,base]{\rmfamily\fontsize{10.000000}{12.000000}\selectfont \(\displaystyle \omega/2 = \omega^\prime_0\)}%
\end{pgfscope}%
\begin{pgfscope}%
\pgfsetbuttcap%
\pgfsetroundjoin%
\definecolor{currentfill}{rgb}{0.000000,0.000000,0.000000}%
\pgfsetfillcolor{currentfill}%
\pgfsetlinewidth{0.803000pt}%
\definecolor{currentstroke}{rgb}{0.000000,0.000000,0.000000}%
\pgfsetstrokecolor{currentstroke}%
\pgfsetdash{}{0pt}%
\pgfsys@defobject{currentmarker}{\pgfqpoint{-0.048611in}{0.000000in}}{\pgfqpoint{0.000000in}{0.000000in}}{%
\pgfpathmoveto{\pgfqpoint{0.000000in}{0.000000in}}%
\pgfpathlineto{\pgfqpoint{-0.048611in}{0.000000in}}%
\pgfusepath{stroke,fill}%
}%
\begin{pgfscope}%
\pgfsys@transformshift{2.000000in}{1.933796in}%
\pgfsys@useobject{currentmarker}{}%
\end{pgfscope}%
\end{pgfscope}%
\begin{pgfscope}%
\pgftext[x=1.519645in,y=1.885969in,left,base]{\rmfamily\fontsize{10.000000}{12.000000}\selectfont \(\displaystyle \omega - m\)}%
\end{pgfscope}%
\begin{pgfscope}%
\pgfsetbuttcap%
\pgfsetroundjoin%
\definecolor{currentfill}{rgb}{0.000000,0.000000,0.000000}%
\pgfsetfillcolor{currentfill}%
\pgfsetlinewidth{0.803000pt}%
\definecolor{currentstroke}{rgb}{0.000000,0.000000,0.000000}%
\pgfsetstrokecolor{currentstroke}%
\pgfsetdash{}{0pt}%
\pgfsys@defobject{currentmarker}{\pgfqpoint{-0.048611in}{0.000000in}}{\pgfqpoint{0.000000in}{0.000000in}}{%
\pgfpathmoveto{\pgfqpoint{0.000000in}{0.000000in}}%
\pgfpathlineto{\pgfqpoint{-0.048611in}{0.000000in}}%
\pgfusepath{stroke,fill}%
}%
\begin{pgfscope}%
\pgfsys@transformshift{2.000000in}{2.367593in}%
\pgfsys@useobject{currentmarker}{}%
\end{pgfscope}%
\end{pgfscope}%
\begin{pgfscope}%
\pgftext[x=1.811343in,y=2.319765in,left,base]{\rmfamily\fontsize{10.000000}{12.000000}\selectfont \(\displaystyle \omega\)}%
\end{pgfscope}%
\begin{pgfscope}%
\pgfpathrectangle{\pgfqpoint{0.198611in}{0.198611in}}{\pgfqpoint{3.602778in}{2.602778in}}%
\pgfusepath{clip}%
\pgfsetbuttcap%
\pgfsetroundjoin%
\pgfsetlinewidth{0.501875pt}%
\definecolor{currentstroke}{rgb}{0.501961,0.501961,0.501961}%
\pgfsetstrokecolor{currentstroke}%
\pgfsetdash{{1.850000pt}{0.800000pt}}{0.000000pt}%
\pgfpathmoveto{\pgfqpoint{1.219976in}{0.632407in}}%
\pgfpathlineto{\pgfqpoint{1.219976in}{1.500000in}}%
\pgfusepath{stroke}%
\end{pgfscope}%
\begin{pgfscope}%
\pgfpathrectangle{\pgfqpoint{0.198611in}{0.198611in}}{\pgfqpoint{3.602778in}{2.602778in}}%
\pgfusepath{clip}%
\pgfsetbuttcap%
\pgfsetroundjoin%
\pgfsetlinewidth{0.501875pt}%
\definecolor{currentstroke}{rgb}{0.501961,0.501961,0.501961}%
\pgfsetstrokecolor{currentstroke}%
\pgfsetdash{{1.850000pt}{0.800000pt}}{0.000000pt}%
\pgfpathmoveto{\pgfqpoint{2.780024in}{0.632407in}}%
\pgfpathlineto{\pgfqpoint{2.780024in}{1.500000in}}%
\pgfusepath{stroke}%
\end{pgfscope}%
\begin{pgfscope}%
\pgfpathrectangle{\pgfqpoint{0.198611in}{0.198611in}}{\pgfqpoint{3.602778in}{2.602778in}}%
\pgfusepath{clip}%
\pgfsetrectcap%
\pgfsetroundjoin%
\pgfsetlinewidth{0.501875pt}%
\definecolor{currentstroke}{rgb}{0.894118,0.101961,0.109804}%
\pgfsetstrokecolor{currentstroke}%
\pgfsetdash{}{0pt}%
\pgfpathmoveto{\pgfqpoint{0.184722in}{0.566020in}}%
\pgfpathlineto{\pgfqpoint{0.524413in}{0.881513in}}%
\pgfpathlineto{\pgfqpoint{0.768088in}{1.104150in}}%
\pgfpathlineto{\pgfqpoint{0.957613in}{1.273814in}}%
\pgfpathlineto{\pgfqpoint{1.120063in}{1.415436in}}%
\pgfpathlineto{\pgfqpoint{1.255438in}{1.529408in}}%
\pgfpathlineto{\pgfqpoint{1.363738in}{1.616727in}}%
\pgfpathlineto{\pgfqpoint{1.444963in}{1.679103in}}%
\pgfpathlineto{\pgfqpoint{1.526188in}{1.737927in}}%
\pgfpathlineto{\pgfqpoint{1.607413in}{1.792107in}}%
\pgfpathlineto{\pgfqpoint{1.661563in}{1.824956in}}%
\pgfpathlineto{\pgfqpoint{1.715713in}{1.854594in}}%
\pgfpathlineto{\pgfqpoint{1.769863in}{1.880437in}}%
\pgfpathlineto{\pgfqpoint{1.824013in}{1.901850in}}%
\pgfpathlineto{\pgfqpoint{1.878163in}{1.918201in}}%
\pgfpathlineto{\pgfqpoint{1.932313in}{1.928924in}}%
\pgfpathlineto{\pgfqpoint{1.986463in}{1.933600in}}%
\pgfpathlineto{\pgfqpoint{2.013537in}{1.933600in}}%
\pgfpathlineto{\pgfqpoint{2.040612in}{1.932036in}}%
\pgfpathlineto{\pgfqpoint{2.094762in}{1.924297in}}%
\pgfpathlineto{\pgfqpoint{2.148912in}{1.910696in}}%
\pgfpathlineto{\pgfqpoint{2.203062in}{1.891737in}}%
\pgfpathlineto{\pgfqpoint{2.257212in}{1.868029in}}%
\pgfpathlineto{\pgfqpoint{2.311362in}{1.840212in}}%
\pgfpathlineto{\pgfqpoint{2.365512in}{1.808899in}}%
\pgfpathlineto{\pgfqpoint{2.419662in}{1.774644in}}%
\pgfpathlineto{\pgfqpoint{2.500887in}{1.718774in}}%
\pgfpathlineto{\pgfqpoint{2.582112in}{1.658661in}}%
\pgfpathlineto{\pgfqpoint{2.690412in}{1.573580in}}%
\pgfpathlineto{\pgfqpoint{2.798712in}{1.484365in}}%
\pgfpathlineto{\pgfqpoint{2.934087in}{1.368722in}}%
\pgfpathlineto{\pgfqpoint{3.096537in}{1.225745in}}%
\pgfpathlineto{\pgfqpoint{3.313137in}{1.030397in}}%
\pgfpathlineto{\pgfqpoint{3.583887in}{0.781444in}}%
\pgfpathlineto{\pgfqpoint{3.815278in}{0.566020in}}%
\pgfpathlineto{\pgfqpoint{3.815278in}{0.566020in}}%
\pgfusepath{stroke}%
\end{pgfscope}%
\begin{pgfscope}%
\pgfpathrectangle{\pgfqpoint{0.198611in}{0.198611in}}{\pgfqpoint{3.602778in}{2.602778in}}%
\pgfusepath{clip}%
\pgfsetrectcap%
\pgfsetroundjoin%
\pgfsetlinewidth{0.501875pt}%
\definecolor{currentstroke}{rgb}{0.215686,0.494118,0.721569}%
\pgfsetstrokecolor{currentstroke}%
\pgfsetdash{}{0pt}%
\pgfpathmoveto{\pgfqpoint{0.184722in}{2.433980in}}%
\pgfpathlineto{\pgfqpoint{0.524413in}{2.118487in}}%
\pgfpathlineto{\pgfqpoint{0.768088in}{1.895850in}}%
\pgfpathlineto{\pgfqpoint{0.957613in}{1.726186in}}%
\pgfpathlineto{\pgfqpoint{1.120063in}{1.584564in}}%
\pgfpathlineto{\pgfqpoint{1.255438in}{1.470592in}}%
\pgfpathlineto{\pgfqpoint{1.363738in}{1.383273in}}%
\pgfpathlineto{\pgfqpoint{1.444963in}{1.320897in}}%
\pgfpathlineto{\pgfqpoint{1.526188in}{1.262073in}}%
\pgfpathlineto{\pgfqpoint{1.607413in}{1.207893in}}%
\pgfpathlineto{\pgfqpoint{1.661563in}{1.175044in}}%
\pgfpathlineto{\pgfqpoint{1.715713in}{1.145406in}}%
\pgfpathlineto{\pgfqpoint{1.769863in}{1.119563in}}%
\pgfpathlineto{\pgfqpoint{1.824013in}{1.098150in}}%
\pgfpathlineto{\pgfqpoint{1.878163in}{1.081799in}}%
\pgfpathlineto{\pgfqpoint{1.932313in}{1.071076in}}%
\pgfpathlineto{\pgfqpoint{1.986463in}{1.066400in}}%
\pgfpathlineto{\pgfqpoint{2.013537in}{1.066400in}}%
\pgfpathlineto{\pgfqpoint{2.040612in}{1.067964in}}%
\pgfpathlineto{\pgfqpoint{2.094762in}{1.075703in}}%
\pgfpathlineto{\pgfqpoint{2.148912in}{1.089304in}}%
\pgfpathlineto{\pgfqpoint{2.203062in}{1.108263in}}%
\pgfpathlineto{\pgfqpoint{2.257212in}{1.131971in}}%
\pgfpathlineto{\pgfqpoint{2.311362in}{1.159788in}}%
\pgfpathlineto{\pgfqpoint{2.365512in}{1.191101in}}%
\pgfpathlineto{\pgfqpoint{2.419662in}{1.225356in}}%
\pgfpathlineto{\pgfqpoint{2.500887in}{1.281226in}}%
\pgfpathlineto{\pgfqpoint{2.582112in}{1.341339in}}%
\pgfpathlineto{\pgfqpoint{2.690412in}{1.426420in}}%
\pgfpathlineto{\pgfqpoint{2.798712in}{1.515635in}}%
\pgfpathlineto{\pgfqpoint{2.934087in}{1.631278in}}%
\pgfpathlineto{\pgfqpoint{3.096537in}{1.774255in}}%
\pgfpathlineto{\pgfqpoint{3.313137in}{1.969603in}}%
\pgfpathlineto{\pgfqpoint{3.583887in}{2.218556in}}%
\pgfpathlineto{\pgfqpoint{3.815278in}{2.433980in}}%
\pgfpathlineto{\pgfqpoint{3.815278in}{2.433980in}}%
\pgfusepath{stroke}%
\end{pgfscope}%
\begin{pgfscope}%
\pgfpathrectangle{\pgfqpoint{0.198611in}{0.198611in}}{\pgfqpoint{3.602778in}{2.602778in}}%
\pgfusepath{clip}%
\pgfsetbuttcap%
\pgfsetroundjoin%
\pgfsetlinewidth{0.501875pt}%
\definecolor{currentstroke}{rgb}{0.501961,0.501961,0.501961}%
\pgfsetstrokecolor{currentstroke}%
\pgfsetdash{{1.850000pt}{0.800000pt}}{0.000000pt}%
\pgfpathmoveto{\pgfqpoint{1.535234in}{0.184722in}}%
\pgfpathlineto{\pgfqpoint{3.815278in}{2.380971in}}%
\pgfpathlineto{\pgfqpoint{3.815278in}{2.380971in}}%
\pgfusepath{stroke}%
\end{pgfscope}%
\begin{pgfscope}%
\pgfpathrectangle{\pgfqpoint{0.198611in}{0.198611in}}{\pgfqpoint{3.602778in}{2.602778in}}%
\pgfusepath{clip}%
\pgfsetbuttcap%
\pgfsetroundjoin%
\pgfsetlinewidth{0.501875pt}%
\definecolor{currentstroke}{rgb}{0.501961,0.501961,0.501961}%
\pgfsetstrokecolor{currentstroke}%
\pgfsetdash{{1.850000pt}{0.800000pt}}{0.000000pt}%
\pgfpathmoveto{\pgfqpoint{0.184722in}{2.380971in}}%
\pgfpathlineto{\pgfqpoint{2.464766in}{0.184722in}}%
\pgfpathlineto{\pgfqpoint{2.464766in}{0.184722in}}%
\pgfusepath{stroke}%
\end{pgfscope}%
\begin{pgfscope}%
\pgfpathrectangle{\pgfqpoint{0.198611in}{0.198611in}}{\pgfqpoint{3.602778in}{2.602778in}}%
\pgfusepath{clip}%
\pgfsetbuttcap%
\pgfsetroundjoin%
\pgfsetlinewidth{0.501875pt}%
\definecolor{currentstroke}{rgb}{0.501961,0.501961,0.501961}%
\pgfsetstrokecolor{currentstroke}%
\pgfsetdash{{1.850000pt}{0.800000pt}}{0.000000pt}%
\pgfpathmoveto{\pgfqpoint{2.464766in}{2.815278in}}%
\pgfpathlineto{\pgfqpoint{0.184722in}{0.619029in}}%
\pgfpathlineto{\pgfqpoint{0.184722in}{0.619029in}}%
\pgfusepath{stroke}%
\end{pgfscope}%
\begin{pgfscope}%
\pgfpathrectangle{\pgfqpoint{0.198611in}{0.198611in}}{\pgfqpoint{3.602778in}{2.602778in}}%
\pgfusepath{clip}%
\pgfsetbuttcap%
\pgfsetroundjoin%
\pgfsetlinewidth{0.501875pt}%
\definecolor{currentstroke}{rgb}{0.501961,0.501961,0.501961}%
\pgfsetstrokecolor{currentstroke}%
\pgfsetdash{{1.850000pt}{0.800000pt}}{0.000000pt}%
\pgfpathmoveto{\pgfqpoint{3.815278in}{0.619029in}}%
\pgfpathlineto{\pgfqpoint{1.535234in}{2.815278in}}%
\pgfpathlineto{\pgfqpoint{1.535234in}{2.815278in}}%
\pgfusepath{stroke}%
\end{pgfscope}%
\begin{pgfscope}%
\pgfsetrectcap%
\pgfsetmiterjoin%
\pgfsetlinewidth{0.501875pt}%
\definecolor{currentstroke}{rgb}{0.000000,0.000000,0.000000}%
\pgfsetstrokecolor{currentstroke}%
\pgfsetdash{}{0pt}%
\pgfpathmoveto{\pgfqpoint{2.000000in}{0.198611in}}%
\pgfpathlineto{\pgfqpoint{2.000000in}{2.801389in}}%
\pgfusepath{stroke}%
\end{pgfscope}%
\begin{pgfscope}%
\pgfsetrectcap%
\pgfsetmiterjoin%
\pgfsetlinewidth{0.501875pt}%
\definecolor{currentstroke}{rgb}{0.000000,0.000000,0.000000}%
\pgfsetstrokecolor{currentstroke}%
\pgfsetdash{}{0pt}%
\pgfpathmoveto{\pgfqpoint{0.198611in}{0.632407in}}%
\pgfpathlineto{\pgfqpoint{3.801389in}{0.632407in}}%
\pgfusepath{stroke}%
\end{pgfscope}%
\begin{pgfscope}%
\pgfsetroundcap%
\pgfsetroundjoin%
\pgfsetlinewidth{0.501875pt}%
\definecolor{currentstroke}{rgb}{0.000000,0.000000,0.000000}%
\pgfsetstrokecolor{currentstroke}%
\pgfsetdash{}{0pt}%
\pgfpathmoveto{\pgfqpoint{2.000000in}{2.807510in}}%
\pgfpathquadraticcurveto{\pgfqpoint{2.000000in}{2.808331in}}{\pgfqpoint{2.000000in}{2.801389in}}%
\pgfusepath{stroke}%
\end{pgfscope}%
\begin{pgfscope}%
\pgfsetroundcap%
\pgfsetroundjoin%
\pgfsetlinewidth{0.501875pt}%
\definecolor{currentstroke}{rgb}{0.000000,0.000000,0.000000}%
\pgfsetstrokecolor{currentstroke}%
\pgfsetdash{}{0pt}%
\pgfpathmoveto{\pgfqpoint{1.972222in}{2.751954in}}%
\pgfpathlineto{\pgfqpoint{2.000000in}{2.807510in}}%
\pgfpathlineto{\pgfqpoint{2.027778in}{2.751954in}}%
\pgfusepath{stroke}%
\end{pgfscope}%
\begin{pgfscope}%
\pgftext[x=2.000000in,y=2.870833in,,bottom]{\rmfamily\fontsize{10.000000}{12.000000}\selectfont \(\displaystyle \omega^{\prime}\)}%
\end{pgfscope}%
\begin{pgfscope}%
\pgfsetroundcap%
\pgfsetroundjoin%
\pgfsetlinewidth{0.501875pt}%
\definecolor{currentstroke}{rgb}{0.000000,0.000000,0.000000}%
\pgfsetstrokecolor{currentstroke}%
\pgfsetdash{}{0pt}%
\pgfpathmoveto{\pgfqpoint{3.807510in}{0.632407in}}%
\pgfpathquadraticcurveto{\pgfqpoint{3.808332in}{0.632407in}}{\pgfqpoint{3.801389in}{0.632407in}}%
\pgfusepath{stroke}%
\end{pgfscope}%
\begin{pgfscope}%
\pgfsetroundcap%
\pgfsetroundjoin%
\pgfsetlinewidth{0.501875pt}%
\definecolor{currentstroke}{rgb}{0.000000,0.000000,0.000000}%
\pgfsetstrokecolor{currentstroke}%
\pgfsetdash{}{0pt}%
\pgfpathmoveto{\pgfqpoint{3.751955in}{0.660185in}}%
\pgfpathlineto{\pgfqpoint{3.807510in}{0.632407in}}%
\pgfpathlineto{\pgfqpoint{3.751955in}{0.604630in}}%
\pgfusepath{stroke}%
\end{pgfscope}%
\begin{pgfscope}%
\pgftext[x=3.870833in,y=0.632407in,left,]{\rmfamily\fontsize{10.000000}{12.000000}\selectfont \(\displaystyle k^{\prime}\)}%
\end{pgfscope}%
\end{pgfpicture}%
\makeatother%
\endgroup%
} %
        \caption{Das zu berechnende Integral aus \cref{eq:mass_shell_convolution} visualisiert für $k=0$}
        \label{fig:mass_shell_convolution}
    \end{minipage}\hfill
    \begin{minipage}{0.5\textwidth}
        \centering
        \resizebox{\textwidth}{!}{%% Creator: Matplotlib, PGF backend
%%
%% To include the figure in your LaTeX document, write
%%   \input{<filename>.pgf}
%%
%% Make sure the required packages are loaded in your preamble
%%   \usepackage{pgf}
%%
%% Figures using additional raster images can only be included by \input if
%% they are in the same directory as the main LaTeX file. For loading figures
%% from other directories you can use the `import` package
%%   \usepackage{import}
%% and then include the figures with
%%   \import{<path to file>}{<filename>.pgf}
%%
%% Matplotlib used the following preamble
%%   \usepackage[utf8x]{inputenc}
%%   \usepackage[T1]{fontenc}
%%   \usepackage{amssymb}
%%
\begingroup%
\makeatletter%
\begin{pgfpicture}%
\pgfpathrectangle{\pgfpointorigin}{\pgfqpoint{4.000000in}{3.000000in}}%
\pgfusepath{use as bounding box, clip}%
\begin{pgfscope}%
\pgfsetbuttcap%
\pgfsetmiterjoin%
\definecolor{currentfill}{rgb}{1.000000,1.000000,1.000000}%
\pgfsetfillcolor{currentfill}%
\pgfsetlinewidth{0.000000pt}%
\definecolor{currentstroke}{rgb}{1.000000,1.000000,1.000000}%
\pgfsetstrokecolor{currentstroke}%
\pgfsetdash{}{0pt}%
\pgfpathmoveto{\pgfqpoint{0.000000in}{0.000000in}}%
\pgfpathlineto{\pgfqpoint{4.000000in}{0.000000in}}%
\pgfpathlineto{\pgfqpoint{4.000000in}{3.000000in}}%
\pgfpathlineto{\pgfqpoint{0.000000in}{3.000000in}}%
\pgfpathclose%
\pgfusepath{fill}%
\end{pgfscope}%
\begin{pgfscope}%
\pgfsetbuttcap%
\pgfsetmiterjoin%
\definecolor{currentfill}{rgb}{1.000000,1.000000,1.000000}%
\pgfsetfillcolor{currentfill}%
\pgfsetlinewidth{0.000000pt}%
\definecolor{currentstroke}{rgb}{0.000000,0.000000,0.000000}%
\pgfsetstrokecolor{currentstroke}%
\pgfsetstrokeopacity{0.000000}%
\pgfsetdash{}{0pt}%
\pgfpathmoveto{\pgfqpoint{0.198611in}{0.198611in}}%
\pgfpathlineto{\pgfqpoint{3.801389in}{0.198611in}}%
\pgfpathlineto{\pgfqpoint{3.801389in}{2.801389in}}%
\pgfpathlineto{\pgfqpoint{0.198611in}{2.801389in}}%
\pgfpathclose%
\pgfusepath{fill}%
\end{pgfscope}%
\begin{pgfscope}%
\pgfpathrectangle{\pgfqpoint{0.198611in}{0.198611in}}{\pgfqpoint{3.602778in}{2.602778in}} %
\pgfusepath{clip}%
\pgfsetrectcap%
\pgfsetroundjoin%
\pgfsetlinewidth{1.003750pt}%
\definecolor{currentstroke}{rgb}{0.215686,0.494118,0.721569}%
\pgfsetstrokecolor{currentstroke}%
\pgfsetdash{}{0pt}%
\pgfpathmoveto{\pgfqpoint{0.198611in}{1.239722in}}%
\pgfpathlineto{\pgfqpoint{0.235003in}{1.266013in}}%
\pgfpathlineto{\pgfqpoint{0.271395in}{1.292304in}}%
\pgfpathlineto{\pgfqpoint{0.307786in}{1.318594in}}%
\pgfpathlineto{\pgfqpoint{0.344178in}{1.344885in}}%
\pgfpathlineto{\pgfqpoint{0.380570in}{1.371176in}}%
\pgfpathlineto{\pgfqpoint{0.416961in}{1.397466in}}%
\pgfpathlineto{\pgfqpoint{0.453353in}{1.423757in}}%
\pgfpathlineto{\pgfqpoint{0.489745in}{1.450048in}}%
\pgfpathlineto{\pgfqpoint{0.526136in}{1.476338in}}%
\pgfpathlineto{\pgfqpoint{0.562528in}{1.502629in}}%
\pgfpathlineto{\pgfqpoint{0.598920in}{1.528920in}}%
\pgfpathlineto{\pgfqpoint{0.635311in}{1.555210in}}%
\pgfpathlineto{\pgfqpoint{0.671703in}{1.581501in}}%
\pgfpathlineto{\pgfqpoint{0.708095in}{1.607792in}}%
\pgfpathlineto{\pgfqpoint{0.744487in}{1.634082in}}%
\pgfpathlineto{\pgfqpoint{0.780878in}{1.660373in}}%
\pgfpathlineto{\pgfqpoint{0.817270in}{1.686664in}}%
\pgfpathlineto{\pgfqpoint{0.853662in}{1.712955in}}%
\pgfpathlineto{\pgfqpoint{0.890053in}{1.739245in}}%
\pgfpathlineto{\pgfqpoint{0.926445in}{1.765536in}}%
\pgfpathlineto{\pgfqpoint{0.962837in}{1.791827in}}%
\pgfpathlineto{\pgfqpoint{0.999228in}{1.818117in}}%
\pgfpathlineto{\pgfqpoint{1.035620in}{1.844408in}}%
\pgfpathlineto{\pgfqpoint{1.072012in}{1.870699in}}%
\pgfpathlineto{\pgfqpoint{1.108403in}{1.896989in}}%
\pgfpathlineto{\pgfqpoint{1.144795in}{1.923280in}}%
\pgfpathlineto{\pgfqpoint{1.181187in}{1.949571in}}%
\pgfpathlineto{\pgfqpoint{1.217579in}{1.975861in}}%
\pgfpathlineto{\pgfqpoint{1.253970in}{2.002152in}}%
\pgfpathlineto{\pgfqpoint{1.290362in}{2.028443in}}%
\pgfpathlineto{\pgfqpoint{1.326754in}{2.054733in}}%
\pgfpathlineto{\pgfqpoint{1.363145in}{2.081024in}}%
\pgfpathlineto{\pgfqpoint{1.399537in}{2.107315in}}%
\pgfpathlineto{\pgfqpoint{1.435929in}{2.133605in}}%
\pgfpathlineto{\pgfqpoint{1.472320in}{2.159896in}}%
\pgfpathlineto{\pgfqpoint{1.508712in}{2.186187in}}%
\pgfpathlineto{\pgfqpoint{1.545104in}{2.212478in}}%
\pgfpathlineto{\pgfqpoint{1.581496in}{2.238768in}}%
\pgfpathlineto{\pgfqpoint{1.617887in}{2.265059in}}%
\pgfpathlineto{\pgfqpoint{1.654279in}{2.291350in}}%
\pgfpathlineto{\pgfqpoint{1.690671in}{2.317640in}}%
\pgfpathlineto{\pgfqpoint{1.727062in}{2.343931in}}%
\pgfpathlineto{\pgfqpoint{1.763454in}{2.370222in}}%
\pgfpathlineto{\pgfqpoint{1.799846in}{2.396512in}}%
\pgfpathlineto{\pgfqpoint{1.836237in}{2.422803in}}%
\pgfpathlineto{\pgfqpoint{1.872629in}{2.449094in}}%
\pgfpathlineto{\pgfqpoint{1.909021in}{2.475384in}}%
\pgfpathlineto{\pgfqpoint{1.945412in}{2.501675in}}%
\pgfpathlineto{\pgfqpoint{1.981804in}{2.527966in}}%
\pgfpathlineto{\pgfqpoint{2.018196in}{2.554256in}}%
\pgfpathlineto{\pgfqpoint{2.054588in}{2.580547in}}%
\pgfpathlineto{\pgfqpoint{2.090979in}{2.606838in}}%
\pgfpathlineto{\pgfqpoint{2.127371in}{2.633129in}}%
\pgfpathlineto{\pgfqpoint{2.163763in}{2.659419in}}%
\pgfpathlineto{\pgfqpoint{2.200154in}{2.685710in}}%
\pgfpathlineto{\pgfqpoint{2.236546in}{2.712001in}}%
\pgfpathlineto{\pgfqpoint{2.272938in}{2.738291in}}%
\pgfpathlineto{\pgfqpoint{2.309329in}{2.764582in}}%
\pgfpathlineto{\pgfqpoint{2.345721in}{2.790873in}}%
\pgfpathlineto{\pgfqpoint{2.379503in}{2.815278in}}%
\pgfusepath{stroke}%
\end{pgfscope}%
\begin{pgfscope}%
\pgfpathrectangle{\pgfqpoint{0.198611in}{0.198611in}}{\pgfqpoint{3.602778in}{2.602778in}} %
\pgfusepath{clip}%
\pgfsetrectcap%
\pgfsetroundjoin%
\pgfsetlinewidth{1.003750pt}%
\definecolor{currentstroke}{rgb}{0.215686,0.494118,0.721569}%
\pgfsetstrokecolor{currentstroke}%
\pgfsetdash{}{0pt}%
\pgfpathmoveto{\pgfqpoint{1.620497in}{0.184722in}}%
\pgfpathlineto{\pgfqpoint{1.654279in}{0.209127in}}%
\pgfpathlineto{\pgfqpoint{1.690671in}{0.235418in}}%
\pgfpathlineto{\pgfqpoint{1.727062in}{0.261709in}}%
\pgfpathlineto{\pgfqpoint{1.763454in}{0.287999in}}%
\pgfpathlineto{\pgfqpoint{1.799846in}{0.314290in}}%
\pgfpathlineto{\pgfqpoint{1.836237in}{0.340581in}}%
\pgfpathlineto{\pgfqpoint{1.872629in}{0.366871in}}%
\pgfpathlineto{\pgfqpoint{1.909021in}{0.393162in}}%
\pgfpathlineto{\pgfqpoint{1.945412in}{0.419453in}}%
\pgfpathlineto{\pgfqpoint{1.981804in}{0.445744in}}%
\pgfpathlineto{\pgfqpoint{2.018196in}{0.472034in}}%
\pgfpathlineto{\pgfqpoint{2.054588in}{0.498325in}}%
\pgfpathlineto{\pgfqpoint{2.090979in}{0.524616in}}%
\pgfpathlineto{\pgfqpoint{2.127371in}{0.550906in}}%
\pgfpathlineto{\pgfqpoint{2.163763in}{0.577197in}}%
\pgfpathlineto{\pgfqpoint{2.200154in}{0.603488in}}%
\pgfpathlineto{\pgfqpoint{2.236546in}{0.629778in}}%
\pgfpathlineto{\pgfqpoint{2.272938in}{0.656069in}}%
\pgfpathlineto{\pgfqpoint{2.309329in}{0.682360in}}%
\pgfpathlineto{\pgfqpoint{2.345721in}{0.708650in}}%
\pgfpathlineto{\pgfqpoint{2.382113in}{0.734941in}}%
\pgfpathlineto{\pgfqpoint{2.418504in}{0.761232in}}%
\pgfpathlineto{\pgfqpoint{2.454896in}{0.787522in}}%
\pgfpathlineto{\pgfqpoint{2.491288in}{0.813813in}}%
\pgfpathlineto{\pgfqpoint{2.527680in}{0.840104in}}%
\pgfpathlineto{\pgfqpoint{2.564071in}{0.866395in}}%
\pgfpathlineto{\pgfqpoint{2.600463in}{0.892685in}}%
\pgfpathlineto{\pgfqpoint{2.636855in}{0.918976in}}%
\pgfpathlineto{\pgfqpoint{2.673246in}{0.945267in}}%
\pgfpathlineto{\pgfqpoint{2.709638in}{0.971557in}}%
\pgfpathlineto{\pgfqpoint{2.746030in}{0.997848in}}%
\pgfpathlineto{\pgfqpoint{2.782421in}{1.024139in}}%
\pgfpathlineto{\pgfqpoint{2.818813in}{1.050429in}}%
\pgfpathlineto{\pgfqpoint{2.855205in}{1.076720in}}%
\pgfpathlineto{\pgfqpoint{2.891597in}{1.103011in}}%
\pgfpathlineto{\pgfqpoint{2.927988in}{1.129301in}}%
\pgfpathlineto{\pgfqpoint{2.964380in}{1.155592in}}%
\pgfpathlineto{\pgfqpoint{3.000772in}{1.181883in}}%
\pgfpathlineto{\pgfqpoint{3.037163in}{1.208173in}}%
\pgfpathlineto{\pgfqpoint{3.073555in}{1.234464in}}%
\pgfpathlineto{\pgfqpoint{3.109947in}{1.260755in}}%
\pgfpathlineto{\pgfqpoint{3.146338in}{1.287045in}}%
\pgfpathlineto{\pgfqpoint{3.182730in}{1.313336in}}%
\pgfpathlineto{\pgfqpoint{3.219122in}{1.339627in}}%
\pgfpathlineto{\pgfqpoint{3.255513in}{1.365918in}}%
\pgfpathlineto{\pgfqpoint{3.291905in}{1.392208in}}%
\pgfpathlineto{\pgfqpoint{3.328297in}{1.418499in}}%
\pgfpathlineto{\pgfqpoint{3.364689in}{1.444790in}}%
\pgfpathlineto{\pgfqpoint{3.401080in}{1.471080in}}%
\pgfpathlineto{\pgfqpoint{3.437472in}{1.497371in}}%
\pgfpathlineto{\pgfqpoint{3.473864in}{1.523662in}}%
\pgfpathlineto{\pgfqpoint{3.510255in}{1.549952in}}%
\pgfpathlineto{\pgfqpoint{3.546647in}{1.576243in}}%
\pgfpathlineto{\pgfqpoint{3.583039in}{1.602534in}}%
\pgfpathlineto{\pgfqpoint{3.619430in}{1.628824in}}%
\pgfpathlineto{\pgfqpoint{3.655822in}{1.655115in}}%
\pgfpathlineto{\pgfqpoint{3.692214in}{1.681406in}}%
\pgfpathlineto{\pgfqpoint{3.728605in}{1.707696in}}%
\pgfpathlineto{\pgfqpoint{3.764997in}{1.733987in}}%
\pgfpathlineto{\pgfqpoint{3.801389in}{1.760278in}}%
\pgfusepath{stroke}%
\end{pgfscope}%
\begin{pgfscope}%
\pgfpathrectangle{\pgfqpoint{0.198611in}{0.198611in}}{\pgfqpoint{3.602778in}{2.602778in}} %
\pgfusepath{clip}%
\pgfsetrectcap%
\pgfsetroundjoin%
\pgfsetlinewidth{1.003750pt}%
\definecolor{currentstroke}{rgb}{0.894118,0.101961,0.109804}%
\pgfsetstrokecolor{currentstroke}%
\pgfsetdash{}{0pt}%
\pgfpathmoveto{\pgfqpoint{1.620497in}{2.815278in}}%
\pgfpathlineto{\pgfqpoint{1.654279in}{2.790873in}}%
\pgfpathlineto{\pgfqpoint{1.690671in}{2.764582in}}%
\pgfpathlineto{\pgfqpoint{1.727062in}{2.738291in}}%
\pgfpathlineto{\pgfqpoint{1.763454in}{2.712001in}}%
\pgfpathlineto{\pgfqpoint{1.799846in}{2.685710in}}%
\pgfpathlineto{\pgfqpoint{1.836237in}{2.659419in}}%
\pgfpathlineto{\pgfqpoint{1.872629in}{2.633129in}}%
\pgfpathlineto{\pgfqpoint{1.909021in}{2.606838in}}%
\pgfpathlineto{\pgfqpoint{1.945412in}{2.580547in}}%
\pgfpathlineto{\pgfqpoint{1.981804in}{2.554256in}}%
\pgfpathlineto{\pgfqpoint{2.018196in}{2.527966in}}%
\pgfpathlineto{\pgfqpoint{2.054588in}{2.501675in}}%
\pgfpathlineto{\pgfqpoint{2.090979in}{2.475384in}}%
\pgfpathlineto{\pgfqpoint{2.127371in}{2.449094in}}%
\pgfpathlineto{\pgfqpoint{2.163763in}{2.422803in}}%
\pgfpathlineto{\pgfqpoint{2.200154in}{2.396512in}}%
\pgfpathlineto{\pgfqpoint{2.236546in}{2.370222in}}%
\pgfpathlineto{\pgfqpoint{2.272938in}{2.343931in}}%
\pgfpathlineto{\pgfqpoint{2.309329in}{2.317640in}}%
\pgfpathlineto{\pgfqpoint{2.345721in}{2.291350in}}%
\pgfpathlineto{\pgfqpoint{2.382113in}{2.265059in}}%
\pgfpathlineto{\pgfqpoint{2.418504in}{2.238768in}}%
\pgfpathlineto{\pgfqpoint{2.454896in}{2.212478in}}%
\pgfpathlineto{\pgfqpoint{2.491288in}{2.186187in}}%
\pgfpathlineto{\pgfqpoint{2.527680in}{2.159896in}}%
\pgfpathlineto{\pgfqpoint{2.564071in}{2.133605in}}%
\pgfpathlineto{\pgfqpoint{2.600463in}{2.107315in}}%
\pgfpathlineto{\pgfqpoint{2.636855in}{2.081024in}}%
\pgfpathlineto{\pgfqpoint{2.673246in}{2.054733in}}%
\pgfpathlineto{\pgfqpoint{2.709638in}{2.028443in}}%
\pgfpathlineto{\pgfqpoint{2.746030in}{2.002152in}}%
\pgfpathlineto{\pgfqpoint{2.782421in}{1.975861in}}%
\pgfpathlineto{\pgfqpoint{2.818813in}{1.949571in}}%
\pgfpathlineto{\pgfqpoint{2.855205in}{1.923280in}}%
\pgfpathlineto{\pgfqpoint{2.891597in}{1.896989in}}%
\pgfpathlineto{\pgfqpoint{2.927988in}{1.870699in}}%
\pgfpathlineto{\pgfqpoint{2.964380in}{1.844408in}}%
\pgfpathlineto{\pgfqpoint{3.000772in}{1.818117in}}%
\pgfpathlineto{\pgfqpoint{3.037163in}{1.791827in}}%
\pgfpathlineto{\pgfqpoint{3.073555in}{1.765536in}}%
\pgfpathlineto{\pgfqpoint{3.109947in}{1.739245in}}%
\pgfpathlineto{\pgfqpoint{3.146338in}{1.712955in}}%
\pgfpathlineto{\pgfqpoint{3.182730in}{1.686664in}}%
\pgfpathlineto{\pgfqpoint{3.219122in}{1.660373in}}%
\pgfpathlineto{\pgfqpoint{3.255513in}{1.634082in}}%
\pgfpathlineto{\pgfqpoint{3.291905in}{1.607792in}}%
\pgfpathlineto{\pgfqpoint{3.328297in}{1.581501in}}%
\pgfpathlineto{\pgfqpoint{3.364689in}{1.555210in}}%
\pgfpathlineto{\pgfqpoint{3.401080in}{1.528920in}}%
\pgfpathlineto{\pgfqpoint{3.437472in}{1.502629in}}%
\pgfpathlineto{\pgfqpoint{3.473864in}{1.476338in}}%
\pgfpathlineto{\pgfqpoint{3.510255in}{1.450048in}}%
\pgfpathlineto{\pgfqpoint{3.546647in}{1.423757in}}%
\pgfpathlineto{\pgfqpoint{3.583039in}{1.397466in}}%
\pgfpathlineto{\pgfqpoint{3.619430in}{1.371176in}}%
\pgfpathlineto{\pgfqpoint{3.655822in}{1.344885in}}%
\pgfpathlineto{\pgfqpoint{3.692214in}{1.318594in}}%
\pgfpathlineto{\pgfqpoint{3.728605in}{1.292304in}}%
\pgfpathlineto{\pgfqpoint{3.764997in}{1.266013in}}%
\pgfpathlineto{\pgfqpoint{3.801389in}{1.239722in}}%
\pgfusepath{stroke}%
\end{pgfscope}%
\begin{pgfscope}%
\pgfpathrectangle{\pgfqpoint{0.198611in}{0.198611in}}{\pgfqpoint{3.602778in}{2.602778in}} %
\pgfusepath{clip}%
\pgfsetrectcap%
\pgfsetroundjoin%
\pgfsetlinewidth{1.003750pt}%
\definecolor{currentstroke}{rgb}{0.894118,0.101961,0.109804}%
\pgfsetstrokecolor{currentstroke}%
\pgfsetdash{}{0pt}%
\pgfpathmoveto{\pgfqpoint{0.198611in}{1.760278in}}%
\pgfpathlineto{\pgfqpoint{0.235003in}{1.733987in}}%
\pgfpathlineto{\pgfqpoint{0.271395in}{1.707696in}}%
\pgfpathlineto{\pgfqpoint{0.307786in}{1.681406in}}%
\pgfpathlineto{\pgfqpoint{0.344178in}{1.655115in}}%
\pgfpathlineto{\pgfqpoint{0.380570in}{1.628824in}}%
\pgfpathlineto{\pgfqpoint{0.416961in}{1.602534in}}%
\pgfpathlineto{\pgfqpoint{0.453353in}{1.576243in}}%
\pgfpathlineto{\pgfqpoint{0.489745in}{1.549952in}}%
\pgfpathlineto{\pgfqpoint{0.526136in}{1.523662in}}%
\pgfpathlineto{\pgfqpoint{0.562528in}{1.497371in}}%
\pgfpathlineto{\pgfqpoint{0.598920in}{1.471080in}}%
\pgfpathlineto{\pgfqpoint{0.635311in}{1.444790in}}%
\pgfpathlineto{\pgfqpoint{0.671703in}{1.418499in}}%
\pgfpathlineto{\pgfqpoint{0.708095in}{1.392208in}}%
\pgfpathlineto{\pgfqpoint{0.744487in}{1.365918in}}%
\pgfpathlineto{\pgfqpoint{0.780878in}{1.339627in}}%
\pgfpathlineto{\pgfqpoint{0.817270in}{1.313336in}}%
\pgfpathlineto{\pgfqpoint{0.853662in}{1.287045in}}%
\pgfpathlineto{\pgfqpoint{0.890053in}{1.260755in}}%
\pgfpathlineto{\pgfqpoint{0.926445in}{1.234464in}}%
\pgfpathlineto{\pgfqpoint{0.962837in}{1.208173in}}%
\pgfpathlineto{\pgfqpoint{0.999228in}{1.181883in}}%
\pgfpathlineto{\pgfqpoint{1.035620in}{1.155592in}}%
\pgfpathlineto{\pgfqpoint{1.072012in}{1.129301in}}%
\pgfpathlineto{\pgfqpoint{1.108403in}{1.103011in}}%
\pgfpathlineto{\pgfqpoint{1.144795in}{1.076720in}}%
\pgfpathlineto{\pgfqpoint{1.181187in}{1.050429in}}%
\pgfpathlineto{\pgfqpoint{1.217579in}{1.024139in}}%
\pgfpathlineto{\pgfqpoint{1.253970in}{0.997848in}}%
\pgfpathlineto{\pgfqpoint{1.290362in}{0.971557in}}%
\pgfpathlineto{\pgfqpoint{1.326754in}{0.945267in}}%
\pgfpathlineto{\pgfqpoint{1.363145in}{0.918976in}}%
\pgfpathlineto{\pgfqpoint{1.399537in}{0.892685in}}%
\pgfpathlineto{\pgfqpoint{1.435929in}{0.866395in}}%
\pgfpathlineto{\pgfqpoint{1.472320in}{0.840104in}}%
\pgfpathlineto{\pgfqpoint{1.508712in}{0.813813in}}%
\pgfpathlineto{\pgfqpoint{1.545104in}{0.787522in}}%
\pgfpathlineto{\pgfqpoint{1.581496in}{0.761232in}}%
\pgfpathlineto{\pgfqpoint{1.617887in}{0.734941in}}%
\pgfpathlineto{\pgfqpoint{1.654279in}{0.708650in}}%
\pgfpathlineto{\pgfqpoint{1.690671in}{0.682360in}}%
\pgfpathlineto{\pgfqpoint{1.727062in}{0.656069in}}%
\pgfpathlineto{\pgfqpoint{1.763454in}{0.629778in}}%
\pgfpathlineto{\pgfqpoint{1.799846in}{0.603488in}}%
\pgfpathlineto{\pgfqpoint{1.836237in}{0.577197in}}%
\pgfpathlineto{\pgfqpoint{1.872629in}{0.550906in}}%
\pgfpathlineto{\pgfqpoint{1.909021in}{0.524616in}}%
\pgfpathlineto{\pgfqpoint{1.945412in}{0.498325in}}%
\pgfpathlineto{\pgfqpoint{1.981804in}{0.472034in}}%
\pgfpathlineto{\pgfqpoint{2.018196in}{0.445744in}}%
\pgfpathlineto{\pgfqpoint{2.054588in}{0.419453in}}%
\pgfpathlineto{\pgfqpoint{2.090979in}{0.393162in}}%
\pgfpathlineto{\pgfqpoint{2.127371in}{0.366871in}}%
\pgfpathlineto{\pgfqpoint{2.163763in}{0.340581in}}%
\pgfpathlineto{\pgfqpoint{2.200154in}{0.314290in}}%
\pgfpathlineto{\pgfqpoint{2.236546in}{0.287999in}}%
\pgfpathlineto{\pgfqpoint{2.272938in}{0.261709in}}%
\pgfpathlineto{\pgfqpoint{2.309329in}{0.235418in}}%
\pgfpathlineto{\pgfqpoint{2.345721in}{0.209127in}}%
\pgfpathlineto{\pgfqpoint{2.379503in}{0.184722in}}%
\pgfusepath{stroke}%
\end{pgfscope}%
\begin{pgfscope}%
\pgfpathrectangle{\pgfqpoint{0.198611in}{0.198611in}}{\pgfqpoint{3.602778in}{2.602778in}} %
\pgfusepath{clip}%
\pgfsetrectcap%
\pgfsetroundjoin%
\pgfsetlinewidth{0.501875pt}%
\definecolor{currentstroke}{rgb}{0.501961,0.501961,0.501961}%
\pgfsetstrokecolor{currentstroke}%
\pgfsetdash{}{0pt}%
\pgfpathmoveto{\pgfqpoint{0.955194in}{1.213694in}}%
\pgfpathlineto{\pgfqpoint{2.000000in}{2.541111in}}%
\pgfusepath{stroke}%
\end{pgfscope}%
\begin{pgfscope}%
\pgfpathrectangle{\pgfqpoint{0.198611in}{0.198611in}}{\pgfqpoint{3.602778in}{2.602778in}} %
\pgfusepath{clip}%
\pgfsetrectcap%
\pgfsetroundjoin%
\pgfsetlinewidth{0.501875pt}%
\definecolor{currentstroke}{rgb}{0.501961,0.501961,0.501961}%
\pgfsetstrokecolor{currentstroke}%
\pgfsetdash{}{0pt}%
\pgfpathmoveto{\pgfqpoint{2.000000in}{0.458889in}}%
\pgfpathlineto{\pgfqpoint{3.441111in}{0.458889in}}%
\pgfusepath{stroke}%
\end{pgfscope}%
\begin{pgfscope}%
\pgfpathrectangle{\pgfqpoint{0.198611in}{0.198611in}}{\pgfqpoint{3.602778in}{2.602778in}} %
\pgfusepath{clip}%
\pgfsetrectcap%
\pgfsetroundjoin%
\pgfsetlinewidth{0.501875pt}%
\definecolor{currentstroke}{rgb}{0.501961,0.501961,0.501961}%
\pgfsetstrokecolor{currentstroke}%
\pgfsetdash{}{0pt}%
\pgfpathmoveto{\pgfqpoint{3.441111in}{0.458889in}}%
\pgfpathlineto{\pgfqpoint{3.441111in}{1.500000in}}%
\pgfusepath{stroke}%
\end{pgfscope}%
\begin{pgfscope}%
\pgftext[x=2.720556in,y=0.289708in,left,base]{\rmfamily\fontsize{10.000000}{12.000000}\selectfont d\(\displaystyle k^\prime\)}%
\end{pgfscope}%
\begin{pgfscope}%
\pgftext[x=3.477139in,y=0.914375in,left,base]{\rmfamily\fontsize{10.000000}{12.000000}\selectfont d\(\displaystyle \omega^\prime\)}%
\end{pgfscope}%
\begin{pgfscope}%
\pgftext[x=2.180139in,y=0.497931in,left,base]{\rmfamily\fontsize{10.000000}{12.000000}\selectfont \(\displaystyle \alpha\)}%
\end{pgfscope}%
\begin{pgfscope}%
\pgftext[x=0.703000in,y=1.460958in,left,base]{\rmfamily\fontsize{10.000000}{12.000000}\selectfont \(\displaystyle 2 \alpha\)}%
\end{pgfscope}%
\begin{pgfscope}%
\pgftext[x=1.189375in,y=2.085625in,left,base]{\rmfamily\fontsize{10.000000}{12.000000}\selectfont \(\displaystyle l\)}%
\end{pgfscope}%
\begin{pgfscope}%
\pgftext[x=1.549653in,y=1.825347in,left,base]{\rmfamily\fontsize{10.000000}{12.000000}\selectfont \(\displaystyle h\)}%
\end{pgfscope}%
\end{pgfpicture}%
\makeatother%
\endgroup%
}
        \caption{Die Kreuzungstelle bei $k_{0+}'$ von ganz nah angeschaut}
        \label{fig:schulgeometrie}
    \end{minipage}
\end{figure}

Da $\Delta_m$ Lorenz-invariant ist, sind $\Delta_m^2$ und $\rwhat{\Delta}_m^{*2}$ es auch. Es genügt also $\rwhat{\Delta_m^{*2}}$ für $k=0$ und positive $\omega$ zu berechnen. Alle anderen Werte holen wir uns dann aus der Lorenz-Invarianz. An \cref{fig:mass_shell_convolution} sehen wir schon, dass das Faltungsintegral \cref{eq:mass_shell_convolution} nur dann ungleich null ist, wenn $(\omega, k)$ oberhalb oder auf der 2$m$-Massenschale liegen. Es ist also insbesondere $\omega > 0$.

Um nun das Integral über zwei sich schneidende lineare\footnote{Linear in dem Sinne, dass die Distribution entlang einer Linie getragen ist. Nicht das es eine lineare Distribution ist} $\delta$-Distributionen zu berechnen bedienen wir uns eines Physikertricks und stellen uns die $\delta$-Distribution als Grenzwert ($h \rightarrow 0$) einer $\frac{1}{h}$-hohen und $h$-breiten Rechtecksfunktion vor. Dann ist das Integral über die sich schneidenden Rechteckfunktionen proportional zu der Schnittfläche und damit zu $l \cdot h$ in \cref{fig:schulgeometrie}. Außerdem schneiden sich die beiden Hyperbeln für $\omega \rightarrow +\infty$ in einem rechten Winkel, das Faltungsintegral ergibt hier also 2 da es zwei Schnittpunkte gibt.

Aus \cref{fig:schulgeometrie} lesen wir ab:

\begin{align}
    \tan(\alpha) &= \frac{\d \omega'}{\d k'}
    ~~\textrm{ und }~~
    \frac{h}{l} = \sin (2 \alpha) \nonumber \\
    \Rightarrow l &=
    \frac{h}{\sin\left(2 \arctan \left(\frac{\d \omega'}{\d k'}\right)\right)}
    = \frac{h \left(\left(\frac{\d \omega'}{\d k'}\right)^2+1\right)}
           {2 \frac{\d \omega'}{\d k'}}
    \label{eq:schulgeometrie2}
\end{align}

außerdem gilt

\begin{equation}
    \omega' = \sqrt{k^{\prime 2} + m^2}
    ~~\Longrightarrow~~
    \frac{\d \omega'}{\d k'} = \frac{k'}{\sqrt{k^{\prime 2}+m^2}}
    \label{eq:schulgeometrie3}
\end{equation}

Wenn wir nun \cref{eq:schulgeometrie2,eq:schulgeometrie3} sowie die vorhergehenden Gedanken kombinieren erhalten wir

\begin{align}
    \rwhat{\Delta_m}^{*2} (\omega, 0) &=
    \eqref{eq:mass_shell_convolution}
    \nonumber \\ &=
    C \frac{
        \left.\left(\d \omega'/\d k'\right)^2\right|_{k'_0} +1
    }
    {
        \left. \d \omega' / \d k'\right|_{k'_0}
    }
    \Theta(\omega^2-(2m)^2)
    \nonumber \\ &=
    C \frac{
        \sqrt{k_0^{\prime 2}+m^2}(2 k_0^{\prime 2}+m^2)
    }{
        2 k'_0 (k_0^2+m^2)
    }\Theta(\dots)
    \nonumber \\ &=
    C \frac{\sqrt{\frac{1}{4} \omega^2 -m^2+m^2} (\omega^2-4m^2+m^2)}
    {\sqrt{\omega^2-4m^2}(\frac{1}{4}\omega^2-m^2+m^2)}
    \Theta(\dots)
    \nonumber \\ &=
    C \frac{\omega^2-3m^2}{\omega \sqrt{\omega^2-4m^2}}\Theta(\dots)
    \stackrel{C=2}{=}
    2 \frac{\omega^2-3m^2}{\omega \sqrt{\omega^2-4m^2}}\Theta(\dots)
\end{align}
% \begin{dmath}
%     \rwhat{\Delta_m}^{*2} (\omega, 0) =
%     \eqref{eq:mass_shell_convolution}
%     =
%     C \frac{
%         \left.\left(\d \omega'/\d k'\right)^2\right|_{k'_0} +1
%     }
%     {
%         \left. \d \omega' / \d k'\right|_{k'_0}
%     }
%     \Theta(\omega^2-(2m)^2)
%     =
%     C \frac{
%         \sqrt{k_0^{\prime 2}+m^2}(2 k_0^{\prime 2}+m^2)
%     }{
%         2 k'_0 (k_0^2+m^2)
%     }\Theta(\dots)
%     =
%     C \frac{\sqrt{\frac{1}{4} \omega^2 -m^2+m^2} (\omega^2-4m^2+m^2)}
%     {\sqrt{\omega^2-4m^2}(\frac{1}{4}\omega^2-m^2+m^2)}
%     \Theta(\dots)
%     =
%     C \frac{\omega^2-3m^2}{\omega \sqrt{\omega^2-4m^2}}\Theta(\dots)
%     \stackrel{C=2}{=}
%     2 \frac{\omega^2-3m^2}{\omega \sqrt{\omega^2-4m^2}}\Theta(\dots)
%     \nonumber
% \end{dmath}

Jetzt erhalten wir $\rwhat{\Delta}_m^{*2}(\omega, k)$ für beliebige $k$ noch aus der Lorenz-Invarianz:

\begin{dmath}
    \rwhat{\Delta}_m^{*2}(\omega, k)
    \stackrel{(\omega,k) \sim (\sqrt{\omega^2-k^2},0)}{=}
    \rwhat{\Delta}_m^{*2}(\sqrt{\omega^2-k^2},0)
    = 2 \frac{\omega^2-k^2-3m^2}
              {\sqrt{\omega^2-k^2}\sqrt{\omega^2-k^2-4m^2}}
              \Theta(\omega^2-k^2-4m^2)
\end{dmath}

Es ist zu beachten, dass die Heaviside-Funktion genau bei der ersten Nullstelle der zweiten Wurzel im Nenner abschneidet und alle weiteren Nullstellen sowohl des Nenners als auch des Zählers außerhalb der $2m$-Massenschale und damit außerhalb des Trägers der Heaviside-Funktion liegen.


\begin{figure}
    \centering
    \begin{minipage}{0.55\textwidth}
        \centering
        \resizebox{\textwidth}{!}{%% Creator: Matplotlib, PGF backend
%%
%% To include the figure in your LaTeX document, write
%%   \input{<filename>.pgf}
%%
%% Make sure the required packages are loaded in your preamble
%%   \usepackage{pgf}
%%
%% Figures using additional raster images can only be included by \input if
%% they are in the same directory as the main LaTeX file. For loading figures
%% from other directories you can use the `import` package
%%   \usepackage{import}
%% and then include the figures with
%%   \import{<path to file>}{<filename>.pgf}
%%
%% Matplotlib used the following preamble
%%   \usepackage[utf8x]{inputenc}
%%   \usepackage[T1]{fontenc}
%%   \usepackage{amssymb}
%%
\begingroup%
\makeatletter%
\begin{pgfpicture}%
\pgfpathrectangle{\pgfpointorigin}{\pgfqpoint{4.000000in}{2.200000in}}%
\pgfusepath{use as bounding box, clip}%
\begin{pgfscope}%
\pgfsetbuttcap%
\pgfsetmiterjoin%
\definecolor{currentfill}{rgb}{1.000000,1.000000,1.000000}%
\pgfsetfillcolor{currentfill}%
\pgfsetlinewidth{0.000000pt}%
\definecolor{currentstroke}{rgb}{1.000000,1.000000,1.000000}%
\pgfsetstrokecolor{currentstroke}%
\pgfsetdash{}{0pt}%
\pgfpathmoveto{\pgfqpoint{0.000000in}{0.000000in}}%
\pgfpathlineto{\pgfqpoint{4.000000in}{0.000000in}}%
\pgfpathlineto{\pgfqpoint{4.000000in}{2.200000in}}%
\pgfpathlineto{\pgfqpoint{0.000000in}{2.200000in}}%
\pgfpathclose%
\pgfusepath{fill}%
\end{pgfscope}%
\begin{pgfscope}%
\pgfsetbuttcap%
\pgfsetmiterjoin%
\definecolor{currentfill}{rgb}{1.000000,1.000000,1.000000}%
\pgfsetfillcolor{currentfill}%
\pgfsetlinewidth{0.000000pt}%
\definecolor{currentstroke}{rgb}{0.000000,0.000000,0.000000}%
\pgfsetstrokecolor{currentstroke}%
\pgfsetstrokeopacity{0.000000}%
\pgfsetdash{}{0pt}%
\pgfpathmoveto{\pgfqpoint{0.198611in}{0.333208in}}%
\pgfpathlineto{\pgfqpoint{3.119722in}{0.333208in}}%
\pgfpathlineto{\pgfqpoint{3.119722in}{1.866792in}}%
\pgfpathlineto{\pgfqpoint{0.198611in}{1.866792in}}%
\pgfpathclose%
\pgfusepath{fill}%
\end{pgfscope}%
\begin{pgfscope}%
\pgfpathrectangle{\pgfqpoint{0.198611in}{0.333208in}}{\pgfqpoint{2.921111in}{1.533583in}} %
\pgfusepath{clip}%
\pgfsys@transformshift{0.198611in}{0.333208in}%
\pgftext[left,bottom]{\pgfimage[interpolate=true,width=2.921667in,height=1.535000in]{delta_m2-img0.png}}%
\end{pgfscope}%
\begin{pgfscope}%
\pgfpathrectangle{\pgfqpoint{0.198611in}{0.333208in}}{\pgfqpoint{2.921111in}{1.533583in}} %
\pgfusepath{clip}%
\pgfsetrectcap%
\pgfsetroundjoin%
\pgfsetlinewidth{0.501875pt}%
\definecolor{currentstroke}{rgb}{0.894118,0.101961,0.109804}%
\pgfsetstrokecolor{currentstroke}%
\pgfsetdash{}{0pt}%
\pgfpathmoveto{\pgfqpoint{0.208387in}{1.868458in}}%
\pgfpathlineto{\pgfqpoint{0.409353in}{1.669314in}}%
\pgfpathlineto{\pgfqpoint{0.573263in}{1.507381in}}%
\pgfpathlineto{\pgfqpoint{0.707903in}{1.374861in}}%
\pgfpathlineto{\pgfqpoint{0.819128in}{1.265885in}}%
\pgfpathlineto{\pgfqpoint{0.912791in}{1.174617in}}%
\pgfpathlineto{\pgfqpoint{0.994746in}{1.095284in}}%
\pgfpathlineto{\pgfqpoint{1.059139in}{1.033424in}}%
\pgfpathlineto{\pgfqpoint{1.117678in}{0.977674in}}%
\pgfpathlineto{\pgfqpoint{1.170364in}{0.928022in}}%
\pgfpathlineto{\pgfqpoint{1.217195in}{0.884431in}}%
\pgfpathlineto{\pgfqpoint{1.258172in}{0.846836in}}%
\pgfpathlineto{\pgfqpoint{1.293296in}{0.815128in}}%
\pgfpathlineto{\pgfqpoint{1.322566in}{0.789161in}}%
\pgfpathlineto{\pgfqpoint{1.351835in}{0.763705in}}%
\pgfpathlineto{\pgfqpoint{1.375251in}{0.743786in}}%
\pgfpathlineto{\pgfqpoint{1.398667in}{0.724343in}}%
\pgfpathlineto{\pgfqpoint{1.422082in}{0.705469in}}%
\pgfpathlineto{\pgfqpoint{1.439644in}{0.691756in}}%
\pgfpathlineto{\pgfqpoint{1.457206in}{0.678485in}}%
\pgfpathlineto{\pgfqpoint{1.474768in}{0.665725in}}%
\pgfpathlineto{\pgfqpoint{1.492330in}{0.653554in}}%
\pgfpathlineto{\pgfqpoint{1.504038in}{0.645812in}}%
\pgfpathlineto{\pgfqpoint{1.515745in}{0.638403in}}%
\pgfpathlineto{\pgfqpoint{1.527453in}{0.631358in}}%
\pgfpathlineto{\pgfqpoint{1.539161in}{0.624715in}}%
\pgfpathlineto{\pgfqpoint{1.550869in}{0.618510in}}%
\pgfpathlineto{\pgfqpoint{1.562577in}{0.612782in}}%
\pgfpathlineto{\pgfqpoint{1.574285in}{0.607573in}}%
\pgfpathlineto{\pgfqpoint{1.585993in}{0.602924in}}%
\pgfpathlineto{\pgfqpoint{1.597700in}{0.598875in}}%
\pgfpathlineto{\pgfqpoint{1.609408in}{0.595465in}}%
\pgfpathlineto{\pgfqpoint{1.621116in}{0.592729in}}%
\pgfpathlineto{\pgfqpoint{1.632824in}{0.590696in}}%
\pgfpathlineto{\pgfqpoint{1.644532in}{0.589391in}}%
\pgfpathlineto{\pgfqpoint{1.650386in}{0.589017in}}%
\pgfpathlineto{\pgfqpoint{1.656240in}{0.588829in}}%
\pgfpathlineto{\pgfqpoint{1.662094in}{0.588829in}}%
\pgfpathlineto{\pgfqpoint{1.667948in}{0.589017in}}%
\pgfpathlineto{\pgfqpoint{1.673801in}{0.589391in}}%
\pgfpathlineto{\pgfqpoint{1.685509in}{0.590696in}}%
\pgfpathlineto{\pgfqpoint{1.697217in}{0.592729in}}%
\pgfpathlineto{\pgfqpoint{1.708925in}{0.595465in}}%
\pgfpathlineto{\pgfqpoint{1.720633in}{0.598875in}}%
\pgfpathlineto{\pgfqpoint{1.732341in}{0.602924in}}%
\pgfpathlineto{\pgfqpoint{1.744049in}{0.607573in}}%
\pgfpathlineto{\pgfqpoint{1.755757in}{0.612782in}}%
\pgfpathlineto{\pgfqpoint{1.767464in}{0.618510in}}%
\pgfpathlineto{\pgfqpoint{1.779172in}{0.624715in}}%
\pgfpathlineto{\pgfqpoint{1.790880in}{0.631358in}}%
\pgfpathlineto{\pgfqpoint{1.802588in}{0.638403in}}%
\pgfpathlineto{\pgfqpoint{1.814296in}{0.645812in}}%
\pgfpathlineto{\pgfqpoint{1.826004in}{0.653554in}}%
\pgfpathlineto{\pgfqpoint{1.843565in}{0.665725in}}%
\pgfpathlineto{\pgfqpoint{1.861127in}{0.678485in}}%
\pgfpathlineto{\pgfqpoint{1.878689in}{0.691756in}}%
\pgfpathlineto{\pgfqpoint{1.896251in}{0.705469in}}%
\pgfpathlineto{\pgfqpoint{1.913813in}{0.719567in}}%
\pgfpathlineto{\pgfqpoint{1.937228in}{0.738877in}}%
\pgfpathlineto{\pgfqpoint{1.960644in}{0.758685in}}%
\pgfpathlineto{\pgfqpoint{1.989914in}{0.784026in}}%
\pgfpathlineto{\pgfqpoint{2.019183in}{0.809899in}}%
\pgfpathlineto{\pgfqpoint{2.054307in}{0.841515in}}%
\pgfpathlineto{\pgfqpoint{2.089431in}{0.873632in}}%
\pgfpathlineto{\pgfqpoint{2.130408in}{0.911607in}}%
\pgfpathlineto{\pgfqpoint{2.177239in}{0.955537in}}%
\pgfpathlineto{\pgfqpoint{2.229925in}{1.005483in}}%
\pgfpathlineto{\pgfqpoint{2.288464in}{1.061482in}}%
\pgfpathlineto{\pgfqpoint{2.358711in}{1.129212in}}%
\pgfpathlineto{\pgfqpoint{2.440666in}{1.208778in}}%
\pgfpathlineto{\pgfqpoint{2.534329in}{1.300239in}}%
\pgfpathlineto{\pgfqpoint{2.645554in}{1.409377in}}%
\pgfpathlineto{\pgfqpoint{2.780194in}{1.542033in}}%
\pgfpathlineto{\pgfqpoint{2.944104in}{1.704079in}}%
\pgfpathlineto{\pgfqpoint{3.109946in}{1.868458in}}%
\pgfpathlineto{\pgfqpoint{3.109946in}{1.868458in}}%
\pgfusepath{stroke}%
\end{pgfscope}%
\begin{pgfscope}%
\pgfpathrectangle{\pgfqpoint{0.198611in}{0.333208in}}{\pgfqpoint{2.921111in}{1.533583in}} %
\pgfusepath{clip}%
\pgfsetrectcap%
\pgfsetroundjoin%
\pgfsetlinewidth{0.200750pt}%
\definecolor{currentstroke}{rgb}{0.993248,0.906157,0.143936}%
\pgfsetstrokecolor{currentstroke}%
\pgfsetdash{}{0pt}%
\pgfpathmoveto{\pgfqpoint{0.243269in}{1.868458in}}%
\pgfpathlineto{\pgfqpoint{0.344959in}{1.770226in}}%
\pgfpathlineto{\pgfqpoint{0.438622in}{1.680228in}}%
\pgfpathlineto{\pgfqpoint{0.526431in}{1.596369in}}%
\pgfpathlineto{\pgfqpoint{0.602532in}{1.524182in}}%
\pgfpathlineto{\pgfqpoint{0.672779in}{1.458037in}}%
\pgfpathlineto{\pgfqpoint{0.737173in}{1.397901in}}%
\pgfpathlineto{\pgfqpoint{0.795712in}{1.343722in}}%
\pgfpathlineto{\pgfqpoint{0.848397in}{1.295434in}}%
\pgfpathlineto{\pgfqpoint{0.901083in}{1.247674in}}%
\pgfpathlineto{\pgfqpoint{0.947914in}{1.205740in}}%
\pgfpathlineto{\pgfqpoint{0.988892in}{1.169515in}}%
\pgfpathlineto{\pgfqpoint{1.029869in}{1.133795in}}%
\pgfpathlineto{\pgfqpoint{1.064993in}{1.103638in}}%
\pgfpathlineto{\pgfqpoint{1.100116in}{1.073966in}}%
\pgfpathlineto{\pgfqpoint{1.129386in}{1.049660in}}%
\pgfpathlineto{\pgfqpoint{1.158656in}{1.025782in}}%
\pgfpathlineto{\pgfqpoint{1.187925in}{1.002386in}}%
\pgfpathlineto{\pgfqpoint{1.217195in}{0.979530in}}%
\pgfpathlineto{\pgfqpoint{1.240611in}{0.961678in}}%
\pgfpathlineto{\pgfqpoint{1.264026in}{0.944253in}}%
\pgfpathlineto{\pgfqpoint{1.287442in}{0.927298in}}%
\pgfpathlineto{\pgfqpoint{1.310858in}{0.910860in}}%
\pgfpathlineto{\pgfqpoint{1.334274in}{0.894992in}}%
\pgfpathlineto{\pgfqpoint{1.351835in}{0.883498in}}%
\pgfpathlineto{\pgfqpoint{1.369397in}{0.872383in}}%
\pgfpathlineto{\pgfqpoint{1.386959in}{0.861674in}}%
\pgfpathlineto{\pgfqpoint{1.404521in}{0.851400in}}%
\pgfpathlineto{\pgfqpoint{1.422082in}{0.841593in}}%
\pgfpathlineto{\pgfqpoint{1.439644in}{0.832284in}}%
\pgfpathlineto{\pgfqpoint{1.457206in}{0.823506in}}%
\pgfpathlineto{\pgfqpoint{1.474768in}{0.815295in}}%
\pgfpathlineto{\pgfqpoint{1.486476in}{0.810153in}}%
\pgfpathlineto{\pgfqpoint{1.498184in}{0.805287in}}%
\pgfpathlineto{\pgfqpoint{1.509891in}{0.800710in}}%
\pgfpathlineto{\pgfqpoint{1.521599in}{0.796430in}}%
\pgfpathlineto{\pgfqpoint{1.533307in}{0.792458in}}%
\pgfpathlineto{\pgfqpoint{1.545015in}{0.788802in}}%
\pgfpathlineto{\pgfqpoint{1.556723in}{0.785474in}}%
\pgfpathlineto{\pgfqpoint{1.568431in}{0.782480in}}%
\pgfpathlineto{\pgfqpoint{1.580139in}{0.779829in}}%
\pgfpathlineto{\pgfqpoint{1.591846in}{0.777529in}}%
\pgfpathlineto{\pgfqpoint{1.603554in}{0.775586in}}%
\pgfpathlineto{\pgfqpoint{1.615262in}{0.774005in}}%
\pgfpathlineto{\pgfqpoint{1.626970in}{0.772792in}}%
\pgfpathlineto{\pgfqpoint{1.638678in}{0.771949in}}%
\pgfpathlineto{\pgfqpoint{1.650386in}{0.771481in}}%
\pgfpathlineto{\pgfqpoint{1.662094in}{0.771387in}}%
\pgfpathlineto{\pgfqpoint{1.673801in}{0.771668in}}%
\pgfpathlineto{\pgfqpoint{1.685509in}{0.772324in}}%
\pgfpathlineto{\pgfqpoint{1.697217in}{0.773352in}}%
\pgfpathlineto{\pgfqpoint{1.708925in}{0.774750in}}%
\pgfpathlineto{\pgfqpoint{1.720633in}{0.776512in}}%
\pgfpathlineto{\pgfqpoint{1.732341in}{0.778635in}}%
\pgfpathlineto{\pgfqpoint{1.744049in}{0.781111in}}%
\pgfpathlineto{\pgfqpoint{1.755757in}{0.783934in}}%
\pgfpathlineto{\pgfqpoint{1.767464in}{0.787097in}}%
\pgfpathlineto{\pgfqpoint{1.779172in}{0.790590in}}%
\pgfpathlineto{\pgfqpoint{1.790880in}{0.794405in}}%
\pgfpathlineto{\pgfqpoint{1.802588in}{0.798532in}}%
\pgfpathlineto{\pgfqpoint{1.814296in}{0.802962in}}%
\pgfpathlineto{\pgfqpoint{1.826004in}{0.807685in}}%
\pgfpathlineto{\pgfqpoint{1.837712in}{0.812690in}}%
\pgfpathlineto{\pgfqpoint{1.855273in}{0.820705in}}%
\pgfpathlineto{\pgfqpoint{1.872835in}{0.829297in}}%
\pgfpathlineto{\pgfqpoint{1.890397in}{0.838433in}}%
\pgfpathlineto{\pgfqpoint{1.907959in}{0.848078in}}%
\pgfpathlineto{\pgfqpoint{1.925520in}{0.858199in}}%
\pgfpathlineto{\pgfqpoint{1.943082in}{0.868767in}}%
\pgfpathlineto{\pgfqpoint{1.960644in}{0.879750in}}%
\pgfpathlineto{\pgfqpoint{1.978206in}{0.891120in}}%
\pgfpathlineto{\pgfqpoint{1.995768in}{0.902851in}}%
\pgfpathlineto{\pgfqpoint{2.019183in}{0.919011in}}%
\pgfpathlineto{\pgfqpoint{2.042599in}{0.935714in}}%
\pgfpathlineto{\pgfqpoint{2.066015in}{0.952910in}}%
\pgfpathlineto{\pgfqpoint{2.089431in}{0.970553in}}%
\pgfpathlineto{\pgfqpoint{2.112846in}{0.988603in}}%
\pgfpathlineto{\pgfqpoint{2.142116in}{1.011683in}}%
\pgfpathlineto{\pgfqpoint{2.171386in}{1.035279in}}%
\pgfpathlineto{\pgfqpoint{2.200655in}{1.059334in}}%
\pgfpathlineto{\pgfqpoint{2.235779in}{1.088737in}}%
\pgfpathlineto{\pgfqpoint{2.270902in}{1.118659in}}%
\pgfpathlineto{\pgfqpoint{2.306026in}{1.149037in}}%
\pgfpathlineto{\pgfqpoint{2.347003in}{1.184982in}}%
\pgfpathlineto{\pgfqpoint{2.387981in}{1.221403in}}%
\pgfpathlineto{\pgfqpoint{2.434812in}{1.263530in}}%
\pgfpathlineto{\pgfqpoint{2.487498in}{1.311476in}}%
\pgfpathlineto{\pgfqpoint{2.540183in}{1.359922in}}%
\pgfpathlineto{\pgfqpoint{2.598722in}{1.414250in}}%
\pgfpathlineto{\pgfqpoint{2.663116in}{1.474524in}}%
\pgfpathlineto{\pgfqpoint{2.733363in}{1.540795in}}%
\pgfpathlineto{\pgfqpoint{2.809464in}{1.613096in}}%
\pgfpathlineto{\pgfqpoint{2.897273in}{1.697063in}}%
\pgfpathlineto{\pgfqpoint{2.990936in}{1.787154in}}%
\pgfpathlineto{\pgfqpoint{3.075065in}{1.868458in}}%
\pgfpathlineto{\pgfqpoint{3.075065in}{1.868458in}}%
\pgfusepath{stroke}%
\end{pgfscope}%
\begin{pgfscope}%
\pgfpathrectangle{\pgfqpoint{0.198611in}{0.333208in}}{\pgfqpoint{2.921111in}{1.533583in}} %
\pgfusepath{clip}%
\pgfsetbuttcap%
\pgfsetroundjoin%
\pgfsetlinewidth{0.501875pt}%
\definecolor{currentstroke}{rgb}{0.501961,0.501961,0.501961}%
\pgfsetstrokecolor{currentstroke}%
\pgfsetdash{{1.850000pt}{0.800000pt}}{0.000000pt}%
\pgfpathmoveto{\pgfqpoint{1.584472in}{0.331542in}}%
\pgfpathlineto{\pgfqpoint{3.119722in}{1.866792in}}%
\pgfpathlineto{\pgfqpoint{3.119722in}{1.866792in}}%
\pgfusepath{stroke}%
\end{pgfscope}%
\begin{pgfscope}%
\pgfpathrectangle{\pgfqpoint{0.198611in}{0.333208in}}{\pgfqpoint{2.921111in}{1.533583in}} %
\pgfusepath{clip}%
\pgfsetbuttcap%
\pgfsetroundjoin%
\pgfsetlinewidth{0.501875pt}%
\definecolor{currentstroke}{rgb}{0.501961,0.501961,0.501961}%
\pgfsetstrokecolor{currentstroke}%
\pgfsetdash{{1.850000pt}{0.800000pt}}{0.000000pt}%
\pgfpathmoveto{\pgfqpoint{0.198611in}{1.866792in}}%
\pgfpathlineto{\pgfqpoint{1.733861in}{0.331542in}}%
\pgfpathlineto{\pgfqpoint{1.733861in}{0.331542in}}%
\pgfusepath{stroke}%
\end{pgfscope}%
\begin{pgfscope}%
\pgfsetrectcap%
\pgfsetmiterjoin%
\pgfsetlinewidth{0.501875pt}%
\definecolor{currentstroke}{rgb}{0.000000,0.000000,0.000000}%
\pgfsetstrokecolor{currentstroke}%
\pgfsetdash{}{0pt}%
\pgfpathmoveto{\pgfqpoint{1.659167in}{0.333208in}}%
\pgfpathlineto{\pgfqpoint{1.659167in}{1.866792in}}%
\pgfusepath{stroke}%
\end{pgfscope}%
\begin{pgfscope}%
\pgfsetrectcap%
\pgfsetmiterjoin%
\pgfsetlinewidth{0.501875pt}%
\definecolor{currentstroke}{rgb}{0.000000,0.000000,0.000000}%
\pgfsetstrokecolor{currentstroke}%
\pgfsetdash{}{0pt}%
\pgfpathmoveto{\pgfqpoint{0.198611in}{0.406236in}}%
\pgfpathlineto{\pgfqpoint{3.119722in}{0.406236in}}%
\pgfusepath{stroke}%
\end{pgfscope}%
\begin{pgfscope}%
\pgfsetroundcap%
\pgfsetroundjoin%
\pgfsetlinewidth{0.501875pt}%
\definecolor{currentstroke}{rgb}{0.000000,0.000000,0.000000}%
\pgfsetstrokecolor{currentstroke}%
\pgfsetdash{}{0pt}%
\pgfpathmoveto{\pgfqpoint{1.659167in}{1.872920in}}%
\pgfpathquadraticcurveto{\pgfqpoint{1.659167in}{1.873738in}}{\pgfqpoint{1.659167in}{1.866792in}}%
\pgfusepath{stroke}%
\end{pgfscope}%
\begin{pgfscope}%
\pgfsetroundcap%
\pgfsetroundjoin%
\pgfsetlinewidth{0.501875pt}%
\definecolor{currentstroke}{rgb}{0.000000,0.000000,0.000000}%
\pgfsetstrokecolor{currentstroke}%
\pgfsetdash{}{0pt}%
\pgfpathmoveto{\pgfqpoint{1.631389in}{1.817364in}}%
\pgfpathlineto{\pgfqpoint{1.659167in}{1.872920in}}%
\pgfpathlineto{\pgfqpoint{1.686944in}{1.817364in}}%
\pgfusepath{stroke}%
\end{pgfscope}%
\begin{pgfscope}%
\pgftext[x=1.659167in,y=1.936236in,,bottom]{\rmfamily\fontsize{10.000000}{12.000000}\selectfont \(\displaystyle \omega\)}%
\end{pgfscope}%
\begin{pgfscope}%
\pgfsetroundcap%
\pgfsetroundjoin%
\pgfsetlinewidth{0.501875pt}%
\definecolor{currentstroke}{rgb}{0.000000,0.000000,0.000000}%
\pgfsetstrokecolor{currentstroke}%
\pgfsetdash{}{0pt}%
\pgfpathmoveto{\pgfqpoint{3.125847in}{0.406236in}}%
\pgfpathquadraticcurveto{\pgfqpoint{3.126667in}{0.406236in}}{\pgfqpoint{3.119722in}{0.406236in}}%
\pgfusepath{stroke}%
\end{pgfscope}%
\begin{pgfscope}%
\pgfsetroundcap%
\pgfsetroundjoin%
\pgfsetlinewidth{0.501875pt}%
\definecolor{currentstroke}{rgb}{0.000000,0.000000,0.000000}%
\pgfsetstrokecolor{currentstroke}%
\pgfsetdash{}{0pt}%
\pgfpathmoveto{\pgfqpoint{3.070292in}{0.434014in}}%
\pgfpathlineto{\pgfqpoint{3.125847in}{0.406236in}}%
\pgfpathlineto{\pgfqpoint{3.070292in}{0.378458in}}%
\pgfusepath{stroke}%
\end{pgfscope}%
\begin{pgfscope}%
\pgftext[x=3.189167in,y=0.406236in,left,]{\rmfamily\fontsize{10.000000}{12.000000}\selectfont \(\displaystyle k\)}%
\end{pgfscope}%
\begin{pgfscope}%
\pgftext[x=2.298160in,y=0.953944in,left,base]{\rmfamily\fontsize{10.000000}{12.000000}\selectfont \(\displaystyle supp~(\hat\Delta_m(-\cdot))\)}%
\end{pgfscope}%
\begin{pgfscope}%
\pgftext[x=1.111458in,y=1.501653in,left,base]{\rmfamily\fontsize{10.000000}{12.000000}\selectfont \(\displaystyle \hat\Delta_m^{*2}(-\cdot)\)}%
\end{pgfscope}%
\begin{pgfscope}%
\pgfpathrectangle{\pgfqpoint{3.302292in}{0.435000in}}{\pgfqpoint{0.063333in}{1.330000in}} %
\pgfusepath{clip}%
\pgfsetbuttcap%
\pgfsetmiterjoin%
\definecolor{currentfill}{rgb}{1.000000,1.000000,1.000000}%
\pgfsetfillcolor{currentfill}%
\pgfsetlinewidth{0.010037pt}%
\definecolor{currentstroke}{rgb}{1.000000,1.000000,1.000000}%
\pgfsetstrokecolor{currentstroke}%
\pgfsetdash{}{0pt}%
\pgfpathmoveto{\pgfqpoint{3.302292in}{0.435000in}}%
\pgfpathlineto{\pgfqpoint{3.302292in}{0.439948in}}%
\pgfpathlineto{\pgfqpoint{3.302292in}{1.701667in}}%
\pgfpathlineto{\pgfqpoint{3.333958in}{1.765000in}}%
\pgfpathlineto{\pgfqpoint{3.333958in}{1.765000in}}%
\pgfpathlineto{\pgfqpoint{3.365625in}{1.701667in}}%
\pgfpathlineto{\pgfqpoint{3.365625in}{0.439948in}}%
\pgfpathlineto{\pgfqpoint{3.365625in}{0.435000in}}%
\pgfpathclose%
\pgfusepath{stroke,fill}%
\end{pgfscope}%
\begin{pgfscope}%
\pgfsys@transformshift{3.301667in}{0.435000in}%
\pgftext[left,bottom]{\pgfimage[interpolate=true,width=0.063333in,height=1.330000in]{delta_m2-img1.png}}%
\end{pgfscope}%
\begin{pgfscope}%
\pgfsetbuttcap%
\pgfsetroundjoin%
\definecolor{currentfill}{rgb}{0.000000,0.000000,0.000000}%
\pgfsetfillcolor{currentfill}%
\pgfsetlinewidth{0.803000pt}%
\definecolor{currentstroke}{rgb}{0.000000,0.000000,0.000000}%
\pgfsetstrokecolor{currentstroke}%
\pgfsetdash{}{0pt}%
\pgfsys@defobject{currentmarker}{\pgfqpoint{0.000000in}{0.000000in}}{\pgfqpoint{0.048611in}{0.000000in}}{%
\pgfpathmoveto{\pgfqpoint{0.000000in}{0.000000in}}%
\pgfpathlineto{\pgfqpoint{0.048611in}{0.000000in}}%
\pgfusepath{stroke,fill}%
}%
\begin{pgfscope}%
\pgfsys@transformshift{3.365625in}{0.435000in}%
\pgfsys@useobject{currentmarker}{}%
\end{pgfscope}%
\end{pgfscope}%
\begin{pgfscope}%
\pgftext[x=3.462847in,y=0.387172in,left,base]{\rmfamily\fontsize{10.000000}{12.000000}\selectfont \(\displaystyle 1.5\)}%
\end{pgfscope}%
\begin{pgfscope}%
\pgfsetbuttcap%
\pgfsetroundjoin%
\definecolor{currentfill}{rgb}{0.000000,0.000000,0.000000}%
\pgfsetfillcolor{currentfill}%
\pgfsetlinewidth{0.803000pt}%
\definecolor{currentstroke}{rgb}{0.000000,0.000000,0.000000}%
\pgfsetstrokecolor{currentstroke}%
\pgfsetdash{}{0pt}%
\pgfsys@defobject{currentmarker}{\pgfqpoint{0.000000in}{0.000000in}}{\pgfqpoint{0.048611in}{0.000000in}}{%
\pgfpathmoveto{\pgfqpoint{0.000000in}{0.000000in}}%
\pgfpathlineto{\pgfqpoint{0.048611in}{0.000000in}}%
\pgfusepath{stroke,fill}%
}%
\begin{pgfscope}%
\pgfsys@transformshift{3.365625in}{0.688333in}%
\pgfsys@useobject{currentmarker}{}%
\end{pgfscope}%
\end{pgfscope}%
\begin{pgfscope}%
\pgftext[x=3.462847in,y=0.640506in,left,base]{\rmfamily\fontsize{10.000000}{12.000000}\selectfont \(\displaystyle 2.0\)}%
\end{pgfscope}%
\begin{pgfscope}%
\pgfsetbuttcap%
\pgfsetroundjoin%
\definecolor{currentfill}{rgb}{0.000000,0.000000,0.000000}%
\pgfsetfillcolor{currentfill}%
\pgfsetlinewidth{0.803000pt}%
\definecolor{currentstroke}{rgb}{0.000000,0.000000,0.000000}%
\pgfsetstrokecolor{currentstroke}%
\pgfsetdash{}{0pt}%
\pgfsys@defobject{currentmarker}{\pgfqpoint{0.000000in}{0.000000in}}{\pgfqpoint{0.048611in}{0.000000in}}{%
\pgfpathmoveto{\pgfqpoint{0.000000in}{0.000000in}}%
\pgfpathlineto{\pgfqpoint{0.048611in}{0.000000in}}%
\pgfusepath{stroke,fill}%
}%
\begin{pgfscope}%
\pgfsys@transformshift{3.365625in}{0.941667in}%
\pgfsys@useobject{currentmarker}{}%
\end{pgfscope}%
\end{pgfscope}%
\begin{pgfscope}%
\pgftext[x=3.462847in,y=0.893839in,left,base]{\rmfamily\fontsize{10.000000}{12.000000}\selectfont \(\displaystyle 2.5\)}%
\end{pgfscope}%
\begin{pgfscope}%
\pgfsetbuttcap%
\pgfsetroundjoin%
\definecolor{currentfill}{rgb}{0.000000,0.000000,0.000000}%
\pgfsetfillcolor{currentfill}%
\pgfsetlinewidth{0.803000pt}%
\definecolor{currentstroke}{rgb}{0.000000,0.000000,0.000000}%
\pgfsetstrokecolor{currentstroke}%
\pgfsetdash{}{0pt}%
\pgfsys@defobject{currentmarker}{\pgfqpoint{0.000000in}{0.000000in}}{\pgfqpoint{0.048611in}{0.000000in}}{%
\pgfpathmoveto{\pgfqpoint{0.000000in}{0.000000in}}%
\pgfpathlineto{\pgfqpoint{0.048611in}{0.000000in}}%
\pgfusepath{stroke,fill}%
}%
\begin{pgfscope}%
\pgfsys@transformshift{3.365625in}{1.195000in}%
\pgfsys@useobject{currentmarker}{}%
\end{pgfscope}%
\end{pgfscope}%
\begin{pgfscope}%
\pgftext[x=3.462847in,y=1.147172in,left,base]{\rmfamily\fontsize{10.000000}{12.000000}\selectfont \(\displaystyle 3.0\)}%
\end{pgfscope}%
\begin{pgfscope}%
\pgfsetbuttcap%
\pgfsetroundjoin%
\definecolor{currentfill}{rgb}{0.000000,0.000000,0.000000}%
\pgfsetfillcolor{currentfill}%
\pgfsetlinewidth{0.803000pt}%
\definecolor{currentstroke}{rgb}{0.000000,0.000000,0.000000}%
\pgfsetstrokecolor{currentstroke}%
\pgfsetdash{}{0pt}%
\pgfsys@defobject{currentmarker}{\pgfqpoint{0.000000in}{0.000000in}}{\pgfqpoint{0.048611in}{0.000000in}}{%
\pgfpathmoveto{\pgfqpoint{0.000000in}{0.000000in}}%
\pgfpathlineto{\pgfqpoint{0.048611in}{0.000000in}}%
\pgfusepath{stroke,fill}%
}%
\begin{pgfscope}%
\pgfsys@transformshift{3.365625in}{1.448333in}%
\pgfsys@useobject{currentmarker}{}%
\end{pgfscope}%
\end{pgfscope}%
\begin{pgfscope}%
\pgftext[x=3.462847in,y=1.400506in,left,base]{\rmfamily\fontsize{10.000000}{12.000000}\selectfont \(\displaystyle 3.5\)}%
\end{pgfscope}%
\begin{pgfscope}%
\pgfsetbuttcap%
\pgfsetroundjoin%
\definecolor{currentfill}{rgb}{0.000000,0.000000,0.000000}%
\pgfsetfillcolor{currentfill}%
\pgfsetlinewidth{0.803000pt}%
\definecolor{currentstroke}{rgb}{0.000000,0.000000,0.000000}%
\pgfsetstrokecolor{currentstroke}%
\pgfsetdash{}{0pt}%
\pgfsys@defobject{currentmarker}{\pgfqpoint{0.000000in}{0.000000in}}{\pgfqpoint{0.048611in}{0.000000in}}{%
\pgfpathmoveto{\pgfqpoint{0.000000in}{0.000000in}}%
\pgfpathlineto{\pgfqpoint{0.048611in}{0.000000in}}%
\pgfusepath{stroke,fill}%
}%
\begin{pgfscope}%
\pgfsys@transformshift{3.365625in}{1.701667in}%
\pgfsys@useobject{currentmarker}{}%
\end{pgfscope}%
\end{pgfscope}%
\begin{pgfscope}%
\pgftext[x=3.462847in,y=1.653839in,left,base]{\rmfamily\fontsize{10.000000}{12.000000}\selectfont \(\displaystyle 4.0\)}%
\end{pgfscope}%
\begin{pgfscope}%
\pgfsetbuttcap%
\pgfsetmiterjoin%
\pgfsetlinewidth{0.501875pt}%
\definecolor{currentstroke}{rgb}{0.000000,0.000000,0.000000}%
\pgfsetstrokecolor{currentstroke}%
\pgfsetdash{}{0pt}%
\pgfpathmoveto{\pgfqpoint{3.302292in}{0.435000in}}%
\pgfpathlineto{\pgfqpoint{3.302292in}{0.439948in}}%
\pgfpathlineto{\pgfqpoint{3.302292in}{1.701667in}}%
\pgfpathlineto{\pgfqpoint{3.333958in}{1.765000in}}%
\pgfpathlineto{\pgfqpoint{3.333958in}{1.765000in}}%
\pgfpathlineto{\pgfqpoint{3.365625in}{1.701667in}}%
\pgfpathlineto{\pgfqpoint{3.365625in}{0.439948in}}%
\pgfpathlineto{\pgfqpoint{3.365625in}{0.435000in}}%
\pgfpathclose%
\pgfusepath{stroke}%
\end{pgfscope}%
\end{pgfpicture}%
\makeatother%
\endgroup%
} %
        \caption{Plot von $\hat{\Delta}_m^{*2}$ und $\hat{\Delta}_m$.
        Je weiter wir uns von der 2m-Massenschale wegbewegen, desto konstanter
        wird $\hat{\Delta_m}^{*2}$ und ist singulär genau auf der $2m$-Massenschale}
        \label{fig:delta_2m}
    \end{minipage}\hfill
    \begin{minipage}{0.45\textwidth}
        \centering
        \resizebox{\textwidth}{!}{%% Creator: Matplotlib, PGF backend
%%
%% To include the figure in your LaTeX document, write
%%   \input{<filename>.pgf}
%%
%% Make sure the required packages are loaded in your preamble
%%   \usepackage{pgf}
%%
%% Figures using additional raster images can only be included by \input if
%% they are in the same directory as the main LaTeX file. For loading figures
%% from other directories you can use the `import` package
%%   \usepackage{import}
%% and then include the figures with
%%   \import{<path to file>}{<filename>.pgf}
%%
%% Matplotlib used the following preamble
%%   \usepackage[utf8x]{inputenc}
%%   \usepackage[T1]{fontenc}
%%   \usepackage{amssymb}
%%
\begingroup%
\makeatletter%
\begin{pgfpicture}%
\pgfpathrectangle{\pgfpointorigin}{\pgfqpoint{4.000000in}{2.200000in}}%
\pgfusepath{use as bounding box, clip}%
\begin{pgfscope}%
\pgfsetbuttcap%
\pgfsetmiterjoin%
\definecolor{currentfill}{rgb}{1.000000,1.000000,1.000000}%
\pgfsetfillcolor{currentfill}%
\pgfsetlinewidth{0.000000pt}%
\definecolor{currentstroke}{rgb}{1.000000,1.000000,1.000000}%
\pgfsetstrokecolor{currentstroke}%
\pgfsetdash{}{0pt}%
\pgfpathmoveto{\pgfqpoint{0.000000in}{0.000000in}}%
\pgfpathlineto{\pgfqpoint{4.000000in}{0.000000in}}%
\pgfpathlineto{\pgfqpoint{4.000000in}{2.200000in}}%
\pgfpathlineto{\pgfqpoint{0.000000in}{2.200000in}}%
\pgfpathclose%
\pgfusepath{fill}%
\end{pgfscope}%
\begin{pgfscope}%
\pgfsetbuttcap%
\pgfsetmiterjoin%
\definecolor{currentfill}{rgb}{1.000000,1.000000,1.000000}%
\pgfsetfillcolor{currentfill}%
\pgfsetlinewidth{0.000000pt}%
\definecolor{currentstroke}{rgb}{0.000000,0.000000,0.000000}%
\pgfsetstrokecolor{currentstroke}%
\pgfsetstrokeopacity{0.000000}%
\pgfsetdash{}{0pt}%
\pgfpathmoveto{\pgfqpoint{0.198611in}{0.198611in}}%
\pgfpathlineto{\pgfqpoint{3.801389in}{0.198611in}}%
\pgfpathlineto{\pgfqpoint{3.801389in}{2.001389in}}%
\pgfpathlineto{\pgfqpoint{0.198611in}{2.001389in}}%
\pgfpathclose%
\pgfusepath{fill}%
\end{pgfscope}%
\begin{pgfscope}%
\pgfpathrectangle{\pgfqpoint{0.198611in}{0.198611in}}{\pgfqpoint{3.602778in}{1.802778in}} %
\pgfusepath{clip}%
\pgfsetbuttcap%
\pgfsetroundjoin%
\pgfsetlinewidth{0.501875pt}%
\definecolor{currentstroke}{rgb}{0.501961,0.501961,0.501961}%
\pgfsetstrokecolor{currentstroke}%
\pgfsetdash{{1.850000pt}{0.800000pt}}{0.000000pt}%
\pgfpathmoveto{\pgfqpoint{0.767471in}{0.184722in}}%
\pgfpathlineto{\pgfqpoint{0.767471in}{2.001389in}}%
\pgfusepath{stroke}%
\end{pgfscope}%
\begin{pgfscope}%
\pgfpathrectangle{\pgfqpoint{0.198611in}{0.198611in}}{\pgfqpoint{3.602778in}{1.802778in}} %
\pgfusepath{clip}%
\pgfsetrectcap%
\pgfsetroundjoin%
\pgfsetlinewidth{1.003750pt}%
\definecolor{currentstroke}{rgb}{0.894118,0.101961,0.109804}%
\pgfsetstrokecolor{currentstroke}%
\pgfsetdash{}{0pt}%
\pgfpathmoveto{\pgfqpoint{0.770354in}{2.015278in}}%
\pgfpathlineto{\pgfqpoint{0.776582in}{1.099358in}}%
\pgfpathlineto{\pgfqpoint{0.784174in}{0.905618in}}%
\pgfpathlineto{\pgfqpoint{0.791766in}{0.815405in}}%
\pgfpathlineto{\pgfqpoint{0.799359in}{0.761699in}}%
\pgfpathlineto{\pgfqpoint{0.806951in}{0.725675in}}%
\pgfpathlineto{\pgfqpoint{0.814544in}{0.699731in}}%
\pgfpathlineto{\pgfqpoint{0.822136in}{0.680141in}}%
\pgfpathlineto{\pgfqpoint{0.837321in}{0.652591in}}%
\pgfpathlineto{\pgfqpoint{0.852506in}{0.634292in}}%
\pgfpathlineto{\pgfqpoint{0.867690in}{0.621422in}}%
\pgfpathlineto{\pgfqpoint{0.882875in}{0.612017in}}%
\pgfpathlineto{\pgfqpoint{0.905652in}{0.602089in}}%
\pgfpathlineto{\pgfqpoint{0.928429in}{0.595392in}}%
\pgfpathlineto{\pgfqpoint{0.958799in}{0.589572in}}%
\pgfpathlineto{\pgfqpoint{0.996761in}{0.585335in}}%
\pgfpathlineto{\pgfqpoint{1.049908in}{0.582586in}}%
\pgfpathlineto{\pgfqpoint{1.125831in}{0.581888in}}%
\pgfpathlineto{\pgfqpoint{1.254902in}{0.584131in}}%
\pgfpathlineto{\pgfqpoint{2.112842in}{0.604087in}}%
\pgfpathlineto{\pgfqpoint{2.606347in}{0.610424in}}%
\pgfpathlineto{\pgfqpoint{3.266885in}{0.615635in}}%
\pgfpathlineto{\pgfqpoint{3.815278in}{0.618380in}}%
\pgfpathlineto{\pgfqpoint{3.815278in}{0.618380in}}%
\pgfusepath{stroke}%
\end{pgfscope}%
\begin{pgfscope}%
\pgfpathrectangle{\pgfqpoint{0.198611in}{0.198611in}}{\pgfqpoint{3.602778in}{1.802778in}} %
\pgfusepath{clip}%
\pgfsetrectcap%
\pgfsetroundjoin%
\pgfsetlinewidth{1.003750pt}%
\definecolor{currentstroke}{rgb}{0.894118,0.101961,0.109804}%
\pgfsetstrokecolor{currentstroke}%
\pgfsetdash{}{0pt}%
\pgfpathmoveto{\pgfqpoint{0.184722in}{0.284458in}}%
\pgfpathlineto{\pgfqpoint{0.430369in}{0.284458in}}%
\pgfpathlineto{\pgfqpoint{0.767471in}{0.284458in}}%
\pgfusepath{stroke}%
\end{pgfscope}%
\begin{pgfscope}%
\pgfpathrectangle{\pgfqpoint{0.198611in}{0.198611in}}{\pgfqpoint{3.602778in}{1.802778in}} %
\pgfusepath{clip}%
\pgfsetbuttcap%
\pgfsetroundjoin%
\pgfsetlinewidth{0.501875pt}%
\definecolor{currentstroke}{rgb}{0.501961,0.501961,0.501961}%
\pgfsetstrokecolor{currentstroke}%
\pgfsetdash{{1.850000pt}{0.800000pt}}{0.000000pt}%
\pgfpathmoveto{\pgfqpoint{0.184722in}{0.627844in}}%
\pgfpathlineto{\pgfqpoint{3.815278in}{0.627844in}}%
\pgfusepath{stroke}%
\end{pgfscope}%
\begin{pgfscope}%
\pgfsetrectcap%
\pgfsetmiterjoin%
\pgfsetlinewidth{0.501875pt}%
\definecolor{currentstroke}{rgb}{0.000000,0.000000,0.000000}%
\pgfsetstrokecolor{currentstroke}%
\pgfsetdash{}{0pt}%
\pgfpathmoveto{\pgfqpoint{0.388231in}{0.198611in}}%
\pgfpathlineto{\pgfqpoint{0.388231in}{2.001389in}}%
\pgfusepath{stroke}%
\end{pgfscope}%
\begin{pgfscope}%
\pgfsetrectcap%
\pgfsetmiterjoin%
\pgfsetlinewidth{0.501875pt}%
\definecolor{currentstroke}{rgb}{0.000000,0.000000,0.000000}%
\pgfsetstrokecolor{currentstroke}%
\pgfsetdash{}{0pt}%
\pgfpathmoveto{\pgfqpoint{0.198611in}{0.284458in}}%
\pgfpathlineto{\pgfqpoint{3.801389in}{0.284458in}}%
\pgfusepath{stroke}%
\end{pgfscope}%
\begin{pgfscope}%
\pgftext[x=0.843319in,y=0.335966in,left,base]{\rmfamily\fontsize{10.000000}{12.000000}\selectfont \(\displaystyle \omega = 2 m\)}%
\end{pgfscope}%
\begin{pgfscope}%
\pgftext[x=0.160687in,y=0.645013in,left,base]{\rmfamily\fontsize{10.000000}{12.000000}\selectfont \(\displaystyle \hat \Delta^{*2} = 2\)}%
\end{pgfscope}%
\begin{pgfscope}%
\pgftext[x=0.843319in,y=0.971230in,left,base]{\rmfamily\fontsize{10.000000}{12.000000}\selectfont \(\displaystyle \approx \frac{1}{\sqrt{\omega}}\)}%
\end{pgfscope}%
\begin{pgfscope}%
\pgftext[x=3.042909in,y=0.645013in,left,base]{\rmfamily\fontsize{10.000000}{12.000000}\selectfont \(\displaystyle \approx 2\)}%
\end{pgfscope}%
\begin{pgfscope}%
\pgfsetroundcap%
\pgfsetroundjoin%
\pgfsetlinewidth{0.501875pt}%
\definecolor{currentstroke}{rgb}{0.000000,0.000000,0.000000}%
\pgfsetstrokecolor{currentstroke}%
\pgfsetdash{}{0pt}%
\pgfpathmoveto{\pgfqpoint{0.388231in}{2.007506in}}%
\pgfpathquadraticcurveto{\pgfqpoint{0.388231in}{2.008330in}}{\pgfqpoint{0.388231in}{2.001389in}}%
\pgfusepath{stroke}%
\end{pgfscope}%
\begin{pgfscope}%
\pgfsetroundcap%
\pgfsetroundjoin%
\pgfsetlinewidth{0.501875pt}%
\definecolor{currentstroke}{rgb}{0.000000,0.000000,0.000000}%
\pgfsetstrokecolor{currentstroke}%
\pgfsetdash{}{0pt}%
\pgfpathmoveto{\pgfqpoint{0.360453in}{1.951951in}}%
\pgfpathlineto{\pgfqpoint{0.388231in}{2.007506in}}%
\pgfpathlineto{\pgfqpoint{0.416009in}{1.951951in}}%
\pgfusepath{stroke}%
\end{pgfscope}%
\begin{pgfscope}%
\pgftext[x=0.388231in,y=2.070833in,,bottom]{\rmfamily\fontsize{10.000000}{12.000000}\selectfont \(\displaystyle \hat\Delta^{*2} ~(\omega, 0)\)}%
\end{pgfscope}%
\begin{pgfscope}%
\pgfsetroundcap%
\pgfsetroundjoin%
\pgfsetlinewidth{0.501875pt}%
\definecolor{currentstroke}{rgb}{0.000000,0.000000,0.000000}%
\pgfsetstrokecolor{currentstroke}%
\pgfsetdash{}{0pt}%
\pgfpathmoveto{\pgfqpoint{3.807500in}{0.284458in}}%
\pgfpathquadraticcurveto{\pgfqpoint{3.808327in}{0.284458in}}{\pgfqpoint{3.801389in}{0.284458in}}%
\pgfusepath{stroke}%
\end{pgfscope}%
\begin{pgfscope}%
\pgfsetroundcap%
\pgfsetroundjoin%
\pgfsetlinewidth{0.501875pt}%
\definecolor{currentstroke}{rgb}{0.000000,0.000000,0.000000}%
\pgfsetstrokecolor{currentstroke}%
\pgfsetdash{}{0pt}%
\pgfpathmoveto{\pgfqpoint{3.751945in}{0.312235in}}%
\pgfpathlineto{\pgfqpoint{3.807500in}{0.284458in}}%
\pgfpathlineto{\pgfqpoint{3.751945in}{0.256680in}}%
\pgfusepath{stroke}%
\end{pgfscope}%
\begin{pgfscope}%
\pgftext[x=3.870833in,y=0.284458in,left,]{\rmfamily\fontsize{10.000000}{12.000000}\selectfont \(\displaystyle \omega\)}%
\end{pgfscope}%
\end{pgfpicture}%
\makeatother%
\endgroup%
}
        \caption{Plot von $\left.\hat{\Delta_m}^{*2}\right|_{k=0}$ um das asymptotische Verhalten für $\omega \rightarrow 0$ und $\omega \rightarrow \infty$ zu verdeutlichen}
        \label{fig:delta_2m_k0}
    \end{minipage}
\end{figure}


\todo[color=green]{$supp (\psi_{ast})$ einzeichnen oder nicht in \cref{fig:delta_2m}?}


\todo{spacing der figures anpassen}

\subsection{\dots und nun zur Wellenfrontmenge} % (fold)
\label{sec:dots_und_nun_zur_wellenfrontmenge}

Mit diesem Ausdruck für $\rwhat{\Delta}_m^{*2}$ können wir uns nun der Wellenfrontmenge widmen.

\subsubsection*{\texorpdfstring{Fall $|s|>1$}{Fall s>1}}
Genau wie im Fall $s \neq 1$ bei der massiven Zweipunktfunktion (vgl. \cref{sec:die_wellenfrontmenge_von_delta_m}) ist hier nichts zu tun, da für $a$ klein genug wieder

\begin{align}
    supp (\hat\psi_{ast}) \cap supp (\rwhat{\Delta}_m^{*2}) = \varnothing
    ~~\Longrightarrow~~
    \left\langle\hat\psi_{ast}, \rwhat{\Delta}_m^{*2}\right\rangle = 0
\label{eq:delta_m2_s>1}
\end{align}

gilt.

\subsubsection*{\texorpdfstring{Fall $s|<|1$}{Fall s|<|1}}
Hier bedienen wir uns direkt bei \ref{Eq:substitution2} und schreiben

\begin{align}
    & \kern -.5em
    \left\langle \rwhat{\Delta}_m^{*2}, \rwhat{\psi}_{ast} \right\rangle
    \label{eq:psi_ast_delta_m2_s<1}
    \\ &=
    2 a^{-\frac{3}{4}} \int \frac{
    \hat\psi_1(\omega)~\hat\psi_2(k) \left(
        \omega^2 a^{-2} - \omega^2\left(a^{-\frac{1}{2}}k+sa^{-1}\right)^2
            -3m^2
            \right)
    }
    {
        \sqrt{\omega^2 a^{-2}-\omega^2\left(a^{-\frac{1}{2}} k +s^{-1}\right)^2}
        \sqrt{\omega^2 a^{-2}-\omega^2\left(
            a^{-\frac{1}{2}} k +sa^{-1}\right)^2
            -4m^2
             }
    }
    \nonumber \\ & \kern 2em\cdot
    \Theta \left(\omega^2-k^2-4m^2\right)
      e^{-i \omega \left(\frac{t'-sx'}{a}+k \frac{x'}{\sqrt{a}}\right)}
    \omega \d \omega \d k
%    =
    \nonumber \\ &\kern -2em \underset{\Delta s := 1-s^2 > 0}{=} \kern -1.5em
     2 a^{-\frac{3}{4}} \int \frac{
        \hat\psi_1(\omega)~ \hat\psi_2(k) \cancel{a^{-2}} \left(
        \omega^2 \left(\Delta s - 2 a^{\frac{1}{2}} k s - ak^2
                \right) - 3a^2m^2
        \right)
     e^{\dots} \Theta(\dots) \cancel{\omega}
     }
     {
        \cancel{\omega} \cancel{a^{-2}}
        \sqrt{\Delta s -2a^{\frac{1}{2}}ks - ak^2}
            \sqrt{\Delta s \omega^2 -2a^{\frac{1}{2}} \omega^2 k s
                    - a\omega^2k^2-4 a^2 m^2}
     }
     \d \omega \d k
\end{align}
% \begin{dmath}
%     \left\langle \rwhat{\psi}_{ast},\rwhat{\Delta}_m^{*2}\right\rangle
%     =
%     2 a^{-\frac{3}{4}} \int \frac{
%     \hat\psi_1(\omega)~\hat\psi_2(k) \left(
%         \omega^2 a^{-2} - \omega^2\left(a^{-\frac{1}{2}}k+sa^{-1}\right)^2
%             -3m^2
%             \right)
%     }
%     {
%         \sqrt{\omega^2 a^{-2}-\omega^2\left(a^{-\frac{1}{2}} k +s^{-1}\right)^2}
%         \sqrt{\omega^2 a^{-2}-\omega^2\left(
%             a^{-\frac{1}{2}} k +sa^{-1}\right)^2
%             -4m^2
%              }
%     }
%     \cdot
%     \Theta \left(\omega^2-k^2-4m^2\right)
%       e^{-i \omega \left(\frac{t'-sx'}{a}+k \frac{x'}{\sqrt{a}}\right)}
%     \omega \d \omega \d k
% %    =
%     \kern -4em \underset{\Delta s := 1-s^2 > 0}{=} \kern -1.5em
%      2 a^{-\frac{3}{4}} \int \frac{
%         \hat\psi_1(\omega)~ \hat\psi_2(k) \cancel{a^{-2}} \left(
%         \omega^2 \left(\Delta s - 2 a^{\frac{1}{2}} k s - ak^2
%                 \right) - 3a^2m^2
%         \right)
%      e^{\dots} \Theta(\dots) \cancel{\omega}
%      }
%      {
%         \cancel{\omega} \cancel{a^{-2}}
%         \sqrt{\Delta s -2a^{\frac{1}{2}}ks - ak^2}
%             \sqrt{\Delta s \omega^2 -2a^{\frac{1}{2}} \omega^2 k s
%                     - a\omega^2k^2-4 a^2 m^2}
%      }
%      \d \omega \d k
% \label{eq:psi_ast_delta_m2_s<1}
% \end{dmath}

Für hinreichend kleine $a$ können wir den Integranden nun majorisieren

\begin{dmath*}
    \left|
    2 \frac{
        \hat\psi_1(\omega) ~\hat\psi_2(k) \omega^2 \Delta s \Theta(\dots)}
    {
        \sqrt{\Delta s} \sqrt{\Delta s \omega^2}
    }
    \right|
    \geq
    \left|
    \frac{
        \hat\psi_1(\omega)~ \hat\psi_2(k) \left(
        \omega^2 \left(\Delta s - 2 a^{\frac{1}{2}} k s - ak^2
                \right) - 3a^2m^2
        \right)
         \Theta(\dots)
     }
     {
        \sqrt{\Delta s -2a^{\frac{1}{2}}ks - ak^2}
            \sqrt{\Delta s \omega^2 -2a^{\frac{1}{2}} \omega^2 k s
                    - a\omega^2k^2-4 a^2 m^2}
     }
    \right|
\end{dmath*}

und dürfen also Lebesgue verwenden und schreiben

\begin{align*}
    \lim_{a \rightarrow 0} \int \dots ~\d \omega \d k
    &=
    \int \lim_{a \rightarrow 0} \dots ~\d \omega \d k
    \\ &=
    2 a^{-\frac{3}{4}} \int
    \frac{\hat\psi_1(\omega) ~\hat\psi_2(k) ~\omega^{\cancel{2}}
        ~\cancel{\Delta s} ~\Theta(\dots)
    }
    {
        \cancel{\sqrt{\Delta s}}\cancel{\sqrt{\Delta s}}\cancel{\omega}
    }
    e^{-i \omega \left(\frac{t'-sx'}{a} + k \frac{x'}{\sqrt{a}}\right)}
    \d \omega \d k
    \\ & \kern -.6em\underset{k\rightarrow \frac{k}{\omega}}{=}
    2 a^{-\frac{3}{4}} \int
    \hat\psi_1(\omega) ~\hat\psi_2\left(\tfrac{k}{\omega}\right)
    e^{-i\omega \frac{t'-sx'}{a} + ik \frac{x'}{\sqrt{a}}}
    \d \omega \d k \\
    \\ &=
    2 a^{-\frac{3}{4}} \psi \left(\frac{t'-sx'}{a}, \frac{x'}{a}\right)
\end{align*}
% \begin{dmath*}
%     \lim_{a \rightarrow 0} \int \dots ~\d \omega \d k
%     = \int \lim_{a \rightarrow 0} \dots ~\d \omega \d k
%     = 2 a^{-\frac{3}{4}} \int
%     \frac{\hat\psi_1(\omega) ~\hat\psi_2(k) ~\omega^{\cancel{2}}
%         ~\cancel{\Delta s} ~\Theta(\dots)
%     }
%     {
%         \cancel{\sqrt{\Delta s}}\cancel{\sqrt{\Delta s}}\cancel{\omega}
%     }
%     e^{-i \omega \left(\frac{t'-sx'}{a} + k \frac{x'}{\sqrt{a}}\right)}
%     \d \omega \d k
%     \underset{k\rightarrow \frac{k}{\omega}}{=}
%     2 a^{-\frac{3}{4}} \int
%     \hat\psi_1(\omega) ~\hat\psi_2\left(\tfrac{k}{\omega}\right)
%     e^{-i\omega \frac{t'-sx'}{a} + ik \frac{x'}{\sqrt{a}}}
%     \d \omega \d k \\
%     =
%     2 a^{-\frac{3}{4}} \psi \left(\frac{t'-sx'}{a}, \frac{x'}{a}\right)
% \end{dmath*}

Und da Shearlets nach \cref{prop:shearlets_decay_rapidly} schnell abfallen erhalten wir schließlich

\begin{dmath}
    \left\langle \Delta_m^{2}, \psi_{ast}\right\rangle
    = 2 a^{-\frac{3}{4}} \psi \left(\frac{t'-sx'}{a}, \frac{x'}{a}\right)
    \sim O(a^k) ~\forall k \hiderel \in \mathbb{N} \condition{falls $(t',x') \neq 0$}
    \sim O(a^{-\frac{3}{4}}) \condition{falls $(t',x') = 0$}
\label{eq:delta_m2_s<1}
\end{dmath}

\subsubsection*{\texorpdfstring{Fall $s = -1$}{Fall s = 1}}
% Bevor die folgenden Formelmonster allzu abschreckend werden, erklären wir hier kurz die Strategie. Für $s=1$ liegt der Träger des Shearlets ja genau auf der Diagonalen (vgl. Abb. \ref{fig:supp_psi_hat}) und damit schließich auf der Singularität von $\rwhat{\Delta_m}^{*2}$ welche sich asymptotisch an die Diagonale schmiegt. Die Strategie wird nun sein, durch Variablentransformation den Integrationsbereich so zu verzerren, dass die

\begin{align}
\label{eq:psi_ast_delta_m2}
    &\kern -1em
    \left\langle \rwhat{\Delta}_m^{*2}, \hat\psi_{ast} \right\rangle
    \\ &=
    2 a^{-\frac{3}{4}} \int
    \frac{
        \hat\psi_1(\omega)\,\hat\psi_2(k)
        \left(\omega^2\left(
            a^{-2}\cancel{(1-s^2)}-2a^{-\frac{3}{2}} k s -a^{-1}k^2
            \right)
            -3m^2
        \right)
    }
    {
        \sqrt{
            \omega^2\left(a^{-2}\cancel{(1-s^2)}-2a^{-\frac{3}{2}} k s -a^{-1}k^2\right)
        }
        \sqrt{
            \omega^2\left(a^{-2}\cancel{(1-s^2)}-2a^{-\frac{3}{2}}k s  -a^{-1}k^2\right) - 4m^2
        }
    }
    \nonumber \\ & \kern 2em \cdot
    \Theta\left(
            \omega^2\left(a^{-2}\cancel{(1-s^2)}-2a^{-\frac{3}{2}} -a^{-1}k^2\right) - 4m^2
        \right)
    e^{-i\omega\left(\frac{t'-sx'}{a}+k \frac{x'}{\sqrt a}\right)}
    \cdot
    \omega \d \omega \d k
    \nonumber \\ &=
    2 a^{-\frac{3}{4}} \int
    \frac{
        \hat\psi_1(\omega)\,\hat\psi_2(k)\,\cancel{a^{-\frac{3}{2}}}
        \left(
            2 \omega^2k-a^{\frac{1}{2}}\omega^2k^2-a^{\frac{3}{2}}3m^2
        \right)
    }
    {
        \cancel{a^{-\frac{3}{2}}} \cancel{\omega}
        \sqrt{2 k-a^{\frac{1}{2}}k^2}
        \sqrt{2 \omega^2k-a^{\frac{1}{2}}\omega^2k^2-a^{\frac{3}{2}}4m^2}
    }
    \nonumber \\ &\kern 2em\cdot
    \Theta\left(2 \omega^2k-a^{\frac{1}{2}}\omega^2k^2-a^{\frac{3}{2}}4m^2
          \right)
    \cdot
    e^{-i\omega\left(\frac{t'+x'}{a}+k \frac{x'}{\sqrt a}\right)}
    \cancel{\omega} \d \omega \d k
    \nonumber \\ &=
    2 a^{-\frac{3}{4}} \int
    \underbrace{
    \left\{
        \int \frac{
            \hat\psi_2(k)\
            \left(
                2 \omega^2k-a^{\frac{1}{2}}\omega^2k^2-a^{\frac{3}{2}}3m^2
            \right)
            \Theta(\dots)
            e^{-i\omega k \frac{x'}{\sqrt a}}
        }
        {
            \sqrt{2 k-a^{\frac{1}{2}}k^2}
            \sqrt{2 \omega^2k-a^{\frac{1}{2}}\omega^2k^2-a^{\frac{3}{2}}4m^2}
        }
        \d k
    \right\}
    }_{=: ~\hat{f}_a(\omega)}
    \nonumber \\ & \kern 2em\cdot
    \hat\psi_1(\omega)
    e^{-i\omega\left(\frac{t'+x'}{a}\right)}
    \d \omega
    \nonumber \\ &=
    2 a^{-\frac{3}{4}} \int
    \hat f_a(\omega)\, \hat\psi_1(\omega)
    e^{-i\omega\left(\frac{t'+x'}{a}\right)}
    \d \omega
\end{align}
% \begin{dmath}
% \label{eq:psi_ast_delta_m2}
%     \left\langle \hat\psi_{ast},\rwhat{\Delta}_m^{*2} \right\rangle
%     =
%     2 a^{-\frac{3}{4}} \int
%     \frac{
%         \hat\psi_1(\omega)\,\hat\psi_2(k)
%         \left(\omega^2\left(
%             a^{-2}\cancel{(1-s^2)}-2a^{-\frac{3}{2}} k s -a^{-1}k^2
%             \right)
%             -3m^2
%         \right)
%     }
%     {
%         \sqrt{
%             \omega^2\left(a^{-2}\cancel{(1-s^2)}-2a^{-\frac{3}{2}} k s -a^{-1}k^2\right)
%         }
%         \sqrt{
%             \omega^2\left(a^{-2}\cancel{(1-s^2)}-2a^{-\frac{3}{2}}k s  -a^{-1}k^2\right) - 4m^2
%         }
%     }
%     \cdot
%     \Theta\left(
%             \omega^2\left(a^{-2}\cancel{(1-s^2)}-2a^{-\frac{3}{2}} -a^{-1}k^2\right) - 4m^2
%         \right)
%     e^{-i\omega\left(\frac{t'-sx'}{a}+k \frac{x'}{\sqrt a}\right)}
%     \cdot
%     \omega \d \omega \d k
%     =
%     2 a^{-\frac{3}{4}} \int
%     \frac{
%         \hat\psi_1(\omega)\,\hat\psi_2(k)\,\cancel{a^{-\frac{3}{2}}}
%         \left(
%             2 \omega^2k-a^{\frac{1}{2}}\omega^2k^2-a^{\frac{3}{2}}3m^2
%         \right)
%     }
%     {
%         \cancel{a^{-\frac{3}{2}}} \cancel{\omega}
%         \sqrt{2 k-a^{\frac{1}{2}}k^2}
%         \sqrt{2 \omega^2k-a^{\frac{1}{2}}\omega^2k^2-a^{\frac{3}{2}}4m^2}
%     }
%     \cdot
%     \Theta\left(2 \omega^2k-a^{\frac{1}{2}}\omega^2k^2-a^{\frac{3}{2}}4m^2
%           \right)
%     \cdot
%     e^{-i\omega\left(\frac{t'+x'}{a}+k \frac{x'}{\sqrt a}\right)}
%     \cancel{\omega} \d \omega \d k
%     =
%     2 a^{-\frac{3}{4}} \int
%     \underbrace{
%     \left\{
%         \int \frac{
%             \hat\psi_2(k)\
%             \left(
%                 2 \omega^2k-a^{\frac{1}{2}}\omega^2k^2-a^{\frac{3}{2}}3m^2
%             \right)
%             \Theta(\dots)
%             e^{-i\omega k \frac{x'}{\sqrt a}}
%         }
%         {
%             \sqrt{2 k-a^{\frac{1}{2}}k^2}
%             \sqrt{2 \omega^2k-a^{\frac{1}{2}}\omega^2k^2-a^{\frac{3}{2}}4m^2}
%         }
%         \d k
%     \right\}
%     }_{=: ~\hat{f}_a(\omega)}
%     \cdot
%     \hat\psi_1(\omega)
%     e^{-i\omega\left(\frac{t'+x'}{a}\right)}
%     \d \omega
%     =
%     2 a^{-\frac{3}{4}} \int
%     \hat f_a(\omega)\, \hat\psi_1(\omega)
%     e^{-i\omega\left(\frac{t'+x'}{a}\right)}
%     \d \omega
% \end{dmath}

Nun müssen wir also $\hat f_a(\omega)$ genauer betrachten: $\hat\psi_2(k) \in C^\infty_c(\mathbb{\hat R})$. $\Theta$ schneidet genau bei der ersten Nullstelle des Nenners ab. Deshalb verschieben wir durch eine Substitution $k \rightarrow k'$ den Integrationsbereich genau so, dass diese Nullstelle bei $k' = 0$ liegt.

Sei also $k_0 := \frac{\omega-\sqrt{\omega^2 - 4a^2m^2}}{\sqrt{a}\omega}$ die relevante Nullstelle des Nenners am Integrationsbereich.
% (denn
% \begin{equation*}
% 2 \omega^2k-a^{\frac{1}{2}}\omega^2k^2-a^{\frac{3}{4}}4m^2
% =
%  -a^{\frac{1}{2}}\left(k-\frac{\omega-\sqrt{\omega^2-4a^2m^2}}{\sqrt a \omega}\right)\left(k-\frac{\omega+\sqrt{\omega^2-4a^2m^2}}{\sqrt a \omega}\right)
% \end{equation*}).
Dann ist die $a$-Abhängigkeit von $k_0$ in erster Näherung gegeben durch $0 < k_0 = \frac{2m^2}{\omega^2}a^{\frac{3}{2}} + O\left(a^{\frac{7}{2}}\right) =: c_\omega a^{\frac{3}{2}} + O\left(a^{\frac{7}{2}}\right)$
und mit $k'=k-k_0$ gelten folgende Ausdrücke für den Nenner und den Zähler:

\subparagraph*{Zähler}

\begin{dmath*}
    2 \omega^2k-a^{\frac{1}{2}}\omega^2k^2-a^{\frac{3}{2}}3m^2
    =
    2\omega^2(k'+k_0)-a^{\frac{1}{2}}\omega^2(k'+k_0)^2-a^{\frac{3}{2}}3m^2
    =
    2 \omega^2 k' + 2 \omega^2 \frac{2m^2}{\omega^2} a^{\frac{3}{2}}
        + 2 \omega^2 O\left(a^{\frac{7}{2}}\right) - a^{\frac{1}{2}} \omega^2(k'+k_0)^2
        -a^{\frac{3}{2}}3m^2
    =
    2 \omega^2k'+a^{\frac{3}{2}}m^2-a^{\frac{1}{2}}\omega^2(k'+k_0)^2
        + O\left(a^{\frac{7}{2}}\right)
\end{dmath*}

\subparagraph*{Nenner}
\begin{dmath*}
    \sqrt{2 k-a^{\frac{1}{2}}k^2}
    \sqrt{2 \omega^2k-a^{\frac{1}{2}}\omega^2k^2-a^{\frac{3}{2}}4m^2}
    =
    \sqrt{2-a^{\frac{1}{2}}(k'+k_0)} \sqrt{k'+k_0}
    \cdot
    \underbrace{
    \sqrt{
            -a^{\frac{1}{2}}\omega^2\left(k'-\tfrac{2\sqrt{\omega^2-4a^2m^2}}
                    {\sqrt a \omega}\right)
        }
    }_{= \sqrt{2-a^{\frac{1}{2}}\omega^2 k' + O\left(a^{\frac{3}{2}}\right)}}
    \sqrt{k'}
\end{dmath*}

Nun ist es an der Zeit für das alte Spiel von "`finde eine integrierbare Majorante, um Lebesgue verwenden und alle Terme mit positiver $a$-Potenz wegschmeißen zu dürfen\footnote{so lange sie in einer Summe mit mindestens einem Term \emph{ohne} positive $a$-Potenz auftauchen}"'

\begin{align}
&\kern -2em
    \left|
    \frac{
        \hat\psi_2(k'+k_0) \left(
        2 \omega^2 k' + a^{\frac{3}{2}}m^2-a^{-\frac{1}{2}}(k'+k_0)^2
            + O \left(a^{\frac{7}{2}}\right)
        \right)
    }
    {
        \sqrt{k'} \sqrt{k'+k_0} \sqrt{2-a^{\frac{1}{2}}(k'+k_0)}
        \sqrt{2-a^{\frac{1}{2}}\omega^2 k' + O\left(a^{\frac{3}{2}}\right)}
    }
    \Theta(k')
    \right|
    \nonumber \\ &\leq
    \frac{\textrm{const}}{\sqrt{k'}}
    \frac{
        \cancel{2}\omega^2 k' + a^{\frac{3}{2}}m^2
        -a^{\frac{1}{2}} \omega^2 (k'+k_0)^2 + O\left(a^{\frac{7}{2}}\right)
    }
    {
        \sqrt{k' + k_0}\cancel{\sqrt{2}} \cancel{\sqrt{2}}
    }\Theta(k')
    \nonumber \\ &\leq
    \frac{\textrm{const}}{\sqrt{k'}}
    \left(
        \frac{\omega^2 k'}{\sqrt{k'}}
        - \frac{a^{\frac{1}{2}}\omega^2(k'+k_0)^2}{\sqrt{k'+k_0}}
        + \frac{a^{\frac{3}{2}}m^2}{\sqrt{k'+k_0}}
        + \frac{O\left(a^{\frac{7}{2}}\right)}{\sqrt{k_0}}
    \right)\Theta(k')
    \nonumber \\ &=
    \frac{\textrm{const}}{\sqrt{k'}}
    \Bigg(
        \omega^2 \sqrt{k'} - a^{\frac{1}{2}}\omega^2 (k'+k_0)^{\frac{3}{2}}
        + \underbrace{\frac{a^{\frac{3}{2}}m^2}
                    {\sqrt{\frac{2m^2}{\omega^2}a^{\frac{3}{2}}+O(a^{\frac{7}{2}})}}}_{O\left(a^{\frac{3}{4}}\right)}
        + \dots
    \Bigg)\Theta(k')
    \nonumber \\&\leq
    \frac{\textrm{const}}{\sqrt{k'}}\Theta(k')
    \label{eq:lange_1/sqrt_abschaetzerei}
\end{align}
% \begin{dmath}
%     \left|
%     \frac{
%         \hat\psi_2(k'+k_0) \left(
%         2 \omega^2 k' + a^{\frac{3}{2}}m^2-a^{-\frac{1}{2}}(k'+k_0)^2
%             + O \left(a^{\frac{7}{2}}\right)
%         \right)
%     }
%     {
%         \sqrt{k'} \sqrt{k'+k_0} \sqrt{2-a^{\frac{1}{2}}(k'+k_0)}
%         \sqrt{2-a^{\frac{1}{2}}\omega^2 k' + O\left(a^{\frac{3}{2}}\right)}
%     }
%     \Theta(k')
%     \right|
%     \leq
%     \frac{\textrm{const}}{\sqrt{k'}}
%     \frac{
%         \cancel{2}\omega^2 k' + a^{\frac{3}{2}}m^2
%         -a^{\frac{1}{2}} \omega^2 (k'+k_0)^2 + O\left(a^{\frac{7}{2}}\right)
%     }
%     {
%         \sqrt{k' + k_0}\cancel{\sqrt{2}} \cancel{\sqrt{2}}
%     }\Theta(k')
%     \leq
%     \frac{\textrm{const}}{\sqrt{k'}}
%     \left(
%         \frac{\omega^2 k'}{\sqrt{k'}}
%         - \frac{a^{\frac{1}{2}}\omega^2(k'+k_0)^2}{\sqrt{k'+k_0}}
%         + \frac{a^{\frac{3}{2}}m^2}{\sqrt{k'+k_0}}
%         + \frac{O\left(a^{\frac{7}{2}}\right)}{\sqrt{k_0}}
%     \right)\Theta(k')
%     =
%     \frac{\textrm{const}}{\sqrt{k'}}
%     \Bigg(
%         \omega^2 \sqrt{k'} - a^{\frac{1}{2}}\omega^2 (k'+k_0)^{\frac{3}{2}}
%         + \underbrace{\frac{a^{\frac{3}{2}}m^2}
%                     {\sqrt{\frac{2m^2}{\omega^2}a^{\frac{3}{2}}+O(a^{\frac{7}{2}})}}}_{O\left(a^{\frac{3}{4}}\right)}
%         + \dots
%     \Bigg)\Theta(k')
%     \leq
%     \frac{\textrm{const}}{\sqrt{k'}}\Theta(k')
%     \label{eq:lange_1/sqrt_abschaetzerei}
% \end{dmath}

Der letzte Ausdruck ist eine integrierbare Majorante und in den Abschätzungen wurde u.a. verwendet, das $\hat\psi_2$ kompakt getragen und beschränkt ist. In "`const'" wurden immer notwendige aber letzten Endes irrelevante Vorfaktoren gesammelt wie z.B. $\Vert \hat\psi_2\Vert_\infty$.

Der Integrand für $\hat f_a$ konvergiert punktweise (vgl. \cref{eq:psi_ast_delta_m2})

\begin{dmath}
\frac{
        \hat\psi_2(k)\
        \left(
            2 \omega^2k-a^{\frac{1}{2}}\omega^2k^2-a^{\frac{3}{2}}3m^2
        \right)
        \Theta(\dots)
        e^{-i\omega k \frac{x'}{\sqrt a}}
    }
    {
        \sqrt{2 k-a^{\frac{1}{2}}k^2}
        \sqrt{2 \omega^2k-a^{\frac{1}{2}}\omega^2k^2-a^{\frac{3}{2}}4m^2}
    }
    \rightarrow
    \hat \psi_2(k) \omega \Theta(k) e^{-i\omega k \frac{x'}{\sqrt{a}}}
\label{eq:langer_sqrt_bruch_punktweise_konvergenz}
\end{dmath}

und wir können also schreiben

\todo{Tief in uns drinnen, wissen wir alle, dass die Behauptung mit dem $O(x^{-1})$ stimmt, müssen wir sie also wirklich noch zeigen?}

\begin{dmath}
    \hat f_a(\omega) \hiderel \rightarrow \hat f_0(\omega)
    =
    \int \omega\, \hat\psi_2(k)\, \Theta(k)\, e^{-i\omega k \frac{x'}{\sqrt{a}}} \d k
    = \omega (\hat\psi_2 \cdot \Theta)^\vee (-\omega x'/\sqrt a)
    = \omega \left(\hat\psi_2^\vee * \Theta^\vee \right)(-\omega x'/\sqrt a)
    = \omega \left(\psi_2 * \left(\delta + i\mathcal{P}(1/x)\right)
             \right)(-\omega x'/\sqrt a)
    = \omega \bigg[
                \underbrace{
                    \psi_2(-\omega x'/\sqrt a)}_{O(a^k) \; \forall k \in \mathbb{N}}
                + \underbrace{i
                    \underbrace{\left(\psi_2 * \mathcal{P}(1/x)\right)}_{O(x^{-1})}
                    (-\omega x'/\sqrt a)
                }_{
                    O\left((-\omega x'/\sqrt a)^{-1}\right)
                    = O\left(a^{\frac{1}{2}}\right)
                   }
             \bigg]
    \sim O \left(a^{\frac{1}{2}}\right) \condition{falls $x' \neq 0$}
    \sim \textrm{const} \condition{falls $x'=0$}
\label{eq:faltung_mit_1/x_rechnung}
\end{dmath}

Wir dürfen also, falls $x' \neq 0$, folgende Abschätzung für $\hat f_a(\omega)$ für $a \to 0$ machen:

\begin{equation*}
    \hat f_a(\omega) = \omega C a^{\frac{1}{2}} +o\left(a^{\frac{1}{2}}\right)
\end{equation*}

Setzen wir dies nun schließlich wieder in unseren letzten Ausdruck in \cref{eq:psi_ast_delta_m2} ein, erhalten wir endlich

\todo{hier fehlen wenn man es ganz genau nimmt noch $O(a^{\frac{7}{2}})$-Terme.
    Wird da noch eine Bemerkung zu geschrieben, oder nehme ich die ganz brav mit?}

\todo{
    Fall x'=t'=0 fehlt noch! Erwartet wird $a^{-\frac{3}{4}}$
}

\begin{dmath}
\label{eq:delta_m2_s=1}
    \left\langle \rwhat{\Delta}_m^{*2}, \hat\psi_{ast}\right\rangle
    = 2 a^{-\frac{3}{4}} \int
         \underbrace{
             \omega C a^{\frac{1}{2}} \hat\psi_1(\omega)
             }_{
                \in C_c^\infty (\hat{\mathbb{R}})
             }
        e^{-i\omega \left(\frac{t'+x'}{a}\right)} \d \omega
    = 2 a^{-\frac{1}{4}} C \left(\omega \hat\psi_1(\omega)\right)^\vee
        \left(-\frac{t'+x'}{a}\right)
    \sim O (a^{-\frac{1}{4}}) \condition{falls $t' = -x' \neq 0$}
    \sim (a^{-\frac{3}{4}}) \condition{falls $t'=0=x'$}
    \sim O (a^k) ~~ \forall k \hiderel \in \mathbb{N}
    \condition{\textrm{sonst}}
\end{dmath}

\subsection{Zusammenfassung und Vergleich der Ergebnisse}
Wenn wir die Ergebnisse aus \cref{eq:delta_m2_s>1,eq:delta_m2_s<1,eq:delta_m2_s=1} zusammenfassen, haben wir für die Wellenfrontmenge von $\rwhat{\Delta}_m^{*2}$

\begin{table}[h]
\centering
\label{tab:wavefrontset_delta_m2}
\begin{tabular}{l|cccc}
        & $(t',x') = 0$      & $t'=x' \neq 0$     & $t'=-x' \neq 0$    & $t' \neq \pm x'$ \\ \hline
$s=1$   & $a^{-\frac{3}{4}}$ & $a^{-\frac{1}{4}}$ & $a^k$              & $a^k$            \\
$s=-1$  & $a^{-\frac{3}{4}}$ & $a^k$              & $a^{-\frac{1}{4}}$ & $a^k$            \\
$|s|<1$ & $a^{-\frac{3}{4}}$ & $a^k$              & $a^k$              & $a^k$            \\
$|s|>1$ & $a^k$              & $a^k$              & $a^k$              & $a^k$
\end{tabular}
\caption{Konvergenzordnung von $\mathcal{S}_{\Delta_m^2}(a,s,(t',x'))$ im Limit $a \to 0$ für alle interessanten Kombinationen von $s$ und $(t',x')$}
\end{table}

Erfreulicherweise deckt sich dies wieder mit den Ergebnissen von \textcite[Cor. 3.70]{Schulz2014}, welcher für allgemeine Potenzen von $\Delta_+$ folgendes erhält:

\begin{equation*}
    WF_{SG}^\psi (\Delta_+^k) \subset
    WF_{SG}^\psi(\Delta_+) \cup
    \{\left\langle0,0;-\lambda, |x|\right\rangle\ \big| \, |x| \hiderel\in \hat{\mathbb{R}}, \lambda \hiderel > |x|\}
\end{equation*}

% section dots_und_nun_zur_wellenfrontmenge (end)








% section die_wellenfrontmenge_von_delta_m_2_ (end)


% !TEX root = main.tex
% !TEX spellcheck=de_DE
%%%%%%%%%%%%%%%%%%%%%%%%%%%%%%%%%%%%%%%%%%%%%%%%%%%%%%%%%%%%%%%%%%%%%%%%%%%%%%%
% % Berechnen der Wellenfrontmenge von Delta_m_twisted
%%%%%%%%%%%%%%%%%%%%%%%%%%%%%%%%%%%%%%%%%%%%%%%%%%%%%%%%%%%%%%%%%%%%%%%%%%%%%%%

\section{\texorpdfstring{Die Wellenfrontmenge von $\Delta_m^{\star 2}$}
         {Die Wellenfrontmenge der getwisteten Zweipunktfunktion}} % (fold)
\label{sec:die_wellenfrontmenge_von_delta_m2_twisted}

Bevor wir uns aber der Wellenfrontmenge widmen können, brauchen wir einen Ausdruck für die Fouriertransformierte $\rwhat{\Delta}_m^{\circledast 2}$ von $\Delta_m^{\star 2}$.

\subsection{\texorpdfstring{$\hat\Delta_m^{\circledast 2}$ berechnen}
            {Die getwistete Zweipunktfunktion berechnen}} % (fold)
\label{sec:delta_m2_twisted_berechnen}

Sammeln wir zunächst einmal die Zutaten, die wir für die getwistete Faltung der massiven Zweipunktfunktion mit sich selber brauchen:

\begin{dgroup}
    \begin{dmath}
        \rwhat{\Delta}_m = \delta(\omega^2-k^w-m^2)\Theta(\omega)\\
        \textrm{die Fouriertransformierte der massiven Zweipunktfunktion}
    \label{eq:material_fuer_delta_m2_twisted_a}
    \end{dmath}
    \begin{dmath}
        \Omega = \begin{pmatrix}
            0 & 1 \\ -1 & 1
        \end{pmatrix}
        \\ \textrm{die kanonische symplektische Matrix auf } \mathbb{R}^n
    \label{eq:material_fuer_delta_m2_twisted_b}
    \end{dmath}
\end{dgroup}

mit \cref{def:twisted_convolution,eq:material_fuer_delta_m2_twisted_a,eq:material_fuer_delta_m2_twisted_b} erhalten wir also

\begin{dmath}
    \rwhat{\Delta}_m^{\circledast 2} (\omega, k)
    = \int
    \delta(\omega^{\prime 2}-k^{\prime 2}-m^2)
    \delta((\omega' - \omega)^2 - (k-k')^2 -m^2)
    \cdot
    \Theta(\omega') \Theta(\omega - \omega')
    e^{\frac{i}{2}(\omega'k-\omega k')}
    \d \omega' \d k'
\end{dmath}

und damit das selbe Integral wie in \cref{eq:mass_shell_convolution} bis auf einen zusätzlichen Phasenfaktor. Nachdem wir gezeigt haben, dass auch dieser Lorentz-Invariant ist, können wir das Integral mit dem selben Trick wie in \cref{sec:delta_m2_berechnen} berechnen.

\begin{proposition}[$\Omega_{std}$ ist Lorentz-invariant für $n=2$]
\label{prop:omega_ist_Lorentz_invariant}
    $\Omega_{std}$ ist Lorentz-invariant für $n=2$
\\[1em]
\emph{Beweis}\\
    Eine einfache Rechnung zeigt
    \begin{dmath*}
        \begin{pmatrix}
            \cosh \beta & \sinh \beta \\ -\sinh \beta & \cosh \beta
        \end{pmatrix}
        \begin{pmatrix}
            0 & 1 \\ -1 & 0
        \end{pmatrix}
        \begin{pmatrix}
            \cosh \beta & -\sinh \beta \\ \sinh \beta & \cosh \beta
        \end{pmatrix}
        =
        \begin{pmatrix}
            0 & 1 \\ -1 & 0
        \end{pmatrix}
    \end{dmath*}
    für alle $\beta \in \mathbb{R}$.
\end{proposition}

Mit \cref{prop:omega_ist_Lorentz_invariant} ist $\rwhat{\Delta}_m^{\circledast 2}$ Lorentz-Invariant und es reicht aus $\rwhat{\Delta}_m^{\circledast 2} (\omega, 0)$ zu berechnen.

Die beiden Kreuzungspunkte der $\delta$-Distributionen liegen bei (vgl. \cref{fig:mass_shell_convolution})

\begin{equation*}
    \left(\omega'_0,k'_{0\pm}\right) = \left(\frac{\omega}{2}, \pm \sqrt{\left(\frac{\omega}{2}\right)^2-m^2}\right)
\end{equation*}


Die "`Fläche"' der Kreuzungspunkte der $\delta$-Distributionen wurde in
 \cref{sec:delta_m2_berechnen} berechnet und ist

\begin{equation*}
A = \frac{\omega^2-3m^2}{\omega \sqrt{\omega^2-4m^2}}.
\end{equation*}

Der Phasenfaktor nimmt bei den Kreuzungspunkten folgende Werte an:
\begin{dmath*}
    e^{\frac{i}{2}\Omega \left((\omega, k),(\omega'_0,k'_{0\pm})\right)}
    =
    e^{\pm \frac{i}{2}\left(-\omega^2\sqrt{\frac{1}{4}-\frac{m^2}{\omega^2}}\right)}
\end{dmath*}


Kombinieren wir also die vorhergehenden Resultate erhalten wir

\begin{align*}
    \rwhat{\Delta}_m^{\circledast 2} (\omega, 0)
    &=
    A e^{\frac{i}{2}\Omega \left((\omega, k),(\omega'_0,k'_{0+})\right)}
    + A e^{\frac{i}{2}\Omega \left((\omega, k),(\omega'_0,k'_{0-})\right)}
    \\&=
    \frac{\omega^2-3m^2}{\omega \sqrt{\omega^2-4m^2}}
    \left\{
        e^{-\frac{i}{2}\omega^2\sqrt{\frac{1}{4}-\frac{m^2}{\omega^2}}}
      + e^{\frac{i}{2}\omega^2\sqrt{\frac{1}{4}-\frac{m^2}{\omega^2}}}
    \right\}
    \Theta\left(\omega^2-4m^2\right)
    \\&=
    2 \frac{\omega^2 -3m^2}{\omega \sqrt{\omega^2-4m^2}}
    \cos \left(\varphi(\omega^2)\right) \Theta\left(\omega^2-4m^2\right),
\end{align*}
% \begin{dmath*}
%     \rwhat{\Delta}_m^{\circledast 2} (\omega, 0)
%     =
%     A e^{\frac{i}{2}\Omega \left((\omega, k),(\omega'_0,k'_{0+})\right)}
%     + A e^{\frac{i}{2}\Omega \left((\omega, k),(\omega'_0,k'_{0-})\right)}
%     =
%     \frac{\omega^2-3m^2}{\omega \sqrt{\omega^2-4m^2}}
%     \left\{
%         e^{-\frac{i}{2}\omega^2\sqrt{\frac{1}{4}-\frac{m^2}{\omega^2}}}
%       + e^{\frac{i}{2}\omega^2\sqrt{\frac{1}{4}-\frac{m^2}{\omega^2}}}
%     \right\}
%     \Theta\left(\omega^2-4m^2\right)
%     =
%     2 \frac{\omega^2 -3m^2}{\omega \sqrt{\omega^2-4m^2}}
%     \cos \left(\varphi(\omega^2)\right) \Theta\left(\omega^2-4m^2\right)
% \end{dmath*}

wobei im letzten Schritt noch implizit $\varphi(\omega^2)$ definiert wurde.
Und mit Lorentz-Invarianz erhalten wir schließlich

\begin{align}
    \rwhat{\Delta}_m^{\circledast 2} (\omega, k)
    &=
    \rwhat{\Delta}_m^{\circledast 2} (\sqrt{\omega^2-k^2}, 0)
    \nonumber \\ &=
    2\frac{\omega^2-k^2-3m^2}{\sqrt{\omega^2-k^2} \sqrt{\omega^2-k^2-4m^2}}
    \cos \left(\frac{k^2-\omega^2}{2}
    \sqrt{\frac{1}{4}+\frac{m^2}{k^2-\omega^2}}
    \right)
    \nonumber \\ & \kern 12em\cdot
    \Theta \left(\omega^2-k^2-4m^2\right)
    \nonumber \\ &=
    \rwhat{\Delta}_m^{* 2}(\omega, k) \cos (\varphi(\omega^2-k^2))
    \Theta\left(\omega^2-k^2-4m^2\right),
\end{align}
% \begin{dmath}
%     \rwhat{\Delta}_m^{\circledast 2} (\omega, k)
%     =
%     \rwhat{\Delta}_m^{\circledast 2} (\sqrt{\omega^2-k^2}, 0)
%     =
%     2\frac{\omega^2-k^2-3m^2}{\sqrt{\omega^2-k^2} \sqrt{\omega^2-k^2-4m^2}}
%     \cos \left(\frac{k^2-\omega^2}{2}
%     \sqrt{\frac{1}{4}+\frac{m^2}{k^2-\omega^2}}
%     \right)
%     \Theta \left(\omega^2-k^2-4m^2\right)
%     =
%     \rwhat{\Delta}_m^{* 2}(\omega, k) \cos (\varphi(\omega^2-k^2))
% \end{dmath}

\subsection{
\texorpdfstring{\dots und nun zur Wellenfrontmenge von $\hat{\Delta}_m^{\circledast 2}$}{... und nun zur Wellenfrontmenge der getwisteten Zweipunktfunktion}} % (fold)
\label{sec:dots_und_nun_zur_wellenfrontmenge_von_delta_m2_twisted}

\begin{figure}
    \centering
    \begin{minipage}{0.55\textwidth}
        \centering
        \resizebox{\textwidth}{!}{%% Creator: Matplotlib, PGF backend
%%
%% To include the figure in your LaTeX document, write
%%   \input{<filename>.pgf}
%%
%% Make sure the required packages are loaded in your preamble
%%   \usepackage{pgf}
%%
%% Figures using additional raster images can only be included by \input if
%% they are in the same directory as the main LaTeX file. For loading figures
%% from other directories you can use the `import` package
%%   \usepackage{import}
%% and then include the figures with
%%   \import{<path to file>}{<filename>.pgf}
%%
%% Matplotlib used the following preamble
%%   \usepackage[utf8x]{inputenc}
%%   \usepackage[T1]{fontenc}
%%   \usepackage{amssymb}
%%
\begingroup%
\makeatletter%
\begin{pgfpicture}%
\pgfpathrectangle{\pgfpointorigin}{\pgfqpoint{4.000000in}{2.200000in}}%
\pgfusepath{use as bounding box, clip}%
\begin{pgfscope}%
\pgfsetbuttcap%
\pgfsetmiterjoin%
\definecolor{currentfill}{rgb}{1.000000,1.000000,1.000000}%
\pgfsetfillcolor{currentfill}%
\pgfsetlinewidth{0.000000pt}%
\definecolor{currentstroke}{rgb}{1.000000,1.000000,1.000000}%
\pgfsetstrokecolor{currentstroke}%
\pgfsetdash{}{0pt}%
\pgfpathmoveto{\pgfqpoint{0.000000in}{0.000000in}}%
\pgfpathlineto{\pgfqpoint{4.000000in}{0.000000in}}%
\pgfpathlineto{\pgfqpoint{4.000000in}{2.200000in}}%
\pgfpathlineto{\pgfqpoint{0.000000in}{2.200000in}}%
\pgfpathclose%
\pgfusepath{fill}%
\end{pgfscope}%
\begin{pgfscope}%
\pgfsetbuttcap%
\pgfsetmiterjoin%
\definecolor{currentfill}{rgb}{1.000000,1.000000,1.000000}%
\pgfsetfillcolor{currentfill}%
\pgfsetlinewidth{0.000000pt}%
\definecolor{currentstroke}{rgb}{0.000000,0.000000,0.000000}%
\pgfsetstrokecolor{currentstroke}%
\pgfsetstrokeopacity{0.000000}%
\pgfsetdash{}{0pt}%
\pgfpathmoveto{\pgfqpoint{0.198611in}{0.333208in}}%
\pgfpathlineto{\pgfqpoint{3.119722in}{0.333208in}}%
\pgfpathlineto{\pgfqpoint{3.119722in}{1.866792in}}%
\pgfpathlineto{\pgfqpoint{0.198611in}{1.866792in}}%
\pgfpathclose%
\pgfusepath{fill}%
\end{pgfscope}%
\begin{pgfscope}%
\pgfpathrectangle{\pgfqpoint{0.198611in}{0.333208in}}{\pgfqpoint{2.921111in}{1.533583in}} %
\pgfusepath{clip}%
\pgfsys@transformshift{0.198611in}{0.333208in}%
\pgftext[left,bottom]{\pgfimage[interpolate=true,width=2.921667in,height=1.535000in]{delta_m2_twisted-img0.png}}%
\end{pgfscope}%
\begin{pgfscope}%
\pgfpathrectangle{\pgfqpoint{0.198611in}{0.333208in}}{\pgfqpoint{2.921111in}{1.533583in}} %
\pgfusepath{clip}%
\pgfsetrectcap%
\pgfsetroundjoin%
\pgfsetlinewidth{0.501875pt}%
\definecolor{currentstroke}{rgb}{0.894118,0.101961,0.109804}%
\pgfsetstrokecolor{currentstroke}%
\pgfsetdash{}{0pt}%
\pgfpathmoveto{\pgfqpoint{0.204257in}{1.868458in}}%
\pgfpathlineto{\pgfqpoint{0.450330in}{1.623864in}}%
\pgfpathlineto{\pgfqpoint{0.643510in}{1.432341in}}%
\pgfpathlineto{\pgfqpoint{0.795712in}{1.281957in}}%
\pgfpathlineto{\pgfqpoint{0.912791in}{1.166768in}}%
\pgfpathlineto{\pgfqpoint{1.006453in}{1.075091in}}%
\pgfpathlineto{\pgfqpoint{1.088408in}{0.995386in}}%
\pgfpathlineto{\pgfqpoint{1.152802in}{0.933244in}}%
\pgfpathlineto{\pgfqpoint{1.211341in}{0.877278in}}%
\pgfpathlineto{\pgfqpoint{1.258172in}{0.833001in}}%
\pgfpathlineto{\pgfqpoint{1.299150in}{0.794752in}}%
\pgfpathlineto{\pgfqpoint{1.334274in}{0.762449in}}%
\pgfpathlineto{\pgfqpoint{1.363543in}{0.735972in}}%
\pgfpathlineto{\pgfqpoint{1.392813in}{0.710007in}}%
\pgfpathlineto{\pgfqpoint{1.416229in}{0.689699in}}%
\pgfpathlineto{\pgfqpoint{1.439644in}{0.669907in}}%
\pgfpathlineto{\pgfqpoint{1.457206in}{0.655476in}}%
\pgfpathlineto{\pgfqpoint{1.474768in}{0.641470in}}%
\pgfpathlineto{\pgfqpoint{1.492330in}{0.627972in}}%
\pgfpathlineto{\pgfqpoint{1.509891in}{0.615079in}}%
\pgfpathlineto{\pgfqpoint{1.521599in}{0.606878in}}%
\pgfpathlineto{\pgfqpoint{1.533307in}{0.599039in}}%
\pgfpathlineto{\pgfqpoint{1.545015in}{0.591608in}}%
\pgfpathlineto{\pgfqpoint{1.556723in}{0.584637in}}%
\pgfpathlineto{\pgfqpoint{1.568431in}{0.578182in}}%
\pgfpathlineto{\pgfqpoint{1.580139in}{0.572301in}}%
\pgfpathlineto{\pgfqpoint{1.591846in}{0.567060in}}%
\pgfpathlineto{\pgfqpoint{1.603554in}{0.562521in}}%
\pgfpathlineto{\pgfqpoint{1.609408in}{0.560535in}}%
\pgfpathlineto{\pgfqpoint{1.615262in}{0.558748in}}%
\pgfpathlineto{\pgfqpoint{1.621116in}{0.557167in}}%
\pgfpathlineto{\pgfqpoint{1.626970in}{0.555798in}}%
\pgfpathlineto{\pgfqpoint{1.632824in}{0.554648in}}%
\pgfpathlineto{\pgfqpoint{1.638678in}{0.553722in}}%
\pgfpathlineto{\pgfqpoint{1.644532in}{0.553023in}}%
\pgfpathlineto{\pgfqpoint{1.650386in}{0.552555in}}%
\pgfpathlineto{\pgfqpoint{1.656240in}{0.552321in}}%
\pgfpathlineto{\pgfqpoint{1.662094in}{0.552321in}}%
\pgfpathlineto{\pgfqpoint{1.667948in}{0.552555in}}%
\pgfpathlineto{\pgfqpoint{1.673801in}{0.553023in}}%
\pgfpathlineto{\pgfqpoint{1.679655in}{0.553722in}}%
\pgfpathlineto{\pgfqpoint{1.685509in}{0.554648in}}%
\pgfpathlineto{\pgfqpoint{1.691363in}{0.555798in}}%
\pgfpathlineto{\pgfqpoint{1.697217in}{0.557167in}}%
\pgfpathlineto{\pgfqpoint{1.703071in}{0.558748in}}%
\pgfpathlineto{\pgfqpoint{1.708925in}{0.560535in}}%
\pgfpathlineto{\pgfqpoint{1.720633in}{0.564698in}}%
\pgfpathlineto{\pgfqpoint{1.732341in}{0.569597in}}%
\pgfpathlineto{\pgfqpoint{1.744049in}{0.575166in}}%
\pgfpathlineto{\pgfqpoint{1.755757in}{0.581341in}}%
\pgfpathlineto{\pgfqpoint{1.767464in}{0.588062in}}%
\pgfpathlineto{\pgfqpoint{1.779172in}{0.595269in}}%
\pgfpathlineto{\pgfqpoint{1.790880in}{0.602910in}}%
\pgfpathlineto{\pgfqpoint{1.802588in}{0.610936in}}%
\pgfpathlineto{\pgfqpoint{1.814296in}{0.619302in}}%
\pgfpathlineto{\pgfqpoint{1.831858in}{0.632409in}}%
\pgfpathlineto{\pgfqpoint{1.849419in}{0.646087in}}%
\pgfpathlineto{\pgfqpoint{1.866981in}{0.660242in}}%
\pgfpathlineto{\pgfqpoint{1.884543in}{0.674800in}}%
\pgfpathlineto{\pgfqpoint{1.907959in}{0.694732in}}%
\pgfpathlineto{\pgfqpoint{1.931374in}{0.715152in}}%
\pgfpathlineto{\pgfqpoint{1.960644in}{0.741230in}}%
\pgfpathlineto{\pgfqpoint{1.989914in}{0.767796in}}%
\pgfpathlineto{\pgfqpoint{2.025037in}{0.800182in}}%
\pgfpathlineto{\pgfqpoint{2.066015in}{0.838506in}}%
\pgfpathlineto{\pgfqpoint{2.112846in}{0.882846in}}%
\pgfpathlineto{\pgfqpoint{2.165532in}{0.933244in}}%
\pgfpathlineto{\pgfqpoint{2.229925in}{0.995386in}}%
\pgfpathlineto{\pgfqpoint{2.306026in}{1.069379in}}%
\pgfpathlineto{\pgfqpoint{2.393835in}{1.155282in}}%
\pgfpathlineto{\pgfqpoint{2.505060in}{1.264646in}}%
\pgfpathlineto{\pgfqpoint{2.639700in}{1.397588in}}%
\pgfpathlineto{\pgfqpoint{2.809464in}{1.565769in}}%
\pgfpathlineto{\pgfqpoint{3.020205in}{1.775089in}}%
\pgfpathlineto{\pgfqpoint{3.114076in}{1.868458in}}%
\pgfpathlineto{\pgfqpoint{3.114076in}{1.868458in}}%
\pgfusepath{stroke}%
\end{pgfscope}%
\begin{pgfscope}%
\pgfpathrectangle{\pgfqpoint{0.198611in}{0.333208in}}{\pgfqpoint{2.921111in}{1.533583in}} %
\pgfusepath{clip}%
\pgfsetrectcap%
\pgfsetroundjoin%
\pgfsetlinewidth{0.200750pt}%
\definecolor{currentstroke}{rgb}{0.993248,0.906157,0.143936}%
\pgfsetstrokecolor{currentstroke}%
\pgfsetdash{}{0pt}%
\pgfpathmoveto{\pgfqpoint{0.226419in}{1.868458in}}%
\pgfpathlineto{\pgfqpoint{0.356667in}{1.741090in}}%
\pgfpathlineto{\pgfqpoint{0.473746in}{1.627118in}}%
\pgfpathlineto{\pgfqpoint{0.573263in}{1.530743in}}%
\pgfpathlineto{\pgfqpoint{0.661072in}{1.446199in}}%
\pgfpathlineto{\pgfqpoint{0.737173in}{1.373398in}}%
\pgfpathlineto{\pgfqpoint{0.807420in}{1.306681in}}%
\pgfpathlineto{\pgfqpoint{0.871813in}{1.246030in}}%
\pgfpathlineto{\pgfqpoint{0.930352in}{1.191411in}}%
\pgfpathlineto{\pgfqpoint{0.983038in}{1.142768in}}%
\pgfpathlineto{\pgfqpoint{1.029869in}{1.100026in}}%
\pgfpathlineto{\pgfqpoint{1.070847in}{1.063084in}}%
\pgfpathlineto{\pgfqpoint{1.111824in}{1.026649in}}%
\pgfpathlineto{\pgfqpoint{1.146948in}{0.995895in}}%
\pgfpathlineto{\pgfqpoint{1.182071in}{0.965654in}}%
\pgfpathlineto{\pgfqpoint{1.211341in}{0.940911in}}%
\pgfpathlineto{\pgfqpoint{1.240611in}{0.916646in}}%
\pgfpathlineto{\pgfqpoint{1.269880in}{0.892932in}}%
\pgfpathlineto{\pgfqpoint{1.293296in}{0.874414in}}%
\pgfpathlineto{\pgfqpoint{1.316712in}{0.856352in}}%
\pgfpathlineto{\pgfqpoint{1.340127in}{0.838804in}}%
\pgfpathlineto{\pgfqpoint{1.363543in}{0.821835in}}%
\pgfpathlineto{\pgfqpoint{1.381105in}{0.809532in}}%
\pgfpathlineto{\pgfqpoint{1.398667in}{0.797630in}}%
\pgfpathlineto{\pgfqpoint{1.416229in}{0.786167in}}%
\pgfpathlineto{\pgfqpoint{1.433790in}{0.775185in}}%
\pgfpathlineto{\pgfqpoint{1.451352in}{0.764727in}}%
\pgfpathlineto{\pgfqpoint{1.468914in}{0.754840in}}%
\pgfpathlineto{\pgfqpoint{1.480622in}{0.748591in}}%
\pgfpathlineto{\pgfqpoint{1.492330in}{0.742634in}}%
\pgfpathlineto{\pgfqpoint{1.504038in}{0.736984in}}%
\pgfpathlineto{\pgfqpoint{1.515745in}{0.731657in}}%
\pgfpathlineto{\pgfqpoint{1.527453in}{0.726669in}}%
\pgfpathlineto{\pgfqpoint{1.539161in}{0.722037in}}%
\pgfpathlineto{\pgfqpoint{1.550869in}{0.717776in}}%
\pgfpathlineto{\pgfqpoint{1.562577in}{0.713902in}}%
\pgfpathlineto{\pgfqpoint{1.574285in}{0.710430in}}%
\pgfpathlineto{\pgfqpoint{1.585993in}{0.707373in}}%
\pgfpathlineto{\pgfqpoint{1.597700in}{0.704744in}}%
\pgfpathlineto{\pgfqpoint{1.609408in}{0.702555in}}%
\pgfpathlineto{\pgfqpoint{1.621116in}{0.700815in}}%
\pgfpathlineto{\pgfqpoint{1.632824in}{0.699533in}}%
\pgfpathlineto{\pgfqpoint{1.644532in}{0.698714in}}%
\pgfpathlineto{\pgfqpoint{1.656240in}{0.698362in}}%
\pgfpathlineto{\pgfqpoint{1.667948in}{0.698479in}}%
\pgfpathlineto{\pgfqpoint{1.679655in}{0.699065in}}%
\pgfpathlineto{\pgfqpoint{1.691363in}{0.700116in}}%
\pgfpathlineto{\pgfqpoint{1.703071in}{0.701628in}}%
\pgfpathlineto{\pgfqpoint{1.714779in}{0.703594in}}%
\pgfpathlineto{\pgfqpoint{1.726487in}{0.706004in}}%
\pgfpathlineto{\pgfqpoint{1.738195in}{0.708849in}}%
\pgfpathlineto{\pgfqpoint{1.749903in}{0.712115in}}%
\pgfpathlineto{\pgfqpoint{1.761610in}{0.715790in}}%
\pgfpathlineto{\pgfqpoint{1.773318in}{0.719859in}}%
\pgfpathlineto{\pgfqpoint{1.785026in}{0.724308in}}%
\pgfpathlineto{\pgfqpoint{1.796734in}{0.729119in}}%
\pgfpathlineto{\pgfqpoint{1.808442in}{0.734279in}}%
\pgfpathlineto{\pgfqpoint{1.820150in}{0.739769in}}%
\pgfpathlineto{\pgfqpoint{1.831858in}{0.745575in}}%
\pgfpathlineto{\pgfqpoint{1.843565in}{0.751680in}}%
\pgfpathlineto{\pgfqpoint{1.861127in}{0.761366in}}%
\pgfpathlineto{\pgfqpoint{1.878689in}{0.771639in}}%
\pgfpathlineto{\pgfqpoint{1.896251in}{0.782451in}}%
\pgfpathlineto{\pgfqpoint{1.913813in}{0.793758in}}%
\pgfpathlineto{\pgfqpoint{1.931374in}{0.805518in}}%
\pgfpathlineto{\pgfqpoint{1.948936in}{0.817691in}}%
\pgfpathlineto{\pgfqpoint{1.966498in}{0.830243in}}%
\pgfpathlineto{\pgfqpoint{1.989914in}{0.847510in}}%
\pgfpathlineto{\pgfqpoint{2.013329in}{0.865322in}}%
\pgfpathlineto{\pgfqpoint{2.036745in}{0.883619in}}%
\pgfpathlineto{\pgfqpoint{2.060161in}{0.902346in}}%
\pgfpathlineto{\pgfqpoint{2.089431in}{0.926290in}}%
\pgfpathlineto{\pgfqpoint{2.118700in}{0.950754in}}%
\pgfpathlineto{\pgfqpoint{2.147970in}{0.975672in}}%
\pgfpathlineto{\pgfqpoint{2.183093in}{1.006093in}}%
\pgfpathlineto{\pgfqpoint{2.218217in}{1.037002in}}%
\pgfpathlineto{\pgfqpoint{2.259195in}{1.073591in}}%
\pgfpathlineto{\pgfqpoint{2.300172in}{1.110663in}}%
\pgfpathlineto{\pgfqpoint{2.347003in}{1.153530in}}%
\pgfpathlineto{\pgfqpoint{2.399689in}{1.202290in}}%
\pgfpathlineto{\pgfqpoint{2.458228in}{1.257017in}}%
\pgfpathlineto{\pgfqpoint{2.522621in}{1.317764in}}%
\pgfpathlineto{\pgfqpoint{2.592869in}{1.384565in}}%
\pgfpathlineto{\pgfqpoint{2.668970in}{1.457441in}}%
\pgfpathlineto{\pgfqpoint{2.756779in}{1.542053in}}%
\pgfpathlineto{\pgfqpoint{2.856295in}{1.638489in}}%
\pgfpathlineto{\pgfqpoint{2.967520in}{1.746802in}}%
\pgfpathlineto{\pgfqpoint{3.091914in}{1.868458in}}%
\pgfpathlineto{\pgfqpoint{3.091914in}{1.868458in}}%
\pgfusepath{stroke}%
\end{pgfscope}%
\begin{pgfscope}%
\pgfpathrectangle{\pgfqpoint{0.198611in}{0.333208in}}{\pgfqpoint{2.921111in}{1.533583in}} %
\pgfusepath{clip}%
\pgfsetbuttcap%
\pgfsetroundjoin%
\pgfsetlinewidth{0.501875pt}%
\definecolor{currentstroke}{rgb}{0.501961,0.501961,0.501961}%
\pgfsetstrokecolor{currentstroke}%
\pgfsetdash{{1.850000pt}{0.800000pt}}{0.000000pt}%
\pgfpathmoveto{\pgfqpoint{1.584472in}{0.331542in}}%
\pgfpathlineto{\pgfqpoint{3.119722in}{1.866792in}}%
\pgfpathlineto{\pgfqpoint{3.119722in}{1.866792in}}%
\pgfusepath{stroke}%
\end{pgfscope}%
\begin{pgfscope}%
\pgfpathrectangle{\pgfqpoint{0.198611in}{0.333208in}}{\pgfqpoint{2.921111in}{1.533583in}} %
\pgfusepath{clip}%
\pgfsetbuttcap%
\pgfsetroundjoin%
\pgfsetlinewidth{0.501875pt}%
\definecolor{currentstroke}{rgb}{0.501961,0.501961,0.501961}%
\pgfsetstrokecolor{currentstroke}%
\pgfsetdash{{1.850000pt}{0.800000pt}}{0.000000pt}%
\pgfpathmoveto{\pgfqpoint{0.198611in}{1.866792in}}%
\pgfpathlineto{\pgfqpoint{1.733861in}{0.331542in}}%
\pgfpathlineto{\pgfqpoint{1.733861in}{0.331542in}}%
\pgfusepath{stroke}%
\end{pgfscope}%
\begin{pgfscope}%
\pgfsetrectcap%
\pgfsetmiterjoin%
\pgfsetlinewidth{0.501875pt}%
\definecolor{currentstroke}{rgb}{0.000000,0.000000,0.000000}%
\pgfsetstrokecolor{currentstroke}%
\pgfsetdash{}{0pt}%
\pgfpathmoveto{\pgfqpoint{1.659167in}{0.333208in}}%
\pgfpathlineto{\pgfqpoint{1.659167in}{1.866792in}}%
\pgfusepath{stroke}%
\end{pgfscope}%
\begin{pgfscope}%
\pgfsetrectcap%
\pgfsetmiterjoin%
\pgfsetlinewidth{0.501875pt}%
\definecolor{currentstroke}{rgb}{0.000000,0.000000,0.000000}%
\pgfsetstrokecolor{currentstroke}%
\pgfsetdash{}{0pt}%
\pgfpathmoveto{\pgfqpoint{0.198611in}{0.406236in}}%
\pgfpathlineto{\pgfqpoint{3.119722in}{0.406236in}}%
\pgfusepath{stroke}%
\end{pgfscope}%
\begin{pgfscope}%
\pgfsetroundcap%
\pgfsetroundjoin%
\pgfsetlinewidth{0.501875pt}%
\definecolor{currentstroke}{rgb}{0.000000,0.000000,0.000000}%
\pgfsetstrokecolor{currentstroke}%
\pgfsetdash{}{0pt}%
\pgfpathmoveto{\pgfqpoint{1.659167in}{1.872920in}}%
\pgfpathquadraticcurveto{\pgfqpoint{1.659167in}{1.873738in}}{\pgfqpoint{1.659167in}{1.866792in}}%
\pgfusepath{stroke}%
\end{pgfscope}%
\begin{pgfscope}%
\pgfsetroundcap%
\pgfsetroundjoin%
\pgfsetlinewidth{0.501875pt}%
\definecolor{currentstroke}{rgb}{0.000000,0.000000,0.000000}%
\pgfsetstrokecolor{currentstroke}%
\pgfsetdash{}{0pt}%
\pgfpathmoveto{\pgfqpoint{1.631389in}{1.817364in}}%
\pgfpathlineto{\pgfqpoint{1.659167in}{1.872920in}}%
\pgfpathlineto{\pgfqpoint{1.686944in}{1.817364in}}%
\pgfusepath{stroke}%
\end{pgfscope}%
\begin{pgfscope}%
\pgftext[x=1.659167in,y=1.936236in,,bottom]{\rmfamily\fontsize{10.000000}{12.000000}\selectfont \(\displaystyle \omega\)}%
\end{pgfscope}%
\begin{pgfscope}%
\pgfsetroundcap%
\pgfsetroundjoin%
\pgfsetlinewidth{0.501875pt}%
\definecolor{currentstroke}{rgb}{0.000000,0.000000,0.000000}%
\pgfsetstrokecolor{currentstroke}%
\pgfsetdash{}{0pt}%
\pgfpathmoveto{\pgfqpoint{3.125847in}{0.406236in}}%
\pgfpathquadraticcurveto{\pgfqpoint{3.126667in}{0.406236in}}{\pgfqpoint{3.119722in}{0.406236in}}%
\pgfusepath{stroke}%
\end{pgfscope}%
\begin{pgfscope}%
\pgfsetroundcap%
\pgfsetroundjoin%
\pgfsetlinewidth{0.501875pt}%
\definecolor{currentstroke}{rgb}{0.000000,0.000000,0.000000}%
\pgfsetstrokecolor{currentstroke}%
\pgfsetdash{}{0pt}%
\pgfpathmoveto{\pgfqpoint{3.070292in}{0.434014in}}%
\pgfpathlineto{\pgfqpoint{3.125847in}{0.406236in}}%
\pgfpathlineto{\pgfqpoint{3.070292in}{0.378458in}}%
\pgfusepath{stroke}%
\end{pgfscope}%
\begin{pgfscope}%
\pgftext[x=3.189167in,y=0.406236in,left,]{\rmfamily\fontsize{10.000000}{12.000000}\selectfont \(\displaystyle k\)}%
\end{pgfscope}%
\begin{pgfscope}%
\pgftext[x=2.316417in,y=0.990458in,left,base]{\rmfamily\fontsize{10.000000}{12.000000}\selectfont \(\displaystyle supp~(\hat\Delta_m)\)}%
\end{pgfscope}%
\begin{pgfscope}%
\pgftext[x=1.367056in,y=1.136514in,left,base]{\rmfamily\fontsize{10.000000}{12.000000}\selectfont \(\displaystyle \hat\Delta_m^{\circledast 2}\)}%
\end{pgfscope}%
\begin{pgfscope}%
\pgfpathrectangle{\pgfqpoint{3.302292in}{0.435000in}}{\pgfqpoint{0.063333in}{1.330000in}} %
\pgfusepath{clip}%
\pgfsetbuttcap%
\pgfsetmiterjoin%
\definecolor{currentfill}{rgb}{1.000000,1.000000,1.000000}%
\pgfsetfillcolor{currentfill}%
\pgfsetlinewidth{0.010037pt}%
\definecolor{currentstroke}{rgb}{1.000000,1.000000,1.000000}%
\pgfsetstrokecolor{currentstroke}%
\pgfsetdash{}{0pt}%
\pgfpathmoveto{\pgfqpoint{3.302292in}{0.435000in}}%
\pgfpathlineto{\pgfqpoint{3.302292in}{0.439948in}}%
\pgfpathlineto{\pgfqpoint{3.302292in}{1.701667in}}%
\pgfpathlineto{\pgfqpoint{3.333958in}{1.765000in}}%
\pgfpathlineto{\pgfqpoint{3.333958in}{1.765000in}}%
\pgfpathlineto{\pgfqpoint{3.365625in}{1.701667in}}%
\pgfpathlineto{\pgfqpoint{3.365625in}{0.439948in}}%
\pgfpathlineto{\pgfqpoint{3.365625in}{0.435000in}}%
\pgfpathclose%
\pgfusepath{stroke,fill}%
\end{pgfscope}%
\begin{pgfscope}%
\pgfsys@transformshift{3.301667in}{0.435000in}%
\pgftext[left,bottom]{\pgfimage[interpolate=true,width=0.063333in,height=1.330000in]{delta_m2_twisted-img1.png}}%
\end{pgfscope}%
\begin{pgfscope}%
\pgfsetbuttcap%
\pgfsetroundjoin%
\definecolor{currentfill}{rgb}{0.000000,0.000000,0.000000}%
\pgfsetfillcolor{currentfill}%
\pgfsetlinewidth{0.803000pt}%
\definecolor{currentstroke}{rgb}{0.000000,0.000000,0.000000}%
\pgfsetstrokecolor{currentstroke}%
\pgfsetdash{}{0pt}%
\pgfsys@defobject{currentmarker}{\pgfqpoint{0.000000in}{0.000000in}}{\pgfqpoint{0.048611in}{0.000000in}}{%
\pgfpathmoveto{\pgfqpoint{0.000000in}{0.000000in}}%
\pgfpathlineto{\pgfqpoint{0.048611in}{0.000000in}}%
\pgfusepath{stroke,fill}%
}%
\begin{pgfscope}%
\pgfsys@transformshift{3.365625in}{0.435000in}%
\pgfsys@useobject{currentmarker}{}%
\end{pgfscope}%
\end{pgfscope}%
\begin{pgfscope}%
\pgftext[x=3.462847in,y=0.387172in,left,base]{\rmfamily\fontsize{10.000000}{12.000000}\selectfont \(\displaystyle -4\)}%
\end{pgfscope}%
\begin{pgfscope}%
\pgfsetbuttcap%
\pgfsetroundjoin%
\definecolor{currentfill}{rgb}{0.000000,0.000000,0.000000}%
\pgfsetfillcolor{currentfill}%
\pgfsetlinewidth{0.803000pt}%
\definecolor{currentstroke}{rgb}{0.000000,0.000000,0.000000}%
\pgfsetstrokecolor{currentstroke}%
\pgfsetdash{}{0pt}%
\pgfsys@defobject{currentmarker}{\pgfqpoint{0.000000in}{0.000000in}}{\pgfqpoint{0.048611in}{0.000000in}}{%
\pgfpathmoveto{\pgfqpoint{0.000000in}{0.000000in}}%
\pgfpathlineto{\pgfqpoint{0.048611in}{0.000000in}}%
\pgfusepath{stroke,fill}%
}%
\begin{pgfscope}%
\pgfsys@transformshift{3.365625in}{0.593333in}%
\pgfsys@useobject{currentmarker}{}%
\end{pgfscope}%
\end{pgfscope}%
\begin{pgfscope}%
\pgftext[x=3.462847in,y=0.545506in,left,base]{\rmfamily\fontsize{10.000000}{12.000000}\selectfont \(\displaystyle -3\)}%
\end{pgfscope}%
\begin{pgfscope}%
\pgfsetbuttcap%
\pgfsetroundjoin%
\definecolor{currentfill}{rgb}{0.000000,0.000000,0.000000}%
\pgfsetfillcolor{currentfill}%
\pgfsetlinewidth{0.803000pt}%
\definecolor{currentstroke}{rgb}{0.000000,0.000000,0.000000}%
\pgfsetstrokecolor{currentstroke}%
\pgfsetdash{}{0pt}%
\pgfsys@defobject{currentmarker}{\pgfqpoint{0.000000in}{0.000000in}}{\pgfqpoint{0.048611in}{0.000000in}}{%
\pgfpathmoveto{\pgfqpoint{0.000000in}{0.000000in}}%
\pgfpathlineto{\pgfqpoint{0.048611in}{0.000000in}}%
\pgfusepath{stroke,fill}%
}%
\begin{pgfscope}%
\pgfsys@transformshift{3.365625in}{0.751667in}%
\pgfsys@useobject{currentmarker}{}%
\end{pgfscope}%
\end{pgfscope}%
\begin{pgfscope}%
\pgftext[x=3.462847in,y=0.703839in,left,base]{\rmfamily\fontsize{10.000000}{12.000000}\selectfont \(\displaystyle -2\)}%
\end{pgfscope}%
\begin{pgfscope}%
\pgfsetbuttcap%
\pgfsetroundjoin%
\definecolor{currentfill}{rgb}{0.000000,0.000000,0.000000}%
\pgfsetfillcolor{currentfill}%
\pgfsetlinewidth{0.803000pt}%
\definecolor{currentstroke}{rgb}{0.000000,0.000000,0.000000}%
\pgfsetstrokecolor{currentstroke}%
\pgfsetdash{}{0pt}%
\pgfsys@defobject{currentmarker}{\pgfqpoint{0.000000in}{0.000000in}}{\pgfqpoint{0.048611in}{0.000000in}}{%
\pgfpathmoveto{\pgfqpoint{0.000000in}{0.000000in}}%
\pgfpathlineto{\pgfqpoint{0.048611in}{0.000000in}}%
\pgfusepath{stroke,fill}%
}%
\begin{pgfscope}%
\pgfsys@transformshift{3.365625in}{0.910000in}%
\pgfsys@useobject{currentmarker}{}%
\end{pgfscope}%
\end{pgfscope}%
\begin{pgfscope}%
\pgftext[x=3.462847in,y=0.862172in,left,base]{\rmfamily\fontsize{10.000000}{12.000000}\selectfont \(\displaystyle -1\)}%
\end{pgfscope}%
\begin{pgfscope}%
\pgfsetbuttcap%
\pgfsetroundjoin%
\definecolor{currentfill}{rgb}{0.000000,0.000000,0.000000}%
\pgfsetfillcolor{currentfill}%
\pgfsetlinewidth{0.803000pt}%
\definecolor{currentstroke}{rgb}{0.000000,0.000000,0.000000}%
\pgfsetstrokecolor{currentstroke}%
\pgfsetdash{}{0pt}%
\pgfsys@defobject{currentmarker}{\pgfqpoint{0.000000in}{0.000000in}}{\pgfqpoint{0.048611in}{0.000000in}}{%
\pgfpathmoveto{\pgfqpoint{0.000000in}{0.000000in}}%
\pgfpathlineto{\pgfqpoint{0.048611in}{0.000000in}}%
\pgfusepath{stroke,fill}%
}%
\begin{pgfscope}%
\pgfsys@transformshift{3.365625in}{1.068333in}%
\pgfsys@useobject{currentmarker}{}%
\end{pgfscope}%
\end{pgfscope}%
\begin{pgfscope}%
\pgftext[x=3.462847in,y=1.020506in,left,base]{\rmfamily\fontsize{10.000000}{12.000000}\selectfont \(\displaystyle 0\)}%
\end{pgfscope}%
\begin{pgfscope}%
\pgfsetbuttcap%
\pgfsetroundjoin%
\definecolor{currentfill}{rgb}{0.000000,0.000000,0.000000}%
\pgfsetfillcolor{currentfill}%
\pgfsetlinewidth{0.803000pt}%
\definecolor{currentstroke}{rgb}{0.000000,0.000000,0.000000}%
\pgfsetstrokecolor{currentstroke}%
\pgfsetdash{}{0pt}%
\pgfsys@defobject{currentmarker}{\pgfqpoint{0.000000in}{0.000000in}}{\pgfqpoint{0.048611in}{0.000000in}}{%
\pgfpathmoveto{\pgfqpoint{0.000000in}{0.000000in}}%
\pgfpathlineto{\pgfqpoint{0.048611in}{0.000000in}}%
\pgfusepath{stroke,fill}%
}%
\begin{pgfscope}%
\pgfsys@transformshift{3.365625in}{1.226667in}%
\pgfsys@useobject{currentmarker}{}%
\end{pgfscope}%
\end{pgfscope}%
\begin{pgfscope}%
\pgftext[x=3.462847in,y=1.178839in,left,base]{\rmfamily\fontsize{10.000000}{12.000000}\selectfont \(\displaystyle 1\)}%
\end{pgfscope}%
\begin{pgfscope}%
\pgfsetbuttcap%
\pgfsetroundjoin%
\definecolor{currentfill}{rgb}{0.000000,0.000000,0.000000}%
\pgfsetfillcolor{currentfill}%
\pgfsetlinewidth{0.803000pt}%
\definecolor{currentstroke}{rgb}{0.000000,0.000000,0.000000}%
\pgfsetstrokecolor{currentstroke}%
\pgfsetdash{}{0pt}%
\pgfsys@defobject{currentmarker}{\pgfqpoint{0.000000in}{0.000000in}}{\pgfqpoint{0.048611in}{0.000000in}}{%
\pgfpathmoveto{\pgfqpoint{0.000000in}{0.000000in}}%
\pgfpathlineto{\pgfqpoint{0.048611in}{0.000000in}}%
\pgfusepath{stroke,fill}%
}%
\begin{pgfscope}%
\pgfsys@transformshift{3.365625in}{1.385000in}%
\pgfsys@useobject{currentmarker}{}%
\end{pgfscope}%
\end{pgfscope}%
\begin{pgfscope}%
\pgftext[x=3.462847in,y=1.337172in,left,base]{\rmfamily\fontsize{10.000000}{12.000000}\selectfont \(\displaystyle 2\)}%
\end{pgfscope}%
\begin{pgfscope}%
\pgfsetbuttcap%
\pgfsetroundjoin%
\definecolor{currentfill}{rgb}{0.000000,0.000000,0.000000}%
\pgfsetfillcolor{currentfill}%
\pgfsetlinewidth{0.803000pt}%
\definecolor{currentstroke}{rgb}{0.000000,0.000000,0.000000}%
\pgfsetstrokecolor{currentstroke}%
\pgfsetdash{}{0pt}%
\pgfsys@defobject{currentmarker}{\pgfqpoint{0.000000in}{0.000000in}}{\pgfqpoint{0.048611in}{0.000000in}}{%
\pgfpathmoveto{\pgfqpoint{0.000000in}{0.000000in}}%
\pgfpathlineto{\pgfqpoint{0.048611in}{0.000000in}}%
\pgfusepath{stroke,fill}%
}%
\begin{pgfscope}%
\pgfsys@transformshift{3.365625in}{1.543333in}%
\pgfsys@useobject{currentmarker}{}%
\end{pgfscope}%
\end{pgfscope}%
\begin{pgfscope}%
\pgftext[x=3.462847in,y=1.495506in,left,base]{\rmfamily\fontsize{10.000000}{12.000000}\selectfont \(\displaystyle 3\)}%
\end{pgfscope}%
\begin{pgfscope}%
\pgfsetbuttcap%
\pgfsetroundjoin%
\definecolor{currentfill}{rgb}{0.000000,0.000000,0.000000}%
\pgfsetfillcolor{currentfill}%
\pgfsetlinewidth{0.803000pt}%
\definecolor{currentstroke}{rgb}{0.000000,0.000000,0.000000}%
\pgfsetstrokecolor{currentstroke}%
\pgfsetdash{}{0pt}%
\pgfsys@defobject{currentmarker}{\pgfqpoint{0.000000in}{0.000000in}}{\pgfqpoint{0.048611in}{0.000000in}}{%
\pgfpathmoveto{\pgfqpoint{0.000000in}{0.000000in}}%
\pgfpathlineto{\pgfqpoint{0.048611in}{0.000000in}}%
\pgfusepath{stroke,fill}%
}%
\begin{pgfscope}%
\pgfsys@transformshift{3.365625in}{1.701667in}%
\pgfsys@useobject{currentmarker}{}%
\end{pgfscope}%
\end{pgfscope}%
\begin{pgfscope}%
\pgftext[x=3.462847in,y=1.653839in,left,base]{\rmfamily\fontsize{10.000000}{12.000000}\selectfont \(\displaystyle 4\)}%
\end{pgfscope}%
\begin{pgfscope}%
\pgfsetbuttcap%
\pgfsetmiterjoin%
\pgfsetlinewidth{0.501875pt}%
\definecolor{currentstroke}{rgb}{0.000000,0.000000,0.000000}%
\pgfsetstrokecolor{currentstroke}%
\pgfsetdash{}{0pt}%
\pgfpathmoveto{\pgfqpoint{3.302292in}{0.435000in}}%
\pgfpathlineto{\pgfqpoint{3.302292in}{0.439948in}}%
\pgfpathlineto{\pgfqpoint{3.302292in}{1.701667in}}%
\pgfpathlineto{\pgfqpoint{3.333958in}{1.765000in}}%
\pgfpathlineto{\pgfqpoint{3.333958in}{1.765000in}}%
\pgfpathlineto{\pgfqpoint{3.365625in}{1.701667in}}%
\pgfpathlineto{\pgfqpoint{3.365625in}{0.439948in}}%
\pgfpathlineto{\pgfqpoint{3.365625in}{0.435000in}}%
\pgfpathclose%
\pgfusepath{stroke}%
\end{pgfscope}%
\end{pgfpicture}%
\makeatother%
\endgroup%
} %
        \caption{Plot von $\hat\Delta_m^{\circledast 2}$ und $\hat\Delta_m$. Wieder liegt der Träger von $\hat\Delta_m^{\circledast 2}$ in der kausalen Zukunft.
        }
        \label{fig:delta_2m_twisted}
    \end{minipage}\hfill
    \begin{minipage}{0.45\textwidth}
        \centering
        \resizebox{\textwidth}{!}{%% Creator: Matplotlib, PGF backend
%%
%% To include the figure in your LaTeX document, write
%%   \input{<filename>.pgf}
%%
%% Make sure the required packages are loaded in your preamble
%%   \usepackage{pgf}
%%
%% Figures using additional raster images can only be included by \input if
%% they are in the same directory as the main LaTeX file. For loading figures
%% from other directories you can use the `import` package
%%   \usepackage{import}
%% and then include the figures with
%%   \import{<path to file>}{<filename>.pgf}
%%
%% Matplotlib used the following preamble
%%   \usepackage[utf8x]{inputenc}
%%   \usepackage[T1]{fontenc}
%%   \usepackage{amssymb}
%%
\begingroup%
\makeatletter%
\begin{pgfpicture}%
\pgfpathrectangle{\pgfpointorigin}{\pgfqpoint{4.000000in}{2.200000in}}%
\pgfusepath{use as bounding box, clip}%
\begin{pgfscope}%
\pgfsetbuttcap%
\pgfsetmiterjoin%
\definecolor{currentfill}{rgb}{1.000000,1.000000,1.000000}%
\pgfsetfillcolor{currentfill}%
\pgfsetlinewidth{0.000000pt}%
\definecolor{currentstroke}{rgb}{1.000000,1.000000,1.000000}%
\pgfsetstrokecolor{currentstroke}%
\pgfsetdash{}{0pt}%
\pgfpathmoveto{\pgfqpoint{0.000000in}{0.000000in}}%
\pgfpathlineto{\pgfqpoint{4.000000in}{0.000000in}}%
\pgfpathlineto{\pgfqpoint{4.000000in}{2.200000in}}%
\pgfpathlineto{\pgfqpoint{0.000000in}{2.200000in}}%
\pgfpathclose%
\pgfusepath{fill}%
\end{pgfscope}%
\begin{pgfscope}%
\pgfsetbuttcap%
\pgfsetmiterjoin%
\definecolor{currentfill}{rgb}{1.000000,1.000000,1.000000}%
\pgfsetfillcolor{currentfill}%
\pgfsetlinewidth{0.000000pt}%
\definecolor{currentstroke}{rgb}{0.000000,0.000000,0.000000}%
\pgfsetstrokecolor{currentstroke}%
\pgfsetstrokeopacity{0.000000}%
\pgfsetdash{}{0pt}%
\pgfpathmoveto{\pgfqpoint{0.198611in}{0.198611in}}%
\pgfpathlineto{\pgfqpoint{3.801389in}{0.198611in}}%
\pgfpathlineto{\pgfqpoint{3.801389in}{2.001389in}}%
\pgfpathlineto{\pgfqpoint{0.198611in}{2.001389in}}%
\pgfpathclose%
\pgfusepath{fill}%
\end{pgfscope}%
\begin{pgfscope}%
\pgfpathrectangle{\pgfqpoint{0.198611in}{0.198611in}}{\pgfqpoint{3.602778in}{1.802778in}} %
\pgfusepath{clip}%
\pgfsetbuttcap%
\pgfsetroundjoin%
\pgfsetlinewidth{0.501875pt}%
\definecolor{currentstroke}{rgb}{0.501961,0.501961,0.501961}%
\pgfsetstrokecolor{currentstroke}%
\pgfsetdash{{1.850000pt}{0.800000pt}}{0.000000pt}%
\pgfpathmoveto{\pgfqpoint{0.507421in}{0.198611in}}%
\pgfpathlineto{\pgfqpoint{0.507421in}{2.001389in}}%
\pgfusepath{stroke}%
\end{pgfscope}%
\begin{pgfscope}%
\pgfpathrectangle{\pgfqpoint{0.198611in}{0.198611in}}{\pgfqpoint{3.602778in}{1.802778in}} %
\pgfusepath{clip}%
\pgfsetrectcap%
\pgfsetroundjoin%
\pgfsetlinewidth{1.003750pt}%
\definecolor{currentstroke}{rgb}{0.894118,0.101961,0.109804}%
\pgfsetstrokecolor{currentstroke}%
\pgfsetdash{}{0pt}%
\pgfpathmoveto{\pgfqpoint{0.508967in}{2.015278in}}%
\pgfpathlineto{\pgfqpoint{0.512367in}{1.239514in}}%
\pgfpathlineto{\pgfqpoint{0.516488in}{1.075048in}}%
\pgfpathlineto{\pgfqpoint{0.520610in}{0.997750in}}%
\pgfpathlineto{\pgfqpoint{0.528853in}{0.919550in}}%
\pgfpathlineto{\pgfqpoint{0.537096in}{0.878538in}}%
\pgfpathlineto{\pgfqpoint{0.545339in}{0.852596in}}%
\pgfpathlineto{\pgfqpoint{0.557704in}{0.827014in}}%
\pgfpathlineto{\pgfqpoint{0.570069in}{0.809537in}}%
\pgfpathlineto{\pgfqpoint{0.586555in}{0.792473in}}%
\pgfpathlineto{\pgfqpoint{0.607163in}{0.776035in}}%
\pgfpathlineto{\pgfqpoint{0.644257in}{0.751554in}}%
\pgfpathlineto{\pgfqpoint{0.722567in}{0.700702in}}%
\pgfpathlineto{\pgfqpoint{0.767905in}{0.667245in}}%
\pgfpathlineto{\pgfqpoint{0.813242in}{0.630138in}}%
\pgfpathlineto{\pgfqpoint{0.866823in}{0.582041in}}%
\pgfpathlineto{\pgfqpoint{0.936889in}{0.514374in}}%
\pgfpathlineto{\pgfqpoint{1.044051in}{0.410719in}}%
\pgfpathlineto{\pgfqpoint{1.089388in}{0.371484in}}%
\pgfpathlineto{\pgfqpoint{1.126482in}{0.343447in}}%
\pgfpathlineto{\pgfqpoint{1.159455in}{0.322565in}}%
\pgfpathlineto{\pgfqpoint{1.188306in}{0.308037in}}%
\pgfpathlineto{\pgfqpoint{1.213035in}{0.298752in}}%
\pgfpathlineto{\pgfqpoint{1.237765in}{0.292685in}}%
\pgfpathlineto{\pgfqpoint{1.262494in}{0.290085in}}%
\pgfpathlineto{\pgfqpoint{1.283102in}{0.290717in}}%
\pgfpathlineto{\pgfqpoint{1.303710in}{0.293999in}}%
\pgfpathlineto{\pgfqpoint{1.324318in}{0.300000in}}%
\pgfpathlineto{\pgfqpoint{1.344926in}{0.308762in}}%
\pgfpathlineto{\pgfqpoint{1.369656in}{0.322934in}}%
\pgfpathlineto{\pgfqpoint{1.394385in}{0.341047in}}%
\pgfpathlineto{\pgfqpoint{1.419115in}{0.362977in}}%
\pgfpathlineto{\pgfqpoint{1.447966in}{0.393105in}}%
\pgfpathlineto{\pgfqpoint{1.476817in}{0.427669in}}%
\pgfpathlineto{\pgfqpoint{1.509789in}{0.471785in}}%
\pgfpathlineto{\pgfqpoint{1.555127in}{0.538306in}}%
\pgfpathlineto{\pgfqpoint{1.670531in}{0.711358in}}%
\pgfpathlineto{\pgfqpoint{1.703504in}{0.753514in}}%
\pgfpathlineto{\pgfqpoint{1.728233in}{0.780752in}}%
\pgfpathlineto{\pgfqpoint{1.752963in}{0.803368in}}%
\pgfpathlineto{\pgfqpoint{1.773571in}{0.818170in}}%
\pgfpathlineto{\pgfqpoint{1.794179in}{0.828905in}}%
\pgfpathlineto{\pgfqpoint{1.810665in}{0.834357in}}%
\pgfpathlineto{\pgfqpoint{1.827151in}{0.836879in}}%
\pgfpathlineto{\pgfqpoint{1.843637in}{0.836372in}}%
\pgfpathlineto{\pgfqpoint{1.860124in}{0.832770in}}%
\pgfpathlineto{\pgfqpoint{1.876610in}{0.826041in}}%
\pgfpathlineto{\pgfqpoint{1.893096in}{0.816188in}}%
\pgfpathlineto{\pgfqpoint{1.909583in}{0.803257in}}%
\pgfpathlineto{\pgfqpoint{1.930191in}{0.782894in}}%
\pgfpathlineto{\pgfqpoint{1.950799in}{0.758109in}}%
\pgfpathlineto{\pgfqpoint{1.975528in}{0.723038in}}%
\pgfpathlineto{\pgfqpoint{2.004379in}{0.675812in}}%
\pgfpathlineto{\pgfqpoint{2.037352in}{0.615530in}}%
\pgfpathlineto{\pgfqpoint{2.156878in}{0.389578in}}%
\pgfpathlineto{\pgfqpoint{2.181607in}{0.351914in}}%
\pgfpathlineto{\pgfqpoint{2.202215in}{0.325446in}}%
\pgfpathlineto{\pgfqpoint{2.222823in}{0.304249in}}%
\pgfpathlineto{\pgfqpoint{2.239309in}{0.291526in}}%
\pgfpathlineto{\pgfqpoint{2.255796in}{0.282869in}}%
\pgfpathlineto{\pgfqpoint{2.272282in}{0.278496in}}%
\pgfpathlineto{\pgfqpoint{2.284647in}{0.278124in}}%
\pgfpathlineto{\pgfqpoint{2.297011in}{0.280291in}}%
\pgfpathlineto{\pgfqpoint{2.309376in}{0.285014in}}%
\pgfpathlineto{\pgfqpoint{2.325863in}{0.295273in}}%
\pgfpathlineto{\pgfqpoint{2.342349in}{0.309979in}}%
\pgfpathlineto{\pgfqpoint{2.358835in}{0.328969in}}%
\pgfpathlineto{\pgfqpoint{2.379443in}{0.358351in}}%
\pgfpathlineto{\pgfqpoint{2.400051in}{0.393393in}}%
\pgfpathlineto{\pgfqpoint{2.424781in}{0.441728in}}%
\pgfpathlineto{\pgfqpoint{2.457753in}{0.513890in}}%
\pgfpathlineto{\pgfqpoint{2.544306in}{0.708885in}}%
\pgfpathlineto{\pgfqpoint{2.569036in}{0.755540in}}%
\pgfpathlineto{\pgfqpoint{2.589644in}{0.788173in}}%
\pgfpathlineto{\pgfqpoint{2.606130in}{0.809348in}}%
\pgfpathlineto{\pgfqpoint{2.622616in}{0.825585in}}%
\pgfpathlineto{\pgfqpoint{2.634981in}{0.834264in}}%
\pgfpathlineto{\pgfqpoint{2.647346in}{0.839787in}}%
\pgfpathlineto{\pgfqpoint{2.659711in}{0.842047in}}%
\pgfpathlineto{\pgfqpoint{2.672075in}{0.840976in}}%
\pgfpathlineto{\pgfqpoint{2.684440in}{0.836543in}}%
\pgfpathlineto{\pgfqpoint{2.696805in}{0.828761in}}%
\pgfpathlineto{\pgfqpoint{2.709170in}{0.817680in}}%
\pgfpathlineto{\pgfqpoint{2.725656in}{0.797945in}}%
\pgfpathlineto{\pgfqpoint{2.742142in}{0.772858in}}%
\pgfpathlineto{\pgfqpoint{2.762750in}{0.734702in}}%
\pgfpathlineto{\pgfqpoint{2.787480in}{0.680551in}}%
\pgfpathlineto{\pgfqpoint{2.816331in}{0.609133in}}%
\pgfpathlineto{\pgfqpoint{2.894641in}{0.409492in}}%
\pgfpathlineto{\pgfqpoint{2.919370in}{0.357397in}}%
\pgfpathlineto{\pgfqpoint{2.935857in}{0.328508in}}%
\pgfpathlineto{\pgfqpoint{2.952343in}{0.305241in}}%
\pgfpathlineto{\pgfqpoint{2.964708in}{0.291893in}}%
\pgfpathlineto{\pgfqpoint{2.977073in}{0.282325in}}%
\pgfpathlineto{\pgfqpoint{2.989437in}{0.276720in}}%
\pgfpathlineto{\pgfqpoint{3.001802in}{0.275204in}}%
\pgfpathlineto{\pgfqpoint{3.014167in}{0.277847in}}%
\pgfpathlineto{\pgfqpoint{3.026531in}{0.284657in}}%
\pgfpathlineto{\pgfqpoint{3.038896in}{0.295579in}}%
\pgfpathlineto{\pgfqpoint{3.051261in}{0.310497in}}%
\pgfpathlineto{\pgfqpoint{3.067747in}{0.336280in}}%
\pgfpathlineto{\pgfqpoint{3.084234in}{0.368236in}}%
\pgfpathlineto{\pgfqpoint{3.104842in}{0.415650in}}%
\pgfpathlineto{\pgfqpoint{3.129571in}{0.480943in}}%
\pgfpathlineto{\pgfqpoint{3.224367in}{0.742097in}}%
\pgfpathlineto{\pgfqpoint{3.244975in}{0.784575in}}%
\pgfpathlineto{\pgfqpoint{3.261462in}{0.811228in}}%
\pgfpathlineto{\pgfqpoint{3.273826in}{0.826335in}}%
\pgfpathlineto{\pgfqpoint{3.286191in}{0.836913in}}%
\pgfpathlineto{\pgfqpoint{3.298556in}{0.842731in}}%
\pgfpathlineto{\pgfqpoint{3.306799in}{0.843886in}}%
\pgfpathlineto{\pgfqpoint{3.315042in}{0.842833in}}%
\pgfpathlineto{\pgfqpoint{3.323285in}{0.839564in}}%
\pgfpathlineto{\pgfqpoint{3.335650in}{0.830533in}}%
\pgfpathlineto{\pgfqpoint{3.348015in}{0.816653in}}%
\pgfpathlineto{\pgfqpoint{3.360380in}{0.798122in}}%
\pgfpathlineto{\pgfqpoint{3.376866in}{0.766686in}}%
\pgfpathlineto{\pgfqpoint{3.393352in}{0.728412in}}%
\pgfpathlineto{\pgfqpoint{3.413960in}{0.672732in}}%
\pgfpathlineto{\pgfqpoint{3.442811in}{0.585136in}}%
\pgfpathlineto{\pgfqpoint{3.496392in}{0.420707in}}%
\pgfpathlineto{\pgfqpoint{3.517000in}{0.367010in}}%
\pgfpathlineto{\pgfqpoint{3.533486in}{0.331208in}}%
\pgfpathlineto{\pgfqpoint{3.549972in}{0.303232in}}%
\pgfpathlineto{\pgfqpoint{3.562337in}{0.288053in}}%
\pgfpathlineto{\pgfqpoint{3.574702in}{0.278251in}}%
\pgfpathlineto{\pgfqpoint{3.582945in}{0.274831in}}%
\pgfpathlineto{\pgfqpoint{3.591188in}{0.273960in}}%
\pgfpathlineto{\pgfqpoint{3.599431in}{0.275663in}}%
\pgfpathlineto{\pgfqpoint{3.607675in}{0.279941in}}%
\pgfpathlineto{\pgfqpoint{3.620039in}{0.291130in}}%
\pgfpathlineto{\pgfqpoint{3.632404in}{0.307881in}}%
\pgfpathlineto{\pgfqpoint{3.644769in}{0.329898in}}%
\pgfpathlineto{\pgfqpoint{3.661255in}{0.366721in}}%
\pgfpathlineto{\pgfqpoint{3.681863in}{0.422910in}}%
\pgfpathlineto{\pgfqpoint{3.706593in}{0.501009in}}%
\pgfpathlineto{\pgfqpoint{3.772538in}{0.716534in}}%
\pgfpathlineto{\pgfqpoint{3.793146in}{0.770446in}}%
\pgfpathlineto{\pgfqpoint{3.801389in}{0.788581in}}%
\pgfpathlineto{\pgfqpoint{3.801389in}{0.788581in}}%
\pgfusepath{stroke}%
\end{pgfscope}%
\begin{pgfscope}%
\pgfpathrectangle{\pgfqpoint{0.198611in}{0.198611in}}{\pgfqpoint{3.602778in}{1.802778in}} %
\pgfusepath{clip}%
\pgfsetrectcap%
\pgfsetroundjoin%
\pgfsetlinewidth{1.003750pt}%
\definecolor{currentstroke}{rgb}{0.894118,0.101961,0.109804}%
\pgfsetstrokecolor{currentstroke}%
\pgfsetdash{}{0pt}%
\pgfpathmoveto{\pgfqpoint{0.184722in}{0.559167in}}%
\pgfpathlineto{\pgfqpoint{0.324422in}{0.559167in}}%
\pgfpathlineto{\pgfqpoint{0.507421in}{0.559167in}}%
\pgfusepath{stroke}%
\end{pgfscope}%
\begin{pgfscope}%
\pgfpathrectangle{\pgfqpoint{0.198611in}{0.198611in}}{\pgfqpoint{3.602778in}{1.802778in}} %
\pgfusepath{clip}%
\pgfsetbuttcap%
\pgfsetroundjoin%
\pgfsetlinewidth{0.501875pt}%
\definecolor{currentstroke}{rgb}{0.501961,0.501961,0.501961}%
\pgfsetstrokecolor{currentstroke}%
\pgfsetdash{{1.850000pt}{0.800000pt}}{0.000000pt}%
\pgfpathmoveto{\pgfqpoint{0.184722in}{0.847611in}}%
\pgfpathlineto{\pgfqpoint{3.815278in}{0.847611in}}%
\pgfusepath{stroke}%
\end{pgfscope}%
\begin{pgfscope}%
\pgfsetrectcap%
\pgfsetmiterjoin%
\pgfsetlinewidth{0.501875pt}%
\definecolor{currentstroke}{rgb}{0.000000,0.000000,0.000000}%
\pgfsetstrokecolor{currentstroke}%
\pgfsetdash{}{0pt}%
\pgfpathmoveto{\pgfqpoint{0.301548in}{0.198611in}}%
\pgfpathlineto{\pgfqpoint{0.301548in}{2.001389in}}%
\pgfusepath{stroke}%
\end{pgfscope}%
\begin{pgfscope}%
\pgfsetrectcap%
\pgfsetmiterjoin%
\pgfsetlinewidth{0.501875pt}%
\definecolor{currentstroke}{rgb}{0.000000,0.000000,0.000000}%
\pgfsetstrokecolor{currentstroke}%
\pgfsetdash{}{0pt}%
\pgfpathmoveto{\pgfqpoint{0.198611in}{0.559167in}}%
\pgfpathlineto{\pgfqpoint{3.801389in}{0.559167in}}%
\pgfusepath{stroke}%
\end{pgfscope}%
\begin{pgfscope}%
\pgftext[x=0.548595in,y=0.342833in,left,base]{\rmfamily\fontsize{10.000000}{12.000000}\selectfont \(\displaystyle \omega = 2 m\)}%
\end{pgfscope}%
\begin{pgfscope}%
\pgftext[x=0.548595in,y=1.136056in,left,base]{\rmfamily\fontsize{10.000000}{12.000000}\selectfont \(\displaystyle \approx \frac{1}{\sqrt{\omega}}\)}%
\end{pgfscope}%
\begin{pgfscope}%
\pgftext[x=2.566151in,y=0.876456in,left,base]{\rmfamily\fontsize{10.000000}{12.000000}\selectfont \(\displaystyle \approx 2 \cos\left(\frac{\omega^2}{2}\right)\)}%
\end{pgfscope}%
\begin{pgfscope}%
\pgfsetroundcap%
\pgfsetroundjoin%
\pgfsetlinewidth{0.501875pt}%
\definecolor{currentstroke}{rgb}{0.000000,0.000000,0.000000}%
\pgfsetstrokecolor{currentstroke}%
\pgfsetdash{}{0pt}%
\pgfpathmoveto{\pgfqpoint{0.301548in}{2.007506in}}%
\pgfpathquadraticcurveto{\pgfqpoint{0.301548in}{2.008330in}}{\pgfqpoint{0.301548in}{2.001389in}}%
\pgfusepath{stroke}%
\end{pgfscope}%
\begin{pgfscope}%
\pgfsetroundcap%
\pgfsetroundjoin%
\pgfsetlinewidth{0.501875pt}%
\definecolor{currentstroke}{rgb}{0.000000,0.000000,0.000000}%
\pgfsetstrokecolor{currentstroke}%
\pgfsetdash{}{0pt}%
\pgfpathmoveto{\pgfqpoint{0.273770in}{1.951951in}}%
\pgfpathlineto{\pgfqpoint{0.301548in}{2.007506in}}%
\pgfpathlineto{\pgfqpoint{0.329325in}{1.951951in}}%
\pgfusepath{stroke}%
\end{pgfscope}%
\begin{pgfscope}%
\pgftext[x=0.301548in,y=2.070833in,,bottom]{\rmfamily\fontsize{10.000000}{12.000000}\selectfont \(\displaystyle \hat\Delta^{\circledast 2} ~(\omega, 0)\)}%
\end{pgfscope}%
\begin{pgfscope}%
\pgfsetroundcap%
\pgfsetroundjoin%
\pgfsetlinewidth{0.501875pt}%
\definecolor{currentstroke}{rgb}{0.000000,0.000000,0.000000}%
\pgfsetstrokecolor{currentstroke}%
\pgfsetdash{}{0pt}%
\pgfpathmoveto{\pgfqpoint{3.807500in}{0.559167in}}%
\pgfpathquadraticcurveto{\pgfqpoint{3.808327in}{0.559167in}}{\pgfqpoint{3.801389in}{0.559167in}}%
\pgfusepath{stroke}%
\end{pgfscope}%
\begin{pgfscope}%
\pgfsetroundcap%
\pgfsetroundjoin%
\pgfsetlinewidth{0.501875pt}%
\definecolor{currentstroke}{rgb}{0.000000,0.000000,0.000000}%
\pgfsetstrokecolor{currentstroke}%
\pgfsetdash{}{0pt}%
\pgfpathmoveto{\pgfqpoint{3.751945in}{0.586944in}}%
\pgfpathlineto{\pgfqpoint{3.807500in}{0.559167in}}%
\pgfpathlineto{\pgfqpoint{3.751945in}{0.531389in}}%
\pgfusepath{stroke}%
\end{pgfscope}%
\begin{pgfscope}%
\pgftext[x=3.870833in,y=0.559167in,left,]{\rmfamily\fontsize{10.000000}{12.000000}\selectfont \(\displaystyle \omega\)}%
\end{pgfscope}%
\end{pgfpicture}%
\makeatother%
\endgroup%
}
        \caption{Plot von $\left.\hat{\Delta}_m^{\circledast 2}\right|_{k=0}$ um das asymptotische Verhalten für $\omega \rightarrow 0$ und $\omega \rightarrow \infty$ zu verdeutlichen}
        \label{fig:delta_2m_twisted_k0}
    \end{minipage}
\end{figure}

\subsubsection*{Fall $|s| > 1$}
Wir bedienen uns wieder genau des selben Arguments, wie in \cref{eq:delta_m2_s>1} und dürfen direkt schreiben:

\begin{equation}
    \left\langle \rwhat{\Delta}_m^{\circledast 2}, \hat\psi_{ast}\right\rangle
    = 0 \condition{für alle $a$ klein genug}
\label{eq:delta_m2_twisted_s>1}
\end{equation}


\subsubsection*{Fall $|s| < 1, (x,t) \neq 0$}
Da
$\rwhat\Delta_m^{\circledast 2} = \rwhat\Delta_m^{* 2} \cos(\dots)$ können wir direkt mit dem Ausdruck (\ref{eq:psi_ast_delta_m2_s<1}) $\cdot \cos$ weiter arbeiten und genau die selben Abschätzungen machen. $\cos(\varphi)$ ist bekanntermaßen beschränkt.

\begin{dmath}
    \left\langle \rwhat{\Delta}_m^{\circledast 2}, \rwhat{\psi}_{ast}
    \right\rangle
    =
     2 a^{-\frac{3}{4}} \int \frac{
    \hat\psi_1(\omega)~ \hat\psi_2(k) \left(
    \omega^2 \left(\Delta s - 2 a^{\frac{1}{2}} k s - ak^2
            \right) - 3a^2m^2
    \right)
     }
     {
        \sqrt{\Delta s -2a^{\frac{1}{2}}ks - ak^2}
            \sqrt{\Delta s \omega^2 -2a^{\frac{1}{2}} \omega^2 k s
                    - a\omega^2k^2-4 a^2 m^2}
     }
     \cdot
     \Theta(\cdots)
     \cos(\varphi(\omega^2-k^2))
     e^{-i \omega \left(\frac{t'-sx'}{a}+k \frac{x'}{\sqrt{a}}\right)}
     \d \omega \d k
     \leq
     2 a^{-\frac{3}{4}} \int
     \omega \hat\psi_1(\omega)\, \hat\psi_2(k)
     e^{-i \omega \left(\frac{t'-sx'}{a}+k \frac{x'}{\sqrt{a}}\right)}
     \d \omega \d k
     \sim O(a^k) ~~ \forall k \hiderel \in \mathbb{N}
\label{eq:delta_m2_twisted_s<1_x_neq_0}
\end{dmath}


\subsubsection*{Fall $|s| < 1, (x,t) = 0$}
In diesem Fall lassen wir den $\cos$-Faktor in \cref{eq:delta_m2_twisted_s<1_x_neq_0} in der ersten Ungleichung nicht heraus fallen, dafür wird der $e^\cdots$-Faktor 1. Den $\cos$-Faktor schreiben wir als Summe von $e$-Funktionen und erhalten

% \begin{equation}
% \begin{aligned}
%     \left\langle \rwhat{\Delta}_m^{\circledast 2}, \rwhat{\psi}_{ast}
%     \right\rangle
%     \\&=
%     2 a^{-\frac{3}{4}} \int
%     \omega \hat\psi_1(\omega) \hat\psi_2(k)
%     \left\{
%         \exp\left(i a^{-2} \frac{
%         \omega^2 (\Delta s - 2 a^{\frac{1}{2}} k s - a k^2)
%         }{2}
%         \sqrt{\frac{1}{4} - \frac{a^2 m^2}{\omega^2(
%             \Delta s - 2 a^{\frac{1}{2}} k s - ak^2
%         )}}
%         \right)
%         \\ &+
%         \exp (-i \cdots)
%     \right\}
%     \d \omega \d k
%     \\ &=
%     2 a^{-\frac{3}{4}} \int
%     \cancel{\sqrt{\omega}} \hat\psi_1(\sqrt{\omega}) \hat\psi_2(k)
%     \left\{
%         \exp\left(i a^{-2} \frac{
%         \omega (\Delta s - 2 a^{\frac{1}{2}} k s - a k^2)
%         }{2}
%         \sqrt{\frac{1}{4} - \frac{a^2 m^2}{\omega(
%             \Delta s - 2 a^{\frac{1}{2}} k s - ak^2
%         )}}
%         \right)
%         \\ &+ \mathrm{c.c.}
%     \right\}
%     \frac{{\d \omega \d k}}{\cancel{\sqrt{\omega}}}
%     \\ &=
%     2 a^{-\frac{3}{4}} \int \left\{
%         \int
%         \hat\psi_1(\sqrt\omega)
%         \left\{
%             \exp
%             \left(ia^{-2} \left(\frac{\omega \Delta s}{4}
%                                 + O\left(a^{\frac{1}{2}}\right)\right)
%             \right)
%             \\ &+
%             \mathrm{c.c.}
%         \right\}
%         \d \omega
%     \right\}
%     \hat \psi_2(k) \d k
%     \\ &=
%     2 a^{-\frac{3}{4}} \int
%     \underbrace{
%     \left\{
%     (\hat\psi_1 \circ \sqrt{\cdot })^\vee
%     \left(\frac{\Delta s}{4a^2}\right)
%      + (\hat\psi_1 \circ \sqrt{\cdot })^\vee
%     \left(-\frac{\Delta s}{4a^2}\right)
%     + \mathrm{c.c.}
%     \right\}}_{
%     \sim O(a^k) ~\forall k \in \mathbb{N}
%     }
%     \psi_2(k) \d k
%     \\ &
%     \sim O(a^k) ~~ \forall k \hiderel \in \mathbb{N}
% \label{eq:delta_m2_twisted_s<1_x=0}
% \end{aligned}
% \end{equation}
\begin{dmath}
    \left\langle \rwhat{\Delta}_m^{\circledast 2}, \rwhat{\psi}_{ast}
    \right\rangle
    =
    2 a^{-\frac{3}{4}} \int
    \omega \hat\psi_1(\omega) \hat\psi_2(k)
    \left\{
        \exp\left(i a^{-2} \frac{
        \omega^2 (\Delta s - 2 a^{\frac{1}{2}} k s - a k^2)
        }{2}
        \sqrt{\frac{1}{4} - \frac{a^2 m^2}{\omega^2(
            \Delta s - 2 a^{\frac{1}{2}} k s - ak^2
        )}}
        \right)
        +\exp (-i \cdots)
    \right\}
    \d \omega \d k
    =
    2 a^{-\frac{3}{4}} \int
    \cancel{\sqrt{\omega}} \hat\psi_1(\sqrt{\omega}) \hat\psi_2(k)
    \left\{
        \exp\left(i a^{-2} \frac{
        \omega (\Delta s - 2 a^{\frac{1}{2}} k s - a k^2)
        }{2}
        \sqrt{\frac{1}{4} - \frac{a^2 m^2}{\omega(
            \Delta s - 2 a^{\frac{1}{2}} k s - ak^2
        )}}
        \right)
        + \mathrm{c.c.}
    \right\}
    \frac{{\d \omega \d k}}{\cancel{\sqrt{\omega}}}
    =
    2 a^{-\frac{3}{4}} \int \left\{
        \int
        \hat\psi_1(\sqrt\omega)
        \left\{
            \exp
            \left(ia^{-2} \left(\frac{\omega \Delta s}{4}
                                + O\left(a^{\frac{1}{2}}\right)\right)
            \right)
            + \mathrm{c.c.}
        \right\}
        \d \omega
    \right\}
    \hat \psi_2(k) \d k
    =
    2 a^{-\frac{3}{4}} \int
    \underbrace{
    \left\{
    (\hat\psi_1 \circ \sqrt{\cdot })^\vee
    \left(\frac{\Delta s}{4a^2}\right)
     + (\hat\psi_1 \circ \sqrt{\cdot })^\vee
    \left(-\frac{\Delta s}{4a^2}\right)
    + \mathrm{c.c.}
    \right\}}_{
    \sim O(a^k) ~\forall k \in \mathbb{N}
    }
    \psi_2(k) \d k
    \sim O(a^k) ~~ \forall k \hiderel \in \mathbb{N}
\label{eq:delta_m2_twisted_s<1_x=0}
\end{dmath}

wobei bei der Substition $\omega \to \sqrt{\omega}$ in der zweiten Zeile wichtig ist, dass $0 \notin supp (\hat\psi_1)$, also auch nach der Substitution noch $\hat\psi_1 \in C_c^\infty (\mathbb{R})$ ist.


\subsubsection*{Fall $s = -1$}

Da $\rwhat\Delta_m^{\circledast 2} = \rwhat\Delta_m^{* 2} \cos(\dots)$ ist, haben wir bis auf den $\cos$-Faktor die selben Analysis zu betreiben, wie für $\rwhat\Delta_m^{*2}$.

\begin{align}
    & \kern -2em\left\langle \rwhat{\Delta}_m^{\circledast 2}, \rwhat{\psi}_{a-1t}
    \right\rangle
    \nonumber \\ &=
    2 a^{-\frac{3}{4}} \int
    \underbrace{\frac{
        \hat\psi_1(\omega) \hat\psi_2(k'+k_0)
        \left(
        2\omega^2(k'+k_0)-a^{\frac{1}{2}}\omega^2(k'+k_0)^2-a^{\frac{3}{2}}3m^2
        \right)
        \Theta(k')
    }
    {
        \sqrt{k'} \sqrt{k'+k_0}
        \sqrt{2-a^{\frac{1}{2}}(k'+k_0)}
        \sqrt{-a^{\frac{1}{2}}\omega^2\left(k'-\tfrac{2\sqrt{\omega^2-4a^2m^2}}
                    {\sqrt a \omega}\right)}
    }}_{i)}
    \nonumber \\ & \kern 2em\cdot
    \cos
    \underbrace{\left(
        \frac{2\omega^2(k'+k_0)-a^{\frac{1}{2}}\omega^2(k'+k_0)^2}
             {2 a^{\frac{3}{2}}}
        \sqrt{
            \frac{1}{4}
            + \frac{a^{\frac{3}{2}} m^2}
                   {2\omega^2(k'+k_0)-a^{\frac{1}{2}}\omega^2(k'+k_0)^2}
        }
    \right)}_{ii)}
    \nonumber \\ & \kern 2em\cdot
    e^{-i\omega\left(\frac{t'+x'}{a}+\frac{(k'+k_0)x'}{\sqrt a}\right)}
    \d \omega \d k'
\label{eq:psi_a-1t_delta_m2_twisted}
\end{align}
% \begin{dmath}
%     \left\langle \rwhat{\Delta}_m^{\circledast 2}, \rwhat{\psi}_{a-1t}
%     \right\rangle
%     =
%     2 a^{-\frac{3}{4}} \int
%     \underbrace{\frac{
%         \hat\psi_1(\omega) \hat\psi_2(k'+k_0)
%         \left(
%         2\omega^2(k'+k_0)-a^{\frac{1}{2}}\omega^2(k'+k_0)^2-a^{\frac{3}{2}}3m^2
%         \right)
%         \Theta(k')
%     }
%     {
%         \sqrt{k'} \sqrt{k'+k_0}
%         \sqrt{2-a^{\frac{1}{2}}(k'+k_0)}
%         \sqrt{-a^{\frac{1}{2}}\omega^2\left(k'-\tfrac{2\sqrt{\omega^2-4a^2m^2}}
%                     {\sqrt a \omega}\right)}
%     }}_{i)}
%     \cdot
%     \cos
%     \underbrace{\left(
%         \frac{2\omega^2(k'+k_0)-a^{\frac{1}{2}}\omega^2(k'+k_0)^2}
%              {2 a^{\frac{3}{2}}}
%         \sqrt{
%             \frac{1}{4}
%             + \frac{a^{\frac{3}{2}} m^2}
%                    {2\omega^2(k'+k_0)-a^{\frac{1}{2}}\omega^2(k'+k_0)^2}
%         }
%     \right)}_{ii)}
%     \cdot
%     e^{-i\omega\left(\frac{t'+x'}{a}+\frac{(k'+k_0)x'}{\sqrt a}\right)}
%     \d \omega \d k'
% \label{eq:psi_a-1t_delta_m2_twisted}
% \end{dmath}

Genau wie in \cref{eq:lange_1/sqrt_abschaetzerei} können wir für $i)$ wieder abschätzen\footnote{Da cos beschränkt ist, spielt er bei den Abschätzungen keine Rolle}

\begin{dmath*}
    % \frac{
    %     \hat\psi_1(\omega) \hat\psi_2(k'+k_0)
    %     \left(
    %     2\omega^2(k'+k_0)-a^{\frac{1}{2}}\omega^2(k'+k_0)^2-a^{\frac{3}{2}}3m^2
    %     \right)
    %     \Theta(k')
    % }
    % {
    %     \sqrt{k'} \sqrt{k'+k_0}
    %     \sqrt{2-a^{\frac{1}{2}}(k'+k_0)}
    %     \sqrt{-a^{\frac{1}{2}}\omega^2\left(k'-\tfrac{2\sqrt{\omega^2-4a^2m^2}}
    %                 {\sqrt a \omega}\right)}
    % }
    i)
    \leq
    \frac{\textrm{const}}{\sqrt{k'}} \Theta(k')
\end{dmath*}

Damit dürfen wir wieder Lebesgue anwenden, um den Grenzwert $a \to 0$ des Integrals zu berechnen.
Des Weiteren ist analog zu \cref{eq:langer_sqrt_bruch_punktweise_konvergenz}

\begin{dmath}
    i)
    \stackrel{\textrm{\scriptsize punktweise f.ü.}}{\longrightarrow}
    \omega\,\hat\psi_1(\omega) \,\hat\psi_2(k') \Theta(k').
\label{eq:delta_m2_twisted_s=-1_material_a}
\end{dmath}

Widmen wir uns also dem Argument des Kosinus $ii)$:

\begin{dmath*}
    \frac{2\omega^2(k'+k_0)-a^{\frac{1}{2}}\omega^2(k'+k_0)^2}
         {2 a^{\frac{3}{2}}}
    \sqrt{
        \frac{1}{4}
        + \frac{a^{\frac{3}{2}} m^2}
               {2\omega^2(k'+k_0)-a^{\frac{1}{2}}\omega^2(k'+k_0)^2}
    }
    =
    \frac{\omega^2 (k'+k'0)(2-a^{\frac{1}{2}}(k_+k_0))}
         {2 a^{\frac{3}{2}}}
    \sqrt{
        \frac{1}{4}
        + \frac{a^{\frac{3}{2}}m^2}
               {\omega^2(k'+k_0)(a^{\frac{1}{2}}(k'+k_0)-2)}
    }\\
    \stackrel{\textrm{\scriptsize punktweise, außer } k'=0}{\longrightarrow}
    \frac{\omega^2 k' a^{-\frac{3}{2}}}{2}
\end{dmath*}

\begin{dmath}
\Longrightarrow ~~
    \cos\big(ii)\big)
    \stackrel{\textrm{\scriptsize punktweise, außer } k'=0}{\longrightarrow}
    \cos\left(\frac{\omega^2 k' a^{-\frac{3}{2}}}{2}\right)
\label{eq:delta_m2_twisted_s=-1_material_b}
\end{dmath}

Einsetzen von \cref{eq:delta_m2_twisted_s=-1_material_a,eq:delta_m2_twisted_s=-1_material_b} in \cref{eq:psi_a-1t_delta_m2_twisted} ergibt mit Lebesgue

\begin{dmath*}
    \lim_{a \to 0}
    \left\langle \rwhat{\Delta}_m^{\circledast 2}, \rwhat{\psi}_{a-1t}
    \right\rangle
    =
    2 a^{-\frac{3}{4}}
    \int \omega \,\hat\psi_1(\omega) \,\hat\psi_2(k')
         \cos\left(a^{-\frac{3}{2}}\frac{\omega^2 k'}{2}\right)
         e^{-i\omega k' \frac{x'}{\sqrt a}}
         e^{-i \omega \frac{t'+x'}{a}}
         \d \omega \d k'
    =
    a^{-\frac{3}{4}} \int
    \underbrace{\left\{
            \int \hat\psi_2(k')\Theta(k')
            \left(e^{i a^{-\frac{3}{2}}\frac{\omega^2 k'}{2}} + e^{-i\cdots}\right)
            e^{-i\omega k' \frac{x'}{\sqrt a}}
            \d k'
        \right\}}_{=: \hat f_a(\omega)}
    \cdot
    \omega \hat\psi_1(\omega)e^{-i\omega \frac{t'+x'}{a}} \d \omega
\end{dmath*}

Nun betrachten wir $\hat f_a(\omega)$ und erhalten analog zu
\cref{eq:faltung_mit_1/x_rechnung}

\begin{align*}
    \hat f_a(\omega)
    &=
    \int \hat\psi_2(k')\Theta(k')
    \left(
        e^{i a^{-\frac{3}{2}}\left(\frac{\omega^2 k'}{2} + O(a^1)\right)}
        + e^{-i \cdots}
    \right) \d k
    \\ &\stackrel{a \to 0}{\longrightarrow}
    \int \hat\psi_2(k')\Theta(k')
    \left(
        e^{i a^{-\frac{3}{2}}\left(\frac{\omega^2 k'}{2}\right)}
        + \mathrm{c.c}
    \right) \d k'
    \\ &=
    \bigg[
        \underbrace{
            \psi_2\left(-\frac{\omega^2}{2 a^{\frac{3}{2}}}\right)
        }_{O(a^k) \; \forall k \in \mathbb{N}}
        + \underbrace{i
            \underbrace{\left(\psi_2 * \mathcal{P}(1/x)\right)}_{O(x^{-1})}
            \left(-\frac{\omega^2}{2 a^{\frac{3}{2}}}\right)
        }_{
            O\left(\left(-\frac{\omega^2}{2 a^{\frac{3}{2}}}\right)^{-1}\right)
            = O\left(a^{\frac{3}{2}}\right)
           }
     + \mathrm{c.c}
     \bigg]
     \\ &\sim
    O\left(a^{\frac{3}{2}}\right).
\end{align*}
% \begin{dmath*}
%     \hat f_a(\omega)
%     =
%     \int \hat\psi_2(k')\Theta(k')
%     \left(
%         e^{i a^{-\frac{3}{2}}\left(\frac{\omega^2 k'}{2} + O(a^1)\right)}
%         + e^{-i \cdots}
%     \right) \d k
%     \\
%     \stackrel{a \to 0}{\longrightarrow}
%     \int \hat\psi_2(k')\Theta(k')
%     \left(
%         e^{i a^{-\frac{3}{2}}\left(\frac{\omega^2 k'}{2}\right)}
%         + e^{-i \cdots}
%     \right) \d k'
%     =
%     \bigg[
%         \underbrace{
%             \psi_2\left(-\frac{\omega^2}{2 a^{\frac{3}{2}}}\right)
%         }_{O(a^k) \; \forall k \in \mathbb{N}}
%         + \underbrace{i
%             \underbrace{\left(\psi_2 * \mathcal{P}(1/x)\right)}_{O(x^{-1})}
%             \left(-\frac{\omega^2}{2 a^{\frac{3}{2}}}\right)
%         }_{
%             O\left(\left(-\frac{\omega^2}{2 a^{\frac{3}{2}}}\right)^{-1}\right)
%             = O\left(a^{\frac{3}{2}}\right)
%            }
%      + (\textrm{anderer Term})
%      \bigg]
%      \sim O\left(a^{\frac{3}{2}}\right)
% \end{dmath*}

Also dürfen wir für $a \to 0$ schreiben $\hat f_a(\omega) = C a^{\frac{3}{2}} + o\left(a^{\frac{3}{2}}\right)$ und landen bei

\begin{align}
    \lim_{a \to 0}
    \left\langle \rwhat{\Delta}_m^{\circledast 2}, \rwhat{\psi}_{a-1t}
    \right\rangle
    &=
    a^{-\frac{3}{4}} \int C a^{\frac{3}{2}} \omega \hat\psi_1(\omega)
    e^{-i\omega \frac{t'+x'}{a}}
    \d \omega
    \nonumber \\ &\sim O\left(a^{\frac{3}{4}}\right) \condition{falls $t'=-x'$}
    \nonumber \\ &\sim O\left(a^k\right) ~~ \forall k \hiderel \in \mathbb{N}
                              \condition{sonst}
\label{eq:delta_m2_twisted_s=-1}
\end{align}
% \begin{dmath}
%     \lim_{a \to 0}
%     \left\langle \rwhat{\Delta}_m^{\circledast 2}, \rwhat{\psi}_{a-1t}
%     \right\rangle
%     =
%     a^{-\frac{3}{4}} \int C a^{\frac{3}{2}} \omega \hat\psi_1(\omega)
%     e^{-i\omega \frac{t'+x'}{a}}
%     \d \omega
%     \sim O\left(a^{\frac{3}{4}}\right) \condition{falls $t'=-x'$}
%     \sim O\left(a^k\right) ~~ \forall k \hiderel \in \mathbb{N}
%                               \condition{sonst}
% \label{eq:delta_m2_twisted_s=-1}
% \end{dmath}

\subsection{Zusammenfassung und Vergleich der Ergebnisse}
Fassen wir die Ergebnisse aus \cref{eq:delta_m2_twisted_s>1,eq:delta_m2_twisted_s<1_x_neq_0,eq:delta_m2_twisted_s<1_x=0,eq:delta_m2_twisted_s=-1} wieder in einer Übersichtstabelle zusammen:

\begin{table}[h]
\centering
\begin{tabular}{l|cccc}
        & $(t',x') = 0$     & $t'=x' \neq 0$    & $t'=-x' \neq 0$   & $t' \neq \pm x'$ \\ \hline
$s=1$   & $a^{\frac{3}{4}}$ & $a^{\frac{3}{4}}$ & $a^k$             & $a^k$            \\
$s=-1$  & $a^{\frac{3}{4}}$ & $a^k$             & $a^{\frac{3}{4}}$ & $a^k$            \\
$|s|<1$ & $a^k$             & $a^k$             & $a^k$             & $a^k$            \\
$|s|>1$ & $a^k$             & $a^k$             & $a^k$             & $a^k$
\end{tabular}
\caption{Konvergenzordnung von $\mathcal{S}_{\Delta_m^{\star 2}}(a,s,(t',x'))$ im Limit $a \to 0$ für alle interessanten Kombinationen von $s$ und $(t',x')$}
\label{tab:wavefrontset_delta_m2_twisted}
\end{table}

Auch diesmal stimmen die Ergebnisse mit denen von \textcite[Prop. 3.72]{Schulz2014}\footnote{So weit sie gegeben wurden} überein, welcher für alle Potenzen des getwisteten Produkts $\Delta_m^{\star k}$ erhält:

\begin{equation*}
\left\langle t,x; \omega, k \right\rangle \hiderel\in WF(\Delta_m^{\star k})
\Rightarrow
- \omega \geq |k|
\end{equation*}
% section dots_und_nun_zur_wellenfrontmenge_von_ (end)


% section die_wellenfrontmenge_von_ (end)



%%%%%%%%%%%%%%%%%%%%%%%%%%%%%%%%%%%%%%%%%%%%%%%%%%%%%%%%%%%%%%%%%%%%%%%%%%%%%%%%
% % Section 3
%%%%%%%%%%%%%%%%%%%%%%%%%%%%%%%%%%%%%%%%%%%%%%%%%%%%%%%%%%%%%%%%%%%%%%%%%%%%%%%%
\section{Berechnen von $WF(G_F)$} % (fold)
\label{sec:berechnen_von_}

\subsection{Ausdrücke für $\left< \psi_{ast}, G_F\right>$} % (fold)
\label{sec:psiast_gf}

Ab jetzt werden wir der Namenskonvention der Physiker in der SRT folgen und unsere
Ortsraumvariablen mit $x = (t, x)$ und unsere Impulsraumvariablen mit $\xi = (\omega, k)$
bezeichnen sowie konsequenterweise das Minkowskiskalarprodukt $x \cdot \xi = \omega t - k x$
verwenden. Des weiteren wird der Verschiebungsparameter mit  $t = (t', x')$ bezeichnet.

Die massive skalare Zweipunktfunktion bzw. der Feynmanpropagator im Impulsraum ist dann
gegeben durch (\textcite{Schwartz2014}, (6.34))

\begin{equation}
\label{eq:gf}
    \hat G_F(\omega, k) = \frac{1}{m^2 - \omega^2 + k^2 - i 0^+}
\end{equation}

Setzen wir dies in unsere Ausdrücke für $\left< \psi_{ast}, f\right>$ aus \eqref{eq:psi_ast_f_1}
bzw. \eqref{eq:psi_ast_f_2} ergibt sich, unter Verwendung des Minkowskiskalaprodukts,

\begin{align}
\left< \hat\psi_{ast}, \hat G_F \right> &=
    \int \hat \psi_{ast}(\omega, t) ~\hat G_F(\omega, t) ~\d \omega \d k
    \nonumber \\
    &=
    a^{\frac{3}{4}} \iint \frac{
        \hat \psi_1 (a \omega)
        ~\hat \psi_2 \left(a^{-\frac{1}{2}}\frac{k}{\omega} - s\right)
        ~ e^{-i\omega t' + i k x'}
    }
    {
        m^2 - \omega^2 + k^2 - i 0^+
    }
    \d \omega \d k
    \nonumber \\
    &=
    a^{-\frac{3}{4}} \iint \frac{
        \hat\psi_1(\omega)
        ~\hat \psi_2\left(\tfrac{k}{w}\right)
        ~e^{-i \omega \frac{t' - sx'}{a} + ik \frac{x'}{\sqrt{a}}}
    }
    {
        m^2 - \left(\frac{\omega}{a}\right)^2
        + \left(\frac{\omega s}{a} + \frac{k}{\sqrt{a}}\right)^2 - i0^+
    }
    \d \omega \d k \nonumber \\
    &=
    a^{-\frac{3}{4}}
    \kern -2em \iint
    \limits_{
    \substack{
        \omega \in [-2, -\frac{1}{2}]\cup[\frac{1}{2},2] \\
        \left|\frac{k}{2}-s\right| \leq \sqrt{ax}
        }
    }
    \kern -1.5em
    \frac{
        \hat\psi_1(\omega)
        ~\hat \psi_2\left(\tfrac{k}{\omega}\right)
        ~e^{-i \omega \frac{t' - sx'}{a} + ik \frac{x'}{\sqrt{a}}}
    }
    {
        m^2 + a^{-2} \omega^2 (s^2 - 1) + a^{-\frac{3}{2}} 2 s \omega k + a^{-1} k  - i0^+
    }
    \d \omega \d k
    \label{eq:psi_ast_gf_1}
\end{align}

\todo{Integral hübsch machen. Größeres Integralzeichen?}

und mit der anderen Substitution analog

\begin{align}
    \left< \hat\psi_{ast}, \hat G_F \right>
    &=
    a^{-\frac{3}{4}}
    \kern -1em
    \iint \limits_{\substack{
        |\omega|~ \in~ [\frac{1}{2},2] \\
        k ~\in~ [-1,1]
        }
    }
    \kern -1em
    \frac{
        \omega ~\hat \psi_1(\omega) ~\hat \psi_2(k)
        e^{-i \omega \left(\frac{t' - sx'}{a} + \frac{kx'}{\sqrt{a}}\right)}
    }
    {
        m^2 - \omega^2(a^{-2}(1-s^2)-a^{-1}k^2 - 2 k s a^{-\frac{3}{2}})
    }
    \d \omega \d k
    \label{eq:psi_ast_gf_2}
\end{align}

wobei sich die Integrationsbereiche aus den Forderungen an den Träger von $\psi$
(vgl. \eqref{eq:supp_psi}) ergeben.




Nach Satz \eqref{thm:main_theorem} genügt es zu bestimmen, an welchen Punkten
$(t', x')$ und in welche Richtungen $s$ $\mathcal{S}_f(a,s,(t',x'))$ nicht schnell-fallend
in $a^{-1}$ ist, um die Wellenfrontmenge zu bestimmen. Da wir keine explizite
erzeugende Funktion $\psi$ angegeben haben, werden wir uns dabei Argumente bedienen,
die alleine auf den allgemeinen Eigenschaften von $\psi_{ast}$ beruhen, aber nicht
einer expliziten Form.
\todo{In Textform beschreiben, was die grobe Strategie ist, also wie der Integrand
vernünfitg vereinfacht wird und welche Eigenschaften von¸$\psi$ wie eingehen.}
\todo{Hier schon die Ergebnisse als Satz angeben, und dann Beweis hinschreiben?}
\todo{Bemerkung einfügen, warum dass auch ziemlich unmöglich ist}

Das allgemeine Vorgehen wird dabei folgendes sein: Die Ausdrücke in \eqref{eq:psi_ast_gf_1}
und \eqref{eq:psi_ast_gf_2} genau anstarren, um zu sehen für welche Werte von
$(t',x')$ und $s$ potentiell interessante Dinge geschehen, also z.B. Terme im Nenner
weg fallen, oder die Phase konstant wird. Dann werden diese Werte von $(t',x')$ und
$s$ eingesetzt und alles so weit vereinfacht und genähert -- im Rahmen des Erlaubten, ohne
das Verhalten für $a \rightarrow 0$ zu ändern --, bis die $a$-Abhängigkeit abgelesen
werden kann. Entscheidende Zutaten sind dabei der beschränkte Träger von $\hat \psi$
und der schnelle Abfall von $\psi$.


\subsubsection*{Fall $s=1, t' = 0 = x'$}
Nach \eqref{eq:psi_ast_gf_2} erhalten wir mit $s=1, t' = 0 = x'$

\begin{align*}
    \left< \hat\psi_{a10}, \hat G_F \right>
    &=
    \int a^{-\frac{3}{4}} \frac{
        \omega ~\hat \psi_1(\omega) ~\hat \psi_2(k)
    }
    {
        m^2+\omega^2 (a^{-1}k^2 + a^{-\frac{3}{2}}2 k )
    }
    \d \omega \d k \\
    &=
    \int a^{\frac{3}{4}} \frac{
        \omega ~\hat \psi_1(\omega) ~\hat \psi_2(k)
    }
    {
        a^\frac{3}{2} m^2+\omega^2 (a^\frac{1}{2}k^2 + 2 k )
    }
    \d \omega \d k
\end{align*}

Da aber $|\omega| \in [\frac{1}{2},2]$ und $k \in [-1,1]$ ist, ist für hinreichend
kleine $a$ (und für genau die interessieren wir uns ja)

\begin{equation*}
    \left|
        \frac{\omega ~\hat \psi_1(\omega) ~\hat \psi_2(k)}{k \omega^2}
    \right|
    \geq
    \left|
        \frac{\omega ~\hat \psi_1(\omega) ~\hat \psi_2(k)}
        {a^\frac{3}{2}m^2+a^\frac{1}{2}\omega^2 k+2k \omega^2}
    \right|
\end{equation*}

eine integrierbare (im Sinne des Cauchy-Hauptwertes) Majorante für den Integranden.

\todo{Warum ist Cauchy-Hauptwert hier erlaubt? Weiter ausführe, warum es diese Majorante tut?}

Wir dürfen uns also des Lebesgueschen Konvergenzsatzes bedienen und schreiben

\begin{equation}
    \lim_{a \rightarrow 0} \left< \hat\psi_{a10}, \hat G_F \right> =
    a^\frac{3}{4} \int \frac{
    \omega ~\hat \psi_1(\omega) ~\hat \psi_2(k)
    }
    {
    2k \omega^2
    }
    \d \omega \d k
    \sim O(a^\frac{3}{4})
\end{equation}

Für $s = -1$ erhalten wir genau das selbe Ergebniss, da ja der $\omega^2 (1-s^2)$-Term
im Nenner genauso wieder verschwindet.

\subsubsection*{Fall $s \neq \pm 1, t' = 0 = x'$}
In diesem Fall verschwindet der $\omega^2 (1-s^2)$-Term im Nenner nicht und
dementsprechend folgt

\begin{align*}
    \left< \hat\psi_{as0}, \hat G_F \right>
    &=
    \int a^{-\frac{3}{4}} \frac{
        \omega ~\hat \psi_1(\omega) ~\hat \psi_2(k)
    }
    {
        m^2-\omega^2 ((1-s^2) - a^{-1}k^2 - a^{-\frac{3}{2}}2 k )
    }
    \d \omega \d k \\
    &=
    \int a^{\frac{5}{4}} \frac{
        \omega ~\hat \psi_1(\omega) ~\hat \psi_2(k)
    }
    {
        a^2 m^2+\omega^2 (s^2-1) + a \omega^2 k^2 + a^\frac{1}{2}2 \omega^2 k s
    }
    \d \omega \d k
\end{align*}

Analog zum vorigen Teil ist, diesmal sogar ohne den Cauchy-Hauptwert bemühen zu
müssen,

\todo{Überall wo es sein muss $\lim_{a \rightarrow 0}$ dazu schreiben, oder sagen
dass der Limit überall impliziert ist}

\begin{equation*}
    \left|
        \frac{2 \omega ~\hat \psi_1(\omega) ~\hat \psi_2(k)}{\omega^2 (1-s^2)}
    \right|
    \geq
    \left|
        \frac{
        \omega ~\hat \psi_1(\omega) ~\hat \psi_2(k)
    }
    {
        a^2 m^2+\omega^2 (s^2-1) + a \omega^2 k^2 + a^\frac{1}{2}2 \omega^2 k s
    }
    \right|
\end{equation*}

dass eine integrierbare Majorante ist (in der Tat ja sogar in $C_c^\infty (\mathbb{R}^2)$)
Damit können wir folgende Abschätzung treffen:

\begin{equation*}
    \lim_{a \rightarrow 0} \left< \hat\psi_{as0}, \hat G_F \right> =
    a^\frac{5}{4} \int \frac{2 \omega ~\hat \psi_1(\omega) ~\hat \psi_2(k)}
    {\omega^2 (1-s^2)}
    \d \omega \d k
    \sim O(a^\frac{5}{4})
\end{equation*}


\subsubsection*{Fall $s \neq \pm 1, (t', s') \neq 0$}
In diesem Fall benutzen wir wieder die erste Substitution \eqref{eq:psi_ast_gf_1}
und klammern wie schon in den beiden vorigen Teilen die höchste negative
Potenz von $a$ im Nenner aus.

\begin{align}
\Rightarrow ~
    \left< \hat\psi_{ast}, \hat G_F \right>
    &=
    a^\frac{5}{4} \int \frac{
        \hat \psi_1 (\omega) ~\hat \psi_2 \left(\frac{k}{\omega}\right)
        ~ e^{-i \omega \left(\frac{t'-sx'}{a}\right) + i k \frac{x'}{\sqrt{a}}}
    }
    {
        a^2 m^2 - \omega^2 (1-s^2) + a^\frac{1}{2} s \omega k +a k^2
    }
    \d \omega \d k
\end{align}

und da immer noch $0 \notin supp(\psi_1)$ gilt ist ein weiteres mal eine integrierbare Majorante gegeben durch

\begin{equation}
    2\frac{\hat \psi_1 (\omega)~\hat\psi_2 \left(\frac{k}{\omega}\right)}
    {\omega^2(s^2-1)}
\end{equation}

In der Tat ist sogar

\begin{equation}
    \hat f(\omega, k) := \frac{\hat \psi_1 (\omega)~\hat\psi_2 \left(\frac{k}{\omega}\right)}
    {\omega^2(s^2-1)}
    \in C_c^\infty (\hat{\mathbb{R}}^2)
\end{equation}

da $\psi_1$ und $\psi_2$ getragen sind. Demnach ist die Fourierinverse von
$\hat f$, $f := \mathcal{F}^{-1}(\hat f) \in \mathcal{S}(\mathbb{R}^2)$, also schnell
fallend. Damit können wir schließlich abschätzen

\begin{align}
    \left| \left< \hat\psi_{ast}, \hat G_F \right> \right|
    &=
    a^\frac{5}{4} \left|  \int \hat f(\omega, k)
    ~e^{-i \omega \left(\frac{t'-sx'}{a}\right) + ik \frac{x'}{\sqrt{a}}}
    \d \omega \d k
    \right|
    \nonumber \\
    &=
    a^\frac{5}{4} \left| f \left(\frac{t'-sx}{a}, \frac{x'}{\sqrt{a}}\right) \right|
    \leq
    a^\frac{5}{4} C_k\left(
    1 + \left\lVert \substack{(t'-sx')/a \\ x'/\sqrt{a}} \right\rVert
    \right)^{-k}
    \nonumber \\
    &\leq
    a^\frac{5}{4} \frac{C_k}{2} a^{\frac{k}{2}} \left\lVert
    \substack{(t'-sx') \\ x'} \right\rVert^{-k}
    \sim O\left(a^\frac{5/2+k}{2}\right) ~~ \forall k \in \mathbb{N}
    \nonumber \\[1em]
    \Rightarrow
     \left| \left< \hat\psi_{ast}, \hat G_F \right> \right|
     &\sim
     O\left(a^k\right) ~~ \forall k \in \mathbb{N}
\end{align}


\subsubsection*{Fall $s = 1, (t', x') \neq 0$}
Auch in diesem Fall nutzen wir wieder den ersten Ausdruck für
$\left< \hat\psi_{a1t}, \hat G_F \right>$ aus \eqref{eq:psi_ast_gf_1} und sorgen
wir auch bisher jedes Mal dafür, dass wir im Nenner nur noch positive Potenzen von
$a$ und einen von $a$ unabhängigen Term haben. Dann sieht das ganze so aus:

\begin{equation*}
    \left< \hat\psi_{a1t}, \hat G_F \right>
    =
    a^\frac{3}{4} \int \frac{\hat \psi_1(\omega)
    ~\hat\psi_2 \left(\frac{k}{\omega}\right)
    ~ e^{-i \omega \left(\frac{t'-x'}{a}\right) + i k \frac{x'}{\sqrt{a}}}
    }
    {
        a^\frac{3}{2} m^2 + a^\frac{1}{2} k^2 + 2 \omega k
    }
    \d \omega \d k
\end{equation*}

wo wir im $\lim_{a \rightarrow 0}$ wieder doe $a$-Potenzen im Nenner weg fallen lassen
und auch dieses Mal dafür wieder den Cauchy-Hauptwert bemühen müssen, um den
Lebesgueschen Konvergenzsatz benutzen zu dürfen.
Weiter geht's:

\begin{align}
    &=
    a^\frac{3}{4} \int \frac{
    \hat \psi_1(\omega)
    ~\hat\psi_2 \left(\frac{k}{\omega}\right)
    ~ e^{-i \omega \left(\frac{t'-x'}{a}\right) + i k \frac{x'}{\sqrt{a}}}
    }
    {
    2\omega k
    } \d \omega \d k \nonumber \\
    &= a^\frac{3}{4} \int
    \underbrace{\left\{ \int \frac{\hat \psi_2\left(\frac{k}{\omega}\right)
        ~e^{ik\frac{x'}{\sqrt{a}}}
        }
        {
            2 k \omega
        }
        \d k
        \right\}}_{=: \hat f_a (\omega)}
    \hat \psi_1(\omega) e^{-i \omega \left(\frac{t'-x'}{a}\right)}
    \d \omega
\end{align}

und um hier weiter zu kommen, schauen wir uns $\hat f_a$ genauer an. Sei dazu
$\Psi_2(\omega) := \int_{-\infty}^\omega \psi_2(\omega ') \d \omega '
    -  \int_{\omega}^{+ \infty} \psi_2(\omega ') \d \omega '$ eine
Stammfunktion von $\psi_2$. Dies ist offenbar $C^\infty$ und beschränkt, da
 $\hat \psi_2 \in C^\infty_c$. Mithilfe von Fourieridentitäten und Substitution können wir nun weiter rechnen:

\begin{align*}
    \hat f_a (\omega) &=
    \int \frac{\hat \psi_2\left(\frac{k}{\omega}\right)}{2k\omega}
    e^{i k \frac{x'}{\sqrt{a}}}
    \d \omega \\
    &\stackrel{i)}{=}
    \int \frac{\hat \psi_2 (k)}{2k}e^{ik\frac{x' \omega}{\sqrt{a}}}
    \d \omega \\
    &\stackrel{ii)}{=} \frac{i}{2}  \Psi_2\left(\frac{x' \omega}{\sqrt{a}}\right)
\end{align*}

Hier wurde in $i)$ einfach $k \rightarrow \omega k$ substituiert und im Schritt $ii)$
wurde genutzt, dass $f(x) = \mathrm{sgn}(x) \leftrightarrow \hat f(k) \sim \frac{1}{k}$.
Nun stecken wir diese Erkenntnisse in unseren vorigen Ausdruck und erhalten

\begin{align}
 \left< \hat\psi_{a1t}, \hat G_F \right>
    &=
    \frac{i a^\frac{3}{4}}{2} \int \Psi_2\left(\frac{x' \omega}{\sqrt{a}}\right)
    ~\hat \psi_1(\omega)
    ~ e^{-i \omega \left(\frac{t'-x'}{a}\right)}
    \d \omega \d k
    \nonumber \\
    &\sim O\left(a^\frac{3}{4}\right) \kern 1em \textrm{ ; für } t'=x'
    \nonumber \\
    &\sim O\left(a^k\right) ~ \forall k \in \mathbb{N} \kern 1em \textrm{;   andernfalls}
\end{align}

Im letzten Schritt wurde wieder genutzt, dass
$\Psi_2\left(\frac{x' \omega}{\sqrt{a}}\right) ~\hat \psi_1(\omega) \in \mathcal{S}(\mathbb{R})$
ist, und demnach eine schnell fallende Fouriertransformierte hat.

% Es gilt $\psi_2(0) = 1$, da nach Konstruktion
% $\Vert \psi_2 \Vert_1 = 1$. Außerdem können wir $\psi_2$ so wählen, dass es
% auf einer ganzen offenen Umgebung von 0 konstant 1 ist. Durch geschicktes addieren
% einer 0 können wir nun schreiben

% \begin{equation*}
%     \hat f_a (\omega) =
%     \int \frac{\hat \psi_2 \left(\frac{k}{\omega}\right) - 1}{2k\omega}
%     e^{ik\frac{x'}{\sqrt{a}}} \d k
%     + \int \frac{1}{2k\omega}
%     e^{ik\frac{x'}{\sqrt{a}}} \d k
% \end{equation*}

% wobei das Symbol des ersten Terms glatt ist, da $\psi_2 (0) = 0$. Der zweite Term
% wird also für $a^{-\frac{1}{2}} \rightarrow \infty$ dominieren. Für diesen gilt:
% \todo{Diese Argument verfeinern, oder mindestens raus finden, warum es denn
% zulässig ist.}

% \begin{equation*}
%     \int \frac{e^{ik\frac{x'}{\sqrt{a}}}}{2k\omega} \d k
%     = \frac{2 \pi i}{2 \omega} \mathrm{sgn}\left(\frac{x'}{\sqrt{a}}\right)
% \end{equation*}

% als Hauptwertintegral. Bedenkend dass $\psi_1 \in \mathcal{S} (\mathbb{R})$ und
% $\hat \psi_1 = 0$ in einer Umgebung von $0$ sowie
% $\mathrm{sgn}\left(\frac{x'}{\sqrt{a}}\right) = \mathrm{sgn}\left(x'\right)$  für $a>0$
% können wir also schließlich abschätzen

% \begin{align}
%     \lim_{a \rightarrow 0}\left< \hat\psi_{a1t}, \hat G_F \right>
%     &= \lim_{a \rightarrow 0} a^\frac{3}{4} C \int \frac{\mathrm{sgn}(x')
%     ~\hat \psi_1(\omega)}{\omega}
%     e^{-i\omega \frac{t'-x}{a}} \d \omega
%     \nonumber \\
%     &\sim O\left(a^\frac{3}{4}\right) \kern 1em \textrm{ ; für } t'=x'
%     \nonumber \\
%     &\sim O\left(a^k\right) ~ \forall k \in \mathbb{N} \kern 1em \textrm{;   andernfalls}
% \end{align}

% wobei in $C$ alle irrelevanten Vorfaktoren gesammelt wurden.

Das analoge Ergebnis erhält man auch für $s=-1$ und $t' = -x'$
% section berechnen_von_ (end)


%!TEX root = main.tex
%!TEX spellcheck=de_DE
%%%%%%%%%%%%%%%%%%%%%%%%%%%%%%%%%%%%%%%%%%%%%%%%%%%%%%%%%%%%%%%%%%%%%%%%%%%%%%%
% % Section 2
%%%%%%%%%%%%%%%%%%%%%%%%%%%%%%%%%%%%%%%%%%%%%%%%%%%%%%%%%%%%%%%%%%%%%%%%%%%%%%%

\section{Ausblick} % (fold)
\label{sec:ausblick}

\subsection{\texorpdfstring{Ausdehnen von \cref{thm:main_theorem} auf $\mathcal{S}'$}{Ausdehnen auf Distributionen}} % (fold)
\label{sec:ausdehnen_von_thm:main_theorem}
Wie in \cref{rem:shearlets_no_distributions} angesprochen, zeigt der Beweis von \textcite{Kutyniok2008} \cref{thm:main_theorem} nur für beschränkte Funtkionen und nicht für allgemeine temperierte Distributionen in $\mathcal{S}'$. So werden alle Hilfslemmata für \cref{proof:main_theorem} nur für solche Funktionen bewiesen. Wir glauben aber, dass sich der Beweis auf alle temperierten Distributionen ausdehnen lässt, dank der Tatsache dass "`temperierte Distributionen polynomiell beschränkt sind"':

\begin{theorem}[Struktursatz für temperierte Distributionen]
\label{thm:struktursatz}
    Sei $X \subset \mathbb{R}^n$ offen. Sei $f \in \mathcal{S}'(X)$. Dann gibt es ein $F \in C(X)$ und $C \in \mathbb{R}, N \in \mathbb{N}$ s.d. für alle $x \in X$
    \begin{equation*}
        |F(x)| \leq C (1+|x| )^N
    \end{equation*}
    (also F polynomiell beschränkt ist) und
    \begin{equation*}
        f = \partial^\alpha F
    \end{equation*}
    als distributionelle Ableitung

    \emph{Beweis} \\[.5em]
    Der Beweis findet sich in \textcite[S. 97]{Friedlander1998}.
\end{theorem}


Leider fehlt aufgrund des stetigen Studienfortschritts die Zeit, diesen Beweis komplett auszuarbeiten. Der Beweis der auf temperierte Distributionen ausgeweiteten \cref{prop:shearlets_decay_rapidly} und wie das polynomielle Wachstum der temperierten Distributionen soll hier aber beispielhaft skizziert werden.  Mit ähnlichen Tricks lassen sich hoffentlich auch alle anderen Hilfslemmata auf temperierte Distributionen ausweiten.

\begin{lemma}[Verfeinerung von \cref{prop:shearlets_decay_rapidly}]
    \label{lemm:shearlets_decay_quickly_schwartz}
    Sei $f \in \mathcal{S}'(\mathbb{R}^2)$ und $supp(f) \subset U$. Sei $t \notin U$. Dann gilt für alle $k \in \mathbb{N}$
    \begin{equation*}
        |\left\langle f, \psi_{ast}\right\rangle| \leq C_k \left(1+a^{-1} d(t,U)\right)^{-k}
    \end{equation*}
    ab einem hinreichend kleinen $a$.
\end{lemma}

Vor dem Beweis zwei Worte zur Bedeutung des Lemmas: Der hauptsächliche Nutzen des Lemmas ist die Aussage, dass wir mit Shearlets, also Schwartzfunktionen und \emph{nicht} kompakt getragenen Funktionen, die lokalen Eigenschaften von temperierten Distributionen untersuchen können, da wir alles was $\delta$-weit von $t$ entfernt exponentiell schnell (in $a$) nicht mehr sehen. Dies ist möglich, da temperierte Distributionen in einem geeigneten Sinne nur polynomiell schnell wachsen.

\begin{proof}
    Nach \cref{thm:struktursatz} gibt es ein polynomiell beschränktes $F \in C(\mathbb{R}^2)$ s.d. $f = \partial^\alpha F$. O.B.d.A können wir annehmen, dass $f$ auf $B(0,\delta)$ verschwindet (im distributionellen Sinne) und damit auch $F$. Dann zeigen wir die Aussage für $t=0$:

    \begin{dmath*}
        |\left \langle f, \psi_{as0} \right\rangle|
        \stackrel{\textrm{formal}}{=}
        \left\lvert
        \int f(x) \psi\left(\left(\begin{smallmatrix}
            a & -\sqrt{a}s \\ 0 & \sqrt a
        \end{smallmatrix}
        \right)^{-1}\left(x-0\right)\right)
        \d x \right\rvert \\
        =
        \left\lvert \int F(x) \partial^{\alpha}\left[
        \psi\left(\left(\begin{smallmatrix}
            a^{-1} & \frac{s}{\sqrt a} \\ 0 & a^{-\frac{1}{2}}
        \end{smallmatrix}
        \right)^{-1}x\right)\right]
        \d x \right\rvert\\
        \leq
        \int \limits_{|x| \geq \delta}
        C (1+|x|)^N
        a^{-|\alpha|}
        \underbrace{
        \left[
            \left\lvert \partial_{x_1}^{|\alpha|} \right\rvert
            + \left\lvert \partial^\alpha \psi \right\rvert
        \right]
        }_{=: \phi \in \mathcal{S}}
        \left(\begin{smallmatrix}
            a^{-1} & \frac{s}{\sqrt{a}} \\ 0 & a^{-\frac{1}{2}}
        \end{smallmatrix}\right) x \d x\\
        \leq
        \int \limits_{|x| \geq \delta}
        C (1+|x|)^N
        a^{-|\alpha|}
        C_k \left(
        1+ \left\lvert \left( \begin{smallmatrix}
            a^{-1} & \frac{s}{\sqrt{a}} \\ 0 & a^{-\frac{1}{2}}
        \end{smallmatrix}\right) x \right\rvert
        \right)^{-k}
        \d x\\
        \leq
        \int \limits_{|x| \geq \delta}
        C C_k a^{-|\alpha|} (1+|x|)^N (1+a^{-1}|x|)^{-k} \d x\\
        \leq
        C C_k a^{-|\alpha|}
        \int \limits_{|x| \geq \delta}
        (1+a^{-1}|x|)^{N-k} \d x\\
        =
        C C_k a^{-|\alpha|} 2 \pi
        \int\limits_\delta^\infty
        (1+a^{-1}r)^{N-k} \d r\\
        =
        C C_k a^{-|\alpha|} 2 \pi
        \frac{(a+\delta)\left(1+\frac{\delta}{a}\right)^{N-k}(a+(k-N-1)\delta)}{(k-N-1)(k-N-2)}\\
        \leq
        C'_k\left(1+\frac{\delta}{a}\right)^{N-k-|\alpha|}
    \end{dmath*}
    Was die Aussage für $N-k-|\alpha|$ und damit auch für alle $k$ zeigt.
\end{proof}

% Die entscheidende Aussage des Lemmas ist, dass die Shearlettransformation Eigenschaften von $f$, die mindestens $delta$-weit von $t$ entfernt sind "`exponentiell schnell nicht mehr sieht"'. Obwohl die Shearlets nicht getragen sind!

% section ausdehnen_von_thm:main_theorem (end)

\subsection{Hörmanders Kriterium abschwächen}
\label{sec:hoermanders_crit_abschwaechen}
Um die Wellenfrontmenge einer Distribution zu bestimmen, muss man diese per Definition erst einmal mit einer kompakt getragenen Funktion lokalisieren, um eine kompakt getragene Distribution zu erhalten, deren Fouriertransformation sich berechnen lässt. Ist der Ansatz aus \cref{sec:ausdehnen_von_thm:main_theorem} erfolgreich, so zeigt dies, dass temperierte Distributionen \emph{nicht} mit einer kompakt getragenen Funktion lokalisert werden müssen; Exponentieller Abfall ist genug, um die Wellenfrontmenge zu bestimmen.

Nach der Anschauung aus \cref{fig:faltung_strahlen} ist Hörmanders Kriterium zwar hinreichend, um das punktweise Produkt zweier Distributionen zu definieren über die Faltung ihrer Fouriertransformierten, aber nicht notwendig. Ein Beispiel dafür ist die Heaviside-Funktion. Offenbar existiert $\Theta(x)^2$, aber Hörmanders Kriterium wird an der $0$ verletzt, da $\hat \Theta (k) = \frac{i}{k} + \delta(k)$. Es ist ausreichend, wenn die lokalisierten Fouriertransformierten in entgegengesetzter im Produkt mit $|k|^{-d-\epsilon}$ abfallen, damit das Faltungsintegral existiert.

Die Hoffnung ist also, dass man an der $a$-Potenz mit der $\left\langle f, \psi_{ast} \right\rangle$ für $a \to 0$ skaliert ablesen kann, wie schnell
	$\lim_{|k| \to \infty} \rwhat{\phi f} (k)$
abfällt für $\frac{k_2}{k_1} = s$ und $\phi$ beliebig nah um $t$ lokalisiert.

Der Ansatz dabei wäre es, zunächst einmal diese $a$-Potenzen bei gut verstandenen Distributionen, z.B. Polynomen oder Ableitungen der $\delta$-Distribution auszurechnen und mit dem Abfallverhalten der lokalisierten Fouriertransformierten zu vergleichen, um dann den genauen Zusammenhang zu raten. Dann wird man versuchen diesen mit den Techniken und Lemmata aus \cref{sec:beweis_von_thm:main_theorem} zu beweisen.

\subsection{Höherdimensionale Shearlets}
Eine beinahe schon peinlich offensichtliche Frage ist: Wie steht es denn damit, das ganze Geschäft der Shearlets mal auf höhere Dimensionen auszudehnen und auch dort eine schöne Technik zum bestimmen von Wellenfrontmengen zu erhalten?

\textcite{Guo2006} diskutieren Verallgemeinerungen der Schergruppe in höheren Dimensionen und entwickeln daraus auch diskrete Wavelets. Aus z.B. \cref{fig:delta_m} wird auch deutlich, was die richtige höhere Verallgemeinerung der parabolischen Skalierung ist. Nämlich

\begin{equation*}
\begin{pmatrix}
k_1 \\ k_2 \\ \vdots \\ k_n
\end{pmatrix}
\mapsto
\begin{pmatrix}
	a & 0 		& \cdots & 0\\
	0 & \sqrt a & 		 & 	\vdots\\
	0 & 			& \ddots & 0 \\
	0 & \cdots  & 	0    & \sqrt{a}
\end{pmatrix}
\begin{pmatrix}
k_1 \\ k_2 \\ \vdots \\ k_n
\end{pmatrix},
\end{equation*}

denn diese sorgt wieder dafür, dass der Träger von $\psi_{ast}$ im Fourierraum für $a \to 0$ wieder einer immer spitzer werdenden Nadel gleicht. Die Wahl $\sqrt{a}$ statt $a^\delta$ für irgendein anderes $a<1$ ist ziemlich willkürlich. \textcite{Kutyniok2008} schreiben auch, dass sie für $\delta \neq \frac{1}{2}$ die Wellenfrontmenge an Beispielen genau so gut bestimmen konnten, wie für $\delta = \frac{1}{2}$. Tatsächlich geht $\delta = \frac{1}{2}$ nur bei dem Beweis von \cref{lemm:lemma57} explizit ein. Aber sicher lässt sich \cref{thm:main_theorem} auch mit einem \cref{lemm:lemma57} beweisen, das leicht andere Exponenten hat.\footnote{Stellt sich nur die Frage, warum man das überhaupt wollte. $\delta = \frac{1}{2}$ ist doch ein ziemlich schöne Wahl.}

Allerdings zeigen die Rechnungen ab \cref{sec:die_wellenfrontmenge_von_delta_m_2_}, dass schon im 2-dimensionalen Fall die Rechungen schnell unübersichtlich werden und notwendigen Abschätzungen sich häufig nur schwer rechtfertigen lassen. Dementsprechend mag es möglich sein, höherdimensionale Shearlets zu entwickeln mit denen es möglich ist Wellenfrontmengen zu bestimmen. Aber in der Praxis werden die Rechnungen bei Distributionen, deren Wellenfrontmengen nicht ohnehin schon einfach zu berechnen sind, sicher sehr unübersichtlich. Oder ich habe mich bei meinen Abschätzungen einfach selten dämlich angestellt.


\subsection{Berechnung des Skalengrads mittels Shearlets}
Eine weitere Größe der mikrolokalen Analysis, die eventuell durch die Shearlettransformation bestimmt werden kann ist der Skalengrad. Er ist definiert wie folgt:

\begin{definition}[Skalengrad]
\label{def:skalengrad}
    Sei $u \in \mathcal{D}'(\Omega),~ \Omega \subset \mathbb{R}^n ~$ offen. Dann ist der Skalengrad $sd(u)$ definiert als
    \begin{equation*}
        \inf_{\omega} \left\{\lambda^\omega \left\langle u_\lambda, \phi \right\rangle
        \stackrel{\lambda \to 0}{\to} 0 \condition{für alle $\phi$} \right\}
    \end{equation*}

    wobei $u_\lambda$ definiert ist über
    \begin{equation*}
        \lambda^{-n} \left\langle u, \phi\left(\tfrac{\cdot}{\lambda}\right)\right\rangle
    \end{equation*}

    also falls $u \in C^\infty$:
    \begin{equation*}
        u_\lambda (x) = u(\lambda x)
    \end{equation*}
\end{definition}

Eine einfache Rechnung zeigt z.B. für die $\delta$-Distribution und ihre Ableitungen, dass

\begin{equation*}
    sd(\delta^{(\alpha)}) = n + |\alpha|
    .
\end{equation*}

Mit der Shearlettransformation erhalten wir aber

\begin{dmath*}
    \left\langle \delta_{x_1}^\alpha \otimes \delta_{x_2}, \psi_{a00} \right\rangle
    =
    \left.\partial_{x_1}^\alpha \left(a^{-\frac{3}{4}}\psi\left(\frac{x_1}{a}, \frac{x_2}{\sqrt a}\right)\right)\right|_{x=0} \\
    =
    a^{-\frac{3}{4}} a^{-\alpha} \partial_{x_1}^\alpha \psi(0)
    \hiderel \sim a^{ - \alpha-\frac{3}{4}}
\end{dmath*}

und bei Ableitung in die andere Richtung


\begin{dmath*}
    \left\langle \delta_{x_1} \otimes \delta_{x_2}^\alpha, \psi_{a00} \right\rangle
    =
    \left.\partial_{x_2}^\alpha \left(a^{-\frac{3}{4}}\psi\left(\frac{x_1}{a}, \frac{x_2}{\sqrt a}\right)\right)\right|_{x=0} \\
    =
    a^{-\frac{3}{4}} a^{-\frac{\alpha}{2}} \partial_{x_2}^\alpha \psi(0)
    \hiderel \sim a^{-\frac{\alpha}{2}-\frac{3}{4}}
\end{dmath*}

Und falls wir $s \neq 0$ wählen wird das ganze nur noch unübersichtlicher, da wir Mischterme erhalten. Dieses Beispiel legt also nahe, dass es einen Zusammenhang zwischen dem Skalengrad einer Distribution und dem Abfallverhalten der Shearlettransformation bei $t=0$ gibt. Aber die parabolische Skalierung in $a$ und Scherung in $s$ sorgen dafür, dass sie sich nicht mehr ganz einfach ablesen lässt.
% section ausblick (end)


%!TEX root = main.tex

\section{Fazit für Physiker} % (fold)
\label{sec:fazit_fuer_physiker}

Die Berechnungen in \cref{sec:die_wellenfrontmenge_von_delta_m,sec:die_wellenfrontmenge_von_delta_m2_twisted,sec:die_wellenfrontmenge_von_delta_m_2_} und Abschätzungsungetüme wie \cref{eq:psi_ast_delta_m2} zeigen deutlich, dass \cref{thm:main_theorem} zwar eine theoretische Möglichkeit liefert Wellenfrontmengen auszurechnen, es aber kein sehr praktikabler Ansatz ist. So wurde auch $\psi_{ast}$ nie konkret angegeben, sondern nur darauf hingewiesen, dass es Funktionen gibt, die all das erfüllen, was wir brauchen (also schneller Abfall und gewisse Eigenschaften des Trägers der Fouriertransformierten). Eins würden diese Funktionen aber sicher \emph{nicht} erfüllen: Dass $\int \psi_{ast}(x)\,f(x) \d x$ für eine größere Klasse von Funktionen tatsächlich analytisch zu berechnen ist, und nicht nur gewisse Schranken für den Abfall in $a$ gegeben werden können.

Ähnlich sieht es bei der Berechnung des Skalengrads mithilfe von Shearlets aus (vgl. \cref{sec:scaling_degree}): Es sieht so aus, als sei es theoretisch möglich. Aber mit gewissem Aufwand bei den Abschätzungen verbunden.

Umso erfreulicher ist, dass die Ergebnisse für die berechneten Wellenfrontmengen mit den bisher bekannten übereinstimmen. Im Falle des getwisteten Produkts der Zweipunktfunktion konnte das Ergebnis von \textcite{Schulz2014} ja sogar verschärft und gezeigt werden, dass das getwistete Produkt bei $0$ nicht ganz so singulär ist, wie das ungetwistete.

Höherdimensionale Verallgemeinerungen der Shearlets müssten mit noch mehr Scherparametern arbeiten -- im drei dimensionalen Fall 3, in 4D schon  6 -- welche dann in den Ausdrücken auftauchen. Dann müsste man schlau erkennen, für welche Kombinationen dieser Scherpararemeter an welchen Orten $t$ $\left< f, \psi_{ast} \right>$ nicht schnell abfällt und abschätzen, wie schnell genau es abfällt.
Wenn die Verzweiflung also sehr groß ist, man viel Zeit, Papier und höherdimensionale Shearlets zur Verfügung hat \emph{könnte} die Shearlettransformation eine theoretische Möglichkeit sein Wellenfrontmengen temperierter Distributionen auszurechnen. Aber eigentlich eher nicht.

Oder natürlich wir haben etwas ganz wichtiges übersehen, und es ist doch alles nicht so hoffnungslos.

% section fazit_für_physiker (end)


%!TEX root = main.tex
\section{Fazit für Mathematiker} % (fold)
\label{sec:fazit_für_mathematiker}

Erfreulicherweise\footnote{und glücklicherweise, sonst hätten wir uns ja verrechnet} stimmen die Ergebnisse für berechneten Wellenfrontmengen mit denen in der Literatur überein. Über die Wellenfrontmenge hinaus erhalten wir mit dem Exponenten von $a$ auch noch weitere Informationen wie langsam die lokalisierte Fouriertransformierte in eine gewisse Richtung abfällt. Größtenteils offen ist, inwiefern diese zusätzlichen Informationen genutzt werden können. Allerdings finden sich in \cref{sec:hoermanders_crit_abschwaechen,sec:scaling_degree} schon Ansätze.

In den Komplikationen und langen Abschätzungen in \cref{sec:die_wellenfrontmenge_von_delta_m,sec:die_wellenfrontmenge_von_delta_m_2_,sec:die_wellenfrontmenge_von_delta_m2_twisted} zeigt sich, dass die Shearlettransformation nur bedingt geeignet ist, um Wellenfrontmengen von temperierten Distributionen zu berechnen.

Dennoch sind im Ausblick mit der Ausweitung von \cref{thm:main_theorem} auf temperierte Distributionen und  der Konstruktion höherdimensionaler Shearlets interessante Fragen aufgekommen, bei deren Bearbeitung sicher temperierte Distributionen und auch etwas Geometrie besser verstanden werden können.

% section fazit_für_mathematiker (end)

\printbibliography

\Declaration

\end{document}
