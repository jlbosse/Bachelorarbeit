%!TEX root = main.tex
% !TEX spellcheck=de_DE
%%%%%%%%%%%%%%%%%%%%%%%%%%%%%%%%%%%%%%%%%%%%%%%%%%%%%%%%%%%%%%%%%%%%%%%%%%%%%%%
% % Der Heaviside-Function
%%%%%%%%%%%%%%%%%%%%%%%%%%%%%%%%%%%%%%%%%%%%%%%%%%%%%%%%%%%%%%%%%%%%%%%%%%%%%%%

\section{\texorpdfstring{Die Wellenfrontmenge von $\Theta$}
        {Die Wellenfrontmenge der Heaviside-Funktion}} % (fold)
\label{sec:die_wellenfrontmenge_der_heaviside_function}

Wie in \cref{sec:die_zweipunktfunktionen_und_warum_wir_sie_potenzieren_wollen} erklärt, sind Potenzen des Feynmanpropagators gegeben durch Potenzen der Zweipunktfunktion $\Delta_m$ und der Heaviside-Funktion $\Theta$. Dementsprechend, muss auch die Wellenfrontmenge von $\Theta$ berechnet werden, aber dies ist glücklicherweise auch mit unseren Shearlet-Methoden relativ einfach.

Da $\Theta$ die Stammfunktion von $\delta$ (im distributionellen Sinne) ist, können wir die Fouriertransformierte dank der üblichen Fourierrechenregeln direkt hinschreiben:\footnote{Wieder nur korrekt bis auf Vorfaktoren von $2\pi$}

\begin{equation}
    \rwhat{\Theta(t)\otimes 1(x)}(\omega,k) = \rwhat{\Theta}(\omega) \otimes \rwhat{\delta}(k) = \left(\delta(\omega) + \frac{i}\omega{}\right) \delta (k)
    \label{eq:heaviside_fourier_transform}
\end{equation}.

\subsubsection*{Fall $s \neq 0$}

\begin{equation*}
supp (\rwhat{\Theta(t)\otimes 1(x)}) = \{(\omega, k) \in \hat{\mathbb{R}} | k = 0\}
\end{equation*}

 und nach \cref{eq:supp_psi}

\begin{equation*}
    supp(\hat \psi) \subset \left\{k \in  \hat{\mathbb{R}}^2 ~\Big| ~k_1 \in \left[\frac{1}{2 a} , \frac{2}{a}\right], \left|\frac{k_2}{k_1} - s\right| \leq \sqrt{a} \right\}
\end{equation*}.

Also gilt für hinreichend große $a$:

\begin{equation}
 supp (\hat\psi_{ast}) \cap supp (\rwhat{\Theta(t)\otimes 1(x)}) = \varnothing
 ~~\Longrightarrow~~
\left\langle \rwhat{\Theta(t)\otimes 1(x)},  \hat\psi_{ast}\right\rangle = 0
\label{eq:heaviside_s_neq_0}
\end{equation}

\subsubsection*{Fall $s = 0$}
Mit \cref{eq:heaviside_fourier_transform} können $\left<\hat \Theta \otimes \hat 1, \hat \psi_{ast}\right>$ direkt berechnen mit dem Ausdruck für $\hat \psi_{ast}$ aus \cref{rem:psi_hat}:

\begin{align}
    \left\langle \hat\Theta \otimes \hat 1, \hat \psi_{ast} \right\rangle
    &=
    a^{\frac{3}{4}} \int \hat \psi_1(a \omega) \hat \psi_2\left(
                a^{-\frac{1}{2}}\left(\tfrac{k}{\omega}\right)\right)
    \left(\delta(\omega) + \frac{i}{\omega}\right) \delta(k)
    e^{-i\omega t' + i kx'}
    \d \omega \d k
    \nonumber \\ &=
    \underbrace{a^{\frac{3}{4}} \hat\psi_1(0) \hat\psi_2(0)
    }_{=0, \textrm{ da } \hat\psi_1(0) = 0}
    +
    i a^{\frac{3}{4}}
    \int \frac{\hat\psi_1(\omega) \hat\psi_2(0)}
        {\omega} e^{-i \omega t'} \d \omega
    \nonumber \\ &=
    i a^{\frac{3}{4}} \hat\psi_2(0) \int
    \underbrace{\frac{\psi_1(\omega)}{\omega}}_{\in C_c^\infty}
    e^{-i\omega\frac{t'}{a}} \d \omega
    \nonumber \\ &=
    O\left(a^{\frac{3}{4}}\right) \condition{falls $t=0$}
    \nonumber \\ &=
    O\left(a^{k}\right) ~ \forall k \in \mathbb{N} \condition{falls $t\neq0$}.
    \label{eq:heaviside_s=0}
\end{align}

Wobei im letzten Schritt genutzt wurde, dass $\hat\psi_1(0) = 0$, $\frac{\hat\psi_1(\omega)}{\omega}$ also glatt ist und somit eine schnell fallende Fouriertransformierte hat.


\subsection{Zusammenfassung und Vergleich der Ergebnisse}
Und einmal der vollständig halber, die Ergebnisse aus \cref{eq:heaviside_s=0,eq:heaviside_s_neq_0} tabellarisch dargestellt:

\begin{table}[h]
\centering
\begin{tabular}{l|cc}
           & \multicolumn{1}{l}{$t'=0$} & \multicolumn{1}{l}{$t' \neq 0$} \\ \hline
$s = 0$    & $a^{\frac{3}{4}}$          & $a^k$                           \\
$s \neq 0$ & $a^k$                      & $a^k$
\end{tabular}
\caption{Konvergenzordnung von $\mathcal{S}_{\Theta \otimes 1} (a,s,(t',x'))$ im Limit $a \rightarrow 0$ für alle interessanten Kombinationen von $s$ und $(t',x')$}
\label{tab:wavefront_set_heaviside}
\end{table}



% section die_wellenfrontmenge_von_ (end)
