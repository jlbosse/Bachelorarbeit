%!TEX root = main.tex
% !TEX spellcheck=de_DE
%%%%%%%%%%%%%%%%%%%%%%%%%%%%%%%%%%%%%%%%%%%%%%%%%%%%%%%%%%%%%%%%%%%%%%%%%%%%%%%
% % Section 3
%%%%%%%%%%%%%%%%%%%%%%%%%%%%%%%%%%%%%%%%%%%%%%%%%%%%%%%%%%%%%%%%%%%%%%%%%%%%%%%
\section{\texorpdfstring{Berechnen von $WF(G_F)$}
        {Berechnen von WF(Gf)}} % (fold)
\label{sec:berechnen_von_}

\subsection{\texorpdfstring{Ausdrücke für $\left< \psi_{ast}, G_F\right>$}
        {Ausdrücke für psi ast, gf}} % (fold)
\label{sec:psiast_gf}

Der Feynmanpropgator ist einerseits definiert durch \cref{eq:feynman_propgator_as_product}, kann aber im Impulsraum auch geschrieben werden als (\textcite{Schwartz2014}, (6.34))

\begin{equation}
\label{eq:gf}
    \hat G_F(\omega, k) = \frac{1}{m^2 - \omega^2 + k^2 - i 0^+}
\end{equation}

Setzen wir dies in unsere Ausdrücke für $\left< \psi_{ast}, f\right>$ aus \ref{eq:substitution1}
bzw. \ref{eq:substitution2} ergibt sich, unter Verwendung des Minkowskiskalaprodukts,

\begin{align}
\left<  \hat G_F, \hat\psi_{ast} \right> &=
    \int \hat \psi_{ast}(\omega, t) ~\hat G_F(\omega, t) ~\d \omega \d k
    \nonumber \\
    &=
    a^{\frac{3}{4}} \iint \frac{
        \hat \psi_1 (a \omega)
        ~\hat \psi_2 \left(a^{-\frac{1}{2}}\frac{k}{\omega} - s\right)
        ~ e^{-i\omega t' + i k x'}
    }
    {
        m^2 - \omega^2 + k^2 - i 0^+
    }
    \d \omega \d k
    \nonumber \\
    &=
    a^{-\frac{3}{4}} \iint \frac{
        \hat\psi_1(\omega)
        ~\hat \psi_2\left(\tfrac{k}{w}\right)
        ~e^{-i \omega \frac{t' - sx'}{a} + ik \frac{x'}{\sqrt{a}}}
    }
    {
        m^2 - \left(\frac{\omega}{a}\right)^2
        + \left(\frac{\omega s}{a} + \frac{k}{\sqrt{a}}\right)^2 - i0^+
    }
    \d \omega \d k \nonumber \\
    &=
    a^{-\frac{3}{4}}
    \kern -2em \iint
    \limits_{
    \substack{
        \omega \in [-2, -\frac{1}{2}]\cup[\frac{1}{2},2] \\
        \left|\frac{k}{2}-s\right| \leq \sqrt{ax}
        }
    }
    \kern -1.5em
    \frac{
        \hat\psi_1(\omega)
        ~\hat \psi_2\left(\tfrac{k}{\omega}\right)
        ~e^{-i \omega \frac{t' - sx'}{a} + ik \frac{x'}{\sqrt{a}}}
    }
    {
        m^2 + a^{-2} \omega^2 (s^2 - 1) + a^{-\frac{3}{2}} 2 s \omega k + a^{-1} k  - i0^+
    }
    \d \omega \d k
    \label{eq:psi_ast_gf_1}
\end{align}

und mit der anderen Substitution analog

\begin{align}
    \left<  \hat G_F, \hat\psi_{ast} \right>
    &=
    a^{-\frac{3}{4}}
    \kern -1em
    \iint \limits_{\substack{
        |\omega|~ \in~ [\frac{1}{2},2] \\
        k ~\in~ [-1,1]
        }
    }
    \kern -1em
    \frac{
        \omega ~\hat \psi_1(\omega) ~\hat \psi_2(k)
        e^{-i \omega \left(\frac{t' - sx'}{a} + \frac{kx'}{\sqrt{a}}\right)}
    }
    {
        m^2 - \omega^2(a^{-2}(1-s^2)-a^{-1}k^2 - 2 k s a^{-\frac{3}{2}})
    }
    \d \omega \d k
    \label{eq:psi_ast_gf_2}
\end{align}

wobei sich die Integrationsbereiche aus den Forderungen an den Träger von $\psi$
(vgl. \eqref{eq:supp_psi}) ergeben.


\subsubsection*{Fall $s=1, t' = 0 = x'$}
Nach \eqref{eq:psi_ast_gf_2} erhalten wir mit $s=1, t' = 0 = x'$

\begin{align*}
    \left<  \hat G_F,\hat\psi_{a10} \right>
    &=
    \int a^{-\frac{3}{4}} \frac{
        \omega ~\hat \psi_1(\omega) ~\hat \psi_2(k)
    }
    {
        m^2+\omega^2 (a^{-1}k^2 + a^{-\frac{3}{2}}2 k )
    }
    \d \omega \d k \\
    &=
    \int a^{\frac{3}{4}} \frac{
        \omega ~\hat \psi_1(\omega) ~\hat \psi_2(k)
    }
    {
        a^{\frac{3}{2}} m^2+\omega^2 (a^{\frac{1}{2}}k^2 + 2 k )
    }
    \d \omega \d k
\end{align*}

Da aber $|\omega| \in [\frac{1}{2},2]$ und $k \in [-1,1]$ ist, ist für hinreichend
kleine $a$ (und für genau die interessieren wir uns ja)

\begin{equation*}
    \left|
        \frac{\omega ~\hat \psi_1(\omega) ~\hat \psi_2(k)}{k \omega^2}
    \right|
    \geq
    \left|
        \frac{\omega ~\hat \psi_1(\omega) ~\hat \psi_2(k)}
        {a^{\frac{3}{2}}m^2+a^{\frac{1}{2}}\omega^2 k+2k \omega^2}
    \right|
\end{equation*}

eine integrierbare (im Sinne des Cauchy-Hauptwertes) Majorante für den Integranden.

Wir dürfen uns also des Lebesgueschen Konvergenzsatzes bedienen und schreiben

\begin{equation}
    \lim_{a \rightarrow 0} \left< \hat\psi_{a10}, \hat G_F \right> =
    a^{\frac{3}{4}} \int \frac{
    \omega ~\hat \psi_1(\omega) ~\hat \psi_2(k)
    }
    {
    2k \omega^2
    }
    \d \omega \d k
    \sim O(a^{\frac{3}{4}})
    \label{eq:gf_s=1_tx=0}
\end{equation}

Für $s = -1$ erhalten wir genau das selbe Ergebniss, da ja der $\omega^2 (1-s^2)$-Term
im Nenner genauso wieder verschwindet.

\subsubsection*{Fall $s \neq \pm 1, t' = 0 = x'$}
In diesem Fall verschwindet der $\omega^2 (1-s^2)$-Term im Nenner nicht und
dementsprechend folgt

\begin{align*}
    \left< \hat G_F, \hat\psi_{as0} \right>
    &=
    \int a^{-\frac{3}{4}} \frac{
        \omega ~\hat \psi_1(\omega) ~\hat \psi_2(k)
    }
    {
        m^2-\omega^2 ((1-s^2) - a^{-1}k^2 - a^{-\frac{3}{2}}2 k )
    }
    \d \omega \d k \\
    &=
    \int a^{\frac{5}{4}} \frac{
        \omega ~\hat \psi_1(\omega) ~\hat \psi_2(k)
    }
    {
        a^2 m^2+\omega^2 (s^2-1) + a \omega^2 k^2 + a^{\frac{1}{2}}2 \omega^2 k s
    }
    \d \omega \d k
    \label{eq:gf_s=-1_tx=0}
\end{align*}

Analog zum vorigen Teil ist, diesmal sogar ohne den Cauchy-Hauptwert bemühen zu
müssen,

\todo{Überall wo es sein muss $\lim_{a \rightarrow 0}$ dazu schreiben, oder sagen
dass der Limit überall impliziert ist}

\begin{equation*}
    \left|
        \frac{2 \omega ~\hat \psi_1(\omega) ~\hat \psi_2(k)}{\omega^2 (1-s^2)}
    \right|
    \geq
    \left|
        \frac{
        \omega ~\hat \psi_1(\omega) ~\hat \psi_2(k)
    }
    {
        a^2 m^2+\omega^2 (s^2-1) + a \omega^2 k^2 + a^{\frac{1}{2}}2 \omega^2 k s
    }
    \right|
\end{equation*}

dass eine integrierbare Majorante ist (in der Tat ja sogar in $C_c^\infty (\mathbb{R}^2)$)
Damit können wir folgende Abschätzung treffen:

\begin{equation}
    \lim_{a \rightarrow 0} \left< \hat G_F, \hat\psi_{as0} \right> =
    a^{\frac{5}{4}} \int \frac{2 \omega ~\hat \psi_1(\omega) ~\hat \psi_2(k)}
    {\omega^2 (1-s^2)}
    \d \omega \d k
    \sim O(a^{\frac{5}{4}})
    \label{eq:gf_s_neq_1_xt=0}
\end{equation}


\subsubsection*{Fall $s \neq \pm 1, (t', s') \neq 0$}
In diesem Fall benutzen wir wieder die erste Substitution \eqref{eq:psi_ast_gf_1}
und klammern wie schon in den beiden vorigen Teilen die höchste negative
Potenz von $a$ im Nenner aus.

\begin{align}
\Rightarrow ~
    \left< \hat G_F,\hat\psi_{ast} \right>
    &=
    a^{\frac{5}{4}} \int \frac{
        \hat \psi_1 (\omega) ~\hat \psi_2 \left(\frac{k}{\omega}\right)
        ~ e^{-i \omega \left(\frac{t'-sx'}{a}\right) + i k \frac{x'}{\sqrt{a}}}
    }
    {
        a^2 m^2 - \omega^2 (1-s^2) + a^{\frac{1}{2}} s \omega k +a k^2
    }
    \d \omega \d k
\end{align}

und da immer noch $0 \notin supp(\psi_1)$ gilt ist ein weiteres mal eine integrierbare Majorante gegeben durch

\begin{equation}
    2\frac{\hat \psi_1 (\omega)~\hat\psi_2 \left(\frac{k}{\omega}\right)}
    {\omega^2(s^2-1)}
\end{equation}

In der Tat ist sogar

\begin{equation}
    \hat f(\omega, k) := \frac{\hat \psi_1 (\omega)~\hat\psi_2 \left(\frac{k}{\omega}\right)}
    {\omega^2(s^2-1)}
    \in C_c^\infty (\hat{\mathbb{R}}^2)
\end{equation}

da $\psi_1$ und $\psi_2$ getragen sind. Demnach ist die Fourierinverse von
$\hat f$, $f := (\hat f)^\vee \in \mathcal{S}(\mathbb{R}^2)$, also schnell
fallend. Damit können wir schließlich abschätzen

\begin{align}
    \left| \left<  \hat G_F,\hat\psi_{ast} \right> \right|
    &=
    a^{\frac{5}{4}} \left|  \int \hat f(\omega, k)
    ~e^{-i \omega \left(\frac{t'-sx'}{a}\right) + ik \frac{x'}{\sqrt{a}}}
    \d \omega \d k
    \right|
    \nonumber \\
    &=
    a^{\frac{5}{4}} \left| f \left(\frac{t'-sx}{a}, \frac{x'}{\sqrt{a}}\right) \right|
    \leq
    a^{\frac{5}{4}} C_k\left(
    1 + \left\lVert \substack{(t'-sx')/a \\ x'/\sqrt{a}} \right\rVert
    \right)^{-k}
    \nonumber \\
    &\leq
    a^{\frac{5}{4}} \frac{C_k}{2} a^{\frac{k}{2}} \left\lVert
    \substack{(t'-sx') \\ x'} \right\rVert^{-k}
    \sim O\left(a^{\frac{5/2+k}{2}}\right) ~~ \forall k \in \mathbb{N}
    \nonumber \\[1em]
    \Rightarrow
     \left| \left< \hat G_F, \hat\psi_{ast} \right> \right|
     &\sim
     O\left(a^k\right) ~~ \forall k \in \mathbb{N}
     \label{eq:gf_s_neq=1_xt_neq_0}
\end{align}


\subsubsection*{Fall $s = 1, (t', x') \neq 0$}
Auch in diesem Fall nutzen wir wieder den ersten Ausdruck für
$\left< \hat\psi_{a1t}, \hat G_F \right>$ aus \eqref{eq:psi_ast_gf_1} und sorgen
wir auch bisher jedes Mal dafür, dass wir im Nenner nur noch positive Potenzen von
$a$ und einen von $a$ unabhängigen Term haben. Dann sieht das ganze so aus:

\begin{equation*}
    \left< \hat G_F, \hat\psi_{a1t} \right>
    =
    a^{\frac{3}{4}} \int \frac{\hat \psi_1(\omega)
    ~\hat\psi_2 \left(\frac{k}{\omega}\right)
    ~ e^{-i \omega \left(\frac{t'-x'}{a}\right) + i k \frac{x'}{\sqrt{a}}}
    }
    {
        a^{\frac{3}{2}} m^2 + a^{\frac{1}{2}} k^2 + 2 \omega k
    }
    \d \omega \d k
\end{equation*}

wo wir im $\lim_{a \rightarrow 0}$ wieder doe $a$-Potenzen im Nenner weg fallen lassen
und auch dieses Mal dafür wieder den Cauchy-Hauptwert bemühen müssen, um den
Lebesgueschen Konvergenzsatz benutzen zu dürfen.
Weiter geht's:

\begin{align}
    &=
    a^{\frac{3}{4}} \int \frac{
    \hat \psi_1(\omega)
    ~\hat\psi_2 \left(\frac{k}{\omega}\right)
    ~ e^{-i \omega \left(\frac{t'-x'}{a}\right) + i k \frac{x'}{\sqrt{a}}}
    }
    {
    2\omega k
    } \d \omega \d k \nonumber \\
    &= a^{\frac{3}{4}} \int
    \underbrace{\left\{ \int \frac{\hat \psi_2\left(\frac{k}{\omega}\right)
        ~e^{ik\frac{x'}{\sqrt{a}}}
        }
        {
            2 k \omega
        }
        \d k
        \right\}}_{=: \hat f_a (\omega)}
    \hat \psi_1(\omega) e^{-i \omega \left(\frac{t'-x'}{a}\right)}
    \d \omega
\end{align}

und um hier weiter zu kommen, schauen wir uns $\hat f_a$ genauer an. Sei dazu
$\Psi_2(\omega) := \int_{-\infty}^\omega \psi_2(\omega ') \d \omega '
    -  \int_{\omega}^{+ \infty} \psi_2(\omega ') \d \omega '$ eine
Stammfunktion von $\psi_2$. Dies ist offenbar $C^\infty$ und beschränkt, da
 $\hat \psi_2 \in C^\infty_c$. Mithilfe von Fourieridentitäten und Substitution können wir nun weiter rechnen:

\begin{align*}
    \hat f_a (\omega) &=
    \int \frac{\hat \psi_2\left(\frac{k}{\omega}\right)}{2k\omega}
    e^{i k \frac{x'}{\sqrt{a}}}
    \d \omega \\
    &\stackrel{i)}{=}
    \int \frac{\hat \psi_2 (k)}{2k}e^{ik\frac{x' \omega}{\sqrt{a}}}
    \d \omega \\
    &\stackrel{ii)}{=} \frac{i}{2}  \Psi_2\left(\frac{x' \omega}{\sqrt{a}}\right)
\end{align*}

Hier wurde in $i)$ einfach $k \rightarrow \omega k$ substituiert und im Schritt $ii)$
wurde genutzt, dass $f(x) = \mathrm{sgn}(x) \leftrightarrow \hat f(k) \sim \frac{1}{k}$.
Nun stecken wir diese Erkenntnisse in unseren vorigen Ausdruck und erhalten

\begin{dmath}
 \left\langle \hat G_F, \hat\psi_{a1t} \right\rangle
    =
    \frac{i a^{\frac{3}{4}}}{2} \int \Psi_2\left(\frac{x' \omega}{\sqrt{a}}\right)
    ~\hat \psi_1(\omega)
    ~ e^{-i \omega \left(\frac{t'-x'}{a}\right)}
    \d \omega \d k
    \sim O\left(a^{\frac{3}{4}}\right)  \condition{\textrm{für } t'=x'}
    \sim O\left(a^k\right) ~ \forall k \hiderel \in \mathbb{N} \condition{sonst}
    \label{eq:gf_s=1_xt_neq_0}
\end{dmath}

Im letzten Schritt wurde wieder genutzt, dass
$\Psi_2\left(\frac{x' \omega}{\sqrt{a}}\right) ~\hat \psi_1(\omega) \in \mathcal{S}(\mathbb{R})$
ist, und demnach eine schnell fallende Fouriertransformierte hat.

\subsection{Zusammenfassung und Vergleich der Ergebnisse}
Fassen wir wie bisher schon die Ergebnisse aus\cref{eq:gf_s_neq_1_xt=0,eq:gf_s=1_xt_neq_0,eq:gf_s_neq=1_xt_neq_0,eq:gf_s=1_tx=0} wieder in einer Übersichtstabelle zusammen:


\begin{table}[]
\centering
\begin{tabular}{l|cccc}
        & \multicolumn{1}{l}{$(t', x') = (0, 0)$} & \multicolumn{1}{l}{$t' = x'$} & \multicolumn{1}{l}{$t' = -x'$} & \multicolumn{1}{l}{$t' \neq \pm x'$} \\ \hline
$s=1$   & $a^{\frac{3}{4}}$                       & $a^{\frac{3}{4}}$             & $a^k$                          & $a^k$                                \\
$s=-1$  &$a^{\frac{3}{4}}$                       & $a^k$                         & $a^{\frac{3}{4}}$              & $a^k$                                \\
$s \neq \pm 1$ & $a^{\frac{5}{4}}$                       & $a^k$                         & $a^k$                          & $a^k$                                \\
\end{tabular}
\caption{Konvergenzordnung von $\mathcal{S}_{G_F}(a,s,(t',x'))$ im Limit $a \to 0$ für alle interessanten Kombinationen von $s$ und $(t',x')$}
\label{tab:wavefrontset_gf}
\end{table}




% Es gilt $\psi_2(0) = 1$, da nach Konstruktion
% $\Vert \psi_2 \Vert_1 = 1$. Außerdem können wir $\psi_2$ so wählen, dass es
% auf einer ganzen offenen Umgebung von 0 konstant 1 ist. Durch geschicktes addieren
% einer 0 können wir nun schreiben

% \begin{equation*}
%     \hat f_a (\omega) =
%     \int \frac{\hat \psi_2 \left(\frac{k}{\omega}\right) - 1}{2k\omega}
%     e^{ik\frac{x'}{\sqrt{a}}} \d k
%     + \int \frac{1}{2k\omega}
%     e^{ik\frac{x'}{\sqrt{a}}} \d k
% \end{equation*}

% wobei das Symbol des ersten Terms glatt ist, da $\psi_2 (0) = 0$. Der zweite Term
% wird also für $a^{-\frac{1}{2}} \rightarrow \infty$ dominieren. Für diesen gilt:
% \todo{Diese Argument verfeinern, oder mindestens raus finden, warum es denn
% zulässig ist.}

% \begin{equation*}
%     \int \frac{e^{ik\frac{x'}{\sqrt{a}}}}{2k\omega} \d k
%     = \frac{2 \pi i}{2 \omega} \mathrm{sgn}\left(\frac{x'}{\sqrt{a}}\right)
% \end{equation*}

% als Hauptwertintegral. Bedenkend dass $\psi_1 \in \mathcal{S} (\mathbb{R})$ und
% $\hat \psi_1 = 0$ in einer Umgebung von $0$ sowie
% $\mathrm{sgn}\left(\frac{x'}{\sqrt{a}}\right) = \mathrm{sgn}\left(x'\right)$  für $a>0$
% können wir also schließlich abschätzen

% \begin{align}
%     \lim_{a \rightarrow 0}\left< \hat\psi_{a1t}, \hat G_F \right>
%     &= \lim_{a \rightarrow 0} a^\frac{3}{4} C \int \frac{\mathrm{sgn}(x')
%     ~\hat \psi_1(\omega)}{\omega}
%     e^{-i\omega \frac{t'-x}{a}} \d \omega
%     \nonumber \\
%     &\sim O\left(a^\frac{3}{4}\right) \kern 1em \textrm{ ; für } t'=x'
%     \nonumber \\
%     &\sim O\left(a^k\right) ~ \forall k \in \mathbb{N} \kern 1em \textrm{;   andernfalls}
% \end{align}

% wobei in $C$ alle irrelevanten Vorfaktoren gesammelt wurden.

Das analoge Ergebnis erhält man auch für $s=-1$ und $t' = -x'$
% section berechnen_von_ (end)
