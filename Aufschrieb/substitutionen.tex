%!TEX root = main.tex
%!TEX spellcheck=de_DE
%%%%%%%%%%%%%%%%%%%%%%%%%%%%%%%%%%%%%%%%%%%%%%%%%%%%%%%%%%%%%%%%%%%%%%%%%%%%%%%
% % Section 2
%%%%%%%%%%%%%%%%%%%%%%%%%%%%%%%%%%%%%%%%%%%%%%%%%%%%%%%%%%%%%%%%%%%%%%%%%%%%%%%
\section{\texorpdfstring{Zwei nützliche Substitionen für  $\left<\psi_{ast}, f\right>$}{Zwei nützliche Substitutionen}}
\label{sec:substitutionen}

\begin{remark}[Notation]
    Da wir ab jetzt Distributionen aus der Physik betrachten, für die es üblich ist als Variablen $(t, x)$ und als Variablen im Fourierraum $(\omega, k)$ zu verwenden, schreiben wir statt $(x_1, x_2)$ ab jetzt $(t,x)$ und statt $(k_1, k_2)$ schreiben wir $(\omega, k)$. Außerdem verwenden wir das Minkowski-Skalarprodukt für die Fouriertransformation d.h.
    \begin{equation*}
        \hat f (\omega, k) \coloneqq \int f(t,x) e^{-i\omega t + i k x}
        \d t \d x
        ,
    \end{equation*}
    wieder um den Konventionen in der Physik gerecht zu werden.
\end{remark}

Zunächst werden wir zwei verschiedene Ausdrücke für $\left< f, \psi_{ast} \right>$
im Fourierraum herleiten, welche fast immer Ausgangspunkt für unsere Abschätzungen sein werden.

Sei also $\psi$ ein Shearlet wie in \cref{cor:psi_hat}. Sei $f$ die zu
analysierende fouriertransformierbare Funktion (oder Distribution) in
$\mathcal{S}' (\mathbb{R}^2)$. Dann ist $\mathcal{S}_f (ast)$ gegeben durch

\begin{align*}
\left<f, \psi_{ast}\right> &= \left< \hat f, \hat\psi_{ast}\right> \\
 &= \int a^{\frac{3}{4}} e^{-i \omega t + ikx} \hat \psi_1(a \omega)
    \hat \psi_2 \left(a^{-\frac{1}{2}} \left(\frac{k}{\omega} - s\right)\right)
    \hat f (\omega,k) \d \omega \d k
\end{align*}

und nach "`entscheren"' und "`deskalieren"', also der Substitution

\begin{equation}
\begin{aligned}[c]
a \omega_1 &= \omega'\\
a^{-\frac{1}{2}} \left(\frac{k}{\omega} - s\right) &=\frac{k'}{\omega'}\\
\end{aligned}
\qquad\Longleftrightarrow\qquad
\begin{aligned}[c]
\omega &= \frac{\omega'}{a}\\
k &= \frac{\omega' s}{a} + a^{-\frac{1}{2}} k'\\
\end{aligned}
\label{eq:substitution1_coords}
\end{equation}

\begin{equation*}
\Rightarrow
\d \omega \d k = a^{-\frac{3}{2}} \d \omega' \d k'
\end{equation*}

ergibt sich folgendes für $\left<\psi_{ast}, f\right>$:

\begin{align}
    \left\langle f , \psi_{ast}\right\rangle
    &=  \left\langle \hat f, \hat\psi_{ast}\right\rangle \nonumber \\
    &=  \iint a^{-\frac{3}{4}}~\hat \psi_1(\omega') ~\hat \psi_2 \left(\tfrac{k'}{\omega'}\right)
    ~\hat f \left(\tfrac{\omega'}{a}, \tfrac{\omega' s}{a} + \tfrac{k'}{\sqrt{a}}\right)
    ~e^{-i\frac{\omega'}{a}(t'+sx') - i \frac{k' x'}{\sqrt a}}
    \d \omega' \d k'
\refstepcounter{equation}
\tag{Substitution 1, (\theequation)}
\label{eq:substitution1}
\end{align}

Wie man sieht, tauchen in den Argumente von $\hat\psi_1$ und $\hat\psi_2$ nun die Parameter $a,s,t$ gar nicht mehr auf, und wir können nun verwenden, was wir aus \eqref{sec:shearlets} über deren Träger wissen.
Alternativ und mit ähnlichem Ergebnis kann auch folgende Substitution

\begin{equation}
\begin{aligned}[c]
a \omega &= \omega'\\
a^{-\frac{1}{2}} \left(\frac{k}{\omega} - s\right) &= k'\\
\end{aligned}
\qquad\Longleftrightarrow\qquad
\begin{aligned}[c]
\omega &= \frac{\omega'}{a}\\
k &= \left( a^{\frac{1}{2}} k' +s \right) \frac{\omega'}{a}\\
\end{aligned}
\label{eq:substitution2_coords}
\end{equation}

\begin{equation*}
\Rightarrow
\d \omega \d k = a^{-\frac{3}{2}} \omega \d \omega' \d k'
\end{equation*}

gewählt werden, wodurch wieder alle Parameter $(a,s,t)$ aus den Argumenten von $\hat\psi_1, \hat\psi_2$
verschwinden und sich

\begin{align}
    \left<f, \psi_{ast}\right>
    =  \iint a^{-\frac{3}{4}}~ k_1~ \hat \psi_1(\omega')~ \hat \psi_2 (k')~
    \hat f \left(\tfrac{\omega'}{a}, \omega' \left(a^{-\frac{1}{2}}k' + s a^{-1}\right)\right)
    ~e^{-i \omega' \left(\frac{t'+s x'}{a} + \frac{k' x'}{\sqrt{a}}\right)}
    \d \omega' \d k'
\refstepcounter{equation}
\tag{Substitution 2, (\theequation)}
\label{eq:substitution2}
\end{align}

ergibt. Dabei ist zu beachten, dass diese Substitution zulässig ist, obwohl sie
die Orientierung \emph{nicht} erhält und \emph{keine} Bijektion ist. Aber
der kritische Bereich, nämlich $\omega = 0$, liegt nicht im Träger von $\hat{\psi}$.

Beiden Substitution gemein ist aber, dass danach
$0=\omega \notin supp (\hat\psi)$ und dass $supp (\psi)$ sowohl in $k$ als auch in $\omega$ beschränkt ist. $\omega$ kann also sowohl nach unten als auch nach oben durch eine Konstante abgeschätzt werden, wann immer dies der Sache dienlich ist. Auch $k$ kann zumindest nach oben immer durch eine Konstante abgeschätzt werden.

\begin{figure}[h]
    \centering
    \begin{minipage}{0.5\textwidth}
        \centering
        \resizebox{\textwidth}{!}{%% Creator: Matplotlib, PGF backend
%%
%% To include the figure in your LaTeX document, write
%%   \input{<filename>.pgf}
%%
%% Make sure the required packages are loaded in your preamble
%%   \usepackage{pgf}
%%
%% Figures using additional raster images can only be included by \input if
%% they are in the same directory as the main LaTeX file. For loading figures
%% from other directories you can use the `import` package
%%   \usepackage{import}
%% and then include the figures with
%%   \import{<path to file>}{<filename>.pgf}
%%
%% Matplotlib used the following preamble
%%   \usepackage[utf8x]{inputenc}
%%   \usepackage[T1]{fontenc}
%%   \usepackage{amssymb}
%%
\begingroup%
\makeatletter%
\begin{pgfpicture}%
\pgfpathrectangle{\pgfpointorigin}{\pgfqpoint{4.000000in}{2.800000in}}%
\pgfusepath{use as bounding box, clip}%
\begin{pgfscope}%
\pgfsetbuttcap%
\pgfsetmiterjoin%
\definecolor{currentfill}{rgb}{1.000000,1.000000,1.000000}%
\pgfsetfillcolor{currentfill}%
\pgfsetlinewidth{0.000000pt}%
\definecolor{currentstroke}{rgb}{1.000000,1.000000,1.000000}%
\pgfsetstrokecolor{currentstroke}%
\pgfsetdash{}{0pt}%
\pgfpathmoveto{\pgfqpoint{0.000000in}{0.000000in}}%
\pgfpathlineto{\pgfqpoint{4.000000in}{0.000000in}}%
\pgfpathlineto{\pgfqpoint{4.000000in}{2.800000in}}%
\pgfpathlineto{\pgfqpoint{0.000000in}{2.800000in}}%
\pgfpathclose%
\pgfusepath{fill}%
\end{pgfscope}%
\begin{pgfscope}%
\pgfsetbuttcap%
\pgfsetmiterjoin%
\definecolor{currentfill}{rgb}{1.000000,1.000000,1.000000}%
\pgfsetfillcolor{currentfill}%
\pgfsetlinewidth{0.000000pt}%
\definecolor{currentstroke}{rgb}{0.000000,0.000000,0.000000}%
\pgfsetstrokecolor{currentstroke}%
\pgfsetstrokeopacity{0.000000}%
\pgfsetdash{}{0pt}%
\pgfpathmoveto{\pgfqpoint{0.198611in}{0.198611in}}%
\pgfpathlineto{\pgfqpoint{3.801389in}{0.198611in}}%
\pgfpathlineto{\pgfqpoint{3.801389in}{2.601389in}}%
\pgfpathlineto{\pgfqpoint{0.198611in}{2.601389in}}%
\pgfpathclose%
\pgfusepath{fill}%
\end{pgfscope}%
\begin{pgfscope}%
\pgfpathrectangle{\pgfqpoint{0.198611in}{0.198611in}}{\pgfqpoint{3.602778in}{2.402778in}}%
\pgfusepath{clip}%
\pgfsetbuttcap%
\pgfsetmiterjoin%
\definecolor{currentfill}{rgb}{0.500000,0.500000,0.500000}%
\pgfsetfillcolor{currentfill}%
\pgfsetfillopacity{0.500000}%
\pgfsetlinewidth{0.501875pt}%
\definecolor{currentstroke}{rgb}{0.000000,0.000000,0.000000}%
\pgfsetstrokecolor{currentstroke}%
\pgfsetdash{}{0pt}%
\pgfpathmoveto{\pgfqpoint{1.963972in}{1.433372in}}%
\pgfpathlineto{\pgfqpoint{2.036028in}{1.433372in}}%
\pgfpathlineto{\pgfqpoint{2.144111in}{1.533488in}}%
\pgfpathlineto{\pgfqpoint{1.855889in}{1.533488in}}%
\pgfpathclose%
\pgfusepath{stroke,fill}%
\end{pgfscope}%
\begin{pgfscope}%
\pgfpathrectangle{\pgfqpoint{0.198611in}{0.198611in}}{\pgfqpoint{3.602778in}{2.402778in}}%
\pgfusepath{clip}%
\pgfsetbuttcap%
\pgfsetmiterjoin%
\definecolor{currentfill}{rgb}{0.500000,0.500000,0.500000}%
\pgfsetfillcolor{currentfill}%
\pgfsetfillopacity{0.500000}%
\pgfsetlinewidth{0.501875pt}%
\definecolor{currentstroke}{rgb}{0.000000,0.000000,0.000000}%
\pgfsetstrokecolor{currentstroke}%
\pgfsetdash{}{0pt}%
\pgfpathmoveto{\pgfqpoint{2.036028in}{1.366628in}}%
\pgfpathlineto{\pgfqpoint{1.963972in}{1.366628in}}%
\pgfpathlineto{\pgfqpoint{1.855889in}{1.266512in}}%
\pgfpathlineto{\pgfqpoint{2.144111in}{1.266512in}}%
\pgfpathclose%
\pgfusepath{stroke,fill}%
\end{pgfscope}%
\begin{pgfscope}%
\pgfpathrectangle{\pgfqpoint{0.198611in}{0.198611in}}{\pgfqpoint{3.602778in}{2.402778in}}%
\pgfusepath{clip}%
\pgfsetbuttcap%
\pgfsetmiterjoin%
\definecolor{currentfill}{rgb}{0.500000,0.500000,0.500000}%
\pgfsetfillcolor{currentfill}%
\pgfsetfillopacity{0.500000}%
\pgfsetlinewidth{0.501875pt}%
\definecolor{currentstroke}{rgb}{0.000000,0.000000,0.000000}%
\pgfsetstrokecolor{currentstroke}%
\pgfsetdash{}{0pt}%
\pgfpathmoveto{\pgfqpoint{2.246348in}{1.733719in}}%
\pgfpathlineto{\pgfqpoint{2.474208in}{1.733719in}}%
\pgfpathlineto{\pgfqpoint{3.896830in}{2.734877in}}%
\pgfpathlineto{\pgfqpoint{2.985392in}{2.734877in}}%
\pgfpathclose%
\pgfusepath{stroke,fill}%
\end{pgfscope}%
\begin{pgfscope}%
\pgfpathrectangle{\pgfqpoint{0.198611in}{0.198611in}}{\pgfqpoint{3.602778in}{2.402778in}}%
\pgfusepath{clip}%
\pgfsetbuttcap%
\pgfsetmiterjoin%
\definecolor{currentfill}{rgb}{0.500000,0.500000,0.500000}%
\pgfsetfillcolor{currentfill}%
\pgfsetfillopacity{0.500000}%
\pgfsetlinewidth{0.501875pt}%
\definecolor{currentstroke}{rgb}{0.000000,0.000000,0.000000}%
\pgfsetstrokecolor{currentstroke}%
\pgfsetdash{}{0pt}%
\pgfpathmoveto{\pgfqpoint{1.753652in}{1.066281in}}%
\pgfpathlineto{\pgfqpoint{1.525792in}{1.066281in}}%
\pgfpathlineto{\pgfqpoint{0.103170in}{0.065123in}}%
\pgfpathlineto{\pgfqpoint{1.014608in}{0.065123in}}%
\pgfpathclose%
\pgfusepath{stroke,fill}%
\end{pgfscope}%
\begin{pgfscope}%
\pgfpathrectangle{\pgfqpoint{0.198611in}{0.198611in}}{\pgfqpoint{3.602778in}{2.402778in}}%
\pgfusepath{clip}%
\pgfsetbuttcap%
\pgfsetroundjoin%
\pgfsetlinewidth{0.501875pt}%
\definecolor{currentstroke}{rgb}{0.501961,0.501961,0.501961}%
\pgfsetstrokecolor{currentstroke}%
\pgfsetdash{{1.850000pt}{0.800000pt}}{0.000000pt}%
\pgfpathmoveto{\pgfqpoint{0.688006in}{0.184722in}}%
\pgfpathlineto{\pgfqpoint{3.311994in}{2.615278in}}%
\pgfpathlineto{\pgfqpoint{3.311994in}{2.615278in}}%
\pgfusepath{stroke}%
\end{pgfscope}%
\begin{pgfscope}%
\pgfpathrectangle{\pgfqpoint{0.198611in}{0.198611in}}{\pgfqpoint{3.602778in}{2.402778in}}%
\pgfusepath{clip}%
\pgfsetbuttcap%
\pgfsetroundjoin%
\pgfsetlinewidth{0.501875pt}%
\definecolor{currentstroke}{rgb}{0.501961,0.501961,0.501961}%
\pgfsetstrokecolor{currentstroke}%
\pgfsetdash{{1.850000pt}{0.800000pt}}{0.000000pt}%
\pgfpathmoveto{\pgfqpoint{0.688006in}{2.615278in}}%
\pgfpathlineto{\pgfqpoint{3.311994in}{0.184722in}}%
\pgfpathlineto{\pgfqpoint{3.311994in}{0.184722in}}%
\pgfusepath{stroke}%
\end{pgfscope}%
\begin{pgfscope}%
\pgfsetrectcap%
\pgfsetmiterjoin%
\pgfsetlinewidth{0.501875pt}%
\definecolor{currentstroke}{rgb}{0.000000,0.000000,0.000000}%
\pgfsetstrokecolor{currentstroke}%
\pgfsetdash{}{0pt}%
\pgfpathmoveto{\pgfqpoint{2.000000in}{0.198611in}}%
\pgfpathlineto{\pgfqpoint{2.000000in}{2.601389in}}%
\pgfusepath{stroke}%
\end{pgfscope}%
\begin{pgfscope}%
\pgfsetrectcap%
\pgfsetmiterjoin%
\pgfsetlinewidth{0.501875pt}%
\definecolor{currentstroke}{rgb}{0.000000,0.000000,0.000000}%
\pgfsetstrokecolor{currentstroke}%
\pgfsetdash{}{0pt}%
\pgfpathmoveto{\pgfqpoint{0.198611in}{1.400000in}}%
\pgfpathlineto{\pgfqpoint{3.801389in}{1.400000in}}%
\pgfusepath{stroke}%
\end{pgfscope}%
\begin{pgfscope}%
\pgfsetroundcap%
\pgfsetroundjoin%
\pgfsetlinewidth{0.501875pt}%
\definecolor{currentstroke}{rgb}{0.000000,0.000000,0.000000}%
\pgfsetstrokecolor{currentstroke}%
\pgfsetdash{}{0pt}%
\pgfpathmoveto{\pgfqpoint{3.032230in}{2.375700in}}%
\pgfpathquadraticcurveto{\pgfqpoint{2.526526in}{1.930390in}}{\pgfqpoint{2.026649in}{1.490210in}}%
\pgfusepath{stroke}%
\end{pgfscope}%
\begin{pgfscope}%
\pgfsetroundcap%
\pgfsetroundjoin%
\pgfsetlinewidth{0.501875pt}%
\definecolor{currentstroke}{rgb}{0.000000,0.000000,0.000000}%
\pgfsetstrokecolor{currentstroke}%
\pgfsetdash{}{0pt}%
\pgfpathmoveto{\pgfqpoint{2.086701in}{1.506078in}}%
\pgfpathlineto{\pgfqpoint{2.026649in}{1.490210in}}%
\pgfpathlineto{\pgfqpoint{2.049986in}{1.547773in}}%
\pgfusepath{stroke}%
\end{pgfscope}%
\begin{pgfscope}%
\pgftext[x=3.080833in,y=2.401157in,left,base]{\rmfamily\fontsize{10.000000}{12.000000}\selectfont \(\displaystyle {\cdot}\)}%
\end{pgfscope}%
\begin{pgfscope}%
\pgftext[x=2.288222in,y=1.600231in,left,base]{\rmfamily\fontsize{10.000000}{12.000000}\selectfont Substitution 1}%
\end{pgfscope}%
\begin{pgfscope}%
\pgfsetroundcap%
\pgfsetroundjoin%
\pgfsetlinewidth{0.501875pt}%
\definecolor{currentstroke}{rgb}{0.000000,0.000000,0.000000}%
\pgfsetstrokecolor{currentstroke}%
\pgfsetdash{}{0pt}%
\pgfpathmoveto{\pgfqpoint{2.000000in}{2.607510in}}%
\pgfpathquadraticcurveto{\pgfqpoint{2.000000in}{2.608331in}}{\pgfqpoint{2.000000in}{2.601389in}}%
\pgfusepath{stroke}%
\end{pgfscope}%
\begin{pgfscope}%
\pgfsetroundcap%
\pgfsetroundjoin%
\pgfsetlinewidth{0.501875pt}%
\definecolor{currentstroke}{rgb}{0.000000,0.000000,0.000000}%
\pgfsetstrokecolor{currentstroke}%
\pgfsetdash{}{0pt}%
\pgfpathmoveto{\pgfqpoint{1.972222in}{2.551954in}}%
\pgfpathlineto{\pgfqpoint{2.000000in}{2.607510in}}%
\pgfpathlineto{\pgfqpoint{2.027778in}{2.551954in}}%
\pgfusepath{stroke}%
\end{pgfscope}%
\begin{pgfscope}%
\pgftext[x=2.000000in,y=2.670833in,,bottom]{\rmfamily\fontsize{10.000000}{12.000000}\selectfont \(\displaystyle \omega\)}%
\end{pgfscope}%
\begin{pgfscope}%
\pgfsetroundcap%
\pgfsetroundjoin%
\pgfsetlinewidth{0.501875pt}%
\definecolor{currentstroke}{rgb}{0.000000,0.000000,0.000000}%
\pgfsetstrokecolor{currentstroke}%
\pgfsetdash{}{0pt}%
\pgfpathmoveto{\pgfqpoint{3.807488in}{1.400000in}}%
\pgfpathquadraticcurveto{\pgfqpoint{3.808320in}{1.400000in}}{\pgfqpoint{3.801389in}{1.400000in}}%
\pgfusepath{stroke}%
\end{pgfscope}%
\begin{pgfscope}%
\pgfsetroundcap%
\pgfsetroundjoin%
\pgfsetlinewidth{0.501875pt}%
\definecolor{currentstroke}{rgb}{0.000000,0.000000,0.000000}%
\pgfsetstrokecolor{currentstroke}%
\pgfsetdash{}{0pt}%
\pgfpathmoveto{\pgfqpoint{3.751932in}{1.427778in}}%
\pgfpathlineto{\pgfqpoint{3.807488in}{1.400000in}}%
\pgfpathlineto{\pgfqpoint{3.751932in}{1.372222in}}%
\pgfusepath{stroke}%
\end{pgfscope}%
\begin{pgfscope}%
\pgftext[x=3.870833in,y=1.400000in,left,]{\rmfamily\fontsize{10.000000}{12.000000}\selectfont \(\displaystyle k\)}%
\end{pgfscope}%
\end{pgfpicture}%
\makeatother%
\endgroup%
} %
        \caption{Der Träger von $\hat\psi$ vor und nach der Substitution aus \cref{eq:substitution1_coords}}
        \label{fig:supp_psi_substitution1}
    \end{minipage}\hfill
    \begin{minipage}{0.5\textwidth}
        \centering
        \resizebox{\textwidth}{!}{%% Creator: Matplotlib, PGF backend
%%
%% To include the figure in your LaTeX document, write
%%   \input{<filename>.pgf}
%%
%% Make sure the required packages are loaded in your preamble
%%   \usepackage{pgf}
%%
%% Figures using additional raster images can only be included by \input if
%% they are in the same directory as the main LaTeX file. For loading figures
%% from other directories you can use the `import` package
%%   \usepackage{import}
%% and then include the figures with
%%   \import{<path to file>}{<filename>.pgf}
%%
%% Matplotlib used the following preamble
%%   \usepackage[utf8x]{inputenc}
%%   \usepackage[T1]{fontenc}
%%   \usepackage{amssymb}
%%
\begingroup%
\makeatletter%
\begin{pgfpicture}%
\pgfpathrectangle{\pgfpointorigin}{\pgfqpoint{4.000000in}{2.000000in}}%
\pgfusepath{use as bounding box, clip}%
\begin{pgfscope}%
\pgfsetbuttcap%
\pgfsetmiterjoin%
\definecolor{currentfill}{rgb}{1.000000,1.000000,1.000000}%
\pgfsetfillcolor{currentfill}%
\pgfsetlinewidth{0.000000pt}%
\definecolor{currentstroke}{rgb}{1.000000,1.000000,1.000000}%
\pgfsetstrokecolor{currentstroke}%
\pgfsetdash{}{0pt}%
\pgfpathmoveto{\pgfqpoint{0.000000in}{0.000000in}}%
\pgfpathlineto{\pgfqpoint{4.000000in}{0.000000in}}%
\pgfpathlineto{\pgfqpoint{4.000000in}{2.000000in}}%
\pgfpathlineto{\pgfqpoint{0.000000in}{2.000000in}}%
\pgfpathclose%
\pgfusepath{fill}%
\end{pgfscope}%
\begin{pgfscope}%
\pgfsetbuttcap%
\pgfsetmiterjoin%
\definecolor{currentfill}{rgb}{1.000000,1.000000,1.000000}%
\pgfsetfillcolor{currentfill}%
\pgfsetlinewidth{0.000000pt}%
\definecolor{currentstroke}{rgb}{0.000000,0.000000,0.000000}%
\pgfsetstrokecolor{currentstroke}%
\pgfsetstrokeopacity{0.000000}%
\pgfsetdash{}{0pt}%
\pgfpathmoveto{\pgfqpoint{0.198611in}{0.198611in}}%
\pgfpathlineto{\pgfqpoint{3.801389in}{0.198611in}}%
\pgfpathlineto{\pgfqpoint{3.801389in}{1.801389in}}%
\pgfpathlineto{\pgfqpoint{0.198611in}{1.801389in}}%
\pgfpathclose%
\pgfusepath{fill}%
\end{pgfscope}%
\begin{pgfscope}%
\pgfpathrectangle{\pgfqpoint{0.198611in}{0.198611in}}{\pgfqpoint{3.602778in}{1.602778in}} %
\pgfusepath{clip}%
\pgfsetbuttcap%
\pgfsetmiterjoin%
\definecolor{currentfill}{rgb}{0.500000,0.500000,0.500000}%
\pgfsetfillcolor{currentfill}%
\pgfsetfillopacity{0.500000}%
\pgfsetlinewidth{0.501875pt}%
\definecolor{currentstroke}{rgb}{0.000000,0.000000,0.000000}%
\pgfsetstrokecolor{currentstroke}%
\pgfsetdash{}{0pt}%
\pgfpathmoveto{\pgfqpoint{1.909931in}{0.398958in}}%
\pgfpathlineto{\pgfqpoint{1.909931in}{0.519167in}}%
\pgfpathlineto{\pgfqpoint{2.090069in}{0.519167in}}%
\pgfpathlineto{\pgfqpoint{2.090069in}{0.398958in}}%
\pgfpathclose%
\pgfusepath{stroke,fill}%
\end{pgfscope}%
\begin{pgfscope}%
\pgfpathrectangle{\pgfqpoint{0.198611in}{0.198611in}}{\pgfqpoint{3.602778in}{1.602778in}} %
\pgfusepath{clip}%
\pgfsetbuttcap%
\pgfsetmiterjoin%
\definecolor{currentfill}{rgb}{0.500000,0.500000,0.500000}%
\pgfsetfillcolor{currentfill}%
\pgfsetfillopacity{0.500000}%
\pgfsetlinewidth{0.501875pt}%
\definecolor{currentstroke}{rgb}{0.000000,0.000000,0.000000}%
\pgfsetstrokecolor{currentstroke}%
\pgfsetdash{}{0pt}%
\pgfpathmoveto{\pgfqpoint{2.307935in}{0.759583in}}%
\pgfpathlineto{\pgfqpoint{2.592760in}{0.759583in}}%
\pgfpathlineto{\pgfqpoint{4.371038in}{1.961667in}}%
\pgfpathlineto{\pgfqpoint{3.231740in}{1.961667in}}%
\pgfpathclose%
\pgfusepath{stroke,fill}%
\end{pgfscope}%
\begin{pgfscope}%
\pgfpathrectangle{\pgfqpoint{0.198611in}{0.198611in}}{\pgfqpoint{3.602778in}{1.602778in}} %
\pgfusepath{clip}%
\pgfsetbuttcap%
\pgfsetroundjoin%
\pgfsetlinewidth{0.501875pt}%
\definecolor{currentstroke}{rgb}{0.501961,0.501961,0.501961}%
\pgfsetstrokecolor{currentstroke}%
\pgfsetdash{{1.850000pt}{0.800000pt}}{0.000000pt}%
\pgfpathmoveto{\pgfqpoint{1.804251in}{0.184722in}}%
\pgfpathlineto{\pgfqpoint{3.636860in}{1.815278in}}%
\pgfpathlineto{\pgfqpoint{3.636860in}{1.815278in}}%
\pgfusepath{stroke}%
\end{pgfscope}%
\begin{pgfscope}%
\pgfpathrectangle{\pgfqpoint{0.198611in}{0.198611in}}{\pgfqpoint{3.602778in}{1.602778in}} %
\pgfusepath{clip}%
\pgfsetbuttcap%
\pgfsetroundjoin%
\pgfsetlinewidth{0.501875pt}%
\definecolor{currentstroke}{rgb}{0.501961,0.501961,0.501961}%
\pgfsetstrokecolor{currentstroke}%
\pgfsetdash{{1.850000pt}{0.800000pt}}{0.000000pt}%
\pgfpathmoveto{\pgfqpoint{0.363140in}{1.815278in}}%
\pgfpathlineto{\pgfqpoint{2.195749in}{0.184722in}}%
\pgfpathlineto{\pgfqpoint{2.195749in}{0.184722in}}%
\pgfusepath{stroke}%
\end{pgfscope}%
\begin{pgfscope}%
\pgfsetrectcap%
\pgfsetmiterjoin%
\pgfsetlinewidth{0.501875pt}%
\definecolor{currentstroke}{rgb}{0.000000,0.000000,0.000000}%
\pgfsetstrokecolor{currentstroke}%
\pgfsetdash{}{0pt}%
\pgfpathmoveto{\pgfqpoint{2.000000in}{0.198611in}}%
\pgfpathlineto{\pgfqpoint{2.000000in}{1.801389in}}%
\pgfusepath{stroke}%
\end{pgfscope}%
\begin{pgfscope}%
\pgfsetrectcap%
\pgfsetmiterjoin%
\pgfsetlinewidth{0.501875pt}%
\definecolor{currentstroke}{rgb}{0.000000,0.000000,0.000000}%
\pgfsetstrokecolor{currentstroke}%
\pgfsetdash{}{0pt}%
\pgfpathmoveto{\pgfqpoint{0.198611in}{0.358889in}}%
\pgfpathlineto{\pgfqpoint{3.801389in}{0.358889in}}%
\pgfusepath{stroke}%
\end{pgfscope}%
\begin{pgfscope}%
\pgfsetroundcap%
\pgfsetroundjoin%
\pgfsetlinewidth{0.501875pt}%
\definecolor{currentstroke}{rgb}{0.000000,0.000000,0.000000}%
\pgfsetstrokecolor{currentstroke}%
\pgfsetdash{}{0pt}%
\pgfpathmoveto{\pgfqpoint{3.302084in}{1.537713in}}%
\pgfpathquadraticcurveto{\pgfqpoint{2.661671in}{0.997340in}}{\pgfqpoint{2.027193in}{0.461973in}}%
\pgfusepath{stroke}%
\end{pgfscope}%
\begin{pgfscope}%
\pgfsetroundcap%
\pgfsetroundjoin%
\pgfsetlinewidth{0.501875pt}%
\definecolor{currentstroke}{rgb}{0.000000,0.000000,0.000000}%
\pgfsetstrokecolor{currentstroke}%
\pgfsetdash{}{0pt}%
\pgfpathmoveto{\pgfqpoint{2.087566in}{0.476570in}}%
\pgfpathlineto{\pgfqpoint{2.027193in}{0.461973in}}%
\pgfpathlineto{\pgfqpoint{2.051739in}{0.519030in}}%
\pgfusepath{stroke}%
\end{pgfscope}%
\begin{pgfscope}%
\pgftext[x=3.351042in,y=1.560972in,left,base]{\rmfamily\fontsize{10.000000}{12.000000}\selectfont \(\displaystyle {\cdot}\)}%
\end{pgfscope}%
\begin{pgfscope}%
\pgftext[x=2.360278in,y=0.599306in,left,base]{\rmfamily\fontsize{10.000000}{12.000000}\selectfont Substitution 1}%
\end{pgfscope}%
\begin{pgfscope}%
\pgfsetroundcap%
\pgfsetroundjoin%
\pgfsetlinewidth{0.501875pt}%
\definecolor{currentstroke}{rgb}{0.000000,0.000000,0.000000}%
\pgfsetstrokecolor{currentstroke}%
\pgfsetdash{}{0pt}%
\pgfpathmoveto{\pgfqpoint{2.000000in}{1.807510in}}%
\pgfpathquadraticcurveto{\pgfqpoint{2.000000in}{1.808331in}}{\pgfqpoint{2.000000in}{1.801389in}}%
\pgfusepath{stroke}%
\end{pgfscope}%
\begin{pgfscope}%
\pgfsetroundcap%
\pgfsetroundjoin%
\pgfsetlinewidth{0.501875pt}%
\definecolor{currentstroke}{rgb}{0.000000,0.000000,0.000000}%
\pgfsetstrokecolor{currentstroke}%
\pgfsetdash{}{0pt}%
\pgfpathmoveto{\pgfqpoint{1.972222in}{1.751954in}}%
\pgfpathlineto{\pgfqpoint{2.000000in}{1.807510in}}%
\pgfpathlineto{\pgfqpoint{2.027778in}{1.751954in}}%
\pgfusepath{stroke}%
\end{pgfscope}%
\begin{pgfscope}%
\pgftext[x=2.000000in,y=1.870833in,,bottom]{\rmfamily\fontsize{10.000000}{12.000000}\selectfont \(\displaystyle \omega\)}%
\end{pgfscope}%
\begin{pgfscope}%
\pgfsetroundcap%
\pgfsetroundjoin%
\pgfsetlinewidth{0.501875pt}%
\definecolor{currentstroke}{rgb}{0.000000,0.000000,0.000000}%
\pgfsetstrokecolor{currentstroke}%
\pgfsetdash{}{0pt}%
\pgfpathmoveto{\pgfqpoint{3.807488in}{0.358889in}}%
\pgfpathquadraticcurveto{\pgfqpoint{3.808320in}{0.358889in}}{\pgfqpoint{3.801389in}{0.358889in}}%
\pgfusepath{stroke}%
\end{pgfscope}%
\begin{pgfscope}%
\pgfsetroundcap%
\pgfsetroundjoin%
\pgfsetlinewidth{0.501875pt}%
\definecolor{currentstroke}{rgb}{0.000000,0.000000,0.000000}%
\pgfsetstrokecolor{currentstroke}%
\pgfsetdash{}{0pt}%
\pgfpathmoveto{\pgfqpoint{3.751932in}{0.386667in}}%
\pgfpathlineto{\pgfqpoint{3.807488in}{0.358889in}}%
\pgfpathlineto{\pgfqpoint{3.751932in}{0.331111in}}%
\pgfusepath{stroke}%
\end{pgfscope}%
\begin{pgfscope}%
\pgftext[x=3.870833in,y=0.358889in,left,]{\rmfamily\fontsize{10.000000}{12.000000}\selectfont \(\displaystyle k\)}%
\end{pgfscope}%
\end{pgfpicture}%
\makeatother%
\endgroup%
}
        \caption{Der Träger von $\hat\psi$ vor und nach der Substitution aus \cref{eq:substitution2_coords}}
        \label{fig:supp_psi_substitution2}
    \end{minipage}
\end{figure}
