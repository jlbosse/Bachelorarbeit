%!TEX root = main.tex
%%%%%%%%%%%%%%%%%%%%%%%%%%%%%%%%%%%%%%%%%%%%%%%%%%%%%%%%%%%%%%%%%%%%%%%%%%%%%%%
% % Section 1
%%%%%%%%%%%%%%%%%%%%%%%%%%%%%%%%%%%%%%%%%%%%%%%%%%%%%%%%%%%%%%%%%%%%%%%%%%%%%%%
\section{Wavelettransformation und die Wellenfrontmenge} % (fold)
\label{sec:shearlets}

Eine Riesz-Basis oder Hamel-Basis für $L^2(\mathbb{R}^n)$ die aus Funktionen der Form

\begin{equation*}
    \left\{\psi_{at}(x) = a^{-\frac{n}{2}}\psi\left(a^{-1}(x-t)\right)  | t \in \mathbb{Z}^n \textrm{ oder } t \in \mathbb{R}^n, a = 2^{-j}, j \in \mathbb{Z} \textrm{ oder } a \in \mathbb{R}\right\}
\end{equation*}

besteht heißt \textit{Waveletbasis} oder \textit{Multiskalenanlyse} für $L^2(\mathbb{R}^n)$. Der Parameter $t$ heißt \textit{Verschiebungsparameter} und verschiebt das Wavelet an alle Orte des $\mathbb{R}^n$ während der \textit{Skalierungsparameter} $a$ für $a \to 0$ immer genauer lokalisiert. Der Faktor $a^{-\frac{n}{2}}$ sorgt dafür, dass die $L^2$-Norm aller $\psi_{at}$ gleich ist. In der Fourierdomäne wird die Verschiebung zum Phasenfaktor und der Träger mit verschwindendem $a$ immer \emph{größer}. Noch einmal als Formel:

\begin{align*}
    supp (\psi_{at}) &= a \cdot supp(\psi)+t \\
    \Longleftrightarrow ~~
    supp (\hat\psi_{at}) &= a^{-1} \cdot supp(\hat\psi_{at})
\end{align*}

Ist $t \in \mathbb{Z}^n$ (oder auf einem anderen diskreten Gitter in $\mathbb{R}^n$) und $a = 2^{-j}$, so spricht man von der \textit{diskreten Wavelettransformation}. Ist $t \in \mathbb{R}^n$ und $G \subseteq GL(n,\mathbb{R})$ so ist es eine \textit{stetige Wavelettransformation} (engl. \textit{continuous wavelet transform}). Die Wavelettransformation einer Funktion (später auch temperierten Distribution) $f$ ist dann definiert als

\begin{equation}
    \mathcal{W}_f (a,t) = a^{-\frac{n}{2}} \int f(x) \psi \left(a^{-1}(x-t)\right) \d x
\end{equation}

Ist $\psi$ eine glatte Funktion und $f$ bei $t$ glatt, so fällt $\mathcal{W}_f (a,t)$ schnell ab für $a \to 0$.
% Dies ist schnell einzusehen, wenn man sich überlegt, dass $\psi \in C^k$ impliziert, dass $\hat\psi$ eine $k$-fache Nullstelle bei $0$ hat und $\int x^l \psi(x) \d x = (-i)^{l} \partial_k^l \hat\psi(0) = 0 ~~\textrm{falls}~ l<k $.

\todo[color=green]{Kurze Plausibel machen, warum dem so ist?}
Umgekehrt fällt auch $\mathcal{W}_f(a,t)$ \emph{genau dann} nicht schnell ab, wenn $f$ bei $t$ \emph{nicht} glatt ist. Also ist die Wavelettransformation in der Lage $sing~ supp (f)$ zu identifizieren. Allerdings sind die klassischen  Wavelettransformationen mit isotroper Skalierung nicht in der Lage die  Orientierung der Singularitäten aufzulösen. Sie besitzen ja gar keinen Orientierungsparameter.

Um dem abzuhelfen, oder besser: um auch dieses Feature noch zu erhalten, gibt es verschiedene Verallgemeinerungen der Wavelets, die mit anisotropischer Skalierung und einem Richtungsparameter arbeiten. Die bekanntesten Beispiele sind die \textit{Curvelets} von \textcite{Candes2005} sowie die \textit{Shearlets} von \textcite{Kutyniok2008}. Beide Ansätze sind bisher nur in zwei Dimensionen im Detail untersucht \todo{besseres Wort} und arbeiten mit parabolischer Skalierung. Im Fall der \textit{Curvelets} wird die Richtungsabhängigkeit durch Drehmatrizen implementiert, während die \textit{Shearlets} mit Scherungen arbeiten. Konkret sieht dass dann so aus:

\todo[color=green]{Hier die Formeln hin schreiben, oder ganz drauf verzichten}

\begin{align*}
    \psi_{a\theta t}(x) &= \det (AR)^{-\frac{1}{2}}
                    \psi \left((AR)^{-1}(x-t) \right)
    \condition{für Curvelets}\\
    \psi_{ast}(x) &= \det (AS)^{-\frac{1}{2}}
                    \psi \left((AS)^{-1}(x-t)\right)
    \condition{für Shearlets}
\end{align*}

\begin{dgroup*}
\begin{dsuspend}
    mit der parabolischen Skalierungsmatrix
\end{dsuspend}
\begin{dmath*}
    A = \begin{pmatrix}  a & 0 \\ 0 & \sqrt a\end{pmatrix}
\end{dmath*}
\begin{dsuspend}
    der Drehmatrix
\end{dsuspend}
\begin{dmath*}
    R = \begin{pmatrix}  \cos \theta & \sin \theta \\ - \sin \theta & \cos \theta \end{pmatrix}
\end{dmath*}
\begin{dsuspend}
    und der Scherungsmatrix
\end{dsuspend}
\begin{dmath*}
    S = \begin{pmatrix}  1 & -s \\ 0 & 1\end{pmatrix}
\end{dmath*}
\end{dgroup*}

Beide Ansätze sind in der Lage, die Wellenfrontmenge einer Distribution zu identifizieren. Allerdings sind die Rechnungen bei den \textit{Shearlets} in der praktischen Umsetzung einfacher, wenn auch sie von einem ästhetischen Standpunkt nicht ganz so befriedigend, da sie nicht inhärent Rotationsinvariant sind, also nicht alle Symmetrien unseres Raumes abbilden. Aber nach allzu viel Ästhetik kann man in dieser Arbeit, mit Hinblick auf die Rechnungen ab \cref{sec:die_wellenfrontmenge_von_delta_m}, ohnehin nicht fragen.

Jetzt also in mehr Details zur Konstruktion der Shearlets und deren Eigenschaften:


%%%%%%%%%%%%%%%%%%%%%%%%%%%%%%%%%%%%%%%%%%%%%%%%%%%%%%%%%%%%%%%%%%%%%%%%%%%%%%%
% % Section 2, die Shearlets
%%%%%%%%%%%%%%%%%%%%%%%%%%%%%%%%%%%%%%%%%%%%%%%%%%%%%%%%%%%%%%%%%%%%%%%%%%%%%%%
\subsection{Konstruktion und Eigenschaften der Shearlets} % (fold)
\label{sec:konstruktion_und_eigenschaften_der_shearlets}

\todo[color=green]{In dieser Sektion nur die wichtigsten Ergebnisse des Papers angeben, oder auch Beweise oder zumindest Beweisskizzen, damit man sieht wie alles zusammen spielt?}

Der folgende Abschnitt basiert größtenteils auf der Arbeit in \textcite{Kutyniok2008}, allerdings wurde die Notation den später berechneten Problemen angepasst und als Ortskoordination $(t,x)$ verwendet sowie als Koordinaten des Dualraums $(\omega, k)$ entsprechend der Konvention in der Physik. Auch Fouriertransformation wird mit $e^{-i\omega t + ikx}$ gerechnet. Dies ändert allerdings nichts essentielles. Da wir später auch komplexwertige Distributionen analysieren wollen, deren Wellenfrontmenge nicht zwingend punktsymmetrisch um den Ursprung (in der Richtung, nicht im Ort) sind, werden wir Shearlets verwenden, deren Fouriertransformierte asymetrischen Träger hat, indem wir die Shearlets aus \cite{Kutyniok2008} jeweils in zwei Shearlets aufteilen, eines mit Träger im Frequenzbereich "`nach vorne"', und eines mit Träger "`nach hinten"'. \todo{Grafik mit der "Parzellierung" des Frequenzbereiches}

\begin{definition}[Shearlettransformation]
ssg
\end{definition}

% section konstruktion_und_eigenschaften_der_shearlets (end)

\begin{proposition}[$\psi_{ast}$ fällt schnell ab]
\label{prop:shearlets_decay_rapidly}
Sei $\psi \in L^2(\mathbb{R}^2)$ ein Shearlet wie definiert und $M$ so ne Trafomatrix. Dann gilt für alle $k \in  \mathbb{N}$, dass es eine konstante $C_k$ gibt s.d. für alle $x \in \mathbb{R}^2$ gilt

\begin{dmath*}
    \left| \psi_{ast}(x) \right|
    \leq
    C_k \left| \det M \right|^{-\frac{1}{2}}\left(1+|M^{-1}(x-t)|^2\right)^{-k}
    = C_k a^{-\frac{3}{4}}\left(1+a^{-2}\left(x_1-t_1\right)^2
        + 2 a^{-2}s\left(x_1-t_1\right)\left(x_2-t_2\right)
        + a^{-1}\left(1+a^{-1}s^2\right)\left(x_2-t_2\right)^2
    \right)^{-k}
\end{dmath*}

Und insbesondere ist $C_k = \frac{15}{2}\frac{\sqrt{a} + s}{a^2}\left(\Vert \hat\psi \Vert_\infty + \Vert \Laplace^k \hat\psi \Vert_\infty\right)$

\end{proposition}

\begin{theorem}[$\mathcal{S}_f(a,s,t)$ misst $WF(f)$]
\label{thm:main_theorem}
    Sei $\mathcal{D} = \mathcal{D}_1 \cup \mathcal{D}_2$ wobei
    $\mathcal{D}_1$ = \{
        $(t_0, s_0) \in \mathbb{R}^2 \times [-1,1] \big|$
        $|\mathcal{S}_f (a, s, t)| = O(a^k)$ gleichmäßig $\forall k \in \mathbb{N}
        , \forall t \in U$ Umgebung von $(t_0, s_0)$
    \}
    und $\mathcal{D}_2$ analog für $\psi^{(v)}$

    Dann gilt $WF(f)^c = \mathcal{D}$
\end{theorem}

\todo{diesen Satz richtig hin schreiben und ordentlich setzen}
\todo{Stil und Nummerierung für Sätze, Propositionen etc. anpassen}

\begin{corollary}[WF(f) misst $sing ~supp (\psi)$]
Sei $\mathcal{R} =$ \{
    $t_0 \in \mathcal{R}^2 \big|$ $|\mathcal{S}_f(a,s,t)| = O(a^k)$
    $\forall k \in \mathbb{N}, \forall t \in U$ Umgebung von $t_0$
    \}

    Dann gilt $sing ~supp (\psi)^c = \mathcal{R}$
\end{corollary}

\begin{remark}[Träger von $\hat\psi_{ast}$]

\begin{figure}[h]
\centering
%% Creator: Matplotlib, PGF backend
%%
%% To include the figure in your LaTeX document, write
%%   \input{<filename>.pgf}
%%
%% Make sure the required packages are loaded in your preamble
%%   \usepackage{pgf}
%%
%% Figures using additional raster images can only be included by \input if
%% they are in the same directory as the main LaTeX file. For loading figures
%% from other directories you can use the `import` package
%%   \usepackage{import}
%% and then include the figures with
%%   \import{<path to file>}{<filename>.pgf}
%%
%% Matplotlib used the following preamble
%%   \usepackage[utf8x]{inputenc}
%%   \usepackage[T1]{fontenc}
%%   \usepackage{amssymb}
%%
\begingroup%
\makeatletter%
\begin{pgfpicture}%
\pgfpathrectangle{\pgfpointorigin}{\pgfqpoint{4.000000in}{2.000000in}}%
\pgfusepath{use as bounding box, clip}%
\begin{pgfscope}%
\pgfsetbuttcap%
\pgfsetmiterjoin%
\definecolor{currentfill}{rgb}{1.000000,1.000000,1.000000}%
\pgfsetfillcolor{currentfill}%
\pgfsetlinewidth{0.000000pt}%
\definecolor{currentstroke}{rgb}{1.000000,1.000000,1.000000}%
\pgfsetstrokecolor{currentstroke}%
\pgfsetdash{}{0pt}%
\pgfpathmoveto{\pgfqpoint{0.000000in}{0.000000in}}%
\pgfpathlineto{\pgfqpoint{4.000000in}{0.000000in}}%
\pgfpathlineto{\pgfqpoint{4.000000in}{2.000000in}}%
\pgfpathlineto{\pgfqpoint{0.000000in}{2.000000in}}%
\pgfpathclose%
\pgfusepath{fill}%
\end{pgfscope}%
\begin{pgfscope}%
\pgfsetbuttcap%
\pgfsetmiterjoin%
\definecolor{currentfill}{rgb}{1.000000,1.000000,1.000000}%
\pgfsetfillcolor{currentfill}%
\pgfsetlinewidth{0.000000pt}%
\definecolor{currentstroke}{rgb}{0.000000,0.000000,0.000000}%
\pgfsetstrokecolor{currentstroke}%
\pgfsetstrokeopacity{0.000000}%
\pgfsetdash{}{0pt}%
\pgfpathmoveto{\pgfqpoint{0.198611in}{0.198611in}}%
\pgfpathlineto{\pgfqpoint{3.801389in}{0.198611in}}%
\pgfpathlineto{\pgfqpoint{3.801389in}{1.801389in}}%
\pgfpathlineto{\pgfqpoint{0.198611in}{1.801389in}}%
\pgfpathclose%
\pgfusepath{fill}%
\end{pgfscope}%
\begin{pgfscope}%
\pgfpathrectangle{\pgfqpoint{0.198611in}{0.198611in}}{\pgfqpoint{3.602778in}{1.602778in}}%
\pgfusepath{clip}%
\pgfsetbuttcap%
\pgfsetmiterjoin%
\definecolor{currentfill}{rgb}{0.500000,0.500000,0.500000}%
\pgfsetfillcolor{currentfill}%
\pgfsetfillopacity{0.500000}%
\pgfsetlinewidth{0.501875pt}%
\definecolor{currentstroke}{rgb}{0.000000,0.000000,0.000000}%
\pgfsetstrokecolor{currentstroke}%
\pgfsetdash{}{0pt}%
\pgfpathmoveto{\pgfqpoint{1.954965in}{0.398958in}}%
\pgfpathlineto{\pgfqpoint{2.045035in}{0.398958in}}%
\pgfpathlineto{\pgfqpoint{2.180139in}{0.519167in}}%
\pgfpathlineto{\pgfqpoint{1.819861in}{0.519167in}}%
\pgfpathclose%
\pgfusepath{stroke,fill}%
\end{pgfscope}%
\begin{pgfscope}%
\pgfpathrectangle{\pgfqpoint{0.198611in}{0.198611in}}{\pgfqpoint{3.602778in}{1.602778in}}%
\pgfusepath{clip}%
\pgfsetbuttcap%
\pgfsetmiterjoin%
\definecolor{currentfill}{rgb}{0.500000,0.500000,0.500000}%
\pgfsetfillcolor{currentfill}%
\pgfsetfillopacity{0.500000}%
\pgfsetlinewidth{0.501875pt}%
\definecolor{currentstroke}{rgb}{0.000000,0.000000,0.000000}%
\pgfsetstrokecolor{currentstroke}%
\pgfsetdash{}{0pt}%
\pgfpathmoveto{\pgfqpoint{1.883721in}{0.626019in}}%
\pgfpathlineto{\pgfqpoint{2.116279in}{0.626019in}}%
\pgfpathlineto{\pgfqpoint{2.465117in}{1.427407in}}%
\pgfpathlineto{\pgfqpoint{1.534883in}{1.427407in}}%
\pgfpathclose%
\pgfusepath{stroke,fill}%
\end{pgfscope}%
\begin{pgfscope}%
\pgfpathrectangle{\pgfqpoint{0.198611in}{0.198611in}}{\pgfqpoint{3.602778in}{1.602778in}}%
\pgfusepath{clip}%
\pgfsetbuttcap%
\pgfsetmiterjoin%
\definecolor{currentfill}{rgb}{0.500000,0.500000,0.500000}%
\pgfsetfillcolor{currentfill}%
\pgfsetfillopacity{0.500000}%
\pgfsetlinewidth{0.501875pt}%
\definecolor{currentstroke}{rgb}{0.000000,0.000000,0.000000}%
\pgfsetstrokecolor{currentstroke}%
\pgfsetdash{}{0pt}%
\pgfpathmoveto{\pgfqpoint{2.183952in}{0.626019in}}%
\pgfpathlineto{\pgfqpoint{2.416511in}{0.626019in}}%
\pgfpathlineto{\pgfqpoint{3.666043in}{1.427407in}}%
\pgfpathlineto{\pgfqpoint{2.735809in}{1.427407in}}%
\pgfpathclose%
\pgfusepath{stroke,fill}%
\end{pgfscope}%
\begin{pgfscope}%
\pgfpathrectangle{\pgfqpoint{0.198611in}{0.198611in}}{\pgfqpoint{3.602778in}{1.602778in}}%
\pgfusepath{clip}%
\pgfsetbuttcap%
\pgfsetroundjoin%
\pgfsetlinewidth{0.501875pt}%
\definecolor{currentstroke}{rgb}{0.501961,0.501961,0.501961}%
\pgfsetstrokecolor{currentstroke}%
\pgfsetdash{{1.850000pt}{0.800000pt}}{0.000000pt}%
\pgfpathmoveto{\pgfqpoint{1.804251in}{0.184722in}}%
\pgfpathlineto{\pgfqpoint{3.636860in}{1.815278in}}%
\pgfpathlineto{\pgfqpoint{3.636860in}{1.815278in}}%
\pgfusepath{stroke}%
\end{pgfscope}%
\begin{pgfscope}%
\pgfpathrectangle{\pgfqpoint{0.198611in}{0.198611in}}{\pgfqpoint{3.602778in}{1.602778in}}%
\pgfusepath{clip}%
\pgfsetbuttcap%
\pgfsetroundjoin%
\pgfsetlinewidth{0.501875pt}%
\definecolor{currentstroke}{rgb}{0.501961,0.501961,0.501961}%
\pgfsetstrokecolor{currentstroke}%
\pgfsetdash{{1.850000pt}{0.800000pt}}{0.000000pt}%
\pgfpathmoveto{\pgfqpoint{0.363140in}{1.815278in}}%
\pgfpathlineto{\pgfqpoint{2.195749in}{0.184722in}}%
\pgfpathlineto{\pgfqpoint{2.195749in}{0.184722in}}%
\pgfusepath{stroke}%
\end{pgfscope}%
\begin{pgfscope}%
\pgfsetrectcap%
\pgfsetmiterjoin%
\pgfsetlinewidth{0.501875pt}%
\definecolor{currentstroke}{rgb}{0.000000,0.000000,0.000000}%
\pgfsetstrokecolor{currentstroke}%
\pgfsetdash{}{0pt}%
\pgfpathmoveto{\pgfqpoint{2.000000in}{0.198611in}}%
\pgfpathlineto{\pgfqpoint{2.000000in}{1.801389in}}%
\pgfusepath{stroke}%
\end{pgfscope}%
\begin{pgfscope}%
\pgfsetrectcap%
\pgfsetmiterjoin%
\pgfsetlinewidth{0.501875pt}%
\definecolor{currentstroke}{rgb}{0.000000,0.000000,0.000000}%
\pgfsetstrokecolor{currentstroke}%
\pgfsetdash{}{0pt}%
\pgfpathmoveto{\pgfqpoint{0.198611in}{0.358889in}}%
\pgfpathlineto{\pgfqpoint{3.801389in}{0.358889in}}%
\pgfusepath{stroke}%
\end{pgfscope}%
\begin{pgfscope}%
\pgfsetroundcap%
\pgfsetroundjoin%
\pgfsetlinewidth{0.501875pt}%
\definecolor{currentstroke}{rgb}{0.000000,0.000000,0.000000}%
\pgfsetstrokecolor{currentstroke}%
\pgfsetdash{}{0pt}%
\pgfpathmoveto{\pgfqpoint{1.413964in}{0.497069in}}%
\pgfpathquadraticcurveto{\pgfqpoint{1.625531in}{0.488637in}}{\pgfqpoint{1.829340in}{0.480514in}}%
\pgfusepath{stroke}%
\end{pgfscope}%
\begin{pgfscope}%
\pgfsetroundcap%
\pgfsetroundjoin%
\pgfsetlinewidth{0.501875pt}%
\definecolor{currentstroke}{rgb}{0.000000,0.000000,0.000000}%
\pgfsetstrokecolor{currentstroke}%
\pgfsetdash{}{0pt}%
\pgfpathmoveto{\pgfqpoint{1.774935in}{0.510482in}}%
\pgfpathlineto{\pgfqpoint{1.829340in}{0.480514in}}%
\pgfpathlineto{\pgfqpoint{1.772722in}{0.454971in}}%
\pgfusepath{stroke}%
\end{pgfscope}%
\begin{pgfscope}%
\pgftext[x=0.648958in,y=0.479097in,left,base]{\rmfamily\fontsize{10.000000}{12.000000}\selectfont \(\displaystyle a = 1, s = 0\)}%
\end{pgfscope}%
\begin{pgfscope}%
\pgfsetroundcap%
\pgfsetroundjoin%
\pgfsetlinewidth{0.501875pt}%
\definecolor{currentstroke}{rgb}{0.000000,0.000000,0.000000}%
\pgfsetstrokecolor{currentstroke}%
\pgfsetdash{}{0pt}%
\pgfpathmoveto{\pgfqpoint{1.141127in}{1.096480in}}%
\pgfpathquadraticcurveto{\pgfqpoint{1.354038in}{1.088809in}}{\pgfqpoint{1.559190in}{1.081418in}}%
\pgfusepath{stroke}%
\end{pgfscope}%
\begin{pgfscope}%
\pgfsetroundcap%
\pgfsetroundjoin%
\pgfsetlinewidth{0.501875pt}%
\definecolor{currentstroke}{rgb}{0.000000,0.000000,0.000000}%
\pgfsetstrokecolor{currentstroke}%
\pgfsetdash{}{0pt}%
\pgfpathmoveto{\pgfqpoint{1.504671in}{1.111178in}}%
\pgfpathlineto{\pgfqpoint{1.559190in}{1.081418in}}%
\pgfpathlineto{\pgfqpoint{1.502670in}{1.055658in}}%
\pgfusepath{stroke}%
\end{pgfscope}%
\begin{pgfscope}%
\pgftext[x=0.198611in,y=1.080139in,left,base]{\rmfamily\fontsize{10.000000}{12.000000}\selectfont \(\displaystyle a = 0.15, s = 0\)}%
\end{pgfscope}%
\begin{pgfscope}%
\pgfsetroundcap%
\pgfsetroundjoin%
\pgfsetlinewidth{0.501875pt}%
\definecolor{currentstroke}{rgb}{0.000000,0.000000,0.000000}%
\pgfsetstrokecolor{currentstroke}%
\pgfsetdash{}{0pt}%
\pgfpathmoveto{\pgfqpoint{3.348730in}{0.656645in}}%
\pgfpathquadraticcurveto{\pgfqpoint{3.136663in}{0.781233in}}{\pgfqpoint{2.931289in}{0.901887in}}%
\pgfusepath{stroke}%
\end{pgfscope}%
\begin{pgfscope}%
\pgfsetroundcap%
\pgfsetroundjoin%
\pgfsetlinewidth{0.501875pt}%
\definecolor{currentstroke}{rgb}{0.000000,0.000000,0.000000}%
\pgfsetstrokecolor{currentstroke}%
\pgfsetdash{}{0pt}%
\pgfpathmoveto{\pgfqpoint{2.965119in}{0.849795in}}%
\pgfpathlineto{\pgfqpoint{2.931289in}{0.901887in}}%
\pgfpathlineto{\pgfqpoint{2.993261in}{0.897696in}}%
\pgfusepath{stroke}%
\end{pgfscope}%
\begin{pgfscope}%
\pgftext[x=3.080833in,y=0.519167in,left,base]{\rmfamily\fontsize{10.000000}{12.000000}\selectfont \(\displaystyle a = 0.15, s = 1\)}%
\end{pgfscope}%
\begin{pgfscope}%
\pgfsetroundcap%
\pgfsetroundjoin%
\pgfsetlinewidth{0.501875pt}%
\definecolor{currentstroke}{rgb}{0.000000,0.000000,0.000000}%
\pgfsetstrokecolor{currentstroke}%
\pgfsetdash{}{0pt}%
\pgfpathmoveto{\pgfqpoint{2.000000in}{1.807510in}}%
\pgfpathquadraticcurveto{\pgfqpoint{2.000000in}{1.808331in}}{\pgfqpoint{2.000000in}{1.801389in}}%
\pgfusepath{stroke}%
\end{pgfscope}%
\begin{pgfscope}%
\pgfsetroundcap%
\pgfsetroundjoin%
\pgfsetlinewidth{0.501875pt}%
\definecolor{currentstroke}{rgb}{0.000000,0.000000,0.000000}%
\pgfsetstrokecolor{currentstroke}%
\pgfsetdash{}{0pt}%
\pgfpathmoveto{\pgfqpoint{1.972222in}{1.751954in}}%
\pgfpathlineto{\pgfqpoint{2.000000in}{1.807510in}}%
\pgfpathlineto{\pgfqpoint{2.027778in}{1.751954in}}%
\pgfusepath{stroke}%
\end{pgfscope}%
\begin{pgfscope}%
\pgftext[x=2.000000in,y=1.870833in,,bottom]{\rmfamily\fontsize{10.000000}{12.000000}\selectfont \(\displaystyle \omega\)}%
\end{pgfscope}%
\begin{pgfscope}%
\pgfsetroundcap%
\pgfsetroundjoin%
\pgfsetlinewidth{0.501875pt}%
\definecolor{currentstroke}{rgb}{0.000000,0.000000,0.000000}%
\pgfsetstrokecolor{currentstroke}%
\pgfsetdash{}{0pt}%
\pgfpathmoveto{\pgfqpoint{3.807488in}{0.358889in}}%
\pgfpathquadraticcurveto{\pgfqpoint{3.808320in}{0.358889in}}{\pgfqpoint{3.801389in}{0.358889in}}%
\pgfusepath{stroke}%
\end{pgfscope}%
\begin{pgfscope}%
\pgfsetroundcap%
\pgfsetroundjoin%
\pgfsetlinewidth{0.501875pt}%
\definecolor{currentstroke}{rgb}{0.000000,0.000000,0.000000}%
\pgfsetstrokecolor{currentstroke}%
\pgfsetdash{}{0pt}%
\pgfpathmoveto{\pgfqpoint{3.751932in}{0.386667in}}%
\pgfpathlineto{\pgfqpoint{3.807488in}{0.358889in}}%
\pgfpathlineto{\pgfqpoint{3.751932in}{0.331111in}}%
\pgfusepath{stroke}%
\end{pgfscope}%
\begin{pgfscope}%
\pgftext[x=3.870833in,y=0.358889in,left,]{\rmfamily\fontsize{10.000000}{12.000000}\selectfont \(\displaystyle k\)}%
\end{pgfscope}%
\end{pgfpicture}%
\makeatother%
\endgroup%

\label{fig:supp_psi_hat}
\caption{Der Träger von $\hat \psi_{ast}$ für verschiedene $a, s$. Man sieht gut,
wie $supp (\hat \psi_{ast})$ für kleinere $a$ in immer spitzeren Kegeln liegt.}
\end{figure}

\label{cor:psi_hat}
Im Fourierraum ist $\hat{\psi}_{ast}$ gegeben durch

\begin{equation}
    \hat \psi_{ast}{(\xi_1, \xi_2)} = a^{\frac{3}{4}}e^{-i\xi \cdot t}\hat\psi_1(a \xi_1) \hat\psi_{2}\left(a^{-\frac{1}{2}}\left(\frac{\xi_2}{\xi_1}-s\right)\right)
\label{eq:hat_psi_ast}
\end{equation}

und es gilt

\begin{equation}
\label{eq:supp_psi}
    supp(\hat \psi) \subset \left\{\xi \in  \hat{\mathbb{R}}^2 ~\Big| ~|\xi_1| \in \left[\frac{1}{2 a} , \frac{2}{a}\right], \left|\frac{\xi_2}{\xi_1} - s\right| \leq \sqrt{a} \right\}
\end{equation}

\end{remark}



% section allgemeines_gelaber_über_shearlets (end)


%%%%%%%%%%%%%%%%%%%%%%%%%%%%%%%%%%%%%%%%%%%%%%%%%%%%%%%%%%%%%%%%%%%%%%%%%%%%%%%%
% % Section 2
%%%%%%%%%%%%%%%%%%%%%%%%%%%%%%%%%%%%%%%%%%%%%%%%%%%%%%%%%%%%%%%%%%%%%%%%%%%%%%%%
\section{\texorpdfstring{Zwei nützliche Substitionen für  $\left<\psi_{ast}, f\right>$}{zwei nützliche Substitutionen}}
\label{sec:substitutionen}

\todo[color=green]{mit $(\omega, k)$ als Variablennamen arbeiten, um zum Rest des Textes zu passen, oder mit $(\xi_1, \xi_2)$ um zu \textcite{Kutyniok2008} zu passen?}

Zunächst werden wir zwei verschiedene Ausdrücke für $\left<\psi_{ast}, f\right>$
im Fourierraum herleiten, welche fast immer Ausgangspunkt für unsere Abschätzungen sein werden.

Sei also $\psi$ ein Shearlet wie in \cref{cor:psi_hat}. Sei $f$ die zu
analysierende fouriertransformierbare Funktion (oder Distribution) in
$\mathcal{D}' (\mathbb{R}^2)$. Dann ist $\mathcal{S}_f (ast)$ gegeben durch

\begin{align*}
\left< \psi_{ast}, f \right> &= \left<\hat\psi_{ast}, \hat f\right> \\
 &= \int a^{\frac{3}{4}} e^{-i \xi \cdot t} \hat \psi_1(a \xi_1)
    \hat \psi_2 \left(a^{-\frac{1}{2}} \left(\frac{\xi_2}{\xi_1} - s\right)\right)
    \hat f (\xi) \d \xi
\end{align*}

\todo{entscheiden, was mit dem fehlenden Faktor $\frac{1}{(2 \pi)^n}$ geschieht}
und nach "`entscheren"' und "`deskalieren"', also der Substitution

\begin{equation}
\begin{aligned}[c]
a \xi_1 &= k_1\\
a^{-\frac{1}{2}} \left(\frac{\xi_2}{\xi_1} - s\right) &=\frac{k_2}{k_1}\\
\end{aligned}
\qquad\Longleftrightarrow\qquad
\begin{aligned}[c]
\xi_1 &= \frac{k_1}{a}\\
\xi_2 &= \frac{k_1 s}{a} + a^{-\frac{1}{2}} k_2\\
\end{aligned}
\label{eq:substitution1_coords}
\end{equation}

\begin{equation*}
\Rightarrow
\d \xi_1 \d \xi_2 = a^{-\frac{3}{2}} \d k_1 \d k_2
\end{equation*}

ergibt sich folgendes für $\left<\psi_{ast}, f\right>$:

\todo{\texttt{owntag} fixen}

\begin{align}
    \left\langle\psi_{ast},f\right\rangle
    &=  \left\langle\hat\psi_{ast},\hat f\right\rangle \nonumber \\
    &=  \iint a^{-\frac{3}{4}}~\hat \psi_1(k_1) ~\hat \psi_2 \left(\tfrac{k_2}{k_1}\right)
    ~\hat f \left(\tfrac{k_1}{a}, \tfrac{k_1 s}{a} + \tfrac{k_2}{\sqrt{a}}\right)
    ~e^{-i\frac{k_1}{a}(t_1+t_2 s) - i \frac{k_2 t_2}{\sqrt a}}
    \d k_1 \d k_2
\owntag[substitution1]{Substitution 1}
\end{align}

Wie man sieht, tauchen in den Argumente von $\hat\psi_1$ und $\hat\psi_2$ nun die Parameter $a,s,t$ gar nicht mehr auf, und wir können nun verwenden, was wir aus \ref{sec:shearlets} über deren Träger wissen.
Alternativ und mit ähnlichem Ergebniss kann auch folgende Substitution

\begin{equation}
\begin{aligned}[c]
a \xi_1 &= k_1\\
a^{-\frac{1}{2}} \left(\frac{\xi_2}{\xi_1} - s\right) &= k_2\\
\end{aligned}
\qquad\Longleftrightarrow\qquad
\begin{aligned}[c]
\xi_1 &= \frac{k_1}{a}\\
\xi_2 &= \left( a^{\frac{1}{2}} k_2 +s \right) \frac{k_1}{a}\\
\end{aligned}
\label{eq:substitution2_coords}
\end{equation}

\begin{equation*}
\Rightarrow
\d \xi_1 \d \xi_2 = a^{-\frac{3}{2}} k_1 \d k_1 \d k_2
\end{equation*}

gewählt werden, wodurch wieder alle Parameter $(a,s,t)$ aus den Argumenten von $\hat\psi_1, \hat\psi_2$
verschwinden und sich

\begin{align}
    \left<\psi_{ast},f\right>
    =  \iint a^{-\frac{3}{4}}~ k_1~ \hat \psi_1(k_1)~ \hat \psi_2 (k_2)~
    \hat f \left(\tfrac{k_1}{a}, k_1 \left(a^{-\frac{1}{2}}k_2 + s a^{-1}\right)\right)
    ~e^{-i k_1 \left(\frac{t_1+s t_2}{a} + \frac{k_2 t_2}{\sqrt{a}}\right)}
    \d k_1 \d k_2
\owntag[substitution2]{Substitution 2}
\end{align}

ergibt. Dabei ist zu beachten, dass diese Substitution zulässig ist, obwohl sie
die Orientierung \emph{nicht} erhält und \emph{keine} Bijektion ist. Aber
der kritische Bereich, nämlich $\xi_1 = 0$, liegt nicht im Träger von $\rwhat{\psi}$.

Beiden Substitution gemein ist aber, dass danach
$0=\omega \notin supp (\hat\psi)$ und dass $supp (\psi)$ sowohl in $k$ als auch in $\omega$ beschränkt ist. $\omega$ kann also sowohl nach unten als auch nach oben durch eine Konstante abgeschätzt werden, wannimmer dies der Sache dienlich ist. Auch $k$ kann zumindest nach oben immer durche eine Konstante abgeschätzt werden.

\begin{figure}[h]
    \centering
    \begin{minipage}{0.5\textwidth}
        \centering
        \resizebox{\textwidth}{!}{%% Creator: Matplotlib, PGF backend
%%
%% To include the figure in your LaTeX document, write
%%   \input{<filename>.pgf}
%%
%% Make sure the required packages are loaded in your preamble
%%   \usepackage{pgf}
%%
%% Figures using additional raster images can only be included by \input if
%% they are in the same directory as the main LaTeX file. For loading figures
%% from other directories you can use the `import` package
%%   \usepackage{import}
%% and then include the figures with
%%   \import{<path to file>}{<filename>.pgf}
%%
%% Matplotlib used the following preamble
%%   \usepackage[utf8x]{inputenc}
%%   \usepackage[T1]{fontenc}
%%   \usepackage{amssymb}
%%
\begingroup%
\makeatletter%
\begin{pgfpicture}%
\pgfpathrectangle{\pgfpointorigin}{\pgfqpoint{4.000000in}{2.800000in}}%
\pgfusepath{use as bounding box, clip}%
\begin{pgfscope}%
\pgfsetbuttcap%
\pgfsetmiterjoin%
\definecolor{currentfill}{rgb}{1.000000,1.000000,1.000000}%
\pgfsetfillcolor{currentfill}%
\pgfsetlinewidth{0.000000pt}%
\definecolor{currentstroke}{rgb}{1.000000,1.000000,1.000000}%
\pgfsetstrokecolor{currentstroke}%
\pgfsetdash{}{0pt}%
\pgfpathmoveto{\pgfqpoint{0.000000in}{0.000000in}}%
\pgfpathlineto{\pgfqpoint{4.000000in}{0.000000in}}%
\pgfpathlineto{\pgfqpoint{4.000000in}{2.800000in}}%
\pgfpathlineto{\pgfqpoint{0.000000in}{2.800000in}}%
\pgfpathclose%
\pgfusepath{fill}%
\end{pgfscope}%
\begin{pgfscope}%
\pgfsetbuttcap%
\pgfsetmiterjoin%
\definecolor{currentfill}{rgb}{1.000000,1.000000,1.000000}%
\pgfsetfillcolor{currentfill}%
\pgfsetlinewidth{0.000000pt}%
\definecolor{currentstroke}{rgb}{0.000000,0.000000,0.000000}%
\pgfsetstrokecolor{currentstroke}%
\pgfsetstrokeopacity{0.000000}%
\pgfsetdash{}{0pt}%
\pgfpathmoveto{\pgfqpoint{0.198611in}{0.198611in}}%
\pgfpathlineto{\pgfqpoint{3.801389in}{0.198611in}}%
\pgfpathlineto{\pgfqpoint{3.801389in}{2.601389in}}%
\pgfpathlineto{\pgfqpoint{0.198611in}{2.601389in}}%
\pgfpathclose%
\pgfusepath{fill}%
\end{pgfscope}%
\begin{pgfscope}%
\pgfpathrectangle{\pgfqpoint{0.198611in}{0.198611in}}{\pgfqpoint{3.602778in}{2.402778in}}%
\pgfusepath{clip}%
\pgfsetbuttcap%
\pgfsetmiterjoin%
\definecolor{currentfill}{rgb}{0.500000,0.500000,0.500000}%
\pgfsetfillcolor{currentfill}%
\pgfsetfillopacity{0.500000}%
\pgfsetlinewidth{0.501875pt}%
\definecolor{currentstroke}{rgb}{0.000000,0.000000,0.000000}%
\pgfsetstrokecolor{currentstroke}%
\pgfsetdash{}{0pt}%
\pgfpathmoveto{\pgfqpoint{1.963972in}{1.433372in}}%
\pgfpathlineto{\pgfqpoint{2.036028in}{1.433372in}}%
\pgfpathlineto{\pgfqpoint{2.144111in}{1.533488in}}%
\pgfpathlineto{\pgfqpoint{1.855889in}{1.533488in}}%
\pgfpathclose%
\pgfusepath{stroke,fill}%
\end{pgfscope}%
\begin{pgfscope}%
\pgfpathrectangle{\pgfqpoint{0.198611in}{0.198611in}}{\pgfqpoint{3.602778in}{2.402778in}}%
\pgfusepath{clip}%
\pgfsetbuttcap%
\pgfsetmiterjoin%
\definecolor{currentfill}{rgb}{0.500000,0.500000,0.500000}%
\pgfsetfillcolor{currentfill}%
\pgfsetfillopacity{0.500000}%
\pgfsetlinewidth{0.501875pt}%
\definecolor{currentstroke}{rgb}{0.000000,0.000000,0.000000}%
\pgfsetstrokecolor{currentstroke}%
\pgfsetdash{}{0pt}%
\pgfpathmoveto{\pgfqpoint{2.036028in}{1.366628in}}%
\pgfpathlineto{\pgfqpoint{1.963972in}{1.366628in}}%
\pgfpathlineto{\pgfqpoint{1.855889in}{1.266512in}}%
\pgfpathlineto{\pgfqpoint{2.144111in}{1.266512in}}%
\pgfpathclose%
\pgfusepath{stroke,fill}%
\end{pgfscope}%
\begin{pgfscope}%
\pgfpathrectangle{\pgfqpoint{0.198611in}{0.198611in}}{\pgfqpoint{3.602778in}{2.402778in}}%
\pgfusepath{clip}%
\pgfsetbuttcap%
\pgfsetmiterjoin%
\definecolor{currentfill}{rgb}{0.500000,0.500000,0.500000}%
\pgfsetfillcolor{currentfill}%
\pgfsetfillopacity{0.500000}%
\pgfsetlinewidth{0.501875pt}%
\definecolor{currentstroke}{rgb}{0.000000,0.000000,0.000000}%
\pgfsetstrokecolor{currentstroke}%
\pgfsetdash{}{0pt}%
\pgfpathmoveto{\pgfqpoint{2.246348in}{1.733719in}}%
\pgfpathlineto{\pgfqpoint{2.474208in}{1.733719in}}%
\pgfpathlineto{\pgfqpoint{3.896830in}{2.734877in}}%
\pgfpathlineto{\pgfqpoint{2.985392in}{2.734877in}}%
\pgfpathclose%
\pgfusepath{stroke,fill}%
\end{pgfscope}%
\begin{pgfscope}%
\pgfpathrectangle{\pgfqpoint{0.198611in}{0.198611in}}{\pgfqpoint{3.602778in}{2.402778in}}%
\pgfusepath{clip}%
\pgfsetbuttcap%
\pgfsetmiterjoin%
\definecolor{currentfill}{rgb}{0.500000,0.500000,0.500000}%
\pgfsetfillcolor{currentfill}%
\pgfsetfillopacity{0.500000}%
\pgfsetlinewidth{0.501875pt}%
\definecolor{currentstroke}{rgb}{0.000000,0.000000,0.000000}%
\pgfsetstrokecolor{currentstroke}%
\pgfsetdash{}{0pt}%
\pgfpathmoveto{\pgfqpoint{1.753652in}{1.066281in}}%
\pgfpathlineto{\pgfqpoint{1.525792in}{1.066281in}}%
\pgfpathlineto{\pgfqpoint{0.103170in}{0.065123in}}%
\pgfpathlineto{\pgfqpoint{1.014608in}{0.065123in}}%
\pgfpathclose%
\pgfusepath{stroke,fill}%
\end{pgfscope}%
\begin{pgfscope}%
\pgfpathrectangle{\pgfqpoint{0.198611in}{0.198611in}}{\pgfqpoint{3.602778in}{2.402778in}}%
\pgfusepath{clip}%
\pgfsetbuttcap%
\pgfsetroundjoin%
\pgfsetlinewidth{0.501875pt}%
\definecolor{currentstroke}{rgb}{0.501961,0.501961,0.501961}%
\pgfsetstrokecolor{currentstroke}%
\pgfsetdash{{1.850000pt}{0.800000pt}}{0.000000pt}%
\pgfpathmoveto{\pgfqpoint{0.688006in}{0.184722in}}%
\pgfpathlineto{\pgfqpoint{3.311994in}{2.615278in}}%
\pgfpathlineto{\pgfqpoint{3.311994in}{2.615278in}}%
\pgfusepath{stroke}%
\end{pgfscope}%
\begin{pgfscope}%
\pgfpathrectangle{\pgfqpoint{0.198611in}{0.198611in}}{\pgfqpoint{3.602778in}{2.402778in}}%
\pgfusepath{clip}%
\pgfsetbuttcap%
\pgfsetroundjoin%
\pgfsetlinewidth{0.501875pt}%
\definecolor{currentstroke}{rgb}{0.501961,0.501961,0.501961}%
\pgfsetstrokecolor{currentstroke}%
\pgfsetdash{{1.850000pt}{0.800000pt}}{0.000000pt}%
\pgfpathmoveto{\pgfqpoint{0.688006in}{2.615278in}}%
\pgfpathlineto{\pgfqpoint{3.311994in}{0.184722in}}%
\pgfpathlineto{\pgfqpoint{3.311994in}{0.184722in}}%
\pgfusepath{stroke}%
\end{pgfscope}%
\begin{pgfscope}%
\pgfsetrectcap%
\pgfsetmiterjoin%
\pgfsetlinewidth{0.501875pt}%
\definecolor{currentstroke}{rgb}{0.000000,0.000000,0.000000}%
\pgfsetstrokecolor{currentstroke}%
\pgfsetdash{}{0pt}%
\pgfpathmoveto{\pgfqpoint{2.000000in}{0.198611in}}%
\pgfpathlineto{\pgfqpoint{2.000000in}{2.601389in}}%
\pgfusepath{stroke}%
\end{pgfscope}%
\begin{pgfscope}%
\pgfsetrectcap%
\pgfsetmiterjoin%
\pgfsetlinewidth{0.501875pt}%
\definecolor{currentstroke}{rgb}{0.000000,0.000000,0.000000}%
\pgfsetstrokecolor{currentstroke}%
\pgfsetdash{}{0pt}%
\pgfpathmoveto{\pgfqpoint{0.198611in}{1.400000in}}%
\pgfpathlineto{\pgfqpoint{3.801389in}{1.400000in}}%
\pgfusepath{stroke}%
\end{pgfscope}%
\begin{pgfscope}%
\pgfsetroundcap%
\pgfsetroundjoin%
\pgfsetlinewidth{0.501875pt}%
\definecolor{currentstroke}{rgb}{0.000000,0.000000,0.000000}%
\pgfsetstrokecolor{currentstroke}%
\pgfsetdash{}{0pt}%
\pgfpathmoveto{\pgfqpoint{3.032230in}{2.375700in}}%
\pgfpathquadraticcurveto{\pgfqpoint{2.526526in}{1.930390in}}{\pgfqpoint{2.026649in}{1.490210in}}%
\pgfusepath{stroke}%
\end{pgfscope}%
\begin{pgfscope}%
\pgfsetroundcap%
\pgfsetroundjoin%
\pgfsetlinewidth{0.501875pt}%
\definecolor{currentstroke}{rgb}{0.000000,0.000000,0.000000}%
\pgfsetstrokecolor{currentstroke}%
\pgfsetdash{}{0pt}%
\pgfpathmoveto{\pgfqpoint{2.086701in}{1.506078in}}%
\pgfpathlineto{\pgfqpoint{2.026649in}{1.490210in}}%
\pgfpathlineto{\pgfqpoint{2.049986in}{1.547773in}}%
\pgfusepath{stroke}%
\end{pgfscope}%
\begin{pgfscope}%
\pgftext[x=3.080833in,y=2.401157in,left,base]{\rmfamily\fontsize{10.000000}{12.000000}\selectfont \(\displaystyle {\cdot}\)}%
\end{pgfscope}%
\begin{pgfscope}%
\pgftext[x=2.288222in,y=1.600231in,left,base]{\rmfamily\fontsize{10.000000}{12.000000}\selectfont Substitution 1}%
\end{pgfscope}%
\begin{pgfscope}%
\pgfsetroundcap%
\pgfsetroundjoin%
\pgfsetlinewidth{0.501875pt}%
\definecolor{currentstroke}{rgb}{0.000000,0.000000,0.000000}%
\pgfsetstrokecolor{currentstroke}%
\pgfsetdash{}{0pt}%
\pgfpathmoveto{\pgfqpoint{2.000000in}{2.607510in}}%
\pgfpathquadraticcurveto{\pgfqpoint{2.000000in}{2.608331in}}{\pgfqpoint{2.000000in}{2.601389in}}%
\pgfusepath{stroke}%
\end{pgfscope}%
\begin{pgfscope}%
\pgfsetroundcap%
\pgfsetroundjoin%
\pgfsetlinewidth{0.501875pt}%
\definecolor{currentstroke}{rgb}{0.000000,0.000000,0.000000}%
\pgfsetstrokecolor{currentstroke}%
\pgfsetdash{}{0pt}%
\pgfpathmoveto{\pgfqpoint{1.972222in}{2.551954in}}%
\pgfpathlineto{\pgfqpoint{2.000000in}{2.607510in}}%
\pgfpathlineto{\pgfqpoint{2.027778in}{2.551954in}}%
\pgfusepath{stroke}%
\end{pgfscope}%
\begin{pgfscope}%
\pgftext[x=2.000000in,y=2.670833in,,bottom]{\rmfamily\fontsize{10.000000}{12.000000}\selectfont \(\displaystyle \omega\)}%
\end{pgfscope}%
\begin{pgfscope}%
\pgfsetroundcap%
\pgfsetroundjoin%
\pgfsetlinewidth{0.501875pt}%
\definecolor{currentstroke}{rgb}{0.000000,0.000000,0.000000}%
\pgfsetstrokecolor{currentstroke}%
\pgfsetdash{}{0pt}%
\pgfpathmoveto{\pgfqpoint{3.807488in}{1.400000in}}%
\pgfpathquadraticcurveto{\pgfqpoint{3.808320in}{1.400000in}}{\pgfqpoint{3.801389in}{1.400000in}}%
\pgfusepath{stroke}%
\end{pgfscope}%
\begin{pgfscope}%
\pgfsetroundcap%
\pgfsetroundjoin%
\pgfsetlinewidth{0.501875pt}%
\definecolor{currentstroke}{rgb}{0.000000,0.000000,0.000000}%
\pgfsetstrokecolor{currentstroke}%
\pgfsetdash{}{0pt}%
\pgfpathmoveto{\pgfqpoint{3.751932in}{1.427778in}}%
\pgfpathlineto{\pgfqpoint{3.807488in}{1.400000in}}%
\pgfpathlineto{\pgfqpoint{3.751932in}{1.372222in}}%
\pgfusepath{stroke}%
\end{pgfscope}%
\begin{pgfscope}%
\pgftext[x=3.870833in,y=1.400000in,left,]{\rmfamily\fontsize{10.000000}{12.000000}\selectfont \(\displaystyle k\)}%
\end{pgfscope}%
\end{pgfpicture}%
\makeatother%
\endgroup%
} %
        \caption{Der Träger von $\hat\psi$ vor und nach der Substitution aus \cref{eq:substitution1_coords}}
        \label{fig:supp_psi_substitution1}
    \end{minipage}\hfill
    \begin{minipage}{0.5\textwidth}
        \centering
        \resizebox{\textwidth}{!}{%% Creator: Matplotlib, PGF backend
%%
%% To include the figure in your LaTeX document, write
%%   \input{<filename>.pgf}
%%
%% Make sure the required packages are loaded in your preamble
%%   \usepackage{pgf}
%%
%% Figures using additional raster images can only be included by \input if
%% they are in the same directory as the main LaTeX file. For loading figures
%% from other directories you can use the `import` package
%%   \usepackage{import}
%% and then include the figures with
%%   \import{<path to file>}{<filename>.pgf}
%%
%% Matplotlib used the following preamble
%%   \usepackage[utf8x]{inputenc}
%%   \usepackage[T1]{fontenc}
%%   \usepackage{amssymb}
%%
\begingroup%
\makeatletter%
\begin{pgfpicture}%
\pgfpathrectangle{\pgfpointorigin}{\pgfqpoint{4.000000in}{2.000000in}}%
\pgfusepath{use as bounding box, clip}%
\begin{pgfscope}%
\pgfsetbuttcap%
\pgfsetmiterjoin%
\definecolor{currentfill}{rgb}{1.000000,1.000000,1.000000}%
\pgfsetfillcolor{currentfill}%
\pgfsetlinewidth{0.000000pt}%
\definecolor{currentstroke}{rgb}{1.000000,1.000000,1.000000}%
\pgfsetstrokecolor{currentstroke}%
\pgfsetdash{}{0pt}%
\pgfpathmoveto{\pgfqpoint{0.000000in}{0.000000in}}%
\pgfpathlineto{\pgfqpoint{4.000000in}{0.000000in}}%
\pgfpathlineto{\pgfqpoint{4.000000in}{2.000000in}}%
\pgfpathlineto{\pgfqpoint{0.000000in}{2.000000in}}%
\pgfpathclose%
\pgfusepath{fill}%
\end{pgfscope}%
\begin{pgfscope}%
\pgfsetbuttcap%
\pgfsetmiterjoin%
\definecolor{currentfill}{rgb}{1.000000,1.000000,1.000000}%
\pgfsetfillcolor{currentfill}%
\pgfsetlinewidth{0.000000pt}%
\definecolor{currentstroke}{rgb}{0.000000,0.000000,0.000000}%
\pgfsetstrokecolor{currentstroke}%
\pgfsetstrokeopacity{0.000000}%
\pgfsetdash{}{0pt}%
\pgfpathmoveto{\pgfqpoint{0.198611in}{0.198611in}}%
\pgfpathlineto{\pgfqpoint{3.801389in}{0.198611in}}%
\pgfpathlineto{\pgfqpoint{3.801389in}{1.801389in}}%
\pgfpathlineto{\pgfqpoint{0.198611in}{1.801389in}}%
\pgfpathclose%
\pgfusepath{fill}%
\end{pgfscope}%
\begin{pgfscope}%
\pgfpathrectangle{\pgfqpoint{0.198611in}{0.198611in}}{\pgfqpoint{3.602778in}{1.602778in}} %
\pgfusepath{clip}%
\pgfsetbuttcap%
\pgfsetmiterjoin%
\definecolor{currentfill}{rgb}{0.500000,0.500000,0.500000}%
\pgfsetfillcolor{currentfill}%
\pgfsetfillopacity{0.500000}%
\pgfsetlinewidth{0.501875pt}%
\definecolor{currentstroke}{rgb}{0.000000,0.000000,0.000000}%
\pgfsetstrokecolor{currentstroke}%
\pgfsetdash{}{0pt}%
\pgfpathmoveto{\pgfqpoint{1.909931in}{0.398958in}}%
\pgfpathlineto{\pgfqpoint{1.909931in}{0.519167in}}%
\pgfpathlineto{\pgfqpoint{2.090069in}{0.519167in}}%
\pgfpathlineto{\pgfqpoint{2.090069in}{0.398958in}}%
\pgfpathclose%
\pgfusepath{stroke,fill}%
\end{pgfscope}%
\begin{pgfscope}%
\pgfpathrectangle{\pgfqpoint{0.198611in}{0.198611in}}{\pgfqpoint{3.602778in}{1.602778in}} %
\pgfusepath{clip}%
\pgfsetbuttcap%
\pgfsetmiterjoin%
\definecolor{currentfill}{rgb}{0.500000,0.500000,0.500000}%
\pgfsetfillcolor{currentfill}%
\pgfsetfillopacity{0.500000}%
\pgfsetlinewidth{0.501875pt}%
\definecolor{currentstroke}{rgb}{0.000000,0.000000,0.000000}%
\pgfsetstrokecolor{currentstroke}%
\pgfsetdash{}{0pt}%
\pgfpathmoveto{\pgfqpoint{2.307935in}{0.759583in}}%
\pgfpathlineto{\pgfqpoint{2.592760in}{0.759583in}}%
\pgfpathlineto{\pgfqpoint{4.371038in}{1.961667in}}%
\pgfpathlineto{\pgfqpoint{3.231740in}{1.961667in}}%
\pgfpathclose%
\pgfusepath{stroke,fill}%
\end{pgfscope}%
\begin{pgfscope}%
\pgfpathrectangle{\pgfqpoint{0.198611in}{0.198611in}}{\pgfqpoint{3.602778in}{1.602778in}} %
\pgfusepath{clip}%
\pgfsetbuttcap%
\pgfsetroundjoin%
\pgfsetlinewidth{0.501875pt}%
\definecolor{currentstroke}{rgb}{0.501961,0.501961,0.501961}%
\pgfsetstrokecolor{currentstroke}%
\pgfsetdash{{1.850000pt}{0.800000pt}}{0.000000pt}%
\pgfpathmoveto{\pgfqpoint{1.804251in}{0.184722in}}%
\pgfpathlineto{\pgfqpoint{3.636860in}{1.815278in}}%
\pgfpathlineto{\pgfqpoint{3.636860in}{1.815278in}}%
\pgfusepath{stroke}%
\end{pgfscope}%
\begin{pgfscope}%
\pgfpathrectangle{\pgfqpoint{0.198611in}{0.198611in}}{\pgfqpoint{3.602778in}{1.602778in}} %
\pgfusepath{clip}%
\pgfsetbuttcap%
\pgfsetroundjoin%
\pgfsetlinewidth{0.501875pt}%
\definecolor{currentstroke}{rgb}{0.501961,0.501961,0.501961}%
\pgfsetstrokecolor{currentstroke}%
\pgfsetdash{{1.850000pt}{0.800000pt}}{0.000000pt}%
\pgfpathmoveto{\pgfqpoint{0.363140in}{1.815278in}}%
\pgfpathlineto{\pgfqpoint{2.195749in}{0.184722in}}%
\pgfpathlineto{\pgfqpoint{2.195749in}{0.184722in}}%
\pgfusepath{stroke}%
\end{pgfscope}%
\begin{pgfscope}%
\pgfsetrectcap%
\pgfsetmiterjoin%
\pgfsetlinewidth{0.501875pt}%
\definecolor{currentstroke}{rgb}{0.000000,0.000000,0.000000}%
\pgfsetstrokecolor{currentstroke}%
\pgfsetdash{}{0pt}%
\pgfpathmoveto{\pgfqpoint{2.000000in}{0.198611in}}%
\pgfpathlineto{\pgfqpoint{2.000000in}{1.801389in}}%
\pgfusepath{stroke}%
\end{pgfscope}%
\begin{pgfscope}%
\pgfsetrectcap%
\pgfsetmiterjoin%
\pgfsetlinewidth{0.501875pt}%
\definecolor{currentstroke}{rgb}{0.000000,0.000000,0.000000}%
\pgfsetstrokecolor{currentstroke}%
\pgfsetdash{}{0pt}%
\pgfpathmoveto{\pgfqpoint{0.198611in}{0.358889in}}%
\pgfpathlineto{\pgfqpoint{3.801389in}{0.358889in}}%
\pgfusepath{stroke}%
\end{pgfscope}%
\begin{pgfscope}%
\pgfsetroundcap%
\pgfsetroundjoin%
\pgfsetlinewidth{0.501875pt}%
\definecolor{currentstroke}{rgb}{0.000000,0.000000,0.000000}%
\pgfsetstrokecolor{currentstroke}%
\pgfsetdash{}{0pt}%
\pgfpathmoveto{\pgfqpoint{3.302084in}{1.537713in}}%
\pgfpathquadraticcurveto{\pgfqpoint{2.661671in}{0.997340in}}{\pgfqpoint{2.027193in}{0.461973in}}%
\pgfusepath{stroke}%
\end{pgfscope}%
\begin{pgfscope}%
\pgfsetroundcap%
\pgfsetroundjoin%
\pgfsetlinewidth{0.501875pt}%
\definecolor{currentstroke}{rgb}{0.000000,0.000000,0.000000}%
\pgfsetstrokecolor{currentstroke}%
\pgfsetdash{}{0pt}%
\pgfpathmoveto{\pgfqpoint{2.087566in}{0.476570in}}%
\pgfpathlineto{\pgfqpoint{2.027193in}{0.461973in}}%
\pgfpathlineto{\pgfqpoint{2.051739in}{0.519030in}}%
\pgfusepath{stroke}%
\end{pgfscope}%
\begin{pgfscope}%
\pgftext[x=3.351042in,y=1.560972in,left,base]{\rmfamily\fontsize{10.000000}{12.000000}\selectfont \(\displaystyle {\cdot}\)}%
\end{pgfscope}%
\begin{pgfscope}%
\pgftext[x=2.360278in,y=0.599306in,left,base]{\rmfamily\fontsize{10.000000}{12.000000}\selectfont Substitution 1}%
\end{pgfscope}%
\begin{pgfscope}%
\pgfsetroundcap%
\pgfsetroundjoin%
\pgfsetlinewidth{0.501875pt}%
\definecolor{currentstroke}{rgb}{0.000000,0.000000,0.000000}%
\pgfsetstrokecolor{currentstroke}%
\pgfsetdash{}{0pt}%
\pgfpathmoveto{\pgfqpoint{2.000000in}{1.807510in}}%
\pgfpathquadraticcurveto{\pgfqpoint{2.000000in}{1.808331in}}{\pgfqpoint{2.000000in}{1.801389in}}%
\pgfusepath{stroke}%
\end{pgfscope}%
\begin{pgfscope}%
\pgfsetroundcap%
\pgfsetroundjoin%
\pgfsetlinewidth{0.501875pt}%
\definecolor{currentstroke}{rgb}{0.000000,0.000000,0.000000}%
\pgfsetstrokecolor{currentstroke}%
\pgfsetdash{}{0pt}%
\pgfpathmoveto{\pgfqpoint{1.972222in}{1.751954in}}%
\pgfpathlineto{\pgfqpoint{2.000000in}{1.807510in}}%
\pgfpathlineto{\pgfqpoint{2.027778in}{1.751954in}}%
\pgfusepath{stroke}%
\end{pgfscope}%
\begin{pgfscope}%
\pgftext[x=2.000000in,y=1.870833in,,bottom]{\rmfamily\fontsize{10.000000}{12.000000}\selectfont \(\displaystyle \omega\)}%
\end{pgfscope}%
\begin{pgfscope}%
\pgfsetroundcap%
\pgfsetroundjoin%
\pgfsetlinewidth{0.501875pt}%
\definecolor{currentstroke}{rgb}{0.000000,0.000000,0.000000}%
\pgfsetstrokecolor{currentstroke}%
\pgfsetdash{}{0pt}%
\pgfpathmoveto{\pgfqpoint{3.807488in}{0.358889in}}%
\pgfpathquadraticcurveto{\pgfqpoint{3.808320in}{0.358889in}}{\pgfqpoint{3.801389in}{0.358889in}}%
\pgfusepath{stroke}%
\end{pgfscope}%
\begin{pgfscope}%
\pgfsetroundcap%
\pgfsetroundjoin%
\pgfsetlinewidth{0.501875pt}%
\definecolor{currentstroke}{rgb}{0.000000,0.000000,0.000000}%
\pgfsetstrokecolor{currentstroke}%
\pgfsetdash{}{0pt}%
\pgfpathmoveto{\pgfqpoint{3.751932in}{0.386667in}}%
\pgfpathlineto{\pgfqpoint{3.807488in}{0.358889in}}%
\pgfpathlineto{\pgfqpoint{3.751932in}{0.331111in}}%
\pgfusepath{stroke}%
\end{pgfscope}%
\begin{pgfscope}%
\pgftext[x=3.870833in,y=0.358889in,left,]{\rmfamily\fontsize{10.000000}{12.000000}\selectfont \(\displaystyle k\)}%
\end{pgfscope}%
\end{pgfpicture}%
\makeatother%
\endgroup%
}
        \caption{Der Träger von $\hat\psi$ vor und nach der Substitution aus \cref{eq:substitution2_coords}}
        \label{fig:supp_psi_substitution2}
    \end{minipage}
\end{figure}
