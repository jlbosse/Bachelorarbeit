%%%%%%%%%%%%%%%%%%%%%%%%%%%%%%%%%%%%%%%%%%%%%%%%%%%%%%%%%%%%%%%%%%%%%%%%%%%%%%%%
% % Section 1
%%%%%%%%%%%%%%%%%%%%%%%%%%%%%%%%%%%%%%%%%%%%%%%%%%%%%%%%%%%%%%%%%%%%%%%%%%%%%%%%
\section{Allgemeines Gelaber über Shearlets} % (fold)
\label{sec:allgemeines_gelaber_ueber_shearlets}

\begin{theorem}[$\mathcal{S}_f(a,s,t)$ misst $WF(f)$]
\label{thm:main_theorem}
    Sei $\mathcal{D} = \mathcal{D}_1 \cup \mathcal{D}_2$ wobei
    $\mathcal{D}_1$ = \{
        $(t_0, s_0) \in \mathbb{R}^2 \times [-1,1] \big|$
        $|\mathcal{S}_f (a, s, t)| = O(a^k)$ gleichmäßig $\forall k \in \mathbb{N}
        , \forall t \in U$ Umgebung von $(t_0, s_0)$
    \}
    und $\mathcal{D}_2$ analog für $\psi^{(v)}$

    Dann gilt $WF(f)^c = \mathcal{D}$
\end{theorem}

\todo{diesen Satz richtig hin schreiben und ordentlich setzen}
\todo{Stil und Nummerierung für Sätze, Propositionen etc. anpassen}

\begin{corollary}[WF(f) misst $sing ~supp (\psi)$]
Sei $\mathcal{R} =$ \{
    $t_0 \in \mathcal{R}^2 \big|$ $|\mathcal{S}_f(a,s,t)| = O(a^k)$
    $\forall k \in \mathbb{N}, \forall t \in U$ Umgebung von $t_0$
    \}

    Dann gilt $sing ~supp (\psi)^c = \mathcal{R}$
\end{corollary}

\begin{remark}[Träger von $\psi$]

\begin{figure}[h]
\centering
%% Creator: Matplotlib, PGF backend
%%
%% To include the figure in your LaTeX document, write
%%   \input{<filename>.pgf}
%%
%% Make sure the required packages are loaded in your preamble
%%   \usepackage{pgf}
%%
%% Figures using additional raster images can only be included by \input if
%% they are in the same directory as the main LaTeX file. For loading figures
%% from other directories you can use the `import` package
%%   \usepackage{import}
%% and then include the figures with
%%   \import{<path to file>}{<filename>.pgf}
%%
%% Matplotlib used the following preamble
%%   \usepackage[utf8x]{inputenc}
%%   \usepackage[T1]{fontenc}
%%   \usepackage{amssymb}
%%
\begingroup%
\makeatletter%
\begin{pgfpicture}%
\pgfpathrectangle{\pgfpointorigin}{\pgfqpoint{4.000000in}{2.000000in}}%
\pgfusepath{use as bounding box, clip}%
\begin{pgfscope}%
\pgfsetbuttcap%
\pgfsetmiterjoin%
\definecolor{currentfill}{rgb}{1.000000,1.000000,1.000000}%
\pgfsetfillcolor{currentfill}%
\pgfsetlinewidth{0.000000pt}%
\definecolor{currentstroke}{rgb}{1.000000,1.000000,1.000000}%
\pgfsetstrokecolor{currentstroke}%
\pgfsetdash{}{0pt}%
\pgfpathmoveto{\pgfqpoint{0.000000in}{0.000000in}}%
\pgfpathlineto{\pgfqpoint{4.000000in}{0.000000in}}%
\pgfpathlineto{\pgfqpoint{4.000000in}{2.000000in}}%
\pgfpathlineto{\pgfqpoint{0.000000in}{2.000000in}}%
\pgfpathclose%
\pgfusepath{fill}%
\end{pgfscope}%
\begin{pgfscope}%
\pgfsetbuttcap%
\pgfsetmiterjoin%
\definecolor{currentfill}{rgb}{1.000000,1.000000,1.000000}%
\pgfsetfillcolor{currentfill}%
\pgfsetlinewidth{0.000000pt}%
\definecolor{currentstroke}{rgb}{0.000000,0.000000,0.000000}%
\pgfsetstrokecolor{currentstroke}%
\pgfsetstrokeopacity{0.000000}%
\pgfsetdash{}{0pt}%
\pgfpathmoveto{\pgfqpoint{0.500000in}{0.250000in}}%
\pgfpathlineto{\pgfqpoint{3.600000in}{0.250000in}}%
\pgfpathlineto{\pgfqpoint{3.600000in}{1.760000in}}%
\pgfpathlineto{\pgfqpoint{0.500000in}{1.760000in}}%
\pgfpathclose%
\pgfusepath{fill}%
\end{pgfscope}%
\begin{pgfscope}%
\pgfpathrectangle{\pgfqpoint{0.500000in}{0.250000in}}{\pgfqpoint{3.100000in}{1.510000in}} %
\pgfusepath{clip}%
\pgfsetbuttcap%
\pgfsetmiterjoin%
\definecolor{currentfill}{rgb}{0.500000,0.500000,0.500000}%
\pgfsetfillcolor{currentfill}%
\pgfsetfillopacity{0.500000}%
\pgfsetlinewidth{0.501875pt}%
\definecolor{currentstroke}{rgb}{0.000000,0.000000,0.000000}%
\pgfsetstrokecolor{currentstroke}%
\pgfsetdash{}{0pt}%
\pgfpathmoveto{\pgfqpoint{2.011250in}{0.438750in}}%
\pgfpathlineto{\pgfqpoint{2.088750in}{0.438750in}}%
\pgfpathlineto{\pgfqpoint{2.205000in}{0.552000in}}%
\pgfpathlineto{\pgfqpoint{1.895000in}{0.552000in}}%
\pgfpathclose%
\pgfusepath{stroke,fill}%
\end{pgfscope}%
\begin{pgfscope}%
\pgfpathrectangle{\pgfqpoint{0.500000in}{0.250000in}}{\pgfqpoint{3.100000in}{1.510000in}} %
\pgfusepath{clip}%
\pgfsetbuttcap%
\pgfsetmiterjoin%
\definecolor{currentfill}{rgb}{0.500000,0.500000,0.500000}%
\pgfsetfillcolor{currentfill}%
\pgfsetfillopacity{0.500000}%
\pgfsetlinewidth{0.501875pt}%
\definecolor{currentstroke}{rgb}{0.000000,0.000000,0.000000}%
\pgfsetstrokecolor{currentstroke}%
\pgfsetdash{}{0pt}%
\pgfpathmoveto{\pgfqpoint{1.949948in}{0.652667in}}%
\pgfpathlineto{\pgfqpoint{2.150052in}{0.652667in}}%
\pgfpathlineto{\pgfqpoint{2.450208in}{1.407667in}}%
\pgfpathlineto{\pgfqpoint{1.649792in}{1.407667in}}%
\pgfpathclose%
\pgfusepath{stroke,fill}%
\end{pgfscope}%
\begin{pgfscope}%
\pgfpathrectangle{\pgfqpoint{0.500000in}{0.250000in}}{\pgfqpoint{3.100000in}{1.510000in}} %
\pgfusepath{clip}%
\pgfsetbuttcap%
\pgfsetmiterjoin%
\definecolor{currentfill}{rgb}{0.500000,0.500000,0.500000}%
\pgfsetfillcolor{currentfill}%
\pgfsetfillopacity{0.500000}%
\pgfsetlinewidth{0.501875pt}%
\definecolor{currentstroke}{rgb}{0.000000,0.000000,0.000000}%
\pgfsetstrokecolor{currentstroke}%
\pgfsetdash{}{0pt}%
\pgfpathmoveto{\pgfqpoint{2.208281in}{0.652667in}}%
\pgfpathlineto{\pgfqpoint{2.408385in}{0.652667in}}%
\pgfpathlineto{\pgfqpoint{3.483542in}{1.407667in}}%
\pgfpathlineto{\pgfqpoint{2.683125in}{1.407667in}}%
\pgfpathclose%
\pgfusepath{stroke,fill}%
\end{pgfscope}%
\begin{pgfscope}%
\pgfpathrectangle{\pgfqpoint{0.500000in}{0.250000in}}{\pgfqpoint{3.100000in}{1.510000in}} %
\pgfusepath{clip}%
\pgfsetbuttcap%
\pgfsetroundjoin%
\pgfsetlinewidth{0.501875pt}%
\definecolor{currentstroke}{rgb}{0.501961,0.501961,0.501961}%
\pgfsetstrokecolor{currentstroke}%
\pgfsetdash{{1.850000pt}{0.800000pt}}{0.000000pt}%
\pgfpathmoveto{\pgfqpoint{1.880743in}{0.236111in}}%
\pgfpathlineto{\pgfqpoint{3.459257in}{1.773889in}}%
\pgfpathlineto{\pgfqpoint{3.459257in}{1.773889in}}%
\pgfusepath{stroke}%
\end{pgfscope}%
\begin{pgfscope}%
\pgfpathrectangle{\pgfqpoint{0.500000in}{0.250000in}}{\pgfqpoint{3.100000in}{1.510000in}} %
\pgfusepath{clip}%
\pgfsetbuttcap%
\pgfsetroundjoin%
\pgfsetlinewidth{0.501875pt}%
\definecolor{currentstroke}{rgb}{0.501961,0.501961,0.501961}%
\pgfsetstrokecolor{currentstroke}%
\pgfsetdash{{1.850000pt}{0.800000pt}}{0.000000pt}%
\pgfpathmoveto{\pgfqpoint{0.640743in}{1.773889in}}%
\pgfpathlineto{\pgfqpoint{2.219257in}{0.236111in}}%
\pgfpathlineto{\pgfqpoint{2.219257in}{0.236111in}}%
\pgfusepath{stroke}%
\end{pgfscope}%
\begin{pgfscope}%
\pgfsetrectcap%
\pgfsetmiterjoin%
\pgfsetlinewidth{0.501875pt}%
\definecolor{currentstroke}{rgb}{0.000000,0.000000,0.000000}%
\pgfsetstrokecolor{currentstroke}%
\pgfsetdash{}{0pt}%
\pgfpathmoveto{\pgfqpoint{2.050000in}{0.250000in}}%
\pgfpathlineto{\pgfqpoint{2.050000in}{1.760000in}}%
\pgfusepath{stroke}%
\end{pgfscope}%
\begin{pgfscope}%
\pgfsetrectcap%
\pgfsetmiterjoin%
\pgfsetlinewidth{0.501875pt}%
\definecolor{currentstroke}{rgb}{0.000000,0.000000,0.000000}%
\pgfsetstrokecolor{currentstroke}%
\pgfsetdash{}{0pt}%
\pgfpathmoveto{\pgfqpoint{0.500000in}{0.401000in}}%
\pgfpathlineto{\pgfqpoint{3.600000in}{0.401000in}}%
\pgfusepath{stroke}%
\end{pgfscope}%
\begin{pgfscope}%
\pgfsetroundcap%
\pgfsetroundjoin%
\pgfsetlinewidth{0.501875pt}%
\definecolor{currentstroke}{rgb}{0.000000,0.000000,0.000000}%
\pgfsetstrokecolor{currentstroke}%
\pgfsetdash{}{0pt}%
\pgfpathmoveto{\pgfqpoint{1.652485in}{0.528210in}}%
\pgfpathquadraticcurveto{\pgfqpoint{1.779212in}{0.521920in}}{\pgfqpoint{1.898185in}{0.516015in}}%
\pgfusepath{stroke}%
\end{pgfscope}%
\begin{pgfscope}%
\pgfsetroundcap%
\pgfsetroundjoin%
\pgfsetlinewidth{0.501875pt}%
\definecolor{currentstroke}{rgb}{0.000000,0.000000,0.000000}%
\pgfsetstrokecolor{currentstroke}%
\pgfsetdash{}{0pt}%
\pgfpathmoveto{\pgfqpoint{1.844075in}{0.546513in}}%
\pgfpathlineto{\pgfqpoint{1.898185in}{0.516015in}}%
\pgfpathlineto{\pgfqpoint{1.841321in}{0.491026in}}%
\pgfusepath{stroke}%
\end{pgfscope}%
\begin{pgfscope}%
\pgftext[x=0.887500in,y=0.514250in,left,base]{\rmfamily\fontsize{10.000000}{12.000000}\selectfont \(\displaystyle a = 1, s = 0\)}%
\end{pgfscope}%
\begin{pgfscope}%
\pgfsetroundcap%
\pgfsetroundjoin%
\pgfsetlinewidth{0.501875pt}%
\definecolor{currentstroke}{rgb}{0.000000,0.000000,0.000000}%
\pgfsetstrokecolor{currentstroke}%
\pgfsetdash{}{0pt}%
\pgfpathmoveto{\pgfqpoint{1.442466in}{1.092221in}}%
\pgfpathquadraticcurveto{\pgfqpoint{1.557990in}{1.086989in}}{\pgfqpoint{1.665758in}{1.082108in}}%
\pgfusepath{stroke}%
\end{pgfscope}%
\begin{pgfscope}%
\pgfsetroundcap%
\pgfsetroundjoin%
\pgfsetlinewidth{0.501875pt}%
\definecolor{currentstroke}{rgb}{0.000000,0.000000,0.000000}%
\pgfsetstrokecolor{currentstroke}%
\pgfsetdash{}{0pt}%
\pgfpathmoveto{\pgfqpoint{1.611516in}{1.112371in}}%
\pgfpathlineto{\pgfqpoint{1.665758in}{1.082108in}}%
\pgfpathlineto{\pgfqpoint{1.609002in}{1.056872in}}%
\pgfusepath{stroke}%
\end{pgfscope}%
\begin{pgfscope}%
\pgftext[x=0.500000in,y=1.080500in,left,base]{\rmfamily\fontsize{10.000000}{12.000000}\selectfont \(\displaystyle a = 0.15, s = 0\)}%
\end{pgfscope}%
\begin{pgfscope}%
\pgfsetroundcap%
\pgfsetroundjoin%
\pgfsetlinewidth{0.501875pt}%
\definecolor{currentstroke}{rgb}{0.000000,0.000000,0.000000}%
\pgfsetstrokecolor{currentstroke}%
\pgfsetdash{}{0pt}%
\pgfpathmoveto{\pgfqpoint{3.243990in}{0.689247in}}%
\pgfpathquadraticcurveto{\pgfqpoint{3.046540in}{0.802467in}}{\pgfqpoint{2.855825in}{0.911824in}}%
\pgfusepath{stroke}%
\end{pgfscope}%
\begin{pgfscope}%
\pgfsetroundcap%
\pgfsetroundjoin%
\pgfsetlinewidth{0.501875pt}%
\definecolor{currentstroke}{rgb}{0.000000,0.000000,0.000000}%
\pgfsetstrokecolor{currentstroke}%
\pgfsetdash{}{0pt}%
\pgfpathmoveto{\pgfqpoint{2.890202in}{0.860092in}}%
\pgfpathlineto{\pgfqpoint{2.855825in}{0.911824in}}%
\pgfpathlineto{\pgfqpoint{2.917837in}{0.908287in}}%
\pgfusepath{stroke}%
\end{pgfscope}%
\begin{pgfscope}%
\pgftext[x=2.980000in,y=0.552000in,left,base]{\rmfamily\fontsize{10.000000}{12.000000}\selectfont \(\displaystyle a = 0.15, s = 1\)}%
\end{pgfscope}%
\begin{pgfscope}%
\pgfsetroundcap%
\pgfsetroundjoin%
\pgfsetlinewidth{0.501875pt}%
\definecolor{currentstroke}{rgb}{0.000000,0.000000,0.000000}%
\pgfsetstrokecolor{currentstroke}%
\pgfsetdash{}{0pt}%
\pgfpathmoveto{\pgfqpoint{2.050000in}{1.766125in}}%
\pgfpathquadraticcurveto{\pgfqpoint{2.050000in}{1.766944in}}{\pgfqpoint{2.050000in}{1.760000in}}%
\pgfusepath{stroke}%
\end{pgfscope}%
\begin{pgfscope}%
\pgfsetroundcap%
\pgfsetroundjoin%
\pgfsetlinewidth{0.501875pt}%
\definecolor{currentstroke}{rgb}{0.000000,0.000000,0.000000}%
\pgfsetstrokecolor{currentstroke}%
\pgfsetdash{}{0pt}%
\pgfpathmoveto{\pgfqpoint{2.022222in}{1.710569in}}%
\pgfpathlineto{\pgfqpoint{2.050000in}{1.766125in}}%
\pgfpathlineto{\pgfqpoint{2.077778in}{1.710569in}}%
\pgfusepath{stroke}%
\end{pgfscope}%
\begin{pgfscope}%
\pgftext[x=2.050000in,y=1.829444in,,bottom]{\rmfamily\fontsize{10.000000}{12.000000}\selectfont \(\displaystyle k_1 / \omega\)}%
\end{pgfscope}%
\begin{pgfscope}%
\pgfsetroundcap%
\pgfsetroundjoin%
\pgfsetlinewidth{0.501875pt}%
\definecolor{currentstroke}{rgb}{0.000000,0.000000,0.000000}%
\pgfsetstrokecolor{currentstroke}%
\pgfsetdash{}{0pt}%
\pgfpathmoveto{\pgfqpoint{3.606104in}{0.401000in}}%
\pgfpathquadraticcurveto{\pgfqpoint{3.606934in}{0.401000in}}{\pgfqpoint{3.600000in}{0.401000in}}%
\pgfusepath{stroke}%
\end{pgfscope}%
\begin{pgfscope}%
\pgfsetroundcap%
\pgfsetroundjoin%
\pgfsetlinewidth{0.501875pt}%
\definecolor{currentstroke}{rgb}{0.000000,0.000000,0.000000}%
\pgfsetstrokecolor{currentstroke}%
\pgfsetdash{}{0pt}%
\pgfpathmoveto{\pgfqpoint{3.550548in}{0.428778in}}%
\pgfpathlineto{\pgfqpoint{3.606104in}{0.401000in}}%
\pgfpathlineto{\pgfqpoint{3.550548in}{0.373222in}}%
\pgfusepath{stroke}%
\end{pgfscope}%
\begin{pgfscope}%
\pgftext[x=3.669444in,y=0.401000in,left,]{\rmfamily\fontsize{10.000000}{12.000000}\selectfont \(\displaystyle k_2 / k\)}%
\end{pgfscope}%
\end{pgfpicture}%
\makeatother%
\endgroup%

\caption{Der Träger von $\hat \psi_{ast}$ für verschiedene $a, s$. Man sieht gut,
wie $supp (\hat \psi_{ast})$ für kleinere $a$ in immer kleineren Kegeln liegt.}
\end{figure}

\label{cor:psi_hat}
Im Fourierraum ist $\hat{\psi}_{ast}$ gegeben durch

\begin{equation}
    \hat \psi_{ast}{(\xi_1, \xi_2)} = a^{\frac{3}{4}}e^{-i\xi \cdot t}\hat\psi_1(a \xi_1) \hat\psi_{2}\left(a^{-\frac{1}{2}}\left(\frac{\xi_2}{\xi_1}-s\right)\right)
\end{equation}

und es gilt

\begin{equation}
\label{eq:supp_psi}
    supp(\hat \psi) \subset \left\{\xi \in  \hat{\mathbb{R}}^2 ~\Big| ~|\xi_1| \in \left[\frac{1}{2 a} , \frac{2}{a}\right], \left|\frac{\xi_2}{\xi_1} - s\right| \leq \sqrt{a} \right\}
\end{equation}

\end{remark}



% section allgemeines_gelaber_über_shearlets (end)


%%%%%%%%%%%%%%%%%%%%%%%%%%%%%%%%%%%%%%%%%%%%%%%%%%%%%%%%%%%%%%%%%%%%%%%%%%%%%%%%
% % Section 2
%%%%%%%%%%%%%%%%%%%%%%%%%%%%%%%%%%%%%%%%%%%%%%%%%%%%%%%%%%%%%%%%%%%%%%%%%%%%%%%%
\section{Zwei nützliche Substitionen für  $\left<\psi_{ast}, f\right>$}
\label{sec:substitutionen}

Zunächst werden wir zwei verschiedene Ausdrücke für $\left<\psi_{ast}, f\right>$
im Fourierraum herleiten, welche sich im dann folgenden als nützlich erweisen werden.

Sei also $\psi$ ein Shearlet wie in Korollar \ref{cor:psi_hat}. Sei $f$ die zu
analysierende fouriertransformierbare Funktion (oder Distribution) in
$\mathcal{D} \prime (\mathbb{R}^2)$. Dann ist $\mathcal{S}_f (ast)$ gegeben durch

\begin{align*}
\left< \psi_{ast}, f \right> &= \left<\hat\psi_{ast}, \hat f\right> \\
 &= \int a^\frac{3}{4} e^{-i \xi \cdot t} \hat \psi_1(a \xi_1)
    \hat \psi_2 \left(a^{-\frac{1}{2}} \left(\frac{\xi_2}{\xi_1} - s\right)\right)
    \hat f (\xi) \d \xi
\end{align*}

\todo{etnscheiden, was mit dem fehlenden Faktor $\frac{1}{(2 \pi)^n}$ geschieht}
und nach "`entscheren"' und "`deskalieren"', also der Substitution

\begin{equation*}
\begin{aligned}[c]
a \xi_1 &= k_1\\
a^{-\frac{1}{2}} \left(\frac{\xi_2}{\xi_1} - s\right) &=\frac{k_2}{k_1}\\
\end{aligned}
\qquad\Longleftrightarrow\qquad
\begin{aligned}[c]
\xi_1 &= \frac{k_1}{a}\\
\xi_2 &= \frac{k_1 s}{a} + a^{-\frac{1}{2}} k_2\\
\end{aligned}
\end{equation*}

\begin{equation*}
\Rightarrow
\d \xi_1 \d \xi_2 = a^{-\frac{3}{2}} \d k_1 \d k_2
\end{equation*}

ergibt sich folgendes für $\left<\psi_{ast}, f\right>$:

\begin{equation}
\label{eq:psi_ast_f_1}
    =  \iint a^{-\frac{3}{4}}~\hat \psi_1(k_1) ~\hat \psi_2 \left(\tfrac{k_2}{k_1}\right)
    ~\hat f \left(\tfrac{k_1}{a}, \tfrac{k_1 s}{a} + \tfrac{k_2}{\sqrt{a}}\right)
    ~e^{-i\frac{k_1}{a}(t_1+t_2 s) - i \frac{k_2 t_2}{\sqrt(a)}}
    \d k_1 \d k_2
\end{equation}

\todo{herausfinden, wie die Gleichungen auch Kapitelnummern erhalten}

Alternativ kann auch folgende Substitution

\begin{equation*}
\begin{aligned}[c]
a \xi_1 &= k_1\\
a^{-\frac{1}{2}} \left(\frac{\xi_2}{\xi_1} - s\right) &= k_2\\
\end{aligned}
\qquad\Longleftrightarrow\qquad
\begin{aligned}[c]
\xi_1 &= \frac{k_1}{a}\\
\xi_2 &= \left( a^\frac{1}{2} k_2 +s \right) \frac{k_1}{a}\\
\end{aligned}
\end{equation*}

\begin{equation*}
\Rightarrow
\d \xi_1 \d \xi_2 = a^{-\frac{3}{2}} k_1 \d k_1 \d k_2
\end{equation*}

gewählt werden, wodurch alle Parameter aus den Argumenten von $\psi_1, \psi_2$
verschwinden und sich

\begin{align}
\label{eq:psi_ast_f_2}
    =  \iint a^{-\frac{3}{4}}~ k_1~ \hat \psi_1(k_1)~ \hat \psi_2 (k_2)~
    \hat f \left(\tfrac{k_1}{a}, k_1 \left(a^{-\frac{1}{2}}k_2 + s a^{-1}\right)\right)
    ~e^{-i k_1 \left(\frac{t_1+s t_2}{a} + \frac{k_2 t_2}{\sqrt{a}}\right)}
    \d k_1 \d k_2
\end{align}

ergibt. Dabei ist zu beachten, dass diese Substitution zulässig ist, obwohl sie
die Oriertierung \emph{nicht} erhält und \emph{nicht} keine Bijektion ist. Aber
der kritische Bereich, nämlich $\xi_1 = 0$, liegt nicht im Träger von $\psi$.

\todo{Grafik basteln, die $supp ~\psi$ vor und nach der Substitution zeigt.}
