%!TEX root = main.tex
%%%%%%%%%%%%%%%%%%%%%%%%%%%%%%%%%%%%%%%%%%%%%%%%%%%%%%%%%%%%%%%%%%%%%%%%%%%%%%%
% % Section 1
%%%%%%%%%%%%%%%%%%%%%%%%%%%%%%%%%%%%%%%%%%%%%%%%%%%%%%%%%%%%%%%%%%%%%%%%%%%%%%%
\section{Wavelettransformation und die Wellenfrontmenge} % (fold)
\label{sec:shearlets}

Die klassische Fouriertransformation $f(x) \mapsto \int f(x) e^{-ik x} \d x$
zerlegt eine Funktion in ihre verschiedenen Frequenzanteile und misst nach dem Satz von Payley-Wiener dabei auch die Regularität der Funktion. Es gilt nämlich $f \in C^N(\mathbb{R}^n) \cap L^1(\mathbb{R}^n) \Rightarrow \hat f(k) = O(k^N) $ für $|\xi| \to \infty$. Leider "`sieht"' die Fouriertransformation aber nicht, wo $f$ singulär ist. Das hängt damit zusammen, dass die "`Basisfunktionen"', die ebenen Wellen, nicht lokalisiert sind. Das Argument der Fouriertransformation $\xi$ kontrolliert die Richtung und Skala, die von der Basisfunktion $e^{-ikx}$ aufgelöst werden. Ortsauflösung der Singularitäten gibt uns die




%%%%%%%%%%%%%%%%%%%%%%%%%%%%%%%%%%%%%%%%%%%%%%%%%%%%%%%%%%%%%%%%%%%%%%%%%%%%%%%
% % Wavelets
%%%%%%%%%%%%%%%%%%%%%%%%%%%%%%%%%%%%%%%%%%%%%%%%%%%%%%%%%%%%%%%%%%%%%%%%%%%%%%%
\subsection{Wavelettransformation} % (fold)
\label{sec:wavelettransformation}

% section wavelettransformation (end)

Einen Schritt in die richtige Richtung, nämlich die Ortsauflösung der Singularitäten, macht die Wavelettransformation. Hier wird wieder eine Familie von Basisfunktionen erzeugt von einem \textit{Mutterwavelet} $\psi$. Anders als die Ebenen Wellen ist $\psi$ aber lokalisiert -- häufig sogar kompakt getragen -- und die Basis wird erzeugt durch Verschieben und Skalieren des Mutterwavelets.

Eine Hamel-Basis für $L^2(\mathbb{R}^n)$ die aus Funktionen der Form

\begin{equation*}
    \left\{\psi_{at}(x) = a^{-\frac{n}{2}}\psi\left(a^{-1}(x-t)\right)  |~ t \in \mathbb{R}^n,  ~a \in \mathbb{R}\right\}
\end{equation*}

 mit einem \textit{Mutterwavelet} $\psi \in C^\infty (\mathbb{R}^n)$ besteht heißt \textit{Waveletbasis} für $L^2(\mathbb{R}^n)$. Der Parameter $t$ heißt \textit{Verschiebungsparameter} und verschiebt das Wavelet an alle Orte des $\mathbb{R}^n$ während der \textit{Skalierungsparameter} $a$ für $a \to 0$ immer genauer lokalisiert. Der Faktor $a^{-\frac{n}{2}}$ sorgt dafür, dass die $L^2$-Norm aller $\psi_{at}$ gleich ist. In der Fourierdomäne wird die Verschiebung zum Phasenfaktor und der Träger mit verschwindendem $a$ immer \emph{größer}.
% Noch einmal als Formel:

% \begin{align*}
%     supp (\psi_{at}) &= a \cdot supp(\psi)+t \\
%     \Longleftrightarrow ~~
%     supp (\hat\psi_{at}) &= a^{-1} \cdot supp(\hat\psi_{at})
% \end{align*}

% Ist $t \in \mathbb{Z}^n$ (oder auf einem anderen diskreten Gitter in $\mathbb{R}^n$) und $a = 2^{-j}$, so spricht man von der \textit{diskreten Wavelettransformation}. Ist $t \in \mathbb{R}^n$ und $G \subseteq GL(n,\mathbb{R})$ so ist es eine \textit{stetige Wavelettransformation} (engl. \textit{continuous wavelet transform}). Die Wavelettransformation einer Funktion (später auch temperierten Distribution) $f$ ist dann definiert als

Analog zur Definition der Fouriertransformation ist die Wavelettransformation von $f$ die Projektion auf die Basisfunktionen:
\begin{equation}
    \mathcal{W}_f (a,t) = \left\langle\psi_{at},f\right\rangle
    = a^{-\frac{n}{2}} \int f(x) \psi \left(a^{-1}(x-t)\right) \d x
\end{equation}

Ist $\psi$ eine glatte Funktion und $f$ bei $t$ glatt, so fällt $\mathcal{W}_f (a,t)$ schnell ab für $a \to 0$.
% Dies ist schnell einzusehen, wenn man sich überlegt, dass $\psi \in C^k$ impliziert, dass $\hat\psi$ eine $k$-fache Nullstelle bei $0$ hat und $\int x^l \psi(x) \d x = (-i)^{l} \partial_k^l \hat\psi(0) = 0 ~~\textrm{falls}~ l<k $.
\todo[color=green]{Kurze Plausibel machen, warum dem so ist?}
Umgekehrt fällt auch $\mathcal{W}_f(a,t)$ \emph{genau dann} nicht schnell ab, wenn $f$ bei $t$ \emph{nicht} glatt ist. Also löst die Wavelettransformation die Lage der Singularitäten von $f$ auf. Allerdings sind die klassischen  Wavelettransformationen mit isotroper Skalierung nicht in der Lage die  Orientierung der Singularitäten aufzulösen. Sie besitzen ja gar keinen Orientierungsparameter.

Um auch hier noch weiter zu kommen, muss einerseits ein Richtungsparameter eingeführt werden und andererseits dafür gesorgt werden, dass die Basisfunktionen mit immer feinerer Skala immer orientierter werden. Deshalb gibt es


%%%%%%%%%%%%%%%%%%%%%%%%%%%%%%%%%%%%%%%%%%%%%%%%%%%%%%%%%%%%%%%%%%%%%%%%%%%%%%%
% % gerichtete Wavelets
%%%%%%%%%%%%%%%%%%%%%%%%%%%%%%%%%%%%%%%%%%%%%%%%%%%%%%%%%%%%%%%%%%%%%%%%%%%%%%%
\subsection{Verallgemeinerte, gerichtete Wavelets} % (fold)
\label{sec:verallgemeinerte_gerichtete_wavelets}

% section verallgemeinerte_gerichtete_wavelets (end)

 Die bekanntesten Beispiele solcher gerichteter Wavelets sind die \textit{Curvelets} von \textcite{Candes2005} sowie die \textit{Shearlets} von \textcite{Kutyniok2008}. Die mit feiner werdenden Skala schärfer werdende Skalierung wird in beiden Fällen durch parabolische Skalierung implementiert. D.h. in Richtung der Orientierung wird mit $a$ skaliert, während in den Richtungen senkrecht dazu mit $\sqrt a$ skaliert wird. In zwei Dimensionen gibt es nur eine weitere senkrechte Richtung, aber später wird deutlich werden, dass dies in mehr Dimensionen die richtige Verallgemeinerung sein muss.

 Die Richtung der Curvelets wird durch Drehmatrizen implementiert, die auf die Variablen $(t,x)$ wirken, während bei den Shearlets die Variablen $(t,x)$ geschert werden.

%  Beide Ansätze sind bisher nur in zwei Dimensionen im Detail untersucht und arbeiten mit parabolischer Skalierung. Im Fall der \textit{Curvelets} wird die Richtungsabhängigkeit durch Drehmatrizen implementiert, während die \textit{Shearlets} mit Scherungen arbeiten. Konkret sieht dass dann so aus:

% \todo[color=green]{Hier die Formeln hin schreiben, oder ganz drauf verzichten}

% \begin{align*}
%     \psi_{a\theta t}(x) &= \det (AR)^{-\frac{1}{2}}
%                     \psi \left((AR)^{-1}(x-t) \right)
%     \condition{für Curvelets}\\
%     \psi_{ast}(x) &= \det (AS)^{-\frac{1}{2}}
%                     \psi \left((AS)^{-1}(x-t)\right)
%     \condition{für Shearlets}
% \end{align*}

% \begin{dgroup*}
% \begin{dsuspend}
%     mit der parabolischen Skalierungsmatrix
% \end{dsuspend}
% \begin{dmath*}
%     A = \begin{pmatrix}  a & 0 \\ 0 & \sqrt a\end{pmatrix}
% \end{dmath*}
% \begin{dsuspend}
%     der Drehmatrix
% \end{dsuspend}
% \begin{dmath*}
%     R = \begin{pmatrix}  \cos \theta & \sin \theta \\ - \sin \theta & \cos \theta \end{pmatrix}
% \end{dmath*}
% \begin{dsuspend}
%     und der Scherungsmatrix
% \end{dsuspend}
% \begin{dmath*}
%     S = \begin{pmatrix}  1 & -s \\ 0 & 1\end{pmatrix}
% \end{dmath*}
% \end{dgroup*}

Beide Ansätze sind in der Lage, die Wellenfrontmenge einer Distribution zu identifizieren. Allerdings sind die Rechnungen bei den \textit{Shearlets} in der praktischen Umsetzung einfacher, wenn auch sie von einem ästhetischen Standpunkt nicht ganz so befriedigend, da sie nicht inhärent Rotationsinvariant sind, also nicht alle Symmetrien unseres Raumes abbilden. Aber nach allzu viel Ästhetik kann man in dieser Arbeit, mit Hinblick auf die Rechnungen ab \cref{sec:die_wellenfrontmenge_von_delta_m}, ohnehin nicht fragen.

Bevor wir die konkrete Konstruktion der Shearlets widmen, brauchen wir noch ein kleines bisschen Theorie, welche Möglichkeiten wir überhaupt haben, um die Konstruktion der Wavelets zu verallgemeinern. Die weitestgehende Verallgemeinerung von "`verschiebe und skaliere ein Mutterwavelet"' um ein reproduzierendes System zu erhalten ist "`verschiebe es und lasse eine beliebge invertierbare Matrix auf die Koordinaten wirken"'. Wir definieren also eine Wirkung der affinen Gruppe $\mathbb{A}^n$ auf Funktionen $\psi \in L^2(\mathbb{R}^n)$ via

\begin{equation*}
   ((M,x') ,\psi (x)) \mapsto |\det M ^{-\frac{1}{2}}|  \psi\left(M^{-1}(x-x')\right) \eqqcolon \psi_{M,x'} (x)
\end{equation*}

mit

\begin{equation*}
    (M,x') \in \mathbb{A}^n = GL(n,\mathbb{R}) \ltimes \mathbb{R}^n
\end{equation*}

Im Allgemeinen wird man aber nicht die ganze affine Gruppe benötigen, um ein reproduzierendes System zu erhalten, sondern nur alle Verschiebungen und eine Untermenge\footnote{Auch nicht notwendigerweise Untergruppe} der $GL(n,\mathbb{R})$. Wann ein Mutterwavelet und die Wirkung einer solchen Untermenge ein reproduzierendes System erzeugen, sagt uns der nächste Satz:

\begin{theorem}[Zulässigkeitskriterium]
\label{thm:admissibility_criterion}
    Sei $\psi \in L^2(\mathbb{R}^n)$.
    Sei $G \subset GL(n,\mathbb{R})$, $\d \mu(M)$ ein Maß auf $G$, im Falle einer Untergruppe z.B. das Haarmaß und es gelte
    \begin{equation}
        \Delta(\psi)(\xi) = \int_G |\hat\psi(M^t \xi)|^2 |\det M| \d \mu(M) = 1
    \label{eq:admissibility criterion}
    \end{equation}

    Dann ist $(\psi, G\ltimes \mathbb{R}^n)$ ein reproduzierendes System für $L^2(\mathbb{R}^n)$, also für alle

    \begin{equation}
        f = \int_{\mathbb{R}^n} \int_G \left\langle \psi_{M,x'},f\right\rangle
            \psi_{M,x'} \d \mu (M) \d x'
    \end{equation}
\end{theorem}

\begin{remark}[\ref{thm:admissibility_criterion} ist Calderon]
    Für $n=1$ ist $GL(1,\mathbb{R}) = (R^*, \cdot)$ mit dem Haarmaß $\d \mu(a) = \frac{\d \lambda(a)}{a}$ und \cref{eq:admissibility criterion} wird zu

    \begin{equation}
        \int_0^\infty \left\lvert \hat\psi(a k) \right\rvert^2 \frac{\d \lambda(a)}{a} = 1 \condition{für fast alle $k \in \hat{\mathbb{R}}$}
    \end{equation}
    \cref{eq:admissibility criterion} ist also das mehrdimensionale Analogon zu Calderons Kriterium \cite[S. 105]{Mallat2008}
\end{remark}

Jetzt aber mehr Details zur Konstruktion der Shearlets und deren Eigenschaften:


%%%%%%%%%%%%%%%%%%%%%%%%%%%%%%%%%%%%%%%%%%%%%%%%%%%%%%%%%%%%%%%%%%%%%%%%%%%%%%%
% % Die Shearlets
%%%%%%%%%%%%%%%%%%%%%%%%%%%%%%%%%%%%%%%%%%%%%%%%%%%%%%%%%%%%%%%%%%%%%%%%%%%%%%%
\subsection{Konstruktion, Eigenschaften der Shearlets und ein wichtiger Satz} % (fold)
\label{sec:konstruktion_und_eigenschaften_der_shearlets}

Der folgende Abschnitt basiert größtenteils auf der Arbeit in \textcite{Kutyniok2008}, allerdings wurde die Notation den später berechneten Problemen angepasst und als Ortskoordination $(t,x)$ verwendet sowie als Koordinaten des Dualraums $(\omega, k)$ entsprechend der Konvention in der Physik. Auch Fouriertransformation wird mit $e^{-i\omega t + ikx}$ gerechnet. Dies ändert allerdings nichts essentielles. Da wir später auch komplexwertige Distributionen analysieren wollen, deren Wellenfrontmenge nicht zwingend punktsymmetrisch um den Ursprung (in der Richtung, nicht im Ort) sind, werden wir Shearlets verwenden, deren Fouriertransformierte asymetrischen Träger hat, indem wir die Shearlets aus \cite{Kutyniok2008} jeweils in zwei Shearlets aufteilen, eines mit Träger im Frequenzbereich "`nach vorne"', und eines mit Träger "`nach hinten"'. \todo{Grafik mit der "Parzellierung" des Frequenzbereiches}

\begin{definition}[Shearlettransformation]
\label{def:shearletttransformationI}
Seien
% \begin{dgroup*}
\begin{equation}
    \psi_1 \in L^2(\mathbb{R})
            \textrm{ mit }supp(\hat\psi_1) \subseteq \left[\frac{1}{2},2\right]
            \textrm{ und } \psi_1 \textrm{ erfüllt \cref{eq:admissibility criterion}}
            \footnote{mit $G = (\mathbb{R}^*, \cdot), |\det M| d \lambda(M) = \frac{\d a}{a}$}
\label{eq:psi_1}
\end{equation}
\begin{equation}
    \psi_2 \in L^2(\mathbb{R})
            \textrm{ mit } supp(\hat\psi_1) \subseteq \left[-1,1\right]
            \textrm{ und } \left\Vert\psi_2 \right\Vert_2 = 1
\label{eq:psi_2}
\end{equation}
% \end{dgroup*}
und $\psi \in C^\infty(\mathbb{R}^2)$ implizit definiert durch

\begin{equation}
    \hat \psi(\omega,k) = \hat\psi_1(\omega) \hat \psi_2 \left(\frac{k}{\omega}\right)
\end{equation}

Sei des weiteren

\begin{equation}
    G = \left\{M_{as} \in GL(2,\mathbb{R}) \Big | M_{as} = \left(\begin{smallmatrix}
        a & -\sqrt a s \\ 0 & \sqrt a
    \end{smallmatrix}\right)
    , a \in [0,1], s \in [-2,2]
    \right\}
    \label{eq:schermatrizen}
\end{equation}

Dann ist für $t \in \mathbb{R}^n, M_{as} \in G$ die Shearlettransformation von $f$ bezüglich $\psi$ definiert als

\begin{equation}
    \mathcal{S}_f(a,s,(t',x'))) =
    \left\langle D_{M_{as}}T_{(t',x')}\psi, f
    \right\rangle
    = a^{-\frac{3}{4}}\int f(t,x) \psi\left(
    \left(\begin{smallmatrix}
        a & -\sqrt a s \\ 0 & \sqrt a
    \end{smallmatrix}\right)^{-1}
    \left(\begin{smallmatrix}
        t-t' \\ x-x'
    \end{smallmatrix}\right)
    \right) \d t \d x
\end{equation}

\end{definition}

Bevor es mit dem Text weiter geht, noch eine kurze Bemerkung zu Vereinfachung der Notation:

\begin{remark}[Notation]
    Der Kompaktheit halber schreiben wir auch
    \begin{equation*}
        \psi_{ast} (t,x) \coloneqq \psi_{as{(t',x')}} (t,x) \coloneqq
        a^{-\frac{3}{4}} \psi\left(
    \left(\begin{smallmatrix}
        a & -\sqrt a s \\ 0 & \sqrt a
    \end{smallmatrix}\right)^{-1}
    \left(\begin{smallmatrix}
        t-t' \\ x-x'
    \end{smallmatrix}\right)
    \right)
    \end{equation*}
    wobei in diesem Fall der Index $t$ für den Verschiebungsparameter steht, nicht für die Koordinate $t$.
\end{remark}

Offenbar können $\psi_1$ und $\psi_2$ in \cref{def:shearletttransformationI} problemlos auch so gewählt werden, dass $\hat\psi_1$ und $\hat\psi_2$ glatt sind; wir stellen in \cref{eq:psi_1,eq:psi_2} ja keine allzu restriktiven Anforderungen an sie. Dann ist $\psi_{ast}$ eine Schwartz-Funktion für alle $(a,s,(t',x'))$ und die Shearlettransformation temperierter Distributionen ist wohldefiniert.  Die Anforderungen aus \cref{eq:psi_1,eq:psi_2} sind so gewählt, dass \cref{eq:admissibility criterion} von $\psi$ erfüllt wird, und gleichzeitig die konkreten Rechnungen zur Bestimmung der Wellenfrontmenge auch analytisch möglich sind.

Der kompakte Träger von $\hat\psi$ in der Frequenzdomäne erlaubt einfachere
Abschätzungen von Ausdrücken der Form $\left<\hat \psi_{ast}, \hat f\right>$, ist aber \emph{nicht} zwingend notwendig, um mit diesem Shearlet die Wellenfrontmenge zu bestimmen.


Die Wirkung der Scher- und Skalierungsmatrizen aus \cref{eq:schermatrizen} versteht man am besten in der Frequenzdomäne. Mit $a \to 0$ wird $\hat \psi$ immer weiter "`nach außen"' geschoben in der Frequenzdomäne und der Träger wird gleichzeitig immer schmaler bzw. gerichteter. Dies ist genau die Anisotropie, die uns erlaubt, nicht nur die Position der Singularitäten, sondern auch ihre Orientierung zu erkennen. Der Parameter bestimmt die Scherung des Trägers von $\psi$. Für $s=0$ ist der Träger um $k=0$ herum lokalisiert, für $s = \pm 1$ um die Diagonale bzw Antidiagonale. Siehe auch \cref{fig:supp_psi_hat} und \cref{rem:psi_hat}.


\begin{remark}[Eigenschaften von $\hat\psi_{ast}$]
\label{rem:psi_hat}
\begin{figure}[h]
\centering
%% Creator: Matplotlib, PGF backend
%%
%% To include the figure in your LaTeX document, write
%%   \input{<filename>.pgf}
%%
%% Make sure the required packages are loaded in your preamble
%%   \usepackage{pgf}
%%
%% Figures using additional raster images can only be included by \input if
%% they are in the same directory as the main LaTeX file. For loading figures
%% from other directories you can use the `import` package
%%   \usepackage{import}
%% and then include the figures with
%%   \import{<path to file>}{<filename>.pgf}
%%
%% Matplotlib used the following preamble
%%   \usepackage[utf8x]{inputenc}
%%   \usepackage[T1]{fontenc}
%%   \usepackage{amssymb}
%%
\begingroup%
\makeatletter%
\begin{pgfpicture}%
\pgfpathrectangle{\pgfpointorigin}{\pgfqpoint{4.000000in}{2.000000in}}%
\pgfusepath{use as bounding box, clip}%
\begin{pgfscope}%
\pgfsetbuttcap%
\pgfsetmiterjoin%
\definecolor{currentfill}{rgb}{1.000000,1.000000,1.000000}%
\pgfsetfillcolor{currentfill}%
\pgfsetlinewidth{0.000000pt}%
\definecolor{currentstroke}{rgb}{1.000000,1.000000,1.000000}%
\pgfsetstrokecolor{currentstroke}%
\pgfsetdash{}{0pt}%
\pgfpathmoveto{\pgfqpoint{0.000000in}{0.000000in}}%
\pgfpathlineto{\pgfqpoint{4.000000in}{0.000000in}}%
\pgfpathlineto{\pgfqpoint{4.000000in}{2.000000in}}%
\pgfpathlineto{\pgfqpoint{0.000000in}{2.000000in}}%
\pgfpathclose%
\pgfusepath{fill}%
\end{pgfscope}%
\begin{pgfscope}%
\pgfsetbuttcap%
\pgfsetmiterjoin%
\definecolor{currentfill}{rgb}{1.000000,1.000000,1.000000}%
\pgfsetfillcolor{currentfill}%
\pgfsetlinewidth{0.000000pt}%
\definecolor{currentstroke}{rgb}{0.000000,0.000000,0.000000}%
\pgfsetstrokecolor{currentstroke}%
\pgfsetstrokeopacity{0.000000}%
\pgfsetdash{}{0pt}%
\pgfpathmoveto{\pgfqpoint{0.500000in}{0.250000in}}%
\pgfpathlineto{\pgfqpoint{3.600000in}{0.250000in}}%
\pgfpathlineto{\pgfqpoint{3.600000in}{1.760000in}}%
\pgfpathlineto{\pgfqpoint{0.500000in}{1.760000in}}%
\pgfpathclose%
\pgfusepath{fill}%
\end{pgfscope}%
\begin{pgfscope}%
\pgfpathrectangle{\pgfqpoint{0.500000in}{0.250000in}}{\pgfqpoint{3.100000in}{1.510000in}} %
\pgfusepath{clip}%
\pgfsetbuttcap%
\pgfsetmiterjoin%
\definecolor{currentfill}{rgb}{0.500000,0.500000,0.500000}%
\pgfsetfillcolor{currentfill}%
\pgfsetfillopacity{0.500000}%
\pgfsetlinewidth{0.501875pt}%
\definecolor{currentstroke}{rgb}{0.000000,0.000000,0.000000}%
\pgfsetstrokecolor{currentstroke}%
\pgfsetdash{}{0pt}%
\pgfpathmoveto{\pgfqpoint{2.011250in}{0.438750in}}%
\pgfpathlineto{\pgfqpoint{2.088750in}{0.438750in}}%
\pgfpathlineto{\pgfqpoint{2.205000in}{0.552000in}}%
\pgfpathlineto{\pgfqpoint{1.895000in}{0.552000in}}%
\pgfpathclose%
\pgfusepath{stroke,fill}%
\end{pgfscope}%
\begin{pgfscope}%
\pgfpathrectangle{\pgfqpoint{0.500000in}{0.250000in}}{\pgfqpoint{3.100000in}{1.510000in}} %
\pgfusepath{clip}%
\pgfsetbuttcap%
\pgfsetmiterjoin%
\definecolor{currentfill}{rgb}{0.500000,0.500000,0.500000}%
\pgfsetfillcolor{currentfill}%
\pgfsetfillopacity{0.500000}%
\pgfsetlinewidth{0.501875pt}%
\definecolor{currentstroke}{rgb}{0.000000,0.000000,0.000000}%
\pgfsetstrokecolor{currentstroke}%
\pgfsetdash{}{0pt}%
\pgfpathmoveto{\pgfqpoint{1.949948in}{0.652667in}}%
\pgfpathlineto{\pgfqpoint{2.150052in}{0.652667in}}%
\pgfpathlineto{\pgfqpoint{2.450208in}{1.407667in}}%
\pgfpathlineto{\pgfqpoint{1.649792in}{1.407667in}}%
\pgfpathclose%
\pgfusepath{stroke,fill}%
\end{pgfscope}%
\begin{pgfscope}%
\pgfpathrectangle{\pgfqpoint{0.500000in}{0.250000in}}{\pgfqpoint{3.100000in}{1.510000in}} %
\pgfusepath{clip}%
\pgfsetbuttcap%
\pgfsetmiterjoin%
\definecolor{currentfill}{rgb}{0.500000,0.500000,0.500000}%
\pgfsetfillcolor{currentfill}%
\pgfsetfillopacity{0.500000}%
\pgfsetlinewidth{0.501875pt}%
\definecolor{currentstroke}{rgb}{0.000000,0.000000,0.000000}%
\pgfsetstrokecolor{currentstroke}%
\pgfsetdash{}{0pt}%
\pgfpathmoveto{\pgfqpoint{2.208281in}{0.652667in}}%
\pgfpathlineto{\pgfqpoint{2.408385in}{0.652667in}}%
\pgfpathlineto{\pgfqpoint{3.483542in}{1.407667in}}%
\pgfpathlineto{\pgfqpoint{2.683125in}{1.407667in}}%
\pgfpathclose%
\pgfusepath{stroke,fill}%
\end{pgfscope}%
\begin{pgfscope}%
\pgfpathrectangle{\pgfqpoint{0.500000in}{0.250000in}}{\pgfqpoint{3.100000in}{1.510000in}} %
\pgfusepath{clip}%
\pgfsetbuttcap%
\pgfsetroundjoin%
\pgfsetlinewidth{0.501875pt}%
\definecolor{currentstroke}{rgb}{0.501961,0.501961,0.501961}%
\pgfsetstrokecolor{currentstroke}%
\pgfsetdash{{1.850000pt}{0.800000pt}}{0.000000pt}%
\pgfpathmoveto{\pgfqpoint{1.880743in}{0.236111in}}%
\pgfpathlineto{\pgfqpoint{3.459257in}{1.773889in}}%
\pgfpathlineto{\pgfqpoint{3.459257in}{1.773889in}}%
\pgfusepath{stroke}%
\end{pgfscope}%
\begin{pgfscope}%
\pgfpathrectangle{\pgfqpoint{0.500000in}{0.250000in}}{\pgfqpoint{3.100000in}{1.510000in}} %
\pgfusepath{clip}%
\pgfsetbuttcap%
\pgfsetroundjoin%
\pgfsetlinewidth{0.501875pt}%
\definecolor{currentstroke}{rgb}{0.501961,0.501961,0.501961}%
\pgfsetstrokecolor{currentstroke}%
\pgfsetdash{{1.850000pt}{0.800000pt}}{0.000000pt}%
\pgfpathmoveto{\pgfqpoint{0.640743in}{1.773889in}}%
\pgfpathlineto{\pgfqpoint{2.219257in}{0.236111in}}%
\pgfpathlineto{\pgfqpoint{2.219257in}{0.236111in}}%
\pgfusepath{stroke}%
\end{pgfscope}%
\begin{pgfscope}%
\pgfsetrectcap%
\pgfsetmiterjoin%
\pgfsetlinewidth{0.501875pt}%
\definecolor{currentstroke}{rgb}{0.000000,0.000000,0.000000}%
\pgfsetstrokecolor{currentstroke}%
\pgfsetdash{}{0pt}%
\pgfpathmoveto{\pgfqpoint{2.050000in}{0.250000in}}%
\pgfpathlineto{\pgfqpoint{2.050000in}{1.760000in}}%
\pgfusepath{stroke}%
\end{pgfscope}%
\begin{pgfscope}%
\pgfsetrectcap%
\pgfsetmiterjoin%
\pgfsetlinewidth{0.501875pt}%
\definecolor{currentstroke}{rgb}{0.000000,0.000000,0.000000}%
\pgfsetstrokecolor{currentstroke}%
\pgfsetdash{}{0pt}%
\pgfpathmoveto{\pgfqpoint{0.500000in}{0.401000in}}%
\pgfpathlineto{\pgfqpoint{3.600000in}{0.401000in}}%
\pgfusepath{stroke}%
\end{pgfscope}%
\begin{pgfscope}%
\pgfsetroundcap%
\pgfsetroundjoin%
\pgfsetlinewidth{0.501875pt}%
\definecolor{currentstroke}{rgb}{0.000000,0.000000,0.000000}%
\pgfsetstrokecolor{currentstroke}%
\pgfsetdash{}{0pt}%
\pgfpathmoveto{\pgfqpoint{1.652485in}{0.528210in}}%
\pgfpathquadraticcurveto{\pgfqpoint{1.779212in}{0.521920in}}{\pgfqpoint{1.898185in}{0.516015in}}%
\pgfusepath{stroke}%
\end{pgfscope}%
\begin{pgfscope}%
\pgfsetroundcap%
\pgfsetroundjoin%
\pgfsetlinewidth{0.501875pt}%
\definecolor{currentstroke}{rgb}{0.000000,0.000000,0.000000}%
\pgfsetstrokecolor{currentstroke}%
\pgfsetdash{}{0pt}%
\pgfpathmoveto{\pgfqpoint{1.844075in}{0.546513in}}%
\pgfpathlineto{\pgfqpoint{1.898185in}{0.516015in}}%
\pgfpathlineto{\pgfqpoint{1.841321in}{0.491026in}}%
\pgfusepath{stroke}%
\end{pgfscope}%
\begin{pgfscope}%
\pgftext[x=0.887500in,y=0.514250in,left,base]{\rmfamily\fontsize{10.000000}{12.000000}\selectfont \(\displaystyle a = 1, s = 0\)}%
\end{pgfscope}%
\begin{pgfscope}%
\pgfsetroundcap%
\pgfsetroundjoin%
\pgfsetlinewidth{0.501875pt}%
\definecolor{currentstroke}{rgb}{0.000000,0.000000,0.000000}%
\pgfsetstrokecolor{currentstroke}%
\pgfsetdash{}{0pt}%
\pgfpathmoveto{\pgfqpoint{1.442466in}{1.092221in}}%
\pgfpathquadraticcurveto{\pgfqpoint{1.557990in}{1.086989in}}{\pgfqpoint{1.665758in}{1.082108in}}%
\pgfusepath{stroke}%
\end{pgfscope}%
\begin{pgfscope}%
\pgfsetroundcap%
\pgfsetroundjoin%
\pgfsetlinewidth{0.501875pt}%
\definecolor{currentstroke}{rgb}{0.000000,0.000000,0.000000}%
\pgfsetstrokecolor{currentstroke}%
\pgfsetdash{}{0pt}%
\pgfpathmoveto{\pgfqpoint{1.611516in}{1.112371in}}%
\pgfpathlineto{\pgfqpoint{1.665758in}{1.082108in}}%
\pgfpathlineto{\pgfqpoint{1.609002in}{1.056872in}}%
\pgfusepath{stroke}%
\end{pgfscope}%
\begin{pgfscope}%
\pgftext[x=0.500000in,y=1.080500in,left,base]{\rmfamily\fontsize{10.000000}{12.000000}\selectfont \(\displaystyle a = 0.15, s = 0\)}%
\end{pgfscope}%
\begin{pgfscope}%
\pgfsetroundcap%
\pgfsetroundjoin%
\pgfsetlinewidth{0.501875pt}%
\definecolor{currentstroke}{rgb}{0.000000,0.000000,0.000000}%
\pgfsetstrokecolor{currentstroke}%
\pgfsetdash{}{0pt}%
\pgfpathmoveto{\pgfqpoint{3.243990in}{0.689247in}}%
\pgfpathquadraticcurveto{\pgfqpoint{3.046540in}{0.802467in}}{\pgfqpoint{2.855825in}{0.911824in}}%
\pgfusepath{stroke}%
\end{pgfscope}%
\begin{pgfscope}%
\pgfsetroundcap%
\pgfsetroundjoin%
\pgfsetlinewidth{0.501875pt}%
\definecolor{currentstroke}{rgb}{0.000000,0.000000,0.000000}%
\pgfsetstrokecolor{currentstroke}%
\pgfsetdash{}{0pt}%
\pgfpathmoveto{\pgfqpoint{2.890202in}{0.860092in}}%
\pgfpathlineto{\pgfqpoint{2.855825in}{0.911824in}}%
\pgfpathlineto{\pgfqpoint{2.917837in}{0.908287in}}%
\pgfusepath{stroke}%
\end{pgfscope}%
\begin{pgfscope}%
\pgftext[x=2.980000in,y=0.552000in,left,base]{\rmfamily\fontsize{10.000000}{12.000000}\selectfont \(\displaystyle a = 0.15, s = 1\)}%
\end{pgfscope}%
\begin{pgfscope}%
\pgfsetroundcap%
\pgfsetroundjoin%
\pgfsetlinewidth{0.501875pt}%
\definecolor{currentstroke}{rgb}{0.000000,0.000000,0.000000}%
\pgfsetstrokecolor{currentstroke}%
\pgfsetdash{}{0pt}%
\pgfpathmoveto{\pgfqpoint{2.050000in}{1.766125in}}%
\pgfpathquadraticcurveto{\pgfqpoint{2.050000in}{1.766944in}}{\pgfqpoint{2.050000in}{1.760000in}}%
\pgfusepath{stroke}%
\end{pgfscope}%
\begin{pgfscope}%
\pgfsetroundcap%
\pgfsetroundjoin%
\pgfsetlinewidth{0.501875pt}%
\definecolor{currentstroke}{rgb}{0.000000,0.000000,0.000000}%
\pgfsetstrokecolor{currentstroke}%
\pgfsetdash{}{0pt}%
\pgfpathmoveto{\pgfqpoint{2.022222in}{1.710569in}}%
\pgfpathlineto{\pgfqpoint{2.050000in}{1.766125in}}%
\pgfpathlineto{\pgfqpoint{2.077778in}{1.710569in}}%
\pgfusepath{stroke}%
\end{pgfscope}%
\begin{pgfscope}%
\pgftext[x=2.050000in,y=1.829444in,,bottom]{\rmfamily\fontsize{10.000000}{12.000000}\selectfont \(\displaystyle k_1 / \omega\)}%
\end{pgfscope}%
\begin{pgfscope}%
\pgfsetroundcap%
\pgfsetroundjoin%
\pgfsetlinewidth{0.501875pt}%
\definecolor{currentstroke}{rgb}{0.000000,0.000000,0.000000}%
\pgfsetstrokecolor{currentstroke}%
\pgfsetdash{}{0pt}%
\pgfpathmoveto{\pgfqpoint{3.606104in}{0.401000in}}%
\pgfpathquadraticcurveto{\pgfqpoint{3.606934in}{0.401000in}}{\pgfqpoint{3.600000in}{0.401000in}}%
\pgfusepath{stroke}%
\end{pgfscope}%
\begin{pgfscope}%
\pgfsetroundcap%
\pgfsetroundjoin%
\pgfsetlinewidth{0.501875pt}%
\definecolor{currentstroke}{rgb}{0.000000,0.000000,0.000000}%
\pgfsetstrokecolor{currentstroke}%
\pgfsetdash{}{0pt}%
\pgfpathmoveto{\pgfqpoint{3.550548in}{0.428778in}}%
\pgfpathlineto{\pgfqpoint{3.606104in}{0.401000in}}%
\pgfpathlineto{\pgfqpoint{3.550548in}{0.373222in}}%
\pgfusepath{stroke}%
\end{pgfscope}%
\begin{pgfscope}%
\pgftext[x=3.669444in,y=0.401000in,left,]{\rmfamily\fontsize{10.000000}{12.000000}\selectfont \(\displaystyle k_2 / k\)}%
\end{pgfscope}%
\end{pgfpicture}%
\makeatother%
\endgroup%

\caption{Der Träger von $\hat \psi_{ast}$ für verschiedene $a, s$. Man sieht gut,
wie $supp (\hat \psi_{ast})$ für kleinere $a$ in immer spitzeren Kegeln liegt.}
\label{fig:supp_psi_hat}
\end{figure}

\label{cor:psi_hat}
Im Fourierraum ist $\hat{\psi}_{ast}$ gegeben durch

\begin{equation}
    \hat \psi_{ast}{(\omega, k)} = a^{\frac{3}{4}}e^{-i\omega t' + ikx'}\hat\psi_1(a \omega) \hat\psi_{2}\left(a^{-\frac{1}{2}}\left(\frac{k}{\omega}-s\right)\right)
\label{eq:hat_psi_ast}
\end{equation}

und es gilt

\begin{equation}
\label{eq:supp_psi}
    supp(\hat \psi) \subset \left\{(\omega, k) \in  \hat{\mathbb{R}}^2 ~\Big| ~\omega \in \left[\frac{1}{2 a} , \frac{2}{a}\right], \left|\frac{k}{\omega} - s\right| \leq \sqrt{a} \right\}
\end{equation}
\end{remark}

Eine weitere Eigenschaft, die aus dieser Definition der Shearlets folgt, ist der schnelle Abfall der Shearlets Abseits von $(t',x')$

\begin{proposition}[$\psi_{ast}$ fällt schnell ab]
\label{prop:shearlets_decay_rapidly}
Sei $\psi \in L^2(\mathbb{R}^2)$ ein Shearlet wie in \cref{def:shearletttransformationI} und $M_{as}$ wie in \cref{eq:schermatrizen}. Dann gilt für alle $N \in  \mathbb{N}$, dass es eine konstante $C_N$ gibt s.d. für alle $(t,x) \in \mathbb{R}^2$ gilt

\begin{dmath*}
    \left| \psi_{ast}(t,x) \right|
    \leq
    C_k \left| \det M_{as} \right|^{-\frac{1}{2}}\left(1+\left|M_{as}^{-1}
    \left(
    \begin{smallmatrix}
        t-t' \\ x-x'
    \end{smallmatrix}\right)
    \right|^2\right)^{-N}
    = C_k a^{-\frac{3}{4}}\left(1+a^{-2}\left(t-t'\right)^2
        + 2 a^{-2}s\left(t-t'\right)\left(x-x'\right)
        + a^{-1}\left(1+a^{-1}s^2\right)\left(x-x'\right)^2
    \right)^{-N}
\end{dmath*}

Und insbesondere ist $C_N = \frac{15}{2}\frac{\sqrt{a} + s}{a^2}\left(\Vert \hat\psi \Vert_\infty + \Vert \Laplace^N \hat\psi \Vert_\infty\right)$

\end{proposition}


Wer bisher aufmerksam mitgelesen hat und sich \cref{fig:supp_psi_hat} genau angeschaut hat, wird bemerkt haben, dass $supp(\hat\psi_{ast})$ für alle $a,s$ quasi nur im Kegel positiver Energie und Masse (also $\omega \geq 0, \omega^2 \geq k^2$) liegt. Wie der folgende Satz zeigt, erzeugt $\psi$ auch nur für solche $f$ ein reproduzierendes System.
\todo[color=green]{Die konkrete Konstruktion für die volle Transformation angeben, oder reicht der Hinweis, dass ich mit Spiegeln und Drehen alles sehen kann?}

\begin{theorem}[$\psi$ reproduziert $L^2(C)^\vee$]
    Sei
    \begin{equation}
        C= \left\{(\omega,k)\in \hat{\mathbb{R}}^2
        \Big| \omega \geq 2 \textrm{ und } \left\lvert\tfrac{k}{\omega}\right\rvert \leq 1
        \right\}
    \end{equation}
    und
    \begin{equation}
        L^2(C)^\vee = \{f \in L^2(\mathbb{R}^2) ~|~ supp (\hat f) \subset C\}
    \end{equation}

    Dann ist $\psi$ aus \cref{def:shearletttransformationI} ein reproduzierendes System für $L^2(C)^\vee$, also für alle
    $f \in L^2(C)^\vee$
    \begin{equation}
        f(t,x) = \int_{\mathbb{R}^2} \int_{-2}^2 \int_0^1
                \left\langle\psi_{as(t',x')},f\right\rangle \psi_{as(t',x')}(t,x)
                \frac{\d a}{a^3} \d s \d t' \d x'
    \end{equation}
\end{theorem}

Um nun nicht nur ein reproduzierendes System für $L^2(C)^\vee$, sondern für ganz $L^2(\mathbb{R}^2)$ zu erhalten, muss $\hat \psi$ noch in den rechten, linken und rückwärts liegenden Kegel gedreht und geschoben werden. Zusätzlich muss noch eine weitere Funktion $W$ gefunden werden, welche die groben Skalen (also $|\omega|, |k| \leq 2$) auflöst. Die Konstruktion dazu findet sich in \textcite[pp. 7]{Kutyniok2008}. Hier verzichten wir aber darauf, da wir in den späteren Rechnungen nur Distributionen betrachten, die im der Frequenzdomäne im oberen Kegel liegen und die groben Skalen bei der Bestimmung der Wellenfrontmenge per Definition nicht interessant sind.

\todo{L2 ist Summer der vier Quadranten plus grobe Strukturen}

% Wer bisher aufmerksam mitgelesen hat und sich \cref{fig:supp_psi_hat} genau angeschaut hat, wird bemerkt haben, dass $supp(\hat\psi_{ast})$ für alle $a,s$ quasi nur im Quadranten positiver Energie und Masse (also $\omega \geq 0, \omega^2 \geq k^2$) liegt. Also wird $\mathcal{S}_f(a,s,t)$ auch nur für solche $f$ von Nutzen sein. Diesen Mangel wollen wir nun durch "`Drehen und Spiegeln"' von $\hat \psi$ und Einführen einer weiteren Analysefunktion $W$ für die groben Skalen beheben. Mehr der Vollständigkeit halber, als aus wirklicher Notwendigkeit, denn in den späteren Kapiteln betrachten wir doch nur Distributionen deren Träger komplett im Vorwärtsquadranten(?) liegt. Und die groben Skalen interessieren beim Bestimmen der Wellenfrontmenge ohnehin nicht.

% \begin{definition}[Shearlettransformation, Fortsetzung]
% Seien $\psi_1$ und $\psi_2$ wie in \cref{def:shearletttransformationI} und

% \begin{equation}
%     G^{(v)} = \left\{M_{as} \in GL(2,\mathbb{R}) \Big | M_{as} = \left(\begin{smallmatrix}
%         \sqrt a & 0 \\ -\sqrt a s &  a
%     \end{smallmatrix}\right)
%     , a \in [0,1], s \in [-2,2]
%     \right\}
% \end{equation}


% \end{definition}

% \begin{theorem}[$\{\psi(i), W\}$ sind ein reproduzierendes System]

% \end{theorem}

% section konstruktion_und_eigenschaften_der_shearlets (end)


Das große Versprechen der Shearlettransformation war ja, dass sie in der Lage ist nicht nur Position, sondern auch Orientierung der Singularitäten aufzulösen. Das dem auch sie ist, ist Aussage des folgenden Satzes:

\begin{theorem}[$\mathcal{S}_f(a,s,t)$ misst $WF(f)$]
\label{thm:main_theorem}
    Sei $f \in \mathcal{S}(C)^\vee$ (wobei $\mathcal{S}(C)^\vee$ analog zu $L^2(C)^\vee$ definiert ist).
    Sei $\mathcal{D}$ die Menge der $((x',t'), s)$ s.d. $\mathcal{S}(a,s,(t',x'))$ schnell verschwindet. Genauer

    \begin{dmath*}
        \mathcal{D} = \left\{
        ((t'_0,x'_0),s_0) \hiderel \in \mathbb{R}^2 \times [-1,1]~ \big|\\ ~\textrm{ für  $((t',x'),s)$ in einer Umgebung $U$ von } ((t'_0,x'_0),s_0) \hiderel :\\
        |S_f (a,s,(t',x'))| = O(a^k) \textrm{ für } a \hiderel \to 0, \forall k \hiderel \in \mathbb{N} \\ \textrm{ mit $O(\cdot)$ gleichmäßig über } ((t',x'),s) \hiderel \in U
        \right\}
    \end{dmath*}

    Dann gilt $WF(f)^c = \mathcal{D}$
\end{theorem}

Wenn wir die Wellenfrontmenge einer Distribution kennen, kenne wir trivialerweise auch ihren singulären Träger:

\begin{corollary}[$WF(f)$ misst $sing ~supp (\psi)$]
Sei $f$ wie eben und $\mathcal{R}$ die Projektion von $\mathcal{D}$ auf die erste Ortskomponente. Also

\begin{equation}
    \mathcal{R} = \pi (\mathcal{D})~~;~~~~
    \pi : ((t',x'),s) \mapsto (t',x')
\end{equation}

Dann gilt $sing ~supp (f)^c = \mathcal{R}$
\end{corollary}


Der Beweis für \cref{thm:main_theorem} findet sich in \textcite[S.19 ff]{Kutyniok2008} und soll hier nur plausibel gemacht werden.
\todo{Beweis noch besser verstehen und hier plausibel machen! Wichtig: Welche Zutaten spielen wie zusammen?}


\todo{Beweis ordentlich formatieren}
\begin{proof}[Beweis von \ref{thm:main_theorem}]
Zunächst die einfachere Richtung, nämlich $WF(f)^c \subseteq \mathcal{D}$.
Wir nehmen also einen gerichteten regulären Punkt $((t_0,x_0),s_0) \in WF(f)^c$ und zeigen, dass er auch in $\mathcal{D}$ liegt. Dazu zerlegen wir $f$ zunächst wie folgt:
 Da $f$ bei $(t_0, x_0)$ in Richtung $s_0$ regulär ist, gibt es per Definition der Wellenfrontmenge ein $\phi \in C_0^\infty(\mathcal{R}^2)$ s.d. $\phi = 1$ in einer Umgebung von $(t_0, x_0)$ und für alle $N \in \mathbb{N}$ $\rwhat{\phi f} = O(1+|(\omega,k)|)^{-N}$ für $\frac{k}{\omega}$ in einer Umgebung von $s_0$. Dementsprechend ist $(1-\phi)f = 0$ in einer Umgebung von $(t_0, x_0)$ und es gilt

 \begin{equation}
     \mathcal{S}_f (a,s,(t',x')) = \left\langle \psi_{ast},\phi f \right\rangle
                                + \left\langle \psi_{ast},(1-\phi) f \right\rangle
 \label{eq:schlaue sache}
 \end{equation}

\begin{figure}[h]
\centering
%% Creator: Matplotlib, PGF backend
%%
%% To include the figure in your LaTeX document, write
%%   \input{<filename>.pgf}
%%
%% Make sure the required packages are loaded in your preamble
%%   \usepackage{pgf}
%%
%% Figures using additional raster images can only be included by \input if
%% they are in the same directory as the main LaTeX file. For loading figures
%% from other directories you can use the `import` package
%%   \usepackage{import}
%% and then include the figures with
%%   \import{<path to file>}{<filename>.pgf}
%%
%% Matplotlib used the following preamble
%%   \usepackage[utf8x]{inputenc}
%%   \usepackage[T1]{fontenc}
%%   \usepackage{amssymb}
%%
\begingroup%
\makeatletter%
\begin{pgfpicture}%
\pgfpathrectangle{\pgfpointorigin}{\pgfqpoint{4.000000in}{2.200000in}}%
\pgfusepath{use as bounding box, clip}%
\begin{pgfscope}%
\pgfsetbuttcap%
\pgfsetmiterjoin%
\definecolor{currentfill}{rgb}{1.000000,1.000000,1.000000}%
\pgfsetfillcolor{currentfill}%
\pgfsetlinewidth{0.000000pt}%
\definecolor{currentstroke}{rgb}{1.000000,1.000000,1.000000}%
\pgfsetstrokecolor{currentstroke}%
\pgfsetdash{}{0pt}%
\pgfpathmoveto{\pgfqpoint{0.000000in}{0.000000in}}%
\pgfpathlineto{\pgfqpoint{4.000000in}{0.000000in}}%
\pgfpathlineto{\pgfqpoint{4.000000in}{2.200000in}}%
\pgfpathlineto{\pgfqpoint{0.000000in}{2.200000in}}%
\pgfpathclose%
\pgfusepath{fill}%
\end{pgfscope}%
\begin{pgfscope}%
\pgfsetbuttcap%
\pgfsetmiterjoin%
\definecolor{currentfill}{rgb}{1.000000,1.000000,1.000000}%
\pgfsetfillcolor{currentfill}%
\pgfsetlinewidth{0.000000pt}%
\definecolor{currentstroke}{rgb}{0.000000,0.000000,0.000000}%
\pgfsetstrokecolor{currentstroke}%
\pgfsetstrokeopacity{0.000000}%
\pgfsetdash{}{0pt}%
\pgfpathmoveto{\pgfqpoint{0.500000in}{0.275000in}}%
\pgfpathlineto{\pgfqpoint{3.600000in}{0.275000in}}%
\pgfpathlineto{\pgfqpoint{3.600000in}{1.936000in}}%
\pgfpathlineto{\pgfqpoint{0.500000in}{1.936000in}}%
\pgfpathclose%
\pgfusepath{fill}%
\end{pgfscope}%
\begin{pgfscope}%
\pgfsetbuttcap%
\pgfsetroundjoin%
\definecolor{currentfill}{rgb}{0.000000,0.000000,0.000000}%
\pgfsetfillcolor{currentfill}%
\pgfsetlinewidth{0.803000pt}%
\definecolor{currentstroke}{rgb}{0.000000,0.000000,0.000000}%
\pgfsetstrokecolor{currentstroke}%
\pgfsetdash{}{0pt}%
\pgfsys@defobject{currentmarker}{\pgfqpoint{0.000000in}{-0.048611in}}{\pgfqpoint{0.000000in}{0.000000in}}{%
\pgfpathmoveto{\pgfqpoint{0.000000in}{0.000000in}}%
\pgfpathlineto{\pgfqpoint{0.000000in}{-0.048611in}}%
\pgfusepath{stroke,fill}%
}%
\begin{pgfscope}%
\pgfsys@transformshift{2.050000in}{0.344208in}%
\pgfsys@useobject{currentmarker}{}%
\end{pgfscope}%
\end{pgfscope}%
\begin{pgfscope}%
\pgftext[x=2.050000in,y=0.246986in,,top]{\rmfamily\fontsize{10.000000}{12.000000}\selectfont \(\displaystyle (t_0, x_0)\)}%
\end{pgfscope}%
\begin{pgfscope}%
\pgfpathrectangle{\pgfqpoint{0.500000in}{0.275000in}}{\pgfqpoint{3.100000in}{1.661000in}}%
\pgfusepath{clip}%
\pgfsetrectcap%
\pgfsetroundjoin%
\pgfsetlinewidth{0.501875pt}%
\definecolor{currentstroke}{rgb}{0.894118,0.101961,0.109804}%
\pgfsetstrokecolor{currentstroke}%
\pgfsetdash{}{0pt}%
\pgfpathmoveto{\pgfqpoint{0.500000in}{0.344208in}}%
\pgfpathlineto{\pgfqpoint{1.778478in}{0.345597in}}%
\pgfpathlineto{\pgfqpoint{1.803303in}{0.348841in}}%
\pgfpathlineto{\pgfqpoint{1.818819in}{0.353486in}}%
\pgfpathlineto{\pgfqpoint{1.831231in}{0.359861in}}%
\pgfpathlineto{\pgfqpoint{1.843644in}{0.369866in}}%
\pgfpathlineto{\pgfqpoint{1.856056in}{0.385070in}}%
\pgfpathlineto{\pgfqpoint{1.865365in}{0.401049in}}%
\pgfpathlineto{\pgfqpoint{1.874675in}{0.422003in}}%
\pgfpathlineto{\pgfqpoint{1.883984in}{0.448968in}}%
\pgfpathlineto{\pgfqpoint{1.896396in}{0.496107in}}%
\pgfpathlineto{\pgfqpoint{1.908809in}{0.558191in}}%
\pgfpathlineto{\pgfqpoint{1.921221in}{0.637077in}}%
\pgfpathlineto{\pgfqpoint{1.936737in}{0.760517in}}%
\pgfpathlineto{\pgfqpoint{1.952252in}{0.909895in}}%
\pgfpathlineto{\pgfqpoint{1.977077in}{1.185411in}}%
\pgfpathlineto{\pgfqpoint{2.005005in}{1.489311in}}%
\pgfpathlineto{\pgfqpoint{2.017417in}{1.597376in}}%
\pgfpathlineto{\pgfqpoint{2.026727in}{1.659912in}}%
\pgfpathlineto{\pgfqpoint{2.036036in}{1.703327in}}%
\pgfpathlineto{\pgfqpoint{2.042242in}{1.720595in}}%
\pgfpathlineto{\pgfqpoint{2.048448in}{1.728063in}}%
\pgfpathlineto{\pgfqpoint{2.051552in}{1.728063in}}%
\pgfpathlineto{\pgfqpoint{2.054655in}{1.725569in}}%
\pgfpathlineto{\pgfqpoint{2.060861in}{1.713168in}}%
\pgfpathlineto{\pgfqpoint{2.067067in}{1.691126in}}%
\pgfpathlineto{\pgfqpoint{2.076376in}{1.641065in}}%
\pgfpathlineto{\pgfqpoint{2.085686in}{1.572759in}}%
\pgfpathlineto{\pgfqpoint{2.098098in}{1.458747in}}%
\pgfpathlineto{\pgfqpoint{2.119820in}{1.221041in}}%
\pgfpathlineto{\pgfqpoint{2.150851in}{0.878170in}}%
\pgfpathlineto{\pgfqpoint{2.169469in}{0.707845in}}%
\pgfpathlineto{\pgfqpoint{2.184985in}{0.595451in}}%
\pgfpathlineto{\pgfqpoint{2.197397in}{0.525148in}}%
\pgfpathlineto{\pgfqpoint{2.209810in}{0.470810in}}%
\pgfpathlineto{\pgfqpoint{2.222222in}{0.430271in}}%
\pgfpathlineto{\pgfqpoint{2.234635in}{0.401049in}}%
\pgfpathlineto{\pgfqpoint{2.247047in}{0.380681in}}%
\pgfpathlineto{\pgfqpoint{2.259459in}{0.366946in}}%
\pgfpathlineto{\pgfqpoint{2.271872in}{0.357980in}}%
\pgfpathlineto{\pgfqpoint{2.287387in}{0.351274in}}%
\pgfpathlineto{\pgfqpoint{2.306006in}{0.347197in}}%
\pgfpathlineto{\pgfqpoint{2.333934in}{0.344936in}}%
\pgfpathlineto{\pgfqpoint{2.395996in}{0.344227in}}%
\pgfpathlineto{\pgfqpoint{3.600000in}{0.344208in}}%
\pgfpathlineto{\pgfqpoint{3.600000in}{0.344208in}}%
\pgfusepath{stroke}%
\end{pgfscope}%
\begin{pgfscope}%
\pgfpathrectangle{\pgfqpoint{0.500000in}{0.275000in}}{\pgfqpoint{3.100000in}{1.661000in}}%
\pgfusepath{clip}%
\pgfsetrectcap%
\pgfsetroundjoin%
\pgfsetlinewidth{0.501875pt}%
\definecolor{currentstroke}{rgb}{0.215686,0.494118,0.721569}%
\pgfsetstrokecolor{currentstroke}%
\pgfsetdash{}{0pt}%
\pgfpathmoveto{\pgfqpoint{0.500000in}{0.344208in}}%
\pgfpathlineto{\pgfqpoint{1.744344in}{0.344919in}}%
\pgfpathlineto{\pgfqpoint{1.750551in}{0.350590in}}%
\pgfpathlineto{\pgfqpoint{1.756757in}{0.361837in}}%
\pgfpathlineto{\pgfqpoint{1.766066in}{0.388737in}}%
\pgfpathlineto{\pgfqpoint{1.775375in}{0.426758in}}%
\pgfpathlineto{\pgfqpoint{1.787788in}{0.492274in}}%
\pgfpathlineto{\pgfqpoint{1.803303in}{0.591792in}}%
\pgfpathlineto{\pgfqpoint{1.849850in}{0.906003in}}%
\pgfpathlineto{\pgfqpoint{1.862262in}{0.967573in}}%
\pgfpathlineto{\pgfqpoint{1.871572in}{1.002023in}}%
\pgfpathlineto{\pgfqpoint{1.880881in}{1.024974in}}%
\pgfpathlineto{\pgfqpoint{1.887087in}{1.033451in}}%
\pgfpathlineto{\pgfqpoint{1.893293in}{1.036292in}}%
\pgfpathlineto{\pgfqpoint{2.209810in}{1.035581in}}%
\pgfpathlineto{\pgfqpoint{2.216016in}{1.029910in}}%
\pgfpathlineto{\pgfqpoint{2.222222in}{1.018663in}}%
\pgfpathlineto{\pgfqpoint{2.231532in}{0.991763in}}%
\pgfpathlineto{\pgfqpoint{2.240841in}{0.953742in}}%
\pgfpathlineto{\pgfqpoint{2.253253in}{0.888226in}}%
\pgfpathlineto{\pgfqpoint{2.268769in}{0.788708in}}%
\pgfpathlineto{\pgfqpoint{2.315315in}{0.474497in}}%
\pgfpathlineto{\pgfqpoint{2.327728in}{0.412927in}}%
\pgfpathlineto{\pgfqpoint{2.337037in}{0.378477in}}%
\pgfpathlineto{\pgfqpoint{2.346346in}{0.355526in}}%
\pgfpathlineto{\pgfqpoint{2.352553in}{0.347049in}}%
\pgfpathlineto{\pgfqpoint{2.358759in}{0.344208in}}%
\pgfpathlineto{\pgfqpoint{3.600000in}{0.344208in}}%
\pgfpathlineto{\pgfqpoint{3.600000in}{0.344208in}}%
\pgfusepath{stroke}%
\end{pgfscope}%
\begin{pgfscope}%
\pgfpathrectangle{\pgfqpoint{0.500000in}{0.275000in}}{\pgfqpoint{3.100000in}{1.661000in}}%
\pgfusepath{clip}%
\pgfsetrectcap%
\pgfsetroundjoin%
\pgfsetlinewidth{0.501875pt}%
\definecolor{currentstroke}{rgb}{0.301961,0.686275,0.290196}%
\pgfsetstrokecolor{currentstroke}%
\pgfsetdash{}{0pt}%
\pgfpathmoveto{\pgfqpoint{0.500000in}{1.036292in}}%
\pgfpathlineto{\pgfqpoint{1.744344in}{1.035581in}}%
\pgfpathlineto{\pgfqpoint{1.750551in}{1.029910in}}%
\pgfpathlineto{\pgfqpoint{1.756757in}{1.018663in}}%
\pgfpathlineto{\pgfqpoint{1.766066in}{0.991763in}}%
\pgfpathlineto{\pgfqpoint{1.775375in}{0.953742in}}%
\pgfpathlineto{\pgfqpoint{1.787788in}{0.888226in}}%
\pgfpathlineto{\pgfqpoint{1.803303in}{0.788708in}}%
\pgfpathlineto{\pgfqpoint{1.849850in}{0.474497in}}%
\pgfpathlineto{\pgfqpoint{1.862262in}{0.412927in}}%
\pgfpathlineto{\pgfqpoint{1.871572in}{0.378477in}}%
\pgfpathlineto{\pgfqpoint{1.880881in}{0.355526in}}%
\pgfpathlineto{\pgfqpoint{1.887087in}{0.347049in}}%
\pgfpathlineto{\pgfqpoint{1.893293in}{0.344208in}}%
\pgfpathlineto{\pgfqpoint{2.209810in}{0.344919in}}%
\pgfpathlineto{\pgfqpoint{2.216016in}{0.350590in}}%
\pgfpathlineto{\pgfqpoint{2.222222in}{0.361837in}}%
\pgfpathlineto{\pgfqpoint{2.231532in}{0.388737in}}%
\pgfpathlineto{\pgfqpoint{2.240841in}{0.426758in}}%
\pgfpathlineto{\pgfqpoint{2.253253in}{0.492274in}}%
\pgfpathlineto{\pgfqpoint{2.268769in}{0.591792in}}%
\pgfpathlineto{\pgfqpoint{2.315315in}{0.906003in}}%
\pgfpathlineto{\pgfqpoint{2.327728in}{0.967573in}}%
\pgfpathlineto{\pgfqpoint{2.337037in}{1.002023in}}%
\pgfpathlineto{\pgfqpoint{2.346346in}{1.024974in}}%
\pgfpathlineto{\pgfqpoint{2.352553in}{1.033451in}}%
\pgfpathlineto{\pgfqpoint{2.358759in}{1.036292in}}%
\pgfpathlineto{\pgfqpoint{3.600000in}{1.036292in}}%
\pgfpathlineto{\pgfqpoint{3.600000in}{1.036292in}}%
\pgfusepath{stroke}%
\end{pgfscope}%
\begin{pgfscope}%
\pgfsetrectcap%
\pgfsetmiterjoin%
\pgfsetlinewidth{0.501875pt}%
\definecolor{currentstroke}{rgb}{0.000000,0.000000,0.000000}%
\pgfsetstrokecolor{currentstroke}%
\pgfsetdash{}{0pt}%
\pgfpathmoveto{\pgfqpoint{2.050000in}{0.275000in}}%
\pgfpathlineto{\pgfqpoint{2.050000in}{1.936000in}}%
\pgfusepath{stroke}%
\end{pgfscope}%
\begin{pgfscope}%
\pgfsetrectcap%
\pgfsetmiterjoin%
\pgfsetlinewidth{0.501875pt}%
\definecolor{currentstroke}{rgb}{0.000000,0.000000,0.000000}%
\pgfsetstrokecolor{currentstroke}%
\pgfsetdash{}{0pt}%
\pgfpathmoveto{\pgfqpoint{0.500000in}{0.344208in}}%
\pgfpathlineto{\pgfqpoint{3.600000in}{0.344208in}}%
\pgfusepath{stroke}%
\end{pgfscope}%
\begin{pgfscope}%
\pgfsetroundcap%
\pgfsetroundjoin%
\pgfsetlinewidth{0.501875pt}%
\definecolor{currentstroke}{rgb}{0.000000,0.000000,0.000000}%
\pgfsetstrokecolor{currentstroke}%
\pgfsetdash{}{0pt}%
\pgfpathmoveto{\pgfqpoint{2.050000in}{1.942121in}}%
\pgfpathquadraticcurveto{\pgfqpoint{2.050000in}{1.942943in}}{\pgfqpoint{2.050000in}{1.936000in}}%
\pgfusepath{stroke}%
\end{pgfscope}%
\begin{pgfscope}%
\pgfsetroundcap%
\pgfsetroundjoin%
\pgfsetlinewidth{0.501875pt}%
\definecolor{currentstroke}{rgb}{0.000000,0.000000,0.000000}%
\pgfsetstrokecolor{currentstroke}%
\pgfsetdash{}{0pt}%
\pgfpathmoveto{\pgfqpoint{2.022222in}{1.886565in}}%
\pgfpathlineto{\pgfqpoint{2.050000in}{1.942121in}}%
\pgfpathlineto{\pgfqpoint{2.077778in}{1.886565in}}%
\pgfusepath{stroke}%
\end{pgfscope}%
\begin{pgfscope}%
\pgftext[x=2.050000in,y=2.005444in,,bottom]{\rmfamily\fontsize{10.000000}{12.000000}\selectfont \(\displaystyle  f\)}%
\end{pgfscope}%
\begin{pgfscope}%
\pgfsetroundcap%
\pgfsetroundjoin%
\pgfsetlinewidth{0.501875pt}%
\definecolor{currentstroke}{rgb}{0.000000,0.000000,0.000000}%
\pgfsetstrokecolor{currentstroke}%
\pgfsetdash{}{0pt}%
\pgfpathmoveto{\pgfqpoint{3.606111in}{0.344208in}}%
\pgfpathquadraticcurveto{\pgfqpoint{3.606938in}{0.344208in}}{\pgfqpoint{3.600000in}{0.344208in}}%
\pgfusepath{stroke}%
\end{pgfscope}%
\begin{pgfscope}%
\pgfsetroundcap%
\pgfsetroundjoin%
\pgfsetlinewidth{0.501875pt}%
\definecolor{currentstroke}{rgb}{0.000000,0.000000,0.000000}%
\pgfsetstrokecolor{currentstroke}%
\pgfsetdash{}{0pt}%
\pgfpathmoveto{\pgfqpoint{3.550556in}{0.371986in}}%
\pgfpathlineto{\pgfqpoint{3.606111in}{0.344208in}}%
\pgfpathlineto{\pgfqpoint{3.550556in}{0.316431in}}%
\pgfusepath{stroke}%
\end{pgfscope}%
\begin{pgfscope}%
\pgftext[x=3.669444in,y=0.344208in,left,]{\rmfamily\fontsize{10.000000}{12.000000}\selectfont \(\displaystyle (x,t)\)}%
\end{pgfscope}%
\begin{pgfscope}%
\pgfsetbuttcap%
\pgfsetmiterjoin%
\definecolor{currentfill}{rgb}{1.000000,1.000000,1.000000}%
\pgfsetfillcolor{currentfill}%
\pgfsetfillopacity{0.800000}%
\pgfsetlinewidth{0.501875pt}%
\definecolor{currentstroke}{rgb}{0.800000,0.800000,0.800000}%
\pgfsetstrokecolor{currentstroke}%
\pgfsetstrokeopacity{0.800000}%
\pgfsetdash{}{0pt}%
\pgfpathmoveto{\pgfqpoint{2.736382in}{1.243871in}}%
\pgfpathlineto{\pgfqpoint{3.502778in}{1.243871in}}%
\pgfpathquadraticcurveto{\pgfqpoint{3.530556in}{1.243871in}}{\pgfqpoint{3.530556in}{1.271648in}}%
\pgfpathlineto{\pgfqpoint{3.530556in}{1.838778in}}%
\pgfpathquadraticcurveto{\pgfqpoint{3.530556in}{1.866556in}}{\pgfqpoint{3.502778in}{1.866556in}}%
\pgfpathlineto{\pgfqpoint{2.736382in}{1.866556in}}%
\pgfpathquadraticcurveto{\pgfqpoint{2.708604in}{1.866556in}}{\pgfqpoint{2.708604in}{1.838778in}}%
\pgfpathlineto{\pgfqpoint{2.708604in}{1.271648in}}%
\pgfpathquadraticcurveto{\pgfqpoint{2.708604in}{1.243871in}}{\pgfqpoint{2.736382in}{1.243871in}}%
\pgfpathclose%
\pgfusepath{stroke,fill}%
\end{pgfscope}%
\begin{pgfscope}%
\pgfsetrectcap%
\pgfsetroundjoin%
\pgfsetlinewidth{0.501875pt}%
\definecolor{currentstroke}{rgb}{0.894118,0.101961,0.109804}%
\pgfsetstrokecolor{currentstroke}%
\pgfsetdash{}{0pt}%
\pgfpathmoveto{\pgfqpoint{2.764160in}{1.762389in}}%
\pgfpathlineto{\pgfqpoint{3.041937in}{1.762389in}}%
\pgfusepath{stroke}%
\end{pgfscope}%
\begin{pgfscope}%
\pgftext[x=3.153049in,y=1.713778in,left,base]{\rmfamily\fontsize{10.000000}{12.000000}\selectfont \(\displaystyle \psi\)}%
\end{pgfscope}%
\begin{pgfscope}%
\pgfsetrectcap%
\pgfsetroundjoin%
\pgfsetlinewidth{0.501875pt}%
\definecolor{currentstroke}{rgb}{0.215686,0.494118,0.721569}%
\pgfsetstrokecolor{currentstroke}%
\pgfsetdash{}{0pt}%
\pgfpathmoveto{\pgfqpoint{2.764160in}{1.568716in}}%
\pgfpathlineto{\pgfqpoint{3.041937in}{1.568716in}}%
\pgfusepath{stroke}%
\end{pgfscope}%
\begin{pgfscope}%
\pgftext[x=3.153049in,y=1.520105in,left,base]{\rmfamily\fontsize{10.000000}{12.000000}\selectfont \(\displaystyle \phi\)}%
\end{pgfscope}%
\begin{pgfscope}%
\pgfsetrectcap%
\pgfsetroundjoin%
\pgfsetlinewidth{0.501875pt}%
\definecolor{currentstroke}{rgb}{0.301961,0.686275,0.290196}%
\pgfsetstrokecolor{currentstroke}%
\pgfsetdash{}{0pt}%
\pgfpathmoveto{\pgfqpoint{2.764160in}{1.375043in}}%
\pgfpathlineto{\pgfqpoint{3.041937in}{1.375043in}}%
\pgfusepath{stroke}%
\end{pgfscope}%
\begin{pgfscope}%
\pgftext[x=3.153049in,y=1.326432in,left,base]{\rmfamily\fontsize{10.000000}{12.000000}\selectfont \(\displaystyle 1-\phi\)}%
\end{pgfscope}%
\end{pgfpicture}%
\makeatother%
\endgroup%

\caption{Die Zerlegung von $f$ um $(t_0,x_0)$ herum visualisiert}
\label{fig:smart_decomposition}
\end{figure}

Da $(1-\phi)f$ in einer Umgebung von $(t_0, x_0)$ verschwindet und nach \cref{prop:shearlets_decay_rapidly} Shearlets außerhalb von $(t',x')$ schnell abfallen für $a \to 0$ fällt auch der zweite Term von \cref{eq:schlaue sache}
für $(t',x') = (t_0,x_0)$ schnell ab. Für den ersten Term überzeugen wir uns anhand von \cref{fig:supp_psi_hat,eq:supp_psi}, dass für $a$ klein genug $supp(\hat\psi_{ast})$ schließlich in jedem noch so kleinen Kegel um $s$ liegt. In einem solchen um $s_0$ fällt aber $\rwhat{\phi f}$ rapide ab nach Vorraussetzung und damit auch der erste Term in \cref{eq:schlaue sache}.

Die beiden entscheidenden Zutaten waren hier also die Tatsache, dass die Shearlets außerhalb von $(t',x')$ rapide Abfallen und damit bei immer feineren Skalen $a$ immer besser lokalisiert werden sowie die Tatsache, dass für $a \to 0$ der Träger im Frequenzbereich in immer engeren Kegeln liegt.

Deutlich schwieriger ist die umgekehrte Inklusion, nämlich dass die Shearlettransformation tatsächlich die ganze Wellenfrontmenge erkennt. Hier geht jetzt auch die Reproduktionseigenschaft der Transformation ein, eben genau dass sie alles sieht.

\end{proof}

% section allgemeines_gelaber_über_shearlets (end)


%%%%%%%%%%%%%%%%%%%%%%%%%%%%%%%%%%%%%%%%%%%%%%%%%%%%%%%%%%%%%%%%%%%%%%%%%%%%%%%%
% % Section 2
%%%%%%%%%%%%%%%%%%%%%%%%%%%%%%%%%%%%%%%%%%%%%%%%%%%%%%%%%%%%%%%%%%%%%%%%%%%%%%%%
\section{\texorpdfstring{Zwei nützliche Substitionen für  $\left<\psi_{ast}, f\right>$}{zwei nützliche Substitutionen}}
\label{sec:substitutionen}

\todo[color=green]{mit $(\omega, k)$ als Variablennamen arbeiten, um zum Rest des Textes zu passen, oder mit $(\xi_1, \xi_2)$ um zu \textcite{Kutyniok2008} zu passen?}

Zunächst werden wir zwei verschiedene Ausdrücke für $\left<\psi_{ast}, f\right>$
im Fourierraum herleiten, welche fast immer Ausgangspunkt für unsere Abschätzungen sein werden.

Sei also $\psi$ ein Shearlet wie in \cref{cor:psi_hat}. Sei $f$ die zu
analysierende fouriertransformierbare Funktion (oder Distribution) in
$\mathcal{D}' (\mathbb{R}^2)$. Dann ist $\mathcal{S}_f (ast)$ gegeben durch

\begin{align*}
\left< \psi_{ast}, f \right> &= \left<\hat\psi_{ast}, \hat f\right> \\
 &= \int a^{\frac{3}{4}} e^{-i \xi \cdot t} \hat \psi_1(a \xi_1)
    \hat \psi_2 \left(a^{-\frac{1}{2}} \left(\frac{\xi_2}{\xi_1} - s\right)\right)
    \hat f (\xi) \d \xi
\end{align*}

\todo{entscheiden, was mit dem fehlenden Faktor $\frac{1}{(2 \pi)^n}$ geschieht}
und nach "`entscheren"' und "`deskalieren"', also der Substitution

\begin{equation}
\begin{aligned}[c]
a \xi_1 &= k_1\\
a^{-\frac{1}{2}} \left(\frac{\xi_2}{\xi_1} - s\right) &=\frac{k_2}{k_1}\\
\end{aligned}
\qquad\Longleftrightarrow\qquad
\begin{aligned}[c]
\xi_1 &= \frac{k_1}{a}\\
\xi_2 &= \frac{k_1 s}{a} + a^{-\frac{1}{2}} k_2\\
\end{aligned}
\label{eq:substitution1_coords}
\end{equation}

\begin{equation*}
\Rightarrow
\d \xi_1 \d \xi_2 = a^{-\frac{3}{2}} \d k_1 \d k_2
\end{equation*}

ergibt sich folgendes für $\left<\psi_{ast}, f\right>$:

\todo{\texttt{owntag} fixen}

\begin{align}
    \left\langle\psi_{ast},f\right\rangle
    &=  \left\langle\hat\psi_{ast},\hat f\right\rangle \nonumber \\
    &=  \iint a^{-\frac{3}{4}}~\hat \psi_1(k_1) ~\hat \psi_2 \left(\tfrac{k_2}{k_1}\right)
    ~\hat f \left(\tfrac{k_1}{a}, \tfrac{k_1 s}{a} + \tfrac{k_2}{\sqrt{a}}\right)
    ~e^{-i\frac{k_1}{a}(t_1+t_2 s) - i \frac{k_2 t_2}{\sqrt a}}
    \d k_1 \d k_2
\owntag[substitution1]{Substitution 1}
\end{align}

Wie man sieht, tauchen in den Argumente von $\hat\psi_1$ und $\hat\psi_2$ nun die Parameter $a,s,t$ gar nicht mehr auf, und wir können nun verwenden, was wir aus \ref{sec:shearlets} über deren Träger wissen.
Alternativ und mit ähnlichem Ergebniss kann auch folgende Substitution

\begin{equation}
\begin{aligned}[c]
a \xi_1 &= k_1\\
a^{-\frac{1}{2}} \left(\frac{\xi_2}{\xi_1} - s\right) &= k_2\\
\end{aligned}
\qquad\Longleftrightarrow\qquad
\begin{aligned}[c]
\xi_1 &= \frac{k_1}{a}\\
\xi_2 &= \left( a^{\frac{1}{2}} k_2 +s \right) \frac{k_1}{a}\\
\end{aligned}
\label{eq:substitution2_coords}
\end{equation}

\begin{equation*}
\Rightarrow
\d \xi_1 \d \xi_2 = a^{-\frac{3}{2}} k_1 \d k_1 \d k_2
\end{equation*}

gewählt werden, wodurch wieder alle Parameter $(a,s,t)$ aus den Argumenten von $\hat\psi_1, \hat\psi_2$
verschwinden und sich

\begin{align}
    \left<\psi_{ast},f\right>
    =  \iint a^{-\frac{3}{4}}~ k_1~ \hat \psi_1(k_1)~ \hat \psi_2 (k_2)~
    \hat f \left(\tfrac{k_1}{a}, k_1 \left(a^{-\frac{1}{2}}k_2 + s a^{-1}\right)\right)
    ~e^{-i k_1 \left(\frac{t_1+s t_2}{a} + \frac{k_2 t_2}{\sqrt{a}}\right)}
    \d k_1 \d k_2
\owntag[substitution2]{Substitution 2}
\end{align}

ergibt. Dabei ist zu beachten, dass diese Substitution zulässig ist, obwohl sie
die Orientierung \emph{nicht} erhält und \emph{keine} Bijektion ist. Aber
der kritische Bereich, nämlich $\xi_1 = 0$, liegt nicht im Träger von $\rwhat{\psi}$.

Beiden Substitution gemein ist aber, dass danach
$0=\omega \notin supp (\hat\psi)$ und dass $supp (\psi)$ sowohl in $k$ als auch in $\omega$ beschränkt ist. $\omega$ kann also sowohl nach unten als auch nach oben durch eine Konstante abgeschätzt werden, wannimmer dies der Sache dienlich ist. Auch $k$ kann zumindest nach oben immer durche eine Konstante abgeschätzt werden.

\begin{figure}[h]
    \centering
    \begin{minipage}{0.5\textwidth}
        \centering
        \resizebox{\textwidth}{!}{%% Creator: Matplotlib, PGF backend
%%
%% To include the figure in your LaTeX document, write
%%   \input{<filename>.pgf}
%%
%% Make sure the required packages are loaded in your preamble
%%   \usepackage{pgf}
%%
%% Figures using additional raster images can only be included by \input if
%% they are in the same directory as the main LaTeX file. For loading figures
%% from other directories you can use the `import` package
%%   \usepackage{import}
%% and then include the figures with
%%   \import{<path to file>}{<filename>.pgf}
%%
%% Matplotlib used the following preamble
%%   \usepackage[utf8x]{inputenc}
%%   \usepackage[T1]{fontenc}
%%   \usepackage{amssymb}
%%
\begingroup%
\makeatletter%
\begin{pgfpicture}%
\pgfpathrectangle{\pgfpointorigin}{\pgfqpoint{4.000000in}{2.800000in}}%
\pgfusepath{use as bounding box, clip}%
\begin{pgfscope}%
\pgfsetbuttcap%
\pgfsetmiterjoin%
\definecolor{currentfill}{rgb}{1.000000,1.000000,1.000000}%
\pgfsetfillcolor{currentfill}%
\pgfsetlinewidth{0.000000pt}%
\definecolor{currentstroke}{rgb}{1.000000,1.000000,1.000000}%
\pgfsetstrokecolor{currentstroke}%
\pgfsetdash{}{0pt}%
\pgfpathmoveto{\pgfqpoint{0.000000in}{0.000000in}}%
\pgfpathlineto{\pgfqpoint{4.000000in}{0.000000in}}%
\pgfpathlineto{\pgfqpoint{4.000000in}{2.800000in}}%
\pgfpathlineto{\pgfqpoint{0.000000in}{2.800000in}}%
\pgfpathclose%
\pgfusepath{fill}%
\end{pgfscope}%
\begin{pgfscope}%
\pgfsetbuttcap%
\pgfsetmiterjoin%
\definecolor{currentfill}{rgb}{1.000000,1.000000,1.000000}%
\pgfsetfillcolor{currentfill}%
\pgfsetlinewidth{0.000000pt}%
\definecolor{currentstroke}{rgb}{0.000000,0.000000,0.000000}%
\pgfsetstrokecolor{currentstroke}%
\pgfsetstrokeopacity{0.000000}%
\pgfsetdash{}{0pt}%
\pgfpathmoveto{\pgfqpoint{0.198611in}{0.198611in}}%
\pgfpathlineto{\pgfqpoint{3.801389in}{0.198611in}}%
\pgfpathlineto{\pgfqpoint{3.801389in}{2.601389in}}%
\pgfpathlineto{\pgfqpoint{0.198611in}{2.601389in}}%
\pgfpathclose%
\pgfusepath{fill}%
\end{pgfscope}%
\begin{pgfscope}%
\pgfpathrectangle{\pgfqpoint{0.198611in}{0.198611in}}{\pgfqpoint{3.602778in}{2.402778in}}%
\pgfusepath{clip}%
\pgfsetbuttcap%
\pgfsetmiterjoin%
\definecolor{currentfill}{rgb}{0.500000,0.500000,0.500000}%
\pgfsetfillcolor{currentfill}%
\pgfsetfillopacity{0.500000}%
\pgfsetlinewidth{0.501875pt}%
\definecolor{currentstroke}{rgb}{0.000000,0.000000,0.000000}%
\pgfsetstrokecolor{currentstroke}%
\pgfsetdash{}{0pt}%
\pgfpathmoveto{\pgfqpoint{1.963972in}{1.433372in}}%
\pgfpathlineto{\pgfqpoint{2.036028in}{1.433372in}}%
\pgfpathlineto{\pgfqpoint{2.144111in}{1.533488in}}%
\pgfpathlineto{\pgfqpoint{1.855889in}{1.533488in}}%
\pgfpathclose%
\pgfusepath{stroke,fill}%
\end{pgfscope}%
\begin{pgfscope}%
\pgfpathrectangle{\pgfqpoint{0.198611in}{0.198611in}}{\pgfqpoint{3.602778in}{2.402778in}}%
\pgfusepath{clip}%
\pgfsetbuttcap%
\pgfsetmiterjoin%
\definecolor{currentfill}{rgb}{0.500000,0.500000,0.500000}%
\pgfsetfillcolor{currentfill}%
\pgfsetfillopacity{0.500000}%
\pgfsetlinewidth{0.501875pt}%
\definecolor{currentstroke}{rgb}{0.000000,0.000000,0.000000}%
\pgfsetstrokecolor{currentstroke}%
\pgfsetdash{}{0pt}%
\pgfpathmoveto{\pgfqpoint{2.036028in}{1.366628in}}%
\pgfpathlineto{\pgfqpoint{1.963972in}{1.366628in}}%
\pgfpathlineto{\pgfqpoint{1.855889in}{1.266512in}}%
\pgfpathlineto{\pgfqpoint{2.144111in}{1.266512in}}%
\pgfpathclose%
\pgfusepath{stroke,fill}%
\end{pgfscope}%
\begin{pgfscope}%
\pgfpathrectangle{\pgfqpoint{0.198611in}{0.198611in}}{\pgfqpoint{3.602778in}{2.402778in}}%
\pgfusepath{clip}%
\pgfsetbuttcap%
\pgfsetmiterjoin%
\definecolor{currentfill}{rgb}{0.500000,0.500000,0.500000}%
\pgfsetfillcolor{currentfill}%
\pgfsetfillopacity{0.500000}%
\pgfsetlinewidth{0.501875pt}%
\definecolor{currentstroke}{rgb}{0.000000,0.000000,0.000000}%
\pgfsetstrokecolor{currentstroke}%
\pgfsetdash{}{0pt}%
\pgfpathmoveto{\pgfqpoint{2.246348in}{1.733719in}}%
\pgfpathlineto{\pgfqpoint{2.474208in}{1.733719in}}%
\pgfpathlineto{\pgfqpoint{3.896830in}{2.734877in}}%
\pgfpathlineto{\pgfqpoint{2.985392in}{2.734877in}}%
\pgfpathclose%
\pgfusepath{stroke,fill}%
\end{pgfscope}%
\begin{pgfscope}%
\pgfpathrectangle{\pgfqpoint{0.198611in}{0.198611in}}{\pgfqpoint{3.602778in}{2.402778in}}%
\pgfusepath{clip}%
\pgfsetbuttcap%
\pgfsetmiterjoin%
\definecolor{currentfill}{rgb}{0.500000,0.500000,0.500000}%
\pgfsetfillcolor{currentfill}%
\pgfsetfillopacity{0.500000}%
\pgfsetlinewidth{0.501875pt}%
\definecolor{currentstroke}{rgb}{0.000000,0.000000,0.000000}%
\pgfsetstrokecolor{currentstroke}%
\pgfsetdash{}{0pt}%
\pgfpathmoveto{\pgfqpoint{1.753652in}{1.066281in}}%
\pgfpathlineto{\pgfqpoint{1.525792in}{1.066281in}}%
\pgfpathlineto{\pgfqpoint{0.103170in}{0.065123in}}%
\pgfpathlineto{\pgfqpoint{1.014608in}{0.065123in}}%
\pgfpathclose%
\pgfusepath{stroke,fill}%
\end{pgfscope}%
\begin{pgfscope}%
\pgfpathrectangle{\pgfqpoint{0.198611in}{0.198611in}}{\pgfqpoint{3.602778in}{2.402778in}}%
\pgfusepath{clip}%
\pgfsetbuttcap%
\pgfsetroundjoin%
\pgfsetlinewidth{0.501875pt}%
\definecolor{currentstroke}{rgb}{0.501961,0.501961,0.501961}%
\pgfsetstrokecolor{currentstroke}%
\pgfsetdash{{1.850000pt}{0.800000pt}}{0.000000pt}%
\pgfpathmoveto{\pgfqpoint{0.688006in}{0.184722in}}%
\pgfpathlineto{\pgfqpoint{3.311994in}{2.615278in}}%
\pgfpathlineto{\pgfqpoint{3.311994in}{2.615278in}}%
\pgfusepath{stroke}%
\end{pgfscope}%
\begin{pgfscope}%
\pgfpathrectangle{\pgfqpoint{0.198611in}{0.198611in}}{\pgfqpoint{3.602778in}{2.402778in}}%
\pgfusepath{clip}%
\pgfsetbuttcap%
\pgfsetroundjoin%
\pgfsetlinewidth{0.501875pt}%
\definecolor{currentstroke}{rgb}{0.501961,0.501961,0.501961}%
\pgfsetstrokecolor{currentstroke}%
\pgfsetdash{{1.850000pt}{0.800000pt}}{0.000000pt}%
\pgfpathmoveto{\pgfqpoint{0.688006in}{2.615278in}}%
\pgfpathlineto{\pgfqpoint{3.311994in}{0.184722in}}%
\pgfpathlineto{\pgfqpoint{3.311994in}{0.184722in}}%
\pgfusepath{stroke}%
\end{pgfscope}%
\begin{pgfscope}%
\pgfsetrectcap%
\pgfsetmiterjoin%
\pgfsetlinewidth{0.501875pt}%
\definecolor{currentstroke}{rgb}{0.000000,0.000000,0.000000}%
\pgfsetstrokecolor{currentstroke}%
\pgfsetdash{}{0pt}%
\pgfpathmoveto{\pgfqpoint{2.000000in}{0.198611in}}%
\pgfpathlineto{\pgfqpoint{2.000000in}{2.601389in}}%
\pgfusepath{stroke}%
\end{pgfscope}%
\begin{pgfscope}%
\pgfsetrectcap%
\pgfsetmiterjoin%
\pgfsetlinewidth{0.501875pt}%
\definecolor{currentstroke}{rgb}{0.000000,0.000000,0.000000}%
\pgfsetstrokecolor{currentstroke}%
\pgfsetdash{}{0pt}%
\pgfpathmoveto{\pgfqpoint{0.198611in}{1.400000in}}%
\pgfpathlineto{\pgfqpoint{3.801389in}{1.400000in}}%
\pgfusepath{stroke}%
\end{pgfscope}%
\begin{pgfscope}%
\pgfsetroundcap%
\pgfsetroundjoin%
\pgfsetlinewidth{0.501875pt}%
\definecolor{currentstroke}{rgb}{0.000000,0.000000,0.000000}%
\pgfsetstrokecolor{currentstroke}%
\pgfsetdash{}{0pt}%
\pgfpathmoveto{\pgfqpoint{3.032230in}{2.375700in}}%
\pgfpathquadraticcurveto{\pgfqpoint{2.526526in}{1.930390in}}{\pgfqpoint{2.026649in}{1.490210in}}%
\pgfusepath{stroke}%
\end{pgfscope}%
\begin{pgfscope}%
\pgfsetroundcap%
\pgfsetroundjoin%
\pgfsetlinewidth{0.501875pt}%
\definecolor{currentstroke}{rgb}{0.000000,0.000000,0.000000}%
\pgfsetstrokecolor{currentstroke}%
\pgfsetdash{}{0pt}%
\pgfpathmoveto{\pgfqpoint{2.086701in}{1.506078in}}%
\pgfpathlineto{\pgfqpoint{2.026649in}{1.490210in}}%
\pgfpathlineto{\pgfqpoint{2.049986in}{1.547773in}}%
\pgfusepath{stroke}%
\end{pgfscope}%
\begin{pgfscope}%
\pgftext[x=3.080833in,y=2.401157in,left,base]{\rmfamily\fontsize{10.000000}{12.000000}\selectfont \(\displaystyle {\cdot}\)}%
\end{pgfscope}%
\begin{pgfscope}%
\pgftext[x=2.288222in,y=1.600231in,left,base]{\rmfamily\fontsize{10.000000}{12.000000}\selectfont Substitution 1}%
\end{pgfscope}%
\begin{pgfscope}%
\pgfsetroundcap%
\pgfsetroundjoin%
\pgfsetlinewidth{0.501875pt}%
\definecolor{currentstroke}{rgb}{0.000000,0.000000,0.000000}%
\pgfsetstrokecolor{currentstroke}%
\pgfsetdash{}{0pt}%
\pgfpathmoveto{\pgfqpoint{2.000000in}{2.607510in}}%
\pgfpathquadraticcurveto{\pgfqpoint{2.000000in}{2.608331in}}{\pgfqpoint{2.000000in}{2.601389in}}%
\pgfusepath{stroke}%
\end{pgfscope}%
\begin{pgfscope}%
\pgfsetroundcap%
\pgfsetroundjoin%
\pgfsetlinewidth{0.501875pt}%
\definecolor{currentstroke}{rgb}{0.000000,0.000000,0.000000}%
\pgfsetstrokecolor{currentstroke}%
\pgfsetdash{}{0pt}%
\pgfpathmoveto{\pgfqpoint{1.972222in}{2.551954in}}%
\pgfpathlineto{\pgfqpoint{2.000000in}{2.607510in}}%
\pgfpathlineto{\pgfqpoint{2.027778in}{2.551954in}}%
\pgfusepath{stroke}%
\end{pgfscope}%
\begin{pgfscope}%
\pgftext[x=2.000000in,y=2.670833in,,bottom]{\rmfamily\fontsize{10.000000}{12.000000}\selectfont \(\displaystyle \omega\)}%
\end{pgfscope}%
\begin{pgfscope}%
\pgfsetroundcap%
\pgfsetroundjoin%
\pgfsetlinewidth{0.501875pt}%
\definecolor{currentstroke}{rgb}{0.000000,0.000000,0.000000}%
\pgfsetstrokecolor{currentstroke}%
\pgfsetdash{}{0pt}%
\pgfpathmoveto{\pgfqpoint{3.807488in}{1.400000in}}%
\pgfpathquadraticcurveto{\pgfqpoint{3.808320in}{1.400000in}}{\pgfqpoint{3.801389in}{1.400000in}}%
\pgfusepath{stroke}%
\end{pgfscope}%
\begin{pgfscope}%
\pgfsetroundcap%
\pgfsetroundjoin%
\pgfsetlinewidth{0.501875pt}%
\definecolor{currentstroke}{rgb}{0.000000,0.000000,0.000000}%
\pgfsetstrokecolor{currentstroke}%
\pgfsetdash{}{0pt}%
\pgfpathmoveto{\pgfqpoint{3.751932in}{1.427778in}}%
\pgfpathlineto{\pgfqpoint{3.807488in}{1.400000in}}%
\pgfpathlineto{\pgfqpoint{3.751932in}{1.372222in}}%
\pgfusepath{stroke}%
\end{pgfscope}%
\begin{pgfscope}%
\pgftext[x=3.870833in,y=1.400000in,left,]{\rmfamily\fontsize{10.000000}{12.000000}\selectfont \(\displaystyle k\)}%
\end{pgfscope}%
\end{pgfpicture}%
\makeatother%
\endgroup%
} %
        \caption{Der Träger von $\hat\psi$ vor und nach der Substitution aus \cref{eq:substitution1_coords}}
        \label{fig:supp_psi_substitution1}
    \end{minipage}\hfill
    \begin{minipage}{0.5\textwidth}
        \centering
        \resizebox{\textwidth}{!}{%% Creator: Matplotlib, PGF backend
%%
%% To include the figure in your LaTeX document, write
%%   \input{<filename>.pgf}
%%
%% Make sure the required packages are loaded in your preamble
%%   \usepackage{pgf}
%%
%% Figures using additional raster images can only be included by \input if
%% they are in the same directory as the main LaTeX file. For loading figures
%% from other directories you can use the `import` package
%%   \usepackage{import}
%% and then include the figures with
%%   \import{<path to file>}{<filename>.pgf}
%%
%% Matplotlib used the following preamble
%%   \usepackage[utf8x]{inputenc}
%%   \usepackage[T1]{fontenc}
%%   \usepackage{amssymb}
%%
\begingroup%
\makeatletter%
\begin{pgfpicture}%
\pgfpathrectangle{\pgfpointorigin}{\pgfqpoint{4.000000in}{2.000000in}}%
\pgfusepath{use as bounding box, clip}%
\begin{pgfscope}%
\pgfsetbuttcap%
\pgfsetmiterjoin%
\definecolor{currentfill}{rgb}{1.000000,1.000000,1.000000}%
\pgfsetfillcolor{currentfill}%
\pgfsetlinewidth{0.000000pt}%
\definecolor{currentstroke}{rgb}{1.000000,1.000000,1.000000}%
\pgfsetstrokecolor{currentstroke}%
\pgfsetdash{}{0pt}%
\pgfpathmoveto{\pgfqpoint{0.000000in}{0.000000in}}%
\pgfpathlineto{\pgfqpoint{4.000000in}{0.000000in}}%
\pgfpathlineto{\pgfqpoint{4.000000in}{2.000000in}}%
\pgfpathlineto{\pgfqpoint{0.000000in}{2.000000in}}%
\pgfpathclose%
\pgfusepath{fill}%
\end{pgfscope}%
\begin{pgfscope}%
\pgfsetbuttcap%
\pgfsetmiterjoin%
\definecolor{currentfill}{rgb}{1.000000,1.000000,1.000000}%
\pgfsetfillcolor{currentfill}%
\pgfsetlinewidth{0.000000pt}%
\definecolor{currentstroke}{rgb}{0.000000,0.000000,0.000000}%
\pgfsetstrokecolor{currentstroke}%
\pgfsetstrokeopacity{0.000000}%
\pgfsetdash{}{0pt}%
\pgfpathmoveto{\pgfqpoint{0.198611in}{0.198611in}}%
\pgfpathlineto{\pgfqpoint{3.801389in}{0.198611in}}%
\pgfpathlineto{\pgfqpoint{3.801389in}{1.801389in}}%
\pgfpathlineto{\pgfqpoint{0.198611in}{1.801389in}}%
\pgfpathclose%
\pgfusepath{fill}%
\end{pgfscope}%
\begin{pgfscope}%
\pgfpathrectangle{\pgfqpoint{0.198611in}{0.198611in}}{\pgfqpoint{3.602778in}{1.602778in}} %
\pgfusepath{clip}%
\pgfsetbuttcap%
\pgfsetmiterjoin%
\definecolor{currentfill}{rgb}{0.500000,0.500000,0.500000}%
\pgfsetfillcolor{currentfill}%
\pgfsetfillopacity{0.500000}%
\pgfsetlinewidth{0.501875pt}%
\definecolor{currentstroke}{rgb}{0.000000,0.000000,0.000000}%
\pgfsetstrokecolor{currentstroke}%
\pgfsetdash{}{0pt}%
\pgfpathmoveto{\pgfqpoint{1.909931in}{0.398958in}}%
\pgfpathlineto{\pgfqpoint{1.909931in}{0.519167in}}%
\pgfpathlineto{\pgfqpoint{2.090069in}{0.519167in}}%
\pgfpathlineto{\pgfqpoint{2.090069in}{0.398958in}}%
\pgfpathclose%
\pgfusepath{stroke,fill}%
\end{pgfscope}%
\begin{pgfscope}%
\pgfpathrectangle{\pgfqpoint{0.198611in}{0.198611in}}{\pgfqpoint{3.602778in}{1.602778in}} %
\pgfusepath{clip}%
\pgfsetbuttcap%
\pgfsetmiterjoin%
\definecolor{currentfill}{rgb}{0.500000,0.500000,0.500000}%
\pgfsetfillcolor{currentfill}%
\pgfsetfillopacity{0.500000}%
\pgfsetlinewidth{0.501875pt}%
\definecolor{currentstroke}{rgb}{0.000000,0.000000,0.000000}%
\pgfsetstrokecolor{currentstroke}%
\pgfsetdash{}{0pt}%
\pgfpathmoveto{\pgfqpoint{2.307935in}{0.759583in}}%
\pgfpathlineto{\pgfqpoint{2.592760in}{0.759583in}}%
\pgfpathlineto{\pgfqpoint{4.371038in}{1.961667in}}%
\pgfpathlineto{\pgfqpoint{3.231740in}{1.961667in}}%
\pgfpathclose%
\pgfusepath{stroke,fill}%
\end{pgfscope}%
\begin{pgfscope}%
\pgfpathrectangle{\pgfqpoint{0.198611in}{0.198611in}}{\pgfqpoint{3.602778in}{1.602778in}} %
\pgfusepath{clip}%
\pgfsetbuttcap%
\pgfsetroundjoin%
\pgfsetlinewidth{0.501875pt}%
\definecolor{currentstroke}{rgb}{0.501961,0.501961,0.501961}%
\pgfsetstrokecolor{currentstroke}%
\pgfsetdash{{1.850000pt}{0.800000pt}}{0.000000pt}%
\pgfpathmoveto{\pgfqpoint{1.804251in}{0.184722in}}%
\pgfpathlineto{\pgfqpoint{3.636860in}{1.815278in}}%
\pgfpathlineto{\pgfqpoint{3.636860in}{1.815278in}}%
\pgfusepath{stroke}%
\end{pgfscope}%
\begin{pgfscope}%
\pgfpathrectangle{\pgfqpoint{0.198611in}{0.198611in}}{\pgfqpoint{3.602778in}{1.602778in}} %
\pgfusepath{clip}%
\pgfsetbuttcap%
\pgfsetroundjoin%
\pgfsetlinewidth{0.501875pt}%
\definecolor{currentstroke}{rgb}{0.501961,0.501961,0.501961}%
\pgfsetstrokecolor{currentstroke}%
\pgfsetdash{{1.850000pt}{0.800000pt}}{0.000000pt}%
\pgfpathmoveto{\pgfqpoint{0.363140in}{1.815278in}}%
\pgfpathlineto{\pgfqpoint{2.195749in}{0.184722in}}%
\pgfpathlineto{\pgfqpoint{2.195749in}{0.184722in}}%
\pgfusepath{stroke}%
\end{pgfscope}%
\begin{pgfscope}%
\pgfsetrectcap%
\pgfsetmiterjoin%
\pgfsetlinewidth{0.501875pt}%
\definecolor{currentstroke}{rgb}{0.000000,0.000000,0.000000}%
\pgfsetstrokecolor{currentstroke}%
\pgfsetdash{}{0pt}%
\pgfpathmoveto{\pgfqpoint{2.000000in}{0.198611in}}%
\pgfpathlineto{\pgfqpoint{2.000000in}{1.801389in}}%
\pgfusepath{stroke}%
\end{pgfscope}%
\begin{pgfscope}%
\pgfsetrectcap%
\pgfsetmiterjoin%
\pgfsetlinewidth{0.501875pt}%
\definecolor{currentstroke}{rgb}{0.000000,0.000000,0.000000}%
\pgfsetstrokecolor{currentstroke}%
\pgfsetdash{}{0pt}%
\pgfpathmoveto{\pgfqpoint{0.198611in}{0.358889in}}%
\pgfpathlineto{\pgfqpoint{3.801389in}{0.358889in}}%
\pgfusepath{stroke}%
\end{pgfscope}%
\begin{pgfscope}%
\pgfsetroundcap%
\pgfsetroundjoin%
\pgfsetlinewidth{0.501875pt}%
\definecolor{currentstroke}{rgb}{0.000000,0.000000,0.000000}%
\pgfsetstrokecolor{currentstroke}%
\pgfsetdash{}{0pt}%
\pgfpathmoveto{\pgfqpoint{3.302084in}{1.537713in}}%
\pgfpathquadraticcurveto{\pgfqpoint{2.661671in}{0.997340in}}{\pgfqpoint{2.027193in}{0.461973in}}%
\pgfusepath{stroke}%
\end{pgfscope}%
\begin{pgfscope}%
\pgfsetroundcap%
\pgfsetroundjoin%
\pgfsetlinewidth{0.501875pt}%
\definecolor{currentstroke}{rgb}{0.000000,0.000000,0.000000}%
\pgfsetstrokecolor{currentstroke}%
\pgfsetdash{}{0pt}%
\pgfpathmoveto{\pgfqpoint{2.087566in}{0.476570in}}%
\pgfpathlineto{\pgfqpoint{2.027193in}{0.461973in}}%
\pgfpathlineto{\pgfqpoint{2.051739in}{0.519030in}}%
\pgfusepath{stroke}%
\end{pgfscope}%
\begin{pgfscope}%
\pgftext[x=3.351042in,y=1.560972in,left,base]{\rmfamily\fontsize{10.000000}{12.000000}\selectfont \(\displaystyle {\cdot}\)}%
\end{pgfscope}%
\begin{pgfscope}%
\pgftext[x=2.360278in,y=0.599306in,left,base]{\rmfamily\fontsize{10.000000}{12.000000}\selectfont Substitution 1}%
\end{pgfscope}%
\begin{pgfscope}%
\pgfsetroundcap%
\pgfsetroundjoin%
\pgfsetlinewidth{0.501875pt}%
\definecolor{currentstroke}{rgb}{0.000000,0.000000,0.000000}%
\pgfsetstrokecolor{currentstroke}%
\pgfsetdash{}{0pt}%
\pgfpathmoveto{\pgfqpoint{2.000000in}{1.807510in}}%
\pgfpathquadraticcurveto{\pgfqpoint{2.000000in}{1.808331in}}{\pgfqpoint{2.000000in}{1.801389in}}%
\pgfusepath{stroke}%
\end{pgfscope}%
\begin{pgfscope}%
\pgfsetroundcap%
\pgfsetroundjoin%
\pgfsetlinewidth{0.501875pt}%
\definecolor{currentstroke}{rgb}{0.000000,0.000000,0.000000}%
\pgfsetstrokecolor{currentstroke}%
\pgfsetdash{}{0pt}%
\pgfpathmoveto{\pgfqpoint{1.972222in}{1.751954in}}%
\pgfpathlineto{\pgfqpoint{2.000000in}{1.807510in}}%
\pgfpathlineto{\pgfqpoint{2.027778in}{1.751954in}}%
\pgfusepath{stroke}%
\end{pgfscope}%
\begin{pgfscope}%
\pgftext[x=2.000000in,y=1.870833in,,bottom]{\rmfamily\fontsize{10.000000}{12.000000}\selectfont \(\displaystyle \omega\)}%
\end{pgfscope}%
\begin{pgfscope}%
\pgfsetroundcap%
\pgfsetroundjoin%
\pgfsetlinewidth{0.501875pt}%
\definecolor{currentstroke}{rgb}{0.000000,0.000000,0.000000}%
\pgfsetstrokecolor{currentstroke}%
\pgfsetdash{}{0pt}%
\pgfpathmoveto{\pgfqpoint{3.807488in}{0.358889in}}%
\pgfpathquadraticcurveto{\pgfqpoint{3.808320in}{0.358889in}}{\pgfqpoint{3.801389in}{0.358889in}}%
\pgfusepath{stroke}%
\end{pgfscope}%
\begin{pgfscope}%
\pgfsetroundcap%
\pgfsetroundjoin%
\pgfsetlinewidth{0.501875pt}%
\definecolor{currentstroke}{rgb}{0.000000,0.000000,0.000000}%
\pgfsetstrokecolor{currentstroke}%
\pgfsetdash{}{0pt}%
\pgfpathmoveto{\pgfqpoint{3.751932in}{0.386667in}}%
\pgfpathlineto{\pgfqpoint{3.807488in}{0.358889in}}%
\pgfpathlineto{\pgfqpoint{3.751932in}{0.331111in}}%
\pgfusepath{stroke}%
\end{pgfscope}%
\begin{pgfscope}%
\pgftext[x=3.870833in,y=0.358889in,left,]{\rmfamily\fontsize{10.000000}{12.000000}\selectfont \(\displaystyle k\)}%
\end{pgfscope}%
\end{pgfpicture}%
\makeatother%
\endgroup%
}
        \caption{Der Träger von $\hat\psi$ vor und nach der Substitution aus \cref{eq:substitution2_coords}}
        \label{fig:supp_psi_substitution2}
    \end{minipage}
\end{figure}
