%!TEX root = main.tex
%%%%%%%%%%%%%%%%%%%%%%%%%%%%%%%%%%%%%%%%%%%%%%%%%%%%%%%%%%%%%%%%%%%%%%%%%%%%%%%
% % Section 1
%%%%%%%%%%%%%%%%%%%%%%%%%%%%%%%%%%%%%%%%%%%%%%%%%%%%%%%%%%%%%%%%%%%%%%%%%%%%%%%
\section{Wavelettransformation und die Wellenfrontmenge} % (fold)
\label{sec:shearlets}

Die klassische Fouriertransformation $f(x) \mapsto \int f(x) e^{-ik x} \d x$
zerlegt eine Funktion in ihre verschiedenen Frequenzanteile und misst nach dem Satz von Payley-Wiener dabei auch die Regularität der Funktion. Es gilt nämlich $f \in C^N(\mathbb{R}^n) \cap L^1(\mathbb{R}^n) \Rightarrow \hat f(k) = O(k^N) $ für $|k| \to \infty$. Leider "`sieht"' die Fouriertransformation aber nicht, an welchen Punkten $x$ die Funktion $f$ singulär (= nicht glatt) ist. Das hängt damit zusammen, dass die "`Basisfunktionen"', die ebenen Wellen, nicht lokalisiert sind. Das Argument der Fouriertransformation $k$ kontrolliert die Richtung und Skala, die von der Basisfunktion $e^{-ikx}$ aufgelöst werden. Zusätzliche Ortsauflösung der Singularitäten gibt uns die


%%%%%%%%%%%%%%%%%%%%%%%%%%%%%%%%%%%%%%%%%%%%%%%%%%%%%%%%%%%%%%%%%%%%%%%%%%%%%%%
% % Wavelets
%%%%%%%%%%%%%%%%%%%%%%%%%%%%%%%%%%%%%%%%%%%%%%%%%%%%%%%%%%%%%%%%%%%%%%%%%%%%%%%
\subsection{Wavelettransformation} % (fold)
\label{sec:wavelettransformation}

% section wavelettransformation (end)

Einen Schritt in die richtige Richtung, nämlich die Ortsauflösung der Singularitäten, macht die Wavelettransformation. Hier wird eine Familie von Basisfunktionen für $L^2(\mathbb{R}^n)$ erzeugt von einem \textit{Mutterwavelet} $\psi$. Anders als die ebenen Wellen ist $\psi$ aber lokalisiert -- häufig sogar kompakt getragen -- und die Basis wird erzeugt durch Verschieben \emph{und} Skalieren des Mutterwavelets.

Eine Hamel-Basis für $L^2(\mathbb{R}^n)$ die aus Funktionen der Form

\begin{equation*}
    \left\{\psi_{at}(x) = a^{-\frac{n}{2}}\psi\left(a^{-1}(x-t)\right)  |~ t \in \mathbb{R}^n,  ~a \in \mathbb{R}\right\}
\end{equation*}

 mit einem \textit{Mutterwavelet} $\psi \in C^\infty (\mathbb{R}^n)$ besteht heißt \textit{stetige Waveletbasis} für $L^2(\mathbb{R}^n)$. Der Parameter $t$ heißt \textit{Verschiebungsparameter} und verschiebt das Wavelet an alle Orte des $\mathbb{R}^n$ während der \textit{Skalierungsparameter} $a$ für $a \to 0$ $\psi$ immer genauer lokalisiert. Der Faktor $a^{-\frac{n}{2}}$ sorgt dafür, dass die $L^2$-Norm aller $\psi_{at}$ gleich ist. In der Fourierdomäne wird die Verschiebung zum Phasenfaktor und der Träger mit verschwindendem $a$ immer \emph{größer}.
% Noch einmal als Formel:

% \begin{align*}
%     supp (\psi_{at}) &= a \cdot supp(\psi)+t \\
%     \Longleftrightarrow ~~
%     supp (\hat\psi_{at}) &= a^{-1} \cdot supp(\hat\psi_{at})
% \end{align*}

% Ist $t \in \mathbb{Z}^n$ (oder auf einem anderen diskreten Gitter in $\mathbb{R}^n$) und $a = 2^{-j}$, so spricht man von der \textit{diskreten Wavelettransformation}. Ist $t \in \mathbb{R}^n$ und $G \subseteq \mathrm{GL}(n,\mathbb{R})$ so ist es eine \textit{stetige Wavelettransformation} (engl. \textit{continuous wavelet transform}). Die Wavelettransformation einer Funktion (später auch temperierten Distribution) $f$ ist dann definiert als

Analog zur Definition der Fouriertransformation ist die stetige Wavelettransformation von $f$ die Projektion auf die Basisfunktionen:
\begin{equation}
    \mathcal{W}_f (a,t) = \left\langle f, \psi_{at} \right\rangle
    = a^{-\frac{n}{2}} \int f(x) \psi \left(a^{-1}(x-t)\right) \d x
\end{equation}

Ist $\psi$ eine glatte Funktion und $f$ bei $t$ glatt, so fällt $\mathcal{W}_f (a,t)$ schnell ab für $a \to 0$.
% Dies ist schnell einzusehen, wenn man sich überlegt, dass $\psi \in C^k$ impliziert, dass $\hat\psi$ eine $k$-fache Nullstelle bei $0$ hat und $\int x^l \psi(x) \d x = (-i)^{l} \partial_k^l \hat\psi(0) = 0 ~~\textrm{falls}~ l<k $.
\todo[color=green]{Kurze Plausibel machen, warum dem so ist?}
Umgekehrt fällt auch $\mathcal{W}_f(a,t)$ \emph{genau dann} nicht schnell ab, wenn $f$ bei $t$ \emph{nicht} glatt ist. Also löst die Wavelettransformation die Lage der Singularitäten von $f$ auf. Allerdings sind die klassischen  Wavelettransformationen mit isotroper Skalierung nicht in der Lage die  Orientierung der Singularitäten aufzulösen. Sie besitzen ja gar keinen Orientierungsparameter.

Um auch die Orientierung aufzulösen, muss einerseits ein Richtungsparameter eingeführt werden und andererseits dafür gesorgt werden, dass die Basisfunktionen mit immer feinerer Skala immer orientierter werden. Deshalb gibt es


%%%%%%%%%%%%%%%%%%%%%%%%%%%%%%%%%%%%%%%%%%%%%%%%%%%%%%%%%%%%%%%%%%%%%%%%%%%%%%%
% % gerichtete Wavelets
%%%%%%%%%%%%%%%%%%%%%%%%%%%%%%%%%%%%%%%%%%%%%%%%%%%%%%%%%%%%%%%%%%%%%%%%%%%%%%%
\subsection{Verallgemeinerte, gerichtete Wavelets} % (fold)
\label{sec:verallgemeinerte_gerichtete_wavelets}

% section verallgemeinerte_gerichtete_wavelets (end)

 Beispiele solcher gerichteter Wavelets sind die \textit{Curvelets} von \textcite{Candes2005}, die \textit{Shearlets} von \textcite{Kutyniok2008} sowie \textit{Contourlets} von \textcite{Contourlets}. Die mit feiner werdender Skala schärfer werdende Orientierung wird in den ersten beiden Fällen durch parabolische Skalierung implementiert. D.h. in Richtung der Orientierung im Fourierraum wird mit $a$ skaliert, während in den Richtungen senkrecht dazu mit $\sqrt a$ skaliert wird. In zwei Dimensionen gibt es nur eine weitere senkrechte Richtung, aber später wird deutlich werden, dass dies in mehr Dimensionen die richtige Verallgemeinerung der parabolischen Skalierung sein muss.

 Die Richtung der Curvelets wird durch Drehmatrizen implementiert, die auf die Variablen $(x_1,x_2)$ wirken, während bei den Shearlets die Variablen $(x_1,x_2)$ geschert werden.

%  Beide Ansätze sind bisher nur in zwei Dimensionen im Detail untersucht und arbeiten mit parabolischer Skalierung. Im Fall der \textit{Curvelets} wird die Richtungsabhängigkeit durch Drehmatrizen implementiert, während die \textit{Shearlets} mit Scherungen arbeiten. Konkret sieht dass dann so aus:

% \todo[color=green]{Hier die Formeln hin schreiben, oder ganz drauf verzichten}

% \begin{align*}
%     \psi_{a\theta t}(x) &= \det (AR)^{-\frac{1}{2}}
%                     \psi \left((AR)^{-1}(x-t) \right)
%     \condition{für Curvelets}\\
%     \psi_{ast}(x) &= \det (AS)^{-\frac{1}{2}}
%                     \psi \left((AS)^{-1}(x-t)\right)
%     \condition{für Shearlets}
% \end{align*}

% \begin{dgroup*}
% \begin{dsuspend}
%     mit der parabolischen Skalierungsmatrix
% \end{dsuspend}
% \begin{dmath*}
%     A = \begin{pmatrix}  a & 0 \\ 0 & \sqrt a\end{pmatrix}
% \end{dmath*}
% \begin{dsuspend}
%     der Drehmatrix
% \end{dsuspend}
% \begin{dmath*}
%     R = \begin{pmatrix}  \cos \theta & \sin \theta \\ - \sin \theta & \cos \theta \end{pmatrix}
% \end{dmath*}
% \begin{dsuspend}
%     und der Scherungsmatrix
% \end{dsuspend}
% \begin{dmath*}
%     S = \begin{pmatrix}  1 & -s \\ 0 & 1\end{pmatrix}
% \end{dmath*}
% \end{dgroup*}

Beide Ansätze sind in der Lage, die Wellenfrontmenge einer Distribution zu identifizieren. Allerdings sind die Rechnungen bei den \textit{Shearlets} in der praktischen Umsetzung einfacher, wenn auch von einem ästhetischen Standpunkt nicht ganz so befriedigend, da sie nicht inhärent rotationsinvariant sind, also nicht alle Symmetrien unseres Raumes abbilden. Aber nach allzu viel Ästhetik sollte man in dieser Arbeit, mit Hinblick auf die Rechnungen ab \cref{sec:die_wellenfrontmenge_von_delta_m}, ohnehin nicht fragen.

Bevor wir die konkrete Konstruktion der Shearlets widmen, brauchen wir noch ein kleines bisschen Theorie, welche Möglichkeiten wir überhaupt haben, um die Konstruktion der Wavelets zu verallgemeinern. Die weitestgehende Verallgemeinerung von "`verschiebe und skaliere ein Mutterwavelet"' um ein reproduzierendes System zu erhalten ist "`verschiebe es und lasse eine beliebge invertierbare Matrix auf die Koordinaten wirken"'. Wir definieren also eine Wirkung der affinen Gruppe $\mathbb{A}^n$ auf Funktionen $\psi \in L^2(\mathbb{R}^n)$ via

\begin{align*}
    \mathbb{A}^n \times L^2(\mathbb{R}^n) &\to L^2(\mathbb{R}^n)\\
   ((M,t) ,\psi (x)) &\mapsto |\det M ^{-\frac{1}{2}}|  \psi\left(M^{-1}(x-t)\right) \eqqcolon \psi_{M,t} (x)\\
\textrm{mit}\kern 8em&\\
    (M,t) &\in \mathbb{A}^n = \mathrm{GL}(n,\mathbb{R}) \ltimes \mathbb{R}^n
\end{align*}

Im Allgemeinen wird man aber nicht die ganze affine Gruppe benötigen, um ein reproduzierendes System zu erhalten, sondern nur alle Verschiebungen und eine Untermenge\footnote{Auch nicht notwendigerweise Untergruppe} der $\mathrm{GL}(n,\mathbb{R})$. Wann ein Mutterwavelet und die Wirkung einer solchen Untermenge ein reproduzierendes System erzeugen, sagt uns der nächste Satz:

\begin{theorem}[Zulässigkeitskriterium]
\label{thm:admissibility_criterion}
    Sei $\psi \in L^2(\mathbb{R}^n)$.
    Sei $G \subset \mathrm{GL}(n,\mathbb{R})$, $\d \mu(M)$ ein Maß auf $G$, im Falle einer Untergruppe z.B. das Haarmaß, und es gelte
    \begin{equation}
        \Delta(\psi)(k) = \int_G |\hat\psi(M^t k)|^2 |\det M| \d \mu(M) = 1
    \label{eq:admissibility criterion}
    \end{equation}

    für fast alle $k \in \hat{\mathbb{R}}^2$.
    Dann ist $(\psi, G\ltimes \mathbb{R}^n)$ ein reproduzierendes System für $L^2(\mathbb{R}^n)$, also gilt für alle $f \in L^2(\mathbb{R}^n)$

    \begin{equation}
        f = \int_{\mathbb{R}^n} \int_G \left\langle \psi_{M,t},f\right\rangle
            \psi_{M,t} \d \mu (M) \d t
    \end{equation}
\end{theorem}
In einer Dimension entspricht das Zulässigkeitskriterium genau Calderons Kriterium:
\begin{remark}[\ref{thm:admissibility_criterion} ist Calderon]
    Für $n=1$ ist $\mathrm{GL}(1,\mathbb{R}) = (\mathbb{R}^*, \cdot)$ mit dem Haarmaß $\d \mu(a) = \frac{\d \lambda(a)}{a}$ und \cref{eq:admissibility criterion} wird zu

    \begin{equation}
        \int_0^\infty \left\lvert \hat\psi(a k) \right\rvert^2 \frac{\d \lambda(a)}{a} = 1 \condition{für fast alle $k \in \hat{\mathbb{R}}$}
        \label{eq:calderon}
    \end{equation}
    \cref{eq:admissibility criterion} ist also das mehrdimensionale Analogon zu Calderons Kriterium \cite[S. 105]{Mallat2008}.
\end{remark}

Jetzt aber mehr Details zur Konstruktion der Shearlets und deren Eigenschaften:


%%%%%%%%%%%%%%%%%%%%%%%%%%%%%%%%%%%%%%%%%%%%%%%%%%%%%%%%%%%%%%%%%%%%%%%%%%%%%%%
% % Die Shearlets
%%%%%%%%%%%%%%%%%%%%%%%%%%%%%%%%%%%%%%%%%%%%%%%%%%%%%%%%%%%%%%%%%%%%%%%%%%%%%%%
\subsection{Konstruktion, Eigenschaften der Shearlets und ein wichtiger Satz} % (fold)
\label{sec:konstruktion_und_eigenschaften_der_shearlets}

Der folgende Abschnitt basiert größtenteils auf der Arbeit von \textcite{Kutyniok2008}.
 Da wir später auch komplexwertige Distributionen analysieren wollen, deren Wellenfrontmenge nicht zwingend punktsymmetrisch um den Ursprung (in der Richtung, nicht im Ort) sind, werden wir Shearlets verwenden, deren Fouriertransformierte asymetrischen Träger hat, indem wir die Shearlets aus \cite{Kutyniok2008} jeweils in zwei Shearlets aufteilen, eines mit Träger im Frequenzbereich "`nach vorne"', und eines mit Träger "`nach hinten"'.

\begin{definition}[Shearlettransformation]
\label{def:shearletttransformationI}
Seien
% \begin{dgroup*}
\begin{equation}
    \psi_1 \in L^2(\mathbb{R})
            \textrm{ mit }supp(\hat\psi_1) \subseteq \left[\frac{1}{2},2\right]
            \textrm{ und } \psi_1 \textrm{ erfülle \cref{eq:calderon}}
            \footnote{mit $G = (\mathbb{R}^*, \cdot), |\det M| d \lambda(M) = \frac{\d a}{a}$}
\label{eq:psi_1}
\end{equation}
\begin{equation}
    \psi_2 \in L^2(\mathbb{R})
            \textrm{ mit } supp(\hat\psi_1) \subseteq \left[-1,1\right]
            \textrm{ und } \left\Vert\psi_2 \right\Vert_2 = 1
\label{eq:psi_2}
\end{equation}
% \end{dgroup*}
und $\psi \in C^\infty(\mathbb{R}^2)$ implizit definiert durch

\begin{equation}
    \hat \psi(k_1,k_2) = \hat\psi_1(k_1) \hat \psi_2 \left(\frac{k_2}{k_1}\right).
\end{equation}

Sei des weiteren

\begin{equation}
    G = \left\{M_{as} \in \mathrm{GL}(2,\mathbb{R}) ~\Big|~ M_{as} = \left(\begin{smallmatrix}
        a & -\sqrt a s \\ 0 & \sqrt a
    \end{smallmatrix}\right)
    , a \in [0,1], s \in [-2,2]
    \right\}
    \label{eq:schermatrizen}
\end{equation}

Dann ist für $t \in \mathbb{R}^n, M_{as} \in G$ die Shearlettransformation von $f$ bezüglich $\psi$ definiert als

\begin{equation}
    \mathcal{S}_f(a,s,t)) =
    \left\langle D_{M_{as}}T_{t}\psi, f
    \right\rangle
    = a^{-\frac{3}{4}}\int f(x) \psi\left(
    \left(\begin{smallmatrix}
        a & -\sqrt a s \\ 0 & \sqrt a
    \end{smallmatrix}\right)^{-1}
    \left(\begin{smallmatrix}
        x-t
    \end{smallmatrix}\right)
    \right) \d x
\end{equation}

\end{definition}

Bevor es mit dem Text weiter geht, noch eine kurze Bemerkung zu Vereinfachung der Notation:

\begin{remark}[Notation]
    Der Kompaktheit halber schreiben wir auch
    \begin{equation*}
        \psi_{ast} (x_1,x_2) \coloneqq
        a^{-\frac{3}{4}} \psi\left(
    \left(\begin{smallmatrix}
        a & -\sqrt a s \\ 0 & \sqrt a
    \end{smallmatrix}\right)^{-1}
    \left(\begin{smallmatrix}
        x-t
    \end{smallmatrix}\right)
    \right)
    \end{equation*}
\end{remark}

Offenbar können $\psi_1$ und $\psi_2$ in \cref{def:shearletttransformationI} problemlos auch so gewählt werden, dass $\hat\psi_1$ und $\hat\psi_2$ glatt sind; wir stellen in \cref{eq:psi_1,eq:psi_2} ja keine allzu restriktiven Anforderungen an sie. Dann ist $\psi_{ast}$ eine Schwartz-Funktion für alle $(a,s,t)$ und die Shearlettransformation temperierter Distributionen ist wohldefiniert.  Die Anforderungen aus \cref{eq:psi_1,eq:psi_2} sind genau so gewählt, dass \cref{eq:admissibility criterion} von $\psi$ erfüllt wird, und gleichzeitig die konkreten Rechnungen zur Bestimmung der Wellenfrontmenge auch analytisch möglich sind.

Der kompakte Träger von $\hat\psi$ in der Frequenzdomäne erlaubt einfachere
Abschätzungen von Ausdrücken der Form $\left<\hat f, \hat \psi_{ast}\right>$, ist aber m.E. \emph{nicht} zwingend notwendig, um mit diesem Shearlet die Wellenfrontmenge zu bestimmen.


Die Wirkung der Scher- und Skalierungsmatrizen aus \cref{eq:schermatrizen} versteht man am besten in der Frequenzdomäne. Mit $a \to 0$ wird $\hat \psi$ immer weiter "`weg vom Ursprung"' geschoben in der Frequenzdomäne und der Träger liegt gleichzeitig in immer engeren Kegeln. Dies ist genau die Anisotropie, die uns erlaubt, nicht nur die Position der Singularitäten, sondern auch ihre Orientierung zu erkennen. Der Parameter $s$ bestimmt die Scherung des Trägers von $\psi$. Für $s=0$ ist der Träger um $k=0$ herum lokalisiert, für $s = \pm 1$ um die Diagonale bzw Antidiagonale. Das ist dargestellt in \cref{fig:supp_psi_hat} und \cref{rem:psi_hat}.


\begin{remark}[Eigenschaften von $\hat\psi_{ast}$]
\label{rem:psi_hat}
\begin{figure}[ht]
\centering
%% Creator: Matplotlib, PGF backend
%%
%% To include the figure in your LaTeX document, write
%%   \input{<filename>.pgf}
%%
%% Make sure the required packages are loaded in your preamble
%%   \usepackage{pgf}
%%
%% Figures using additional raster images can only be included by \input if
%% they are in the same directory as the main LaTeX file. For loading figures
%% from other directories you can use the `import` package
%%   \usepackage{import}
%% and then include the figures with
%%   \import{<path to file>}{<filename>.pgf}
%%
%% Matplotlib used the following preamble
%%   \usepackage[utf8x]{inputenc}
%%   \usepackage[T1]{fontenc}
%%   \usepackage{amssymb}
%%
\begingroup%
\makeatletter%
\begin{pgfpicture}%
\pgfpathrectangle{\pgfpointorigin}{\pgfqpoint{4.000000in}{2.000000in}}%
\pgfusepath{use as bounding box, clip}%
\begin{pgfscope}%
\pgfsetbuttcap%
\pgfsetmiterjoin%
\definecolor{currentfill}{rgb}{1.000000,1.000000,1.000000}%
\pgfsetfillcolor{currentfill}%
\pgfsetlinewidth{0.000000pt}%
\definecolor{currentstroke}{rgb}{1.000000,1.000000,1.000000}%
\pgfsetstrokecolor{currentstroke}%
\pgfsetdash{}{0pt}%
\pgfpathmoveto{\pgfqpoint{0.000000in}{0.000000in}}%
\pgfpathlineto{\pgfqpoint{4.000000in}{0.000000in}}%
\pgfpathlineto{\pgfqpoint{4.000000in}{2.000000in}}%
\pgfpathlineto{\pgfqpoint{0.000000in}{2.000000in}}%
\pgfpathclose%
\pgfusepath{fill}%
\end{pgfscope}%
\begin{pgfscope}%
\pgfsetbuttcap%
\pgfsetmiterjoin%
\definecolor{currentfill}{rgb}{1.000000,1.000000,1.000000}%
\pgfsetfillcolor{currentfill}%
\pgfsetlinewidth{0.000000pt}%
\definecolor{currentstroke}{rgb}{0.000000,0.000000,0.000000}%
\pgfsetstrokecolor{currentstroke}%
\pgfsetstrokeopacity{0.000000}%
\pgfsetdash{}{0pt}%
\pgfpathmoveto{\pgfqpoint{0.500000in}{0.250000in}}%
\pgfpathlineto{\pgfqpoint{3.600000in}{0.250000in}}%
\pgfpathlineto{\pgfqpoint{3.600000in}{1.760000in}}%
\pgfpathlineto{\pgfqpoint{0.500000in}{1.760000in}}%
\pgfpathclose%
\pgfusepath{fill}%
\end{pgfscope}%
\begin{pgfscope}%
\pgfpathrectangle{\pgfqpoint{0.500000in}{0.250000in}}{\pgfqpoint{3.100000in}{1.510000in}} %
\pgfusepath{clip}%
\pgfsetbuttcap%
\pgfsetmiterjoin%
\definecolor{currentfill}{rgb}{0.500000,0.500000,0.500000}%
\pgfsetfillcolor{currentfill}%
\pgfsetfillopacity{0.500000}%
\pgfsetlinewidth{0.501875pt}%
\definecolor{currentstroke}{rgb}{0.000000,0.000000,0.000000}%
\pgfsetstrokecolor{currentstroke}%
\pgfsetdash{}{0pt}%
\pgfpathmoveto{\pgfqpoint{2.011250in}{0.438750in}}%
\pgfpathlineto{\pgfqpoint{2.088750in}{0.438750in}}%
\pgfpathlineto{\pgfqpoint{2.205000in}{0.552000in}}%
\pgfpathlineto{\pgfqpoint{1.895000in}{0.552000in}}%
\pgfpathclose%
\pgfusepath{stroke,fill}%
\end{pgfscope}%
\begin{pgfscope}%
\pgfpathrectangle{\pgfqpoint{0.500000in}{0.250000in}}{\pgfqpoint{3.100000in}{1.510000in}} %
\pgfusepath{clip}%
\pgfsetbuttcap%
\pgfsetmiterjoin%
\definecolor{currentfill}{rgb}{0.500000,0.500000,0.500000}%
\pgfsetfillcolor{currentfill}%
\pgfsetfillopacity{0.500000}%
\pgfsetlinewidth{0.501875pt}%
\definecolor{currentstroke}{rgb}{0.000000,0.000000,0.000000}%
\pgfsetstrokecolor{currentstroke}%
\pgfsetdash{}{0pt}%
\pgfpathmoveto{\pgfqpoint{1.949948in}{0.652667in}}%
\pgfpathlineto{\pgfqpoint{2.150052in}{0.652667in}}%
\pgfpathlineto{\pgfqpoint{2.450208in}{1.407667in}}%
\pgfpathlineto{\pgfqpoint{1.649792in}{1.407667in}}%
\pgfpathclose%
\pgfusepath{stroke,fill}%
\end{pgfscope}%
\begin{pgfscope}%
\pgfpathrectangle{\pgfqpoint{0.500000in}{0.250000in}}{\pgfqpoint{3.100000in}{1.510000in}} %
\pgfusepath{clip}%
\pgfsetbuttcap%
\pgfsetmiterjoin%
\definecolor{currentfill}{rgb}{0.500000,0.500000,0.500000}%
\pgfsetfillcolor{currentfill}%
\pgfsetfillopacity{0.500000}%
\pgfsetlinewidth{0.501875pt}%
\definecolor{currentstroke}{rgb}{0.000000,0.000000,0.000000}%
\pgfsetstrokecolor{currentstroke}%
\pgfsetdash{}{0pt}%
\pgfpathmoveto{\pgfqpoint{2.208281in}{0.652667in}}%
\pgfpathlineto{\pgfqpoint{2.408385in}{0.652667in}}%
\pgfpathlineto{\pgfqpoint{3.483542in}{1.407667in}}%
\pgfpathlineto{\pgfqpoint{2.683125in}{1.407667in}}%
\pgfpathclose%
\pgfusepath{stroke,fill}%
\end{pgfscope}%
\begin{pgfscope}%
\pgfpathrectangle{\pgfqpoint{0.500000in}{0.250000in}}{\pgfqpoint{3.100000in}{1.510000in}} %
\pgfusepath{clip}%
\pgfsetbuttcap%
\pgfsetroundjoin%
\pgfsetlinewidth{0.501875pt}%
\definecolor{currentstroke}{rgb}{0.501961,0.501961,0.501961}%
\pgfsetstrokecolor{currentstroke}%
\pgfsetdash{{1.850000pt}{0.800000pt}}{0.000000pt}%
\pgfpathmoveto{\pgfqpoint{1.880743in}{0.236111in}}%
\pgfpathlineto{\pgfqpoint{3.459257in}{1.773889in}}%
\pgfpathlineto{\pgfqpoint{3.459257in}{1.773889in}}%
\pgfusepath{stroke}%
\end{pgfscope}%
\begin{pgfscope}%
\pgfpathrectangle{\pgfqpoint{0.500000in}{0.250000in}}{\pgfqpoint{3.100000in}{1.510000in}} %
\pgfusepath{clip}%
\pgfsetbuttcap%
\pgfsetroundjoin%
\pgfsetlinewidth{0.501875pt}%
\definecolor{currentstroke}{rgb}{0.501961,0.501961,0.501961}%
\pgfsetstrokecolor{currentstroke}%
\pgfsetdash{{1.850000pt}{0.800000pt}}{0.000000pt}%
\pgfpathmoveto{\pgfqpoint{0.640743in}{1.773889in}}%
\pgfpathlineto{\pgfqpoint{2.219257in}{0.236111in}}%
\pgfpathlineto{\pgfqpoint{2.219257in}{0.236111in}}%
\pgfusepath{stroke}%
\end{pgfscope}%
\begin{pgfscope}%
\pgfsetrectcap%
\pgfsetmiterjoin%
\pgfsetlinewidth{0.501875pt}%
\definecolor{currentstroke}{rgb}{0.000000,0.000000,0.000000}%
\pgfsetstrokecolor{currentstroke}%
\pgfsetdash{}{0pt}%
\pgfpathmoveto{\pgfqpoint{2.050000in}{0.250000in}}%
\pgfpathlineto{\pgfqpoint{2.050000in}{1.760000in}}%
\pgfusepath{stroke}%
\end{pgfscope}%
\begin{pgfscope}%
\pgfsetrectcap%
\pgfsetmiterjoin%
\pgfsetlinewidth{0.501875pt}%
\definecolor{currentstroke}{rgb}{0.000000,0.000000,0.000000}%
\pgfsetstrokecolor{currentstroke}%
\pgfsetdash{}{0pt}%
\pgfpathmoveto{\pgfqpoint{0.500000in}{0.401000in}}%
\pgfpathlineto{\pgfqpoint{3.600000in}{0.401000in}}%
\pgfusepath{stroke}%
\end{pgfscope}%
\begin{pgfscope}%
\pgfsetroundcap%
\pgfsetroundjoin%
\pgfsetlinewidth{0.501875pt}%
\definecolor{currentstroke}{rgb}{0.000000,0.000000,0.000000}%
\pgfsetstrokecolor{currentstroke}%
\pgfsetdash{}{0pt}%
\pgfpathmoveto{\pgfqpoint{1.652485in}{0.528210in}}%
\pgfpathquadraticcurveto{\pgfqpoint{1.779212in}{0.521920in}}{\pgfqpoint{1.898185in}{0.516015in}}%
\pgfusepath{stroke}%
\end{pgfscope}%
\begin{pgfscope}%
\pgfsetroundcap%
\pgfsetroundjoin%
\pgfsetlinewidth{0.501875pt}%
\definecolor{currentstroke}{rgb}{0.000000,0.000000,0.000000}%
\pgfsetstrokecolor{currentstroke}%
\pgfsetdash{}{0pt}%
\pgfpathmoveto{\pgfqpoint{1.844075in}{0.546513in}}%
\pgfpathlineto{\pgfqpoint{1.898185in}{0.516015in}}%
\pgfpathlineto{\pgfqpoint{1.841321in}{0.491026in}}%
\pgfusepath{stroke}%
\end{pgfscope}%
\begin{pgfscope}%
\pgftext[x=0.887500in,y=0.514250in,left,base]{\rmfamily\fontsize{10.000000}{12.000000}\selectfont \(\displaystyle a = 1, s = 0\)}%
\end{pgfscope}%
\begin{pgfscope}%
\pgfsetroundcap%
\pgfsetroundjoin%
\pgfsetlinewidth{0.501875pt}%
\definecolor{currentstroke}{rgb}{0.000000,0.000000,0.000000}%
\pgfsetstrokecolor{currentstroke}%
\pgfsetdash{}{0pt}%
\pgfpathmoveto{\pgfqpoint{1.442466in}{1.092221in}}%
\pgfpathquadraticcurveto{\pgfqpoint{1.557990in}{1.086989in}}{\pgfqpoint{1.665758in}{1.082108in}}%
\pgfusepath{stroke}%
\end{pgfscope}%
\begin{pgfscope}%
\pgfsetroundcap%
\pgfsetroundjoin%
\pgfsetlinewidth{0.501875pt}%
\definecolor{currentstroke}{rgb}{0.000000,0.000000,0.000000}%
\pgfsetstrokecolor{currentstroke}%
\pgfsetdash{}{0pt}%
\pgfpathmoveto{\pgfqpoint{1.611516in}{1.112371in}}%
\pgfpathlineto{\pgfqpoint{1.665758in}{1.082108in}}%
\pgfpathlineto{\pgfqpoint{1.609002in}{1.056872in}}%
\pgfusepath{stroke}%
\end{pgfscope}%
\begin{pgfscope}%
\pgftext[x=0.500000in,y=1.080500in,left,base]{\rmfamily\fontsize{10.000000}{12.000000}\selectfont \(\displaystyle a = 0.15, s = 0\)}%
\end{pgfscope}%
\begin{pgfscope}%
\pgfsetroundcap%
\pgfsetroundjoin%
\pgfsetlinewidth{0.501875pt}%
\definecolor{currentstroke}{rgb}{0.000000,0.000000,0.000000}%
\pgfsetstrokecolor{currentstroke}%
\pgfsetdash{}{0pt}%
\pgfpathmoveto{\pgfqpoint{3.243990in}{0.689247in}}%
\pgfpathquadraticcurveto{\pgfqpoint{3.046540in}{0.802467in}}{\pgfqpoint{2.855825in}{0.911824in}}%
\pgfusepath{stroke}%
\end{pgfscope}%
\begin{pgfscope}%
\pgfsetroundcap%
\pgfsetroundjoin%
\pgfsetlinewidth{0.501875pt}%
\definecolor{currentstroke}{rgb}{0.000000,0.000000,0.000000}%
\pgfsetstrokecolor{currentstroke}%
\pgfsetdash{}{0pt}%
\pgfpathmoveto{\pgfqpoint{2.890202in}{0.860092in}}%
\pgfpathlineto{\pgfqpoint{2.855825in}{0.911824in}}%
\pgfpathlineto{\pgfqpoint{2.917837in}{0.908287in}}%
\pgfusepath{stroke}%
\end{pgfscope}%
\begin{pgfscope}%
\pgftext[x=2.980000in,y=0.552000in,left,base]{\rmfamily\fontsize{10.000000}{12.000000}\selectfont \(\displaystyle a = 0.15, s = 1\)}%
\end{pgfscope}%
\begin{pgfscope}%
\pgfsetroundcap%
\pgfsetroundjoin%
\pgfsetlinewidth{0.501875pt}%
\definecolor{currentstroke}{rgb}{0.000000,0.000000,0.000000}%
\pgfsetstrokecolor{currentstroke}%
\pgfsetdash{}{0pt}%
\pgfpathmoveto{\pgfqpoint{2.050000in}{1.766125in}}%
\pgfpathquadraticcurveto{\pgfqpoint{2.050000in}{1.766944in}}{\pgfqpoint{2.050000in}{1.760000in}}%
\pgfusepath{stroke}%
\end{pgfscope}%
\begin{pgfscope}%
\pgfsetroundcap%
\pgfsetroundjoin%
\pgfsetlinewidth{0.501875pt}%
\definecolor{currentstroke}{rgb}{0.000000,0.000000,0.000000}%
\pgfsetstrokecolor{currentstroke}%
\pgfsetdash{}{0pt}%
\pgfpathmoveto{\pgfqpoint{2.022222in}{1.710569in}}%
\pgfpathlineto{\pgfqpoint{2.050000in}{1.766125in}}%
\pgfpathlineto{\pgfqpoint{2.077778in}{1.710569in}}%
\pgfusepath{stroke}%
\end{pgfscope}%
\begin{pgfscope}%
\pgftext[x=2.050000in,y=1.829444in,,bottom]{\rmfamily\fontsize{10.000000}{12.000000}\selectfont \(\displaystyle k_1 / \omega\)}%
\end{pgfscope}%
\begin{pgfscope}%
\pgfsetroundcap%
\pgfsetroundjoin%
\pgfsetlinewidth{0.501875pt}%
\definecolor{currentstroke}{rgb}{0.000000,0.000000,0.000000}%
\pgfsetstrokecolor{currentstroke}%
\pgfsetdash{}{0pt}%
\pgfpathmoveto{\pgfqpoint{3.606104in}{0.401000in}}%
\pgfpathquadraticcurveto{\pgfqpoint{3.606934in}{0.401000in}}{\pgfqpoint{3.600000in}{0.401000in}}%
\pgfusepath{stroke}%
\end{pgfscope}%
\begin{pgfscope}%
\pgfsetroundcap%
\pgfsetroundjoin%
\pgfsetlinewidth{0.501875pt}%
\definecolor{currentstroke}{rgb}{0.000000,0.000000,0.000000}%
\pgfsetstrokecolor{currentstroke}%
\pgfsetdash{}{0pt}%
\pgfpathmoveto{\pgfqpoint{3.550548in}{0.428778in}}%
\pgfpathlineto{\pgfqpoint{3.606104in}{0.401000in}}%
\pgfpathlineto{\pgfqpoint{3.550548in}{0.373222in}}%
\pgfusepath{stroke}%
\end{pgfscope}%
\begin{pgfscope}%
\pgftext[x=3.669444in,y=0.401000in,left,]{\rmfamily\fontsize{10.000000}{12.000000}\selectfont \(\displaystyle k_2 / k\)}%
\end{pgfscope}%
\end{pgfpicture}%
\makeatother%
\endgroup%

\caption{Der Träger von $\hat \psi_{ast}$ für verschiedene $a, s$. Man sieht gut,
wie $supp (\hat \psi_{ast})$ für kleinere $a$ in immer spitzeren Kegeln liegt. Da wir in den konkreten Rechnungen später $(k_1, k_2) = (\omega, k)$ nennen werden und Minkowski-Diagramme üblicherweise mit $\omega$ auf der $y$-Achse dargestellt werden, haben wir hier beide Namen eingetragen und alles an der Diagonale gespiegelt.}
\label{fig:supp_psi_hat}
\end{figure}

\label{cor:psi_hat}
Im Fourierraum ist $\hat{\psi}_{ast}$ gegeben durch

\begin{equation}
    \hat \psi_{ast}{(k_1, k_2)} = a^{\frac{3}{4}}e^{-ikx}\hat\psi_1(a k_1) \hat\psi_{2}\left(a^{-\frac{1}{2}}\left(\frac{k_2}{k_1}-s\right)\right)
\label{eq:hat_psi_ast}
\end{equation}

und es gilt

\begin{equation}
\label{eq:supp_psi}
    supp(\hat \psi) \subset \left\{k \in  \hat{\mathbb{R}}^2 ~\Big| ~k_1 \in \left[\frac{1}{2 a} , \frac{2}{a}\right], \left|\frac{k_2}{k_1} - s\right| \leq \sqrt{a} \right\}
\end{equation}
\end{remark}

Eine weitere Eigenschaft, die aus dieser Definition der Shearlets folgt, ist der schnelle Abfall der Shearlets abseits von $t$:

\begin{proposition}[$\psi_{ast}$ fällt schnell ab]
\label{prop:shearlets_decay_rapidly}
Sei $\psi \in L^2(\mathbb{R}^2)$ ein Shearlet wie in \cref{def:shearletttransformationI} und $M_{as}$ wie in \cref{eq:schermatrizen}. Dann gilt für alle $N \in  \mathbb{N}$, dass es eine konstante $C_N$ gibt s.d. für alle $t \in \mathbb{R}^2$ gilt

\begin{dmath*}
    \left| \psi_{ast}(x_1,x_2) \right|
    \leq
    C_k \left| \det M_{as} \right|^{-\frac{1}{2}}\left(1+\left|M_{as}^{-1}
    \left(
    \begin{smallmatrix}
        x_1-t_1 \\ x_2-t_2
    \end{smallmatrix}\right)
    \right|^2\right)^{-N}
    = C_k a^{-\frac{3}{4}}\left(1+a^{-2}\left(x_1-t_1\right)^2
        + 2 a^{-2}s\left(x_1-t_1\right)\left(x_2-t_2\right)
        + a^{-1}\left(1+a^{-1}s^2\right)\left(x_2-t_2\right)^2
    \right)^{-N}
\end{dmath*}

Und insbesondere ist $C_N = \frac{15}{2}\frac{\sqrt{a} + s}{a^2}\left(\Vert \hat\psi \Vert_\infty + \Vert \Laplace^N \hat\psi \Vert_\infty\right)$

\end{proposition}


Wer bis hier aufmerksam mitgelesen hat und sich \cref{fig:supp_psi_hat} genau angeschaut hat, wird bemerkt haben, dass $supp(\hat\psi_{ast})$ für alle $a$ und $s$ quasi nur im Quadranten $x_1^2 \geq x_2^2$ und $x_2 \geq 0$ liegt.
Mit den Namen der Physik $(k_1, k_2) = (\omega, k)$ entspricht das dem "`Vorwärtslichtkegel"'. Glücklicherweise liegen alle analysierten Distributionen im gleich definierten $L(C)^\vee$.
Wie der folgende Satz zeigt, erzeugt $\psi$ auch nur für solche $f$ ein reproduzierendes System.

\begin{theorem}[$\psi$ reproduziert $L^2(C)^\vee$]
\label{thm:shearlets_reproduzieren}
    Sei
    \begin{equation*}
        C \coloneqq \left\{(k_1,k_2)\in \hat{\mathbb{R}}^2
        \Big| k_1 \geq 2 \textrm{ und } \left\lvert\tfrac{k_2}{k_1}\right\rvert \leq 1
        \right\}
    \end{equation*}
    und
    \begin{equation}
        L^2(C)^\vee \coloneqq \{f \in L^2(\mathbb{R}^2) ~|~ supp (\hat f) \subset C\}
        \label{eq:L2_cone}
    \end{equation}

    Dann ist $\psi$ aus \cref{def:shearletttransformationI} ein reproduzierendes System für $L^2(C)^\vee$, also gilt für alle
    $f \in L^2(C)^\vee$:
    \begin{equation}
        f(x) = \int_{\mathbb{R}^2} \int_{-2}^2 \int_0^1
                \left\langle f,\psi_{ast}\right\rangle \psi_{ast}(x)
                \frac{\d a}{a^3} \d s \d t.
    \end{equation}
\end{theorem}

Um nun nicht nur ein reproduzierendes System für $L^2(C)^\vee$, sondern für ganz $L^2(\mathbb{R}^2)$ zu erhalten, muss $\hat \psi_{ast}$ noch in den rechten, linken und rückwärts liegenden Kegel gedreht und geschoben werden. Zusätzlich muss noch eine weitere Funktion $W$ gefunden werden, welche die groben Skalen (also $|k_1|, |k_2| \leq 2$) auflöst.

\todo{hier ne eigene Definition draus machen?}

Wir definieren also
\begin{align}
    \hat \psi_{ast}^{(1)} (k_1, k_2) &\coloneqq \hat\psi_{ast} (k_1,k_2),
    \qquad
    \hat\psi_{ast}^{(3)}(k_1,k_2) \coloneqq \hat\psi_{ast} (-k_1,-k_2),
    \nonumber\\
    \hat \psi_{ast}^{(2)} (k_1, k_2) &\coloneqq \hat\psi_{ast} (k_2,k_1),
    \qquad
    \hat\psi_{ast}^{(4)}(k_1, k_2) \coloneqq \hat\psi_{ast} (-k_2,-k_1)
    \label{eq:psi(i)}
\end{align}

Des Weiteren gibt es ein $W(x)$ s.d. $\hat W(k) \in C^\infty (\hat{\mathbb{R}}^2)$ und

\begin{align*}
    \Vert f \Vert^2 =
    \int_{\mathbb{R}^2} |\left\langle f, T_t W \right\rangle|^2 \d t
    + \sum_{i=1}^4 \int |\left\langle f, \psi_{ast} \right\rangle|
    \, \d \mu (a,s,t)
\end{align*}

und somit erhalten wir (dank Parseval) ein reproduzierendes System für ganz $L^2$, da ja
$L^2 (\mathbb{R}^2) = L^2 (\textrm{grobe Skalen})^\vee \oplus \sum_{i=1}^4 L^2 \left(C^{(i)}\right)^\vee$ ist. Mehr Details dazu finden sich in \cite[S. 28 ff]{Kutyniok2008}.

\begin{figure}[ht]
\centering
%% Creator: Matplotlib, PGF backend
%%
%% To include the figure in your LaTeX document, write
%%   \input{<filename>.pgf}
%%
%% Make sure the required packages are loaded in your preamble
%%   \usepackage{pgf}
%%
%% Figures using additional raster images can only be included by \input if
%% they are in the same directory as the main LaTeX file. For loading figures
%% from other directories you can use the `import` package
%%   \usepackage{import}
%% and then include the figures with
%%   \import{<path to file>}{<filename>.pgf}
%%
%% Matplotlib used the following preamble
%%   \usepackage[utf8x]{inputenc}
%%   \usepackage[T1]{fontenc}
%%   \usepackage{amssymb}
%%
\begingroup%
\makeatletter%
\begin{pgfpicture}%
\pgfpathrectangle{\pgfpointorigin}{\pgfqpoint{3.000000in}{3.000000in}}%
\pgfusepath{use as bounding box, clip}%
\begin{pgfscope}%
\pgfsetbuttcap%
\pgfsetmiterjoin%
\definecolor{currentfill}{rgb}{1.000000,1.000000,1.000000}%
\pgfsetfillcolor{currentfill}%
\pgfsetlinewidth{0.000000pt}%
\definecolor{currentstroke}{rgb}{1.000000,1.000000,1.000000}%
\pgfsetstrokecolor{currentstroke}%
\pgfsetdash{}{0pt}%
\pgfpathmoveto{\pgfqpoint{0.000000in}{0.000000in}}%
\pgfpathlineto{\pgfqpoint{3.000000in}{0.000000in}}%
\pgfpathlineto{\pgfqpoint{3.000000in}{3.000000in}}%
\pgfpathlineto{\pgfqpoint{0.000000in}{3.000000in}}%
\pgfpathclose%
\pgfusepath{fill}%
\end{pgfscope}%
\begin{pgfscope}%
\pgfsetbuttcap%
\pgfsetmiterjoin%
\definecolor{currentfill}{rgb}{1.000000,1.000000,1.000000}%
\pgfsetfillcolor{currentfill}%
\pgfsetlinewidth{0.000000pt}%
\definecolor{currentstroke}{rgb}{0.000000,0.000000,0.000000}%
\pgfsetstrokecolor{currentstroke}%
\pgfsetstrokeopacity{0.000000}%
\pgfsetdash{}{0pt}%
\pgfpathmoveto{\pgfqpoint{0.198611in}{0.198611in}}%
\pgfpathlineto{\pgfqpoint{2.801389in}{0.198611in}}%
\pgfpathlineto{\pgfqpoint{2.801389in}{2.801389in}}%
\pgfpathlineto{\pgfqpoint{0.198611in}{2.801389in}}%
\pgfpathclose%
\pgfusepath{fill}%
\end{pgfscope}%
\begin{pgfscope}%
\pgfpathrectangle{\pgfqpoint{0.198611in}{0.198611in}}{\pgfqpoint{2.602778in}{2.602778in}} %
\pgfusepath{clip}%
\pgfsetbuttcap%
\pgfsetroundjoin%
\pgfsetlinewidth{0.501875pt}%
\definecolor{currentstroke}{rgb}{0.501961,0.501961,0.501961}%
\pgfsetstrokecolor{currentstroke}%
\pgfsetdash{{1.850000pt}{0.800000pt}}{0.000000pt}%
\pgfpathmoveto{\pgfqpoint{1.066204in}{1.066204in}}%
\pgfpathlineto{\pgfqpoint{1.066204in}{1.933796in}}%
\pgfusepath{stroke}%
\end{pgfscope}%
\begin{pgfscope}%
\pgfpathrectangle{\pgfqpoint{0.198611in}{0.198611in}}{\pgfqpoint{2.602778in}{2.602778in}} %
\pgfusepath{clip}%
\pgfsetbuttcap%
\pgfsetroundjoin%
\pgfsetlinewidth{0.501875pt}%
\definecolor{currentstroke}{rgb}{0.501961,0.501961,0.501961}%
\pgfsetstrokecolor{currentstroke}%
\pgfsetdash{{1.850000pt}{0.800000pt}}{0.000000pt}%
\pgfpathmoveto{\pgfqpoint{1.933796in}{1.066204in}}%
\pgfpathlineto{\pgfqpoint{1.933796in}{1.933796in}}%
\pgfusepath{stroke}%
\end{pgfscope}%
\begin{pgfscope}%
\pgfpathrectangle{\pgfqpoint{0.198611in}{0.198611in}}{\pgfqpoint{2.602778in}{2.602778in}} %
\pgfusepath{clip}%
\pgfsetbuttcap%
\pgfsetroundjoin%
\pgfsetlinewidth{0.501875pt}%
\definecolor{currentstroke}{rgb}{0.501961,0.501961,0.501961}%
\pgfsetstrokecolor{currentstroke}%
\pgfsetdash{{1.850000pt}{0.800000pt}}{0.000000pt}%
\pgfpathmoveto{\pgfqpoint{1.933796in}{1.933796in}}%
\pgfpathlineto{\pgfqpoint{1.979459in}{1.979459in}}%
\pgfpathlineto{\pgfqpoint{2.025122in}{2.025122in}}%
\pgfpathlineto{\pgfqpoint{2.070785in}{2.070785in}}%
\pgfpathlineto{\pgfqpoint{2.116447in}{2.116447in}}%
\pgfpathlineto{\pgfqpoint{2.162110in}{2.162110in}}%
\pgfpathlineto{\pgfqpoint{2.207773in}{2.207773in}}%
\pgfpathlineto{\pgfqpoint{2.253436in}{2.253436in}}%
\pgfpathlineto{\pgfqpoint{2.299098in}{2.299098in}}%
\pgfpathlineto{\pgfqpoint{2.344761in}{2.344761in}}%
\pgfpathlineto{\pgfqpoint{2.390424in}{2.390424in}}%
\pgfpathlineto{\pgfqpoint{2.436087in}{2.436087in}}%
\pgfpathlineto{\pgfqpoint{2.481750in}{2.481750in}}%
\pgfpathlineto{\pgfqpoint{2.527412in}{2.527412in}}%
\pgfpathlineto{\pgfqpoint{2.573075in}{2.573075in}}%
\pgfpathlineto{\pgfqpoint{2.618738in}{2.618738in}}%
\pgfpathlineto{\pgfqpoint{2.664401in}{2.664401in}}%
\pgfpathlineto{\pgfqpoint{2.710063in}{2.710063in}}%
\pgfpathlineto{\pgfqpoint{2.755726in}{2.755726in}}%
\pgfpathlineto{\pgfqpoint{2.801389in}{2.801389in}}%
\pgfusepath{stroke}%
\end{pgfscope}%
\begin{pgfscope}%
\pgfpathrectangle{\pgfqpoint{0.198611in}{0.198611in}}{\pgfqpoint{2.602778in}{2.602778in}} %
\pgfusepath{clip}%
\pgfsetbuttcap%
\pgfsetroundjoin%
\pgfsetlinewidth{0.501875pt}%
\definecolor{currentstroke}{rgb}{0.501961,0.501961,0.501961}%
\pgfsetstrokecolor{currentstroke}%
\pgfsetdash{{1.850000pt}{0.800000pt}}{0.000000pt}%
\pgfpathmoveto{\pgfqpoint{1.933796in}{1.066204in}}%
\pgfpathlineto{\pgfqpoint{1.979459in}{1.020541in}}%
\pgfpathlineto{\pgfqpoint{2.025122in}{0.974878in}}%
\pgfpathlineto{\pgfqpoint{2.070785in}{0.929215in}}%
\pgfpathlineto{\pgfqpoint{2.116447in}{0.883553in}}%
\pgfpathlineto{\pgfqpoint{2.162110in}{0.837890in}}%
\pgfpathlineto{\pgfqpoint{2.207773in}{0.792227in}}%
\pgfpathlineto{\pgfqpoint{2.253436in}{0.746564in}}%
\pgfpathlineto{\pgfqpoint{2.299098in}{0.700902in}}%
\pgfpathlineto{\pgfqpoint{2.344761in}{0.655239in}}%
\pgfpathlineto{\pgfqpoint{2.390424in}{0.609576in}}%
\pgfpathlineto{\pgfqpoint{2.436087in}{0.563913in}}%
\pgfpathlineto{\pgfqpoint{2.481750in}{0.518250in}}%
\pgfpathlineto{\pgfqpoint{2.527412in}{0.472588in}}%
\pgfpathlineto{\pgfqpoint{2.573075in}{0.426925in}}%
\pgfpathlineto{\pgfqpoint{2.618738in}{0.381262in}}%
\pgfpathlineto{\pgfqpoint{2.664401in}{0.335599in}}%
\pgfpathlineto{\pgfqpoint{2.710063in}{0.289937in}}%
\pgfpathlineto{\pgfqpoint{2.755726in}{0.244274in}}%
\pgfpathlineto{\pgfqpoint{2.801389in}{0.198611in}}%
\pgfusepath{stroke}%
\end{pgfscope}%
\begin{pgfscope}%
\pgfpathrectangle{\pgfqpoint{0.198611in}{0.198611in}}{\pgfqpoint{2.602778in}{2.602778in}} %
\pgfusepath{clip}%
\pgfsetbuttcap%
\pgfsetroundjoin%
\pgfsetlinewidth{0.501875pt}%
\definecolor{currentstroke}{rgb}{0.501961,0.501961,0.501961}%
\pgfsetstrokecolor{currentstroke}%
\pgfsetdash{{1.850000pt}{0.800000pt}}{0.000000pt}%
\pgfpathmoveto{\pgfqpoint{0.198611in}{0.198611in}}%
\pgfpathlineto{\pgfqpoint{0.244274in}{0.244274in}}%
\pgfpathlineto{\pgfqpoint{0.289937in}{0.289937in}}%
\pgfpathlineto{\pgfqpoint{0.335599in}{0.335599in}}%
\pgfpathlineto{\pgfqpoint{0.381262in}{0.381262in}}%
\pgfpathlineto{\pgfqpoint{0.426925in}{0.426925in}}%
\pgfpathlineto{\pgfqpoint{0.472588in}{0.472588in}}%
\pgfpathlineto{\pgfqpoint{0.518250in}{0.518250in}}%
\pgfpathlineto{\pgfqpoint{0.563913in}{0.563913in}}%
\pgfpathlineto{\pgfqpoint{0.609576in}{0.609576in}}%
\pgfpathlineto{\pgfqpoint{0.655239in}{0.655239in}}%
\pgfpathlineto{\pgfqpoint{0.700902in}{0.700902in}}%
\pgfpathlineto{\pgfqpoint{0.746564in}{0.746564in}}%
\pgfpathlineto{\pgfqpoint{0.792227in}{0.792227in}}%
\pgfpathlineto{\pgfqpoint{0.837890in}{0.837890in}}%
\pgfpathlineto{\pgfqpoint{0.883553in}{0.883553in}}%
\pgfpathlineto{\pgfqpoint{0.929215in}{0.929215in}}%
\pgfpathlineto{\pgfqpoint{0.974878in}{0.974878in}}%
\pgfpathlineto{\pgfqpoint{1.020541in}{1.020541in}}%
\pgfpathlineto{\pgfqpoint{1.066204in}{1.066204in}}%
\pgfusepath{stroke}%
\end{pgfscope}%
\begin{pgfscope}%
\pgfpathrectangle{\pgfqpoint{0.198611in}{0.198611in}}{\pgfqpoint{2.602778in}{2.602778in}} %
\pgfusepath{clip}%
\pgfsetbuttcap%
\pgfsetroundjoin%
\pgfsetlinewidth{0.501875pt}%
\definecolor{currentstroke}{rgb}{0.501961,0.501961,0.501961}%
\pgfsetstrokecolor{currentstroke}%
\pgfsetdash{{1.850000pt}{0.800000pt}}{0.000000pt}%
\pgfpathmoveto{\pgfqpoint{0.198611in}{2.801389in}}%
\pgfpathlineto{\pgfqpoint{0.244274in}{2.755726in}}%
\pgfpathlineto{\pgfqpoint{0.289937in}{2.710063in}}%
\pgfpathlineto{\pgfqpoint{0.335599in}{2.664401in}}%
\pgfpathlineto{\pgfqpoint{0.381262in}{2.618738in}}%
\pgfpathlineto{\pgfqpoint{0.426925in}{2.573075in}}%
\pgfpathlineto{\pgfqpoint{0.472588in}{2.527412in}}%
\pgfpathlineto{\pgfqpoint{0.518250in}{2.481750in}}%
\pgfpathlineto{\pgfqpoint{0.563913in}{2.436087in}}%
\pgfpathlineto{\pgfqpoint{0.609576in}{2.390424in}}%
\pgfpathlineto{\pgfqpoint{0.655239in}{2.344761in}}%
\pgfpathlineto{\pgfqpoint{0.700902in}{2.299098in}}%
\pgfpathlineto{\pgfqpoint{0.746564in}{2.253436in}}%
\pgfpathlineto{\pgfqpoint{0.792227in}{2.207773in}}%
\pgfpathlineto{\pgfqpoint{0.837890in}{2.162110in}}%
\pgfpathlineto{\pgfqpoint{0.883553in}{2.116447in}}%
\pgfpathlineto{\pgfqpoint{0.929215in}{2.070785in}}%
\pgfpathlineto{\pgfqpoint{0.974878in}{2.025122in}}%
\pgfpathlineto{\pgfqpoint{1.020541in}{1.979459in}}%
\pgfpathlineto{\pgfqpoint{1.066204in}{1.933796in}}%
\pgfusepath{stroke}%
\end{pgfscope}%
\begin{pgfscope}%
\pgfpathrectangle{\pgfqpoint{0.198611in}{0.198611in}}{\pgfqpoint{2.602778in}{2.602778in}} %
\pgfusepath{clip}%
\pgfsetbuttcap%
\pgfsetroundjoin%
\pgfsetlinewidth{0.501875pt}%
\definecolor{currentstroke}{rgb}{0.501961,0.501961,0.501961}%
\pgfsetstrokecolor{currentstroke}%
\pgfsetdash{{1.850000pt}{0.800000pt}}{0.000000pt}%
\pgfpathmoveto{\pgfqpoint{1.066204in}{1.933796in}}%
\pgfpathlineto{\pgfqpoint{1.933796in}{1.933796in}}%
\pgfusepath{stroke}%
\end{pgfscope}%
\begin{pgfscope}%
\pgfpathrectangle{\pgfqpoint{0.198611in}{0.198611in}}{\pgfqpoint{2.602778in}{2.602778in}} %
\pgfusepath{clip}%
\pgfsetbuttcap%
\pgfsetroundjoin%
\pgfsetlinewidth{0.501875pt}%
\definecolor{currentstroke}{rgb}{0.501961,0.501961,0.501961}%
\pgfsetstrokecolor{currentstroke}%
\pgfsetdash{{1.850000pt}{0.800000pt}}{0.000000pt}%
\pgfpathmoveto{\pgfqpoint{1.066204in}{1.066204in}}%
\pgfpathlineto{\pgfqpoint{1.933796in}{1.066204in}}%
\pgfusepath{stroke}%
\end{pgfscope}%
\begin{pgfscope}%
\pgfsetrectcap%
\pgfsetmiterjoin%
\pgfsetlinewidth{0.501875pt}%
\definecolor{currentstroke}{rgb}{0.000000,0.000000,0.000000}%
\pgfsetstrokecolor{currentstroke}%
\pgfsetdash{}{0pt}%
\pgfpathmoveto{\pgfqpoint{1.500000in}{0.198611in}}%
\pgfpathlineto{\pgfqpoint{1.500000in}{2.801389in}}%
\pgfusepath{stroke}%
\end{pgfscope}%
\begin{pgfscope}%
\pgfsetrectcap%
\pgfsetmiterjoin%
\pgfsetlinewidth{0.501875pt}%
\definecolor{currentstroke}{rgb}{0.000000,0.000000,0.000000}%
\pgfsetstrokecolor{currentstroke}%
\pgfsetdash{}{0pt}%
\pgfpathmoveto{\pgfqpoint{0.198611in}{1.500000in}}%
\pgfpathlineto{\pgfqpoint{2.801389in}{1.500000in}}%
\pgfusepath{stroke}%
\end{pgfscope}%
\begin{pgfscope}%
\pgftext[x=1.565069in,y=2.367593in,left,base]{\rmfamily\fontsize{10.000000}{12.000000}\selectfont \(\displaystyle C = C^{(1)}\)}%
\end{pgfscope}%
\begin{pgfscope}%
\pgftext[x=0.632407in,y=1.565069in,left,base]{\rmfamily\fontsize{10.000000}{12.000000}\selectfont \(\displaystyle C^{(2)}\)}%
\end{pgfscope}%
\begin{pgfscope}%
\pgftext[x=1.565069in,y=0.632407in,left,base]{\rmfamily\fontsize{10.000000}{12.000000}\selectfont \(\displaystyle C^{(3)}\)}%
\end{pgfscope}%
\begin{pgfscope}%
\pgftext[x=2.259144in,y=1.565069in,left,base]{\rmfamily\fontsize{10.000000}{12.000000}\selectfont \(\displaystyle C^{(4)}\)}%
\end{pgfscope}%
\begin{pgfscope}%
\pgftext[x=1.152963in,y=1.565069in,left,base]{\rmfamily\fontsize{10.000000}{12.000000}\selectfont grobe Skalen}%
\end{pgfscope}%
\begin{pgfscope}%
\pgfsetroundcap%
\pgfsetroundjoin%
\pgfsetlinewidth{0.501875pt}%
\definecolor{currentstroke}{rgb}{0.000000,0.000000,0.000000}%
\pgfsetstrokecolor{currentstroke}%
\pgfsetdash{}{0pt}%
\pgfpathmoveto{\pgfqpoint{1.500000in}{2.807510in}}%
\pgfpathquadraticcurveto{\pgfqpoint{1.500000in}{2.808331in}}{\pgfqpoint{1.500000in}{2.801389in}}%
\pgfusepath{stroke}%
\end{pgfscope}%
\begin{pgfscope}%
\pgfsetroundcap%
\pgfsetroundjoin%
\pgfsetlinewidth{0.501875pt}%
\definecolor{currentstroke}{rgb}{0.000000,0.000000,0.000000}%
\pgfsetstrokecolor{currentstroke}%
\pgfsetdash{}{0pt}%
\pgfpathmoveto{\pgfqpoint{1.472222in}{2.751954in}}%
\pgfpathlineto{\pgfqpoint{1.500000in}{2.807510in}}%
\pgfpathlineto{\pgfqpoint{1.527778in}{2.751954in}}%
\pgfusepath{stroke}%
\end{pgfscope}%
\begin{pgfscope}%
\pgftext[x=1.500000in,y=2.870833in,,bottom]{\rmfamily\fontsize{10.000000}{12.000000}\selectfont \(\displaystyle k_1\)}%
\end{pgfscope}%
\begin{pgfscope}%
\pgfsetroundcap%
\pgfsetroundjoin%
\pgfsetlinewidth{0.501875pt}%
\definecolor{currentstroke}{rgb}{0.000000,0.000000,0.000000}%
\pgfsetstrokecolor{currentstroke}%
\pgfsetdash{}{0pt}%
\pgfpathmoveto{\pgfqpoint{2.807604in}{1.500000in}}%
\pgfpathquadraticcurveto{\pgfqpoint{2.808378in}{1.500000in}}{\pgfqpoint{2.801389in}{1.500000in}}%
\pgfusepath{stroke}%
\end{pgfscope}%
\begin{pgfscope}%
\pgfsetroundcap%
\pgfsetroundjoin%
\pgfsetlinewidth{0.501875pt}%
\definecolor{currentstroke}{rgb}{0.000000,0.000000,0.000000}%
\pgfsetstrokecolor{currentstroke}%
\pgfsetdash{}{0pt}%
\pgfpathmoveto{\pgfqpoint{2.752048in}{1.527778in}}%
\pgfpathlineto{\pgfqpoint{2.807604in}{1.500000in}}%
\pgfpathlineto{\pgfqpoint{2.752048in}{1.472222in}}%
\pgfusepath{stroke}%
\end{pgfscope}%
\begin{pgfscope}%
\pgftext[x=2.870833in,y=1.500000in,left,]{\rmfamily\fontsize{10.000000}{12.000000}\selectfont \(\displaystyle k_2\)}%
\end{pgfscope}%
\end{pgfpicture}%
\makeatother%
\endgroup%

\caption{Aufteilung des Fourierraums in vier Quadranten plus einen Teil für die groben Skalen. Die Quadranten $C^{(i)}$ entsprechen den Unterräumen von $L^2(\hat{\mathbb{R}}^2)$ die von $\hat \psi_{ast}^{(i)}$ reproduziert werden.}
\label{fig:quadrants}
\end{figure}

Das große Versprechen der Shearlettransformation war ja, dass sie in der Lage ist nicht nur Position, sondern auch Orientierung der Singularitäten aufzulösen. Dass dem auch so ist, ist Aussage des folgenden Satzes:

\begin{theorem}[$\mathcal{S}_f(a,s,t)$ misst $WF(f)$]
\label{thm:main_theorem}
    Sei $f \in \mathcal{S}'(C)^\vee \cap B(C)^\vee$ (wobei $\mathcal{S}(C)^\vee$ analog zu $L^2(C)^\vee$ definiert ist).
    Sei $\mathcal{D}$ die Menge der $(s, t)$ s.d. $\mathcal{S}(a,s,t)$ schnell verschwindet. Genauer

    \begin{dmath*}
        \mathcal{D} = \left\{
        (s_0, t_0) \hiderel \in \mathbb{R}^2 \times [-1,1]~ \Big| ~\textrm{ für  $(s,t)$ in einer Umgebung $U$ von } (t_0,s_0) \hiderel :\\
        |S_f (a,s,t)| = O(a^k) \textrm{ für } a \hiderel \to 0, \forall k \hiderel \in \mathbb{N}  \textrm{ mit $O(\cdot)$ gleichmäßig über } (s,t) \hiderel \in U
        \right\}
    \end{dmath*}

    Dann gilt $WF(f)^c = \mathcal{D}$
\end{theorem}


\begin{remark}
\label{rem:shearlets_no_distributions}
    Die Einschränkung, dass $f$ beschränkt ist ($f \in B(\mathbb{R}^2)$) ist gravierend und bedeutet zunächst, dass die Shearlettransformation nur bedingt geeignet ist, um Wellenfrontmengen auszurechnen. In \cref{sec:ausblick} soll aber ein Ansatz gegeben werden, wie der Beweis von \cref{thm:main_theorem} hoffentlich auf ganz $\mathcal{S}'(C)^\vee$ ausgedehnt werden kann. Des Weiteren zeigen ja auch die konkreten Rechnungen an Distributionen mit bereits bekannter Wellefrontmenge, dass die Shearlettransformation diese korrekt erkennt.
\end{remark}
Wenn wir die Wellenfrontmenge einer Distribution kennen, kennen wir auch ihren singulären Träger:

\begin{corollary}[$\mathcal{S}_f(a,s,t)$ misst $sing ~supp (\psi)$]
\label{cor:psi_ast_misst_sing_supp}
Sei $f$ wie eben und $\mathcal{R}$ die Projektion von $\mathcal{D}$ auf die  Ortskomponente. Also

\begin{equation*}
    \mathcal{R} = \pi (\mathcal{D})~~;~~~~
    \pi : (t,s) \mapsto t
\end{equation*}

Dann gilt $sing ~supp (f)^c = \mathcal{R}$
\end{corollary}
