%!TEX root = main.tex
%%%%%%%%%%%%%%%%%%%%%%%%%%%%%%%%%%%%%%%%%%%%%%%%%%%%%%%%%%%%%%%%%%%%%%%%%%%%%%%
% % Section 1
%%%%%%%%%%%%%%%%%%%%%%%%%%%%%%%%%%%%%%%%%%%%%%%%%%%%%%%%%%%%%%%%%%%%%%%%%%%%%%%
\section{Shearlets} % (fold)
\label{sec:shearlets}

\todo[color=green]{In dieser Sektion nur die wichtigsten Ergebnisse des Papers angeben, oder auch Beweise oder zumindest Beweisskizzen, damit man sieht wie alles zusammen spielt?}

\begin{proposition}[$\psi_{ast}$ fällt schnell ab]
\label{prop:shearlets_decay_rapidly}
Sei $\psi \in L^2(\mathbb{R}^2)$ ein Shearlet wie definiert und $M$ so ne Trafomatrix. Dann gilt für alle $k \in  \mathbb{N}$, dass es eine konstante $C_k$ gibt s.d. für alle $x \in \mathbb{R}^2$ gilt

\begin{dmath*}
    \left| \psi_{ast}(x) \right|
    \leq
    C_k \left| \det M \right|^{-\frac{1}{2}}\left(1+|M^{-1}(x-t)|^2\right)^{-k}
    = C_k a^{-\frac{3}{4}}\left(1+a^{-2}\left(x_1-t_1\right)^2
        + 2 a^{-2}s\left(x_1-t_1\right)\left(x_2-t_2\right)
        + a^{-1}\left(1+a^{-1}s^2\right)\left(x_2-t_2\right)^2
    \right)^{-k}
\end{dmath*}

Und insbesondere ist $C_k = \frac{15}{2}\frac{\sqrt{a} + s}{a^2}\left(\Vert \hat\psi \Vert_\infty + \Vert \Laplace^k \hat\psi \Vert_\infty\right)$

\end{proposition}

\begin{theorem}[$\mathcal{S}_f(a,s,t)$ misst $WF(f)$]
\label{thm:main_theorem}
    Sei $\mathcal{D} = \mathcal{D}_1 \cup \mathcal{D}_2$ wobei
    $\mathcal{D}_1$ = \{
        $(t_0, s_0) \in \mathbb{R}^2 \times [-1,1] \big|$
        $|\mathcal{S}_f (a, s, t)| = O(a^k)$ gleichmäßig $\forall k \in \mathbb{N}
        , \forall t \in U$ Umgebung von $(t_0, s_0)$
    \}
    und $\mathcal{D}_2$ analog für $\psi^{(v)}$

    Dann gilt $WF(f)^c = \mathcal{D}$
\end{theorem}

\todo{diesen Satz richtig hin schreiben und ordentlich setzen}
\todo{Stil und Nummerierung für Sätze, Propositionen etc. anpassen}

\begin{corollary}[WF(f) misst $sing ~supp (\psi)$]
Sei $\mathcal{R} =$ \{
    $t_0 \in \mathcal{R}^2 \big|$ $|\mathcal{S}_f(a,s,t)| = O(a^k)$
    $\forall k \in \mathbb{N}, \forall t \in U$ Umgebung von $t_0$
    \}

    Dann gilt $sing ~supp (\psi)^c = \mathcal{R}$
\end{corollary}

\begin{remark}[Träger von $\psi$]

\begin{figure}[h]
\centering
%% Creator: Matplotlib, PGF backend
%%
%% To include the figure in your LaTeX document, write
%%   \input{<filename>.pgf}
%%
%% Make sure the required packages are loaded in your preamble
%%   \usepackage{pgf}
%%
%% Figures using additional raster images can only be included by \input if
%% they are in the same directory as the main LaTeX file. For loading figures
%% from other directories you can use the `import` package
%%   \usepackage{import}
%% and then include the figures with
%%   \import{<path to file>}{<filename>.pgf}
%%
%% Matplotlib used the following preamble
%%   \usepackage[utf8x]{inputenc}
%%   \usepackage[T1]{fontenc}
%%   \usepackage{amssymb}
%%
\begingroup%
\makeatletter%
\begin{pgfpicture}%
\pgfpathrectangle{\pgfpointorigin}{\pgfqpoint{4.000000in}{2.000000in}}%
\pgfusepath{use as bounding box, clip}%
\begin{pgfscope}%
\pgfsetbuttcap%
\pgfsetmiterjoin%
\definecolor{currentfill}{rgb}{1.000000,1.000000,1.000000}%
\pgfsetfillcolor{currentfill}%
\pgfsetlinewidth{0.000000pt}%
\definecolor{currentstroke}{rgb}{1.000000,1.000000,1.000000}%
\pgfsetstrokecolor{currentstroke}%
\pgfsetdash{}{0pt}%
\pgfpathmoveto{\pgfqpoint{0.000000in}{0.000000in}}%
\pgfpathlineto{\pgfqpoint{4.000000in}{0.000000in}}%
\pgfpathlineto{\pgfqpoint{4.000000in}{2.000000in}}%
\pgfpathlineto{\pgfqpoint{0.000000in}{2.000000in}}%
\pgfpathclose%
\pgfusepath{fill}%
\end{pgfscope}%
\begin{pgfscope}%
\pgfsetbuttcap%
\pgfsetmiterjoin%
\definecolor{currentfill}{rgb}{1.000000,1.000000,1.000000}%
\pgfsetfillcolor{currentfill}%
\pgfsetlinewidth{0.000000pt}%
\definecolor{currentstroke}{rgb}{0.000000,0.000000,0.000000}%
\pgfsetstrokecolor{currentstroke}%
\pgfsetstrokeopacity{0.000000}%
\pgfsetdash{}{0pt}%
\pgfpathmoveto{\pgfqpoint{0.198611in}{0.198611in}}%
\pgfpathlineto{\pgfqpoint{3.801389in}{0.198611in}}%
\pgfpathlineto{\pgfqpoint{3.801389in}{1.801389in}}%
\pgfpathlineto{\pgfqpoint{0.198611in}{1.801389in}}%
\pgfpathclose%
\pgfusepath{fill}%
\end{pgfscope}%
\begin{pgfscope}%
\pgfpathrectangle{\pgfqpoint{0.198611in}{0.198611in}}{\pgfqpoint{3.602778in}{1.602778in}}%
\pgfusepath{clip}%
\pgfsetbuttcap%
\pgfsetmiterjoin%
\definecolor{currentfill}{rgb}{0.500000,0.500000,0.500000}%
\pgfsetfillcolor{currentfill}%
\pgfsetfillopacity{0.500000}%
\pgfsetlinewidth{0.501875pt}%
\definecolor{currentstroke}{rgb}{0.000000,0.000000,0.000000}%
\pgfsetstrokecolor{currentstroke}%
\pgfsetdash{}{0pt}%
\pgfpathmoveto{\pgfqpoint{1.954965in}{0.398958in}}%
\pgfpathlineto{\pgfqpoint{2.045035in}{0.398958in}}%
\pgfpathlineto{\pgfqpoint{2.180139in}{0.519167in}}%
\pgfpathlineto{\pgfqpoint{1.819861in}{0.519167in}}%
\pgfpathclose%
\pgfusepath{stroke,fill}%
\end{pgfscope}%
\begin{pgfscope}%
\pgfpathrectangle{\pgfqpoint{0.198611in}{0.198611in}}{\pgfqpoint{3.602778in}{1.602778in}}%
\pgfusepath{clip}%
\pgfsetbuttcap%
\pgfsetmiterjoin%
\definecolor{currentfill}{rgb}{0.500000,0.500000,0.500000}%
\pgfsetfillcolor{currentfill}%
\pgfsetfillopacity{0.500000}%
\pgfsetlinewidth{0.501875pt}%
\definecolor{currentstroke}{rgb}{0.000000,0.000000,0.000000}%
\pgfsetstrokecolor{currentstroke}%
\pgfsetdash{}{0pt}%
\pgfpathmoveto{\pgfqpoint{1.883721in}{0.626019in}}%
\pgfpathlineto{\pgfqpoint{2.116279in}{0.626019in}}%
\pgfpathlineto{\pgfqpoint{2.465117in}{1.427407in}}%
\pgfpathlineto{\pgfqpoint{1.534883in}{1.427407in}}%
\pgfpathclose%
\pgfusepath{stroke,fill}%
\end{pgfscope}%
\begin{pgfscope}%
\pgfpathrectangle{\pgfqpoint{0.198611in}{0.198611in}}{\pgfqpoint{3.602778in}{1.602778in}}%
\pgfusepath{clip}%
\pgfsetbuttcap%
\pgfsetmiterjoin%
\definecolor{currentfill}{rgb}{0.500000,0.500000,0.500000}%
\pgfsetfillcolor{currentfill}%
\pgfsetfillopacity{0.500000}%
\pgfsetlinewidth{0.501875pt}%
\definecolor{currentstroke}{rgb}{0.000000,0.000000,0.000000}%
\pgfsetstrokecolor{currentstroke}%
\pgfsetdash{}{0pt}%
\pgfpathmoveto{\pgfqpoint{2.183952in}{0.626019in}}%
\pgfpathlineto{\pgfqpoint{2.416511in}{0.626019in}}%
\pgfpathlineto{\pgfqpoint{3.666043in}{1.427407in}}%
\pgfpathlineto{\pgfqpoint{2.735809in}{1.427407in}}%
\pgfpathclose%
\pgfusepath{stroke,fill}%
\end{pgfscope}%
\begin{pgfscope}%
\pgfpathrectangle{\pgfqpoint{0.198611in}{0.198611in}}{\pgfqpoint{3.602778in}{1.602778in}}%
\pgfusepath{clip}%
\pgfsetbuttcap%
\pgfsetroundjoin%
\pgfsetlinewidth{0.501875pt}%
\definecolor{currentstroke}{rgb}{0.501961,0.501961,0.501961}%
\pgfsetstrokecolor{currentstroke}%
\pgfsetdash{{1.850000pt}{0.800000pt}}{0.000000pt}%
\pgfpathmoveto{\pgfqpoint{1.804251in}{0.184722in}}%
\pgfpathlineto{\pgfqpoint{3.636860in}{1.815278in}}%
\pgfpathlineto{\pgfqpoint{3.636860in}{1.815278in}}%
\pgfusepath{stroke}%
\end{pgfscope}%
\begin{pgfscope}%
\pgfpathrectangle{\pgfqpoint{0.198611in}{0.198611in}}{\pgfqpoint{3.602778in}{1.602778in}}%
\pgfusepath{clip}%
\pgfsetbuttcap%
\pgfsetroundjoin%
\pgfsetlinewidth{0.501875pt}%
\definecolor{currentstroke}{rgb}{0.501961,0.501961,0.501961}%
\pgfsetstrokecolor{currentstroke}%
\pgfsetdash{{1.850000pt}{0.800000pt}}{0.000000pt}%
\pgfpathmoveto{\pgfqpoint{0.363140in}{1.815278in}}%
\pgfpathlineto{\pgfqpoint{2.195749in}{0.184722in}}%
\pgfpathlineto{\pgfqpoint{2.195749in}{0.184722in}}%
\pgfusepath{stroke}%
\end{pgfscope}%
\begin{pgfscope}%
\pgfsetrectcap%
\pgfsetmiterjoin%
\pgfsetlinewidth{0.501875pt}%
\definecolor{currentstroke}{rgb}{0.000000,0.000000,0.000000}%
\pgfsetstrokecolor{currentstroke}%
\pgfsetdash{}{0pt}%
\pgfpathmoveto{\pgfqpoint{2.000000in}{0.198611in}}%
\pgfpathlineto{\pgfqpoint{2.000000in}{1.801389in}}%
\pgfusepath{stroke}%
\end{pgfscope}%
\begin{pgfscope}%
\pgfsetrectcap%
\pgfsetmiterjoin%
\pgfsetlinewidth{0.501875pt}%
\definecolor{currentstroke}{rgb}{0.000000,0.000000,0.000000}%
\pgfsetstrokecolor{currentstroke}%
\pgfsetdash{}{0pt}%
\pgfpathmoveto{\pgfqpoint{0.198611in}{0.358889in}}%
\pgfpathlineto{\pgfqpoint{3.801389in}{0.358889in}}%
\pgfusepath{stroke}%
\end{pgfscope}%
\begin{pgfscope}%
\pgfsetroundcap%
\pgfsetroundjoin%
\pgfsetlinewidth{0.501875pt}%
\definecolor{currentstroke}{rgb}{0.000000,0.000000,0.000000}%
\pgfsetstrokecolor{currentstroke}%
\pgfsetdash{}{0pt}%
\pgfpathmoveto{\pgfqpoint{1.413964in}{0.497069in}}%
\pgfpathquadraticcurveto{\pgfqpoint{1.625531in}{0.488637in}}{\pgfqpoint{1.829340in}{0.480514in}}%
\pgfusepath{stroke}%
\end{pgfscope}%
\begin{pgfscope}%
\pgfsetroundcap%
\pgfsetroundjoin%
\pgfsetlinewidth{0.501875pt}%
\definecolor{currentstroke}{rgb}{0.000000,0.000000,0.000000}%
\pgfsetstrokecolor{currentstroke}%
\pgfsetdash{}{0pt}%
\pgfpathmoveto{\pgfqpoint{1.774935in}{0.510482in}}%
\pgfpathlineto{\pgfqpoint{1.829340in}{0.480514in}}%
\pgfpathlineto{\pgfqpoint{1.772722in}{0.454971in}}%
\pgfusepath{stroke}%
\end{pgfscope}%
\begin{pgfscope}%
\pgftext[x=0.648958in,y=0.479097in,left,base]{\rmfamily\fontsize{10.000000}{12.000000}\selectfont \(\displaystyle a = 1, s = 0\)}%
\end{pgfscope}%
\begin{pgfscope}%
\pgfsetroundcap%
\pgfsetroundjoin%
\pgfsetlinewidth{0.501875pt}%
\definecolor{currentstroke}{rgb}{0.000000,0.000000,0.000000}%
\pgfsetstrokecolor{currentstroke}%
\pgfsetdash{}{0pt}%
\pgfpathmoveto{\pgfqpoint{1.141127in}{1.096480in}}%
\pgfpathquadraticcurveto{\pgfqpoint{1.354038in}{1.088809in}}{\pgfqpoint{1.559190in}{1.081418in}}%
\pgfusepath{stroke}%
\end{pgfscope}%
\begin{pgfscope}%
\pgfsetroundcap%
\pgfsetroundjoin%
\pgfsetlinewidth{0.501875pt}%
\definecolor{currentstroke}{rgb}{0.000000,0.000000,0.000000}%
\pgfsetstrokecolor{currentstroke}%
\pgfsetdash{}{0pt}%
\pgfpathmoveto{\pgfqpoint{1.504671in}{1.111178in}}%
\pgfpathlineto{\pgfqpoint{1.559190in}{1.081418in}}%
\pgfpathlineto{\pgfqpoint{1.502670in}{1.055658in}}%
\pgfusepath{stroke}%
\end{pgfscope}%
\begin{pgfscope}%
\pgftext[x=0.198611in,y=1.080139in,left,base]{\rmfamily\fontsize{10.000000}{12.000000}\selectfont \(\displaystyle a = 0.15, s = 0\)}%
\end{pgfscope}%
\begin{pgfscope}%
\pgfsetroundcap%
\pgfsetroundjoin%
\pgfsetlinewidth{0.501875pt}%
\definecolor{currentstroke}{rgb}{0.000000,0.000000,0.000000}%
\pgfsetstrokecolor{currentstroke}%
\pgfsetdash{}{0pt}%
\pgfpathmoveto{\pgfqpoint{3.348730in}{0.656645in}}%
\pgfpathquadraticcurveto{\pgfqpoint{3.136663in}{0.781233in}}{\pgfqpoint{2.931289in}{0.901887in}}%
\pgfusepath{stroke}%
\end{pgfscope}%
\begin{pgfscope}%
\pgfsetroundcap%
\pgfsetroundjoin%
\pgfsetlinewidth{0.501875pt}%
\definecolor{currentstroke}{rgb}{0.000000,0.000000,0.000000}%
\pgfsetstrokecolor{currentstroke}%
\pgfsetdash{}{0pt}%
\pgfpathmoveto{\pgfqpoint{2.965119in}{0.849795in}}%
\pgfpathlineto{\pgfqpoint{2.931289in}{0.901887in}}%
\pgfpathlineto{\pgfqpoint{2.993261in}{0.897696in}}%
\pgfusepath{stroke}%
\end{pgfscope}%
\begin{pgfscope}%
\pgftext[x=3.080833in,y=0.519167in,left,base]{\rmfamily\fontsize{10.000000}{12.000000}\selectfont \(\displaystyle a = 0.15, s = 1\)}%
\end{pgfscope}%
\begin{pgfscope}%
\pgfsetroundcap%
\pgfsetroundjoin%
\pgfsetlinewidth{0.501875pt}%
\definecolor{currentstroke}{rgb}{0.000000,0.000000,0.000000}%
\pgfsetstrokecolor{currentstroke}%
\pgfsetdash{}{0pt}%
\pgfpathmoveto{\pgfqpoint{2.000000in}{1.807510in}}%
\pgfpathquadraticcurveto{\pgfqpoint{2.000000in}{1.808331in}}{\pgfqpoint{2.000000in}{1.801389in}}%
\pgfusepath{stroke}%
\end{pgfscope}%
\begin{pgfscope}%
\pgfsetroundcap%
\pgfsetroundjoin%
\pgfsetlinewidth{0.501875pt}%
\definecolor{currentstroke}{rgb}{0.000000,0.000000,0.000000}%
\pgfsetstrokecolor{currentstroke}%
\pgfsetdash{}{0pt}%
\pgfpathmoveto{\pgfqpoint{1.972222in}{1.751954in}}%
\pgfpathlineto{\pgfqpoint{2.000000in}{1.807510in}}%
\pgfpathlineto{\pgfqpoint{2.027778in}{1.751954in}}%
\pgfusepath{stroke}%
\end{pgfscope}%
\begin{pgfscope}%
\pgftext[x=2.000000in,y=1.870833in,,bottom]{\rmfamily\fontsize{10.000000}{12.000000}\selectfont \(\displaystyle \omega\)}%
\end{pgfscope}%
\begin{pgfscope}%
\pgfsetroundcap%
\pgfsetroundjoin%
\pgfsetlinewidth{0.501875pt}%
\definecolor{currentstroke}{rgb}{0.000000,0.000000,0.000000}%
\pgfsetstrokecolor{currentstroke}%
\pgfsetdash{}{0pt}%
\pgfpathmoveto{\pgfqpoint{3.807488in}{0.358889in}}%
\pgfpathquadraticcurveto{\pgfqpoint{3.808320in}{0.358889in}}{\pgfqpoint{3.801389in}{0.358889in}}%
\pgfusepath{stroke}%
\end{pgfscope}%
\begin{pgfscope}%
\pgfsetroundcap%
\pgfsetroundjoin%
\pgfsetlinewidth{0.501875pt}%
\definecolor{currentstroke}{rgb}{0.000000,0.000000,0.000000}%
\pgfsetstrokecolor{currentstroke}%
\pgfsetdash{}{0pt}%
\pgfpathmoveto{\pgfqpoint{3.751932in}{0.386667in}}%
\pgfpathlineto{\pgfqpoint{3.807488in}{0.358889in}}%
\pgfpathlineto{\pgfqpoint{3.751932in}{0.331111in}}%
\pgfusepath{stroke}%
\end{pgfscope}%
\begin{pgfscope}%
\pgftext[x=3.870833in,y=0.358889in,left,]{\rmfamily\fontsize{10.000000}{12.000000}\selectfont \(\displaystyle k\)}%
\end{pgfscope}%
\end{pgfpicture}%
\makeatother%
\endgroup%

\label{fig:supp_psi_hat}
\caption{Der Träger von $\hat \psi_{ast}$ für verschiedene $a, s$. Man sieht gut,
wie $supp (\hat \psi_{ast})$ für kleinere $a$ in immer spitzeren Kegeln liegt.}
\end{figure}

\label{cor:psi_hat}
Im Fourierraum ist $\hat{\psi}_{ast}$ gegeben durch

\begin{equation}
    \hat \psi_{ast}{(\xi_1, \xi_2)} = a^{\frac{3}{4}}e^{-i\xi \cdot t}\hat\psi_1(a \xi_1) \hat\psi_{2}\left(a^{-\frac{1}{2}}\left(\frac{\xi_2}{\xi_1}-s\right)\right)
\end{equation}

und es gilt

\begin{equation}
\label{eq:supp_psi}
    supp(\hat \psi) \subset \left\{\xi \in  \hat{\mathbb{R}}^2 ~\Big| ~|\xi_1| \in \left[\frac{1}{2 a} , \frac{2}{a}\right], \left|\frac{\xi_2}{\xi_1} - s\right| \leq \sqrt{a} \right\}
\end{equation}

\end{remark}



% section allgemeines_gelaber_über_shearlets (end)


%%%%%%%%%%%%%%%%%%%%%%%%%%%%%%%%%%%%%%%%%%%%%%%%%%%%%%%%%%%%%%%%%%%%%%%%%%%%%%%%
% % Section 2
%%%%%%%%%%%%%%%%%%%%%%%%%%%%%%%%%%%%%%%%%%%%%%%%%%%%%%%%%%%%%%%%%%%%%%%%%%%%%%%%
\section{\texorpdfstring{Zwei nützliche Substitionen für  $\left<\psi_{ast}, f\right>$}{zwei nützliche Substitutionen}}
\label{sec:substitutionen}

\todo[color=green]{mit $(\omega, k)$ als Variablennamen arbeiten, um zum Rest des Textes zu passen, oder mit $(\xi_1, \xi_2)$ um zu \textcite{Kutyniok2008} zu passen?}

Zunächst werden wir zwei verschiedene Ausdrücke für $\left<\psi_{ast}, f\right>$
im Fourierraum herleiten, welche fast immer Ausgangspunkt für unsere Abschätzungen sein werden.

Sei also $\psi$ ein Shearlet wie in \cref{cor:psi_hat}. Sei $f$ die zu
analysierende fouriertransformierbare Funktion (oder Distribution) in
$\mathcal{D}' (\mathbb{R}^2)$. Dann ist $\mathcal{S}_f (ast)$ gegeben durch

\begin{align*}
\left< \psi_{ast}, f \right> &= \left<\hat\psi_{ast}, \hat f\right> \\
 &= \int a^{\frac{3}{4}} e^{-i \xi \cdot t} \hat \psi_1(a \xi_1)
    \hat \psi_2 \left(a^{-\frac{1}{2}} \left(\frac{\xi_2}{\xi_1} - s\right)\right)
    \hat f (\xi) \d \xi
\end{align*}

\todo{entscheiden, was mit dem fehlenden Faktor $\frac{1}{(2 \pi)^n}$ geschieht}
und nach "`entscheren"' und "`deskalieren"', also der Substitution

\begin{equation*}
\begin{aligned}[c]
a \xi_1 &= k_1\\
a^{-\frac{1}{2}} \left(\frac{\xi_2}{\xi_1} - s\right) &=\frac{k_2}{k_1}\\
\end{aligned}
\qquad\Longleftrightarrow\qquad
\begin{aligned}[c]
\xi_1 &= \frac{k_1}{a}\\
\xi_2 &= \frac{k_1 s}{a} + a^{-\frac{1}{2}} k_2\\
\end{aligned}
\end{equation*}

\begin{equation*}
\Rightarrow
\d \xi_1 \d \xi_2 = a^{-\frac{3}{2}} \d k_1 \d k_2
\end{equation*}

ergibt sich folgendes für $\left<\psi_{ast}, f\right>$:

\todo{\texttt{owntag} fixen}

\begin{align}
    \left\langle\psi_{ast},f\right\rangle
    &=  \left\langle\hat\psi_{ast},\hat f\right\rangle \nonumber \\
    &=  \iint a^{-\frac{3}{4}}~\hat \psi_1(k_1) ~\hat \psi_2 \left(\tfrac{k_2}{k_1}\right)
    ~\hat f \left(\tfrac{k_1}{a}, \tfrac{k_1 s}{a} + \tfrac{k_2}{\sqrt{a}}\right)
    ~e^{-i\frac{k_1}{a}(t_1+t_2 s) - i \frac{k_2 t_2}{\sqrt a}}
    \d k_1 \d k_2
\owntag[substitution1]{Substitution 1}
\end{align}

\todo{herausfinden, wie die Gleichungen auch Kapitelnummern erhalten}

Alternativ kann auch folgende Substitution

\begin{equation*}
\begin{aligned}[c]
a \xi_1 &= k_1\\
a^{-\frac{1}{2}} \left(\frac{\xi_2}{\xi_1} - s\right) &= k_2\\
\end{aligned}
\qquad\Longleftrightarrow\qquad
\begin{aligned}[c]
\xi_1 &= \frac{k_1}{a}\\
\xi_2 &= \left( a^{\frac{1}{2}} k_2 +s \right) \frac{k_1}{a}\\
\end{aligned}
\end{equation*}

\begin{equation*}
\Rightarrow
\d \xi_1 \d \xi_2 = a^{-\frac{3}{2}} k_1 \d k_1 \d k_2
\end{equation*}

gewählt werden, wodurch alle Parameter $(a,s,t)$ aus den Argumenten von $\hat\psi_1, \hat\psi_2$
verschwinden und sich

\begin{align}
    \left<\psi_{ast},f\right>
    =  \iint a^{-\frac{3}{4}}~ k_1~ \hat \psi_1(k_1)~ \hat \psi_2 (k_2)~
    \hat f \left(\tfrac{k_1}{a}, k_1 \left(a^{-\frac{1}{2}}k_2 + s a^{-1}\right)\right)
    ~e^{-i k_1 \left(\frac{t_1+s t_2}{a} + \frac{k_2 t_2}{\sqrt{a}}\right)}
    \d k_1 \d k_2
\owntag[substitution2]{Substitution 2}
\end{align}

ergibt. Dabei ist zu beachten, dass diese Substitution zulässig ist, obwohl sie
die Orientierung \emph{nicht} erhält und \emph{keine} Bijektion ist. Aber
der kritische Bereich, nämlich $\xi_1 = 0$, liegt nicht im Träger von $\rwhat{\psi}$.

\todo{Grafik basteln, die $supp ~\psi$ vor und nach der Substitution zeigt.}
