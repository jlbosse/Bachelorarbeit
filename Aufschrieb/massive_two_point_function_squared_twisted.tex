% !TEX root = main.tex
% !TEX spellcheck=de_DE
%%%%%%%%%%%%%%%%%%%%%%%%%%%%%%%%%%%%%%%%%%%%%%%%%%%%%%%%%%%%%%%%%%%%%%%%%%%%%%%
% % Berechnen der Wellenfrontmenge von Delta_m_twisted
%%%%%%%%%%%%%%%%%%%%%%%%%%%%%%%%%%%%%%%%%%%%%%%%%%%%%%%%%%%%%%%%%%%%%%%%%%%%%%%

\section{\texorpdfstring{Die Wellenfrontmenge von $\Delta_m^{\star 2}$}
         {wellenfrontmenge von delta_m2_twisted}} % (fold)
\label{sec:die_wellenfrontmenge_von_delta_m2_twisted}

\todo{herausfinden, was für ein Symbol man für dieses verdrehte Produkt verwendet}

Hier gehört wohl erstmal Text hin, was diese perverse Windung = (twisted convolution) überhaupt zu bedeuten hat. Leider weiß ich das selber noch nicht.


Die perverse Windung zweier Funktionen ist definiert wie folgt:

\todo{twisted convolution definieren, oder twisted product definieren und zeigen, dass dann der Faltungssatz zu dem mit der twisted convolution wird?}

\begin{definition}[perverse Windung]
\label{def:twisted_convolution}
    Seien $f,g \in $ "`passender Funktionen/Distributionenraum'". Sei $\Omega \in \mathbb{R}^{n \times n}$ eine symplektische Matrix. Dann ist die perverse Windung $(f \circledast g) (x)$ definiert als

    \begin{equation}
        (f \circledast g) \,(x) \coloneqq
        \int f(y) g(x-y)e^{\frac{i}{2} \Omega(x,y)} \d y
    \end{equation}

    Die perverse Windung ist also einfach die gewöhnliche Faltung, die noch mit einem Ortsabhängigen Phasenfaktor verziert wurde.
\end{definition}



\todo{Gibt es schon eine Übersetzung für twisted convolution?}

Bevor wir uns aber der Wellenfrontmenge widmen können, brauchen wir einen Ausdruck für die Fouriertransformierte $\rwhat{\Delta}_m^{\circledast 2}$ von $\Delta_m^{\star 2}$.

\subsection{\texorpdfstring{$\hat\Delta_m^{\circledast 2}$ berechnen}
            {delta_m2_twisted berechnen}} % (fold)
\label{sec:delta_m2_twisted_berechnen}

Sammeln wir zunächst einmal die Zutaten, die wir für die perverse Windung der massiven Zweipunktfunktion mit sich selber brauchen:

\begin{dgroup}
    \begin{dmath*}
        \rwhat{\Delta}_m = \delta(\omega^2-k^w-m^2)\Theta(\omega)\\
        \textrm{die Fouriertransformierte der massiven Zweipunktfunktion}
    \end{dmath*}
    \begin{dmath*}
        \Omega = \begin{pmatrix}
            0 & 1 \\ -1 & 1
        \end{pmatrix}
        \\ \textrm{die kanonische symplektische Matrix auf } \mathbb{R}^n
    \end{dmath*}
    \label{eq:material_fuer_delta_m2_twisted}
\end{dgroup}

mit der Definition aus \ref{def:twisted_convolution} und \eqref{eq:material_fuer_delta_m2_twisted} erhalten wir also

\begin{dmath}
    \rwhat{\Delta}_m^{\circledast 2} (\omega, k)
    = \int
    \delta(\omega^{\prime 2}-k^{\prime 2}-m^2)
    \delta((\omega' - \omega)^2 - (k-k')^2 -m^2)
    \cdot
    \Theta(\omega') \Theta(\omega - \omega')
    e^{\frac{i}{2}(\omega'k-\omega k')}
    \d \omega' \d k'
\end{dmath}

und damit das selbe Integral wie in \eqref{eq:mass_shell_convolution} bis auf einen zusätzlichen Phasenfaktor. Nachdem wir gezeigt haben, dass auch dieser Lorenz-Invariant ist, können wir das Integral mit dem selben Trick wie in Abschnitt \ref{sec:delta_m2_berechnen} berechnen.

\begin{proposition}[$\Omega_{std}$ ist Lorenz-invariant für $n=2$]
\label{prop:omega_ist_lorenz_invariant}
    $\Omega_{std}$ ist Lorenz-invariant für $n=2$
\\[1em]
\emph{Beweis}\\
    Eine einfache Rechnung zeigt
    \begin{dmath*}
        \begin{pmatrix}
            \cosh \beta & \sinh \beta \\ -\sinh \beta & \cosh \beta
        \end{pmatrix}
        \begin{pmatrix}
            0 & 1 \\ -1 & 0
        \end{pmatrix}
        \begin{pmatrix}
            \cosh \beta & -\sinh \beta \\ \sinh \beta & \cosh \beta
        \end{pmatrix}
        =
        \begin{pmatrix}
            0 & 1 \\ -1 & 0
        \end{pmatrix}
    \end{dmath*}
\end{proposition}

Mit Proposition \ref{prop:omega_ist_lorenz_invariant} ist $\rwhat{\Delta}_m^{\circledast 2}$ Lorenz-Invariant und es reicht aus $\rwhat{\Delta}_m^{\circledast 2} (\omega, 0)$ zu berechnen.

Die beiden Kreuzungspunkte der $\delta$-Distributionen liegen bei (vgl. Abb. \ref{fig:mass_shell_convolution})

\begin{equation*}
    \left(\omega'_0,k'_{0\pm}\right) = \left(\frac{\omega}{2}, \pm \sqrt{\left(\frac{\omega}{2}\right)^2-m^2}\right)
\end{equation*}


Die "`Fläche"' der Kreuzungspunkte der $\delta$-Distributionen wurde in
Abschnitt \ref{sec:delta_m2_berechnen} berechnet und ist

\begin{equation*}
A = \frac{\omega^2-3m^2}{\omega \sqrt{\omega^2-4m^2}}
\end{equation*}

Der Phasenfaktor nimmt bei den Kreuzungspunkten folgende Werte an:
\begin{dmath*}
    e^{\frac{i}{2}\Omega \left((\omega, k),(\omega'_0,k'_{0\pm})\right)}
    =
    e^{\pm \frac{i}{2}\left(-\omega^2\sqrt{\frac{1}{4}-\frac{m^2}{\omega^2}}\right)}
\end{dmath*}


Kombinieren wir also die vorhergehenden Resultate erhalten wir

\begin{dmath*}
    \rwhat{\Delta}_m^{\circledast 2} (\omega, 0)
    =
    A e^{\frac{i}{2}\Omega \left((\omega, k),(\omega'_0,k'_{0+})\right)}
    + A e^{\frac{i}{2}\Omega \left((\omega, k),(\omega'_0,k'_{0-})\right)}
    =
    \frac{\omega^2-3m^2}{\omega \sqrt{\omega^2-4m^2}}
    \left\{
        e^{-\frac{i}{2}\omega^2\sqrt{\frac{1}{4}-\frac{m^2}{\omega^2}}}
      + e^{\frac{i}{2}\omega^2\sqrt{\frac{1}{4}-\frac{m^2}{\omega^2}}}
    \right\}
    =
    2 \frac{\omega^2 -3m^2}{\omega \sqrt{\omega^2-4m^2}}
    \cos \left(\varphi(\omega^2)\right)
\end{dmath*}

wobei im letzten Schritt noch implizit $\varphi(\omega^2)$ definiert wurde.
Und mit Lorenz-Invarianz erhalten wir schließlich

\begin{dmath}
    \rwhat{\Delta}_m^{\circledast 2} (\omega, k)
    =
    \rwhat{\Delta}_m^{\circledast 2} (\sqrt{\omega^2-k^2}, 0)
    =
    2\frac{\omega^2-k^2-3m^2}{\sqrt{\omega^2-k^2} \sqrt{\omega^2-k^2-4m^2}}
    \cos \left(\frac{k^2-\omega^2}{2}
    \sqrt{\frac{1}{4}+\frac{m^2}{k^2-\omega^2}}
    \right)
    =
    \rwhat{\Delta}_m^{* 2}(\omega, k) \cos (\varphi(\omega^2-k^2))
\end{dmath}

\subsection{
\texorpdfstring{\dots und nun zur Wellenfrontmenge von ${\hat\Delta}_m^{\circledast 2}$}{und nun zur wellenfrontmenge von delta_m2_twisted}} % (fold)
\label{sec:dots_und_nun_zur_wellenfrontmenge_von_delta_m2_twisted}

\begin{figure}
    \centering
    \begin{minipage}{0.55\textwidth}
        \centering
        \resizebox{\textwidth}{!}{%% Creator: Matplotlib, PGF backend
%%
%% To include the figure in your LaTeX document, write
%%   \input{<filename>.pgf}
%%
%% Make sure the required packages are loaded in your preamble
%%   \usepackage{pgf}
%%
%% Figures using additional raster images can only be included by \input if
%% they are in the same directory as the main LaTeX file. For loading figures
%% from other directories you can use the `import` package
%%   \usepackage{import}
%% and then include the figures with
%%   \import{<path to file>}{<filename>.pgf}
%%
%% Matplotlib used the following preamble
%%   \usepackage[utf8x]{inputenc}
%%   \usepackage[T1]{fontenc}
%%   \usepackage{amssymb}
%%
\begingroup%
\makeatletter%
\begin{pgfpicture}%
\pgfpathrectangle{\pgfpointorigin}{\pgfqpoint{10.000000in}{5.500000in}}%
\pgfusepath{use as bounding box, clip}%
\begin{pgfscope}%
\pgfsetbuttcap%
\pgfsetmiterjoin%
\definecolor{currentfill}{rgb}{1.000000,1.000000,1.000000}%
\pgfsetfillcolor{currentfill}%
\pgfsetlinewidth{0.000000pt}%
\definecolor{currentstroke}{rgb}{1.000000,1.000000,1.000000}%
\pgfsetstrokecolor{currentstroke}%
\pgfsetdash{}{0pt}%
\pgfpathmoveto{\pgfqpoint{0.000000in}{0.000000in}}%
\pgfpathlineto{\pgfqpoint{10.000000in}{0.000000in}}%
\pgfpathlineto{\pgfqpoint{10.000000in}{5.500000in}}%
\pgfpathlineto{\pgfqpoint{0.000000in}{5.500000in}}%
\pgfpathclose%
\pgfusepath{fill}%
\end{pgfscope}%
\begin{pgfscope}%
\pgfsetbuttcap%
\pgfsetmiterjoin%
\definecolor{currentfill}{rgb}{1.000000,1.000000,1.000000}%
\pgfsetfillcolor{currentfill}%
\pgfsetlinewidth{0.000000pt}%
\definecolor{currentstroke}{rgb}{0.000000,0.000000,0.000000}%
\pgfsetstrokecolor{currentstroke}%
\pgfsetstrokeopacity{0.000000}%
\pgfsetdash{}{0pt}%
\pgfpathmoveto{\pgfqpoint{0.198611in}{0.723208in}}%
\pgfpathlineto{\pgfqpoint{7.919722in}{0.723208in}}%
\pgfpathlineto{\pgfqpoint{7.919722in}{4.776792in}}%
\pgfpathlineto{\pgfqpoint{0.198611in}{4.776792in}}%
\pgfpathclose%
\pgfusepath{fill}%
\end{pgfscope}%
\begin{pgfscope}%
\pgfpathrectangle{\pgfqpoint{0.198611in}{0.723208in}}{\pgfqpoint{7.721111in}{4.053583in}} %
\pgfusepath{clip}%
\pgfsys@transformshift{0.198611in}{0.723208in}%
\pgftext[left,bottom]{\pgfimage[interpolate=true,width=7.722222in,height=4.055556in]{delta_m2_twisted-img0.png}}%
\end{pgfscope}%
\begin{pgfscope}%
\pgfpathrectangle{\pgfqpoint{0.198611in}{0.723208in}}{\pgfqpoint{7.721111in}{4.053583in}} %
\pgfusepath{clip}%
\pgfsetrectcap%
\pgfsetroundjoin%
\pgfsetlinewidth{0.501875pt}%
\definecolor{currentstroke}{rgb}{0.894118,0.101961,0.109804}%
\pgfsetstrokecolor{currentstroke}%
\pgfsetdash{}{0pt}%
\pgfpathmoveto{\pgfqpoint{0.204004in}{4.790681in}}%
\pgfpathlineto{\pgfqpoint{1.303835in}{3.698482in}}%
\pgfpathlineto{\pgfqpoint{1.991702in}{3.019436in}}%
\pgfpathlineto{\pgfqpoint{2.447704in}{2.573297in}}%
\pgfpathlineto{\pgfqpoint{2.772315in}{2.259749in}}%
\pgfpathlineto{\pgfqpoint{3.011909in}{2.032385in}}%
\pgfpathlineto{\pgfqpoint{3.197401in}{1.860524in}}%
\pgfpathlineto{\pgfqpoint{3.336520in}{1.735539in}}%
\pgfpathlineto{\pgfqpoint{3.452453in}{1.635361in}}%
\pgfpathlineto{\pgfqpoint{3.545199in}{1.559044in}}%
\pgfpathlineto{\pgfqpoint{3.622487in}{1.499098in}}%
\pgfpathlineto{\pgfqpoint{3.692047in}{1.448980in}}%
\pgfpathlineto{\pgfqpoint{3.753877in}{1.408415in}}%
\pgfpathlineto{\pgfqpoint{3.807979in}{1.376816in}}%
\pgfpathlineto{\pgfqpoint{3.854352in}{1.353258in}}%
\pgfpathlineto{\pgfqpoint{3.900725in}{1.333540in}}%
\pgfpathlineto{\pgfqpoint{3.939370in}{1.320452in}}%
\pgfpathlineto{\pgfqpoint{3.978014in}{1.310729in}}%
\pgfpathlineto{\pgfqpoint{4.016658in}{1.304625in}}%
\pgfpathlineto{\pgfqpoint{4.055302in}{1.302311in}}%
\pgfpathlineto{\pgfqpoint{4.086218in}{1.303238in}}%
\pgfpathlineto{\pgfqpoint{4.124862in}{1.307841in}}%
\pgfpathlineto{\pgfqpoint{4.163506in}{1.316143in}}%
\pgfpathlineto{\pgfqpoint{4.202150in}{1.327919in}}%
\pgfpathlineto{\pgfqpoint{4.240794in}{1.342883in}}%
\pgfpathlineto{\pgfqpoint{4.287167in}{1.364592in}}%
\pgfpathlineto{\pgfqpoint{4.333540in}{1.389860in}}%
\pgfpathlineto{\pgfqpoint{4.387642in}{1.423124in}}%
\pgfpathlineto{\pgfqpoint{4.449473in}{1.465215in}}%
\pgfpathlineto{\pgfqpoint{4.519033in}{1.516666in}}%
\pgfpathlineto{\pgfqpoint{4.596321in}{1.577730in}}%
\pgfpathlineto{\pgfqpoint{4.689067in}{1.655028in}}%
\pgfpathlineto{\pgfqpoint{4.805000in}{1.756061in}}%
\pgfpathlineto{\pgfqpoint{4.944119in}{1.881730in}}%
\pgfpathlineto{\pgfqpoint{5.114153in}{2.039640in}}%
\pgfpathlineto{\pgfqpoint{5.330561in}{2.244951in}}%
\pgfpathlineto{\pgfqpoint{5.616528in}{2.520734in}}%
\pgfpathlineto{\pgfqpoint{6.002970in}{2.898006in}}%
\pgfpathlineto{\pgfqpoint{6.536260in}{3.423232in}}%
\pgfpathlineto{\pgfqpoint{7.309144in}{4.189062in}}%
\pgfpathlineto{\pgfqpoint{7.914330in}{4.790681in}}%
\pgfpathlineto{\pgfqpoint{7.914330in}{4.790681in}}%
\pgfusepath{stroke}%
\end{pgfscope}%
\begin{pgfscope}%
\pgfpathrectangle{\pgfqpoint{0.198611in}{0.723208in}}{\pgfqpoint{7.721111in}{4.053583in}} %
\pgfusepath{clip}%
\pgfsetbuttcap%
\pgfsetroundjoin%
\pgfsetlinewidth{0.501875pt}%
\definecolor{currentstroke}{rgb}{0.501961,0.501961,0.501961}%
\pgfsetstrokecolor{currentstroke}%
\pgfsetdash{{1.850000pt}{0.800000pt}}{0.000000pt}%
\pgfpathmoveto{\pgfqpoint{3.852250in}{0.709319in}}%
\pgfpathlineto{\pgfqpoint{7.919722in}{4.776792in}}%
\pgfpathlineto{\pgfqpoint{7.919722in}{4.776792in}}%
\pgfusepath{stroke}%
\end{pgfscope}%
\begin{pgfscope}%
\pgfpathrectangle{\pgfqpoint{0.198611in}{0.723208in}}{\pgfqpoint{7.721111in}{4.053583in}} %
\pgfusepath{clip}%
\pgfsetbuttcap%
\pgfsetroundjoin%
\pgfsetlinewidth{0.501875pt}%
\definecolor{currentstroke}{rgb}{0.501961,0.501961,0.501961}%
\pgfsetstrokecolor{currentstroke}%
\pgfsetdash{{1.850000pt}{0.800000pt}}{0.000000pt}%
\pgfpathmoveto{\pgfqpoint{0.198611in}{4.776792in}}%
\pgfpathlineto{\pgfqpoint{4.266083in}{0.709319in}}%
\pgfpathlineto{\pgfqpoint{4.266083in}{0.709319in}}%
\pgfusepath{stroke}%
\end{pgfscope}%
\begin{pgfscope}%
\pgfsetrectcap%
\pgfsetmiterjoin%
\pgfsetlinewidth{0.501875pt}%
\definecolor{currentstroke}{rgb}{0.000000,0.000000,0.000000}%
\pgfsetstrokecolor{currentstroke}%
\pgfsetdash{}{0pt}%
\pgfpathmoveto{\pgfqpoint{4.059167in}{0.723208in}}%
\pgfpathlineto{\pgfqpoint{4.059167in}{4.776792in}}%
\pgfusepath{stroke}%
\end{pgfscope}%
\begin{pgfscope}%
\pgfsetrectcap%
\pgfsetmiterjoin%
\pgfsetlinewidth{0.501875pt}%
\definecolor{currentstroke}{rgb}{0.000000,0.000000,0.000000}%
\pgfsetstrokecolor{currentstroke}%
\pgfsetdash{}{0pt}%
\pgfpathmoveto{\pgfqpoint{0.198611in}{0.916236in}}%
\pgfpathlineto{\pgfqpoint{7.919722in}{0.916236in}}%
\pgfusepath{stroke}%
\end{pgfscope}%
\begin{pgfscope}%
\pgfsetroundcap%
\pgfsetroundjoin%
\pgfsetlinewidth{0.501875pt}%
\definecolor{currentstroke}{rgb}{0.000000,0.000000,0.000000}%
\pgfsetstrokecolor{currentstroke}%
\pgfsetdash{}{0pt}%
\pgfpathmoveto{\pgfqpoint{4.059167in}{4.782913in}}%
\pgfpathquadraticcurveto{\pgfqpoint{4.059167in}{4.783734in}}{\pgfqpoint{4.059167in}{4.776792in}}%
\pgfusepath{stroke}%
\end{pgfscope}%
\begin{pgfscope}%
\pgfsetroundcap%
\pgfsetroundjoin%
\pgfsetlinewidth{0.501875pt}%
\definecolor{currentstroke}{rgb}{0.000000,0.000000,0.000000}%
\pgfsetstrokecolor{currentstroke}%
\pgfsetdash{}{0pt}%
\pgfpathmoveto{\pgfqpoint{4.031389in}{4.727357in}}%
\pgfpathlineto{\pgfqpoint{4.059167in}{4.782913in}}%
\pgfpathlineto{\pgfqpoint{4.086944in}{4.727357in}}%
\pgfusepath{stroke}%
\end{pgfscope}%
\begin{pgfscope}%
\pgftext[x=4.059167in,y=4.846236in,,bottom]{\rmfamily\fontsize{10.000000}{12.000000}\selectfont \(\displaystyle \omega\)}%
\end{pgfscope}%
\begin{pgfscope}%
\pgfsetroundcap%
\pgfsetroundjoin%
\pgfsetlinewidth{0.501875pt}%
\definecolor{currentstroke}{rgb}{0.000000,0.000000,0.000000}%
\pgfsetstrokecolor{currentstroke}%
\pgfsetdash{}{0pt}%
\pgfpathmoveto{\pgfqpoint{7.925821in}{0.916236in}}%
\pgfpathquadraticcurveto{\pgfqpoint{7.926654in}{0.916236in}}{\pgfqpoint{7.919722in}{0.916236in}}%
\pgfusepath{stroke}%
\end{pgfscope}%
\begin{pgfscope}%
\pgfsetroundcap%
\pgfsetroundjoin%
\pgfsetlinewidth{0.501875pt}%
\definecolor{currentstroke}{rgb}{0.000000,0.000000,0.000000}%
\pgfsetstrokecolor{currentstroke}%
\pgfsetdash{}{0pt}%
\pgfpathmoveto{\pgfqpoint{7.870265in}{0.944014in}}%
\pgfpathlineto{\pgfqpoint{7.925821in}{0.916236in}}%
\pgfpathlineto{\pgfqpoint{7.870265in}{0.888458in}}%
\pgfusepath{stroke}%
\end{pgfscope}%
\begin{pgfscope}%
\pgftext[x=7.989167in,y=0.916236in,left,]{\rmfamily\fontsize{10.000000}{12.000000}\selectfont \(\displaystyle k\)}%
\end{pgfscope}%
\begin{pgfscope}%
\pgfpathrectangle{\pgfqpoint{8.402292in}{0.930000in}}{\pgfqpoint{0.173333in}{3.640000in}} %
\pgfusepath{clip}%
\pgfsetbuttcap%
\pgfsetmiterjoin%
\definecolor{currentfill}{rgb}{1.000000,1.000000,1.000000}%
\pgfsetfillcolor{currentfill}%
\pgfsetlinewidth{0.010037pt}%
\definecolor{currentstroke}{rgb}{1.000000,1.000000,1.000000}%
\pgfsetstrokecolor{currentstroke}%
\pgfsetdash{}{0pt}%
\pgfpathmoveto{\pgfqpoint{8.402292in}{0.930000in}}%
\pgfpathlineto{\pgfqpoint{8.402292in}{0.943542in}}%
\pgfpathlineto{\pgfqpoint{8.402292in}{4.396667in}}%
\pgfpathlineto{\pgfqpoint{8.488958in}{4.570000in}}%
\pgfpathlineto{\pgfqpoint{8.488958in}{4.570000in}}%
\pgfpathlineto{\pgfqpoint{8.575625in}{4.396667in}}%
\pgfpathlineto{\pgfqpoint{8.575625in}{0.943542in}}%
\pgfpathlineto{\pgfqpoint{8.575625in}{0.930000in}}%
\pgfpathclose%
\pgfusepath{stroke,fill}%
\end{pgfscope}%
\begin{pgfscope}%
\pgfsys@transformshift{8.402778in}{0.930556in}%
\pgftext[left,bottom]{\pgfimage[interpolate=true,width=0.166667in,height=3.638889in]{delta_m2_twisted-img1.png}}%
\end{pgfscope}%
\begin{pgfscope}%
\pgfsetbuttcap%
\pgfsetroundjoin%
\definecolor{currentfill}{rgb}{0.000000,0.000000,0.000000}%
\pgfsetfillcolor{currentfill}%
\pgfsetlinewidth{0.803000pt}%
\definecolor{currentstroke}{rgb}{0.000000,0.000000,0.000000}%
\pgfsetstrokecolor{currentstroke}%
\pgfsetdash{}{0pt}%
\pgfsys@defobject{currentmarker}{\pgfqpoint{0.000000in}{0.000000in}}{\pgfqpoint{0.048611in}{0.000000in}}{%
\pgfpathmoveto{\pgfqpoint{0.000000in}{0.000000in}}%
\pgfpathlineto{\pgfqpoint{0.048611in}{0.000000in}}%
\pgfusepath{stroke,fill}%
}%
\begin{pgfscope}%
\pgfsys@transformshift{8.575625in}{0.930000in}%
\pgfsys@useobject{currentmarker}{}%
\end{pgfscope}%
\end{pgfscope}%
\begin{pgfscope}%
\pgftext[x=8.672847in,y=0.882172in,left,base]{\rmfamily\fontsize{10.000000}{12.000000}\selectfont \(\displaystyle -4\)}%
\end{pgfscope}%
\begin{pgfscope}%
\pgfsetbuttcap%
\pgfsetroundjoin%
\definecolor{currentfill}{rgb}{0.000000,0.000000,0.000000}%
\pgfsetfillcolor{currentfill}%
\pgfsetlinewidth{0.803000pt}%
\definecolor{currentstroke}{rgb}{0.000000,0.000000,0.000000}%
\pgfsetstrokecolor{currentstroke}%
\pgfsetdash{}{0pt}%
\pgfsys@defobject{currentmarker}{\pgfqpoint{0.000000in}{0.000000in}}{\pgfqpoint{0.048611in}{0.000000in}}{%
\pgfpathmoveto{\pgfqpoint{0.000000in}{0.000000in}}%
\pgfpathlineto{\pgfqpoint{0.048611in}{0.000000in}}%
\pgfusepath{stroke,fill}%
}%
\begin{pgfscope}%
\pgfsys@transformshift{8.575625in}{1.363333in}%
\pgfsys@useobject{currentmarker}{}%
\end{pgfscope}%
\end{pgfscope}%
\begin{pgfscope}%
\pgftext[x=8.672847in,y=1.315506in,left,base]{\rmfamily\fontsize{10.000000}{12.000000}\selectfont \(\displaystyle -3\)}%
\end{pgfscope}%
\begin{pgfscope}%
\pgfsetbuttcap%
\pgfsetroundjoin%
\definecolor{currentfill}{rgb}{0.000000,0.000000,0.000000}%
\pgfsetfillcolor{currentfill}%
\pgfsetlinewidth{0.803000pt}%
\definecolor{currentstroke}{rgb}{0.000000,0.000000,0.000000}%
\pgfsetstrokecolor{currentstroke}%
\pgfsetdash{}{0pt}%
\pgfsys@defobject{currentmarker}{\pgfqpoint{0.000000in}{0.000000in}}{\pgfqpoint{0.048611in}{0.000000in}}{%
\pgfpathmoveto{\pgfqpoint{0.000000in}{0.000000in}}%
\pgfpathlineto{\pgfqpoint{0.048611in}{0.000000in}}%
\pgfusepath{stroke,fill}%
}%
\begin{pgfscope}%
\pgfsys@transformshift{8.575625in}{1.796667in}%
\pgfsys@useobject{currentmarker}{}%
\end{pgfscope}%
\end{pgfscope}%
\begin{pgfscope}%
\pgftext[x=8.672847in,y=1.748839in,left,base]{\rmfamily\fontsize{10.000000}{12.000000}\selectfont \(\displaystyle -2\)}%
\end{pgfscope}%
\begin{pgfscope}%
\pgfsetbuttcap%
\pgfsetroundjoin%
\definecolor{currentfill}{rgb}{0.000000,0.000000,0.000000}%
\pgfsetfillcolor{currentfill}%
\pgfsetlinewidth{0.803000pt}%
\definecolor{currentstroke}{rgb}{0.000000,0.000000,0.000000}%
\pgfsetstrokecolor{currentstroke}%
\pgfsetdash{}{0pt}%
\pgfsys@defobject{currentmarker}{\pgfqpoint{0.000000in}{0.000000in}}{\pgfqpoint{0.048611in}{0.000000in}}{%
\pgfpathmoveto{\pgfqpoint{0.000000in}{0.000000in}}%
\pgfpathlineto{\pgfqpoint{0.048611in}{0.000000in}}%
\pgfusepath{stroke,fill}%
}%
\begin{pgfscope}%
\pgfsys@transformshift{8.575625in}{2.230000in}%
\pgfsys@useobject{currentmarker}{}%
\end{pgfscope}%
\end{pgfscope}%
\begin{pgfscope}%
\pgftext[x=8.672847in,y=2.182172in,left,base]{\rmfamily\fontsize{10.000000}{12.000000}\selectfont \(\displaystyle -1\)}%
\end{pgfscope}%
\begin{pgfscope}%
\pgfsetbuttcap%
\pgfsetroundjoin%
\definecolor{currentfill}{rgb}{0.000000,0.000000,0.000000}%
\pgfsetfillcolor{currentfill}%
\pgfsetlinewidth{0.803000pt}%
\definecolor{currentstroke}{rgb}{0.000000,0.000000,0.000000}%
\pgfsetstrokecolor{currentstroke}%
\pgfsetdash{}{0pt}%
\pgfsys@defobject{currentmarker}{\pgfqpoint{0.000000in}{0.000000in}}{\pgfqpoint{0.048611in}{0.000000in}}{%
\pgfpathmoveto{\pgfqpoint{0.000000in}{0.000000in}}%
\pgfpathlineto{\pgfqpoint{0.048611in}{0.000000in}}%
\pgfusepath{stroke,fill}%
}%
\begin{pgfscope}%
\pgfsys@transformshift{8.575625in}{2.663333in}%
\pgfsys@useobject{currentmarker}{}%
\end{pgfscope}%
\end{pgfscope}%
\begin{pgfscope}%
\pgftext[x=8.672847in,y=2.615506in,left,base]{\rmfamily\fontsize{10.000000}{12.000000}\selectfont \(\displaystyle 0\)}%
\end{pgfscope}%
\begin{pgfscope}%
\pgfsetbuttcap%
\pgfsetroundjoin%
\definecolor{currentfill}{rgb}{0.000000,0.000000,0.000000}%
\pgfsetfillcolor{currentfill}%
\pgfsetlinewidth{0.803000pt}%
\definecolor{currentstroke}{rgb}{0.000000,0.000000,0.000000}%
\pgfsetstrokecolor{currentstroke}%
\pgfsetdash{}{0pt}%
\pgfsys@defobject{currentmarker}{\pgfqpoint{0.000000in}{0.000000in}}{\pgfqpoint{0.048611in}{0.000000in}}{%
\pgfpathmoveto{\pgfqpoint{0.000000in}{0.000000in}}%
\pgfpathlineto{\pgfqpoint{0.048611in}{0.000000in}}%
\pgfusepath{stroke,fill}%
}%
\begin{pgfscope}%
\pgfsys@transformshift{8.575625in}{3.096667in}%
\pgfsys@useobject{currentmarker}{}%
\end{pgfscope}%
\end{pgfscope}%
\begin{pgfscope}%
\pgftext[x=8.672847in,y=3.048839in,left,base]{\rmfamily\fontsize{10.000000}{12.000000}\selectfont \(\displaystyle 1\)}%
\end{pgfscope}%
\begin{pgfscope}%
\pgfsetbuttcap%
\pgfsetroundjoin%
\definecolor{currentfill}{rgb}{0.000000,0.000000,0.000000}%
\pgfsetfillcolor{currentfill}%
\pgfsetlinewidth{0.803000pt}%
\definecolor{currentstroke}{rgb}{0.000000,0.000000,0.000000}%
\pgfsetstrokecolor{currentstroke}%
\pgfsetdash{}{0pt}%
\pgfsys@defobject{currentmarker}{\pgfqpoint{0.000000in}{0.000000in}}{\pgfqpoint{0.048611in}{0.000000in}}{%
\pgfpathmoveto{\pgfqpoint{0.000000in}{0.000000in}}%
\pgfpathlineto{\pgfqpoint{0.048611in}{0.000000in}}%
\pgfusepath{stroke,fill}%
}%
\begin{pgfscope}%
\pgfsys@transformshift{8.575625in}{3.530000in}%
\pgfsys@useobject{currentmarker}{}%
\end{pgfscope}%
\end{pgfscope}%
\begin{pgfscope}%
\pgftext[x=8.672847in,y=3.482172in,left,base]{\rmfamily\fontsize{10.000000}{12.000000}\selectfont \(\displaystyle 2\)}%
\end{pgfscope}%
\begin{pgfscope}%
\pgfsetbuttcap%
\pgfsetroundjoin%
\definecolor{currentfill}{rgb}{0.000000,0.000000,0.000000}%
\pgfsetfillcolor{currentfill}%
\pgfsetlinewidth{0.803000pt}%
\definecolor{currentstroke}{rgb}{0.000000,0.000000,0.000000}%
\pgfsetstrokecolor{currentstroke}%
\pgfsetdash{}{0pt}%
\pgfsys@defobject{currentmarker}{\pgfqpoint{0.000000in}{0.000000in}}{\pgfqpoint{0.048611in}{0.000000in}}{%
\pgfpathmoveto{\pgfqpoint{0.000000in}{0.000000in}}%
\pgfpathlineto{\pgfqpoint{0.048611in}{0.000000in}}%
\pgfusepath{stroke,fill}%
}%
\begin{pgfscope}%
\pgfsys@transformshift{8.575625in}{3.963333in}%
\pgfsys@useobject{currentmarker}{}%
\end{pgfscope}%
\end{pgfscope}%
\begin{pgfscope}%
\pgftext[x=8.672847in,y=3.915506in,left,base]{\rmfamily\fontsize{10.000000}{12.000000}\selectfont \(\displaystyle 3\)}%
\end{pgfscope}%
\begin{pgfscope}%
\pgfsetbuttcap%
\pgfsetroundjoin%
\definecolor{currentfill}{rgb}{0.000000,0.000000,0.000000}%
\pgfsetfillcolor{currentfill}%
\pgfsetlinewidth{0.803000pt}%
\definecolor{currentstroke}{rgb}{0.000000,0.000000,0.000000}%
\pgfsetstrokecolor{currentstroke}%
\pgfsetdash{}{0pt}%
\pgfsys@defobject{currentmarker}{\pgfqpoint{0.000000in}{0.000000in}}{\pgfqpoint{0.048611in}{0.000000in}}{%
\pgfpathmoveto{\pgfqpoint{0.000000in}{0.000000in}}%
\pgfpathlineto{\pgfqpoint{0.048611in}{0.000000in}}%
\pgfusepath{stroke,fill}%
}%
\begin{pgfscope}%
\pgfsys@transformshift{8.575625in}{4.396667in}%
\pgfsys@useobject{currentmarker}{}%
\end{pgfscope}%
\end{pgfscope}%
\begin{pgfscope}%
\pgftext[x=8.672847in,y=4.348839in,left,base]{\rmfamily\fontsize{10.000000}{12.000000}\selectfont \(\displaystyle 4\)}%
\end{pgfscope}%
\begin{pgfscope}%
\pgfsetbuttcap%
\pgfsetmiterjoin%
\pgfsetlinewidth{0.501875pt}%
\definecolor{currentstroke}{rgb}{0.000000,0.000000,0.000000}%
\pgfsetstrokecolor{currentstroke}%
\pgfsetdash{}{0pt}%
\pgfpathmoveto{\pgfqpoint{8.402292in}{0.930000in}}%
\pgfpathlineto{\pgfqpoint{8.402292in}{0.943542in}}%
\pgfpathlineto{\pgfqpoint{8.402292in}{4.396667in}}%
\pgfpathlineto{\pgfqpoint{8.488958in}{4.570000in}}%
\pgfpathlineto{\pgfqpoint{8.488958in}{4.570000in}}%
\pgfpathlineto{\pgfqpoint{8.575625in}{4.396667in}}%
\pgfpathlineto{\pgfqpoint{8.575625in}{0.943542in}}%
\pgfpathlineto{\pgfqpoint{8.575625in}{0.930000in}}%
\pgfpathclose%
\pgfusepath{stroke}%
\end{pgfscope}%
\end{pgfpicture}%
\makeatother%
\endgroup%
} %
        \caption{Plot von $\hat\Delta_m^{\circledast 2}$ und $\hat\Delta_m$. Wieder liegt der Träger von $\hat\Delta_m^{\circledast 2}$ in der kausalen Zukunft.
        }
        \label{fig:delta_2m_twisted}
    \end{minipage}\hfill
    \begin{minipage}{0.45\textwidth}
        \centering
        \resizebox{\textwidth}{!}{%% Creator: Matplotlib, PGF backend
%%
%% To include the figure in your LaTeX document, write
%%   \input{<filename>.pgf}
%%
%% Make sure the required packages are loaded in your preamble
%%   \usepackage{pgf}
%%
%% Figures using additional raster images can only be included by \input if
%% they are in the same directory as the main LaTeX file. For loading figures
%% from other directories you can use the `import` package
%%   \usepackage{import}
%% and then include the figures with
%%   \import{<path to file>}{<filename>.pgf}
%%
%% Matplotlib used the following preamble
%%   \usepackage[utf8x]{inputenc}
%%   \usepackage[T1]{fontenc}
%%   \usepackage{amssymb}
%%
\begingroup%
\makeatletter%
\begin{pgfpicture}%
\pgfpathrectangle{\pgfpointorigin}{\pgfqpoint{4.000000in}{2.200000in}}%
\pgfusepath{use as bounding box, clip}%
\begin{pgfscope}%
\pgfsetbuttcap%
\pgfsetmiterjoin%
\definecolor{currentfill}{rgb}{1.000000,1.000000,1.000000}%
\pgfsetfillcolor{currentfill}%
\pgfsetlinewidth{0.000000pt}%
\definecolor{currentstroke}{rgb}{1.000000,1.000000,1.000000}%
\pgfsetstrokecolor{currentstroke}%
\pgfsetdash{}{0pt}%
\pgfpathmoveto{\pgfqpoint{0.000000in}{0.000000in}}%
\pgfpathlineto{\pgfqpoint{4.000000in}{0.000000in}}%
\pgfpathlineto{\pgfqpoint{4.000000in}{2.200000in}}%
\pgfpathlineto{\pgfqpoint{0.000000in}{2.200000in}}%
\pgfpathclose%
\pgfusepath{fill}%
\end{pgfscope}%
\begin{pgfscope}%
\pgfsetbuttcap%
\pgfsetmiterjoin%
\definecolor{currentfill}{rgb}{1.000000,1.000000,1.000000}%
\pgfsetfillcolor{currentfill}%
\pgfsetlinewidth{0.000000pt}%
\definecolor{currentstroke}{rgb}{0.000000,0.000000,0.000000}%
\pgfsetstrokecolor{currentstroke}%
\pgfsetstrokeopacity{0.000000}%
\pgfsetdash{}{0pt}%
\pgfpathmoveto{\pgfqpoint{0.198611in}{0.198611in}}%
\pgfpathlineto{\pgfqpoint{3.801389in}{0.198611in}}%
\pgfpathlineto{\pgfqpoint{3.801389in}{2.001389in}}%
\pgfpathlineto{\pgfqpoint{0.198611in}{2.001389in}}%
\pgfpathclose%
\pgfusepath{fill}%
\end{pgfscope}%
\begin{pgfscope}%
\pgfpathrectangle{\pgfqpoint{0.198611in}{0.198611in}}{\pgfqpoint{3.602778in}{1.802778in}} %
\pgfusepath{clip}%
\pgfsetbuttcap%
\pgfsetroundjoin%
\pgfsetlinewidth{0.501875pt}%
\definecolor{currentstroke}{rgb}{0.501961,0.501961,0.501961}%
\pgfsetstrokecolor{currentstroke}%
\pgfsetdash{{1.850000pt}{0.800000pt}}{0.000000pt}%
\pgfpathmoveto{\pgfqpoint{0.507421in}{0.198611in}}%
\pgfpathlineto{\pgfqpoint{0.507421in}{2.001389in}}%
\pgfusepath{stroke}%
\end{pgfscope}%
\begin{pgfscope}%
\pgfpathrectangle{\pgfqpoint{0.198611in}{0.198611in}}{\pgfqpoint{3.602778in}{1.802778in}} %
\pgfusepath{clip}%
\pgfsetrectcap%
\pgfsetroundjoin%
\pgfsetlinewidth{1.003750pt}%
\definecolor{currentstroke}{rgb}{0.894118,0.101961,0.109804}%
\pgfsetstrokecolor{currentstroke}%
\pgfsetdash{}{0pt}%
\pgfpathmoveto{\pgfqpoint{0.512092in}{2.015278in}}%
\pgfpathlineto{\pgfqpoint{0.512367in}{1.888581in}}%
\pgfpathlineto{\pgfqpoint{0.516488in}{1.548954in}}%
\pgfpathlineto{\pgfqpoint{0.520610in}{1.386168in}}%
\pgfpathlineto{\pgfqpoint{0.528853in}{1.217162in}}%
\pgfpathlineto{\pgfqpoint{0.537096in}{1.125446in}}%
\pgfpathlineto{\pgfqpoint{0.545339in}{1.065712in}}%
\pgfpathlineto{\pgfqpoint{0.557704in}{1.005174in}}%
\pgfpathlineto{\pgfqpoint{0.570069in}{0.962974in}}%
\pgfpathlineto{\pgfqpoint{0.582433in}{0.930877in}}%
\pgfpathlineto{\pgfqpoint{0.598920in}{0.897252in}}%
\pgfpathlineto{\pgfqpoint{0.619528in}{0.863755in}}%
\pgfpathlineto{\pgfqpoint{0.644257in}{0.830408in}}%
\pgfpathlineto{\pgfqpoint{0.677230in}{0.791728in}}%
\pgfpathlineto{\pgfqpoint{0.739054in}{0.725344in}}%
\pgfpathlineto{\pgfqpoint{0.837972in}{0.618605in}}%
\pgfpathlineto{\pgfqpoint{1.044051in}{0.391956in}}%
\pgfpathlineto{\pgfqpoint{1.085266in}{0.353376in}}%
\pgfpathlineto{\pgfqpoint{1.118239in}{0.326322in}}%
\pgfpathlineto{\pgfqpoint{1.147090in}{0.306084in}}%
\pgfpathlineto{\pgfqpoint{1.175941in}{0.289580in}}%
\pgfpathlineto{\pgfqpoint{1.200671in}{0.278774in}}%
\pgfpathlineto{\pgfqpoint{1.225400in}{0.271346in}}%
\pgfpathlineto{\pgfqpoint{1.250130in}{0.267550in}}%
\pgfpathlineto{\pgfqpoint{1.270738in}{0.267317in}}%
\pgfpathlineto{\pgfqpoint{1.291346in}{0.269857in}}%
\pgfpathlineto{\pgfqpoint{1.311953in}{0.275245in}}%
\pgfpathlineto{\pgfqpoint{1.332561in}{0.283528in}}%
\pgfpathlineto{\pgfqpoint{1.353169in}{0.294724in}}%
\pgfpathlineto{\pgfqpoint{1.377899in}{0.311977in}}%
\pgfpathlineto{\pgfqpoint{1.402628in}{0.333297in}}%
\pgfpathlineto{\pgfqpoint{1.427358in}{0.358504in}}%
\pgfpathlineto{\pgfqpoint{1.456209in}{0.392466in}}%
\pgfpathlineto{\pgfqpoint{1.489181in}{0.436571in}}%
\pgfpathlineto{\pgfqpoint{1.526276in}{0.491590in}}%
\pgfpathlineto{\pgfqpoint{1.579856in}{0.577165in}}%
\pgfpathlineto{\pgfqpoint{1.649923in}{0.688630in}}%
\pgfpathlineto{\pgfqpoint{1.687017in}{0.741589in}}%
\pgfpathlineto{\pgfqpoint{1.715868in}{0.777427in}}%
\pgfpathlineto{\pgfqpoint{1.740598in}{0.803323in}}%
\pgfpathlineto{\pgfqpoint{1.761206in}{0.820916in}}%
\pgfpathlineto{\pgfqpoint{1.781814in}{0.834460in}}%
\pgfpathlineto{\pgfqpoint{1.798300in}{0.842148in}}%
\pgfpathlineto{\pgfqpoint{1.814786in}{0.846876in}}%
\pgfpathlineto{\pgfqpoint{1.831273in}{0.848522in}}%
\pgfpathlineto{\pgfqpoint{1.847759in}{0.846999in}}%
\pgfpathlineto{\pgfqpoint{1.864245in}{0.842253in}}%
\pgfpathlineto{\pgfqpoint{1.880732in}{0.834264in}}%
\pgfpathlineto{\pgfqpoint{1.897218in}{0.823053in}}%
\pgfpathlineto{\pgfqpoint{1.917826in}{0.804607in}}%
\pgfpathlineto{\pgfqpoint{1.938434in}{0.781437in}}%
\pgfpathlineto{\pgfqpoint{1.959042in}{0.753857in}}%
\pgfpathlineto{\pgfqpoint{1.983771in}{0.715536in}}%
\pgfpathlineto{\pgfqpoint{2.012622in}{0.664800in}}%
\pgfpathlineto{\pgfqpoint{2.049717in}{0.592828in}}%
\pgfpathlineto{\pgfqpoint{2.140391in}{0.413262in}}%
\pgfpathlineto{\pgfqpoint{2.169242in}{0.364361in}}%
\pgfpathlineto{\pgfqpoint{2.193972in}{0.328891in}}%
\pgfpathlineto{\pgfqpoint{2.214580in}{0.304945in}}%
\pgfpathlineto{\pgfqpoint{2.231066in}{0.289973in}}%
\pgfpathlineto{\pgfqpoint{2.247553in}{0.279056in}}%
\pgfpathlineto{\pgfqpoint{2.264039in}{0.272442in}}%
\pgfpathlineto{\pgfqpoint{2.276404in}{0.270421in}}%
\pgfpathlineto{\pgfqpoint{2.288768in}{0.270983in}}%
\pgfpathlineto{\pgfqpoint{2.301133in}{0.274159in}}%
\pgfpathlineto{\pgfqpoint{2.313498in}{0.279957in}}%
\pgfpathlineto{\pgfqpoint{2.329984in}{0.291732in}}%
\pgfpathlineto{\pgfqpoint{2.346470in}{0.308025in}}%
\pgfpathlineto{\pgfqpoint{2.362957in}{0.328643in}}%
\pgfpathlineto{\pgfqpoint{2.383565in}{0.360074in}}%
\pgfpathlineto{\pgfqpoint{2.404173in}{0.397119in}}%
\pgfpathlineto{\pgfqpoint{2.428902in}{0.447714in}}%
\pgfpathlineto{\pgfqpoint{2.461875in}{0.522473in}}%
\pgfpathlineto{\pgfqpoint{2.540185in}{0.703588in}}%
\pgfpathlineto{\pgfqpoint{2.564914in}{0.752398in}}%
\pgfpathlineto{\pgfqpoint{2.585522in}{0.786944in}}%
\pgfpathlineto{\pgfqpoint{2.602009in}{0.809694in}}%
\pgfpathlineto{\pgfqpoint{2.618495in}{0.827512in}}%
\pgfpathlineto{\pgfqpoint{2.630860in}{0.837367in}}%
\pgfpathlineto{\pgfqpoint{2.643224in}{0.844041in}}%
\pgfpathlineto{\pgfqpoint{2.655589in}{0.847416in}}%
\pgfpathlineto{\pgfqpoint{2.667954in}{0.847411in}}%
\pgfpathlineto{\pgfqpoint{2.680319in}{0.843984in}}%
\pgfpathlineto{\pgfqpoint{2.692683in}{0.837133in}}%
\pgfpathlineto{\pgfqpoint{2.705048in}{0.826900in}}%
\pgfpathlineto{\pgfqpoint{2.721534in}{0.808145in}}%
\pgfpathlineto{\pgfqpoint{2.738021in}{0.783846in}}%
\pgfpathlineto{\pgfqpoint{2.754507in}{0.754452in}}%
\pgfpathlineto{\pgfqpoint{2.775115in}{0.711435in}}%
\pgfpathlineto{\pgfqpoint{2.799844in}{0.652477in}}%
\pgfpathlineto{\pgfqpoint{2.841060in}{0.544076in}}%
\pgfpathlineto{\pgfqpoint{2.886398in}{0.426470in}}%
\pgfpathlineto{\pgfqpoint{2.911127in}{0.370530in}}%
\pgfpathlineto{\pgfqpoint{2.931735in}{0.331451in}}%
\pgfpathlineto{\pgfqpoint{2.948221in}{0.306336in}}%
\pgfpathlineto{\pgfqpoint{2.964708in}{0.287474in}}%
\pgfpathlineto{\pgfqpoint{2.977073in}{0.277785in}}%
\pgfpathlineto{\pgfqpoint{2.989437in}{0.272123in}}%
\pgfpathlineto{\pgfqpoint{3.001802in}{0.270619in}}%
\pgfpathlineto{\pgfqpoint{3.014167in}{0.273340in}}%
\pgfpathlineto{\pgfqpoint{3.026531in}{0.280293in}}%
\pgfpathlineto{\pgfqpoint{3.038896in}{0.291421in}}%
\pgfpathlineto{\pgfqpoint{3.051261in}{0.306604in}}%
\pgfpathlineto{\pgfqpoint{3.067747in}{0.332826in}}%
\pgfpathlineto{\pgfqpoint{3.084234in}{0.365307in}}%
\pgfpathlineto{\pgfqpoint{3.104842in}{0.413477in}}%
\pgfpathlineto{\pgfqpoint{3.129571in}{0.479776in}}%
\pgfpathlineto{\pgfqpoint{3.224367in}{0.744676in}}%
\pgfpathlineto{\pgfqpoint{3.244975in}{0.787714in}}%
\pgfpathlineto{\pgfqpoint{3.261462in}{0.814705in}}%
\pgfpathlineto{\pgfqpoint{3.273826in}{0.829994in}}%
\pgfpathlineto{\pgfqpoint{3.286191in}{0.840690in}}%
\pgfpathlineto{\pgfqpoint{3.298556in}{0.846559in}}%
\pgfpathlineto{\pgfqpoint{3.306799in}{0.847712in}}%
\pgfpathlineto{\pgfqpoint{3.315042in}{0.846627in}}%
\pgfpathlineto{\pgfqpoint{3.323285in}{0.843296in}}%
\pgfpathlineto{\pgfqpoint{3.335650in}{0.834120in}}%
\pgfpathlineto{\pgfqpoint{3.348015in}{0.820032in}}%
\pgfpathlineto{\pgfqpoint{3.360380in}{0.801236in}}%
\pgfpathlineto{\pgfqpoint{3.376866in}{0.769366in}}%
\pgfpathlineto{\pgfqpoint{3.393352in}{0.730577in}}%
\pgfpathlineto{\pgfqpoint{3.413960in}{0.674169in}}%
\pgfpathlineto{\pgfqpoint{3.442811in}{0.585459in}}%
\pgfpathlineto{\pgfqpoint{3.496392in}{0.419034in}}%
\pgfpathlineto{\pgfqpoint{3.517000in}{0.364713in}}%
\pgfpathlineto{\pgfqpoint{3.533486in}{0.328507in}}%
\pgfpathlineto{\pgfqpoint{3.549972in}{0.300226in}}%
\pgfpathlineto{\pgfqpoint{3.562337in}{0.284891in}}%
\pgfpathlineto{\pgfqpoint{3.574702in}{0.274996in}}%
\pgfpathlineto{\pgfqpoint{3.582945in}{0.271550in}}%
\pgfpathlineto{\pgfqpoint{3.591188in}{0.270684in}}%
\pgfpathlineto{\pgfqpoint{3.599431in}{0.272420in}}%
\pgfpathlineto{\pgfqpoint{3.607675in}{0.276761in}}%
\pgfpathlineto{\pgfqpoint{3.620039in}{0.288098in}}%
\pgfpathlineto{\pgfqpoint{3.632404in}{0.305056in}}%
\pgfpathlineto{\pgfqpoint{3.644769in}{0.327337in}}%
\pgfpathlineto{\pgfqpoint{3.661255in}{0.364590in}}%
\pgfpathlineto{\pgfqpoint{3.681863in}{0.421417in}}%
\pgfpathlineto{\pgfqpoint{3.706593in}{0.500380in}}%
\pgfpathlineto{\pgfqpoint{3.772538in}{0.718180in}}%
\pgfpathlineto{\pgfqpoint{3.793146in}{0.772633in}}%
\pgfpathlineto{\pgfqpoint{3.801389in}{0.790946in}}%
\pgfpathlineto{\pgfqpoint{3.801389in}{0.790946in}}%
\pgfusepath{stroke}%
\end{pgfscope}%
\begin{pgfscope}%
\pgfpathrectangle{\pgfqpoint{0.198611in}{0.198611in}}{\pgfqpoint{3.602778in}{1.802778in}} %
\pgfusepath{clip}%
\pgfsetrectcap%
\pgfsetroundjoin%
\pgfsetlinewidth{1.003750pt}%
\definecolor{currentstroke}{rgb}{0.894118,0.101961,0.109804}%
\pgfsetstrokecolor{currentstroke}%
\pgfsetdash{}{0pt}%
\pgfpathmoveto{\pgfqpoint{0.184722in}{0.559167in}}%
\pgfpathlineto{\pgfqpoint{0.324422in}{0.559167in}}%
\pgfpathlineto{\pgfqpoint{0.507421in}{0.559167in}}%
\pgfusepath{stroke}%
\end{pgfscope}%
\begin{pgfscope}%
\pgfpathrectangle{\pgfqpoint{0.198611in}{0.198611in}}{\pgfqpoint{3.602778in}{1.802778in}} %
\pgfusepath{clip}%
\pgfsetbuttcap%
\pgfsetroundjoin%
\pgfsetlinewidth{0.501875pt}%
\definecolor{currentstroke}{rgb}{0.501961,0.501961,0.501961}%
\pgfsetstrokecolor{currentstroke}%
\pgfsetdash{{1.850000pt}{0.800000pt}}{0.000000pt}%
\pgfpathmoveto{\pgfqpoint{0.184722in}{0.847611in}}%
\pgfpathlineto{\pgfqpoint{3.815278in}{0.847611in}}%
\pgfusepath{stroke}%
\end{pgfscope}%
\begin{pgfscope}%
\pgfsetrectcap%
\pgfsetmiterjoin%
\pgfsetlinewidth{0.501875pt}%
\definecolor{currentstroke}{rgb}{0.000000,0.000000,0.000000}%
\pgfsetstrokecolor{currentstroke}%
\pgfsetdash{}{0pt}%
\pgfpathmoveto{\pgfqpoint{0.301548in}{0.198611in}}%
\pgfpathlineto{\pgfqpoint{0.301548in}{2.001389in}}%
\pgfusepath{stroke}%
\end{pgfscope}%
\begin{pgfscope}%
\pgfsetrectcap%
\pgfsetmiterjoin%
\pgfsetlinewidth{0.501875pt}%
\definecolor{currentstroke}{rgb}{0.000000,0.000000,0.000000}%
\pgfsetstrokecolor{currentstroke}%
\pgfsetdash{}{0pt}%
\pgfpathmoveto{\pgfqpoint{0.198611in}{0.559167in}}%
\pgfpathlineto{\pgfqpoint{3.801389in}{0.559167in}}%
\pgfusepath{stroke}%
\end{pgfscope}%
\begin{pgfscope}%
\pgftext[x=0.548595in,y=0.342833in,left,base]{\rmfamily\fontsize{10.000000}{12.000000}\selectfont \(\displaystyle \omega = 2 m\)}%
\end{pgfscope}%
\begin{pgfscope}%
\pgftext[x=0.548595in,y=1.136056in,left,base]{\rmfamily\fontsize{10.000000}{12.000000}\selectfont \(\displaystyle \approx \frac{1}{\sqrt{\omega}}\)}%
\end{pgfscope}%
\begin{pgfscope}%
\pgftext[x=2.566151in,y=0.876456in,left,base]{\rmfamily\fontsize{10.000000}{12.000000}\selectfont \(\displaystyle \approx 2 \cos\left(\frac{\omega^2}{2}\right)\)}%
\end{pgfscope}%
\begin{pgfscope}%
\pgfsetroundcap%
\pgfsetroundjoin%
\pgfsetlinewidth{0.501875pt}%
\definecolor{currentstroke}{rgb}{0.000000,0.000000,0.000000}%
\pgfsetstrokecolor{currentstroke}%
\pgfsetdash{}{0pt}%
\pgfpathmoveto{\pgfqpoint{0.301548in}{2.007506in}}%
\pgfpathquadraticcurveto{\pgfqpoint{0.301548in}{2.008330in}}{\pgfqpoint{0.301548in}{2.001389in}}%
\pgfusepath{stroke}%
\end{pgfscope}%
\begin{pgfscope}%
\pgfsetroundcap%
\pgfsetroundjoin%
\pgfsetlinewidth{0.501875pt}%
\definecolor{currentstroke}{rgb}{0.000000,0.000000,0.000000}%
\pgfsetstrokecolor{currentstroke}%
\pgfsetdash{}{0pt}%
\pgfpathmoveto{\pgfqpoint{0.273770in}{1.951951in}}%
\pgfpathlineto{\pgfqpoint{0.301548in}{2.007506in}}%
\pgfpathlineto{\pgfqpoint{0.329325in}{1.951951in}}%
\pgfusepath{stroke}%
\end{pgfscope}%
\begin{pgfscope}%
\pgftext[x=0.301548in,y=2.070833in,,bottom]{\rmfamily\fontsize{10.000000}{12.000000}\selectfont \(\displaystyle \hat\Delta^{\circledast 2} ~(-\omega, 0)\)}%
\end{pgfscope}%
\begin{pgfscope}%
\pgfsetroundcap%
\pgfsetroundjoin%
\pgfsetlinewidth{0.501875pt}%
\definecolor{currentstroke}{rgb}{0.000000,0.000000,0.000000}%
\pgfsetstrokecolor{currentstroke}%
\pgfsetdash{}{0pt}%
\pgfpathmoveto{\pgfqpoint{3.807500in}{0.559167in}}%
\pgfpathquadraticcurveto{\pgfqpoint{3.808327in}{0.559167in}}{\pgfqpoint{3.801389in}{0.559167in}}%
\pgfusepath{stroke}%
\end{pgfscope}%
\begin{pgfscope}%
\pgfsetroundcap%
\pgfsetroundjoin%
\pgfsetlinewidth{0.501875pt}%
\definecolor{currentstroke}{rgb}{0.000000,0.000000,0.000000}%
\pgfsetstrokecolor{currentstroke}%
\pgfsetdash{}{0pt}%
\pgfpathmoveto{\pgfqpoint{3.751945in}{0.586944in}}%
\pgfpathlineto{\pgfqpoint{3.807500in}{0.559167in}}%
\pgfpathlineto{\pgfqpoint{3.751945in}{0.531389in}}%
\pgfusepath{stroke}%
\end{pgfscope}%
\begin{pgfscope}%
\pgftext[x=3.870833in,y=0.559167in,left,]{\rmfamily\fontsize{10.000000}{12.000000}\selectfont \(\displaystyle \omega\)}%
\end{pgfscope}%
\end{pgfpicture}%
\makeatother%
\endgroup%
}
        \caption{Plot von $\left.\hat{\Delta}_m^{\circledast 2}\right|_{k=0}$ um das asymptotische Verhalten für $\omega \rightarrow 0$ und $\omega \rightarrow \infty$ zu verdeutlichen}
        \label{fig:delta_2m_twisted_k0}
    \end{minipage}
\end{figure}

\subsubsection{Fall $|s| > 1$}
Wir bedienen uns wieder genau des selben Arguments, wie in \eqref{eq:delta_m2_s>1} und dürfen direkt schreiben:

\begin{equation}
\label{eq:delta_m2_twisted_s=1}
    \left\langle\hat\psi_{ast}, \rwhat{\Delta}_m^{\circledast 2}\right\rangle
    = 0 \condition{für alle $a$ klein genug}
\end{equation}


\subsubsection*{Fall $|s| < 1, (x,t) \neq 0$}
Da
$\rwhat\Delta_m^{\circledast 2} = \rwhat\Delta_m^{* 2} \cos(\dots)$ können wir direkt mit dem Ausdruck \eqref{eq:psi_ast_delta_m2_s<1} $\cdot \cos$ weiter arbeiten.

\todo{Referenz zu Delta_m2 s<1 einfügen und argumentieren, dass wir genau die selben Abschätzungen machen.}

\begin{dmath}
\label{eq:delta_m2_twisted_s<1}
    \left\langle \rwhat{\psi}_{ast},\rwhat{\Delta}_m^{\circledast 2}
    \right\rangle
    =
     2 a^{-\frac{3}{4}} \int \frac{
    \hat\psi_1(\omega)~ \hat\psi_2(k) \left(
    \omega^2 \left(\Delta s - 2 a^{\frac{1}{2}} k s - ak^2
            \right) - 3a^2m^2
    \right)
     }
     {
        \sqrt{\Delta s -2a^{\frac{1}{2}}ks - ak^2}
            \sqrt{\Delta s \omega^2 -2a^{\frac{1}{2}} \omega^2 k s
                    - a\omega^2k^2-4 a^2 m^2}
     }
     \cdot
     \Theta(\cdots)
     \cos(\varphi(\omega^2-k^2))
     e^{-i \omega \left(\frac{t'-sx'}{a}+k \frac{x'}{\sqrt{a}}\right)}
     \d \omega \d k
     \leq
     2 a^{-\frac{3}{4}} \int
     \omega \hat\psi_1(\omega)\, \hat\psi_2(k)
     e^{-i \omega \left(\frac{t'-sx'}{a}+k \frac{x'}{\sqrt{a}}\right)}
     \d \omega \d k
     \sim O(a^k) ~~ \forall k \hiderel \in \mathbb{N}
\end{dmath}


\subsubsection*{Fall $|s| < 1, (x,t) = 0$}
In diesem Fall lassen wir den $\cos$-Faktor in \eqref{eq:delta_m2_twisted_s<1} in der ersten Ungleichung nicht heraus fallen, dafür wird der $e^\cdots$-Faktor 1. Den $\cos$-Faktor schreiben wir als Summe von $e$-Funktionen und erhalten

\todo{\dots in exp durch ---"--- ersetzen}

\begin{dmath}
    \left\langle \rwhat{\psi}_{ast},\rwhat{\Delta}_m^{\circledast 2}
    \right\rangle
    =
    2 a^{-\frac{3}{4}} \int
    \omega \hat\psi_1(\omega) \hat\psi_2(k)
    \left\{
        \exp\left(i a^{-2} \frac{
        \omega^2 (\Delta s - 2 a^{\frac{1}{2}} k s - a k^2)
        }{2}
        \sqrt{\frac{1}{4} - \frac{a^2 m^2}{\omega^2(
            \Delta s - 2 a^{\frac{1}{2}} k s - ak^2
        )}}
        \right)
        +\exp (-i \cdots)
    \right\}
    \d \omega \d k
    =
    2 a^{-\frac{3}{4}} \int
    \cancel{\sqrt{\omega}} \hat\psi_1(\sqrt{\omega}) \hat\psi_2(k)
    \left\{
        \exp\left(i a^{-2} \frac{
        \omega (\Delta s - 2 a^{\frac{1}{2}} k s - a k^2)
        }{2}
        \sqrt{\frac{1}{4} - \frac{a^2 m^2}{\omega(
            \Delta s - 2 a^{\frac{1}{2}} k s - ak^2
        )}}
        \right)
        +\exp (-i \cdots)
    \right\}
    \frac{{\d \omega \d k}}{\cancel{\sqrt{\omega}}}
    =
    2 a^{-\frac{3}{4}} \int \left\{
        \int
        \hat\psi_1(\sqrt\omega)
        \left\{
            \exp
            \left(ia^{-2} \left(\frac{\omega \Delta s}{4}
                                + O\left(a^{\frac{1}{2}}\right)\right)
            \right)
            + \exp(-i\cdots)
        \right\}
        \d \omega
    \right\}
    \hat \psi_2(k) \d k
    =
    2 a^{-\frac{3}{4}} \int
    \underbrace{
    (\hat\psi_1 \circ \sqrt{\cdot })^\vee \left(\frac{\Delta s}{4a^2}\right)}_{
    \sim O(a^k) ~\forall k \in \mathbb{N}
    }
    \psi_2(k) \d k
    \sim O(a^k) ~~ \forall k \hiderel \in \mathbb{N}
\end{dmath}

\todo[color=red]{Schritt von dritte in die vierte Zeile rechtfertigen. Vermutlich nicht einfach...}

wobei bei der Substition $\omega \to \sqrt{\omega}$ in der zweiten Zeile wichtig ist, dass $0 \notin supp (\hat\psi_1)$, also auch nach der Substitution noch $\hat\psi_1 \in C_c^\infty (\mathbb{R})$ ist.
% section dots_und_nun_zur_wellenfrontmenge_von_ (end)


% section die_wellenfrontmenge_von_ (end)
