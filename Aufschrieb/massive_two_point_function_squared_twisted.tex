% !TEX root = main.tex
% !TEX spellcheck=de_DE
%%%%%%%%%%%%%%%%%%%%%%%%%%%%%%%%%%%%%%%%%%%%%%%%%%%%%%%%%%%%%%%%%%%%%%%%%%%%%%%
% % Berechnen der Wellenfrontmenge von Delta_m_twisted
%%%%%%%%%%%%%%%%%%%%%%%%%%%%%%%%%%%%%%%%%%%%%%%%%%%%%%%%%%%%%%%%%%%%%%%%%%%%%%%

\section{\texorpdfstring{Die Wellenfrontmenge von $\Delta_m^{\star 2}$}
         {Die Wellenfrontmenge der getwisteten Zweipunktfunktion}} % (fold)
\label{sec:die_wellenfrontmenge_von_delta_m2_twisted}

Bevor wir uns aber der Wellenfrontmenge widmen können, brauchen wir einen Ausdruck für die Fouriertransformierte $\rwhat{\Delta}_m^{\circledast 2}$ von $\Delta_m^{\star 2}$.

\subsection{\texorpdfstring{$\hat\Delta_m^{\circledast 2}$ berechnen}
            {Die getwistete Zweipunktfunktion berechnen}} % (fold)
\label{sec:delta_m2_twisted_berechnen}

Auch für die getwistete Faltung ist schnell nachgerechnet, dass
\begin{equation*}
    \rwhat{\Delta_m} \circledast \rwhat{\Delta_m} (-\omega,-k)
     =
    \overline{ \left(\rwhat{\Delta_m} (-\cdot) \circledast \rwhat{\Delta_m} (-\cdot)\right)}
     (\omega,k).
\end{equation*}
Wie wir später sehen werden, ist die komplexe Konjugation irrelevant, da alles reell.

\begin{dgroup}
    \begin{dmath}
        \rwhat{\Delta}_m(-\cdot) = \delta(\omega^2-k^w-m^2)\Theta(\omega)\\
        \textrm{die Fouriertransformierte der massiven Zweipunktfunktion}
    \label{eq:material_fuer_delta_m2_twisted_a}
    \end{dmath}
    \begin{dmath}
        \Omega = \begin{pmatrix}
            0 & 1 \\ -1 & 1
        \end{pmatrix}
        \\ \textrm{die kanonische symplektische Matrix auf } \mathbb{R}^n
    \label{eq:material_fuer_delta_m2_twisted_b}
    \end{dmath}
\end{dgroup}

mit \cref{def:twisted_convolution,eq:material_fuer_delta_m2_twisted_a,eq:material_fuer_delta_m2_twisted_b} erhalten wir also

\begin{dmath}
    \rwhat{\Delta}_m^{\circledast 2} (-\omega, -k)
    = \int
    \delta(\omega^{\prime 2}-k^{\prime 2}-m^2)
    \delta((\omega' - \omega)^2 - (k-k')^2 -m^2)
    \cdot
    \Theta(\omega') \Theta(\omega - \omega')
    e^{\frac{i}{2}(\omega'k-\omega k')}
    \d \omega' \d k'
\end{dmath}

und damit das selbe Integral wie in \cref{eq:mass_shell_convolution} bis auf einen zusätzlichen Phasenfaktor. Nachdem wir gezeigt haben, dass auch dieser Lorentz-Invariant ist, können wir das Integral mit dem selben Trick wie in \cref{sec:delta_m2_berechnen} berechnen.

\begin{proposition}[$\Omega_{std}$ ist Lorentz-invariant für $n=2$]
\label{prop:omega_ist_Lorentz_invariant}
    $\Omega_{std}$ ist Lorentz-invariant für $n=2$
\\[1em]
\emph{Beweis}\\
    Eine einfache Rechnung zeigt
    \begin{dmath*}
        \begin{pmatrix}
            \cosh \beta & \sinh \beta \\ -\sinh \beta & \cosh \beta
        \end{pmatrix}
        \begin{pmatrix}
            0 & 1 \\ -1 & 0
        \end{pmatrix}
        \begin{pmatrix}
            \cosh \beta & -\sinh \beta \\ \sinh \beta & \cosh \beta
        \end{pmatrix}
        =
        \begin{pmatrix}
            0 & 1 \\ -1 & 0
        \end{pmatrix}
    \end{dmath*}
    für alle $\beta \in \mathbb{R}$.
\end{proposition}

Mit \cref{prop:omega_ist_Lorentz_invariant} ist $\rwhat{\Delta}_m^{\circledast 2}$ Lorentz-Invariant und es reicht aus $\rwhat{\Delta}_m^{\circledast 2} (\omega, 0)$ zu berechnen.

Die beiden Kreuzungspunkte der $\delta$-Distributionen liegen bei (vgl. \cref{fig:mass_shell_convolution})

\begin{equation*}
    \left(\omega'_0,k'_{0\pm}\right) = \left(\frac{\omega}{2}, \pm \sqrt{\left(\frac{\omega}{2}\right)^2-m^2}\right)
\end{equation*}


Die "`Fläche"' der Kreuzungspunkte der $\delta$-Distributionen wurde in
 \cref{sec:delta_m2_berechnen} berechnet und ist

\begin{equation*}
A = \frac{\omega^2-m^2}{\omega \sqrt{\omega^2-4m^2}}.
\end{equation*}

Der Phasenfaktor nimmt bei den Kreuzungspunkten folgende Werte an:
\begin{dmath*}
    e^{\frac{i}{2}\Omega \left((\omega, k),(\omega'_0,k'_{0\pm})\right)}
    =
    e^{\pm \frac{i}{2}\left(-\omega^2\sqrt{\frac{1}{4}-\frac{m^2}{\omega^2}}\right)}
\end{dmath*}


Kombinieren wir also die vorhergehenden Resultate erhalten wir

\begin{align*}
    \rwhat{\Delta}_m^{\circledast 2} (-\omega, 0)
    &=
    A e^{\frac{i}{2}\Omega \left((\omega, k),(\omega'_0,k'_{0+})\right)}
    + A e^{\frac{i}{2}\Omega \left((\omega, k),(\omega'_0,k'_{0-})\right)}
    \\&=
    \frac{\omega^2-2m^2}{\omega \sqrt{\omega^2-4m^2}}
    \left\{
        e^{-\frac{i}{2}\omega^2\sqrt{\frac{1}{4}-\frac{m^2}{\omega^2}}}
      + e^{\frac{i}{2}\omega^2\sqrt{\frac{1}{4}-\frac{m^2}{\omega^2}}}
    \right\}
    \Theta\left(\omega^2-4m^2\right)
    \\&=
    2 \frac{\omega^2 -2m^2}{\omega \sqrt{\omega^2-4m^2}}
    \cos \left(\varphi(\omega^2)\right) \Theta\left(\omega^2-4m^2\right),
\end{align*}
% \begin{dmath*}
%     \rwhat{\Delta}_m^{\circledast 2} (\omega, 0)
%     =
%     A e^{\frac{i}{2}\Omega \left((\omega, k),(\omega'_0,k'_{0+})\right)}
%     + A e^{\frac{i}{2}\Omega \left((\omega, k),(\omega'_0,k'_{0-})\right)}
%     =
%     \frac{\omega^2-3m^2}{\omega \sqrt{\omega^2-4m^2}}
%     \left\{
%         e^{-\frac{i}{2}\omega^2\sqrt{\frac{1}{4}-\frac{m^2}{\omega^2}}}
%       + e^{\frac{i}{2}\omega^2\sqrt{\frac{1}{4}-\frac{m^2}{\omega^2}}}
%     \right\}
%     \Theta\left(\omega^2-4m^2\right)
%     =
%     2 \frac{\omega^2 -3m^2}{\omega \sqrt{\omega^2-4m^2}}
%     \cos \left(\varphi(\omega^2)\right) \Theta\left(\omega^2-4m^2\right)
% \end{dmath*}

wobei im letzten Schritt noch implizit $\varphi(\omega^2)$ definiert wurde.
Und mit Lorentz-Invarianz erhalten wir schließlich

\begin{align}
    \rwhat{\Delta}_m^{\circledast 2} (-\omega, -k)
    &=
    \rwhat{\Delta}_m^{\circledast 2} (-\sqrt{\omega^2-k^2}, 0)
    \nonumber \\ &=
    2\frac{\omega^2-k^2-3m^2}{\sqrt{\omega^2-k^2} \sqrt{\omega^2-k^2-4m^2}}
    \cos \left(\frac{k^2-\omega^2}{2}
    \sqrt{\frac{1}{4}+\frac{m^2}{k^2-\omega^2}}
    \right)
    \nonumber \\ & \kern 12em\cdot
    \Theta \left(\omega^2-k^2-4m^2\right)
    \nonumber \\ &=
    \rwhat{\Delta}_m^{* 2}(\omega, k) \cos (\varphi(\omega^2-k^2))
    \Theta\left(\omega^2-k^2-4m^2\right),
\end{align}
% \begin{dmath}
%     \rwhat{\Delta}_m^{\circledast 2} (\omega, k)
%     =
%     \rwhat{\Delta}_m^{\circledast 2} (\sqrt{\omega^2-k^2}, 0)
%     =
%     2\frac{\omega^2-k^2-3m^2}{\sqrt{\omega^2-k^2} \sqrt{\omega^2-k^2-4m^2}}
%     \cos \left(\frac{k^2-\omega^2}{2}
%     \sqrt{\frac{1}{4}+\frac{m^2}{k^2-\omega^2}}
%     \right)
%     \Theta \left(\omega^2-k^2-4m^2\right)
%     =
%     \rwhat{\Delta}_m^{* 2}(\omega, k) \cos (\varphi(\omega^2-k^2))
% \end{dmath}

\subsection{
\texorpdfstring{\dots und nun zur Wellenfrontmenge von $\hat{\Delta}_m^{\circledast 2}$}{... und nun zur Wellenfrontmenge der getwisteten Zweipunktfunktion}} % (fold)
\label{sec:dots_und_nun_zur_wellenfrontmenge_von_delta_m2_twisted}

\begin{figure}
    \centering
    \begin{minipage}{0.55\textwidth}
        \centering
        \resizebox{\textwidth}{!}{%% Creator: Matplotlib, PGF backend
%%
%% To include the figure in your LaTeX document, write
%%   \input{<filename>.pgf}
%%
%% Make sure the required packages are loaded in your preamble
%%   \usepackage{pgf}
%%
%% Figures using additional raster images can only be included by \input if
%% they are in the same directory as the main LaTeX file. For loading figures
%% from other directories you can use the `import` package
%%   \usepackage{import}
%% and then include the figures with
%%   \import{<path to file>}{<filename>.pgf}
%%
%% Matplotlib used the following preamble
%%   \usepackage[utf8x]{inputenc}
%%   \usepackage[T1]{fontenc}
%%   \usepackage{amssymb}
%%
\begingroup%
\makeatletter%
\begin{pgfpicture}%
\pgfpathrectangle{\pgfpointorigin}{\pgfqpoint{4.000000in}{2.200000in}}%
\pgfusepath{use as bounding box, clip}%
\begin{pgfscope}%
\pgfsetbuttcap%
\pgfsetmiterjoin%
\definecolor{currentfill}{rgb}{1.000000,1.000000,1.000000}%
\pgfsetfillcolor{currentfill}%
\pgfsetlinewidth{0.000000pt}%
\definecolor{currentstroke}{rgb}{1.000000,1.000000,1.000000}%
\pgfsetstrokecolor{currentstroke}%
\pgfsetdash{}{0pt}%
\pgfpathmoveto{\pgfqpoint{0.000000in}{0.000000in}}%
\pgfpathlineto{\pgfqpoint{4.000000in}{0.000000in}}%
\pgfpathlineto{\pgfqpoint{4.000000in}{2.200000in}}%
\pgfpathlineto{\pgfqpoint{0.000000in}{2.200000in}}%
\pgfpathclose%
\pgfusepath{fill}%
\end{pgfscope}%
\begin{pgfscope}%
\pgfsetbuttcap%
\pgfsetmiterjoin%
\definecolor{currentfill}{rgb}{1.000000,1.000000,1.000000}%
\pgfsetfillcolor{currentfill}%
\pgfsetlinewidth{0.000000pt}%
\definecolor{currentstroke}{rgb}{0.000000,0.000000,0.000000}%
\pgfsetstrokecolor{currentstroke}%
\pgfsetstrokeopacity{0.000000}%
\pgfsetdash{}{0pt}%
\pgfpathmoveto{\pgfqpoint{0.198611in}{0.333208in}}%
\pgfpathlineto{\pgfqpoint{3.119722in}{0.333208in}}%
\pgfpathlineto{\pgfqpoint{3.119722in}{1.866792in}}%
\pgfpathlineto{\pgfqpoint{0.198611in}{1.866792in}}%
\pgfpathclose%
\pgfusepath{fill}%
\end{pgfscope}%
\begin{pgfscope}%
\pgfpathrectangle{\pgfqpoint{0.198611in}{0.333208in}}{\pgfqpoint{2.921111in}{1.533583in}} %
\pgfusepath{clip}%
\pgfsys@transformshift{0.198611in}{0.333208in}%
\pgftext[left,bottom]{\pgfimage[interpolate=true,width=2.921667in,height=1.535000in]{delta_m2_twisted-img0.png}}%
\end{pgfscope}%
\begin{pgfscope}%
\pgfpathrectangle{\pgfqpoint{0.198611in}{0.333208in}}{\pgfqpoint{2.921111in}{1.533583in}} %
\pgfusepath{clip}%
\pgfsetrectcap%
\pgfsetroundjoin%
\pgfsetlinewidth{0.501875pt}%
\definecolor{currentstroke}{rgb}{0.894118,0.101961,0.109804}%
\pgfsetstrokecolor{currentstroke}%
\pgfsetdash{}{0pt}%
\pgfpathmoveto{\pgfqpoint{0.204257in}{1.868458in}}%
\pgfpathlineto{\pgfqpoint{0.450330in}{1.623864in}}%
\pgfpathlineto{\pgfqpoint{0.643510in}{1.432341in}}%
\pgfpathlineto{\pgfqpoint{0.795712in}{1.281957in}}%
\pgfpathlineto{\pgfqpoint{0.912791in}{1.166768in}}%
\pgfpathlineto{\pgfqpoint{1.006453in}{1.075091in}}%
\pgfpathlineto{\pgfqpoint{1.088408in}{0.995386in}}%
\pgfpathlineto{\pgfqpoint{1.152802in}{0.933244in}}%
\pgfpathlineto{\pgfqpoint{1.211341in}{0.877278in}}%
\pgfpathlineto{\pgfqpoint{1.258172in}{0.833001in}}%
\pgfpathlineto{\pgfqpoint{1.299150in}{0.794752in}}%
\pgfpathlineto{\pgfqpoint{1.334274in}{0.762449in}}%
\pgfpathlineto{\pgfqpoint{1.363543in}{0.735972in}}%
\pgfpathlineto{\pgfqpoint{1.392813in}{0.710007in}}%
\pgfpathlineto{\pgfqpoint{1.416229in}{0.689699in}}%
\pgfpathlineto{\pgfqpoint{1.439644in}{0.669907in}}%
\pgfpathlineto{\pgfqpoint{1.457206in}{0.655476in}}%
\pgfpathlineto{\pgfqpoint{1.474768in}{0.641470in}}%
\pgfpathlineto{\pgfqpoint{1.492330in}{0.627972in}}%
\pgfpathlineto{\pgfqpoint{1.509891in}{0.615079in}}%
\pgfpathlineto{\pgfqpoint{1.521599in}{0.606878in}}%
\pgfpathlineto{\pgfqpoint{1.533307in}{0.599039in}}%
\pgfpathlineto{\pgfqpoint{1.545015in}{0.591608in}}%
\pgfpathlineto{\pgfqpoint{1.556723in}{0.584637in}}%
\pgfpathlineto{\pgfqpoint{1.568431in}{0.578182in}}%
\pgfpathlineto{\pgfqpoint{1.580139in}{0.572301in}}%
\pgfpathlineto{\pgfqpoint{1.591846in}{0.567060in}}%
\pgfpathlineto{\pgfqpoint{1.603554in}{0.562521in}}%
\pgfpathlineto{\pgfqpoint{1.609408in}{0.560535in}}%
\pgfpathlineto{\pgfqpoint{1.615262in}{0.558748in}}%
\pgfpathlineto{\pgfqpoint{1.621116in}{0.557167in}}%
\pgfpathlineto{\pgfqpoint{1.626970in}{0.555798in}}%
\pgfpathlineto{\pgfqpoint{1.632824in}{0.554648in}}%
\pgfpathlineto{\pgfqpoint{1.638678in}{0.553722in}}%
\pgfpathlineto{\pgfqpoint{1.644532in}{0.553023in}}%
\pgfpathlineto{\pgfqpoint{1.650386in}{0.552555in}}%
\pgfpathlineto{\pgfqpoint{1.656240in}{0.552321in}}%
\pgfpathlineto{\pgfqpoint{1.662094in}{0.552321in}}%
\pgfpathlineto{\pgfqpoint{1.667948in}{0.552555in}}%
\pgfpathlineto{\pgfqpoint{1.673801in}{0.553023in}}%
\pgfpathlineto{\pgfqpoint{1.679655in}{0.553722in}}%
\pgfpathlineto{\pgfqpoint{1.685509in}{0.554648in}}%
\pgfpathlineto{\pgfqpoint{1.691363in}{0.555798in}}%
\pgfpathlineto{\pgfqpoint{1.697217in}{0.557167in}}%
\pgfpathlineto{\pgfqpoint{1.703071in}{0.558748in}}%
\pgfpathlineto{\pgfqpoint{1.708925in}{0.560535in}}%
\pgfpathlineto{\pgfqpoint{1.720633in}{0.564698in}}%
\pgfpathlineto{\pgfqpoint{1.732341in}{0.569597in}}%
\pgfpathlineto{\pgfqpoint{1.744049in}{0.575166in}}%
\pgfpathlineto{\pgfqpoint{1.755757in}{0.581341in}}%
\pgfpathlineto{\pgfqpoint{1.767464in}{0.588062in}}%
\pgfpathlineto{\pgfqpoint{1.779172in}{0.595269in}}%
\pgfpathlineto{\pgfqpoint{1.790880in}{0.602910in}}%
\pgfpathlineto{\pgfqpoint{1.802588in}{0.610936in}}%
\pgfpathlineto{\pgfqpoint{1.814296in}{0.619302in}}%
\pgfpathlineto{\pgfqpoint{1.831858in}{0.632409in}}%
\pgfpathlineto{\pgfqpoint{1.849419in}{0.646087in}}%
\pgfpathlineto{\pgfqpoint{1.866981in}{0.660242in}}%
\pgfpathlineto{\pgfqpoint{1.884543in}{0.674800in}}%
\pgfpathlineto{\pgfqpoint{1.907959in}{0.694732in}}%
\pgfpathlineto{\pgfqpoint{1.931374in}{0.715152in}}%
\pgfpathlineto{\pgfqpoint{1.960644in}{0.741230in}}%
\pgfpathlineto{\pgfqpoint{1.989914in}{0.767796in}}%
\pgfpathlineto{\pgfqpoint{2.025037in}{0.800182in}}%
\pgfpathlineto{\pgfqpoint{2.066015in}{0.838506in}}%
\pgfpathlineto{\pgfqpoint{2.112846in}{0.882846in}}%
\pgfpathlineto{\pgfqpoint{2.165532in}{0.933244in}}%
\pgfpathlineto{\pgfqpoint{2.229925in}{0.995386in}}%
\pgfpathlineto{\pgfqpoint{2.306026in}{1.069379in}}%
\pgfpathlineto{\pgfqpoint{2.393835in}{1.155282in}}%
\pgfpathlineto{\pgfqpoint{2.505060in}{1.264646in}}%
\pgfpathlineto{\pgfqpoint{2.639700in}{1.397588in}}%
\pgfpathlineto{\pgfqpoint{2.809464in}{1.565769in}}%
\pgfpathlineto{\pgfqpoint{3.020205in}{1.775089in}}%
\pgfpathlineto{\pgfqpoint{3.114076in}{1.868458in}}%
\pgfpathlineto{\pgfqpoint{3.114076in}{1.868458in}}%
\pgfusepath{stroke}%
\end{pgfscope}%
\begin{pgfscope}%
\pgfpathrectangle{\pgfqpoint{0.198611in}{0.333208in}}{\pgfqpoint{2.921111in}{1.533583in}} %
\pgfusepath{clip}%
\pgfsetrectcap%
\pgfsetroundjoin%
\pgfsetlinewidth{0.200750pt}%
\definecolor{currentstroke}{rgb}{0.993248,0.906157,0.143936}%
\pgfsetstrokecolor{currentstroke}%
\pgfsetdash{}{0pt}%
\pgfpathmoveto{\pgfqpoint{0.226419in}{1.868458in}}%
\pgfpathlineto{\pgfqpoint{0.356667in}{1.741090in}}%
\pgfpathlineto{\pgfqpoint{0.473746in}{1.627118in}}%
\pgfpathlineto{\pgfqpoint{0.573263in}{1.530743in}}%
\pgfpathlineto{\pgfqpoint{0.661072in}{1.446199in}}%
\pgfpathlineto{\pgfqpoint{0.737173in}{1.373398in}}%
\pgfpathlineto{\pgfqpoint{0.807420in}{1.306681in}}%
\pgfpathlineto{\pgfqpoint{0.871813in}{1.246030in}}%
\pgfpathlineto{\pgfqpoint{0.930352in}{1.191411in}}%
\pgfpathlineto{\pgfqpoint{0.983038in}{1.142768in}}%
\pgfpathlineto{\pgfqpoint{1.029869in}{1.100026in}}%
\pgfpathlineto{\pgfqpoint{1.070847in}{1.063084in}}%
\pgfpathlineto{\pgfqpoint{1.111824in}{1.026649in}}%
\pgfpathlineto{\pgfqpoint{1.146948in}{0.995895in}}%
\pgfpathlineto{\pgfqpoint{1.182071in}{0.965654in}}%
\pgfpathlineto{\pgfqpoint{1.211341in}{0.940911in}}%
\pgfpathlineto{\pgfqpoint{1.240611in}{0.916646in}}%
\pgfpathlineto{\pgfqpoint{1.269880in}{0.892932in}}%
\pgfpathlineto{\pgfqpoint{1.293296in}{0.874414in}}%
\pgfpathlineto{\pgfqpoint{1.316712in}{0.856352in}}%
\pgfpathlineto{\pgfqpoint{1.340127in}{0.838804in}}%
\pgfpathlineto{\pgfqpoint{1.363543in}{0.821835in}}%
\pgfpathlineto{\pgfqpoint{1.381105in}{0.809532in}}%
\pgfpathlineto{\pgfqpoint{1.398667in}{0.797630in}}%
\pgfpathlineto{\pgfqpoint{1.416229in}{0.786167in}}%
\pgfpathlineto{\pgfqpoint{1.433790in}{0.775185in}}%
\pgfpathlineto{\pgfqpoint{1.451352in}{0.764727in}}%
\pgfpathlineto{\pgfqpoint{1.468914in}{0.754840in}}%
\pgfpathlineto{\pgfqpoint{1.480622in}{0.748591in}}%
\pgfpathlineto{\pgfqpoint{1.492330in}{0.742634in}}%
\pgfpathlineto{\pgfqpoint{1.504038in}{0.736984in}}%
\pgfpathlineto{\pgfqpoint{1.515745in}{0.731657in}}%
\pgfpathlineto{\pgfqpoint{1.527453in}{0.726669in}}%
\pgfpathlineto{\pgfqpoint{1.539161in}{0.722037in}}%
\pgfpathlineto{\pgfqpoint{1.550869in}{0.717776in}}%
\pgfpathlineto{\pgfqpoint{1.562577in}{0.713902in}}%
\pgfpathlineto{\pgfqpoint{1.574285in}{0.710430in}}%
\pgfpathlineto{\pgfqpoint{1.585993in}{0.707373in}}%
\pgfpathlineto{\pgfqpoint{1.597700in}{0.704744in}}%
\pgfpathlineto{\pgfqpoint{1.609408in}{0.702555in}}%
\pgfpathlineto{\pgfqpoint{1.621116in}{0.700815in}}%
\pgfpathlineto{\pgfqpoint{1.632824in}{0.699533in}}%
\pgfpathlineto{\pgfqpoint{1.644532in}{0.698714in}}%
\pgfpathlineto{\pgfqpoint{1.656240in}{0.698362in}}%
\pgfpathlineto{\pgfqpoint{1.667948in}{0.698479in}}%
\pgfpathlineto{\pgfqpoint{1.679655in}{0.699065in}}%
\pgfpathlineto{\pgfqpoint{1.691363in}{0.700116in}}%
\pgfpathlineto{\pgfqpoint{1.703071in}{0.701628in}}%
\pgfpathlineto{\pgfqpoint{1.714779in}{0.703594in}}%
\pgfpathlineto{\pgfqpoint{1.726487in}{0.706004in}}%
\pgfpathlineto{\pgfqpoint{1.738195in}{0.708849in}}%
\pgfpathlineto{\pgfqpoint{1.749903in}{0.712115in}}%
\pgfpathlineto{\pgfqpoint{1.761610in}{0.715790in}}%
\pgfpathlineto{\pgfqpoint{1.773318in}{0.719859in}}%
\pgfpathlineto{\pgfqpoint{1.785026in}{0.724308in}}%
\pgfpathlineto{\pgfqpoint{1.796734in}{0.729119in}}%
\pgfpathlineto{\pgfqpoint{1.808442in}{0.734279in}}%
\pgfpathlineto{\pgfqpoint{1.820150in}{0.739769in}}%
\pgfpathlineto{\pgfqpoint{1.831858in}{0.745575in}}%
\pgfpathlineto{\pgfqpoint{1.843565in}{0.751680in}}%
\pgfpathlineto{\pgfqpoint{1.861127in}{0.761366in}}%
\pgfpathlineto{\pgfqpoint{1.878689in}{0.771639in}}%
\pgfpathlineto{\pgfqpoint{1.896251in}{0.782451in}}%
\pgfpathlineto{\pgfqpoint{1.913813in}{0.793758in}}%
\pgfpathlineto{\pgfqpoint{1.931374in}{0.805518in}}%
\pgfpathlineto{\pgfqpoint{1.948936in}{0.817691in}}%
\pgfpathlineto{\pgfqpoint{1.966498in}{0.830243in}}%
\pgfpathlineto{\pgfqpoint{1.989914in}{0.847510in}}%
\pgfpathlineto{\pgfqpoint{2.013329in}{0.865322in}}%
\pgfpathlineto{\pgfqpoint{2.036745in}{0.883619in}}%
\pgfpathlineto{\pgfqpoint{2.060161in}{0.902346in}}%
\pgfpathlineto{\pgfqpoint{2.089431in}{0.926290in}}%
\pgfpathlineto{\pgfqpoint{2.118700in}{0.950754in}}%
\pgfpathlineto{\pgfqpoint{2.147970in}{0.975672in}}%
\pgfpathlineto{\pgfqpoint{2.183093in}{1.006093in}}%
\pgfpathlineto{\pgfqpoint{2.218217in}{1.037002in}}%
\pgfpathlineto{\pgfqpoint{2.259195in}{1.073591in}}%
\pgfpathlineto{\pgfqpoint{2.300172in}{1.110663in}}%
\pgfpathlineto{\pgfqpoint{2.347003in}{1.153530in}}%
\pgfpathlineto{\pgfqpoint{2.399689in}{1.202290in}}%
\pgfpathlineto{\pgfqpoint{2.458228in}{1.257017in}}%
\pgfpathlineto{\pgfqpoint{2.522621in}{1.317764in}}%
\pgfpathlineto{\pgfqpoint{2.592869in}{1.384565in}}%
\pgfpathlineto{\pgfqpoint{2.668970in}{1.457441in}}%
\pgfpathlineto{\pgfqpoint{2.756779in}{1.542053in}}%
\pgfpathlineto{\pgfqpoint{2.856295in}{1.638489in}}%
\pgfpathlineto{\pgfqpoint{2.967520in}{1.746802in}}%
\pgfpathlineto{\pgfqpoint{3.091914in}{1.868458in}}%
\pgfpathlineto{\pgfqpoint{3.091914in}{1.868458in}}%
\pgfusepath{stroke}%
\end{pgfscope}%
\begin{pgfscope}%
\pgfpathrectangle{\pgfqpoint{0.198611in}{0.333208in}}{\pgfqpoint{2.921111in}{1.533583in}} %
\pgfusepath{clip}%
\pgfsetbuttcap%
\pgfsetroundjoin%
\pgfsetlinewidth{0.501875pt}%
\definecolor{currentstroke}{rgb}{0.501961,0.501961,0.501961}%
\pgfsetstrokecolor{currentstroke}%
\pgfsetdash{{1.850000pt}{0.800000pt}}{0.000000pt}%
\pgfpathmoveto{\pgfqpoint{1.584472in}{0.331542in}}%
\pgfpathlineto{\pgfqpoint{3.119722in}{1.866792in}}%
\pgfpathlineto{\pgfqpoint{3.119722in}{1.866792in}}%
\pgfusepath{stroke}%
\end{pgfscope}%
\begin{pgfscope}%
\pgfpathrectangle{\pgfqpoint{0.198611in}{0.333208in}}{\pgfqpoint{2.921111in}{1.533583in}} %
\pgfusepath{clip}%
\pgfsetbuttcap%
\pgfsetroundjoin%
\pgfsetlinewidth{0.501875pt}%
\definecolor{currentstroke}{rgb}{0.501961,0.501961,0.501961}%
\pgfsetstrokecolor{currentstroke}%
\pgfsetdash{{1.850000pt}{0.800000pt}}{0.000000pt}%
\pgfpathmoveto{\pgfqpoint{0.198611in}{1.866792in}}%
\pgfpathlineto{\pgfqpoint{1.733861in}{0.331542in}}%
\pgfpathlineto{\pgfqpoint{1.733861in}{0.331542in}}%
\pgfusepath{stroke}%
\end{pgfscope}%
\begin{pgfscope}%
\pgfsetrectcap%
\pgfsetmiterjoin%
\pgfsetlinewidth{0.501875pt}%
\definecolor{currentstroke}{rgb}{0.000000,0.000000,0.000000}%
\pgfsetstrokecolor{currentstroke}%
\pgfsetdash{}{0pt}%
\pgfpathmoveto{\pgfqpoint{1.659167in}{0.333208in}}%
\pgfpathlineto{\pgfqpoint{1.659167in}{1.866792in}}%
\pgfusepath{stroke}%
\end{pgfscope}%
\begin{pgfscope}%
\pgfsetrectcap%
\pgfsetmiterjoin%
\pgfsetlinewidth{0.501875pt}%
\definecolor{currentstroke}{rgb}{0.000000,0.000000,0.000000}%
\pgfsetstrokecolor{currentstroke}%
\pgfsetdash{}{0pt}%
\pgfpathmoveto{\pgfqpoint{0.198611in}{0.406236in}}%
\pgfpathlineto{\pgfqpoint{3.119722in}{0.406236in}}%
\pgfusepath{stroke}%
\end{pgfscope}%
\begin{pgfscope}%
\pgfsetroundcap%
\pgfsetroundjoin%
\pgfsetlinewidth{0.501875pt}%
\definecolor{currentstroke}{rgb}{0.000000,0.000000,0.000000}%
\pgfsetstrokecolor{currentstroke}%
\pgfsetdash{}{0pt}%
\pgfpathmoveto{\pgfqpoint{1.659167in}{1.872920in}}%
\pgfpathquadraticcurveto{\pgfqpoint{1.659167in}{1.873738in}}{\pgfqpoint{1.659167in}{1.866792in}}%
\pgfusepath{stroke}%
\end{pgfscope}%
\begin{pgfscope}%
\pgfsetroundcap%
\pgfsetroundjoin%
\pgfsetlinewidth{0.501875pt}%
\definecolor{currentstroke}{rgb}{0.000000,0.000000,0.000000}%
\pgfsetstrokecolor{currentstroke}%
\pgfsetdash{}{0pt}%
\pgfpathmoveto{\pgfqpoint{1.631389in}{1.817364in}}%
\pgfpathlineto{\pgfqpoint{1.659167in}{1.872920in}}%
\pgfpathlineto{\pgfqpoint{1.686944in}{1.817364in}}%
\pgfusepath{stroke}%
\end{pgfscope}%
\begin{pgfscope}%
\pgftext[x=1.659167in,y=1.936236in,,bottom]{\rmfamily\fontsize{10.000000}{12.000000}\selectfont \(\displaystyle \omega\)}%
\end{pgfscope}%
\begin{pgfscope}%
\pgfsetroundcap%
\pgfsetroundjoin%
\pgfsetlinewidth{0.501875pt}%
\definecolor{currentstroke}{rgb}{0.000000,0.000000,0.000000}%
\pgfsetstrokecolor{currentstroke}%
\pgfsetdash{}{0pt}%
\pgfpathmoveto{\pgfqpoint{3.125847in}{0.406236in}}%
\pgfpathquadraticcurveto{\pgfqpoint{3.126667in}{0.406236in}}{\pgfqpoint{3.119722in}{0.406236in}}%
\pgfusepath{stroke}%
\end{pgfscope}%
\begin{pgfscope}%
\pgfsetroundcap%
\pgfsetroundjoin%
\pgfsetlinewidth{0.501875pt}%
\definecolor{currentstroke}{rgb}{0.000000,0.000000,0.000000}%
\pgfsetstrokecolor{currentstroke}%
\pgfsetdash{}{0pt}%
\pgfpathmoveto{\pgfqpoint{3.070292in}{0.434014in}}%
\pgfpathlineto{\pgfqpoint{3.125847in}{0.406236in}}%
\pgfpathlineto{\pgfqpoint{3.070292in}{0.378458in}}%
\pgfusepath{stroke}%
\end{pgfscope}%
\begin{pgfscope}%
\pgftext[x=3.189167in,y=0.406236in,left,]{\rmfamily\fontsize{10.000000}{12.000000}\selectfont \(\displaystyle k\)}%
\end{pgfscope}%
\begin{pgfscope}%
\pgftext[x=2.316417in,y=0.990458in,left,base]{\rmfamily\fontsize{10.000000}{12.000000}\selectfont \(\displaystyle supp~(\hat\Delta_m)\)}%
\end{pgfscope}%
\begin{pgfscope}%
\pgftext[x=1.367056in,y=1.136514in,left,base]{\rmfamily\fontsize{10.000000}{12.000000}\selectfont \(\displaystyle \hat\Delta_m^{\circledast 2}\)}%
\end{pgfscope}%
\begin{pgfscope}%
\pgfpathrectangle{\pgfqpoint{3.302292in}{0.435000in}}{\pgfqpoint{0.063333in}{1.330000in}} %
\pgfusepath{clip}%
\pgfsetbuttcap%
\pgfsetmiterjoin%
\definecolor{currentfill}{rgb}{1.000000,1.000000,1.000000}%
\pgfsetfillcolor{currentfill}%
\pgfsetlinewidth{0.010037pt}%
\definecolor{currentstroke}{rgb}{1.000000,1.000000,1.000000}%
\pgfsetstrokecolor{currentstroke}%
\pgfsetdash{}{0pt}%
\pgfpathmoveto{\pgfqpoint{3.302292in}{0.435000in}}%
\pgfpathlineto{\pgfqpoint{3.302292in}{0.439948in}}%
\pgfpathlineto{\pgfqpoint{3.302292in}{1.701667in}}%
\pgfpathlineto{\pgfqpoint{3.333958in}{1.765000in}}%
\pgfpathlineto{\pgfqpoint{3.333958in}{1.765000in}}%
\pgfpathlineto{\pgfqpoint{3.365625in}{1.701667in}}%
\pgfpathlineto{\pgfqpoint{3.365625in}{0.439948in}}%
\pgfpathlineto{\pgfqpoint{3.365625in}{0.435000in}}%
\pgfpathclose%
\pgfusepath{stroke,fill}%
\end{pgfscope}%
\begin{pgfscope}%
\pgfsys@transformshift{3.301667in}{0.435000in}%
\pgftext[left,bottom]{\pgfimage[interpolate=true,width=0.063333in,height=1.330000in]{delta_m2_twisted-img1.png}}%
\end{pgfscope}%
\begin{pgfscope}%
\pgfsetbuttcap%
\pgfsetroundjoin%
\definecolor{currentfill}{rgb}{0.000000,0.000000,0.000000}%
\pgfsetfillcolor{currentfill}%
\pgfsetlinewidth{0.803000pt}%
\definecolor{currentstroke}{rgb}{0.000000,0.000000,0.000000}%
\pgfsetstrokecolor{currentstroke}%
\pgfsetdash{}{0pt}%
\pgfsys@defobject{currentmarker}{\pgfqpoint{0.000000in}{0.000000in}}{\pgfqpoint{0.048611in}{0.000000in}}{%
\pgfpathmoveto{\pgfqpoint{0.000000in}{0.000000in}}%
\pgfpathlineto{\pgfqpoint{0.048611in}{0.000000in}}%
\pgfusepath{stroke,fill}%
}%
\begin{pgfscope}%
\pgfsys@transformshift{3.365625in}{0.435000in}%
\pgfsys@useobject{currentmarker}{}%
\end{pgfscope}%
\end{pgfscope}%
\begin{pgfscope}%
\pgftext[x=3.462847in,y=0.387172in,left,base]{\rmfamily\fontsize{10.000000}{12.000000}\selectfont \(\displaystyle -4\)}%
\end{pgfscope}%
\begin{pgfscope}%
\pgfsetbuttcap%
\pgfsetroundjoin%
\definecolor{currentfill}{rgb}{0.000000,0.000000,0.000000}%
\pgfsetfillcolor{currentfill}%
\pgfsetlinewidth{0.803000pt}%
\definecolor{currentstroke}{rgb}{0.000000,0.000000,0.000000}%
\pgfsetstrokecolor{currentstroke}%
\pgfsetdash{}{0pt}%
\pgfsys@defobject{currentmarker}{\pgfqpoint{0.000000in}{0.000000in}}{\pgfqpoint{0.048611in}{0.000000in}}{%
\pgfpathmoveto{\pgfqpoint{0.000000in}{0.000000in}}%
\pgfpathlineto{\pgfqpoint{0.048611in}{0.000000in}}%
\pgfusepath{stroke,fill}%
}%
\begin{pgfscope}%
\pgfsys@transformshift{3.365625in}{0.593333in}%
\pgfsys@useobject{currentmarker}{}%
\end{pgfscope}%
\end{pgfscope}%
\begin{pgfscope}%
\pgftext[x=3.462847in,y=0.545506in,left,base]{\rmfamily\fontsize{10.000000}{12.000000}\selectfont \(\displaystyle -3\)}%
\end{pgfscope}%
\begin{pgfscope}%
\pgfsetbuttcap%
\pgfsetroundjoin%
\definecolor{currentfill}{rgb}{0.000000,0.000000,0.000000}%
\pgfsetfillcolor{currentfill}%
\pgfsetlinewidth{0.803000pt}%
\definecolor{currentstroke}{rgb}{0.000000,0.000000,0.000000}%
\pgfsetstrokecolor{currentstroke}%
\pgfsetdash{}{0pt}%
\pgfsys@defobject{currentmarker}{\pgfqpoint{0.000000in}{0.000000in}}{\pgfqpoint{0.048611in}{0.000000in}}{%
\pgfpathmoveto{\pgfqpoint{0.000000in}{0.000000in}}%
\pgfpathlineto{\pgfqpoint{0.048611in}{0.000000in}}%
\pgfusepath{stroke,fill}%
}%
\begin{pgfscope}%
\pgfsys@transformshift{3.365625in}{0.751667in}%
\pgfsys@useobject{currentmarker}{}%
\end{pgfscope}%
\end{pgfscope}%
\begin{pgfscope}%
\pgftext[x=3.462847in,y=0.703839in,left,base]{\rmfamily\fontsize{10.000000}{12.000000}\selectfont \(\displaystyle -2\)}%
\end{pgfscope}%
\begin{pgfscope}%
\pgfsetbuttcap%
\pgfsetroundjoin%
\definecolor{currentfill}{rgb}{0.000000,0.000000,0.000000}%
\pgfsetfillcolor{currentfill}%
\pgfsetlinewidth{0.803000pt}%
\definecolor{currentstroke}{rgb}{0.000000,0.000000,0.000000}%
\pgfsetstrokecolor{currentstroke}%
\pgfsetdash{}{0pt}%
\pgfsys@defobject{currentmarker}{\pgfqpoint{0.000000in}{0.000000in}}{\pgfqpoint{0.048611in}{0.000000in}}{%
\pgfpathmoveto{\pgfqpoint{0.000000in}{0.000000in}}%
\pgfpathlineto{\pgfqpoint{0.048611in}{0.000000in}}%
\pgfusepath{stroke,fill}%
}%
\begin{pgfscope}%
\pgfsys@transformshift{3.365625in}{0.910000in}%
\pgfsys@useobject{currentmarker}{}%
\end{pgfscope}%
\end{pgfscope}%
\begin{pgfscope}%
\pgftext[x=3.462847in,y=0.862172in,left,base]{\rmfamily\fontsize{10.000000}{12.000000}\selectfont \(\displaystyle -1\)}%
\end{pgfscope}%
\begin{pgfscope}%
\pgfsetbuttcap%
\pgfsetroundjoin%
\definecolor{currentfill}{rgb}{0.000000,0.000000,0.000000}%
\pgfsetfillcolor{currentfill}%
\pgfsetlinewidth{0.803000pt}%
\definecolor{currentstroke}{rgb}{0.000000,0.000000,0.000000}%
\pgfsetstrokecolor{currentstroke}%
\pgfsetdash{}{0pt}%
\pgfsys@defobject{currentmarker}{\pgfqpoint{0.000000in}{0.000000in}}{\pgfqpoint{0.048611in}{0.000000in}}{%
\pgfpathmoveto{\pgfqpoint{0.000000in}{0.000000in}}%
\pgfpathlineto{\pgfqpoint{0.048611in}{0.000000in}}%
\pgfusepath{stroke,fill}%
}%
\begin{pgfscope}%
\pgfsys@transformshift{3.365625in}{1.068333in}%
\pgfsys@useobject{currentmarker}{}%
\end{pgfscope}%
\end{pgfscope}%
\begin{pgfscope}%
\pgftext[x=3.462847in,y=1.020506in,left,base]{\rmfamily\fontsize{10.000000}{12.000000}\selectfont \(\displaystyle 0\)}%
\end{pgfscope}%
\begin{pgfscope}%
\pgfsetbuttcap%
\pgfsetroundjoin%
\definecolor{currentfill}{rgb}{0.000000,0.000000,0.000000}%
\pgfsetfillcolor{currentfill}%
\pgfsetlinewidth{0.803000pt}%
\definecolor{currentstroke}{rgb}{0.000000,0.000000,0.000000}%
\pgfsetstrokecolor{currentstroke}%
\pgfsetdash{}{0pt}%
\pgfsys@defobject{currentmarker}{\pgfqpoint{0.000000in}{0.000000in}}{\pgfqpoint{0.048611in}{0.000000in}}{%
\pgfpathmoveto{\pgfqpoint{0.000000in}{0.000000in}}%
\pgfpathlineto{\pgfqpoint{0.048611in}{0.000000in}}%
\pgfusepath{stroke,fill}%
}%
\begin{pgfscope}%
\pgfsys@transformshift{3.365625in}{1.226667in}%
\pgfsys@useobject{currentmarker}{}%
\end{pgfscope}%
\end{pgfscope}%
\begin{pgfscope}%
\pgftext[x=3.462847in,y=1.178839in,left,base]{\rmfamily\fontsize{10.000000}{12.000000}\selectfont \(\displaystyle 1\)}%
\end{pgfscope}%
\begin{pgfscope}%
\pgfsetbuttcap%
\pgfsetroundjoin%
\definecolor{currentfill}{rgb}{0.000000,0.000000,0.000000}%
\pgfsetfillcolor{currentfill}%
\pgfsetlinewidth{0.803000pt}%
\definecolor{currentstroke}{rgb}{0.000000,0.000000,0.000000}%
\pgfsetstrokecolor{currentstroke}%
\pgfsetdash{}{0pt}%
\pgfsys@defobject{currentmarker}{\pgfqpoint{0.000000in}{0.000000in}}{\pgfqpoint{0.048611in}{0.000000in}}{%
\pgfpathmoveto{\pgfqpoint{0.000000in}{0.000000in}}%
\pgfpathlineto{\pgfqpoint{0.048611in}{0.000000in}}%
\pgfusepath{stroke,fill}%
}%
\begin{pgfscope}%
\pgfsys@transformshift{3.365625in}{1.385000in}%
\pgfsys@useobject{currentmarker}{}%
\end{pgfscope}%
\end{pgfscope}%
\begin{pgfscope}%
\pgftext[x=3.462847in,y=1.337172in,left,base]{\rmfamily\fontsize{10.000000}{12.000000}\selectfont \(\displaystyle 2\)}%
\end{pgfscope}%
\begin{pgfscope}%
\pgfsetbuttcap%
\pgfsetroundjoin%
\definecolor{currentfill}{rgb}{0.000000,0.000000,0.000000}%
\pgfsetfillcolor{currentfill}%
\pgfsetlinewidth{0.803000pt}%
\definecolor{currentstroke}{rgb}{0.000000,0.000000,0.000000}%
\pgfsetstrokecolor{currentstroke}%
\pgfsetdash{}{0pt}%
\pgfsys@defobject{currentmarker}{\pgfqpoint{0.000000in}{0.000000in}}{\pgfqpoint{0.048611in}{0.000000in}}{%
\pgfpathmoveto{\pgfqpoint{0.000000in}{0.000000in}}%
\pgfpathlineto{\pgfqpoint{0.048611in}{0.000000in}}%
\pgfusepath{stroke,fill}%
}%
\begin{pgfscope}%
\pgfsys@transformshift{3.365625in}{1.543333in}%
\pgfsys@useobject{currentmarker}{}%
\end{pgfscope}%
\end{pgfscope}%
\begin{pgfscope}%
\pgftext[x=3.462847in,y=1.495506in,left,base]{\rmfamily\fontsize{10.000000}{12.000000}\selectfont \(\displaystyle 3\)}%
\end{pgfscope}%
\begin{pgfscope}%
\pgfsetbuttcap%
\pgfsetroundjoin%
\definecolor{currentfill}{rgb}{0.000000,0.000000,0.000000}%
\pgfsetfillcolor{currentfill}%
\pgfsetlinewidth{0.803000pt}%
\definecolor{currentstroke}{rgb}{0.000000,0.000000,0.000000}%
\pgfsetstrokecolor{currentstroke}%
\pgfsetdash{}{0pt}%
\pgfsys@defobject{currentmarker}{\pgfqpoint{0.000000in}{0.000000in}}{\pgfqpoint{0.048611in}{0.000000in}}{%
\pgfpathmoveto{\pgfqpoint{0.000000in}{0.000000in}}%
\pgfpathlineto{\pgfqpoint{0.048611in}{0.000000in}}%
\pgfusepath{stroke,fill}%
}%
\begin{pgfscope}%
\pgfsys@transformshift{3.365625in}{1.701667in}%
\pgfsys@useobject{currentmarker}{}%
\end{pgfscope}%
\end{pgfscope}%
\begin{pgfscope}%
\pgftext[x=3.462847in,y=1.653839in,left,base]{\rmfamily\fontsize{10.000000}{12.000000}\selectfont \(\displaystyle 4\)}%
\end{pgfscope}%
\begin{pgfscope}%
\pgfsetbuttcap%
\pgfsetmiterjoin%
\pgfsetlinewidth{0.501875pt}%
\definecolor{currentstroke}{rgb}{0.000000,0.000000,0.000000}%
\pgfsetstrokecolor{currentstroke}%
\pgfsetdash{}{0pt}%
\pgfpathmoveto{\pgfqpoint{3.302292in}{0.435000in}}%
\pgfpathlineto{\pgfqpoint{3.302292in}{0.439948in}}%
\pgfpathlineto{\pgfqpoint{3.302292in}{1.701667in}}%
\pgfpathlineto{\pgfqpoint{3.333958in}{1.765000in}}%
\pgfpathlineto{\pgfqpoint{3.333958in}{1.765000in}}%
\pgfpathlineto{\pgfqpoint{3.365625in}{1.701667in}}%
\pgfpathlineto{\pgfqpoint{3.365625in}{0.439948in}}%
\pgfpathlineto{\pgfqpoint{3.365625in}{0.435000in}}%
\pgfpathclose%
\pgfusepath{stroke}%
\end{pgfscope}%
\end{pgfpicture}%
\makeatother%
\endgroup%
} %
        \caption{Plot von $\hat\Delta_m^{\circledast 2}(-\cdot)$ und $\hat\Delta_m(-\cdot)$. Wieder liegt der Träger von $\hat\Delta_m^{\circledast 2}(-\cdot)$ oberhalb der $2m$-Massenschale.
        }
        \label{fig:delta_2m_twisted}
    \end{minipage}\hfill
    \begin{minipage}{0.45\textwidth}
        \centering
        \resizebox{\textwidth}{!}{%% Creator: Matplotlib, PGF backend
%%
%% To include the figure in your LaTeX document, write
%%   \input{<filename>.pgf}
%%
%% Make sure the required packages are loaded in your preamble
%%   \usepackage{pgf}
%%
%% Figures using additional raster images can only be included by \input if
%% they are in the same directory as the main LaTeX file. For loading figures
%% from other directories you can use the `import` package
%%   \usepackage{import}
%% and then include the figures with
%%   \import{<path to file>}{<filename>.pgf}
%%
%% Matplotlib used the following preamble
%%   \usepackage[utf8x]{inputenc}
%%   \usepackage[T1]{fontenc}
%%   \usepackage{amssymb}
%%
\begingroup%
\makeatletter%
\begin{pgfpicture}%
\pgfpathrectangle{\pgfpointorigin}{\pgfqpoint{4.000000in}{2.200000in}}%
\pgfusepath{use as bounding box, clip}%
\begin{pgfscope}%
\pgfsetbuttcap%
\pgfsetmiterjoin%
\definecolor{currentfill}{rgb}{1.000000,1.000000,1.000000}%
\pgfsetfillcolor{currentfill}%
\pgfsetlinewidth{0.000000pt}%
\definecolor{currentstroke}{rgb}{1.000000,1.000000,1.000000}%
\pgfsetstrokecolor{currentstroke}%
\pgfsetdash{}{0pt}%
\pgfpathmoveto{\pgfqpoint{0.000000in}{0.000000in}}%
\pgfpathlineto{\pgfqpoint{4.000000in}{0.000000in}}%
\pgfpathlineto{\pgfqpoint{4.000000in}{2.200000in}}%
\pgfpathlineto{\pgfqpoint{0.000000in}{2.200000in}}%
\pgfpathclose%
\pgfusepath{fill}%
\end{pgfscope}%
\begin{pgfscope}%
\pgfsetbuttcap%
\pgfsetmiterjoin%
\definecolor{currentfill}{rgb}{1.000000,1.000000,1.000000}%
\pgfsetfillcolor{currentfill}%
\pgfsetlinewidth{0.000000pt}%
\definecolor{currentstroke}{rgb}{0.000000,0.000000,0.000000}%
\pgfsetstrokecolor{currentstroke}%
\pgfsetstrokeopacity{0.000000}%
\pgfsetdash{}{0pt}%
\pgfpathmoveto{\pgfqpoint{0.198611in}{0.198611in}}%
\pgfpathlineto{\pgfqpoint{3.801389in}{0.198611in}}%
\pgfpathlineto{\pgfqpoint{3.801389in}{2.001389in}}%
\pgfpathlineto{\pgfqpoint{0.198611in}{2.001389in}}%
\pgfpathclose%
\pgfusepath{fill}%
\end{pgfscope}%
\begin{pgfscope}%
\pgfpathrectangle{\pgfqpoint{0.198611in}{0.198611in}}{\pgfqpoint{3.602778in}{1.802778in}} %
\pgfusepath{clip}%
\pgfsetbuttcap%
\pgfsetroundjoin%
\pgfsetlinewidth{0.501875pt}%
\definecolor{currentstroke}{rgb}{0.501961,0.501961,0.501961}%
\pgfsetstrokecolor{currentstroke}%
\pgfsetdash{{1.850000pt}{0.800000pt}}{0.000000pt}%
\pgfpathmoveto{\pgfqpoint{0.507421in}{0.198611in}}%
\pgfpathlineto{\pgfqpoint{0.507421in}{2.001389in}}%
\pgfusepath{stroke}%
\end{pgfscope}%
\begin{pgfscope}%
\pgfpathrectangle{\pgfqpoint{0.198611in}{0.198611in}}{\pgfqpoint{3.602778in}{1.802778in}} %
\pgfusepath{clip}%
\pgfsetrectcap%
\pgfsetroundjoin%
\pgfsetlinewidth{1.003750pt}%
\definecolor{currentstroke}{rgb}{0.894118,0.101961,0.109804}%
\pgfsetstrokecolor{currentstroke}%
\pgfsetdash{}{0pt}%
\pgfpathmoveto{\pgfqpoint{0.508967in}{2.015278in}}%
\pgfpathlineto{\pgfqpoint{0.512367in}{1.239514in}}%
\pgfpathlineto{\pgfqpoint{0.516488in}{1.075048in}}%
\pgfpathlineto{\pgfqpoint{0.520610in}{0.997750in}}%
\pgfpathlineto{\pgfqpoint{0.528853in}{0.919550in}}%
\pgfpathlineto{\pgfqpoint{0.537096in}{0.878538in}}%
\pgfpathlineto{\pgfqpoint{0.545339in}{0.852596in}}%
\pgfpathlineto{\pgfqpoint{0.557704in}{0.827014in}}%
\pgfpathlineto{\pgfqpoint{0.570069in}{0.809537in}}%
\pgfpathlineto{\pgfqpoint{0.586555in}{0.792473in}}%
\pgfpathlineto{\pgfqpoint{0.607163in}{0.776035in}}%
\pgfpathlineto{\pgfqpoint{0.644257in}{0.751554in}}%
\pgfpathlineto{\pgfqpoint{0.722567in}{0.700702in}}%
\pgfpathlineto{\pgfqpoint{0.767905in}{0.667245in}}%
\pgfpathlineto{\pgfqpoint{0.813242in}{0.630138in}}%
\pgfpathlineto{\pgfqpoint{0.866823in}{0.582041in}}%
\pgfpathlineto{\pgfqpoint{0.936889in}{0.514374in}}%
\pgfpathlineto{\pgfqpoint{1.044051in}{0.410719in}}%
\pgfpathlineto{\pgfqpoint{1.089388in}{0.371484in}}%
\pgfpathlineto{\pgfqpoint{1.126482in}{0.343447in}}%
\pgfpathlineto{\pgfqpoint{1.159455in}{0.322565in}}%
\pgfpathlineto{\pgfqpoint{1.188306in}{0.308037in}}%
\pgfpathlineto{\pgfqpoint{1.213035in}{0.298752in}}%
\pgfpathlineto{\pgfqpoint{1.237765in}{0.292685in}}%
\pgfpathlineto{\pgfqpoint{1.262494in}{0.290085in}}%
\pgfpathlineto{\pgfqpoint{1.283102in}{0.290717in}}%
\pgfpathlineto{\pgfqpoint{1.303710in}{0.293999in}}%
\pgfpathlineto{\pgfqpoint{1.324318in}{0.300000in}}%
\pgfpathlineto{\pgfqpoint{1.344926in}{0.308762in}}%
\pgfpathlineto{\pgfqpoint{1.369656in}{0.322934in}}%
\pgfpathlineto{\pgfqpoint{1.394385in}{0.341047in}}%
\pgfpathlineto{\pgfqpoint{1.419115in}{0.362977in}}%
\pgfpathlineto{\pgfqpoint{1.447966in}{0.393105in}}%
\pgfpathlineto{\pgfqpoint{1.476817in}{0.427669in}}%
\pgfpathlineto{\pgfqpoint{1.509789in}{0.471785in}}%
\pgfpathlineto{\pgfqpoint{1.555127in}{0.538306in}}%
\pgfpathlineto{\pgfqpoint{1.670531in}{0.711358in}}%
\pgfpathlineto{\pgfqpoint{1.703504in}{0.753514in}}%
\pgfpathlineto{\pgfqpoint{1.728233in}{0.780752in}}%
\pgfpathlineto{\pgfqpoint{1.752963in}{0.803368in}}%
\pgfpathlineto{\pgfqpoint{1.773571in}{0.818170in}}%
\pgfpathlineto{\pgfqpoint{1.794179in}{0.828905in}}%
\pgfpathlineto{\pgfqpoint{1.810665in}{0.834357in}}%
\pgfpathlineto{\pgfqpoint{1.827151in}{0.836879in}}%
\pgfpathlineto{\pgfqpoint{1.843637in}{0.836372in}}%
\pgfpathlineto{\pgfqpoint{1.860124in}{0.832770in}}%
\pgfpathlineto{\pgfqpoint{1.876610in}{0.826041in}}%
\pgfpathlineto{\pgfqpoint{1.893096in}{0.816188in}}%
\pgfpathlineto{\pgfqpoint{1.909583in}{0.803257in}}%
\pgfpathlineto{\pgfqpoint{1.930191in}{0.782894in}}%
\pgfpathlineto{\pgfqpoint{1.950799in}{0.758109in}}%
\pgfpathlineto{\pgfqpoint{1.975528in}{0.723038in}}%
\pgfpathlineto{\pgfqpoint{2.004379in}{0.675812in}}%
\pgfpathlineto{\pgfqpoint{2.037352in}{0.615530in}}%
\pgfpathlineto{\pgfqpoint{2.156878in}{0.389578in}}%
\pgfpathlineto{\pgfqpoint{2.181607in}{0.351914in}}%
\pgfpathlineto{\pgfqpoint{2.202215in}{0.325446in}}%
\pgfpathlineto{\pgfqpoint{2.222823in}{0.304249in}}%
\pgfpathlineto{\pgfqpoint{2.239309in}{0.291526in}}%
\pgfpathlineto{\pgfqpoint{2.255796in}{0.282869in}}%
\pgfpathlineto{\pgfqpoint{2.272282in}{0.278496in}}%
\pgfpathlineto{\pgfqpoint{2.284647in}{0.278124in}}%
\pgfpathlineto{\pgfqpoint{2.297011in}{0.280291in}}%
\pgfpathlineto{\pgfqpoint{2.309376in}{0.285014in}}%
\pgfpathlineto{\pgfqpoint{2.325863in}{0.295273in}}%
\pgfpathlineto{\pgfqpoint{2.342349in}{0.309979in}}%
\pgfpathlineto{\pgfqpoint{2.358835in}{0.328969in}}%
\pgfpathlineto{\pgfqpoint{2.379443in}{0.358351in}}%
\pgfpathlineto{\pgfqpoint{2.400051in}{0.393393in}}%
\pgfpathlineto{\pgfqpoint{2.424781in}{0.441728in}}%
\pgfpathlineto{\pgfqpoint{2.457753in}{0.513890in}}%
\pgfpathlineto{\pgfqpoint{2.544306in}{0.708885in}}%
\pgfpathlineto{\pgfqpoint{2.569036in}{0.755540in}}%
\pgfpathlineto{\pgfqpoint{2.589644in}{0.788173in}}%
\pgfpathlineto{\pgfqpoint{2.606130in}{0.809348in}}%
\pgfpathlineto{\pgfqpoint{2.622616in}{0.825585in}}%
\pgfpathlineto{\pgfqpoint{2.634981in}{0.834264in}}%
\pgfpathlineto{\pgfqpoint{2.647346in}{0.839787in}}%
\pgfpathlineto{\pgfqpoint{2.659711in}{0.842047in}}%
\pgfpathlineto{\pgfqpoint{2.672075in}{0.840976in}}%
\pgfpathlineto{\pgfqpoint{2.684440in}{0.836543in}}%
\pgfpathlineto{\pgfqpoint{2.696805in}{0.828761in}}%
\pgfpathlineto{\pgfqpoint{2.709170in}{0.817680in}}%
\pgfpathlineto{\pgfqpoint{2.725656in}{0.797945in}}%
\pgfpathlineto{\pgfqpoint{2.742142in}{0.772858in}}%
\pgfpathlineto{\pgfqpoint{2.762750in}{0.734702in}}%
\pgfpathlineto{\pgfqpoint{2.787480in}{0.680551in}}%
\pgfpathlineto{\pgfqpoint{2.816331in}{0.609133in}}%
\pgfpathlineto{\pgfqpoint{2.894641in}{0.409492in}}%
\pgfpathlineto{\pgfqpoint{2.919370in}{0.357397in}}%
\pgfpathlineto{\pgfqpoint{2.935857in}{0.328508in}}%
\pgfpathlineto{\pgfqpoint{2.952343in}{0.305241in}}%
\pgfpathlineto{\pgfqpoint{2.964708in}{0.291893in}}%
\pgfpathlineto{\pgfqpoint{2.977073in}{0.282325in}}%
\pgfpathlineto{\pgfqpoint{2.989437in}{0.276720in}}%
\pgfpathlineto{\pgfqpoint{3.001802in}{0.275204in}}%
\pgfpathlineto{\pgfqpoint{3.014167in}{0.277847in}}%
\pgfpathlineto{\pgfqpoint{3.026531in}{0.284657in}}%
\pgfpathlineto{\pgfqpoint{3.038896in}{0.295579in}}%
\pgfpathlineto{\pgfqpoint{3.051261in}{0.310497in}}%
\pgfpathlineto{\pgfqpoint{3.067747in}{0.336280in}}%
\pgfpathlineto{\pgfqpoint{3.084234in}{0.368236in}}%
\pgfpathlineto{\pgfqpoint{3.104842in}{0.415650in}}%
\pgfpathlineto{\pgfqpoint{3.129571in}{0.480943in}}%
\pgfpathlineto{\pgfqpoint{3.224367in}{0.742097in}}%
\pgfpathlineto{\pgfqpoint{3.244975in}{0.784575in}}%
\pgfpathlineto{\pgfqpoint{3.261462in}{0.811228in}}%
\pgfpathlineto{\pgfqpoint{3.273826in}{0.826335in}}%
\pgfpathlineto{\pgfqpoint{3.286191in}{0.836913in}}%
\pgfpathlineto{\pgfqpoint{3.298556in}{0.842731in}}%
\pgfpathlineto{\pgfqpoint{3.306799in}{0.843886in}}%
\pgfpathlineto{\pgfqpoint{3.315042in}{0.842833in}}%
\pgfpathlineto{\pgfqpoint{3.323285in}{0.839564in}}%
\pgfpathlineto{\pgfqpoint{3.335650in}{0.830533in}}%
\pgfpathlineto{\pgfqpoint{3.348015in}{0.816653in}}%
\pgfpathlineto{\pgfqpoint{3.360380in}{0.798122in}}%
\pgfpathlineto{\pgfqpoint{3.376866in}{0.766686in}}%
\pgfpathlineto{\pgfqpoint{3.393352in}{0.728412in}}%
\pgfpathlineto{\pgfqpoint{3.413960in}{0.672732in}}%
\pgfpathlineto{\pgfqpoint{3.442811in}{0.585136in}}%
\pgfpathlineto{\pgfqpoint{3.496392in}{0.420707in}}%
\pgfpathlineto{\pgfqpoint{3.517000in}{0.367010in}}%
\pgfpathlineto{\pgfqpoint{3.533486in}{0.331208in}}%
\pgfpathlineto{\pgfqpoint{3.549972in}{0.303232in}}%
\pgfpathlineto{\pgfqpoint{3.562337in}{0.288053in}}%
\pgfpathlineto{\pgfqpoint{3.574702in}{0.278251in}}%
\pgfpathlineto{\pgfqpoint{3.582945in}{0.274831in}}%
\pgfpathlineto{\pgfqpoint{3.591188in}{0.273960in}}%
\pgfpathlineto{\pgfqpoint{3.599431in}{0.275663in}}%
\pgfpathlineto{\pgfqpoint{3.607675in}{0.279941in}}%
\pgfpathlineto{\pgfqpoint{3.620039in}{0.291130in}}%
\pgfpathlineto{\pgfqpoint{3.632404in}{0.307881in}}%
\pgfpathlineto{\pgfqpoint{3.644769in}{0.329898in}}%
\pgfpathlineto{\pgfqpoint{3.661255in}{0.366721in}}%
\pgfpathlineto{\pgfqpoint{3.681863in}{0.422910in}}%
\pgfpathlineto{\pgfqpoint{3.706593in}{0.501009in}}%
\pgfpathlineto{\pgfqpoint{3.772538in}{0.716534in}}%
\pgfpathlineto{\pgfqpoint{3.793146in}{0.770446in}}%
\pgfpathlineto{\pgfqpoint{3.801389in}{0.788581in}}%
\pgfpathlineto{\pgfqpoint{3.801389in}{0.788581in}}%
\pgfusepath{stroke}%
\end{pgfscope}%
\begin{pgfscope}%
\pgfpathrectangle{\pgfqpoint{0.198611in}{0.198611in}}{\pgfqpoint{3.602778in}{1.802778in}} %
\pgfusepath{clip}%
\pgfsetrectcap%
\pgfsetroundjoin%
\pgfsetlinewidth{1.003750pt}%
\definecolor{currentstroke}{rgb}{0.894118,0.101961,0.109804}%
\pgfsetstrokecolor{currentstroke}%
\pgfsetdash{}{0pt}%
\pgfpathmoveto{\pgfqpoint{0.184722in}{0.559167in}}%
\pgfpathlineto{\pgfqpoint{0.324422in}{0.559167in}}%
\pgfpathlineto{\pgfqpoint{0.507421in}{0.559167in}}%
\pgfusepath{stroke}%
\end{pgfscope}%
\begin{pgfscope}%
\pgfpathrectangle{\pgfqpoint{0.198611in}{0.198611in}}{\pgfqpoint{3.602778in}{1.802778in}} %
\pgfusepath{clip}%
\pgfsetbuttcap%
\pgfsetroundjoin%
\pgfsetlinewidth{0.501875pt}%
\definecolor{currentstroke}{rgb}{0.501961,0.501961,0.501961}%
\pgfsetstrokecolor{currentstroke}%
\pgfsetdash{{1.850000pt}{0.800000pt}}{0.000000pt}%
\pgfpathmoveto{\pgfqpoint{0.184722in}{0.847611in}}%
\pgfpathlineto{\pgfqpoint{3.815278in}{0.847611in}}%
\pgfusepath{stroke}%
\end{pgfscope}%
\begin{pgfscope}%
\pgfsetrectcap%
\pgfsetmiterjoin%
\pgfsetlinewidth{0.501875pt}%
\definecolor{currentstroke}{rgb}{0.000000,0.000000,0.000000}%
\pgfsetstrokecolor{currentstroke}%
\pgfsetdash{}{0pt}%
\pgfpathmoveto{\pgfqpoint{0.301548in}{0.198611in}}%
\pgfpathlineto{\pgfqpoint{0.301548in}{2.001389in}}%
\pgfusepath{stroke}%
\end{pgfscope}%
\begin{pgfscope}%
\pgfsetrectcap%
\pgfsetmiterjoin%
\pgfsetlinewidth{0.501875pt}%
\definecolor{currentstroke}{rgb}{0.000000,0.000000,0.000000}%
\pgfsetstrokecolor{currentstroke}%
\pgfsetdash{}{0pt}%
\pgfpathmoveto{\pgfqpoint{0.198611in}{0.559167in}}%
\pgfpathlineto{\pgfqpoint{3.801389in}{0.559167in}}%
\pgfusepath{stroke}%
\end{pgfscope}%
\begin{pgfscope}%
\pgftext[x=0.548595in,y=0.342833in,left,base]{\rmfamily\fontsize{10.000000}{12.000000}\selectfont \(\displaystyle \omega = 2 m\)}%
\end{pgfscope}%
\begin{pgfscope}%
\pgftext[x=0.548595in,y=1.136056in,left,base]{\rmfamily\fontsize{10.000000}{12.000000}\selectfont \(\displaystyle \approx \frac{1}{\sqrt{\omega}}\)}%
\end{pgfscope}%
\begin{pgfscope}%
\pgftext[x=2.566151in,y=0.876456in,left,base]{\rmfamily\fontsize{10.000000}{12.000000}\selectfont \(\displaystyle \approx 2 \cos\left(\frac{\omega^2}{2}\right)\)}%
\end{pgfscope}%
\begin{pgfscope}%
\pgfsetroundcap%
\pgfsetroundjoin%
\pgfsetlinewidth{0.501875pt}%
\definecolor{currentstroke}{rgb}{0.000000,0.000000,0.000000}%
\pgfsetstrokecolor{currentstroke}%
\pgfsetdash{}{0pt}%
\pgfpathmoveto{\pgfqpoint{0.301548in}{2.007506in}}%
\pgfpathquadraticcurveto{\pgfqpoint{0.301548in}{2.008330in}}{\pgfqpoint{0.301548in}{2.001389in}}%
\pgfusepath{stroke}%
\end{pgfscope}%
\begin{pgfscope}%
\pgfsetroundcap%
\pgfsetroundjoin%
\pgfsetlinewidth{0.501875pt}%
\definecolor{currentstroke}{rgb}{0.000000,0.000000,0.000000}%
\pgfsetstrokecolor{currentstroke}%
\pgfsetdash{}{0pt}%
\pgfpathmoveto{\pgfqpoint{0.273770in}{1.951951in}}%
\pgfpathlineto{\pgfqpoint{0.301548in}{2.007506in}}%
\pgfpathlineto{\pgfqpoint{0.329325in}{1.951951in}}%
\pgfusepath{stroke}%
\end{pgfscope}%
\begin{pgfscope}%
\pgftext[x=0.301548in,y=2.070833in,,bottom]{\rmfamily\fontsize{10.000000}{12.000000}\selectfont \(\displaystyle \hat\Delta^{\circledast 2} ~(\omega, 0)\)}%
\end{pgfscope}%
\begin{pgfscope}%
\pgfsetroundcap%
\pgfsetroundjoin%
\pgfsetlinewidth{0.501875pt}%
\definecolor{currentstroke}{rgb}{0.000000,0.000000,0.000000}%
\pgfsetstrokecolor{currentstroke}%
\pgfsetdash{}{0pt}%
\pgfpathmoveto{\pgfqpoint{3.807500in}{0.559167in}}%
\pgfpathquadraticcurveto{\pgfqpoint{3.808327in}{0.559167in}}{\pgfqpoint{3.801389in}{0.559167in}}%
\pgfusepath{stroke}%
\end{pgfscope}%
\begin{pgfscope}%
\pgfsetroundcap%
\pgfsetroundjoin%
\pgfsetlinewidth{0.501875pt}%
\definecolor{currentstroke}{rgb}{0.000000,0.000000,0.000000}%
\pgfsetstrokecolor{currentstroke}%
\pgfsetdash{}{0pt}%
\pgfpathmoveto{\pgfqpoint{3.751945in}{0.586944in}}%
\pgfpathlineto{\pgfqpoint{3.807500in}{0.559167in}}%
\pgfpathlineto{\pgfqpoint{3.751945in}{0.531389in}}%
\pgfusepath{stroke}%
\end{pgfscope}%
\begin{pgfscope}%
\pgftext[x=3.870833in,y=0.559167in,left,]{\rmfamily\fontsize{10.000000}{12.000000}\selectfont \(\displaystyle \omega\)}%
\end{pgfscope}%
\end{pgfpicture}%
\makeatother%
\endgroup%
}
        \caption{Plot von $\left.\hat{\Delta}_m^{\circledast 2}\right|_{k=0}(-\cdot)$ um das asymptotische Verhalten für $\omega \rightarrow 0$ und $\omega \rightarrow \infty$ zu verdeutlichen}
        \label{fig:delta_2m_twisted_k0}
    \end{minipage}
\end{figure}

\subsubsection*{Fall $|s| > 1$}
Wir bedienen uns wieder genau des selben Arguments, wie in \cref{eq:delta_m2_s>1} und dürfen direkt schreiben:

\begin{equation}
    \left\langle \rwhat{\Delta}_m^{\circledast 2}, \hat\psi_{ast}^{(3)}\right\rangle
    = \left\langle \rwhat{\Delta}_m^{\circledast 2} (-\cdot), \hat\psi_{ast}\right\rangle
    = 0 \condition{für alle $a$ klein genug}
\label{eq:delta_m2_twisted_s>1}
\end{equation}


\subsubsection*{Fall $|s| < 1, (x,t) \neq 0$}
Da
$\rwhat\Delta_m^{\circledast 2} = \rwhat\Delta_m^{* 2} \cos(\dots)$ können wir direkt mit dem Ausdruck (\ref{eq:psi_ast_delta_m2_s<1}) $\cdot \cos$ weiter arbeiten und genau die selben Abschätzungen machen. $\cos(\varphi)$ ist bekanntermaßen beschränkt.

\begin{align}
    \left\langle \rwhat{\Delta}_m^{\circledast 2}, \rwhat{\psi}_{ast}^{(3)}
    \right\rangle
    &=
    \left\langle \rwhat{\Delta}_m^{\circledast 2}(-\cdot), \rwhat{\psi}_{ast}
    \right\rangle
    \nonumber \\ &=
     2 a^{-\frac{3}{4}} \int \frac{
    \hat\psi_1(\omega)~ \hat\psi_2(k) \left(
    \omega^2 \left(\Delta s - 2 a^{\frac{1}{2}} k s - ak^2
            \right) - 2a^2m^2
    \right)
     }
     {
        \sqrt{\Delta s -2a^{\frac{1}{2}}ks - ak^2}
            \sqrt{\Delta s \omega^2 -2a^{\frac{1}{2}} \omega^2 k s
                    - a\omega^2k^2-4 a^2 m^2}
     }
     \nonumber \\ &\kern 2em\cdot
     \Theta(\cdots)
     \cos(\varphi(\omega^2-k^2))
     e^{-i \omega \left(\frac{t'-sx'}{a}+k \frac{x'}{\sqrt{a}}\right)}
     \d \omega \d k
     \nonumber \\ &\leq
     2 a^{-\frac{3}{4}} \int
     \omega \hat\psi_1(\omega)\, \hat\psi_2(k)
     e^{-i \omega \left(\frac{t'-sx'}{a}+k \frac{x'}{\sqrt{a}}\right)}
     \d \omega \d k
     \nonumber \\ &\sim
     O(a^k) ~~ \forall k \hiderel \in \mathbb{N}
\label{eq:delta_m2_twisted_s<1_x_neq_0}
\end{align}
% \begin{dmath}
% \label{eq:delta_m2_twisted_s<1}
%     \left\langle \rwhat{\Delta}_m^{\circledast 2}, \rwhat{\psi}_{ast}^{(3)}
%     \right\rangle
%     =
%     \left\langle \rwhat{\Delta}_m^{\circledast 2}(-\cdot), \rwhat{\psi}_{ast}
%     \right\rangle
%     =
%      2 a^{-\frac{3}{4}} \int \frac{
%     \hat\psi_1(\omega)~ \hat\psi_2(k) \left(
%     \omega^2 \left(\Delta s - 2 a^{\frac{1}{2}} k s - ak^2
%             \right) - 3a^2m^2
%     \right)
%      }
%      {
%         \sqrt{\Delta s -2a^{\frac{1}{2}}ks - ak^2}
%             \sqrt{\Delta s \omega^2 -2a^{\frac{1}{2}} \omega^2 k s
%                     - a\omega^2k^2-4 a^2 m^2}
%      }
%      \cdot
%      \Theta(\cdots)
%      \cos(\varphi(\omega^2-k^2))
%      e^{-i \omega \left(\frac{t'-sx'}{a}+k \frac{x'}{\sqrt{a}}\right)}
%      \d \omega \d k
%      \leq
%      2 a^{-\frac{3}{4}} \int
%      \omega \hat\psi_1(\omega)\, \hat\psi_2(k)
%      e^{-i \omega \left(\frac{t'-sx'}{a}+k \frac{x'}{\sqrt{a}}\right)}
%      \d \omega \d k
%      \sim O(a^k) ~~ \forall k \hiderel \in \mathbb{N}
% \label{eq:delta_m2_twisted_s<1_x_neq_0}
% \end{dmath}


\subsubsection*{Fall $|s| < 1, (x,t) = 0$}
In diesem Fall lassen wir den $\cos$-Faktor in \cref{eq:delta_m2_twisted_s<1_x_neq_0} in der ersten Ungleichung nicht heraus fallen, dafür wird der $e^\cdots$-Faktor 1. Den $\cos$-Faktor schreiben wir als Summe von $e$-Funktionen und erhalten

% \begin{equation}
% \begin{aligned}
%     \left\langle \rwhat{\Delta}_m^{\circledast 2}, \rwhat{\psi}_{ast}
%     \right\rangle
%     \\&=
%     2 a^{-\frac{3}{4}} \int
%     \omega \hat\psi_1(\omega) \hat\psi_2(k)
%     \left\{
%         \exp\left(i a^{-2} \frac{
%         \omega^2 (\Delta s - 2 a^{\frac{1}{2}} k s - a k^2)
%         }{2}
%         \sqrt{\frac{1}{4} - \frac{a^2 m^2}{\omega^2(
%             \Delta s - 2 a^{\frac{1}{2}} k s - ak^2
%         )}}
%         \right)
%         \\ &+
%         \exp (-i \cdots)
%     \right\}
%     \d \omega \d k
%     \\ &=
%     2 a^{-\frac{3}{4}} \int
%     \cancel{\sqrt{\omega}} \hat\psi_1(\sqrt{\omega}) \hat\psi_2(k)
%     \left\{
%         \exp\left(i a^{-2} \frac{
%         \omega (\Delta s - 2 a^{\frac{1}{2}} k s - a k^2)
%         }{2}
%         \sqrt{\frac{1}{4} - \frac{a^2 m^2}{\omega(
%             \Delta s - 2 a^{\frac{1}{2}} k s - ak^2
%         )}}
%         \right)
%         \\ &+ \mathrm{c.c.}
%     \right\}
%     \frac{{\d \omega \d k}}{\cancel{\sqrt{\omega}}}
%     \\ &=
%     2 a^{-\frac{3}{4}} \int \left\{
%         \int
%         \hat\psi_1(\sqrt\omega)
%         \left\{
%             \exp
%             \left(ia^{-2} \left(\frac{\omega \Delta s}{4}
%                                 + O\left(a^{\frac{1}{2}}\right)\right)
%             \right)
%             \\ &+
%             \mathrm{c.c.}
%         \right\}
%         \d \omega
%     \right\}
%     \hat \psi_2(k) \d k
%     \\ &=
%     2 a^{-\frac{3}{4}} \int
%     \underbrace{
%     \left\{
%     (\hat\psi_1 \circ \sqrt{\cdot })^\vee
%     \left(\frac{\Delta s}{4a^2}\right)
%      + (\hat\psi_1 \circ \sqrt{\cdot })^\vee
%     \left(-\frac{\Delta s}{4a^2}\right)
%     + \mathrm{c.c.}
%     \right\}}_{
%     \sim O(a^k) ~\forall k \in \mathbb{N}
%     }
%     \psi_2(k) \d k
%     \\ &
%     \sim O(a^k) ~~ \forall k \hiderel \in \mathbb{N}
% \label{eq:delta_m2_twisted_s<1_x=0}
% \end{aligned}
% \end{equation}
\begin{dmath}
    \left\langle \rwhat{\Delta}_m^{\circledast 2}(-\cdot), \rwhat{\psi}_{ast}
    \right\rangle
    =
    2 a^{-\frac{3}{4}} \bigint
    \omega \hat\psi_1(\omega) \hat\psi_2(k)
    \left\{
        \exp\left(i a^{-2} \frac{
        \omega^2 (\Delta s - 2 a^{\frac{1}{2}} k s - a k^2)
        }{2}
        \sqrt{\frac{1}{4} - \frac{a^2 m^2}{\omega^2(
            \Delta s - 2 a^{\frac{1}{2}} k s - ak^2
        )}}
        \right)
        +\exp (-i \cdots)
    \right\}
    \d \omega \d k
    =
    2 a^{-\frac{3}{4}} \bigint
    \cancel{\sqrt{\omega}} \hat\psi_1(\sqrt{\omega}) \hat\psi_2(k)
    \left\{
        \exp\left(i a^{-2} \frac{
        \omega (\Delta s - 2 a^{\frac{1}{2}} k s - a k^2)
        }{2}
        \sqrt{\frac{1}{4} - \frac{a^2 m^2}{\omega(
            \Delta s - 2 a^{\frac{1}{2}} k s - ak^2
        )}}
        \right)
        + \mathrm{c.c.}
    \right\}
    \frac{{\d \omega \d k}}{\cancel{\sqrt{\omega}}}
    =
    2 a^{-\frac{3}{4}} \bigint \left\{
        \int
        \hat\psi_1(\sqrt\omega)
        \left\{
            \exp
            \left(ia^{-2} \left(\frac{\omega \Delta s}{4}
                                + O\left(a^{\frac{1}{2}}\right)\right)
            \right)
            + \mathrm{c.c.}
        \right\}
        \d \omega
    \right\}
    \hat \psi_2(k) \d k
    =
    2 a^{-\frac{3}{4}} \bigint
    \underbrace{
    \left\{
    (\hat\psi_1 \circ \sqrt{\cdot })^\vee
    \left(\frac{\Delta s}{4a^2}\right)
     + (\hat\psi_1 \circ \sqrt{\cdot })^\vee
    \left(-\frac{\Delta s}{4a^2}\right)
    + \mathrm{c.c.}
    \right\}}_{
    \sim O(a^k) ~\forall k \in \mathbb{N}
    }
    \psi_2(k) \d k
    \sim O(a^k) ~~ \forall k \hiderel \in \mathbb{N}
\label{eq:delta_m2_twisted_s<1_x=0}
\end{dmath}

wobei bei der Substition $\omega \to \sqrt{\omega}$ in der zweiten Zeile wichtig ist, dass $0 \notin supp (\hat\psi_1)$, also auch nach der Substitution noch $\hat\psi_1 \in C_c^\infty (\mathbb{R})$ ist.


\subsubsection*{Fall $s = -1$}

Da $\rwhat\Delta_m^{\circledast 2} = \rwhat\Delta_m^{* 2} \cos(\dots)$ ist, haben wir bis auf den $\cos$-Faktor die selben Analysis zu betreiben, wie für $\rwhat\Delta_m^{*2}$.

\begin{align}
    & \kern -2em\left\langle \rwhat{\Delta}_m^{\circledast 2}(-\cdot), \rwhat{\psi}_{a-1t}
    \right\rangle
    \nonumber \\ &=
    2 a^{-\frac{3}{4}} \bigint
    \underbrace{\frac{
        \hat\psi_1(\omega) \hat\psi_2(k'+k_0)
        \left(
        2\omega^2(k'+k_0)-a^{\frac{1}{2}}\omega^2(k'+k_0)^2-a^{\frac{3}{2}}2m^2
        \right)
        \Theta(k')
    }
    {
        \sqrt{k'} \sqrt{k'+k_0}
        \sqrt{2-a^{\frac{1}{2}}(k'+k_0)}
        \sqrt{-a^{\frac{1}{2}}\omega^2\left(k'-\tfrac{2\sqrt{\omega^2-4a^2m^2}}
                    {\sqrt a \omega}\right)}
    }}_{i)}
    \nonumber \\ & \kern 2em\cdot
    \cos
    \underbrace{\left(
        \frac{2\omega^2(k'+k_0)-a^{\frac{1}{2}}\omega^2(k'+k_0)^2}
             {2 a^{\frac{3}{2}}}
        \sqrt{
            \frac{1}{4}
            + \frac{a^{\frac{3}{2}} m^2}
                   {2\omega^2(k'+k_0)-a^{\frac{1}{2}}\omega^2(k'+k_0)^2}
        }
    \right)}_{ii)}
    \nonumber \\ & \kern 2em\cdot
    e^{-i\omega\left(\frac{t'+x'}{a}+\frac{(k'+k_0)x'}{\sqrt a}\right)}
    \d \omega \d k'
\label{eq:psi_a-1t_delta_m2_twisted}
\end{align}
% \begin{dmath}
%     \left\langle \rwhat{\Delta}_m^{\circledast 2}, \rwhat{\psi}_{a-1t}
%     \right\rangle
%     =
%     2 a^{-\frac{3}{4}} \int
%     \underbrace{\frac{
%         \hat\psi_1(\omega) \hat\psi_2(k'+k_0)
%         \left(
%         2\omega^2(k'+k_0)-a^{\frac{1}{2}}\omega^2(k'+k_0)^2-a^{\frac{3}{2}}3m^2
%         \right)
%         \Theta(k')
%     }
%     {
%         \sqrt{k'} \sqrt{k'+k_0}
%         \sqrt{2-a^{\frac{1}{2}}(k'+k_0)}
%         \sqrt{-a^{\frac{1}{2}}\omega^2\left(k'-\tfrac{2\sqrt{\omega^2-4a^2m^2}}
%                     {\sqrt a \omega}\right)}
%     }}_{i)}
%     \cdot
%     \cos
%     \underbrace{\left(
%         \frac{2\omega^2(k'+k_0)-a^{\frac{1}{2}}\omega^2(k'+k_0)^2}
%              {2 a^{\frac{3}{2}}}
%         \sqrt{
%             \frac{1}{4}
%             + \frac{a^{\frac{3}{2}} m^2}
%                    {2\omega^2(k'+k_0)-a^{\frac{1}{2}}\omega^2(k'+k_0)^2}
%         }
%     \right)}_{ii)}
%     \cdot
%     e^{-i\omega\left(\frac{t'+x'}{a}+\frac{(k'+k_0)x'}{\sqrt a}\right)}
%     \d \omega \d k'
% \label{eq:psi_a-1t_delta_m2_twisted}
% \end{dmath}

Genau wie in \cref{eq:lange_1/sqrt_abschaetzerei} können wir für $i)$ wieder abschätzen\footnote{Da cos beschränkt ist, spielt er bei den Abschätzungen keine Rolle}

\begin{dmath*}
    % \frac{
    %     \hat\psi_1(\omega) \hat\psi_2(k'+k_0)
    %     \left(
    %     2\omega^2(k'+k_0)-a^{\frac{1}{2}}\omega^2(k'+k_0)^2-a^{\frac{3}{2}}3m^2
    %     \right)
    %     \Theta(k')
    % }
    % {
    %     \sqrt{k'} \sqrt{k'+k_0}
    %     \sqrt{2-a^{\frac{1}{2}}(k'+k_0)}
    %     \sqrt{-a^{\frac{1}{2}}\omega^2\left(k'-\tfrac{2\sqrt{\omega^2-4a^2m^2}}
    %                 {\sqrt a \omega}\right)}
    % }
    i)
    \leq
    \frac{\textrm{const}}{\sqrt{k'}} \Theta(k')
\end{dmath*}

Damit dürfen wir wieder Lebesgue anwenden, um den Grenzwert $a \to 0$ des Integrals zu berechnen.
Des Weiteren ist analog zu \cref{eq:langer_sqrt_bruch_punktweise_konvergenz}

\begin{dmath}
    i)
    \stackrel{\textrm{\scriptsize punktweise f.ü.}}{\longrightarrow}
    \omega\,\hat\psi_1(\omega) \,\hat\psi_2(k') \Theta(k').
\label{eq:delta_m2_twisted_s=-1_material_a}
\end{dmath}

Widmen wir uns also dem Argument des Kosinus $ii)$:

\begin{dmath*}
    \frac{2\omega^2(k'+k_0)-a^{\frac{1}{2}}\omega^2(k'+k_0)^2}
         {2 a^{\frac{3}{2}}}
    \sqrt{
        \frac{1}{4}
        + \frac{a^{\frac{3}{2}} m^2}
               {2\omega^2(k'+k_0)-a^{\frac{1}{2}}\omega^2(k'+k_0)^2}
    }
    =
    \frac{\omega^2 (k'+k'0)(2-a^{\frac{1}{2}}(k_+k_0))}
         {2 a^{\frac{3}{2}}}
    \sqrt{
        \frac{1}{4}
        + \frac{a^{\frac{3}{2}}m^2}
               {\omega^2(k'+k_0)(a^{\frac{1}{2}}(k'+k_0)-2)}
    }\\
    \stackrel{\textrm{\scriptsize punktweise, außer } k'=0}{\longrightarrow}
    \frac{\omega^2 k' a^{-\frac{3}{2}}}{2}
\end{dmath*}

\begin{dmath}
\Longrightarrow ~~
    \cos\big(ii)\big)
    \stackrel{\textrm{\scriptsize punktweise, außer } k'=0}{\longrightarrow}
    \cos\left(\frac{\omega^2 k' a^{-\frac{3}{2}}}{2}\right)
\label{eq:delta_m2_twisted_s=-1_material_b}
\end{dmath}

Einsetzen von \cref{eq:delta_m2_twisted_s=-1_material_a,eq:delta_m2_twisted_s=-1_material_b} in \cref{eq:psi_a-1t_delta_m2_twisted} ergibt mit Lebesgue

\begin{dmath*}
    \lim_{a \to 0}
    \left\langle \rwhat{\Delta}_m^{\circledast 2}(-\cdot), \rwhat{\psi}_{a-1t}
    \right\rangle
    =
    2 a^{-\frac{3}{4}}
    \int \omega \,\hat\psi_1(\omega) \,\hat\psi_2(k')
         \cos\left(a^{-\frac{3}{2}}\frac{\omega^2 k'}{2}\right)
         e^{-i\omega k' \frac{x'}{\sqrt a}}
         e^{-i \omega \frac{t'+x'}{a}}
         \d \omega \d k'
    =
    a^{-\frac{3}{4}} \int
    \underbrace{\left\{
            \int \hat\psi_2(k')\Theta(k')
            \left(e^{i a^{-\frac{3}{2}}\frac{\omega^2 k'}{2}} + e^{-i\cdots}\right)
            e^{-i\omega k' \frac{x'}{\sqrt a}}
            \d k'
        \right\}}_{=: \hat f_a(\omega)}
    \cdot
    \omega \hat\psi_1(\omega)e^{-i\omega \frac{t'+x'}{a}} \d \omega
\end{dmath*}

Nun betrachten wir $\hat f_a(\omega)$ und erhalten analog zu
\cref{eq:faltung_mit_1/x_rechnung}

\begin{align*}
    \hat f_a(\omega)
    &=
    \int \hat\psi_2(k')\Theta(k')
    \left(
        e^{i a^{-\frac{3}{2}}\left(\frac{\omega^2 k'}{2} + O(a^1)\right)}
        + e^{-i \cdots}
    \right) \d k
    \\ &\stackrel{a \to 0}{\longrightarrow}
    \int \hat\psi_2(k')\Theta(k')
    \left(
        e^{i a^{-\frac{3}{2}}\left(\frac{\omega^2 k'}{2}\right)}
        + \mathrm{c.c}
    \right) \d k'
    \\ &=
    \bigg[
        \underbrace{
            \psi_2\left(-\frac{\omega^2}{2 a^{\frac{3}{2}}}\right)
        }_{O(a^k) \; \forall k \in \mathbb{N}}
        + \underbrace{i
            \underbrace{\left(\psi_2 * \mathcal{P}(1/x)\right)}_{O(x^{-1})}
            \left(-\frac{\omega^2}{2 a^{\frac{3}{2}}}\right)
        }_{
            O\left(\left(-\frac{\omega^2}{2 a^{\frac{3}{2}}}\right)^{-1}\right)
            = O\left(a^{\frac{3}{2}}\right)
           }
     + \mathrm{c.c}
     \bigg]
     \\ &\sim
    O\left(a^{\frac{3}{2}}\right).
\end{align*}
% \begin{dmath*}
%     \hat f_a(\omega)
%     =
%     \int \hat\psi_2(k')\Theta(k')
%     \left(
%         e^{i a^{-\frac{3}{2}}\left(\frac{\omega^2 k'}{2} + O(a^1)\right)}
%         + e^{-i \cdots}
%     \right) \d k
%     \\
%     \stackrel{a \to 0}{\longrightarrow}
%     \int \hat\psi_2(k')\Theta(k')
%     \left(
%         e^{i a^{-\frac{3}{2}}\left(\frac{\omega^2 k'}{2}\right)}
%         + e^{-i \cdots}
%     \right) \d k'
%     =
%     \bigg[
%         \underbrace{
%             \psi_2\left(-\frac{\omega^2}{2 a^{\frac{3}{2}}}\right)
%         }_{O(a^k) \; \forall k \in \mathbb{N}}
%         + \underbrace{i
%             \underbrace{\left(\psi_2 * \mathcal{P}(1/x)\right)}_{O(x^{-1})}
%             \left(-\frac{\omega^2}{2 a^{\frac{3}{2}}}\right)
%         }_{
%             O\left(\left(-\frac{\omega^2}{2 a^{\frac{3}{2}}}\right)^{-1}\right)
%             = O\left(a^{\frac{3}{2}}\right)
%            }
%      + (\textrm{anderer Term})
%      \bigg]
%      \sim O\left(a^{\frac{3}{2}}\right)
% \end{dmath*}

Also dürfen wir für $a \to 0$ schreiben $\hat f_a(\omega) = C a^{\frac{3}{2}} + o\left(a^{\frac{3}{2}}\right)$ und landen bei

\begin{align}
    \lim_{a \to 0}
    \left\langle \rwhat{\Delta}_m^{\circledast 2}, \rwhat{\psi}_{a-1t}^{(3)}
    \right\rangle
    &=
    a^{-\frac{3}{4}} \int C a^{\frac{3}{2}} \omega \hat\psi_1(\omega)
    e^{-i\omega \frac{t'+x'}{a}}
    \d \omega
    \nonumber \\ &\sim O\left(a^{\frac{3}{4}}\right) \condition{falls $t'=-x'$}
    \nonumber \\ &\sim O\left(a^k\right) ~~ \forall k \hiderel \in \mathbb{N}
                              \condition{sonst}
\label{eq:delta_m2_twisted_s=-1}
\end{align}
% \begin{dmath}
%     \lim_{a \to 0}
%     \left\langle \rwhat{\Delta}_m^{\circledast 2}, \rwhat{\psi}_{a-1t}
%     \right\rangle
%     =
%     a^{-\frac{3}{4}} \int C a^{\frac{3}{2}} \omega \hat\psi_1(\omega)
%     e^{-i\omega \frac{t'+x'}{a}}
%     \d \omega
%     \sim O\left(a^{\frac{3}{4}}\right) \condition{falls $t'=-x'$}
%     \sim O\left(a^k\right) ~~ \forall k \hiderel \in \mathbb{N}
%                               \condition{sonst}
% \label{eq:delta_m2_twisted_s=-1}
% \end{dmath}

\subsection{Zusammenfassung und Vergleich der Ergebnisse}
Fassen wir die Ergebnisse aus \cref{eq:delta_m2_twisted_s>1,eq:delta_m2_twisted_s<1_x_neq_0,eq:delta_m2_twisted_s<1_x=0,eq:delta_m2_twisted_s=-1} wieder in einer Übersichtstabelle zusammen:

\begin{table}[h]
\centering
\begin{tabular}{l|cccc}
        & $(t',x') = 0$     & $t'=x' \neq 0$    & $t'=-x' \neq 0$   & $t' \neq \pm x'$ \\ \hline
$s=1$   & $a^{\frac{3}{4}}$ & $a^{\frac{3}{4}}$ & $a^k$             & $a^k$            \\
$s=-1$  & $a^{\frac{3}{4}}$ & $a^k$             & $a^{\frac{3}{4}}$ & $a^k$            \\
$|s|<1$ & $a^k$             & $a^k$             & $a^k$             & $a^k$            \\
$|s|>1$ & $a^k$             & $a^k$             & $a^k$             & $a^k$
\end{tabular}
\caption{Konvergenzordnung von $\left<\Delta_m^{\star 2},\psi_{ast}^{(3)}\right>$ im Limit $a \to 0$ für alle interessanten Kombinationen von $s$ und $(t',x')$}
\label{tab:wavefrontset_delta_m2_twisted}
\end{table}

Auch diesmal stimmen die Ergebnisse mit denen von \textcite[Prop. 3.72]{Schulz2014}\footnote{So weit sie gegeben wurden} überein, welcher für alle Potenzen des getwisteten Produkts $\Delta_m^{\star k}$ erhält:

\begin{equation*}
\left\langle t,x; \omega, k \right\rangle \hiderel\in WF(\Delta_m^{\star k})
\Rightarrow
- \omega \geq |k|
\end{equation*}

Das getwistete Produkt ist also bei $0$ in weniger Richtungen singulär, als das ungetwistete. Insbesondere muss $G_F^2$ auf dieser nicht-kommutativen Raumzeit nicht renormiert werden, da das Produkt $\Theta \Delta_m^{\star_M 2}$ wohldefiniert ist:

\begin{corollary}[$\Theta \Delta_m^{\star_M 2}$ ist wohldefiniert]
Nach \cref{tab:wavefrontset_delta_m2_twisted} ist $\Delta_m^{\star_M 2}$ nur in lichtartige Richtungen singulär, während nach \cref{tab:wavefront_set_heaviside} $\Theta\otimes 1$ nur in $\partial_t$-Richtung singulär ist. Also ist deren Produkt nach Hörmanders Kriterium \cref{thm:hoermanders_criterion} wohldefiniert.
\end{corollary}

% section dots_und_nun_zur_wellenfrontmenge_von_ (end)


% section die_wellenfrontmenge_von_ (end)
