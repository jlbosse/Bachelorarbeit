%%%%%%%%%%%%%%%%%%%%%%%%%%%%%%%%%%%%%%%%%%%%%%%%%%%%%%%
% % Lines starting with % are comments, which are ignored.
% % This is a handy way of indicating the date and version of
% % your document, to wit:
% %
% % LaTeX sample file
% % Modified March, 2002
% %
%%%%%%%%%%%%%%%%%%%%%%%%%%%%%%%%%%%%%%%%%%%%%%%%%%%%%%%

\documentclass{scrartcl}
\usepackage{thesisstyle}


%%%%%%%%%%%%%%%%%%%%%%%%%%%%%%%%%%%%%%%%%%%%%%%%%%%%%%%%%%%%%%%%%%%%%%%%%%%%%%%%
% % Import plots
%%%%%%%%%%%%%%%%%%%%%%%%%%%%%%%%%%%%%%%%%%%%%%%%%%%%%%%%%%%%%%%%%%%%%%%%%%%%%%%%


%%%%%%%%%%%%%%%%%%%%%%%%%%%%%%%%%%%%%%%%%%%%%%%%%%%%%%%
% % Title and author(s)
%%%%%%%%%%%%%%%%%%%%%%%%%%%%%%%%%%%%%%%%%%%%%%%%%%%%%%%
\title{Berechnen der Wellenfrontmenge der massiven Zweipunktfunktion mittels Shearlets}
\author{Jan Lukas Bosse\thanks{
                  Georg-August Universitt Göttingen}
        }
%%%%%%%%%%%%%%%%%%%%%%%%%%%%%%%%%%%%%%%%%%%%%%%%%%%%%%%


\begin{document}
\newpage
\maketitle
%%%%%%%%%%%%%%%%%%%%%%%%%%%%%%%%%%%%%%%%%%%%%%%%%%%%%%%%%%%%%
% abstract, keywords and Subject classification are optional.
%%%%%%%%%%%%%%%%%%%%%%%%%%%%%%%%%%%%%%%%%%%%%%%%%%%%%%%%%%%%%%
\begin{abstract}
    Im folgenden werden wir die Wellenfrontmenge der massiven Zweipunktfunktionen
    mittels der Methoden von \textcite{Kutyniok2008} ausrechnen.
\end{abstract}


%%%%%%%%%%%%%%%%%%%%%%%%%%%%%%%%%%%%%%%%%%%%%%%%%%%%%%%%%%%%%%%%%%%%%%%%%%%%%%%%
% % Here is the start of the Text
%%%%%%%%%%%%%%%%%%%%%%%%%%%%%%%%%%%%%%%%%%%%%%%%%%%%%%%%%%%%%%%%%%%%%%%%%%%%%%%%


%%%%%%%%%%%%%%%%%%%%%%%%%%%%%%%%%%%%%%%%%%%%%%%%%%%%%%%%%%%%%%%%%%%%%%%%%%%%%%%%
% % Section 1
%%%%%%%%%%%%%%%%%%%%%%%%%%%%%%%%%%%%%%%%%%%%%%%%%%%%%%%%%%%%%%%%%%%%%%%%%%%%%%%%
\section{Allgemeines Gelaber über Shearlets} % (fold)
\label{sec:allgemeines_gelaber_ueber_shearlets}

\begin{theorem}[$\mathcal{S}_f(a,s,t)$ misst $WF(f)$]
\label{thm:main_theorem}
    Sei $\mathcal{D} = \mathcal{D}_1 \cup \mathcal{D}_2$ wobei
    $\mathcal{D}_1$ = \{
        $(t_0, s_0) \in \mathbb{R}^2 \times [-1,1] \big|$
        $|\mathcal{S}_f (a, s, t)| = O(a^k)$ gleichmäßig $\forall k \in \mathbb{N}
        , \forall t \in U$ Umgebung von $(t_0, s_0)$
    \}
    und $\mathcal{D}_2$ analog für $\psi^{(v)}$

    Dann gilt $WF(f)^c = \mathcal{D}$
\end{theorem}

\todo{diesen Satz richtig hin schreiben und ordentlich setzen}
\todo{Stil und Nummerierung für Sätze, Propositionen etc. anpassen}

\begin{corollary}[WF(f) misst $sing ~supp (\psi)$]
Sei $\mathcal{R} =$ \{
    $t_0 \in \mathcal{R}^2 \big|$ $|\mathcal{S}_f(a,s,t)| = O(a^k)$
    $\forall k \in \mathbb{N}, \forall t \in U$ Umgebung von $t_0$
    \}

    Dann gilt $sing ~supp (\psi)^c = \mathcal{R}$
\end{corollary}

\begin{remark}[Träger von $\psi$]

\begin{figure}[h]
\centering
%% Creator: Matplotlib, PGF backend
%%
%% To include the figure in your LaTeX document, write
%%   \input{<filename>.pgf}
%%
%% Make sure the required packages are loaded in your preamble
%%   \usepackage{pgf}
%%
%% Figures using additional raster images can only be included by \input if
%% they are in the same directory as the main LaTeX file. For loading figures
%% from other directories you can use the `import` package
%%   \usepackage{import}
%% and then include the figures with
%%   \import{<path to file>}{<filename>.pgf}
%%
%% Matplotlib used the following preamble
%%   \usepackage[utf8x]{inputenc}
%%   \usepackage[T1]{fontenc}
%%   \usepackage{amssymb}
%%
\begingroup%
\makeatletter%
\begin{pgfpicture}%
\pgfpathrectangle{\pgfpointorigin}{\pgfqpoint{4.000000in}{2.000000in}}%
\pgfusepath{use as bounding box, clip}%
\begin{pgfscope}%
\pgfsetbuttcap%
\pgfsetmiterjoin%
\definecolor{currentfill}{rgb}{1.000000,1.000000,1.000000}%
\pgfsetfillcolor{currentfill}%
\pgfsetlinewidth{0.000000pt}%
\definecolor{currentstroke}{rgb}{1.000000,1.000000,1.000000}%
\pgfsetstrokecolor{currentstroke}%
\pgfsetdash{}{0pt}%
\pgfpathmoveto{\pgfqpoint{0.000000in}{0.000000in}}%
\pgfpathlineto{\pgfqpoint{4.000000in}{0.000000in}}%
\pgfpathlineto{\pgfqpoint{4.000000in}{2.000000in}}%
\pgfpathlineto{\pgfqpoint{0.000000in}{2.000000in}}%
\pgfpathclose%
\pgfusepath{fill}%
\end{pgfscope}%
\begin{pgfscope}%
\pgfsetbuttcap%
\pgfsetmiterjoin%
\definecolor{currentfill}{rgb}{1.000000,1.000000,1.000000}%
\pgfsetfillcolor{currentfill}%
\pgfsetlinewidth{0.000000pt}%
\definecolor{currentstroke}{rgb}{0.000000,0.000000,0.000000}%
\pgfsetstrokecolor{currentstroke}%
\pgfsetstrokeopacity{0.000000}%
\pgfsetdash{}{0pt}%
\pgfpathmoveto{\pgfqpoint{0.198611in}{0.198611in}}%
\pgfpathlineto{\pgfqpoint{3.801389in}{0.198611in}}%
\pgfpathlineto{\pgfqpoint{3.801389in}{1.801389in}}%
\pgfpathlineto{\pgfqpoint{0.198611in}{1.801389in}}%
\pgfpathclose%
\pgfusepath{fill}%
\end{pgfscope}%
\begin{pgfscope}%
\pgfpathrectangle{\pgfqpoint{0.198611in}{0.198611in}}{\pgfqpoint{3.602778in}{1.602778in}}%
\pgfusepath{clip}%
\pgfsetbuttcap%
\pgfsetmiterjoin%
\definecolor{currentfill}{rgb}{0.500000,0.500000,0.500000}%
\pgfsetfillcolor{currentfill}%
\pgfsetfillopacity{0.500000}%
\pgfsetlinewidth{0.501875pt}%
\definecolor{currentstroke}{rgb}{0.000000,0.000000,0.000000}%
\pgfsetstrokecolor{currentstroke}%
\pgfsetdash{}{0pt}%
\pgfpathmoveto{\pgfqpoint{1.954965in}{0.398958in}}%
\pgfpathlineto{\pgfqpoint{2.045035in}{0.398958in}}%
\pgfpathlineto{\pgfqpoint{2.180139in}{0.519167in}}%
\pgfpathlineto{\pgfqpoint{1.819861in}{0.519167in}}%
\pgfpathclose%
\pgfusepath{stroke,fill}%
\end{pgfscope}%
\begin{pgfscope}%
\pgfpathrectangle{\pgfqpoint{0.198611in}{0.198611in}}{\pgfqpoint{3.602778in}{1.602778in}}%
\pgfusepath{clip}%
\pgfsetbuttcap%
\pgfsetmiterjoin%
\definecolor{currentfill}{rgb}{0.500000,0.500000,0.500000}%
\pgfsetfillcolor{currentfill}%
\pgfsetfillopacity{0.500000}%
\pgfsetlinewidth{0.501875pt}%
\definecolor{currentstroke}{rgb}{0.000000,0.000000,0.000000}%
\pgfsetstrokecolor{currentstroke}%
\pgfsetdash{}{0pt}%
\pgfpathmoveto{\pgfqpoint{1.883721in}{0.626019in}}%
\pgfpathlineto{\pgfqpoint{2.116279in}{0.626019in}}%
\pgfpathlineto{\pgfqpoint{2.465117in}{1.427407in}}%
\pgfpathlineto{\pgfqpoint{1.534883in}{1.427407in}}%
\pgfpathclose%
\pgfusepath{stroke,fill}%
\end{pgfscope}%
\begin{pgfscope}%
\pgfpathrectangle{\pgfqpoint{0.198611in}{0.198611in}}{\pgfqpoint{3.602778in}{1.602778in}}%
\pgfusepath{clip}%
\pgfsetbuttcap%
\pgfsetmiterjoin%
\definecolor{currentfill}{rgb}{0.500000,0.500000,0.500000}%
\pgfsetfillcolor{currentfill}%
\pgfsetfillopacity{0.500000}%
\pgfsetlinewidth{0.501875pt}%
\definecolor{currentstroke}{rgb}{0.000000,0.000000,0.000000}%
\pgfsetstrokecolor{currentstroke}%
\pgfsetdash{}{0pt}%
\pgfpathmoveto{\pgfqpoint{2.183952in}{0.626019in}}%
\pgfpathlineto{\pgfqpoint{2.416511in}{0.626019in}}%
\pgfpathlineto{\pgfqpoint{3.666043in}{1.427407in}}%
\pgfpathlineto{\pgfqpoint{2.735809in}{1.427407in}}%
\pgfpathclose%
\pgfusepath{stroke,fill}%
\end{pgfscope}%
\begin{pgfscope}%
\pgfpathrectangle{\pgfqpoint{0.198611in}{0.198611in}}{\pgfqpoint{3.602778in}{1.602778in}}%
\pgfusepath{clip}%
\pgfsetbuttcap%
\pgfsetroundjoin%
\pgfsetlinewidth{0.501875pt}%
\definecolor{currentstroke}{rgb}{0.501961,0.501961,0.501961}%
\pgfsetstrokecolor{currentstroke}%
\pgfsetdash{{1.850000pt}{0.800000pt}}{0.000000pt}%
\pgfpathmoveto{\pgfqpoint{1.804251in}{0.184722in}}%
\pgfpathlineto{\pgfqpoint{3.636860in}{1.815278in}}%
\pgfpathlineto{\pgfqpoint{3.636860in}{1.815278in}}%
\pgfusepath{stroke}%
\end{pgfscope}%
\begin{pgfscope}%
\pgfpathrectangle{\pgfqpoint{0.198611in}{0.198611in}}{\pgfqpoint{3.602778in}{1.602778in}}%
\pgfusepath{clip}%
\pgfsetbuttcap%
\pgfsetroundjoin%
\pgfsetlinewidth{0.501875pt}%
\definecolor{currentstroke}{rgb}{0.501961,0.501961,0.501961}%
\pgfsetstrokecolor{currentstroke}%
\pgfsetdash{{1.850000pt}{0.800000pt}}{0.000000pt}%
\pgfpathmoveto{\pgfqpoint{0.363140in}{1.815278in}}%
\pgfpathlineto{\pgfqpoint{2.195749in}{0.184722in}}%
\pgfpathlineto{\pgfqpoint{2.195749in}{0.184722in}}%
\pgfusepath{stroke}%
\end{pgfscope}%
\begin{pgfscope}%
\pgfsetrectcap%
\pgfsetmiterjoin%
\pgfsetlinewidth{0.501875pt}%
\definecolor{currentstroke}{rgb}{0.000000,0.000000,0.000000}%
\pgfsetstrokecolor{currentstroke}%
\pgfsetdash{}{0pt}%
\pgfpathmoveto{\pgfqpoint{2.000000in}{0.198611in}}%
\pgfpathlineto{\pgfqpoint{2.000000in}{1.801389in}}%
\pgfusepath{stroke}%
\end{pgfscope}%
\begin{pgfscope}%
\pgfsetrectcap%
\pgfsetmiterjoin%
\pgfsetlinewidth{0.501875pt}%
\definecolor{currentstroke}{rgb}{0.000000,0.000000,0.000000}%
\pgfsetstrokecolor{currentstroke}%
\pgfsetdash{}{0pt}%
\pgfpathmoveto{\pgfqpoint{0.198611in}{0.358889in}}%
\pgfpathlineto{\pgfqpoint{3.801389in}{0.358889in}}%
\pgfusepath{stroke}%
\end{pgfscope}%
\begin{pgfscope}%
\pgfsetroundcap%
\pgfsetroundjoin%
\pgfsetlinewidth{0.501875pt}%
\definecolor{currentstroke}{rgb}{0.000000,0.000000,0.000000}%
\pgfsetstrokecolor{currentstroke}%
\pgfsetdash{}{0pt}%
\pgfpathmoveto{\pgfqpoint{1.413964in}{0.497069in}}%
\pgfpathquadraticcurveto{\pgfqpoint{1.625531in}{0.488637in}}{\pgfqpoint{1.829340in}{0.480514in}}%
\pgfusepath{stroke}%
\end{pgfscope}%
\begin{pgfscope}%
\pgfsetroundcap%
\pgfsetroundjoin%
\pgfsetlinewidth{0.501875pt}%
\definecolor{currentstroke}{rgb}{0.000000,0.000000,0.000000}%
\pgfsetstrokecolor{currentstroke}%
\pgfsetdash{}{0pt}%
\pgfpathmoveto{\pgfqpoint{1.774935in}{0.510482in}}%
\pgfpathlineto{\pgfqpoint{1.829340in}{0.480514in}}%
\pgfpathlineto{\pgfqpoint{1.772722in}{0.454971in}}%
\pgfusepath{stroke}%
\end{pgfscope}%
\begin{pgfscope}%
\pgftext[x=0.648958in,y=0.479097in,left,base]{\rmfamily\fontsize{10.000000}{12.000000}\selectfont \(\displaystyle a = 1, s = 0\)}%
\end{pgfscope}%
\begin{pgfscope}%
\pgfsetroundcap%
\pgfsetroundjoin%
\pgfsetlinewidth{0.501875pt}%
\definecolor{currentstroke}{rgb}{0.000000,0.000000,0.000000}%
\pgfsetstrokecolor{currentstroke}%
\pgfsetdash{}{0pt}%
\pgfpathmoveto{\pgfqpoint{1.141127in}{1.096480in}}%
\pgfpathquadraticcurveto{\pgfqpoint{1.354038in}{1.088809in}}{\pgfqpoint{1.559190in}{1.081418in}}%
\pgfusepath{stroke}%
\end{pgfscope}%
\begin{pgfscope}%
\pgfsetroundcap%
\pgfsetroundjoin%
\pgfsetlinewidth{0.501875pt}%
\definecolor{currentstroke}{rgb}{0.000000,0.000000,0.000000}%
\pgfsetstrokecolor{currentstroke}%
\pgfsetdash{}{0pt}%
\pgfpathmoveto{\pgfqpoint{1.504671in}{1.111178in}}%
\pgfpathlineto{\pgfqpoint{1.559190in}{1.081418in}}%
\pgfpathlineto{\pgfqpoint{1.502670in}{1.055658in}}%
\pgfusepath{stroke}%
\end{pgfscope}%
\begin{pgfscope}%
\pgftext[x=0.198611in,y=1.080139in,left,base]{\rmfamily\fontsize{10.000000}{12.000000}\selectfont \(\displaystyle a = 0.15, s = 0\)}%
\end{pgfscope}%
\begin{pgfscope}%
\pgfsetroundcap%
\pgfsetroundjoin%
\pgfsetlinewidth{0.501875pt}%
\definecolor{currentstroke}{rgb}{0.000000,0.000000,0.000000}%
\pgfsetstrokecolor{currentstroke}%
\pgfsetdash{}{0pt}%
\pgfpathmoveto{\pgfqpoint{3.348730in}{0.656645in}}%
\pgfpathquadraticcurveto{\pgfqpoint{3.136663in}{0.781233in}}{\pgfqpoint{2.931289in}{0.901887in}}%
\pgfusepath{stroke}%
\end{pgfscope}%
\begin{pgfscope}%
\pgfsetroundcap%
\pgfsetroundjoin%
\pgfsetlinewidth{0.501875pt}%
\definecolor{currentstroke}{rgb}{0.000000,0.000000,0.000000}%
\pgfsetstrokecolor{currentstroke}%
\pgfsetdash{}{0pt}%
\pgfpathmoveto{\pgfqpoint{2.965119in}{0.849795in}}%
\pgfpathlineto{\pgfqpoint{2.931289in}{0.901887in}}%
\pgfpathlineto{\pgfqpoint{2.993261in}{0.897696in}}%
\pgfusepath{stroke}%
\end{pgfscope}%
\begin{pgfscope}%
\pgftext[x=3.080833in,y=0.519167in,left,base]{\rmfamily\fontsize{10.000000}{12.000000}\selectfont \(\displaystyle a = 0.15, s = 1\)}%
\end{pgfscope}%
\begin{pgfscope}%
\pgfsetroundcap%
\pgfsetroundjoin%
\pgfsetlinewidth{0.501875pt}%
\definecolor{currentstroke}{rgb}{0.000000,0.000000,0.000000}%
\pgfsetstrokecolor{currentstroke}%
\pgfsetdash{}{0pt}%
\pgfpathmoveto{\pgfqpoint{2.000000in}{1.807510in}}%
\pgfpathquadraticcurveto{\pgfqpoint{2.000000in}{1.808331in}}{\pgfqpoint{2.000000in}{1.801389in}}%
\pgfusepath{stroke}%
\end{pgfscope}%
\begin{pgfscope}%
\pgfsetroundcap%
\pgfsetroundjoin%
\pgfsetlinewidth{0.501875pt}%
\definecolor{currentstroke}{rgb}{0.000000,0.000000,0.000000}%
\pgfsetstrokecolor{currentstroke}%
\pgfsetdash{}{0pt}%
\pgfpathmoveto{\pgfqpoint{1.972222in}{1.751954in}}%
\pgfpathlineto{\pgfqpoint{2.000000in}{1.807510in}}%
\pgfpathlineto{\pgfqpoint{2.027778in}{1.751954in}}%
\pgfusepath{stroke}%
\end{pgfscope}%
\begin{pgfscope}%
\pgftext[x=2.000000in,y=1.870833in,,bottom]{\rmfamily\fontsize{10.000000}{12.000000}\selectfont \(\displaystyle \omega\)}%
\end{pgfscope}%
\begin{pgfscope}%
\pgfsetroundcap%
\pgfsetroundjoin%
\pgfsetlinewidth{0.501875pt}%
\definecolor{currentstroke}{rgb}{0.000000,0.000000,0.000000}%
\pgfsetstrokecolor{currentstroke}%
\pgfsetdash{}{0pt}%
\pgfpathmoveto{\pgfqpoint{3.807488in}{0.358889in}}%
\pgfpathquadraticcurveto{\pgfqpoint{3.808320in}{0.358889in}}{\pgfqpoint{3.801389in}{0.358889in}}%
\pgfusepath{stroke}%
\end{pgfscope}%
\begin{pgfscope}%
\pgfsetroundcap%
\pgfsetroundjoin%
\pgfsetlinewidth{0.501875pt}%
\definecolor{currentstroke}{rgb}{0.000000,0.000000,0.000000}%
\pgfsetstrokecolor{currentstroke}%
\pgfsetdash{}{0pt}%
\pgfpathmoveto{\pgfqpoint{3.751932in}{0.386667in}}%
\pgfpathlineto{\pgfqpoint{3.807488in}{0.358889in}}%
\pgfpathlineto{\pgfqpoint{3.751932in}{0.331111in}}%
\pgfusepath{stroke}%
\end{pgfscope}%
\begin{pgfscope}%
\pgftext[x=3.870833in,y=0.358889in,left,]{\rmfamily\fontsize{10.000000}{12.000000}\selectfont \(\displaystyle k\)}%
\end{pgfscope}%
\end{pgfpicture}%
\makeatother%
\endgroup%

\caption{Der Träger von $\hat \psi_{ast}$ für verschiedene $a, s$. Man sieht gut,
wie $supp (\hat \psi_{ast})$ für kleinere $a$ in immer kleineren Kegeln liegt.}
\end{figure}

\label{cor:psi_hat}
Im Fourierraum ist $\hat{\psi}_{ast}$ gegeben durch

\begin{equation}
    \hat \psi_{ast}{(\xi_1, \xi_2)} = a^{\frac{3}{4}}e^{-i\xi \cdot t}\hat\psi_1(a \xi_1) \hat\psi_{2}\left(a^{-\frac{1}{2}}\left(\frac{\xi_2}{\xi_1}-s\right)\right)
\end{equation}

und es gilt

\begin{equation}
\label{eq:supp_psi}
    supp(\hat \psi) \subset \left\{\xi \in  \hat{\mathbb{R}}^2 ~\Big| ~|\xi_1| \in \left[\frac{1}{2 a} , \frac{2}{a}\right], \left|\frac{\xi_2}{\xi_1} - s\right| \leq \sqrt{a} \right\}
\end{equation}

\end{remark}



% section allgemeines_gelaber_über_shearlets (end)


%%%%%%%%%%%%%%%%%%%%%%%%%%%%%%%%%%%%%%%%%%%%%%%%%%%%%%%%%%%%%%%%%%%%%%%%%%%%%%%%
% % Section 2
%%%%%%%%%%%%%%%%%%%%%%%%%%%%%%%%%%%%%%%%%%%%%%%%%%%%%%%%%%%%%%%%%%%%%%%%%%%%%%%%
\section{Allgemeine Ausdrücke für $\left<\psi_{ast}, f\right>$}
\label{sec:<psi_ast,f>}

Zunächst werden wir zwei verschiedene Ausdrücke für $\left<\psi_{ast}, f\right>$
im Fourierraum herleiten, welche sich im dann folgenden als nützlich erweisen werden.

Sei also $\psi$ ein Shearlet wie in Korollar \ref{cor:psi_hat}. Sei $f$ die zu
analysierende fouriertransformierbare Funktion (oder Distribution) in
$\mathcal{D} \prime (\mathbb{R}^2)$. Dann ist $\mathcal{S}_f (ast)$ gegeben durch

\begin{align*}
\left< \psi_{ast}, f \right> &= \left<\hat\psi_{ast}, \hat f\right> \\
 &= \int a^\frac{3}{4} e^{-i \xi \cdot t} \hat \psi_1(a \xi_1)
    \hat \psi_2 \left(a^{-\frac{1}{2}} \left(\frac{\xi_2}{\xi_1} - s\right)\right)
    \hat f (\xi) \d \xi
\end{align*}

\todo{etnscheiden, was mit dem fehlenden Faktor $\frac{1}{(2 \pi)^n}$ geschieht}
und nach "`entscheren"' und "`deskalieren"', also der Substitution

\begin{equation*}
\begin{aligned}[c]
a \xi_1 &= k_1\\
a^{-\frac{1}{2}} \left(\frac{\xi_2}{\xi_1} - s\right) &=\frac{k_2}{k_1}\\
\end{aligned}
\qquad\Longleftrightarrow\qquad
\begin{aligned}[c]
\xi_1 &= \frac{k_1}{a}\\
\xi_2 &= \frac{k_1 s}{a} + a^{-\frac{1}{2}} k_2\\
\end{aligned}
\end{equation*}

\begin{equation*}
\Rightarrow
\d \xi_1 \d \xi_2 = a^{-\frac{3}{2}} \d k_1 \d k_2
\end{equation*}

ergibt sich folgendes für $\left<\psi_{ast}, f\right>$:

\begin{equation}
\label{eq:psi_ast_f_1}
    =  \iint a^{-\frac{3}{4}}~\hat \psi_1(k_1) ~\hat \psi_2 \left(\tfrac{k_2}{k_1}\right)
    ~\hat f \left(\tfrac{k_1}{a}, \tfrac{k_1 s}{a} + \tfrac{k_2}{\sqrt{a}}\right)
    ~e^{-i\frac{k_1}{a}(t_1+t_2 s) - i \frac{k_2 t_2}{\sqrt(a)}}
    \d k_1 \d k_2
\end{equation}

\todo{herausfinden, wie die Gleichungen auch Kapitelnummern erhalten}

Alternativ kann auch folgende Substitution

\begin{equation*}
\begin{aligned}[c]
a \xi_1 &= k_1\\
a^{-\frac{1}{2}} \left(\frac{\xi_2}{\xi_1} - s\right) &= k_2\\
\end{aligned}
\qquad\Longleftrightarrow\qquad
\begin{aligned}[c]
\xi_1 &= \frac{k_1}{a}\\
\xi_2 &= \left( a^\frac{1}{2} k_2 +s \right) \frac{k_1}{a}\\
\end{aligned}
\end{equation*}

\begin{equation*}
\Rightarrow
\d \xi_1 \d \xi_2 = a^{-\frac{3}{2}} k_1 \d k_1 \d k_2
\end{equation*}

gewählt werden, wodurch alle Parameter aus den Argumenten von $\psi_1, \psi_2$
verschwinden und sich

\begin{align}
\label{eq:psi_ast_f_2}
    =  \iint a^{-\frac{3}{4}}~ k_1~ \hat \psi_1(k_1)~ \hat \psi_2 (k_2)~
    \hat f \left(\tfrac{k_1}{a}, k_1 \left(a^{-\frac{1}{2}}k_2 + s a^{-1}\right)\right)
    ~e^{-i k_1 \left(\frac{t_1+s t_2}{a} + \frac{k_2 t_2}{\sqrt{a}}\right)}
    \d k_1 \d k_2
\end{align}

ergibt. Dabei ist zu beachten, dass diese Substitution zulässig ist, obwohl sie
die Oriertierung \emph{nicht} erhält und \emph{nicht} keine Bijektion ist. Aber
der kritische Bereich, nämlich $\xi_1 = 0$, liegt nicht im Träger von $\psi$.

\todo{Grafik basteln, die $supp ~\psi$ vor und nach der Substitution zeigt.}

\subsection{Ausdrücke für $\left< \psi_{ast}, G_F\right>$} % (fold)
\label{sec:psiast_gf}

Ab jetzt werden wir der Namenskonvention der Physiker in der SRT folgen und unsere
Ortsraumvariablen mit $x = (t, x)$ und unsere Impulsraumvariablen mit $\xi = (\omega, k)$
bezeichnen sowie konsequenterweise das Minkowskiskalarprodukt $x \cdot \xi = \omega t - k x$
verwenden. Des weiteren wird der Verschiebungsparameter mit  $t = (t', x')$ bezeichnet.

Die massive skalare Zweipunktfunktion bzw. der Feynmanpropagator im Impulsraum ist dann
gegeben durch (\textcite{Schwartz2014}, (6.34))

\begin{equation}
\label{eq:gf}
    \hat G_F(\omega, k) = \frac{1}{m^2 - \omega^2 + k^2 - i 0^+}
\end{equation}

Setzen wir dies in unsere Ausdrücke für $\left< \psi_{ast}, f\right>$ aus \eqref{eq:psi_ast_f_1}
bzw. \eqref{eq:psi_ast_f_2} ergibt sich, unter Verwendung des Minkowskiskalaprodukts,

\begin{align}
\left< \hat\psi_{ast}, \hat G_F \right> &=
    \int \hat \psi_{ast}(\omega, t) ~\hat G_F(\omega, t) ~\d \omega \d k
    \nonumber \\
    &=
    a^{\frac{3}{4}} \iint \frac{
        \hat \psi_1 (a \omega)
        ~\hat \psi_2 \left(a^{-\frac{1}{2}}\frac{k}{\omega} - s\right)
        ~ e^{-i\omega t' + i k x'}
    }
    {
        m^2 - \omega^2 + k^2 - i 0^+
    }
    \d \omega \d k
    \nonumber \\
    &=
    a^{-\frac{3}{4}} \iint \frac{
        \hat\psi_1(\omega)
        ~\hat \psi_2\left(\tfrac{k}{w}\right)
        ~e^{-i \omega \frac{t' - sx'}{a} + ik \frac{x'}{\sqrt{a}}}
    }
    {
        m^2 - \left(\frac{\omega}{a}\right)^2
        + \left(\frac{\omega s}{a} + \frac{k}{\sqrt{a}}\right)^2 - i0^+
    }
    \d \omega \d k \nonumber \\
    &=
    a^{-\frac{3}{4}}
    \kern -2em \iint
    \limits_{
    \substack{
        \omega \in [-2, -\frac{1}{2}]\cup[\frac{1}{2},2] \\
        \left|\frac{k}{2}-s\right| \leq \sqrt{ax}
        }
    }
    \kern -1.5em
    \frac{
        \hat\psi_1(\omega)
        ~\hat \psi_2\left(\tfrac{k}{w}\right)
        ~e^{-i \omega \frac{t' - sx'}{a} + ik \frac{x'}{\sqrt{a}}}
    }
    {
        m^2 + a^{-2} \omega^2 (s^2 - 1) + a^{-\frac{3}{2}} 2 s \omega k + a^{-1} k  - i0^+
    }
    \d \omega \d k
    \label{eq:psi_ast_gf_1}
\end{align}

\todo{Integral hübsch machen. Größeres Integralzeichen?}

und mit der anderen Substitution analog

\begin{align}
    \left< \hat\psi_{ast}, \hat G_F \right>
    &=
    a^{-\frac{3}{4}}
    \kern -1em
    \iint \limits_{\substack{
        |\omega|~ \in~ [\frac{1}{2},2] \\
        k ~\in~ [-1,1]
        }
    }
    \kern -1em
    \frac{
        \omega ~\hat \psi_1(\omega) ~\hat \psi_2(k)
        e^{-i \omega \left(\frac{t' - sx'}{a} + \frac{kx'}{\sqrt{a}}\right)}
    }
    {
        m^2 - \omega^2(a^{-2}(1-s^2)-a^{-1}k^2 - 2 k s a^{-\frac{3}{2}})
    }
    \d \omega \d k
    \label{eq:psi_ast_gf_2}
\end{align}

wobei sich die Integrationsbereiche aus den Forderungen an den Träger von $\psi$
(vgl. \eqref{eq:supp_psi}) ergeben.
% section ausdrücke_für_ (end)


%%%%%%%%%%%%%%%%%%%%%%%%%%%%%%%%%%%%%%%%%%%%%%%%%%%%%%%%%%%%%%%%%%%%%%%%%%%%%%%%
% % Section 3
%%%%%%%%%%%%%%%%%%%%%%%%%%%%%%%%%%%%%%%%%%%%%%%%%%%%%%%%%%%%%%%%%%%%%%%%%%%%%%%%
\section{Berechnen von $WF(G_F)$} % (fold)
\label{sec:berechnen_von_}

Nach Satz \eqref{thm:main_theorem} genügt es zu bestimmen, an welchen Punkten
$(t', x')$ und in welche Richtungen $s$ $\mathcal{S}_f(a,s,(t',x'))$ nicht schnell-fallend
in $a^{-1}$ ist, um die Wellenfrontmenge zu bestimmen. Da wir keine explizite
erzeugende Funktion $\psi$ angegeben haben, werden wir uns dabei Argumente bedienen,
die alleine auf den allgemeinen Eigenschaften von $\psi_{ast}$ beruhen, aber nicht
einer expliziten Form.
\todo{In Textform beschreiben, was die grobe Strategie ist, also wie der Integrand
vernünfitg vereinfacht wird und welche Eigenschaften von¸$\psi$ wie eingehen.}
\todo{Hier schon die Ergebnisse als Satz angeben, und dann Beweis hinschreiben?}
\todo{Bemerkung einfügen, warum dass auch ziemlich unmöglich ist}

Das allgemeine Vorgehen wird dabei folgendes sein: Die Ausdrücke in \eqref{eq:psi_ast_gf_1}
und \eqref{eq:psi_ast_gf_2} genau anstarren, um zu sehen für welche Werte von
$(t',x')$ und $s$ potentiell interessante Dinge geschehen, also z.B. Terme im Nenner
weg fallen, oder die Phase konstant wird. Dann werden diese Werte von $(t',x')$ und
$s$ eingesetzt und alles so weit vereinfacht und genähert -- im Rahmen des Erlaubten, ohne
das Verhalten für $a \rightarrow 0$ zu ändern --, bis die $a$-Abhängigkeit abgelesen
werden kann. Entscheidende Zutaten sind dabei der beschränkte Träger von $\hat \psi$
und der schnelle Abfall von $\psi$.


\subsubsection*{Fall $s=1, t' = 0 = x'$}
Nach \eqref{eq:psi_ast_gf_2} erhalten wir mit $s=1, t' = 0 = x'$

\begin{align*}
    \left< \hat\psi_{a10}, \hat G_F \right>
    &=
    \int a^{-\frac{3}{4}} \frac{
        \omega ~\hat \psi_1(\omega) ~\hat \psi_2(k)
    }
    {
        m^2+\omega^2 (a^{-1}k^2 + a^{-\frac{3}{2}}2 k )
    }
    \d \omega \d k \\
    &=
    \int a^{\frac{3}{4}} \frac{
        \omega ~\hat \psi_1(\omega) ~\hat \psi_2(k)
    }
    {
        a^\frac{3}{2} m^2+\omega^2 (a^\frac{1}{2}k^2 + 2 k )
    }
    \d \omega \d k
\end{align*}

Da aber $|\omega| \in [\frac{1}{2},2]$ und $k \in [-1,1]$ ist, ist für hinreichend
kleine $a$ (und für genau die interessieren wir uns ja)

\begin{equation*}
    \left|
        \frac{\omega ~\hat \psi_1(\omega) ~\hat \psi_2(k)}{k \omega^2}
    \right|
    \geq
    \left|
        \frac{\omega ~\hat \psi_1(\omega) ~\hat \psi_2(k)}
        {a^\frac{3}{2}m^2+a^\frac{1}{2}\omega^2 k+2k \omega^2}
    \right|
\end{equation*}

eine integrierbare (im Sinne des Cauchy-Hauptwertes) Majorante für den Integranden.

\todo{Warum ist Cauchy-Hauptwert hier erlaubt? Weiter ausführe, warum es diese Majorante tut?}

Wir dürfen uns also des Lebesgueschen Konvergenzsatzes bedienen und schreiben

\begin{equation}
    \lim_{a \rightarrow 0} \left< \hat\psi_{a10}, \hat G_F \right> =
    a^\frac{3}{4} \int \frac{
    \omega ~\hat \psi_1(\omega) ~\hat \psi_2(k)
    }
    {
    2k \omega^2
    }
    \d \omega \d k
    \sim O(a^\frac{3}{4})
\end{equation}

Für $s = -1$ erhalten wir genau das selbe Ergebniss, da ja der $\omega^2 (1-s^2)$-Term
im Nenner genauso wieder verschwindet.

\subsubsection*{Fall $s \neq \pm 1, t' = 0 = x'$}
In diesem Fall verschwindet der $\omega^2 (1-s^2)$-Term im Nenner nicht und
dementsprechend folgt

\begin{align*}
    \left< \hat\psi_{as0}, \hat G_F \right>
    &=
    \int a^{-\frac{3}{4}} \frac{
        \omega ~\hat \psi_1(\omega) ~\hat \psi_2(k)
    }
    {
        m^2-\omega^2 ((1-s^2) - a^{-1}k^2 - a^{-\frac{3}{2}}2 k )
    }
    \d \omega \d k \\
    &=
    \int a^{\frac{5}{4}} \frac{
        \omega ~\hat \psi_1(\omega) ~\hat \psi_2(k)
    }
    {
        a^2 m^2+\omega^2 (s^2-1) + a \omega^2 k^2 + a^\frac{1}{2}2 \omega^2 k s
    }
    \d \omega \d k
\end{align*}

Analog zum vorigen Teil ist, diesmal sogar ohne den Cauchy-Hauptwert bemühen zu
müssen,

\todo{Überall wo es sein muss $\lim_{a \rightarrow 0}$ dazu schreiben, oder sagen
dass der Limit überall impliziert ist}

\begin{equation*}
    \left|
        \frac{2 \omega ~\hat \psi_1(\omega) ~\hat \psi_2(k)}{\omega^2 (1-s^2)}
    \right|
    \geq
    \left|
        \frac{
        \omega ~\hat \psi_1(\omega) ~\hat \psi_2(k)
    }
    {
        a^2 m^2+\omega^2 (s^2-1) + a \omega^2 k^2 + a^\frac{1}{2}2 \omega^2 k s
    }
    \right|
\end{equation*}

dass eine integrierbare Majorante ist (in der Tat ja sogar in $C_c^\infty (\mathbb{R}^2)$)
Damit können wir folgende Abschätzung treffen:

\begin{equation*}
    \lim_{a \rightarrow 0} \left< \hat\psi_{as0}, \hat G_F \right> =
    a^\frac{5}{4} \int \frac{2 \omega ~\hat \psi_1(\omega) ~\hat \psi_2(k)}
    {\omega^2 (1-s^2)}
    \d \omega \d k
    \sim O(a^\frac{5}{4})
\end{equation*}


\subsubsection*{Fall $s \neq \pm 1, (t', s') \neq 0$}
In diesem Fall benutzen wir wieder die erste Substitution \eqref{eq:psi_ast_gf_1}
und klammern wie schon in den beiden vorigen Teilen die höchste negative
Potenz von $a$ im Nenner aus.

\begin{align}
\Rightarrow ~
    \left< \hat\psi_{ast}, \hat G_F \right>
    &=
    a^\frac{5}{4} \int \frac{
        \hat \psi_1 (\omega) ~\hat \psi_2 \left(\frac{k}{\omega}\right)
        ~ e^{-i \omega \left(\frac{t'-sx'}{a}\right) + i k \frac{x'}{\sqrt{a}}}
    }
    {
        a^2 m^2 - \omega^2 (1-s^2) + a^\frac{1}{2} s \omega k +a k^2
    }
    \d \omega \d k
\end{align}

und da immer noch $0 \notin supp(\psi_1)$ gilt ist ein weiteres mal eine integrierbare Majorante gegeben durch

\begin{equation}
    2\frac{\hat \psi_1 (\omega)~\hat\psi_2 \left(\frac{k}{\omega}\right)}
    {\omega^2(s^2-1)}
\end{equation}

In der Tat ist sogar

\begin{equation}
    \hat f(\omega, k) := \frac{\hat \psi_1 (\omega)~\hat\psi_2 \left(\frac{k}{\omega}\right)}
    {\omega^2(s^2-1)}
    \in C_c^\infty (\hat{\mathbb{R}}^2)
\end{equation}

da $\psi_1$ und $\psi_2$ getragen sind. Demnach ist die Fourierinverse von
$\hat f$, $f := \mathcal{F}^{-1}(\hat f) \in \mathcal{S}(\mathbb{R}^2)$, also schnell
fallend. Damit können wir schließlich abschätzen

\begin{align}
    \left| \left< \hat\psi_{ast}, \hat G_F \right> \right|
    &=
    a^\frac{5}{4} \left|  \int \hat f(\omega, k)
    ~e^{-i \omega \left(\frac{t'-sx'}{a}\right) + ik \frac{x'}{\sqrt{a}}}
    \d \omega \d k
    \right|
    \nonumber \\
    &=
    a^\frac{5}{4} \left| f \left(\frac{t'-sx}{a}, \frac{x'}{\sqrt{a}}\right) \right|
    \leq
    a^\frac{5}{4} C_k\left(
    1 + \left\lVert \substack{(t'-sx')/a \\ x'/\sqrt{a}} \right\rVert
    \right)^{-k}
    \nonumber \\
    &\leq
    a^\frac{5}{4} \frac{C_k}{2} a^{\frac{k}{2}} \left\lVert
    \substack{(t'-sx') \\ x'} \right\rVert^{-k}
    \sim O\left(a^\frac{5/2+k}{2}\right) ~~ \forall k \in \mathbb{N}
    \nonumber \\[1em]
    \Rightarrow
     \left| \left< \hat\psi_{ast}, \hat G_F \right> \right|
     &\sim
     O\left(a^k\right) ~~ \forall k \in \mathbb{N}
\end{align}


\subsubsection*{Fall $s = 1, (t', x') \neq 0$}
Auch in diesem Fall nutzen wir wieder den ersten Ausdruck für
$\left< \hat\psi_{a1t}, \hat G_F \right>$ aus \eqref{eq:psi_ast_gf_1} und sorgen
wir auch bisher jedes Mal dafür, dass wir im Nenner nur noch positive Potenzen von
$a$ und einen von $a$ unabhängigen Term haben. Dann sieht das ganze so aus:

\begin{equation*}
    \left< \hat\psi_{a1t}, \hat G_F \right>
    =
    a^\frac{3}{4} \int \frac{\hat \psi_1(\omega)
    ~\hat\psi_2 \left(\frac{k}{\omega}\right)
    ~ e^{-i \omega \left(\frac{t'-x'}{a}\right) + i k \frac{x'}{\sqrt{a}}}
    }
    {
        a^\frac{3}{2} m^2 + a^\frac{1}{2} k^2 + 2 \omega k
    }
    \d \omega \d k
\end{equation*}

wo wir im $\lim_{a \rightarrow 0}$ wieder doe $a$-Potenzen im Nenner weg fallen lassen
und auch dieses Mal dafür wieder den Cauchy-Hauptwert bemühen müssen, um den
Lebesgueschen Konvergenzsatz benutzen zu dürfen.
Weiter geht's:

\begin{align}
    &=
    a^\frac{3}{4} \int \frac{
    \hat \psi_1(\omega)
    ~\hat\psi_2 \left(\frac{k}{\omega}\right)
    ~ e^{-i \omega \left(\frac{t'-x'}{a}\right) + i k \frac{x'}{\sqrt{a}}}
    }
    {
    2\omega k
    } \d \omega \d k \nonumber \\
    &= a^\frac{3}{4} \int
    \underbrace{\left\{ \int \frac{\hat \psi_2\left(\frac{k}{\omega}\right)
        ~e^{ik\frac{x'}{\sqrt{a}}}
        }
        {
            2 k \omega
        }
        \d k
        \right\}}_{=: \hat f_a (\omega)}
    \hat \psi_1(\omega) e^{-i \omega \left(\frac{t'-x'}{a}\right)}
    \d \omega
\end{align}

und um hier weiter zu kommen, schauen wir uns $\hat f_a$ genauer an. Sei dazu
$\Psi_2(\omega) := \int_{-\infty}^\omega \psi_2(\omega ') \d \omega '
    -  \int_{\omega}^{+ \infty} \psi_2(\omega ') \d \omega '$ eine
Stammfunktion von $\psi_2$. Dies ist offenbar $C^\infty$ und beschränkt, da
 $\hat \psi_2 \in C^\infty_c$. Mithilfe von Fourieridentitäten und Substitution können wir nun weiter rechnen:

\begin{align*}
    \hat f_a (\omega) &=
    \int \frac{\hat \psi_2\left(\frac{k}{\omega}\right)}{2k\omega}
    e^{i k \frac{x'}{\sqrt{a}}}
    \d \omega \\
    &\stackrel{i)}{=}
    \int \frac{\hat \psi_2 (k)}{2k}e^{ik\frac{x' \omega}{\sqrt{a}}}
    \d \omega \\
    &\stackrel{ii)}{=} \frac{i}{2}  \Psi_2\left(\frac{x' \omega}{\sqrt{a}}\right)
\end{align*}

Hier wurde in $i)$ einfach $k \rightarrow \omega k$ substituiert und im Schritt $ii)$
wurde genutzt, dass $f(x) = \mathrm{sgn}(x) \leftrightarrow \hat f(k) \sim \frac{1}{k}$.
Nun stecken wir diese Erkenntnisse in unseren vorigen Ausdruck und erhalten

\begin{align}
 \left< \hat\psi_{a1t}, \hat G_F \right>
    &=
    \frac{i a^\frac{3}{4}}{2} \int \Psi_2\left(\frac{x' \omega}{\sqrt{a}}\right)
    ~\hat \psi_1(\omega)
    ~ e^{-i \omega \left(\frac{t'-x'}{a}\right)}
    \d \omega \d k
    \nonumber \\
    &\sim O\left(a^\frac{3}{4}\right) \kern 1em \textrm{ ; für } t'=x'
    \nonumber \\
    &\sim O\left(a^k\right) ~ \forall k \in \mathbb{N} \kern 1em \textrm{;   andernfalls}
\end{align}

Im letzten Schritt wurde wieder genutzt, dass
$\Psi_2\left(\frac{x' \omega}{\sqrt{a}}\right) ~\hat \psi_1(\omega) \in \mathcal{S}(\mathbb{R})$
ist, und demnach eine schnell fallende Fouriertransformierte hat.

% Es gilt $\psi_2(0) = 1$, da nach Konstruktion
% $\Vert \psi_2 \Vert_1 = 1$. Außerdem können wir $\psi_2$ so wählen, dass es
% auf einer ganzen offenen Umgebung von 0 konstant 1 ist. Durch geschicktes addieren
% einer 0 können wir nun schreiben

% \begin{equation*}
%     \hat f_a (\omega) =
%     \int \frac{\hat \psi_2 \left(\frac{k}{\omega}\right) - 1}{2k\omega}
%     e^{ik\frac{x'}{\sqrt{a}}} \d k
%     + \int \frac{1}{2k\omega}
%     e^{ik\frac{x'}{\sqrt{a}}} \d k
% \end{equation*}

% wobei das Symbol des ersten Terms glatt ist, da $\psi_2 (0) = 0$. Der zweite Term
% wird also für $a^{-\frac{1}{2}} \rightarrow \infty$ dominieren. Für diesen gilt:
% \todo{Diese Argument verfeinern, oder mindestens raus finden, warum es denn
% zulässig ist.}

% \begin{equation*}
%     \int \frac{e^{ik\frac{x'}{\sqrt{a}}}}{2k\omega} \d k
%     = \frac{2 \pi i}{2 \omega} \mathrm{sgn}\left(\frac{x'}{\sqrt{a}}\right)
% \end{equation*}

% als Hauptwertintegral. Bedenkend dass $\psi_1 \in \mathcal{S} (\mathbb{R})$ und
% $\hat \psi_1 = 0$ in einer Umgebung von $0$ sowie
% $\mathrm{sgn}\left(\frac{x'}{\sqrt{a}}\right) = \mathrm{sgn}\left(x'\right)$  für $a>0$
% können wir also schließlich abschätzen

% \begin{align}
%     \lim_{a \rightarrow 0}\left< \hat\psi_{a1t}, \hat G_F \right>
%     &= \lim_{a \rightarrow 0} a^\frac{3}{4} C \int \frac{\mathrm{sgn}(x')
%     ~\hat \psi_1(\omega)}{\omega}
%     e^{-i\omega \frac{t'-x}{a}} \d \omega
%     \nonumber \\
%     &\sim O\left(a^\frac{3}{4}\right) \kern 1em \textrm{ ; für } t'=x'
%     \nonumber \\
%     &\sim O\left(a^k\right) ~ \forall k \in \mathbb{N} \kern 1em \textrm{;   andernfalls}
% \end{align}

% wobei in $C$ alle irrelevanten Vorfaktoren gesammelt wurden.

Das analoge Ergebnis erhält man auch für $s=-1$ und $t' = -x'$
% section berechnen_von_ (end)



\printbibliography

\end{document}
