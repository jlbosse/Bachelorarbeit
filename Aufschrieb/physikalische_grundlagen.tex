%!TEX root = main.tex

\section{Zweipunktfunktionen, Sternprodukte und all die Physik} % (fold)
\label{sec:zweipunktfunktionen_sternprodukte_und_all_die_physik}

In diesem Kapitel wollen wir motivieren, warum die Multiplikation von Distributionen auch für Physiker eine relevante Fragestellung ist, was getwistete Produkte sind und was sie mit nicht-kommutativer Raumzeit zu tun haben.

\subsection{Die Zweipunktfunktionen und warum wir sie potenzieren wollen}
In der störungstheoretischen Quantenfeldtheorie entsprechen schon einfache Feynmandiagramme, wie z.B. das in \cref{fig:feynman-diagramm} (formal) Integralen über Produkte von Distributionen, in diesem Fall dem Feynman-Propagator.

% \begin{tikzpicture}
% \begin{feynman}
% \vertex (a) {\(\phi)};
% \vertex [right=of a] (b);
% \vertex [right=of b] (c);
% \vertex [right=of c] (d) {\(\phi)};
% \diagram*{
% (a) -- [boson, edge label'=\(p\)] (b) %[dot, label=right:\(x_1\)]
% -- [boson, half left, edge label'=\(k\)] (c) %[dot, label=left:\(x_2\)]
% -- [boson, half left, edge label'=\(k-p\)] (b),
% (c) -- [boson, edge label=\(p\)] (d),
% };
% \end{feynman}
% \end{tikzpicture}

\begin{figure}[h]
\begin{equation*}
\feynmandiagram [layered layout, horizontal=b to c] {
a [particle=\(\phi\)] -- [fermion, edge label=\(p\)] b [dot, label=right:\(x_1\)]
-- [fermion, half left, edge label=\(k\)] c [dot, label=left:\(x_2\)]
-- [fermion, half left, edge label=\(k-p\)] b,
c -- [fermion, edge label=\(p\)] d [particle=\(\phi\)],
};
 =  \int G_F(x_1,x_2) \,G_F(x_2,x_1) e^{ik(x_1-x_2)} e^{i(p-k)(x_2-x_1)}
 \d x_1 \d x_2 \d k
\end{equation*}
\caption{Ein einfaches Feynman-Diagramm aus der skalaren $\phi^3$-Theorie und das entsprechende Integral über Feynman-Propagatoren}
\label{fig:feynman-diagramm}
\end{figure}

Der Feynman-Propagator in zwei Dimensionen kann geschrieben werden als zeitgeordnete Zweipunktfunktion (vgl. \textcite{ReedSimon}), also

\begin{equation}
    G_F(t,x)
    =
    \Theta (t)\Delta_m(t,x) + \Theta(-t)\Delta_m(-t,-x)
    \label{eq:feynman_propgator_as_product}
\end{equation}

Wobei $\Theta$ die Heaviside-Funktion bezeichnet. Also sind Potenzen des Feynman-Propagators gegeben durch Potenzen der Zweipunktfunktion und der Heaviside-Funktion. Um zu wissen, wo diese Produkte definiert werden können, muss man deren Wellenfrontmengen kennen; dann liefert Hörmanders Kriterium \ref{thm:hoermanders_criterion} ein Kriterium für die Wohldefiniertheit.

In all dem kann die Zweipunktfunktion $\Delta_m$ geschrieben werden als Fouriertransformierte eines positiven Maßes auf der positiven Massenschale $H_m$ (vgl. \textcite{Schwartz2014}, 24.69):

\begin{equation}
    \Delta_m (t,x) = \int \delta (\omega^2-k^2-m^2)
                    \Theta(-\omega)e^{-i\omega t + i k x} \d \omega \d k
\label{eq:delta_m}
\end{equation}

\subsection{Sternprodukte und getwistete Faltungen}
Die \emph{nicht kommutative Quantenfeldtheorie} beschäftigt sich mit Quantenfeldtheorien in der Größenordnung der \emph{Planck-Skala}. Bei diesen Größenordnungen wird erwartet, dass die Geometrie der Raumzeit nicht mehr kommutativ ist, sich also Ort und Zeit nicht mehr mit beliebiger Präzision messen lassen. Diese Schranken in der Messgenauigkeit lassen sich als Unschärferelation verstehen, ganz analog zur klassischen Unschärferelation zwischen Ort und Impuls.

Bei der Deformationsquantisierung wird das kommutative punktweise Produkt von Funktionen auf dem Phasenraum ersetzt durch ein nicht-kommutatives \emph{Sternprodukt/Moyal-Produkt}, um das nicht-kommutieren von Ort und Impuls zu implementieren.
Mehr Details finden sich in \textcite[Kap. 6]{Waldmann2007}.

Analog zu dieser Konstruktion ersetzen \textcite{Doplicher1995} das kommutative Produkt von Funktionen auf der Raumzeit durch ein Sternprodukt $\star$ auf den Funktionen auf der Raumzeit und erhalten damit eine nicht-kommutative Geometrie. Gemäß dem Faltungssatz für die Fouriertransformierte gibt es auch eine \emph{getwistete Faltung} $\circledast$, s.d. der Faltungssatz erfüllt ist. Also
\begin{equation*}
    \rwhat{f \star g} \, (k) = \hat f \circledast \hat g \, (k)
\end{equation*}

In zwei Dimensionen ist die getwistete Faltung definiert wie folgt:

\begin{definition}[getwistete Faltung]
\label{def:twisted_convolution}
    Seien $f,g \in $ "`passender Funktionen-/Distributionenraum"'. Sei $\Omega \in \mathbb{R}^{2 \times 2}$ eine symplektische Matrix. Dann ist die getwistete Faltung $(f \circledast g) (x)$ definiert als

    \begin{equation}
        (f \circledast g) \,(x) \coloneqq
        \int f(y) g(x-y)e^{\frac{i}{2} \Omega(x,y)} \d y
    \end{equation}

    Die getwistete Faltung ist also einfach die gewöhnliche Faltung, die noch mit einem ortsabhängigen Phasenfaktor verziert wurde.

    Wir verwenden die kanonische symplektische Matrix, also
    $\Omega = \left(\begin{smallmatrix}
        0 & 1 \\ -1 & 0
    \end{smallmatrix}\right)$
\end{definition}

Durch formale Rechnung, Ausschreiben der $e$-Funktion als Potenzreihe und nutzen der Fourieridentitäten $x \cdot \leftrightarrow i \partial_k$ sieht man, dass das Sternprodukt die Form
\todo[color=green]{Drin lassen oder nicht? Ich verwende es ja nicht, aber ohne fühlt sich unvollständig an}

\begin{equation*}
    f \star g = fg + \frac{i}{2} \sum_{i,j} \Pi^{ij}(\partial_if)(\partial_jg) - \frac{1}{8}\sum_{i,j,k,m} \Pi^{ij} \Pi^{km} (\partial_i \partial_k f)(\partial_j \partial_m g) + \dots
\end{equation*}

hat. Dabei ist $\Pi$ der zu $\Omega$ korrespondierende Poisson-Bivektor.


% section zweipunktfunktionen_sternprodukte_und_all_die_physik (end)
