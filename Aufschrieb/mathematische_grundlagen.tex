%!TEX root = main.tex

\section{Fouriertransformation, mikrolokale Analysis und all die Mathematik} % (fold)
\label{sec:fouriertransformation_mikrolokale_analysis_und_all_die_mathematik}

Hier entsteht mal ein Kapitel, in dem die notwendigen mathematischen Begriffe eingeführt und motiviert werden.

\begin{definition}[high frequency set]
\label{def:high_frequency_set}
Sei $f \in \mathcal{E}'(\Omega), \Omega \subset \mathcal{R}^n$ ein kompakt getragene Distribution. Dann definieren wir die Richtungen höher Frequenzen als
\begin{dmath*}
\Sigma (v) = \left\{k \hiderel \in \hat{\mathbb{R}}^n \big| k \textrm{ hat \emph{keine} kegelförmige Umgebung U s.d. } \\ |\hat v (k')| \leq C_N(1+|k|)^{-N} \forall k \hiderel\in V, \forall N  \hiderel\in \mathbb{N} \right\}
\end{dmath*}
und darauf basierend definieren wir noch eine lokalisierte Variante:

Sei $f \in \mathcal{D}'(\Omega), \Omega \subset \mathbb{R}^n$ eine Distribution.
Sei $\mathcal{D}_x$ die Menge der kompakt getragenen glatten Funktionen, die an $x$ nicht verschwinden.
Dann ist die singuläre Faser von $f$ an $x$ definiert als
\begin{dmath*}
\Sigma_x (f) = \bigcap \limits_{\phi \in \mathcal{D}_x} \Sigma (\underbrace{\phi f}_{\in \mathcal{E}'(\mathbb{R}^n)}) 
\end{dmath*}
\end{definition}

Damit können wir uns dann die Wellenfrontmenge definieren:

\begin{definition}[Wellenfrontmenge]
\label{def:wavefrontset}
\end{definition}
Sei $f \in \mathcal{D}'(\Omega), \Omega \subset \mathbb{R}^n$ eine Distribution. Dann ist ihre Wellenfrontmenge definiert als

\begin{dmath*}
WF(f) \coloneqq \left\{
	(x,k) \in \Omega \times (\hat{\mathbb{R}}^n \setminus 0)
	\Big | k \in \Sigma_x(f)
	\right\}
\end{dmath*}
\end{definition}

Anschaulich sagt uns die Wellenfrontmenge wo und in welche Richtungen eine Distribution singulär ist. So ist z.B. die Wellenfrontmenge der $\delta$-Distribution $(0, \mathbb{R}^n \setminus 0)$ oder das der 2-dimensionalen Heaviside-Funktion $1(x)\cdot\Theta(y)$ ist $\{((x,0),(0,1)\cdot \mathbb{R}\setminus 0)\}$



\begin{figure}
\caption{so ne tolle Faltung}
\label{fig:faltung_strahlen}
\end{figure}

% section fouriertransformation_mikrolokale_analysis_und_all_das (end)
