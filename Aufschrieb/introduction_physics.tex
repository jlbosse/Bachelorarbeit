
%!TEX root = main.tex
\chapter{Einleitung für Physiker} % (fold)
\label{sec:einleitung_physics}
\todo[color =green]{noch WF($\Theta$) ausrechnen und dann zeigen wo bei Hörmander die Probelne liegen?}
Einer der Zugänge zur Renormierung in der Quantenfeldtheorie ist die Fortsetzung der auftretenden Produkte von Distributionen\footnote{Wir erinnern uns, dass allgemeine Produkte von Distributionen nicht unbedingt immer definiert sind. Was zum Beispiel soll \(\delta^2\) sein?} auf ganz $\mathbb{R}^{1+d}$. Um zu bestimmen, wo und mit welchen Freiheiten diese fortgesetzt werden können, müssen die Wellenfrontmengen der Faktoren bestimmt werden. Leider ist es notorisch schwierig Wellenfrontmengen für Distributionen, die komplizierter sind als die $\delta$-Distribution und Ableitungen, direkt zu bestimmen. Unter anderem in Modellen der \emph{nichtkommutativen Quantenfeldtheorie} \cite{kappaMinkowski,Doplicher1995,StringLocalized} treten Distributionen auf, deren Wellenfrontmengen mit den bisherigen Methoden nicht bestimmt werden konnten.

Ursprünglich in der Bildbearbeitung und -kompression wurde erkannt und zur Kompression genutzt, dass Wavelettransformationen in der Lage sind, die Singularitätsstruktur von Bildern zu erkennen. D.h. dass die Wavelettransformation mit feiner werdendem Skalenparameter an Singularitäten nicht schnell abfällt, überall sonst aber schon.
Wie \textcite{Kutyniok2008,Candes2005,Contourlets} in ihren respektiven Arbeiten gezeigt haben, lassen sich diese Erkenntnisse auf Distributionen ausweiten und mit anisotropen und gerichteten Wavelets Wellenfrontmengen ausrechnen.

In der vorliegenden Arbeit wollen wir am Beispiel der \emph{Shearlets} untersuchen, wie praktikabel diese Methoden sind, um Wellenfrontmengen komplizierterer Distributionen auszurechnen. Dazu ziehen wir als Beispiele die massive Zweipunktfunktion \(\Delta_m\), ihre (getwisteten) Quadrate \(\Delta_m^{(\star) 2}\) und die Heaviside-Funktion \(\Theta\) heran. Diese sind von besonderem Interesse, da der Feynmanpropagator und Potenzen davon als zentrale zu renormierende Distribution der QFT ein Produkt von Zweipunktfunktion und Heaviside-Funktion ist.

Daneben gibt es noch eine kurze Diskussion, ob und wie es möglich ist, die Ergebnisse auf mehr als nur zwei Dimensionen auszuweiten.
Des Weiteren wird skizziert, welche weiteren Größen der \emph{mikrolokalen Analysis}, wie z.B. der Skalengrad, mithilfe von Shearlets berechnet werden können. Der Skalengrad einer Distribution ist eng verwandt mit dem Abzählen der Potenzen (engl. "`power counting"') in der QFT.

Wir kommen zu dem Ergebnis, dass die Shearlettransformation in zwei Dimensionen zwar eine theoretische Möglichkeit ist, Wellefrontmengen zu berechnen, aber deutlich mehr Arbeit als weniger direkte Methoden. In höheren -- und damit physikalisch relevanteren -- Dimensionen sind noch keine Verallgemeinerung bekannt, aber die konkreten Rechnungen werden sicher nicht übersichtlicher als in zwei Dimensionen.


% section einleitung (end)
