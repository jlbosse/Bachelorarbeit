
%!TEX root = main.tex
\section{Einleitung für Physiker} % (fold)
\label{sec:einleitung_physics}

Einer der Zugänge zur Renormierung in der Quantenfeldtheorie ist die Erweiterung der auftretenden Produkte von Distributionen auf ganz $\mathbb{R}^{1+d}$. Um zu bestimmen, wo und wie diese erweitert werden können, muss die Wellenfrontmenge der Faktoren bestimmt werden. Leider ist es notorisch schwierig Wellenfrontmengen für Distributionen komplizierter als die $\delta$-Distribution und Ableitungen direkt zu bestimmen.

Ursprünglich in der Bildbearbeitung und Kompression wurde erkannt und zur Kompression genutzt, dass Wavelettransformationen in der Lage sind, die Singularitätsstruktur von Bildern zu erkennen.
Wie \textcite {Kutyniok2008} sowie \textcite
{Candes2005} gezeigt haben, lässt sich diese Erkenntnis auf Distributionen ausweiten und mit anisotropen und gerichteten Wavelets Wellenfrontmengen ausrichten.

In der vorliegenden Arbeit wollen wir am Beispiel von \emph{Shearlets} untersuchen, wie praktikabel für die direkte Berechnung diese Methoden für Distributionen abseits der $\delta$-Distribution sind um Wellenfrontmengen zu bestimmen. Daneben gibt es noch eine kurze Diskussion, ob und wie es möglich ist, die Ergebnisse auf mehr als nur zwei Dimensionen auszuweiten.
Des weiteren wird noch eine kurze Diskussion gegeben, welche weiteren Größen der mikrolokalen Analysis mithilfe von Shearlets berechnet werden können.

Wir kommen zu dem Ergebniss, dass die Shearlettransformation schon in zwei Dimensionen zwar eine theoretische Möglichkeit ist, Wellefrontmengen zu berechnen, aber deutlich mehr Arbeit als weniger direkte Methoden. In höheren -- und damit physikalisch relevanteren -- Dimensionen sind noch keine Verallgemeinerung bekannt, aber die konkreten Rechnungen werden sicher nicht übersichtlicher als in zwei Dimensionen.


% section einleitung (end)
