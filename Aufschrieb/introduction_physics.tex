
%!TEX root = main.tex
\chapter{Einleitung für Physiker} % (fold)
\label{sec:einleitung_physics}
\todo[color =green]{noch WF($\Theta$) ausrechnen und dann zeigen wo bei Hörmander die Probelne liegen?}
Einer der Zugänge zur Renormierung in der Quantenfeldtheorie ist die Fortsetzung der auftretenden Produkte von Distributionen auf ganz $\mathbb{R}^{1+d}$. Um zu bestimmen, wo und mit welchen Freiheiten diese fortgesetzt werden können, muss die Wellenfrontmenge der Faktoren bestimmt werden. Leider ist es notorisch schwierig Wellenfrontmengen für Distributionen, die komplizierter sind als die $\delta$-Distribution und Ableitungen, direkt zu bestimmen.

Ursprünglich in der Bildbearbeitung und -kompression wurde erkannt und zur Kompression genutzt, dass Wavelettransformationen in der Lage sind, die Singularitätsstruktur von Bildern zu erkennen.
Wie \textcite {Kutyniok2008} sowie \textcite
{Candes2005} gezeigt haben, lassen sich diese Erkenntnisse auf Distributionen ausweiten und mit anisotropen und gerichteten Wavelets Wellenfrontmengen ausrechnen.

In der vorliegenden Arbeit wollen wir am Beispiel von \emph{Shearlets} untersuchen, wie praktikabel diese Methoden sind, um Wellenfrontmengen komplizierterer Distributionen auszurechnen. Daneben gibt es noch eine kurze Diskussion, ob und wie es möglich ist, die Ergebnisse auf mehr als nur zwei Dimensionen auszuweiten.
Des Weiteren wird skizziert, welche weiteren Größen der \emph{mikrolokalen Analysis}, wie z.B. der Skalengrad, mithilfe von Shearlets berechnet werden können.

Wir kommen zu dem Ergebnis, dass die Shearlettransformation in zwei Dimensionen zwar eine theoretische Möglichkeit ist, Wellefrontmengen zu berechnen, aber deutlich mehr Arbeit als weniger direkte Methoden. In höheren -- und damit physikalisch relevanteren -- Dimensionen sind noch keine Verallgemeinerung bekannt, aber die konkreten Rechnungen werden sicher nicht übersichtlicher als in zwei Dimensionen.


% section einleitung (end)
