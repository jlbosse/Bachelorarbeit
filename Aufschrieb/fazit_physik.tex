%!TEX root = main.tex

\section{Fazit} % (fold)
\label{sec:fazit_fuer_physiker}

Die Berechnungen in \cref{sec:die_wellenfrontmenge_von_delta_m,sec:die_wellenfrontmenge_von_delta_m2_twisted,sec:die_wellenfrontmenge_von_delta_m_2_} und Abschätzungsungetüme wie \cref{eq:psi_ast_delta_m2} zeigen deutlich, dass \cref{thm:main_theorem} zwar eine theoretische Möglichkeit liefert Wellenfrontmengen auszurechnen, es aber kein sehr praktikabler Ansatz ist. So wurde auch $\psi_{ast}$ nie konkret angegeben, sondern nur darauf hingewiesen, dass es Funktionen gibt, die all das erfüllen, was wir brauchen (also schneller Abfall und gewisse Eigenschaften des Trägers der Fouriertransformierten). Eins würden diese Funktionen aber sicher \emph{nicht} erfüllen: Dass $\int \psi_{ast}(x)\,f(x) \d x$ für eine größere Klasse von Funktionen tatsächlich analytisch zu berechnen ist, und nicht nur gewisse Schranken für den Abfall in $a$ gegeben werden können.

Ein weiteres Problem ist, dass die Abschätzungen für $\left<f, \psi_{ast}\right>$ nur möglich sind, wenn $\hat f$ bekannt ist. Dass schließt aber schon Funktionen aus, deren Fouriertransformierte nicht bekannt ist, oder temperierte Distributionen die als oszillierende Integrale gegeben sind.

Ähnlich sieht es bei der Berechnung des Skalengrads mithilfe von Shearlets aus (vgl. \cref{sec:scaling_degree}): Es sieht so aus, als sei es theoretisch möglich. Aber mit gewissem Aufwand bei den Abschätzungen verbunden.

Umso erfreulicher ist, dass die Ergebnisse für die berechneten Wellenfrontmengen mit den bisher bekannten übereinstimmen. Im Falle des getwisteten Produkts der Zweipunktfunktion konnte das Ergebnis von \textcite{Schulz2014} ja sogar verschärft und gezeigt werden, dass das getwistete Produkt bei $0$ nicht ganz so singulär ist, wie das ungetwistete.

Höherdimensionale Verallgemeinerungen der Shearlets müssten mit noch mehr Scherparametern arbeiten -- im drei dimensionalen Fall 3, in 4D schon  6 -- welche dann in den Ausdrücken auftauchen. Dann müsste man schlau erkennen, für welche Kombinationen dieser Scherpararemeter an welchen Orten $t$ $\left< f, \psi_{ast} \right>$ nicht schnell abfällt und abschätzen, wie schnell genau es abfällt.
Wenn die Verzweiflung also sehr groß ist, man viel Zeit, Papier und höherdimensionale Shearlets zur Verfügung hat \emph{könnte} die Shearlettransformation eine theoretische Möglichkeit sein Wellenfrontmengen temperierter Distributionen auszurechnen. Aber eigentlich eher nicht.

Oder natürlich wir haben etwas ganz wichtiges übersehen, und es ist doch alles nicht so hoffnungslos.

% section fazit_für_physiker (end)
