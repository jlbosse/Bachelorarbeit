%!TEX root = main.tex
\section{Fazit für Mathematiker} % (fold)
\label{sec:fazit_für_mathematiker}
\todo{was noch? Sieht nach etwas wenig aus...}
Erfreulicherweise\footnote{und glücklicherweise, sonst hätten wir uns ja verrechnet} stimmen die Ergebnisse für berechneten Wellenfrontmengen mit denen in der Literatur überein. Über die Wellenfrontmenge hinaus erhalten wir mit dem Exponenten von $a$ auch noch weitere Informationen wie langsam die lokalisierte Fouriertransformierte in eine gewisse Richtung abfällt. Größtenteils offen ist, inwiefern diese zusätzlichen Informationen genutzt werden können. Allerdings finden sich in \cref{sec:hoermanders_crit_abschwaechen,sec:scaling_degree} schon Ansätze.

In den Komplikationen und langen Abschätzungen in \cref{sec:die_wellenfrontmenge_von_delta_m,sec:die_wellenfrontmenge_von_delta_m_2_,sec:die_wellenfrontmenge_von_delta_m2_twisted} zeigt sich, dass die Shearlettransformation nur bedingt geeignet ist, um Wellenfrontmengen von temperierten Distributionen zu berechnen. Insbesondere war es immer entscheidend, einen expliziten Ausdruck für die Fouriertransformierte Distribution zu haben. Also sind die Shearletmethoden auf alle temperierten Distributionen, deren Fouriertransformierte nicht explizit ausgerechnet werden kann, nicht anwendbar!

Dennoch sind im Ausblick mit der Ausweitung von \cref{thm:main_theorem} auf temperierte Distributionen und  der Konstruktion höherdimensionaler Shearlets interessante Fragen aufgekommen, bei deren Bearbeitung sicher temperierte Distributionen und auch etwas Geometrie besser verstanden werden können.

% section fazit_für_mathematiker (end)
