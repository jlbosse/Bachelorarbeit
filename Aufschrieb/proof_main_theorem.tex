%!TEX root = main.tex
%%%%%%%%%%%%%%%%%%%%%%%%%%%%%%%%%%%%%%%%%%%%%%%%%%%%%%%%%%%%%%%%%%%%%%%%%%%%%%%
% % Section 1
%%%%%%%%%%%%%%%%%%%%%%%%%%%%%%%%%%%%%%%%%%%%%%%%%%%%%%%%%%%%%%%%%%%%%%%%%%%%%%%
\section{\texorpdfstring{Beweis von \cref{thm:main_theorem}}{Beweis des Hauptsatzes}} % (fold)
\label{sec:beweis_von_thm:main_theorem}

In \cite{Kutyniok2008} wird nur der Beweis von \cref{cor:psi_ast_misst_sing_supp} im Detail ausgeführt. Der Beweis von \cref{thm:main_theorem} wird nur skizziert. Deshalb wollen wir diese Lücke hier füllen, wobei das Vorgehen sich natürlich an dem Beweis von \cref{cor:psi_ast_misst_sing_supp} orientiert.

Bevor wir mit dem Beweis beginnen können, benötigen wir noch ein paar technische Lemmata. Beweise für diese finden sich, wenn nicht gegeben, in \cite{Kutyniok2008}.

\begin{lemma}
\label{lemm:lemma54}
    Sei $g \in L^2(\mathbb{R}^2)$ mit $\Vert g\Vert_\infty < \infty$. Nehme an, dass $supp (g) \subset U$ für ein $U \subset \mathbb{R}^2$ und setze
    $(U^\delta)^c \coloneqq \left\{x \in \mathbb{R}^2 | d(x,U) > \delta\right\}$.
    Dann fällt

    \begin{equation*}
        \hat h(k) \coloneqq \int \limits_0^1 \int \limits_{(B^\delta)^c} \int \limits_{-2}^{2} \left<g,\psi_{ast}\right> \hat \psi_{ast}(k) \d \mu (a,s,t)
    \end{equation*}

    schnell ab für $\Vert k \Vert \to \infty$.
\end{lemma}

Der Beweist findet sich in \cite{Kutyniok2008} und verwendet die selben Tricks, wie wir bei unserem Beweis von \cref{lemm:ruecktrafo_fourier_faellt_schnell_ab}.

Das nächste Lemma ist eine Verfeinerung von Lemma 5.5 in \textcite{Kutyniok2008}, der Beweis wird deshalb erbracht.
\begin{lemma}
\label{lemm:ruecktrafo_fourier_faellt_schnell_ab}

Seien $B(s_0,\Delta) \subset [-2,2]$ und $U \subset \mathbb{R}^2$ beschränkt. Nehme an, dass $G(a,s,t)$ schnell abfällt für $a \to 0$ gleichmäßig über $(s,t) \in  B(s_0,\Delta) \times U$. Dann fällt

\begin{equation*}
    \hat h(k) = \int \limits_0^1 \int \limits_U \int \limits_{-2}^2
    G(a,s,t) \hat \psi_{ast} (k)
        \d s \d t \frac{\d a}{a^3}
\end{equation*}

schnell ab, für $\Vert k \Vert \to \infty$ und $\frac{k_2}{k_1} \in B(s_0, \Delta/2)$ .
\end{lemma}

\begin{proof}
Es sei
\begin{equation*}
    \Gamma_k = \left\{a\in [0,1], s \in [-2,2] ~\Big|~ \tfrac{1}{2} \leq a|k| \leq 2 , \left|s-\tfrac{k_2}{k_1} \right| \leq \sqrt a
                   \right\},
\end{equation*}

also die Menge der $a,s$ s.d. $\psi_{ast}(k) \neq 0$.

Dann können wir dank \cref{eq:supp_psi} abschätzen

\begin{equation*}
    | \hat \psi_{ast} (k)| \leq C' a^{\frac{3}{4}} \chi_{\Gamma_k},
\end{equation*}

und nach Vorraussetzung gilt auch für alle $(s,t) \in B(s_0,\Delta) \times U$

\begin{equation*}
    |G(a,s,t)| \leq C_{N} a^{N}
    \condition{$\forall N \in \mathbb{N}$}.
\end{equation*}

Außerdem sei $S = B(s_0,\Delta/2)$.

Um $\hat h(k)$ abzuschätzen, teilen wir das Integral in die Bereiche auf, in dem G(a,s,t) schnell abfällt und in dem es nicht schnell abfällt:

\begin{dmath*}
    \hat h(k) =
    \underbrace{
    \int \limits_0^1 \int \limits_U \int \limits_{S}
    G(a,s,t) \hat \psi_{ast} (k)
        \d s \d t \frac{\d a}{a^3}}_{i)}
     +
    \underbrace{
     \int \limits_0^1 \int \limits_U \int \limits_{[-2,2]\setminus S}
    G(a,s,t) \hat \psi_{ast} (k)
        \d s \d t \frac{\d a}{a^3}}_{ii)}
\end{dmath*}



\paragraph{zu $i)$}
\begin{align*}
    i) &\leq
     \int \limits_0^1 \int \limits_U \int \limits_{S}
    \left\lvert G(a,s,t)\right\rvert
    \left\lvert \hat \psi_{ast} (k) \right\rvert
        \d s \d t \frac{\d a}{a^3}
    \\ &\leq
    C_N C'
    \int \limits_0^1 \int \limits_U \int \limits_{S}
    a^{\frac{3}{4}} a^N \chi_{\Gamma_k} \d s \d t \frac{\d a}{a^3}
    \\ &\leq
    C_N \int \limits_{\frac{1}{2|k|}}^{\frac{2}{|k|}}
    a^{N-\frac{9}{4}} \d a
    ~\leq~
     C_N |k|^{-N+\frac{7}{4}}
\end{align*}

% \begin{dmath*}
%     i) \leq \int \limits_0^1 \int \limits_U \int \limits_{S}
%     \left\lvert G(a,s,t)\right\rvert
%     \left\lvert \hat \psi_{ast} (k) \right\rvert
%         \d s \d t \frac{\d a}{a^3}
%     \leq
%     C_N C'
%     \int \limits_0^1 \int \limits_U \int \limits_{S}
%     a^{\frac{3}{4}} a^N \chi_{\Gamma_k} \d s \d t \frac{\d a}{a^3}
%     \leq
%     C_N \int \limits_{\frac{1}{2|k|}}^{\frac{2}{|k|}}
%     a^{N-\frac{9}{4}} \d a
%     \le C_N |k|^{-N+\frac{7}{4}}
% \end{dmath*}


\begin{equation*}
    \Longrightarrow ~~
    i) \xrightarrow[a \to 0]{\text{\tiny schnell}} 0
\end{equation*}

\paragraph{zu $ii)$}

\begin{align*}
    ii) &\leq
     \int \limits_0^1 \int \limits_U \int \limits_{[-2,2]\setminus S}
    |G(a,s,t)| |\hat \psi_{ast} (k)|
        \d s \d t \frac{\d a}{a^3}
    \\ &\leq
    C' \int \limits_{0}^{1} \int \limits_{U} \int \limits_{[-2,2]\setminus S}
    |G(a,s,t)| \chi_{\Gamma_k} a^{\frac{3}{4}}
    \d s \d t \frac{\d a}{a^3}
\end{align*}
% \begin{dmath*}
%     ii) \leq
%      \int \limits_0^1 \int \limits_U \int \limits_{[-2,2]\setminus S}
%     |G(a,s,t)| |\hat \psi_{ast} (k)|
%         \d s \d t \frac{\d a}{a^3}
%     \leq
%     C' \int \limits_{0}^{1} \int \limits_{U} \int \limits_{[-2,2]\setminus S}
%     |G(a,s,t)| \chi_{\Gamma_k} a^{\frac{3}{4}}
%     \d s \d t \frac{\d a}{a^3}
% \end{dmath*}

Für alle hinreichend großen $k$ ist aber $\Gamma_k \subset S$, also $\Gamma_k \cap [-1,1]\setminus S = \varnothing$ und demnach das Integral 0. Also

\begin{equation*}
    ii) = 0 \condition{für alle k groß genug}
\end{equation*}
\end{proof}

Noch ein letztes Lemma, bevor wir mit dem Beweis beginnen können:

\begin{lemma}
    [Abschätzungen für $\left<\phi \psi_{a_0st},\psi_{a_1s't'}\right>$]
\label{lemm:lemma57}
Sei $\phi \in C_0^\infty(B(t,\delta))$. Dann gilt für alle $N>0$

\begin{enumerate}
    \item Falls $0 \leq \sqrt{a_0} \leq \sqrt{a_1}\leq \delta \leq 1$
    \begin{equation*}
        |\left<\phi \psi_{a_0st},\psi_{a_1s't'}\right>| \leq
        C_N \left(1+\frac{a_1}{a_0}\right)^{-N}
        \left(1+\frac{|s-s'|^2}{a_1}\right)^{-N}
        \left(1+\frac{\Vert t-t' \Vert^2}{a_1}\right)^{-N}
    \end{equation*}
    \item Falls $0 \leq \sqrt{a_0} \leq \delta \leq \sqrt{a_1} \leq 1$
    \begin{equation*}
        |\left<\phi \psi_{a_0st},\psi_{a_1s't'}\right>| \leq
        C_N \left(1+\frac{a_1}{a_0}\right)^{-N}
        \left(1+\frac{|s-s'|^2}{\delta^2}\right)^{-N}
        \left(1+\frac{\Vert t-t' \Vert^2}{a_1}\right)^{-N}
    \end{equation*}
    \item Falls $\sqrt{a_0} \leq \delta \leq a_1 \leq \sqrt{a_1} \leq 1$
    \begin{equation*}
        |\left<\phi \psi_{a_0st},\psi_{a_1s't'}\right>| \leq
        C_N \left(1+\frac{\delta}{a_0}\right)^{-N}
        \left(1+\frac{\Vert t-t' \Vert^2}{\delta}\right)^{-N}
    \end{equation*}
\end{enumerate}
\end{lemma}

% Das jetzt folgende Lemma fehlt in \textcite{Kutyniok2008}, weshalb auf Seite 26 fälschlicherweise benutzt werden muss, dass $f$ beschränkt ist. Es reicht aber, dass $\mathcal{S}_f(a,s,t)$ für $\Vert t \Vert \to \infty$ polynomiell beschränkt ist. Mehr geht auch gar nicht, da $f$ i.A. eine temperierte Distribution ist.

% \begin{lemma}[$|\left<f,\psi_{ast}\right>|$ ist langsam wachsend]
% Sei $f \in \mathcal{S}'(\mathbb{R}^2)$. Dann gibt es ein $N \in \mathbb{N}$ s.d. für alle $a \in [0,1]$ und $s \in [-1,1]$
% \begin{equation*}
%     \left|\left\langle f, \psi_{ast}\right\rangle\right| \leq C_N (1+\Vert t\Vert^2)^{N}
% \end{equation*}
% \end{lemma}

% \begin{proof}
% $\psi_{ast} \in \mathcal{S}$ für alle $a,s,t$, da $\hat \psi_{ast}$ kompakt getragen ist für alle $a,s,t$. Also gibt es per Definition von $\mathcal{S}'$ $C \in \mathbb{R}, M,N \in \mathbb{N}$ so dass

% \begin{equation*}
%     \left|\left\langle f, \psi_{ast} \right\rangle \right|
%     \leq
%     C \sum_{|\alpha| \leq N, |\beta|\leq M} \sup_{x \in \mathbb{R}^2} \left| x^\alpha \partial^\beta \psi_{ast}(x)\right|
% \end{equation*}
% \todo{hier den Beweis fertig machen}
% \end{proof}

Kommen wir nun zu dem Beweis unseres Hauptsatzes:

\begin{proof}[von \ref{thm:main_theorem}]
\label{proof:main_theorem}
Zunächst die einfachere Richtung, nämlich $WF(f)^c \subseteq \mathcal{D}$.
Wir nehmen also einen gerichteten regulären Punkt $(s_0,t_0) \in WF(f)^c$ und zeigen, dass er auch in $\mathcal{D}$ liegt. Dazu zerlegen wir $f$ zunächst wie folgt:
 Da $f$ bei $t_0$ in Richtung $s_0$ regulär ist, gibt es per Definition der Wellenfrontmenge ein $\phi \in C_0^\infty(\mathbb{R}^2)$ s.d. $\phi = 1$ in einer Umgebung von $t_0$ und für alle $N \in \mathbb{N}$ $\rwhat{\phi f} = O(1+|k|)^{-N}$ für $\frac{k_2}{k_1}$ in einer Umgebung von $s_0$. Dementsprechend ist $(1-\phi)f = 0$ in einer Umgebung von $t_0$ und es gilt

 \begin{equation}
     \mathcal{S}_f (a,s,t = \left\langle \psi_{ast},\phi f \right\rangle
                                + \left\langle \psi_{ast},(1-\phi) f \right\rangle
 \label{eq:schlaue sache}
 \end{equation}

\begin{figure}[h]
\centering
%% Creator: Matplotlib, PGF backend
%%
%% To include the figure in your LaTeX document, write
%%   \input{<filename>.pgf}
%%
%% Make sure the required packages are loaded in your preamble
%%   \usepackage{pgf}
%%
%% Figures using additional raster images can only be included by \input if
%% they are in the same directory as the main LaTeX file. For loading figures
%% from other directories you can use the `import` package
%%   \usepackage{import}
%% and then include the figures with
%%   \import{<path to file>}{<filename>.pgf}
%%
%% Matplotlib used the following preamble
%%   \usepackage[utf8x]{inputenc}
%%   \usepackage[T1]{fontenc}
%%   \usepackage{amssymb}
%%
\begingroup%
\makeatletter%
\begin{pgfpicture}%
\pgfpathrectangle{\pgfpointorigin}{\pgfqpoint{4.000000in}{2.200000in}}%
\pgfusepath{use as bounding box, clip}%
\begin{pgfscope}%
\pgfsetbuttcap%
\pgfsetmiterjoin%
\definecolor{currentfill}{rgb}{1.000000,1.000000,1.000000}%
\pgfsetfillcolor{currentfill}%
\pgfsetlinewidth{0.000000pt}%
\definecolor{currentstroke}{rgb}{1.000000,1.000000,1.000000}%
\pgfsetstrokecolor{currentstroke}%
\pgfsetdash{}{0pt}%
\pgfpathmoveto{\pgfqpoint{0.000000in}{0.000000in}}%
\pgfpathlineto{\pgfqpoint{4.000000in}{0.000000in}}%
\pgfpathlineto{\pgfqpoint{4.000000in}{2.200000in}}%
\pgfpathlineto{\pgfqpoint{0.000000in}{2.200000in}}%
\pgfpathclose%
\pgfusepath{fill}%
\end{pgfscope}%
\begin{pgfscope}%
\pgfsetbuttcap%
\pgfsetmiterjoin%
\definecolor{currentfill}{rgb}{1.000000,1.000000,1.000000}%
\pgfsetfillcolor{currentfill}%
\pgfsetlinewidth{0.000000pt}%
\definecolor{currentstroke}{rgb}{0.000000,0.000000,0.000000}%
\pgfsetstrokecolor{currentstroke}%
\pgfsetstrokeopacity{0.000000}%
\pgfsetdash{}{0pt}%
\pgfpathmoveto{\pgfqpoint{0.500000in}{0.275000in}}%
\pgfpathlineto{\pgfqpoint{3.600000in}{0.275000in}}%
\pgfpathlineto{\pgfqpoint{3.600000in}{1.936000in}}%
\pgfpathlineto{\pgfqpoint{0.500000in}{1.936000in}}%
\pgfpathclose%
\pgfusepath{fill}%
\end{pgfscope}%
\begin{pgfscope}%
\pgfsetbuttcap%
\pgfsetroundjoin%
\definecolor{currentfill}{rgb}{0.000000,0.000000,0.000000}%
\pgfsetfillcolor{currentfill}%
\pgfsetlinewidth{0.803000pt}%
\definecolor{currentstroke}{rgb}{0.000000,0.000000,0.000000}%
\pgfsetstrokecolor{currentstroke}%
\pgfsetdash{}{0pt}%
\pgfsys@defobject{currentmarker}{\pgfqpoint{0.000000in}{-0.048611in}}{\pgfqpoint{0.000000in}{0.000000in}}{%
\pgfpathmoveto{\pgfqpoint{0.000000in}{0.000000in}}%
\pgfpathlineto{\pgfqpoint{0.000000in}{-0.048611in}}%
\pgfusepath{stroke,fill}%
}%
\begin{pgfscope}%
\pgfsys@transformshift{2.050000in}{0.344208in}%
\pgfsys@useobject{currentmarker}{}%
\end{pgfscope}%
\end{pgfscope}%
\begin{pgfscope}%
\pgftext[x=2.050000in,y=0.246986in,,top]{\rmfamily\fontsize{10.000000}{12.000000}\selectfont \(\displaystyle (t_0, x_0)\)}%
\end{pgfscope}%
\begin{pgfscope}%
\pgfpathrectangle{\pgfqpoint{0.500000in}{0.275000in}}{\pgfqpoint{3.100000in}{1.661000in}}%
\pgfusepath{clip}%
\pgfsetrectcap%
\pgfsetroundjoin%
\pgfsetlinewidth{0.501875pt}%
\definecolor{currentstroke}{rgb}{0.894118,0.101961,0.109804}%
\pgfsetstrokecolor{currentstroke}%
\pgfsetdash{}{0pt}%
\pgfpathmoveto{\pgfqpoint{0.500000in}{0.344208in}}%
\pgfpathlineto{\pgfqpoint{1.778478in}{0.345597in}}%
\pgfpathlineto{\pgfqpoint{1.803303in}{0.348841in}}%
\pgfpathlineto{\pgfqpoint{1.818819in}{0.353486in}}%
\pgfpathlineto{\pgfqpoint{1.831231in}{0.359861in}}%
\pgfpathlineto{\pgfqpoint{1.843644in}{0.369866in}}%
\pgfpathlineto{\pgfqpoint{1.856056in}{0.385070in}}%
\pgfpathlineto{\pgfqpoint{1.865365in}{0.401049in}}%
\pgfpathlineto{\pgfqpoint{1.874675in}{0.422003in}}%
\pgfpathlineto{\pgfqpoint{1.883984in}{0.448968in}}%
\pgfpathlineto{\pgfqpoint{1.896396in}{0.496107in}}%
\pgfpathlineto{\pgfqpoint{1.908809in}{0.558191in}}%
\pgfpathlineto{\pgfqpoint{1.921221in}{0.637077in}}%
\pgfpathlineto{\pgfqpoint{1.936737in}{0.760517in}}%
\pgfpathlineto{\pgfqpoint{1.952252in}{0.909895in}}%
\pgfpathlineto{\pgfqpoint{1.977077in}{1.185411in}}%
\pgfpathlineto{\pgfqpoint{2.005005in}{1.489311in}}%
\pgfpathlineto{\pgfqpoint{2.017417in}{1.597376in}}%
\pgfpathlineto{\pgfqpoint{2.026727in}{1.659912in}}%
\pgfpathlineto{\pgfqpoint{2.036036in}{1.703327in}}%
\pgfpathlineto{\pgfqpoint{2.042242in}{1.720595in}}%
\pgfpathlineto{\pgfqpoint{2.048448in}{1.728063in}}%
\pgfpathlineto{\pgfqpoint{2.051552in}{1.728063in}}%
\pgfpathlineto{\pgfqpoint{2.054655in}{1.725569in}}%
\pgfpathlineto{\pgfqpoint{2.060861in}{1.713168in}}%
\pgfpathlineto{\pgfqpoint{2.067067in}{1.691126in}}%
\pgfpathlineto{\pgfqpoint{2.076376in}{1.641065in}}%
\pgfpathlineto{\pgfqpoint{2.085686in}{1.572759in}}%
\pgfpathlineto{\pgfqpoint{2.098098in}{1.458747in}}%
\pgfpathlineto{\pgfqpoint{2.119820in}{1.221041in}}%
\pgfpathlineto{\pgfqpoint{2.150851in}{0.878170in}}%
\pgfpathlineto{\pgfqpoint{2.169469in}{0.707845in}}%
\pgfpathlineto{\pgfqpoint{2.184985in}{0.595451in}}%
\pgfpathlineto{\pgfqpoint{2.197397in}{0.525148in}}%
\pgfpathlineto{\pgfqpoint{2.209810in}{0.470810in}}%
\pgfpathlineto{\pgfqpoint{2.222222in}{0.430271in}}%
\pgfpathlineto{\pgfqpoint{2.234635in}{0.401049in}}%
\pgfpathlineto{\pgfqpoint{2.247047in}{0.380681in}}%
\pgfpathlineto{\pgfqpoint{2.259459in}{0.366946in}}%
\pgfpathlineto{\pgfqpoint{2.271872in}{0.357980in}}%
\pgfpathlineto{\pgfqpoint{2.287387in}{0.351274in}}%
\pgfpathlineto{\pgfqpoint{2.306006in}{0.347197in}}%
\pgfpathlineto{\pgfqpoint{2.333934in}{0.344936in}}%
\pgfpathlineto{\pgfqpoint{2.395996in}{0.344227in}}%
\pgfpathlineto{\pgfqpoint{3.600000in}{0.344208in}}%
\pgfpathlineto{\pgfqpoint{3.600000in}{0.344208in}}%
\pgfusepath{stroke}%
\end{pgfscope}%
\begin{pgfscope}%
\pgfpathrectangle{\pgfqpoint{0.500000in}{0.275000in}}{\pgfqpoint{3.100000in}{1.661000in}}%
\pgfusepath{clip}%
\pgfsetrectcap%
\pgfsetroundjoin%
\pgfsetlinewidth{0.501875pt}%
\definecolor{currentstroke}{rgb}{0.215686,0.494118,0.721569}%
\pgfsetstrokecolor{currentstroke}%
\pgfsetdash{}{0pt}%
\pgfpathmoveto{\pgfqpoint{0.500000in}{0.344208in}}%
\pgfpathlineto{\pgfqpoint{1.744344in}{0.344919in}}%
\pgfpathlineto{\pgfqpoint{1.750551in}{0.350590in}}%
\pgfpathlineto{\pgfqpoint{1.756757in}{0.361837in}}%
\pgfpathlineto{\pgfqpoint{1.766066in}{0.388737in}}%
\pgfpathlineto{\pgfqpoint{1.775375in}{0.426758in}}%
\pgfpathlineto{\pgfqpoint{1.787788in}{0.492274in}}%
\pgfpathlineto{\pgfqpoint{1.803303in}{0.591792in}}%
\pgfpathlineto{\pgfqpoint{1.849850in}{0.906003in}}%
\pgfpathlineto{\pgfqpoint{1.862262in}{0.967573in}}%
\pgfpathlineto{\pgfqpoint{1.871572in}{1.002023in}}%
\pgfpathlineto{\pgfqpoint{1.880881in}{1.024974in}}%
\pgfpathlineto{\pgfqpoint{1.887087in}{1.033451in}}%
\pgfpathlineto{\pgfqpoint{1.893293in}{1.036292in}}%
\pgfpathlineto{\pgfqpoint{2.209810in}{1.035581in}}%
\pgfpathlineto{\pgfqpoint{2.216016in}{1.029910in}}%
\pgfpathlineto{\pgfqpoint{2.222222in}{1.018663in}}%
\pgfpathlineto{\pgfqpoint{2.231532in}{0.991763in}}%
\pgfpathlineto{\pgfqpoint{2.240841in}{0.953742in}}%
\pgfpathlineto{\pgfqpoint{2.253253in}{0.888226in}}%
\pgfpathlineto{\pgfqpoint{2.268769in}{0.788708in}}%
\pgfpathlineto{\pgfqpoint{2.315315in}{0.474497in}}%
\pgfpathlineto{\pgfqpoint{2.327728in}{0.412927in}}%
\pgfpathlineto{\pgfqpoint{2.337037in}{0.378477in}}%
\pgfpathlineto{\pgfqpoint{2.346346in}{0.355526in}}%
\pgfpathlineto{\pgfqpoint{2.352553in}{0.347049in}}%
\pgfpathlineto{\pgfqpoint{2.358759in}{0.344208in}}%
\pgfpathlineto{\pgfqpoint{3.600000in}{0.344208in}}%
\pgfpathlineto{\pgfqpoint{3.600000in}{0.344208in}}%
\pgfusepath{stroke}%
\end{pgfscope}%
\begin{pgfscope}%
\pgfpathrectangle{\pgfqpoint{0.500000in}{0.275000in}}{\pgfqpoint{3.100000in}{1.661000in}}%
\pgfusepath{clip}%
\pgfsetrectcap%
\pgfsetroundjoin%
\pgfsetlinewidth{0.501875pt}%
\definecolor{currentstroke}{rgb}{0.301961,0.686275,0.290196}%
\pgfsetstrokecolor{currentstroke}%
\pgfsetdash{}{0pt}%
\pgfpathmoveto{\pgfqpoint{0.500000in}{1.036292in}}%
\pgfpathlineto{\pgfqpoint{1.744344in}{1.035581in}}%
\pgfpathlineto{\pgfqpoint{1.750551in}{1.029910in}}%
\pgfpathlineto{\pgfqpoint{1.756757in}{1.018663in}}%
\pgfpathlineto{\pgfqpoint{1.766066in}{0.991763in}}%
\pgfpathlineto{\pgfqpoint{1.775375in}{0.953742in}}%
\pgfpathlineto{\pgfqpoint{1.787788in}{0.888226in}}%
\pgfpathlineto{\pgfqpoint{1.803303in}{0.788708in}}%
\pgfpathlineto{\pgfqpoint{1.849850in}{0.474497in}}%
\pgfpathlineto{\pgfqpoint{1.862262in}{0.412927in}}%
\pgfpathlineto{\pgfqpoint{1.871572in}{0.378477in}}%
\pgfpathlineto{\pgfqpoint{1.880881in}{0.355526in}}%
\pgfpathlineto{\pgfqpoint{1.887087in}{0.347049in}}%
\pgfpathlineto{\pgfqpoint{1.893293in}{0.344208in}}%
\pgfpathlineto{\pgfqpoint{2.209810in}{0.344919in}}%
\pgfpathlineto{\pgfqpoint{2.216016in}{0.350590in}}%
\pgfpathlineto{\pgfqpoint{2.222222in}{0.361837in}}%
\pgfpathlineto{\pgfqpoint{2.231532in}{0.388737in}}%
\pgfpathlineto{\pgfqpoint{2.240841in}{0.426758in}}%
\pgfpathlineto{\pgfqpoint{2.253253in}{0.492274in}}%
\pgfpathlineto{\pgfqpoint{2.268769in}{0.591792in}}%
\pgfpathlineto{\pgfqpoint{2.315315in}{0.906003in}}%
\pgfpathlineto{\pgfqpoint{2.327728in}{0.967573in}}%
\pgfpathlineto{\pgfqpoint{2.337037in}{1.002023in}}%
\pgfpathlineto{\pgfqpoint{2.346346in}{1.024974in}}%
\pgfpathlineto{\pgfqpoint{2.352553in}{1.033451in}}%
\pgfpathlineto{\pgfqpoint{2.358759in}{1.036292in}}%
\pgfpathlineto{\pgfqpoint{3.600000in}{1.036292in}}%
\pgfpathlineto{\pgfqpoint{3.600000in}{1.036292in}}%
\pgfusepath{stroke}%
\end{pgfscope}%
\begin{pgfscope}%
\pgfsetrectcap%
\pgfsetmiterjoin%
\pgfsetlinewidth{0.501875pt}%
\definecolor{currentstroke}{rgb}{0.000000,0.000000,0.000000}%
\pgfsetstrokecolor{currentstroke}%
\pgfsetdash{}{0pt}%
\pgfpathmoveto{\pgfqpoint{2.050000in}{0.275000in}}%
\pgfpathlineto{\pgfqpoint{2.050000in}{1.936000in}}%
\pgfusepath{stroke}%
\end{pgfscope}%
\begin{pgfscope}%
\pgfsetrectcap%
\pgfsetmiterjoin%
\pgfsetlinewidth{0.501875pt}%
\definecolor{currentstroke}{rgb}{0.000000,0.000000,0.000000}%
\pgfsetstrokecolor{currentstroke}%
\pgfsetdash{}{0pt}%
\pgfpathmoveto{\pgfqpoint{0.500000in}{0.344208in}}%
\pgfpathlineto{\pgfqpoint{3.600000in}{0.344208in}}%
\pgfusepath{stroke}%
\end{pgfscope}%
\begin{pgfscope}%
\pgfsetroundcap%
\pgfsetroundjoin%
\pgfsetlinewidth{0.501875pt}%
\definecolor{currentstroke}{rgb}{0.000000,0.000000,0.000000}%
\pgfsetstrokecolor{currentstroke}%
\pgfsetdash{}{0pt}%
\pgfpathmoveto{\pgfqpoint{2.050000in}{1.942121in}}%
\pgfpathquadraticcurveto{\pgfqpoint{2.050000in}{1.942943in}}{\pgfqpoint{2.050000in}{1.936000in}}%
\pgfusepath{stroke}%
\end{pgfscope}%
\begin{pgfscope}%
\pgfsetroundcap%
\pgfsetroundjoin%
\pgfsetlinewidth{0.501875pt}%
\definecolor{currentstroke}{rgb}{0.000000,0.000000,0.000000}%
\pgfsetstrokecolor{currentstroke}%
\pgfsetdash{}{0pt}%
\pgfpathmoveto{\pgfqpoint{2.022222in}{1.886565in}}%
\pgfpathlineto{\pgfqpoint{2.050000in}{1.942121in}}%
\pgfpathlineto{\pgfqpoint{2.077778in}{1.886565in}}%
\pgfusepath{stroke}%
\end{pgfscope}%
\begin{pgfscope}%
\pgftext[x=2.050000in,y=2.005444in,,bottom]{\rmfamily\fontsize{10.000000}{12.000000}\selectfont \(\displaystyle  f\)}%
\end{pgfscope}%
\begin{pgfscope}%
\pgfsetroundcap%
\pgfsetroundjoin%
\pgfsetlinewidth{0.501875pt}%
\definecolor{currentstroke}{rgb}{0.000000,0.000000,0.000000}%
\pgfsetstrokecolor{currentstroke}%
\pgfsetdash{}{0pt}%
\pgfpathmoveto{\pgfqpoint{3.606111in}{0.344208in}}%
\pgfpathquadraticcurveto{\pgfqpoint{3.606938in}{0.344208in}}{\pgfqpoint{3.600000in}{0.344208in}}%
\pgfusepath{stroke}%
\end{pgfscope}%
\begin{pgfscope}%
\pgfsetroundcap%
\pgfsetroundjoin%
\pgfsetlinewidth{0.501875pt}%
\definecolor{currentstroke}{rgb}{0.000000,0.000000,0.000000}%
\pgfsetstrokecolor{currentstroke}%
\pgfsetdash{}{0pt}%
\pgfpathmoveto{\pgfqpoint{3.550556in}{0.371986in}}%
\pgfpathlineto{\pgfqpoint{3.606111in}{0.344208in}}%
\pgfpathlineto{\pgfqpoint{3.550556in}{0.316431in}}%
\pgfusepath{stroke}%
\end{pgfscope}%
\begin{pgfscope}%
\pgftext[x=3.669444in,y=0.344208in,left,]{\rmfamily\fontsize{10.000000}{12.000000}\selectfont \(\displaystyle (x,t)\)}%
\end{pgfscope}%
\begin{pgfscope}%
\pgfsetbuttcap%
\pgfsetmiterjoin%
\definecolor{currentfill}{rgb}{1.000000,1.000000,1.000000}%
\pgfsetfillcolor{currentfill}%
\pgfsetfillopacity{0.800000}%
\pgfsetlinewidth{0.501875pt}%
\definecolor{currentstroke}{rgb}{0.800000,0.800000,0.800000}%
\pgfsetstrokecolor{currentstroke}%
\pgfsetstrokeopacity{0.800000}%
\pgfsetdash{}{0pt}%
\pgfpathmoveto{\pgfqpoint{2.736382in}{1.243871in}}%
\pgfpathlineto{\pgfqpoint{3.502778in}{1.243871in}}%
\pgfpathquadraticcurveto{\pgfqpoint{3.530556in}{1.243871in}}{\pgfqpoint{3.530556in}{1.271648in}}%
\pgfpathlineto{\pgfqpoint{3.530556in}{1.838778in}}%
\pgfpathquadraticcurveto{\pgfqpoint{3.530556in}{1.866556in}}{\pgfqpoint{3.502778in}{1.866556in}}%
\pgfpathlineto{\pgfqpoint{2.736382in}{1.866556in}}%
\pgfpathquadraticcurveto{\pgfqpoint{2.708604in}{1.866556in}}{\pgfqpoint{2.708604in}{1.838778in}}%
\pgfpathlineto{\pgfqpoint{2.708604in}{1.271648in}}%
\pgfpathquadraticcurveto{\pgfqpoint{2.708604in}{1.243871in}}{\pgfqpoint{2.736382in}{1.243871in}}%
\pgfpathclose%
\pgfusepath{stroke,fill}%
\end{pgfscope}%
\begin{pgfscope}%
\pgfsetrectcap%
\pgfsetroundjoin%
\pgfsetlinewidth{0.501875pt}%
\definecolor{currentstroke}{rgb}{0.894118,0.101961,0.109804}%
\pgfsetstrokecolor{currentstroke}%
\pgfsetdash{}{0pt}%
\pgfpathmoveto{\pgfqpoint{2.764160in}{1.762389in}}%
\pgfpathlineto{\pgfqpoint{3.041937in}{1.762389in}}%
\pgfusepath{stroke}%
\end{pgfscope}%
\begin{pgfscope}%
\pgftext[x=3.153049in,y=1.713778in,left,base]{\rmfamily\fontsize{10.000000}{12.000000}\selectfont \(\displaystyle \psi\)}%
\end{pgfscope}%
\begin{pgfscope}%
\pgfsetrectcap%
\pgfsetroundjoin%
\pgfsetlinewidth{0.501875pt}%
\definecolor{currentstroke}{rgb}{0.215686,0.494118,0.721569}%
\pgfsetstrokecolor{currentstroke}%
\pgfsetdash{}{0pt}%
\pgfpathmoveto{\pgfqpoint{2.764160in}{1.568716in}}%
\pgfpathlineto{\pgfqpoint{3.041937in}{1.568716in}}%
\pgfusepath{stroke}%
\end{pgfscope}%
\begin{pgfscope}%
\pgftext[x=3.153049in,y=1.520105in,left,base]{\rmfamily\fontsize{10.000000}{12.000000}\selectfont \(\displaystyle \phi\)}%
\end{pgfscope}%
\begin{pgfscope}%
\pgfsetrectcap%
\pgfsetroundjoin%
\pgfsetlinewidth{0.501875pt}%
\definecolor{currentstroke}{rgb}{0.301961,0.686275,0.290196}%
\pgfsetstrokecolor{currentstroke}%
\pgfsetdash{}{0pt}%
\pgfpathmoveto{\pgfqpoint{2.764160in}{1.375043in}}%
\pgfpathlineto{\pgfqpoint{3.041937in}{1.375043in}}%
\pgfusepath{stroke}%
\end{pgfscope}%
\begin{pgfscope}%
\pgftext[x=3.153049in,y=1.326432in,left,base]{\rmfamily\fontsize{10.000000}{12.000000}\selectfont \(\displaystyle 1-\phi\)}%
\end{pgfscope}%
\end{pgfpicture}%
\makeatother%
\endgroup%

\caption{Die Zerlegung von $f$ um $(t_0,x_0)$ herum visualisiert}
\label{fig:smart_decomposition}
\end{figure}

Da $(1-\phi)f$ in einer Umgebung von $t_0$ verschwindet und nach \cref{prop:shearlets_decay_rapidly} Shearlets außerhalb von $t$ schnell abfallen für $a \to 0$ fällt auch der zweite Term von \cref{eq:schlaue sache}
für $t \neq t_0$ schnell ab. Für den ersten Term überzeugen wir uns anhand von \cref{fig:supp_psi_hat,eq:supp_psi}, dass für $a$ klein genug $supp(\hat\psi_{ast})$ schließlich in jedem noch so kleinen Kegel um $s$ liegt. In einem solchen um $s_0$ fällt aber $\rwhat{\phi f}$ rapide ab nach Vorraussetzung und damit auch der erste Term in \eqref{eq:schlaue sache}.

Die beiden entscheidenden Zutaten waren hier also die Tatsache, dass die Shearlets außerhalb von $t$ rapide abfallen und damit bei immer feineren Skalen $a$ immer besser lokalisiert werden sowie die Tatsache, dass für $a \to 0$ der Träger im Frequenzbereich in immer engeren Kegeln liegt.

Deutlich schwieriger ist die umgekehrte Inklusion, nämlich dass die Shearlettransformation tatsächlich die ganze Wellenfrontmenge erkennt. Hier geht jetzt auch die Reproduktionseigenschaft der Transformation ein.

Für die umgekehrte Inklusion $\mathcal{D} \subseteq WF(f)^c$ haben wir zu zeigen, dass falls $\mathcal{S}_f (a,s,t)$ schnell abfällt in einer Umgebung $U$ von $(s_0, t_0)$ für $a \to 0$ dann auch $\rwhat{\phi f} (k)$ schnell abfällt für $\Vert(k)\Vert \to \infty$ für $\frac{k_2}{k_1}$ in einer Umgebung von $s_0$ und ein $\phi$ getragen in einer Umgebung von $t_0$.

Sei also $\mathcal{S}_f(a,s,t) \xrightarrow[a \to 0]{\text{\tiny schnell}} 0$
für $(s,t) \in B(s_0, 2 \Delta) \times B(t_0, 2 \delta) \eqqcolon S \times U$\footnote{Falls das ganze gilt für $(s,t)$ in einer offenen Umgebung von $(s_0,t_0)$, dann auch in solchen Bällen mit passenden $\Delta, \delta$}.
Sei $\phi \in C_0^\infty(B(t_0,\delta))$ und $\phi \equiv 1$ in einer Umgebung von $t_0$. Dann müssen wir zeigen, dass $\rwhat{\phi f}(k) \xrightarrow[a \to 0]{\text{\tiny schnell}} 0$ für $\frac{k_2}{k_1} \in S$

\begin{align*}
    \rwhat{\phi f} (k)
    &=
    \int \phi(x) f(x) e^{-ikx} \d x \\
    \\&\stackrel{\eqref{thm:shearlets_reproduzieren}}{=}
    \iint \left\langle \psi_{ast},\phi f \right\rangle
        \psi_{ast} (x) \d \mu (a,s,t)
        e^{-ikx} \d x
    \\ &=
    \int \left\langle \psi_{ast'},\phi f \right\rangle
    \hat\psi_{ast}(k) \d \mu(a,s,t)
    \\&= \kern -1em
    \underbrace{
        \int \limits_{U \times [-2,2] \times [0,1]} \kern -1em
        \left\langle \psi_{ast},\phi f \right\rangle
        \hat\psi_{ast}(k) \d \mu(a,s,t)
    }_{i)}
    \\& \kern 2em+ \kern -1em
    \underbrace{
        \int \limits_{U^c \times [-2,2] \times [0,1]} \kern -1em
        \left\langle \psi_{ast},\phi f \right\rangle
        \hat\psi_{ast}(k) \d \mu(a,s,t)
    }_{ii)}
\end{align*}
% \begin{dmath*}
%     \rwhat{\phi f} (k)
%     =
%     \int \phi(x) f(x) e^{-ikx} \d x \\
%     \stackrel{\ref{thm:shearlets_reproduzieren}}{=}
%     \iint \left\langle \psi_{ast},\phi f \right\rangle
%         \psi_{ast} (x) \d \mu (a,s,t)
%         e^{-ikx} \d x
%     =
%     \int \left\langle \psi_{ast'},\phi f \right\rangle
%     \hat\psi_{ast}(k) \d \mu(a,s,t)
%     = \kern -1em
%     \underbrace{
%         \int \limits_{U \times [-2,2] \times [0,1]} \kern -1em
%         \left\langle \psi_{ast},\phi f \right\rangle
%         \hat\psi_{ast}(k) \d \mu(a,s,t)
%     }_{i)}
%     + \kern -1em
%     \underbrace{
%         \int \limits_{U^c \times [-2,2] \times [0,1]} \kern -1em
%         \left\langle \psi_{ast},\phi f \right\rangle
%         \hat\psi_{ast}(k) \d \mu(a,s,t)
%     }_{ii)}
% \end{dmath*}

\paragraph*{zu $ii)$}
Per Konstruktion von $\phi$ gilt $d(supp(\phi f, U^c)) = \delta > 0$. Also fällt $ii)$ nach \cref{lemm:lemma54} schnell ab.

\paragraph*{zu $i)$}
Falls $\left<\psi_{ast},\phi f\right> \xrightarrow[a \to 0]{\text{\tiny schnell}} 0$ für $(s,t) \in S \times U$, dann fällt $ii)$ schnell ab, falls $\frac{k_2}{k_2} \in B(s_0,\Delta)$ nach \cref{lemm:ruecktrafo_fourier_faellt_schnell_ab}. Wir zeigen also, dass $\left<\psi_{ast},\phi f\right> \xrightarrow[a \to 0]{\text{\tiny schnell}} 0$ für $(s,t) \in S \times U$.

\begin{align*}
    \left\langle \phi f, \psi_{ast} \right\rangle
    &=
    \int \phi f \psi_{ast} \d x
    \\&=
    \int \phi(x) \int \left\langle f, \psi_{a' s' t'} \right\rangle
    \psi_{a's't'} (x) \d \mu(a',s',t') \,\psi_{ast}(x) \d x
    \\&=
    \int\limits_{0}^{1} \int\limits_{\mathbb{R}^2} \int\limits_{-2}^2
    \left\langle \phi \psi_{ast}, \psi_{a's't'} \right\rangle
    \left\langle f, \psi_{a's't'} \right\rangle
    \d s' \d t' \frac{\d a'}{a^{\prime 3}}
\end{align*}
% \begin{dmath*}
%     \left\langle \phi f, \psi_{ast} \right\rangle
%     =
%     \int \phi f \psi_{ast} \d x
%     =
%     \int \phi(x) \int \left\langle f, \psi_{a' s' t'} \right\rangle
%     \psi_{a's't'} (x) \d \mu(a',s',t') \,\psi_{ast}(x) \d x
%     =
%     \int\limits_{0}^{1} \int\limits_{\mathbb{R}^2} \int\limits_{-2}^2
%     \left\langle \phi \psi_{ast}, \psi_{a's't'} \right\rangle
%     \left\langle f, \psi_{a's't'} \right\rangle
%     \d s' \d t' \frac{\d a'}{a^{\prime 3}}
% \end{dmath*}


Für $a < \delta$\footnote{Und da wir uns für kleine $a$ interessieren, dürfen wir das auch direkt annehmen} teilen wir die Integration über $s'$ auf in die drei Fälle $a)$ $0<a'<a<\delta$, $b)$ $a < a' < \delta$ und $c)$ $\delta < a'$ und nutzen \cref{lemm:lemma57}. Außerdem teilen wir die Integration über $t'$ auf in die Fälle $t' \in U$ und $t' \in U^c$

\paragraph*{Zu $a)$, $t' \in U^c$}
Hier gilt im Integrationsbereich $\Vert t' - t \Vert \geq \delta > 0$. Mit \cref{lemm:lemma57}$.1)$ können wir für alle $N\in \mathbb{N}$ abschätzen wie folgt:
\begin{align*}
 &\kern-2em
 \int\limits_{0}^{a} \int\limits_{U^c} \int\limits_{-2}^2
 \left\langle \phi \psi_{ast}, \psi_{a's't'} \right\rangle
 \left\langle f, \psi_{a's't'} \right\rangle
    \d s' \d t' \frac{\d a'}{a^{\prime 3}}
\\&\leq
\int\limits_{0}^{a} \int\limits_{U^c} \int\limits_{-2}^2
C_N
\left| \left\langle f, \psi_{a's't'} \right\rangle \right|
\left(1+\frac{\Vert t-t' \Vert^2}{a}\right)^{-N} \d \mu(a',s',t')
\\&\leq
\int\limits_{0}^{a} \int\limits_{U^c} \int\limits_{-2}^2
C_N
\left| \left\langle f, \psi_{a's't'} \right\rangle \right|
a^N \Vert t-t' \Vert^{-N} \d \mu(a',s',t')
\\&\leq
C_N a^N
\int\limits_{0}^{a} \int\limits_{U^c} \int\limits_{-2}^2
\left| \left\langle f, \psi_{a's't'} \right\rangle \right|
\Vert t-t' \Vert^{-N} \d \mu(a',s',t')
\end{align*}
% \begin{dmath*}
%  \int\limits_{0}^{a} \int\limits_{U^c} \int\limits_{-2}^2
%  \left\langle \phi \psi_{ast}, \psi_{a's't'} \right\rangle
%  \left\langle f, \psi_{a's't'} \right\rangle
%     \d s' \d t' \frac{\d a'}{a^{\prime 3}}
% \leq
% \int\limits_{0}^{a} \int\limits_{U^c} \int\limits_{-2}^2
% C_N
% \left| \left\langle f, \psi_{a's't'} \right\rangle \right|
% \left(1+\frac{\Vert t-t' \Vert^2}{a}\right)^{-N} \d \mu(a',s',t')
% \leq
% \int\limits_{0}^{a} \int\limits_{U^c} \int\limits_{-2}^2
% C_N
% \left| \left\langle f, \psi_{a's't'} \right\rangle \right|
% a^N \Vert t-t' \Vert^{-N} \d \mu(a',s',t')
% \leq
% C_N a^N
% \int\limits_{0}^{a} \int\limits_{U^c} \int\limits_{-2}^2
% \left| \left\langle f, \psi_{a's't'} \right\rangle \right|
% \Vert t-t' \Vert^{-N} \d \mu(a',s',t')
% \end{dmath*}

Wobei das letzte Integral endlich ist, da $\Vert t-t'\Vert \geq \delta$ im Integrationsbereich und $\left| \left\langle f, \psi_{a's't'} \right\rangle \right|$ beschränkt, da $f$ beschränkt ist. Für kleinere $a$ kann es nur kleiner werden.


\paragraph*{Zu $a)$, $t' \in U$}
Eine kurze Erinnerung: Falls $(s',t') \in S \times U$, gilt nach Vorraussetzung für alle $N \in \mathbb{N}$:
$\left| \left\langle f, \psi_{a's't'} \right\rangle \right| \leq C_N a^N$.
Außerdem gilt für $s' \notin B(s_0,\Delta) : ~$ $|s-s'|  \geq \Delta$ für alle $s$ die wir hier betrachten. Also:

\begin{align*}
&\kern -2em
 \int\limits_{0}^{a} \int\limits_{U} \int\limits_{-2}^2
 \left\langle \phi \psi_{ast}, \psi_{a's't'} \right\rangle
 \left\langle f, \psi_{a's't'} \right\rangle
    \d \mu(a',s',t')
\\&=
 \int\limits_{0}^{a} \int\limits_{U} \int\limits_{B(s_0,\Delta)}
 \dittotikz
    \d \mu(a',s',t')
    +
 \int\limits_{0}^{a} \int\limits_{U} \int\limits_{B(s_0,\Delta)^c}
 \dittotikz
    \d \mu(a',s',t')
\\&=
 \int\limits_{0}^{a} \int\limits_{U} \int\limits_{B(s_0,\Delta)} C_N a^N
 \d \mu(a',s',t')
 \\&\kern +2em+
 \int\limits_{0}^{a} \int\limits_{U} \int\limits_{B(s_0,\Delta)^c}
 |\left\langle f, \psi_{a's't'} \right\rangle|
 \left(1+\frac{|s-s'|^2}{a}\right)^{-N}
 \d \mu(a',s',t')
 \\ &\leq
 C_N a^N +
  \int\limits_{0}^{a} \int\limits_{U} \int\limits_{B(s_0,\Delta)^c}
 \left\langle f, \psi_{a's't'} \right\rangle
 a^N \Delta^{-2N}
 \d \mu(a',s',t')
 \\&\leq
 C_N a^N
\end{align*}
% \begin{dmath*}
%  \int\limits_{0}^{a} \int\limits_{U} \int\limits_{-2}^2
%  \left\langle \phi \psi_{ast}, \psi_{a's't'} \right\rangle
%  \left\langle f, \psi_{a's't'} \right\rangle
%     \d \mu(a',s',t')
% =
%  \int\limits_{0}^{a} \int\limits_{U} \int\limits_{B(s_0,\Delta)}
%  \cdots
%     \d \mu(a',s',t')
%     +
%  \int\limits_{0}^{a} \int\limits_{U} \int\limits_{B(s_0,\Delta)^c}
%  \cdots
%     \d \mu(a',s',t')
% =
%  \int\limits_{0}^{a} \int\limits_{U} \int\limits_{B(s_0,\Delta)} C_N a^N
%  \d \mu(a',s',t')
%  +
%  \int\limits_{0}^{a} \int\limits_{U} \int\limits_{B(s_0,\Delta)^c}
%  |\left\langle f, \psi_{a's't'} \right\rangle|
%  \left(1+\frac{|s-s'|^2}{a}\right)^{-N}
%  \d \mu(a',s',t')
%  \leq
%  C_N a^N +
%   \int\limits_{0}^{a} \int\limits_{U} \int\limits_{B(s_0,\Delta)^c}
%  \left\langle f, \psi_{a's't'} \right\rangle
%  a^N \Delta^{-2N}
%  \d \mu(a',s',t')
%  \leq
%  C_N a^N
% \end{dmath*}

Mit analogen Abschätzungen und \cref{lemm:lemma57} erhalten wir auch noch, dass auch die Integrale zu $b)$ und $c)$ schnell abfallen in $a$.

\raggedleft{$QED$}
\end{proof}



% section beweis_von_thm:main_theorem (end)
