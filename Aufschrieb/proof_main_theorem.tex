%!TEX root = main.tex
%%%%%%%%%%%%%%%%%%%%%%%%%%%%%%%%%%%%%%%%%%%%%%%%%%%%%%%%%%%%%%%%%%%%%%%%%%%%%%%
% % Section 1
%%%%%%%%%%%%%%%%%%%%%%%%%%%%%%%%%%%%%%%%%%%%%%%%%%%%%%%%%%%%%%%%%%%%%%%%%%%%%%%
\section{Beweis von \cref{thm:main_theorem}} % (fold)
\label{sec:beweis_von_thm:main_theorem}


Bevor wir \cref{thm:main_theorem} beweisen können, benötigen wir aber noch ein paar technische Lemmata. Beweise dafür finden sich in \textcite{Kutyniok2008}

\begin{lemma}
\label{lemm:ruecktrafo_fourier_faellt_schnell_ab}

Seien $B(s_0,\Delta) \subset [-1,1]$ und $V \subset \mathbb{R}^2$ beschränkt. Nehme an, dass $G(a,s,t)$ schnell abfällt für $a \to 0$ gleichmäßig für $(s,t) \in  B(s_0,\Delta) \times V$. Dann fällt

\begin{equation*}
    \hat h(k) = \int \limits_0^1 \int \limits_V \int \limits_{[-1,1]}
    G(a,s,t) \hat \psi_{ast} (k)
        \d s \d t \frac{\d a}{a^3}
\end{equation*}

schnell ab, für $\Vert k \Vert \to \infty$ und $\frac{k_2}{k_1}$ in einer Umgebung von $s_0$.
\end{lemma}

\begin{proof}
Es sei
\begin{equation*}
    \Gamma_k = \left\{a\in [0,1], s \in [-1,1] \big| \tfrac{1}{2} \leq a|k| \leq 2 , \left|s-\tfrac{k_2}{k_1} \right| \leq \sqrt a
                   \right\}
\end{equation*}

Dann können wir dank \cref{eq:supp_psi} abschätzen

\begin{equation*}
    | \hat \psi_{ast} (k)| \leq C' a^{\frac{3}{4}} \chi_{\Gamma_k}
\end{equation*}

und nach Vorraussetzung gilt auch

\begin{equation*}
    |G(a,s,t)| \leq C_{N} a^{N} \condition{$\forall N \in \mathbb{N}$}
\end{equation*}

Außerdem sei
\begin{equation*}
    S = B(s_0,\Delta/2)
\end{equation*}

Um $\hat h(k)$ abzuschätzen, teilen wir es in den Bereich auf, in dem G(a,s,t) schnell abfällt, und in dem es nicht schnell abfällt:

\begin{dmath*}
    \hat h(k) =
    \underbrace{
    \int \limits_0^1 \int \limits_V \int \limits_{S}
    G(a,s,t) \hat \psi_{ast} (k)
        \d s \d t \frac{\d a}{a^3}}_{i)}
     +
    \underbrace{
     \int \limits_0^1 \int \limits_V \int \limits_{[-1,1]\setminus S}
    G(a,s,t) \hat \psi_{ast} (k)
        \d s \d t \frac{\d a}{a^3}}_{ii)}
\end{dmath*}


\emph{zu $i)$}

\begin{dmath*}
    i) \leq \int \limits_0^1 \int \limits_V \int \limits_{S}
    \left\lvert G(a,s,t)\right\rvert
    \left\lvert \hat \psi_{ast} (k) \right\rvert
        \d s \d t \frac{\d a}{a^3}
    \leq
    C_N C'
    \int \limits_0^1 \int \limits_V \int \limits_{S}
    a^{\frac{3}{4}} a^N \chi_{\Gamma_k} \d s \d t \frac{\d a}{a^3}
    \leq
    C_N \int \limits_{\frac{1}{2|k|}}^{\frac{2}{|k|}}
    a^{N-\frac{9}{4}} \d a
    \le C_N |k|^{-N+\frac{7}{4}}
\end{dmath*}

$i)$ fällt also schnell ab für $a \to 0$.


\emph{zu $ii)$}

\begin{dmath*}
    ii) \leq
     \int \limits_0^1 \int \limits_V \int \limits_{[-1,1]\setminus S}
    |G(a,s,t)| |\hat \psi_{ast} (k)|
        \d s \d t \frac{\d a}{a^3}
    \leq
    C' \int \limits_{0}^{1} \int \limits_{V} \int \limits_{[-1,1]\setminus S}
    |G(a,s,t)| \chi_{\Gamma_k} a^{\frac{3}{4}}
    \d s \d t \frac{\d a}{a^3}
\end{dmath*}

Für alle hinreichend großen $k$ ist aber $\Gamma_k \subset S$, also $\Gamma_k \cap [-1,1]\setminus S = \varnothing$ und demnach das Integral 0. Also

\begin{equation*}
    ii) = 0 \condition{für alle k groß genug}
\end{equation*}
\end{proof}


\begin{corollary}
    [Abschätzungen für $\left<\phi \psi_{a_0st},\psi_{a_1s't'}\right>$]
Sei $\phi \in C_0^\infty(B(t,\delta))$. Dann gilt für alle $N>0$

\begin{enumerate}
    \item Falls $0 \leq \sqrt{a_0} \leq \sqrt{a_1}\leq \delta \leq 1$
    \begin{equation*}
        |\left<\phi \psi_{a_0st},\psi_{a_1s't'}\right>| \leq
        C_N \left(1+\frac{a_1}{a_0}\right)^{-N}
        \left(1+\frac{|s-s'|^2}{a_1}\right)^{-N}
        \left(1+\frac{\Vert t-t' \Vert^2}{a_1}\right)^{-N}
    \end{equation*}
    \item Falls $0 \leq \sqrt{a_0} \leq \delta \leq \sqrt{a_1} \leq 1$
    \begin{equation*}
        |\left<\phi \psi_{a_0st},\psi_{a_1s't'}\right>| \leq
        C_N \left(1+\frac{a_1}{a_0}\right)^{-N}
        \left(1+\frac{|s-s'|^2}{\delta^2}\right)^{-N}
        \left(1+\frac{\Vert t-t' \Vert^2}{a_1}\right)^{-N}
    \end{equation*}
\end{enumerate}
\end{corollary}

\begin{proof}[von \ref{thm:main_theorem}]
Zunächst die einfachere Richtung, nämlich $WF(f)^c \subseteq \mathcal{D}$.
Wir nehmen also einen gerichteten regulären Punkt $((t_0,x_0),s_0) \in WF(f)^c$ und zeigen, dass er auch in $\mathcal{D}$ liegt. Dazu zerlegen wir $f$ zunächst wie folgt:
 Da $f$ bei $(t_0, x_0)$ in Richtung $s_0$ regulär ist, gibt es per Definition der Wellenfrontmenge ein $\phi \in C_0^\infty(\mathcal{R}^2)$ s.d. $\phi = 1$ in einer Umgebung von $(t_0, x_0)$ und für alle $N \in \mathbb{N}$ $\rwhat{\phi f} = O(1+|(\omega,k)|)^{-N}$ für $\frac{k}{\omega}$ in einer Umgebung von $s_0$. Dementsprechend ist $(1-\phi)f = 0$ in einer Umgebung von $(t_0, x_0)$ und es gilt

 \begin{equation}
     \mathcal{S}_f (a,s,(t',x')) = \left\langle \psi_{ast},\phi f \right\rangle
                                + \left\langle \psi_{ast},(1-\phi) f \right\rangle
 \label{eq:schlaue sache}
 \end{equation}

\begin{figure}[h]
\centering
%% Creator: Matplotlib, PGF backend
%%
%% To include the figure in your LaTeX document, write
%%   \input{<filename>.pgf}
%%
%% Make sure the required packages are loaded in your preamble
%%   \usepackage{pgf}
%%
%% Figures using additional raster images can only be included by \input if
%% they are in the same directory as the main LaTeX file. For loading figures
%% from other directories you can use the `import` package
%%   \usepackage{import}
%% and then include the figures with
%%   \import{<path to file>}{<filename>.pgf}
%%
%% Matplotlib used the following preamble
%%   \usepackage[utf8x]{inputenc}
%%   \usepackage[T1]{fontenc}
%%   \usepackage{amssymb}
%%
\begingroup%
\makeatletter%
\begin{pgfpicture}%
\pgfpathrectangle{\pgfpointorigin}{\pgfqpoint{4.000000in}{2.200000in}}%
\pgfusepath{use as bounding box, clip}%
\begin{pgfscope}%
\pgfsetbuttcap%
\pgfsetmiterjoin%
\definecolor{currentfill}{rgb}{1.000000,1.000000,1.000000}%
\pgfsetfillcolor{currentfill}%
\pgfsetlinewidth{0.000000pt}%
\definecolor{currentstroke}{rgb}{1.000000,1.000000,1.000000}%
\pgfsetstrokecolor{currentstroke}%
\pgfsetdash{}{0pt}%
\pgfpathmoveto{\pgfqpoint{0.000000in}{0.000000in}}%
\pgfpathlineto{\pgfqpoint{4.000000in}{0.000000in}}%
\pgfpathlineto{\pgfqpoint{4.000000in}{2.200000in}}%
\pgfpathlineto{\pgfqpoint{0.000000in}{2.200000in}}%
\pgfpathclose%
\pgfusepath{fill}%
\end{pgfscope}%
\begin{pgfscope}%
\pgfsetbuttcap%
\pgfsetmiterjoin%
\definecolor{currentfill}{rgb}{1.000000,1.000000,1.000000}%
\pgfsetfillcolor{currentfill}%
\pgfsetlinewidth{0.000000pt}%
\definecolor{currentstroke}{rgb}{0.000000,0.000000,0.000000}%
\pgfsetstrokecolor{currentstroke}%
\pgfsetstrokeopacity{0.000000}%
\pgfsetdash{}{0pt}%
\pgfpathmoveto{\pgfqpoint{0.500000in}{0.275000in}}%
\pgfpathlineto{\pgfqpoint{3.600000in}{0.275000in}}%
\pgfpathlineto{\pgfqpoint{3.600000in}{1.936000in}}%
\pgfpathlineto{\pgfqpoint{0.500000in}{1.936000in}}%
\pgfpathclose%
\pgfusepath{fill}%
\end{pgfscope}%
\begin{pgfscope}%
\pgfsetbuttcap%
\pgfsetroundjoin%
\definecolor{currentfill}{rgb}{0.000000,0.000000,0.000000}%
\pgfsetfillcolor{currentfill}%
\pgfsetlinewidth{0.803000pt}%
\definecolor{currentstroke}{rgb}{0.000000,0.000000,0.000000}%
\pgfsetstrokecolor{currentstroke}%
\pgfsetdash{}{0pt}%
\pgfsys@defobject{currentmarker}{\pgfqpoint{0.000000in}{-0.048611in}}{\pgfqpoint{0.000000in}{0.000000in}}{%
\pgfpathmoveto{\pgfqpoint{0.000000in}{0.000000in}}%
\pgfpathlineto{\pgfqpoint{0.000000in}{-0.048611in}}%
\pgfusepath{stroke,fill}%
}%
\begin{pgfscope}%
\pgfsys@transformshift{1.828571in}{0.344208in}%
\pgfsys@useobject{currentmarker}{}%
\end{pgfscope}%
\end{pgfscope}%
\begin{pgfscope}%
\pgftext[x=1.828571in,y=0.246986in,,top]{\rmfamily\fontsize{10.000000}{12.000000}\selectfont \(\displaystyle t_0\)}%
\end{pgfscope}%
\begin{pgfscope}%
\pgfsetbuttcap%
\pgfsetroundjoin%
\definecolor{currentfill}{rgb}{0.000000,0.000000,0.000000}%
\pgfsetfillcolor{currentfill}%
\pgfsetlinewidth{0.803000pt}%
\definecolor{currentstroke}{rgb}{0.000000,0.000000,0.000000}%
\pgfsetstrokecolor{currentstroke}%
\pgfsetdash{}{0pt}%
\pgfsys@defobject{currentmarker}{\pgfqpoint{0.000000in}{-0.048611in}}{\pgfqpoint{0.000000in}{0.000000in}}{%
\pgfpathmoveto{\pgfqpoint{0.000000in}{0.000000in}}%
\pgfpathlineto{\pgfqpoint{0.000000in}{-0.048611in}}%
\pgfusepath{stroke,fill}%
}%
\begin{pgfscope}%
\pgfsys@transformshift{1.939286in}{0.344208in}%
\pgfsys@useobject{currentmarker}{}%
\end{pgfscope}%
\end{pgfscope}%
\begin{pgfscope}%
\pgftext[x=1.939286in,y=0.246986in,,top]{\rmfamily\fontsize{10.000000}{12.000000}\selectfont \(\displaystyle t\)}%
\end{pgfscope}%
\begin{pgfscope}%
\pgfsetbuttcap%
\pgfsetroundjoin%
\definecolor{currentfill}{rgb}{0.000000,0.000000,0.000000}%
\pgfsetfillcolor{currentfill}%
\pgfsetlinewidth{0.803000pt}%
\definecolor{currentstroke}{rgb}{0.000000,0.000000,0.000000}%
\pgfsetstrokecolor{currentstroke}%
\pgfsetdash{}{0pt}%
\pgfsys@defobject{currentmarker}{\pgfqpoint{-0.048611in}{0.000000in}}{\pgfqpoint{0.000000in}{0.000000in}}{%
\pgfpathmoveto{\pgfqpoint{0.000000in}{0.000000in}}%
\pgfpathlineto{\pgfqpoint{-0.048611in}{0.000000in}}%
\pgfusepath{stroke,fill}%
}%
\begin{pgfscope}%
\pgfsys@transformshift{0.721429in}{1.036292in}%
\pgfsys@useobject{currentmarker}{}%
\end{pgfscope}%
\end{pgfscope}%
\begin{pgfscope}%
\pgftext[x=0.554779in,y=0.988464in,left,base]{\rmfamily\fontsize{10.000000}{12.000000}\selectfont 1}%
\end{pgfscope}%
\begin{pgfscope}%
\pgfpathrectangle{\pgfqpoint{0.500000in}{0.275000in}}{\pgfqpoint{3.100000in}{1.661000in}}%
\pgfusepath{clip}%
\pgfsetbuttcap%
\pgfsetroundjoin%
\pgfsetlinewidth{0.501875pt}%
\definecolor{currentstroke}{rgb}{0.501961,0.501961,0.501961}%
\pgfsetstrokecolor{currentstroke}%
\pgfsetdash{{1.850000pt}{0.800000pt}}{0.000000pt}%
\pgfpathmoveto{\pgfqpoint{1.939286in}{0.261111in}}%
\pgfpathlineto{\pgfqpoint{1.939286in}{1.949889in}}%
\pgfusepath{stroke}%
\end{pgfscope}%
\begin{pgfscope}%
\pgfpathrectangle{\pgfqpoint{0.500000in}{0.275000in}}{\pgfqpoint{3.100000in}{1.661000in}}%
\pgfusepath{clip}%
\pgfsetrectcap%
\pgfsetroundjoin%
\pgfsetlinewidth{0.501875pt}%
\definecolor{currentstroke}{rgb}{0.894118,0.101961,0.109804}%
\pgfsetstrokecolor{currentstroke}%
\pgfsetdash{}{0pt}%
\pgfpathmoveto{\pgfqpoint{0.486111in}{0.344208in}}%
\pgfpathlineto{\pgfqpoint{1.551509in}{0.345603in}}%
\pgfpathlineto{\pgfqpoint{1.586973in}{0.348858in}}%
\pgfpathlineto{\pgfqpoint{1.609138in}{0.353517in}}%
\pgfpathlineto{\pgfqpoint{1.626870in}{0.359911in}}%
\pgfpathlineto{\pgfqpoint{1.644602in}{0.369943in}}%
\pgfpathlineto{\pgfqpoint{1.657901in}{0.380785in}}%
\pgfpathlineto{\pgfqpoint{1.671200in}{0.395358in}}%
\pgfpathlineto{\pgfqpoint{1.684499in}{0.414585in}}%
\pgfpathlineto{\pgfqpoint{1.697798in}{0.439480in}}%
\pgfpathlineto{\pgfqpoint{1.711097in}{0.471104in}}%
\pgfpathlineto{\pgfqpoint{1.724396in}{0.510505in}}%
\pgfpathlineto{\pgfqpoint{1.737695in}{0.558631in}}%
\pgfpathlineto{\pgfqpoint{1.750994in}{0.616232in}}%
\pgfpathlineto{\pgfqpoint{1.768726in}{0.708477in}}%
\pgfpathlineto{\pgfqpoint{1.786458in}{0.818128in}}%
\pgfpathlineto{\pgfqpoint{1.808623in}{0.976524in}}%
\pgfpathlineto{\pgfqpoint{1.852953in}{1.327417in}}%
\pgfpathlineto{\pgfqpoint{1.875118in}{1.490059in}}%
\pgfpathlineto{\pgfqpoint{1.892850in}{1.597969in}}%
\pgfpathlineto{\pgfqpoint{1.906149in}{1.660357in}}%
\pgfpathlineto{\pgfqpoint{1.915015in}{1.691459in}}%
\pgfpathlineto{\pgfqpoint{1.923881in}{1.713383in}}%
\pgfpathlineto{\pgfqpoint{1.932747in}{1.725662in}}%
\pgfpathlineto{\pgfqpoint{1.937180in}{1.728093in}}%
\pgfpathlineto{\pgfqpoint{1.941613in}{1.728031in}}%
\pgfpathlineto{\pgfqpoint{1.946046in}{1.725475in}}%
\pgfpathlineto{\pgfqpoint{1.950479in}{1.720439in}}%
\pgfpathlineto{\pgfqpoint{1.959345in}{1.703051in}}%
\pgfpathlineto{\pgfqpoint{1.968211in}{1.676237in}}%
\pgfpathlineto{\pgfqpoint{1.977077in}{1.640567in}}%
\pgfpathlineto{\pgfqpoint{1.990376in}{1.572121in}}%
\pgfpathlineto{\pgfqpoint{2.008108in}{1.457967in}}%
\pgfpathlineto{\pgfqpoint{2.030273in}{1.290874in}}%
\pgfpathlineto{\pgfqpoint{2.083469in}{0.877388in}}%
\pgfpathlineto{\pgfqpoint{2.105634in}{0.732984in}}%
\pgfpathlineto{\pgfqpoint{2.123366in}{0.636529in}}%
\pgfpathlineto{\pgfqpoint{2.141098in}{0.557753in}}%
\pgfpathlineto{\pgfqpoint{2.158830in}{0.495768in}}%
\pgfpathlineto{\pgfqpoint{2.172129in}{0.459204in}}%
\pgfpathlineto{\pgfqpoint{2.185428in}{0.430056in}}%
\pgfpathlineto{\pgfqpoint{2.198727in}{0.407264in}}%
\pgfpathlineto{\pgfqpoint{2.212026in}{0.389778in}}%
\pgfpathlineto{\pgfqpoint{2.225325in}{0.376610in}}%
\pgfpathlineto{\pgfqpoint{2.238624in}{0.366877in}}%
\pgfpathlineto{\pgfqpoint{2.256356in}{0.357936in}}%
\pgfpathlineto{\pgfqpoint{2.274088in}{0.352285in}}%
\pgfpathlineto{\pgfqpoint{2.296253in}{0.348204in}}%
\pgfpathlineto{\pgfqpoint{2.331717in}{0.345389in}}%
\pgfpathlineto{\pgfqpoint{2.398212in}{0.344296in}}%
\pgfpathlineto{\pgfqpoint{2.876977in}{0.344208in}}%
\pgfpathlineto{\pgfqpoint{3.613889in}{0.344208in}}%
\pgfpathlineto{\pgfqpoint{3.613889in}{0.344208in}}%
\pgfusepath{stroke}%
\end{pgfscope}%
\begin{pgfscope}%
\pgfpathrectangle{\pgfqpoint{0.500000in}{0.275000in}}{\pgfqpoint{3.100000in}{1.661000in}}%
\pgfusepath{clip}%
\pgfsetrectcap%
\pgfsetroundjoin%
\pgfsetlinewidth{0.501875pt}%
\definecolor{currentstroke}{rgb}{0.215686,0.494118,0.721569}%
\pgfsetstrokecolor{currentstroke}%
\pgfsetdash{}{0pt}%
\pgfpathmoveto{\pgfqpoint{0.486111in}{0.344208in}}%
\pgfpathlineto{\pgfqpoint{1.409653in}{0.345316in}}%
\pgfpathlineto{\pgfqpoint{1.418519in}{0.348387in}}%
\pgfpathlineto{\pgfqpoint{1.427385in}{0.355340in}}%
\pgfpathlineto{\pgfqpoint{1.436251in}{0.368220in}}%
\pgfpathlineto{\pgfqpoint{1.445117in}{0.389050in}}%
\pgfpathlineto{\pgfqpoint{1.453983in}{0.419471in}}%
\pgfpathlineto{\pgfqpoint{1.462849in}{0.460375in}}%
\pgfpathlineto{\pgfqpoint{1.476148in}{0.540690in}}%
\pgfpathlineto{\pgfqpoint{1.493880in}{0.672835in}}%
\pgfpathlineto{\pgfqpoint{1.516045in}{0.839810in}}%
\pgfpathlineto{\pgfqpoint{1.529344in}{0.920125in}}%
\pgfpathlineto{\pgfqpoint{1.538210in}{0.961029in}}%
\pgfpathlineto{\pgfqpoint{1.547076in}{0.991450in}}%
\pgfpathlineto{\pgfqpoint{1.555942in}{1.012280in}}%
\pgfpathlineto{\pgfqpoint{1.564808in}{1.025160in}}%
\pgfpathlineto{\pgfqpoint{1.573674in}{1.032113in}}%
\pgfpathlineto{\pgfqpoint{1.582540in}{1.035184in}}%
\pgfpathlineto{\pgfqpoint{1.600272in}{1.036290in}}%
\pgfpathlineto{\pgfqpoint{2.074603in}{1.035184in}}%
\pgfpathlineto{\pgfqpoint{2.083469in}{1.032113in}}%
\pgfpathlineto{\pgfqpoint{2.092335in}{1.025160in}}%
\pgfpathlineto{\pgfqpoint{2.101201in}{1.012280in}}%
\pgfpathlineto{\pgfqpoint{2.110067in}{0.991450in}}%
\pgfpathlineto{\pgfqpoint{2.118933in}{0.961029in}}%
\pgfpathlineto{\pgfqpoint{2.127799in}{0.920125in}}%
\pgfpathlineto{\pgfqpoint{2.141098in}{0.839810in}}%
\pgfpathlineto{\pgfqpoint{2.158830in}{0.707665in}}%
\pgfpathlineto{\pgfqpoint{2.180995in}{0.540690in}}%
\pgfpathlineto{\pgfqpoint{2.194294in}{0.460375in}}%
\pgfpathlineto{\pgfqpoint{2.203160in}{0.419471in}}%
\pgfpathlineto{\pgfqpoint{2.212026in}{0.389050in}}%
\pgfpathlineto{\pgfqpoint{2.220892in}{0.368220in}}%
\pgfpathlineto{\pgfqpoint{2.229758in}{0.355340in}}%
\pgfpathlineto{\pgfqpoint{2.238624in}{0.348387in}}%
\pgfpathlineto{\pgfqpoint{2.247490in}{0.345316in}}%
\pgfpathlineto{\pgfqpoint{2.265222in}{0.344210in}}%
\pgfpathlineto{\pgfqpoint{3.613889in}{0.344208in}}%
\pgfpathlineto{\pgfqpoint{3.613889in}{0.344208in}}%
\pgfusepath{stroke}%
\end{pgfscope}%
\begin{pgfscope}%
\pgfpathrectangle{\pgfqpoint{0.500000in}{0.275000in}}{\pgfqpoint{3.100000in}{1.661000in}}%
\pgfusepath{clip}%
\pgfsetrectcap%
\pgfsetroundjoin%
\pgfsetlinewidth{0.501875pt}%
\definecolor{currentstroke}{rgb}{0.301961,0.686275,0.290196}%
\pgfsetstrokecolor{currentstroke}%
\pgfsetdash{}{0pt}%
\pgfpathmoveto{\pgfqpoint{0.486111in}{1.036292in}}%
\pgfpathlineto{\pgfqpoint{1.409653in}{1.035184in}}%
\pgfpathlineto{\pgfqpoint{1.418519in}{1.032113in}}%
\pgfpathlineto{\pgfqpoint{1.427385in}{1.025160in}}%
\pgfpathlineto{\pgfqpoint{1.436251in}{1.012280in}}%
\pgfpathlineto{\pgfqpoint{1.445117in}{0.991450in}}%
\pgfpathlineto{\pgfqpoint{1.453983in}{0.961029in}}%
\pgfpathlineto{\pgfqpoint{1.462849in}{0.920125in}}%
\pgfpathlineto{\pgfqpoint{1.476148in}{0.839810in}}%
\pgfpathlineto{\pgfqpoint{1.493880in}{0.707665in}}%
\pgfpathlineto{\pgfqpoint{1.516045in}{0.540690in}}%
\pgfpathlineto{\pgfqpoint{1.529344in}{0.460375in}}%
\pgfpathlineto{\pgfqpoint{1.538210in}{0.419471in}}%
\pgfpathlineto{\pgfqpoint{1.547076in}{0.389050in}}%
\pgfpathlineto{\pgfqpoint{1.555942in}{0.368220in}}%
\pgfpathlineto{\pgfqpoint{1.564808in}{0.355340in}}%
\pgfpathlineto{\pgfqpoint{1.573674in}{0.348387in}}%
\pgfpathlineto{\pgfqpoint{1.582540in}{0.345316in}}%
\pgfpathlineto{\pgfqpoint{1.600272in}{0.344210in}}%
\pgfpathlineto{\pgfqpoint{2.074603in}{0.345316in}}%
\pgfpathlineto{\pgfqpoint{2.083469in}{0.348387in}}%
\pgfpathlineto{\pgfqpoint{2.092335in}{0.355340in}}%
\pgfpathlineto{\pgfqpoint{2.101201in}{0.368220in}}%
\pgfpathlineto{\pgfqpoint{2.110067in}{0.389050in}}%
\pgfpathlineto{\pgfqpoint{2.118933in}{0.419471in}}%
\pgfpathlineto{\pgfqpoint{2.127799in}{0.460375in}}%
\pgfpathlineto{\pgfqpoint{2.141098in}{0.540690in}}%
\pgfpathlineto{\pgfqpoint{2.158830in}{0.672835in}}%
\pgfpathlineto{\pgfqpoint{2.180995in}{0.839810in}}%
\pgfpathlineto{\pgfqpoint{2.194294in}{0.920125in}}%
\pgfpathlineto{\pgfqpoint{2.203160in}{0.961029in}}%
\pgfpathlineto{\pgfqpoint{2.212026in}{0.991450in}}%
\pgfpathlineto{\pgfqpoint{2.220892in}{1.012280in}}%
\pgfpathlineto{\pgfqpoint{2.229758in}{1.025160in}}%
\pgfpathlineto{\pgfqpoint{2.238624in}{1.032113in}}%
\pgfpathlineto{\pgfqpoint{2.247490in}{1.035184in}}%
\pgfpathlineto{\pgfqpoint{2.265222in}{1.036290in}}%
\pgfpathlineto{\pgfqpoint{3.613889in}{1.036292in}}%
\pgfpathlineto{\pgfqpoint{3.613889in}{1.036292in}}%
\pgfusepath{stroke}%
\end{pgfscope}%
\begin{pgfscope}%
\pgfsetrectcap%
\pgfsetmiterjoin%
\pgfsetlinewidth{0.501875pt}%
\definecolor{currentstroke}{rgb}{0.000000,0.000000,0.000000}%
\pgfsetstrokecolor{currentstroke}%
\pgfsetdash{}{0pt}%
\pgfpathmoveto{\pgfqpoint{0.721429in}{0.275000in}}%
\pgfpathlineto{\pgfqpoint{0.721429in}{1.936000in}}%
\pgfusepath{stroke}%
\end{pgfscope}%
\begin{pgfscope}%
\pgfsetrectcap%
\pgfsetmiterjoin%
\pgfsetlinewidth{0.501875pt}%
\definecolor{currentstroke}{rgb}{0.000000,0.000000,0.000000}%
\pgfsetstrokecolor{currentstroke}%
\pgfsetdash{}{0pt}%
\pgfpathmoveto{\pgfqpoint{0.500000in}{0.344208in}}%
\pgfpathlineto{\pgfqpoint{3.600000in}{0.344208in}}%
\pgfusepath{stroke}%
\end{pgfscope}%
\begin{pgfscope}%
\pgfsetroundcap%
\pgfsetroundjoin%
\pgfsetlinewidth{0.501875pt}%
\definecolor{currentstroke}{rgb}{0.000000,0.000000,0.000000}%
\pgfsetstrokecolor{currentstroke}%
\pgfsetdash{}{0pt}%
\pgfpathmoveto{\pgfqpoint{0.721429in}{1.942121in}}%
\pgfpathquadraticcurveto{\pgfqpoint{0.721429in}{1.942943in}}{\pgfqpoint{0.721429in}{1.936000in}}%
\pgfusepath{stroke}%
\end{pgfscope}%
\begin{pgfscope}%
\pgfsetroundcap%
\pgfsetroundjoin%
\pgfsetlinewidth{0.501875pt}%
\definecolor{currentstroke}{rgb}{0.000000,0.000000,0.000000}%
\pgfsetstrokecolor{currentstroke}%
\pgfsetdash{}{0pt}%
\pgfpathmoveto{\pgfqpoint{0.693651in}{1.886565in}}%
\pgfpathlineto{\pgfqpoint{0.721429in}{1.942121in}}%
\pgfpathlineto{\pgfqpoint{0.749206in}{1.886565in}}%
\pgfusepath{stroke}%
\end{pgfscope}%
\begin{pgfscope}%
\pgftext[x=0.721429in,y=2.005444in,,bottom]{\rmfamily\fontsize{10.000000}{12.000000}\selectfont \(\displaystyle  f\)}%
\end{pgfscope}%
\begin{pgfscope}%
\pgfsetroundcap%
\pgfsetroundjoin%
\pgfsetlinewidth{0.501875pt}%
\definecolor{currentstroke}{rgb}{0.000000,0.000000,0.000000}%
\pgfsetstrokecolor{currentstroke}%
\pgfsetdash{}{0pt}%
\pgfpathmoveto{\pgfqpoint{3.606114in}{0.344208in}}%
\pgfpathquadraticcurveto{\pgfqpoint{3.606939in}{0.344208in}}{\pgfqpoint{3.600000in}{0.344208in}}%
\pgfusepath{stroke}%
\end{pgfscope}%
\begin{pgfscope}%
\pgfsetroundcap%
\pgfsetroundjoin%
\pgfsetlinewidth{0.501875pt}%
\definecolor{currentstroke}{rgb}{0.000000,0.000000,0.000000}%
\pgfsetstrokecolor{currentstroke}%
\pgfsetdash{}{0pt}%
\pgfpathmoveto{\pgfqpoint{3.550559in}{0.371986in}}%
\pgfpathlineto{\pgfqpoint{3.606114in}{0.344208in}}%
\pgfpathlineto{\pgfqpoint{3.550559in}{0.316431in}}%
\pgfusepath{stroke}%
\end{pgfscope}%
\begin{pgfscope}%
\pgftext[x=3.669444in,y=0.344208in,left,]{\rmfamily\fontsize{10.000000}{12.000000}\selectfont \(\displaystyle t\)}%
\end{pgfscope}%
\begin{pgfscope}%
\pgfsetbuttcap%
\pgfsetmiterjoin%
\definecolor{currentfill}{rgb}{1.000000,1.000000,1.000000}%
\pgfsetfillcolor{currentfill}%
\pgfsetfillopacity{0.800000}%
\pgfsetlinewidth{0.501875pt}%
\definecolor{currentstroke}{rgb}{0.800000,0.800000,0.800000}%
\pgfsetstrokecolor{currentstroke}%
\pgfsetstrokeopacity{0.800000}%
\pgfsetdash{}{0pt}%
\pgfpathmoveto{\pgfqpoint{2.736382in}{1.243871in}}%
\pgfpathlineto{\pgfqpoint{3.502778in}{1.243871in}}%
\pgfpathquadraticcurveto{\pgfqpoint{3.530556in}{1.243871in}}{\pgfqpoint{3.530556in}{1.271648in}}%
\pgfpathlineto{\pgfqpoint{3.530556in}{1.838778in}}%
\pgfpathquadraticcurveto{\pgfqpoint{3.530556in}{1.866556in}}{\pgfqpoint{3.502778in}{1.866556in}}%
\pgfpathlineto{\pgfqpoint{2.736382in}{1.866556in}}%
\pgfpathquadraticcurveto{\pgfqpoint{2.708604in}{1.866556in}}{\pgfqpoint{2.708604in}{1.838778in}}%
\pgfpathlineto{\pgfqpoint{2.708604in}{1.271648in}}%
\pgfpathquadraticcurveto{\pgfqpoint{2.708604in}{1.243871in}}{\pgfqpoint{2.736382in}{1.243871in}}%
\pgfpathclose%
\pgfusepath{stroke,fill}%
\end{pgfscope}%
\begin{pgfscope}%
\pgfsetrectcap%
\pgfsetroundjoin%
\pgfsetlinewidth{0.501875pt}%
\definecolor{currentstroke}{rgb}{0.894118,0.101961,0.109804}%
\pgfsetstrokecolor{currentstroke}%
\pgfsetdash{}{0pt}%
\pgfpathmoveto{\pgfqpoint{2.764160in}{1.762389in}}%
\pgfpathlineto{\pgfqpoint{3.041937in}{1.762389in}}%
\pgfusepath{stroke}%
\end{pgfscope}%
\begin{pgfscope}%
\pgftext[x=3.153049in,y=1.713778in,left,base]{\rmfamily\fontsize{10.000000}{12.000000}\selectfont \(\displaystyle \psi_{ast}\)}%
\end{pgfscope}%
\begin{pgfscope}%
\pgfsetrectcap%
\pgfsetroundjoin%
\pgfsetlinewidth{0.501875pt}%
\definecolor{currentstroke}{rgb}{0.215686,0.494118,0.721569}%
\pgfsetstrokecolor{currentstroke}%
\pgfsetdash{}{0pt}%
\pgfpathmoveto{\pgfqpoint{2.764160in}{1.568716in}}%
\pgfpathlineto{\pgfqpoint{3.041937in}{1.568716in}}%
\pgfusepath{stroke}%
\end{pgfscope}%
\begin{pgfscope}%
\pgftext[x=3.153049in,y=1.520105in,left,base]{\rmfamily\fontsize{10.000000}{12.000000}\selectfont \(\displaystyle \phi\)}%
\end{pgfscope}%
\begin{pgfscope}%
\pgfsetrectcap%
\pgfsetroundjoin%
\pgfsetlinewidth{0.501875pt}%
\definecolor{currentstroke}{rgb}{0.301961,0.686275,0.290196}%
\pgfsetstrokecolor{currentstroke}%
\pgfsetdash{}{0pt}%
\pgfpathmoveto{\pgfqpoint{2.764160in}{1.375043in}}%
\pgfpathlineto{\pgfqpoint{3.041937in}{1.375043in}}%
\pgfusepath{stroke}%
\end{pgfscope}%
\begin{pgfscope}%
\pgftext[x=3.153049in,y=1.326432in,left,base]{\rmfamily\fontsize{10.000000}{12.000000}\selectfont \(\displaystyle 1-\phi\)}%
\end{pgfscope}%
\end{pgfpicture}%
\makeatother%
\endgroup%

\caption{Die Zerlegung von $f$ um $(t_0,x_0)$ herum visualisiert}
\label{fig:smart_decomposition}
\end{figure}

Da $(1-\phi)f$ in einer Umgebung von $(t_0, x_0)$ verschwindet und nach \cref{prop:shearlets_decay_rapidly} Shearlets außerhalb von $(t',x')$ schnell abfallen für $a \to 0$ fällt auch der zweite Term von \cref{eq:schlaue sache}
für $(t',x') = (t_0,x_0)$ schnell ab. Für den ersten Term überzeugen wir uns anhand von \cref{fig:supp_psi_hat,eq:supp_psi}, dass für $a$ klein genug $supp(\hat\psi_{ast})$ schließlich in jedem noch so kleinen Kegel um $s$ liegt. In einem solchen um $s_0$ fällt aber $\rwhat{\phi f}$ rapide ab nach Vorraussetzung und damit auch der erste Term in \cref{eq:schlaue sache}.

Die beiden entscheidenden Zutaten waren hier also die Tatsache, dass die Shearlets außerhalb von $(t',x')$ rapide Abfallen und damit bei immer feineren Skalen $a$ immer besser lokalisiert werden sowie die Tatsache, dass für $a \to 0$ der Träger im Frequenzbereich in immer engeren Kegeln liegt.

Deutlich schwieriger ist die umgekehrte Inklusion, nämlich dass die Shearlettransformation tatsächlich die ganze Wellenfrontmenge erkennt. Hier geht jetzt auch die Reproduktionseigenschaft der Transformation ein, eben genau dass sie alles sieht.

Für die umgekehrte Inklusion $\mathcal{D} \subseteq WF(f)^c$ haben wir zu zeigen, dass falls $\mathcal{S}_f (a,s,(t',x'))$ schnell abfällt für $a \to 0$ einer Umgebung $U$ von $(s_0, (t_0, x_0))$ dann auch $\rwhat{\phi f} (\omega,k)$ schnell abfällt für $\Vert(\omega,k)\Vert \to \infty$ für $\frac{k}{\omega}$ in einer Umgebung von $s_0$ und ein $\phi$ getragen in einer Umgebung von $(t_0, x_0)$.

Sei also $\phi \in C^\infty(\pi(U))$, wobei $\pi$ wieder die Projektion auf die Ortskomponente ist. Dann ist

\begin{dmath*}
    \rwhat{\phi f} (\omega, k)
    =
    \int \phi f e^{-i\omega t+ikx} \d t \d x \\
    \stackrel{\ref{thm:shearlets_reproduzieren}}{=}
    \iint \left\langle \psi_{as(t',x')},\phi f \right\rangle
        \psi_{as(t',x')} (t,x) \d \mu (as(t',x'))
        e^{\cdots} \d t \d x
    =
    \int \left\langle \psi_{as(t',x')},\phi f \right\rangle
    \hat\psi_{as(t',x')}(\omega,k) \d \mu(\cdots)
    = \kern -1em
    \underbrace{
        \int \limits_{U \times [0,1]} \kern -1em
        \left\langle \psi_{as(t',x')},\phi f \right\rangle
        \hat\psi_{as(t',x')}(\omega,k) \d \mu(\cdots)
    }_{i)}
    + \kern -1em
    \underbrace{
        \int \limits_{U^c \times [0,1]} \kern -1em
        \left\langle \psi_{as(t',x')},\phi f \right\rangle
        \hat\psi_{as(t',x')}(\omega,k) \d \mu(\cdots)
    }_{ii)}
\end{dmath*}

\emph{zu $ii)$}
Für $\Vert (\omega,k)\Vert \to \infty$ ist $\hat \psi$ nur für $a \to 0$

\end{proof}



% section beweis_von_thm:main_theorem (end)
