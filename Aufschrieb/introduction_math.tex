%!TEX root = main.tex
\chapter{Einleitung für Mathematiker} % (fold)
\label{sec:einleitung_mathematik}

Ursprünglich in der Bildbearbeitung und -kompression wurde erkannt und genutzt, dass die (stetige) Wavelettransformationen einer Funktion $f$ schnell abfällt an Punkten, an denen $f$ glatt ist und langsam an den Singularitäten. Bekanntestes Beispiel dafür ist die JPEG-Kompression, welche auf der Wavelettransformation basiert.

Allerdings ist die klassische Wavelettransformation mit gleichmäßiger Skalierung in alle Richtungen nicht in der Lage, die Orientierung der Singularitäten zu erkennen. Deshalb wurden verschiedene Verallgemeinerungen von Wavelets mit anisotroper Skalierung entwickelt (\cite{Guo2006} \cite{Kutyniok2008} \cite{Candes2005}), die in der Lage sind auch die Orientierung der Singularitäten zu erkennen. Im Fourierraum bedeutet anisotrope Skalierung, dass der Träger der Wavelets mit feiner werdendem Skalenparameter in immer engeren Kegeln liegt. Im Realraum entspricht dies immer flacher werdenden `"Wellenpaketen"' als Testfunktionen. Bei isotroper Skalierung hingegen wird der Träger im Realraum in alle Richtungen gleichmäßig kleiner.

Dies ist eng verwandt mit dem Konzept der \emph{Wellenfrontmenge} aus der \emph{mikrolokalen Analysis}. Die Wellenfrontmenge misst, vereinfacht gesagt, die Lage und Orientierung der Singularitäten von nicht nur Funktionen, sondern auch Distributionen. Das Versprechen in \cite{Kutyniok2008} ist, dass es mit anisotrop skalierenden Wavelettransformationen möglich ist Wellenfrontmengen zu berechnen.

Ziel der vorliegenden Arbeit ist es, genauer zu untersuchen inwiefern die Shearletttransformation von \textcite{Kutyniok2008} geeignet ist, um Wellenfrontmengen zu bestimmen. Dazu werden die Wellenfrontmengen von physikalisch motivierten Distributionen berechnet. Außerdem füllen wir eine kleine Lücke in \cite{Kutyniok2008}, geben einen Ansatz, wie die Ergebnisse von \textcite{Kutyniok2008} auf temperierte Distributionen ausgedehnt werden können und erklären, warum sie nicht auf alle Distributionen ausgedehnt werden können.
Des weiteren wird noch eine kurze Diskussion gegeben, welche weiteren Größen der mikrolokalen Analysis mithilfe von Shearlets berechnet werden können und welche Möglichkeiten es für höherdimensionale Shearlets gibt.


% section einleitung (end)
