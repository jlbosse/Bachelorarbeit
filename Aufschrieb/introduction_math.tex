%!TEX root = main.tex
\section{Einleitung für Mathematiker} % (fold)
\label{sec:einleitung_mathematik}

Ursprünglich in der Bildbearbeitung und Kompression wurde erkannt und genutzt, dass (stetige) Wavelettransformationen einer Funktion $f$ schnell abfallen an Punkten, an denen $f$ glatt ist und langsam an den Singularitäten. Bekanntestes Beispiel dafür ist die JPEG-Kompression, welche auf der Wavelettransformation basiert.

Allerdings ist die klassische Wavelettransformation mit gleichmäßiger Skalierung in alle Richtungen nicht in der Lage die Orientierung der Singularitäten zu erkennen. Es gibt aber verschiedene Verallgemeinerungen von Wavelets mit anisotroper Skalierung (\cite{Guo2006} \cite{Kutyniok2008} \cite{Candes2005}), die in der Lage sind auch die Orientierung der Singularitäten zu erkennen.
In der \emph{mikrolokalen Analysis} misst die \emph{Wellenfrontmenge}, Lage und Orientierung der Singularitäten nicht nur von Funktionen, sondern auch von Distributionen. Das Versprechen in \cite{Kutyniok2008} ist, dass mit der Shearlettransformation möglich ist Wellenfrontmengen zu berechnen.

Ziel der vorliegenden Arbeit ist es, sich genauer anzuschauen inwiefern die Shearletttransformation von \textcite{Kutyniok2008} geeignet ist, um Wellenfrontmengen zu bestimmen. Dazu werden die Wellenfrontmengen von physikalisch motivierten Distributionen berechnet. Außerdem füllen wir eine kleine Lücke in \cite{Kutyniok2008}, geben einen Ansatz, wie die Ergebnisse von \textcite{Kutyniok2008} auf temperierte Distributionen ausgedehnt werden können und erklären, warum sie nicht auf alle Distributionen ausgedehnt werden können.
Des weiteren wird noch eine kurze Diskussion gegeben, welche weiteren Größen der mikrolokalen Analysis mithilfe von Shearlets berechnet werden können und welche Möglichkeiten es für höherdimensionale Shearlets gibt.


% section einleitung (end)
