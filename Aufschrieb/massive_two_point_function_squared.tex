%!TEX root = main.tex
%%%%%%%%%%%%%%%%%%%%%%%%%%%%%%%%%%%%%%%%%%%%%%%%%%%%%%%%%%%%%%%%%%%%%%%%%%%%%%%
% % Berechnen der Wellenfrontmenge von Delta_m
%%%%%%%%%%%%%%%%%%%%%%%%%%%%%%%%%%%%%%%%%%%%%%%%%%%%%%%%%%%%%%%%%%%%%%%%%%%%%%%

\section{\texorpdfstring{Die Wellenfrontmenge von $\Delta_m^2$}
    {Wellenfrontmenge von delta m}} % (fold)
\label{sec:die_wellenfrontmenge_von_delta_m_2_}

Bevor wir die Wellenfrontmenge von $\Delta_m^2$ berechnen können, benötigen wir einen Ausdruck dafür, oder besser noch einen für die Fouriertransformierte davon.

\subsection{\texorpdfstring{$\hat\Delta^{\ast 2}$ berechnen}
    {delta-ast-hat berechnen}}
 Gemäß dem Faltungssatz gilt $\rwhat{\Delta_m^2} = \rwhat \Delta_m * \rwhat \Delta_m = \rwhat\Delta_m^{*2}$. Wir müssen also die Faltung von $\rwhat \Delta_m$ mit sich selber ausrechnen.

\begin{figure}
    \centering
    \begin{minipage}{0.5\textwidth}
        \centering
        \resizebox{\textwidth}{!}{%% Creator: Matplotlib, PGF backend
%%
%% To include the figure in your LaTeX document, write
%%   \input{<filename>.pgf}
%%
%% Make sure the required packages are loaded in your preamble
%%   \usepackage{pgf}
%%
%% Figures using additional raster images can only be included by \input if
%% they are in the same directory as the main LaTeX file. For loading figures
%% from other directories you can use the `import` package
%%   \usepackage{import}
%% and then include the figures with
%%   \import{<path to file>}{<filename>.pgf}
%%
%% Matplotlib used the following preamble
%%   \usepackage[utf8x]{inputenc}
%%   \usepackage[T1]{fontenc}
%%   \usepackage{amssymb}
%%
\begingroup%
\makeatletter%
\begin{pgfpicture}%
\pgfpathrectangle{\pgfpointorigin}{\pgfqpoint{4.000000in}{3.000000in}}%
\pgfusepath{use as bounding box, clip}%
\begin{pgfscope}%
\pgfsetbuttcap%
\pgfsetmiterjoin%
\definecolor{currentfill}{rgb}{1.000000,1.000000,1.000000}%
\pgfsetfillcolor{currentfill}%
\pgfsetlinewidth{0.000000pt}%
\definecolor{currentstroke}{rgb}{1.000000,1.000000,1.000000}%
\pgfsetstrokecolor{currentstroke}%
\pgfsetdash{}{0pt}%
\pgfpathmoveto{\pgfqpoint{0.000000in}{0.000000in}}%
\pgfpathlineto{\pgfqpoint{4.000000in}{0.000000in}}%
\pgfpathlineto{\pgfqpoint{4.000000in}{3.000000in}}%
\pgfpathlineto{\pgfqpoint{0.000000in}{3.000000in}}%
\pgfpathclose%
\pgfusepath{fill}%
\end{pgfscope}%
\begin{pgfscope}%
\pgfsetbuttcap%
\pgfsetmiterjoin%
\definecolor{currentfill}{rgb}{1.000000,1.000000,1.000000}%
\pgfsetfillcolor{currentfill}%
\pgfsetlinewidth{0.000000pt}%
\definecolor{currentstroke}{rgb}{0.000000,0.000000,0.000000}%
\pgfsetstrokecolor{currentstroke}%
\pgfsetstrokeopacity{0.000000}%
\pgfsetdash{}{0pt}%
\pgfpathmoveto{\pgfqpoint{0.198611in}{0.198611in}}%
\pgfpathlineto{\pgfqpoint{3.801389in}{0.198611in}}%
\pgfpathlineto{\pgfqpoint{3.801389in}{2.801389in}}%
\pgfpathlineto{\pgfqpoint{0.198611in}{2.801389in}}%
\pgfpathclose%
\pgfusepath{fill}%
\end{pgfscope}%
\begin{pgfscope}%
\pgfsetbuttcap%
\pgfsetroundjoin%
\definecolor{currentfill}{rgb}{0.000000,0.000000,0.000000}%
\pgfsetfillcolor{currentfill}%
\pgfsetlinewidth{0.803000pt}%
\definecolor{currentstroke}{rgb}{0.000000,0.000000,0.000000}%
\pgfsetstrokecolor{currentstroke}%
\pgfsetdash{}{0pt}%
\pgfsys@defobject{currentmarker}{\pgfqpoint{0.000000in}{-0.048611in}}{\pgfqpoint{0.000000in}{0.000000in}}{%
\pgfpathmoveto{\pgfqpoint{0.000000in}{0.000000in}}%
\pgfpathlineto{\pgfqpoint{0.000000in}{-0.048611in}}%
\pgfusepath{stroke,fill}%
}%
\begin{pgfscope}%
\pgfsys@transformshift{1.219976in}{0.632407in}%
\pgfsys@useobject{currentmarker}{}%
\end{pgfscope}%
\end{pgfscope}%
\begin{pgfscope}%
\pgftext[x=1.219976in,y=0.535185in,,top]{\rmfamily\fontsize{10.000000}{12.000000}\selectfont \(\displaystyle k^\prime_{0-} = -\sqrt{\left(\frac{\omega}{2}\right)^2-m^2}\)}%
\end{pgfscope}%
\begin{pgfscope}%
\pgfsetbuttcap%
\pgfsetroundjoin%
\definecolor{currentfill}{rgb}{0.000000,0.000000,0.000000}%
\pgfsetfillcolor{currentfill}%
\pgfsetlinewidth{0.803000pt}%
\definecolor{currentstroke}{rgb}{0.000000,0.000000,0.000000}%
\pgfsetstrokecolor{currentstroke}%
\pgfsetdash{}{0pt}%
\pgfsys@defobject{currentmarker}{\pgfqpoint{0.000000in}{-0.048611in}}{\pgfqpoint{0.000000in}{0.000000in}}{%
\pgfpathmoveto{\pgfqpoint{0.000000in}{0.000000in}}%
\pgfpathlineto{\pgfqpoint{0.000000in}{-0.048611in}}%
\pgfusepath{stroke,fill}%
}%
\begin{pgfscope}%
\pgfsys@transformshift{2.780024in}{0.632407in}%
\pgfsys@useobject{currentmarker}{}%
\end{pgfscope}%
\end{pgfscope}%
\begin{pgfscope}%
\pgftext[x=2.780024in,y=0.535185in,,top]{\rmfamily\fontsize{10.000000}{12.000000}\selectfont \(\displaystyle k^\prime_{0+} = \sqrt{\left(\frac{\omega}{2}\right)^2-m^2}\)}%
\end{pgfscope}%
\begin{pgfscope}%
\pgfsetbuttcap%
\pgfsetroundjoin%
\definecolor{currentfill}{rgb}{0.000000,0.000000,0.000000}%
\pgfsetfillcolor{currentfill}%
\pgfsetlinewidth{0.803000pt}%
\definecolor{currentstroke}{rgb}{0.000000,0.000000,0.000000}%
\pgfsetstrokecolor{currentstroke}%
\pgfsetdash{}{0pt}%
\pgfsys@defobject{currentmarker}{\pgfqpoint{-0.048611in}{0.000000in}}{\pgfqpoint{0.000000in}{0.000000in}}{%
\pgfpathmoveto{\pgfqpoint{0.000000in}{0.000000in}}%
\pgfpathlineto{\pgfqpoint{-0.048611in}{0.000000in}}%
\pgfusepath{stroke,fill}%
}%
\begin{pgfscope}%
\pgfsys@transformshift{2.000000in}{1.066204in}%
\pgfsys@useobject{currentmarker}{}%
\end{pgfscope}%
\end{pgfscope}%
\begin{pgfscope}%
\pgftext[x=1.780831in,y=1.018376in,left,base]{\rmfamily\fontsize{10.000000}{12.000000}\selectfont \(\displaystyle m\)}%
\end{pgfscope}%
\begin{pgfscope}%
\pgfsetbuttcap%
\pgfsetroundjoin%
\definecolor{currentfill}{rgb}{0.000000,0.000000,0.000000}%
\pgfsetfillcolor{currentfill}%
\pgfsetlinewidth{0.803000pt}%
\definecolor{currentstroke}{rgb}{0.000000,0.000000,0.000000}%
\pgfsetstrokecolor{currentstroke}%
\pgfsetdash{}{0pt}%
\pgfsys@defobject{currentmarker}{\pgfqpoint{-0.048611in}{0.000000in}}{\pgfqpoint{0.000000in}{0.000000in}}{%
\pgfpathmoveto{\pgfqpoint{0.000000in}{0.000000in}}%
\pgfpathlineto{\pgfqpoint{-0.048611in}{0.000000in}}%
\pgfusepath{stroke,fill}%
}%
\begin{pgfscope}%
\pgfsys@transformshift{2.000000in}{1.500000in}%
\pgfsys@useobject{currentmarker}{}%
\end{pgfscope}%
\end{pgfscope}%
\begin{pgfscope}%
\pgftext[x=1.338511in,y=1.444321in,left,base]{\rmfamily\fontsize{10.000000}{12.000000}\selectfont \(\displaystyle \omega/2 = \omega^\prime_0\)}%
\end{pgfscope}%
\begin{pgfscope}%
\pgfsetbuttcap%
\pgfsetroundjoin%
\definecolor{currentfill}{rgb}{0.000000,0.000000,0.000000}%
\pgfsetfillcolor{currentfill}%
\pgfsetlinewidth{0.803000pt}%
\definecolor{currentstroke}{rgb}{0.000000,0.000000,0.000000}%
\pgfsetstrokecolor{currentstroke}%
\pgfsetdash{}{0pt}%
\pgfsys@defobject{currentmarker}{\pgfqpoint{-0.048611in}{0.000000in}}{\pgfqpoint{0.000000in}{0.000000in}}{%
\pgfpathmoveto{\pgfqpoint{0.000000in}{0.000000in}}%
\pgfpathlineto{\pgfqpoint{-0.048611in}{0.000000in}}%
\pgfusepath{stroke,fill}%
}%
\begin{pgfscope}%
\pgfsys@transformshift{2.000000in}{1.933796in}%
\pgfsys@useobject{currentmarker}{}%
\end{pgfscope}%
\end{pgfscope}%
\begin{pgfscope}%
\pgftext[x=1.519645in,y=1.885969in,left,base]{\rmfamily\fontsize{10.000000}{12.000000}\selectfont \(\displaystyle \omega - m\)}%
\end{pgfscope}%
\begin{pgfscope}%
\pgfsetbuttcap%
\pgfsetroundjoin%
\definecolor{currentfill}{rgb}{0.000000,0.000000,0.000000}%
\pgfsetfillcolor{currentfill}%
\pgfsetlinewidth{0.803000pt}%
\definecolor{currentstroke}{rgb}{0.000000,0.000000,0.000000}%
\pgfsetstrokecolor{currentstroke}%
\pgfsetdash{}{0pt}%
\pgfsys@defobject{currentmarker}{\pgfqpoint{-0.048611in}{0.000000in}}{\pgfqpoint{0.000000in}{0.000000in}}{%
\pgfpathmoveto{\pgfqpoint{0.000000in}{0.000000in}}%
\pgfpathlineto{\pgfqpoint{-0.048611in}{0.000000in}}%
\pgfusepath{stroke,fill}%
}%
\begin{pgfscope}%
\pgfsys@transformshift{2.000000in}{2.367593in}%
\pgfsys@useobject{currentmarker}{}%
\end{pgfscope}%
\end{pgfscope}%
\begin{pgfscope}%
\pgftext[x=1.811343in,y=2.319765in,left,base]{\rmfamily\fontsize{10.000000}{12.000000}\selectfont \(\displaystyle \omega\)}%
\end{pgfscope}%
\begin{pgfscope}%
\pgfpathrectangle{\pgfqpoint{0.198611in}{0.198611in}}{\pgfqpoint{3.602778in}{2.602778in}}%
\pgfusepath{clip}%
\pgfsetbuttcap%
\pgfsetroundjoin%
\pgfsetlinewidth{0.501875pt}%
\definecolor{currentstroke}{rgb}{0.501961,0.501961,0.501961}%
\pgfsetstrokecolor{currentstroke}%
\pgfsetdash{{1.850000pt}{0.800000pt}}{0.000000pt}%
\pgfpathmoveto{\pgfqpoint{1.219976in}{0.632407in}}%
\pgfpathlineto{\pgfqpoint{1.219976in}{1.500000in}}%
\pgfusepath{stroke}%
\end{pgfscope}%
\begin{pgfscope}%
\pgfpathrectangle{\pgfqpoint{0.198611in}{0.198611in}}{\pgfqpoint{3.602778in}{2.602778in}}%
\pgfusepath{clip}%
\pgfsetbuttcap%
\pgfsetroundjoin%
\pgfsetlinewidth{0.501875pt}%
\definecolor{currentstroke}{rgb}{0.501961,0.501961,0.501961}%
\pgfsetstrokecolor{currentstroke}%
\pgfsetdash{{1.850000pt}{0.800000pt}}{0.000000pt}%
\pgfpathmoveto{\pgfqpoint{2.780024in}{0.632407in}}%
\pgfpathlineto{\pgfqpoint{2.780024in}{1.500000in}}%
\pgfusepath{stroke}%
\end{pgfscope}%
\begin{pgfscope}%
\pgfpathrectangle{\pgfqpoint{0.198611in}{0.198611in}}{\pgfqpoint{3.602778in}{2.602778in}}%
\pgfusepath{clip}%
\pgfsetrectcap%
\pgfsetroundjoin%
\pgfsetlinewidth{0.501875pt}%
\definecolor{currentstroke}{rgb}{0.894118,0.101961,0.109804}%
\pgfsetstrokecolor{currentstroke}%
\pgfsetdash{}{0pt}%
\pgfpathmoveto{\pgfqpoint{0.184722in}{0.566020in}}%
\pgfpathlineto{\pgfqpoint{0.524413in}{0.881513in}}%
\pgfpathlineto{\pgfqpoint{0.768088in}{1.104150in}}%
\pgfpathlineto{\pgfqpoint{0.957613in}{1.273814in}}%
\pgfpathlineto{\pgfqpoint{1.120063in}{1.415436in}}%
\pgfpathlineto{\pgfqpoint{1.255438in}{1.529408in}}%
\pgfpathlineto{\pgfqpoint{1.363738in}{1.616727in}}%
\pgfpathlineto{\pgfqpoint{1.444963in}{1.679103in}}%
\pgfpathlineto{\pgfqpoint{1.526188in}{1.737927in}}%
\pgfpathlineto{\pgfqpoint{1.607413in}{1.792107in}}%
\pgfpathlineto{\pgfqpoint{1.661563in}{1.824956in}}%
\pgfpathlineto{\pgfqpoint{1.715713in}{1.854594in}}%
\pgfpathlineto{\pgfqpoint{1.769863in}{1.880437in}}%
\pgfpathlineto{\pgfqpoint{1.824013in}{1.901850in}}%
\pgfpathlineto{\pgfqpoint{1.878163in}{1.918201in}}%
\pgfpathlineto{\pgfqpoint{1.932313in}{1.928924in}}%
\pgfpathlineto{\pgfqpoint{1.986463in}{1.933600in}}%
\pgfpathlineto{\pgfqpoint{2.013537in}{1.933600in}}%
\pgfpathlineto{\pgfqpoint{2.040612in}{1.932036in}}%
\pgfpathlineto{\pgfqpoint{2.094762in}{1.924297in}}%
\pgfpathlineto{\pgfqpoint{2.148912in}{1.910696in}}%
\pgfpathlineto{\pgfqpoint{2.203062in}{1.891737in}}%
\pgfpathlineto{\pgfqpoint{2.257212in}{1.868029in}}%
\pgfpathlineto{\pgfqpoint{2.311362in}{1.840212in}}%
\pgfpathlineto{\pgfqpoint{2.365512in}{1.808899in}}%
\pgfpathlineto{\pgfqpoint{2.419662in}{1.774644in}}%
\pgfpathlineto{\pgfqpoint{2.500887in}{1.718774in}}%
\pgfpathlineto{\pgfqpoint{2.582112in}{1.658661in}}%
\pgfpathlineto{\pgfqpoint{2.690412in}{1.573580in}}%
\pgfpathlineto{\pgfqpoint{2.798712in}{1.484365in}}%
\pgfpathlineto{\pgfqpoint{2.934087in}{1.368722in}}%
\pgfpathlineto{\pgfqpoint{3.096537in}{1.225745in}}%
\pgfpathlineto{\pgfqpoint{3.313137in}{1.030397in}}%
\pgfpathlineto{\pgfqpoint{3.583887in}{0.781444in}}%
\pgfpathlineto{\pgfqpoint{3.815278in}{0.566020in}}%
\pgfpathlineto{\pgfqpoint{3.815278in}{0.566020in}}%
\pgfusepath{stroke}%
\end{pgfscope}%
\begin{pgfscope}%
\pgfpathrectangle{\pgfqpoint{0.198611in}{0.198611in}}{\pgfqpoint{3.602778in}{2.602778in}}%
\pgfusepath{clip}%
\pgfsetrectcap%
\pgfsetroundjoin%
\pgfsetlinewidth{0.501875pt}%
\definecolor{currentstroke}{rgb}{0.215686,0.494118,0.721569}%
\pgfsetstrokecolor{currentstroke}%
\pgfsetdash{}{0pt}%
\pgfpathmoveto{\pgfqpoint{0.184722in}{2.433980in}}%
\pgfpathlineto{\pgfqpoint{0.524413in}{2.118487in}}%
\pgfpathlineto{\pgfqpoint{0.768088in}{1.895850in}}%
\pgfpathlineto{\pgfqpoint{0.957613in}{1.726186in}}%
\pgfpathlineto{\pgfqpoint{1.120063in}{1.584564in}}%
\pgfpathlineto{\pgfqpoint{1.255438in}{1.470592in}}%
\pgfpathlineto{\pgfqpoint{1.363738in}{1.383273in}}%
\pgfpathlineto{\pgfqpoint{1.444963in}{1.320897in}}%
\pgfpathlineto{\pgfqpoint{1.526188in}{1.262073in}}%
\pgfpathlineto{\pgfqpoint{1.607413in}{1.207893in}}%
\pgfpathlineto{\pgfqpoint{1.661563in}{1.175044in}}%
\pgfpathlineto{\pgfqpoint{1.715713in}{1.145406in}}%
\pgfpathlineto{\pgfqpoint{1.769863in}{1.119563in}}%
\pgfpathlineto{\pgfqpoint{1.824013in}{1.098150in}}%
\pgfpathlineto{\pgfqpoint{1.878163in}{1.081799in}}%
\pgfpathlineto{\pgfqpoint{1.932313in}{1.071076in}}%
\pgfpathlineto{\pgfqpoint{1.986463in}{1.066400in}}%
\pgfpathlineto{\pgfqpoint{2.013537in}{1.066400in}}%
\pgfpathlineto{\pgfqpoint{2.040612in}{1.067964in}}%
\pgfpathlineto{\pgfqpoint{2.094762in}{1.075703in}}%
\pgfpathlineto{\pgfqpoint{2.148912in}{1.089304in}}%
\pgfpathlineto{\pgfqpoint{2.203062in}{1.108263in}}%
\pgfpathlineto{\pgfqpoint{2.257212in}{1.131971in}}%
\pgfpathlineto{\pgfqpoint{2.311362in}{1.159788in}}%
\pgfpathlineto{\pgfqpoint{2.365512in}{1.191101in}}%
\pgfpathlineto{\pgfqpoint{2.419662in}{1.225356in}}%
\pgfpathlineto{\pgfqpoint{2.500887in}{1.281226in}}%
\pgfpathlineto{\pgfqpoint{2.582112in}{1.341339in}}%
\pgfpathlineto{\pgfqpoint{2.690412in}{1.426420in}}%
\pgfpathlineto{\pgfqpoint{2.798712in}{1.515635in}}%
\pgfpathlineto{\pgfqpoint{2.934087in}{1.631278in}}%
\pgfpathlineto{\pgfqpoint{3.096537in}{1.774255in}}%
\pgfpathlineto{\pgfqpoint{3.313137in}{1.969603in}}%
\pgfpathlineto{\pgfqpoint{3.583887in}{2.218556in}}%
\pgfpathlineto{\pgfqpoint{3.815278in}{2.433980in}}%
\pgfpathlineto{\pgfqpoint{3.815278in}{2.433980in}}%
\pgfusepath{stroke}%
\end{pgfscope}%
\begin{pgfscope}%
\pgfpathrectangle{\pgfqpoint{0.198611in}{0.198611in}}{\pgfqpoint{3.602778in}{2.602778in}}%
\pgfusepath{clip}%
\pgfsetbuttcap%
\pgfsetroundjoin%
\pgfsetlinewidth{0.501875pt}%
\definecolor{currentstroke}{rgb}{0.501961,0.501961,0.501961}%
\pgfsetstrokecolor{currentstroke}%
\pgfsetdash{{1.850000pt}{0.800000pt}}{0.000000pt}%
\pgfpathmoveto{\pgfqpoint{1.535234in}{0.184722in}}%
\pgfpathlineto{\pgfqpoint{3.815278in}{2.380971in}}%
\pgfpathlineto{\pgfqpoint{3.815278in}{2.380971in}}%
\pgfusepath{stroke}%
\end{pgfscope}%
\begin{pgfscope}%
\pgfpathrectangle{\pgfqpoint{0.198611in}{0.198611in}}{\pgfqpoint{3.602778in}{2.602778in}}%
\pgfusepath{clip}%
\pgfsetbuttcap%
\pgfsetroundjoin%
\pgfsetlinewidth{0.501875pt}%
\definecolor{currentstroke}{rgb}{0.501961,0.501961,0.501961}%
\pgfsetstrokecolor{currentstroke}%
\pgfsetdash{{1.850000pt}{0.800000pt}}{0.000000pt}%
\pgfpathmoveto{\pgfqpoint{0.184722in}{2.380971in}}%
\pgfpathlineto{\pgfqpoint{2.464766in}{0.184722in}}%
\pgfpathlineto{\pgfqpoint{2.464766in}{0.184722in}}%
\pgfusepath{stroke}%
\end{pgfscope}%
\begin{pgfscope}%
\pgfpathrectangle{\pgfqpoint{0.198611in}{0.198611in}}{\pgfqpoint{3.602778in}{2.602778in}}%
\pgfusepath{clip}%
\pgfsetbuttcap%
\pgfsetroundjoin%
\pgfsetlinewidth{0.501875pt}%
\definecolor{currentstroke}{rgb}{0.501961,0.501961,0.501961}%
\pgfsetstrokecolor{currentstroke}%
\pgfsetdash{{1.850000pt}{0.800000pt}}{0.000000pt}%
\pgfpathmoveto{\pgfqpoint{2.464766in}{2.815278in}}%
\pgfpathlineto{\pgfqpoint{0.184722in}{0.619029in}}%
\pgfpathlineto{\pgfqpoint{0.184722in}{0.619029in}}%
\pgfusepath{stroke}%
\end{pgfscope}%
\begin{pgfscope}%
\pgfpathrectangle{\pgfqpoint{0.198611in}{0.198611in}}{\pgfqpoint{3.602778in}{2.602778in}}%
\pgfusepath{clip}%
\pgfsetbuttcap%
\pgfsetroundjoin%
\pgfsetlinewidth{0.501875pt}%
\definecolor{currentstroke}{rgb}{0.501961,0.501961,0.501961}%
\pgfsetstrokecolor{currentstroke}%
\pgfsetdash{{1.850000pt}{0.800000pt}}{0.000000pt}%
\pgfpathmoveto{\pgfqpoint{3.815278in}{0.619029in}}%
\pgfpathlineto{\pgfqpoint{1.535234in}{2.815278in}}%
\pgfpathlineto{\pgfqpoint{1.535234in}{2.815278in}}%
\pgfusepath{stroke}%
\end{pgfscope}%
\begin{pgfscope}%
\pgfsetrectcap%
\pgfsetmiterjoin%
\pgfsetlinewidth{0.501875pt}%
\definecolor{currentstroke}{rgb}{0.000000,0.000000,0.000000}%
\pgfsetstrokecolor{currentstroke}%
\pgfsetdash{}{0pt}%
\pgfpathmoveto{\pgfqpoint{2.000000in}{0.198611in}}%
\pgfpathlineto{\pgfqpoint{2.000000in}{2.801389in}}%
\pgfusepath{stroke}%
\end{pgfscope}%
\begin{pgfscope}%
\pgfsetrectcap%
\pgfsetmiterjoin%
\pgfsetlinewidth{0.501875pt}%
\definecolor{currentstroke}{rgb}{0.000000,0.000000,0.000000}%
\pgfsetstrokecolor{currentstroke}%
\pgfsetdash{}{0pt}%
\pgfpathmoveto{\pgfqpoint{0.198611in}{0.632407in}}%
\pgfpathlineto{\pgfqpoint{3.801389in}{0.632407in}}%
\pgfusepath{stroke}%
\end{pgfscope}%
\begin{pgfscope}%
\pgfsetroundcap%
\pgfsetroundjoin%
\pgfsetlinewidth{0.501875pt}%
\definecolor{currentstroke}{rgb}{0.000000,0.000000,0.000000}%
\pgfsetstrokecolor{currentstroke}%
\pgfsetdash{}{0pt}%
\pgfpathmoveto{\pgfqpoint{2.000000in}{2.807510in}}%
\pgfpathquadraticcurveto{\pgfqpoint{2.000000in}{2.808331in}}{\pgfqpoint{2.000000in}{2.801389in}}%
\pgfusepath{stroke}%
\end{pgfscope}%
\begin{pgfscope}%
\pgfsetroundcap%
\pgfsetroundjoin%
\pgfsetlinewidth{0.501875pt}%
\definecolor{currentstroke}{rgb}{0.000000,0.000000,0.000000}%
\pgfsetstrokecolor{currentstroke}%
\pgfsetdash{}{0pt}%
\pgfpathmoveto{\pgfqpoint{1.972222in}{2.751954in}}%
\pgfpathlineto{\pgfqpoint{2.000000in}{2.807510in}}%
\pgfpathlineto{\pgfqpoint{2.027778in}{2.751954in}}%
\pgfusepath{stroke}%
\end{pgfscope}%
\begin{pgfscope}%
\pgftext[x=2.000000in,y=2.870833in,,bottom]{\rmfamily\fontsize{10.000000}{12.000000}\selectfont \(\displaystyle \omega^{\prime}\)}%
\end{pgfscope}%
\begin{pgfscope}%
\pgfsetroundcap%
\pgfsetroundjoin%
\pgfsetlinewidth{0.501875pt}%
\definecolor{currentstroke}{rgb}{0.000000,0.000000,0.000000}%
\pgfsetstrokecolor{currentstroke}%
\pgfsetdash{}{0pt}%
\pgfpathmoveto{\pgfqpoint{3.807510in}{0.632407in}}%
\pgfpathquadraticcurveto{\pgfqpoint{3.808332in}{0.632407in}}{\pgfqpoint{3.801389in}{0.632407in}}%
\pgfusepath{stroke}%
\end{pgfscope}%
\begin{pgfscope}%
\pgfsetroundcap%
\pgfsetroundjoin%
\pgfsetlinewidth{0.501875pt}%
\definecolor{currentstroke}{rgb}{0.000000,0.000000,0.000000}%
\pgfsetstrokecolor{currentstroke}%
\pgfsetdash{}{0pt}%
\pgfpathmoveto{\pgfqpoint{3.751955in}{0.660185in}}%
\pgfpathlineto{\pgfqpoint{3.807510in}{0.632407in}}%
\pgfpathlineto{\pgfqpoint{3.751955in}{0.604630in}}%
\pgfusepath{stroke}%
\end{pgfscope}%
\begin{pgfscope}%
\pgftext[x=3.870833in,y=0.632407in,left,]{\rmfamily\fontsize{10.000000}{12.000000}\selectfont \(\displaystyle k^{\prime}\)}%
\end{pgfscope}%
\end{pgfpicture}%
\makeatother%
\endgroup%
} %
        \caption{Das zu berechnende Integral aus \eqref{eq:mass_shell_convolution} visualisiert}
        \label{fig:mass_shell_convolution}
    \end{minipage}\hfill
    \begin{minipage}{0.5\textwidth}
        \centering
        \resizebox{\textwidth}{!}{%% Creator: Matplotlib, PGF backend
%%
%% To include the figure in your LaTeX document, write
%%   \input{<filename>.pgf}
%%
%% Make sure the required packages are loaded in your preamble
%%   \usepackage{pgf}
%%
%% Figures using additional raster images can only be included by \input if
%% they are in the same directory as the main LaTeX file. For loading figures
%% from other directories you can use the `import` package
%%   \usepackage{import}
%% and then include the figures with
%%   \import{<path to file>}{<filename>.pgf}
%%
%% Matplotlib used the following preamble
%%   \usepackage[utf8x]{inputenc}
%%   \usepackage[T1]{fontenc}
%%   \usepackage{amssymb}
%%
\begingroup%
\makeatletter%
\begin{pgfpicture}%
\pgfpathrectangle{\pgfpointorigin}{\pgfqpoint{4.000000in}{3.000000in}}%
\pgfusepath{use as bounding box, clip}%
\begin{pgfscope}%
\pgfsetbuttcap%
\pgfsetmiterjoin%
\definecolor{currentfill}{rgb}{1.000000,1.000000,1.000000}%
\pgfsetfillcolor{currentfill}%
\pgfsetlinewidth{0.000000pt}%
\definecolor{currentstroke}{rgb}{1.000000,1.000000,1.000000}%
\pgfsetstrokecolor{currentstroke}%
\pgfsetdash{}{0pt}%
\pgfpathmoveto{\pgfqpoint{0.000000in}{0.000000in}}%
\pgfpathlineto{\pgfqpoint{4.000000in}{0.000000in}}%
\pgfpathlineto{\pgfqpoint{4.000000in}{3.000000in}}%
\pgfpathlineto{\pgfqpoint{0.000000in}{3.000000in}}%
\pgfpathclose%
\pgfusepath{fill}%
\end{pgfscope}%
\begin{pgfscope}%
\pgfsetbuttcap%
\pgfsetmiterjoin%
\definecolor{currentfill}{rgb}{1.000000,1.000000,1.000000}%
\pgfsetfillcolor{currentfill}%
\pgfsetlinewidth{0.000000pt}%
\definecolor{currentstroke}{rgb}{0.000000,0.000000,0.000000}%
\pgfsetstrokecolor{currentstroke}%
\pgfsetstrokeopacity{0.000000}%
\pgfsetdash{}{0pt}%
\pgfpathmoveto{\pgfqpoint{0.198611in}{0.198611in}}%
\pgfpathlineto{\pgfqpoint{3.801389in}{0.198611in}}%
\pgfpathlineto{\pgfqpoint{3.801389in}{2.801389in}}%
\pgfpathlineto{\pgfqpoint{0.198611in}{2.801389in}}%
\pgfpathclose%
\pgfusepath{fill}%
\end{pgfscope}%
\begin{pgfscope}%
\pgfpathrectangle{\pgfqpoint{0.198611in}{0.198611in}}{\pgfqpoint{3.602778in}{2.602778in}} %
\pgfusepath{clip}%
\pgfsetrectcap%
\pgfsetroundjoin%
\pgfsetlinewidth{1.003750pt}%
\definecolor{currentstroke}{rgb}{0.215686,0.494118,0.721569}%
\pgfsetstrokecolor{currentstroke}%
\pgfsetdash{}{0pt}%
\pgfpathmoveto{\pgfqpoint{0.198611in}{1.239722in}}%
\pgfpathlineto{\pgfqpoint{0.235003in}{1.266013in}}%
\pgfpathlineto{\pgfqpoint{0.271395in}{1.292304in}}%
\pgfpathlineto{\pgfqpoint{0.307786in}{1.318594in}}%
\pgfpathlineto{\pgfqpoint{0.344178in}{1.344885in}}%
\pgfpathlineto{\pgfqpoint{0.380570in}{1.371176in}}%
\pgfpathlineto{\pgfqpoint{0.416961in}{1.397466in}}%
\pgfpathlineto{\pgfqpoint{0.453353in}{1.423757in}}%
\pgfpathlineto{\pgfqpoint{0.489745in}{1.450048in}}%
\pgfpathlineto{\pgfqpoint{0.526136in}{1.476338in}}%
\pgfpathlineto{\pgfqpoint{0.562528in}{1.502629in}}%
\pgfpathlineto{\pgfqpoint{0.598920in}{1.528920in}}%
\pgfpathlineto{\pgfqpoint{0.635311in}{1.555210in}}%
\pgfpathlineto{\pgfqpoint{0.671703in}{1.581501in}}%
\pgfpathlineto{\pgfqpoint{0.708095in}{1.607792in}}%
\pgfpathlineto{\pgfqpoint{0.744487in}{1.634082in}}%
\pgfpathlineto{\pgfqpoint{0.780878in}{1.660373in}}%
\pgfpathlineto{\pgfqpoint{0.817270in}{1.686664in}}%
\pgfpathlineto{\pgfqpoint{0.853662in}{1.712955in}}%
\pgfpathlineto{\pgfqpoint{0.890053in}{1.739245in}}%
\pgfpathlineto{\pgfqpoint{0.926445in}{1.765536in}}%
\pgfpathlineto{\pgfqpoint{0.962837in}{1.791827in}}%
\pgfpathlineto{\pgfqpoint{0.999228in}{1.818117in}}%
\pgfpathlineto{\pgfqpoint{1.035620in}{1.844408in}}%
\pgfpathlineto{\pgfqpoint{1.072012in}{1.870699in}}%
\pgfpathlineto{\pgfqpoint{1.108403in}{1.896989in}}%
\pgfpathlineto{\pgfqpoint{1.144795in}{1.923280in}}%
\pgfpathlineto{\pgfqpoint{1.181187in}{1.949571in}}%
\pgfpathlineto{\pgfqpoint{1.217579in}{1.975861in}}%
\pgfpathlineto{\pgfqpoint{1.253970in}{2.002152in}}%
\pgfpathlineto{\pgfqpoint{1.290362in}{2.028443in}}%
\pgfpathlineto{\pgfqpoint{1.326754in}{2.054733in}}%
\pgfpathlineto{\pgfqpoint{1.363145in}{2.081024in}}%
\pgfpathlineto{\pgfqpoint{1.399537in}{2.107315in}}%
\pgfpathlineto{\pgfqpoint{1.435929in}{2.133605in}}%
\pgfpathlineto{\pgfqpoint{1.472320in}{2.159896in}}%
\pgfpathlineto{\pgfqpoint{1.508712in}{2.186187in}}%
\pgfpathlineto{\pgfqpoint{1.545104in}{2.212478in}}%
\pgfpathlineto{\pgfqpoint{1.581496in}{2.238768in}}%
\pgfpathlineto{\pgfqpoint{1.617887in}{2.265059in}}%
\pgfpathlineto{\pgfqpoint{1.654279in}{2.291350in}}%
\pgfpathlineto{\pgfqpoint{1.690671in}{2.317640in}}%
\pgfpathlineto{\pgfqpoint{1.727062in}{2.343931in}}%
\pgfpathlineto{\pgfqpoint{1.763454in}{2.370222in}}%
\pgfpathlineto{\pgfqpoint{1.799846in}{2.396512in}}%
\pgfpathlineto{\pgfqpoint{1.836237in}{2.422803in}}%
\pgfpathlineto{\pgfqpoint{1.872629in}{2.449094in}}%
\pgfpathlineto{\pgfqpoint{1.909021in}{2.475384in}}%
\pgfpathlineto{\pgfqpoint{1.945412in}{2.501675in}}%
\pgfpathlineto{\pgfqpoint{1.981804in}{2.527966in}}%
\pgfpathlineto{\pgfqpoint{2.018196in}{2.554256in}}%
\pgfpathlineto{\pgfqpoint{2.054588in}{2.580547in}}%
\pgfpathlineto{\pgfqpoint{2.090979in}{2.606838in}}%
\pgfpathlineto{\pgfqpoint{2.127371in}{2.633129in}}%
\pgfpathlineto{\pgfqpoint{2.163763in}{2.659419in}}%
\pgfpathlineto{\pgfqpoint{2.200154in}{2.685710in}}%
\pgfpathlineto{\pgfqpoint{2.236546in}{2.712001in}}%
\pgfpathlineto{\pgfqpoint{2.272938in}{2.738291in}}%
\pgfpathlineto{\pgfqpoint{2.309329in}{2.764582in}}%
\pgfpathlineto{\pgfqpoint{2.345721in}{2.790873in}}%
\pgfpathlineto{\pgfqpoint{2.379503in}{2.815278in}}%
\pgfusepath{stroke}%
\end{pgfscope}%
\begin{pgfscope}%
\pgfpathrectangle{\pgfqpoint{0.198611in}{0.198611in}}{\pgfqpoint{3.602778in}{2.602778in}} %
\pgfusepath{clip}%
\pgfsetrectcap%
\pgfsetroundjoin%
\pgfsetlinewidth{1.003750pt}%
\definecolor{currentstroke}{rgb}{0.215686,0.494118,0.721569}%
\pgfsetstrokecolor{currentstroke}%
\pgfsetdash{}{0pt}%
\pgfpathmoveto{\pgfqpoint{1.620497in}{0.184722in}}%
\pgfpathlineto{\pgfqpoint{1.654279in}{0.209127in}}%
\pgfpathlineto{\pgfqpoint{1.690671in}{0.235418in}}%
\pgfpathlineto{\pgfqpoint{1.727062in}{0.261709in}}%
\pgfpathlineto{\pgfqpoint{1.763454in}{0.287999in}}%
\pgfpathlineto{\pgfqpoint{1.799846in}{0.314290in}}%
\pgfpathlineto{\pgfqpoint{1.836237in}{0.340581in}}%
\pgfpathlineto{\pgfqpoint{1.872629in}{0.366871in}}%
\pgfpathlineto{\pgfqpoint{1.909021in}{0.393162in}}%
\pgfpathlineto{\pgfqpoint{1.945412in}{0.419453in}}%
\pgfpathlineto{\pgfqpoint{1.981804in}{0.445744in}}%
\pgfpathlineto{\pgfqpoint{2.018196in}{0.472034in}}%
\pgfpathlineto{\pgfqpoint{2.054588in}{0.498325in}}%
\pgfpathlineto{\pgfqpoint{2.090979in}{0.524616in}}%
\pgfpathlineto{\pgfqpoint{2.127371in}{0.550906in}}%
\pgfpathlineto{\pgfqpoint{2.163763in}{0.577197in}}%
\pgfpathlineto{\pgfqpoint{2.200154in}{0.603488in}}%
\pgfpathlineto{\pgfqpoint{2.236546in}{0.629778in}}%
\pgfpathlineto{\pgfqpoint{2.272938in}{0.656069in}}%
\pgfpathlineto{\pgfqpoint{2.309329in}{0.682360in}}%
\pgfpathlineto{\pgfqpoint{2.345721in}{0.708650in}}%
\pgfpathlineto{\pgfqpoint{2.382113in}{0.734941in}}%
\pgfpathlineto{\pgfqpoint{2.418504in}{0.761232in}}%
\pgfpathlineto{\pgfqpoint{2.454896in}{0.787522in}}%
\pgfpathlineto{\pgfqpoint{2.491288in}{0.813813in}}%
\pgfpathlineto{\pgfqpoint{2.527680in}{0.840104in}}%
\pgfpathlineto{\pgfqpoint{2.564071in}{0.866395in}}%
\pgfpathlineto{\pgfqpoint{2.600463in}{0.892685in}}%
\pgfpathlineto{\pgfqpoint{2.636855in}{0.918976in}}%
\pgfpathlineto{\pgfqpoint{2.673246in}{0.945267in}}%
\pgfpathlineto{\pgfqpoint{2.709638in}{0.971557in}}%
\pgfpathlineto{\pgfqpoint{2.746030in}{0.997848in}}%
\pgfpathlineto{\pgfqpoint{2.782421in}{1.024139in}}%
\pgfpathlineto{\pgfqpoint{2.818813in}{1.050429in}}%
\pgfpathlineto{\pgfqpoint{2.855205in}{1.076720in}}%
\pgfpathlineto{\pgfqpoint{2.891597in}{1.103011in}}%
\pgfpathlineto{\pgfqpoint{2.927988in}{1.129301in}}%
\pgfpathlineto{\pgfqpoint{2.964380in}{1.155592in}}%
\pgfpathlineto{\pgfqpoint{3.000772in}{1.181883in}}%
\pgfpathlineto{\pgfqpoint{3.037163in}{1.208173in}}%
\pgfpathlineto{\pgfqpoint{3.073555in}{1.234464in}}%
\pgfpathlineto{\pgfqpoint{3.109947in}{1.260755in}}%
\pgfpathlineto{\pgfqpoint{3.146338in}{1.287045in}}%
\pgfpathlineto{\pgfqpoint{3.182730in}{1.313336in}}%
\pgfpathlineto{\pgfqpoint{3.219122in}{1.339627in}}%
\pgfpathlineto{\pgfqpoint{3.255513in}{1.365918in}}%
\pgfpathlineto{\pgfqpoint{3.291905in}{1.392208in}}%
\pgfpathlineto{\pgfqpoint{3.328297in}{1.418499in}}%
\pgfpathlineto{\pgfqpoint{3.364689in}{1.444790in}}%
\pgfpathlineto{\pgfqpoint{3.401080in}{1.471080in}}%
\pgfpathlineto{\pgfqpoint{3.437472in}{1.497371in}}%
\pgfpathlineto{\pgfqpoint{3.473864in}{1.523662in}}%
\pgfpathlineto{\pgfqpoint{3.510255in}{1.549952in}}%
\pgfpathlineto{\pgfqpoint{3.546647in}{1.576243in}}%
\pgfpathlineto{\pgfqpoint{3.583039in}{1.602534in}}%
\pgfpathlineto{\pgfqpoint{3.619430in}{1.628824in}}%
\pgfpathlineto{\pgfqpoint{3.655822in}{1.655115in}}%
\pgfpathlineto{\pgfqpoint{3.692214in}{1.681406in}}%
\pgfpathlineto{\pgfqpoint{3.728605in}{1.707696in}}%
\pgfpathlineto{\pgfqpoint{3.764997in}{1.733987in}}%
\pgfpathlineto{\pgfqpoint{3.801389in}{1.760278in}}%
\pgfusepath{stroke}%
\end{pgfscope}%
\begin{pgfscope}%
\pgfpathrectangle{\pgfqpoint{0.198611in}{0.198611in}}{\pgfqpoint{3.602778in}{2.602778in}} %
\pgfusepath{clip}%
\pgfsetrectcap%
\pgfsetroundjoin%
\pgfsetlinewidth{1.003750pt}%
\definecolor{currentstroke}{rgb}{0.894118,0.101961,0.109804}%
\pgfsetstrokecolor{currentstroke}%
\pgfsetdash{}{0pt}%
\pgfpathmoveto{\pgfqpoint{1.620497in}{2.815278in}}%
\pgfpathlineto{\pgfqpoint{1.654279in}{2.790873in}}%
\pgfpathlineto{\pgfqpoint{1.690671in}{2.764582in}}%
\pgfpathlineto{\pgfqpoint{1.727062in}{2.738291in}}%
\pgfpathlineto{\pgfqpoint{1.763454in}{2.712001in}}%
\pgfpathlineto{\pgfqpoint{1.799846in}{2.685710in}}%
\pgfpathlineto{\pgfqpoint{1.836237in}{2.659419in}}%
\pgfpathlineto{\pgfqpoint{1.872629in}{2.633129in}}%
\pgfpathlineto{\pgfqpoint{1.909021in}{2.606838in}}%
\pgfpathlineto{\pgfqpoint{1.945412in}{2.580547in}}%
\pgfpathlineto{\pgfqpoint{1.981804in}{2.554256in}}%
\pgfpathlineto{\pgfqpoint{2.018196in}{2.527966in}}%
\pgfpathlineto{\pgfqpoint{2.054588in}{2.501675in}}%
\pgfpathlineto{\pgfqpoint{2.090979in}{2.475384in}}%
\pgfpathlineto{\pgfqpoint{2.127371in}{2.449094in}}%
\pgfpathlineto{\pgfqpoint{2.163763in}{2.422803in}}%
\pgfpathlineto{\pgfqpoint{2.200154in}{2.396512in}}%
\pgfpathlineto{\pgfqpoint{2.236546in}{2.370222in}}%
\pgfpathlineto{\pgfqpoint{2.272938in}{2.343931in}}%
\pgfpathlineto{\pgfqpoint{2.309329in}{2.317640in}}%
\pgfpathlineto{\pgfqpoint{2.345721in}{2.291350in}}%
\pgfpathlineto{\pgfqpoint{2.382113in}{2.265059in}}%
\pgfpathlineto{\pgfqpoint{2.418504in}{2.238768in}}%
\pgfpathlineto{\pgfqpoint{2.454896in}{2.212478in}}%
\pgfpathlineto{\pgfqpoint{2.491288in}{2.186187in}}%
\pgfpathlineto{\pgfqpoint{2.527680in}{2.159896in}}%
\pgfpathlineto{\pgfqpoint{2.564071in}{2.133605in}}%
\pgfpathlineto{\pgfqpoint{2.600463in}{2.107315in}}%
\pgfpathlineto{\pgfqpoint{2.636855in}{2.081024in}}%
\pgfpathlineto{\pgfqpoint{2.673246in}{2.054733in}}%
\pgfpathlineto{\pgfqpoint{2.709638in}{2.028443in}}%
\pgfpathlineto{\pgfqpoint{2.746030in}{2.002152in}}%
\pgfpathlineto{\pgfqpoint{2.782421in}{1.975861in}}%
\pgfpathlineto{\pgfqpoint{2.818813in}{1.949571in}}%
\pgfpathlineto{\pgfqpoint{2.855205in}{1.923280in}}%
\pgfpathlineto{\pgfqpoint{2.891597in}{1.896989in}}%
\pgfpathlineto{\pgfqpoint{2.927988in}{1.870699in}}%
\pgfpathlineto{\pgfqpoint{2.964380in}{1.844408in}}%
\pgfpathlineto{\pgfqpoint{3.000772in}{1.818117in}}%
\pgfpathlineto{\pgfqpoint{3.037163in}{1.791827in}}%
\pgfpathlineto{\pgfqpoint{3.073555in}{1.765536in}}%
\pgfpathlineto{\pgfqpoint{3.109947in}{1.739245in}}%
\pgfpathlineto{\pgfqpoint{3.146338in}{1.712955in}}%
\pgfpathlineto{\pgfqpoint{3.182730in}{1.686664in}}%
\pgfpathlineto{\pgfqpoint{3.219122in}{1.660373in}}%
\pgfpathlineto{\pgfqpoint{3.255513in}{1.634082in}}%
\pgfpathlineto{\pgfqpoint{3.291905in}{1.607792in}}%
\pgfpathlineto{\pgfqpoint{3.328297in}{1.581501in}}%
\pgfpathlineto{\pgfqpoint{3.364689in}{1.555210in}}%
\pgfpathlineto{\pgfqpoint{3.401080in}{1.528920in}}%
\pgfpathlineto{\pgfqpoint{3.437472in}{1.502629in}}%
\pgfpathlineto{\pgfqpoint{3.473864in}{1.476338in}}%
\pgfpathlineto{\pgfqpoint{3.510255in}{1.450048in}}%
\pgfpathlineto{\pgfqpoint{3.546647in}{1.423757in}}%
\pgfpathlineto{\pgfqpoint{3.583039in}{1.397466in}}%
\pgfpathlineto{\pgfqpoint{3.619430in}{1.371176in}}%
\pgfpathlineto{\pgfqpoint{3.655822in}{1.344885in}}%
\pgfpathlineto{\pgfqpoint{3.692214in}{1.318594in}}%
\pgfpathlineto{\pgfqpoint{3.728605in}{1.292304in}}%
\pgfpathlineto{\pgfqpoint{3.764997in}{1.266013in}}%
\pgfpathlineto{\pgfqpoint{3.801389in}{1.239722in}}%
\pgfusepath{stroke}%
\end{pgfscope}%
\begin{pgfscope}%
\pgfpathrectangle{\pgfqpoint{0.198611in}{0.198611in}}{\pgfqpoint{3.602778in}{2.602778in}} %
\pgfusepath{clip}%
\pgfsetrectcap%
\pgfsetroundjoin%
\pgfsetlinewidth{1.003750pt}%
\definecolor{currentstroke}{rgb}{0.894118,0.101961,0.109804}%
\pgfsetstrokecolor{currentstroke}%
\pgfsetdash{}{0pt}%
\pgfpathmoveto{\pgfqpoint{0.198611in}{1.760278in}}%
\pgfpathlineto{\pgfqpoint{0.235003in}{1.733987in}}%
\pgfpathlineto{\pgfqpoint{0.271395in}{1.707696in}}%
\pgfpathlineto{\pgfqpoint{0.307786in}{1.681406in}}%
\pgfpathlineto{\pgfqpoint{0.344178in}{1.655115in}}%
\pgfpathlineto{\pgfqpoint{0.380570in}{1.628824in}}%
\pgfpathlineto{\pgfqpoint{0.416961in}{1.602534in}}%
\pgfpathlineto{\pgfqpoint{0.453353in}{1.576243in}}%
\pgfpathlineto{\pgfqpoint{0.489745in}{1.549952in}}%
\pgfpathlineto{\pgfqpoint{0.526136in}{1.523662in}}%
\pgfpathlineto{\pgfqpoint{0.562528in}{1.497371in}}%
\pgfpathlineto{\pgfqpoint{0.598920in}{1.471080in}}%
\pgfpathlineto{\pgfqpoint{0.635311in}{1.444790in}}%
\pgfpathlineto{\pgfqpoint{0.671703in}{1.418499in}}%
\pgfpathlineto{\pgfqpoint{0.708095in}{1.392208in}}%
\pgfpathlineto{\pgfqpoint{0.744487in}{1.365918in}}%
\pgfpathlineto{\pgfqpoint{0.780878in}{1.339627in}}%
\pgfpathlineto{\pgfqpoint{0.817270in}{1.313336in}}%
\pgfpathlineto{\pgfqpoint{0.853662in}{1.287045in}}%
\pgfpathlineto{\pgfqpoint{0.890053in}{1.260755in}}%
\pgfpathlineto{\pgfqpoint{0.926445in}{1.234464in}}%
\pgfpathlineto{\pgfqpoint{0.962837in}{1.208173in}}%
\pgfpathlineto{\pgfqpoint{0.999228in}{1.181883in}}%
\pgfpathlineto{\pgfqpoint{1.035620in}{1.155592in}}%
\pgfpathlineto{\pgfqpoint{1.072012in}{1.129301in}}%
\pgfpathlineto{\pgfqpoint{1.108403in}{1.103011in}}%
\pgfpathlineto{\pgfqpoint{1.144795in}{1.076720in}}%
\pgfpathlineto{\pgfqpoint{1.181187in}{1.050429in}}%
\pgfpathlineto{\pgfqpoint{1.217579in}{1.024139in}}%
\pgfpathlineto{\pgfqpoint{1.253970in}{0.997848in}}%
\pgfpathlineto{\pgfqpoint{1.290362in}{0.971557in}}%
\pgfpathlineto{\pgfqpoint{1.326754in}{0.945267in}}%
\pgfpathlineto{\pgfqpoint{1.363145in}{0.918976in}}%
\pgfpathlineto{\pgfqpoint{1.399537in}{0.892685in}}%
\pgfpathlineto{\pgfqpoint{1.435929in}{0.866395in}}%
\pgfpathlineto{\pgfqpoint{1.472320in}{0.840104in}}%
\pgfpathlineto{\pgfqpoint{1.508712in}{0.813813in}}%
\pgfpathlineto{\pgfqpoint{1.545104in}{0.787522in}}%
\pgfpathlineto{\pgfqpoint{1.581496in}{0.761232in}}%
\pgfpathlineto{\pgfqpoint{1.617887in}{0.734941in}}%
\pgfpathlineto{\pgfqpoint{1.654279in}{0.708650in}}%
\pgfpathlineto{\pgfqpoint{1.690671in}{0.682360in}}%
\pgfpathlineto{\pgfqpoint{1.727062in}{0.656069in}}%
\pgfpathlineto{\pgfqpoint{1.763454in}{0.629778in}}%
\pgfpathlineto{\pgfqpoint{1.799846in}{0.603488in}}%
\pgfpathlineto{\pgfqpoint{1.836237in}{0.577197in}}%
\pgfpathlineto{\pgfqpoint{1.872629in}{0.550906in}}%
\pgfpathlineto{\pgfqpoint{1.909021in}{0.524616in}}%
\pgfpathlineto{\pgfqpoint{1.945412in}{0.498325in}}%
\pgfpathlineto{\pgfqpoint{1.981804in}{0.472034in}}%
\pgfpathlineto{\pgfqpoint{2.018196in}{0.445744in}}%
\pgfpathlineto{\pgfqpoint{2.054588in}{0.419453in}}%
\pgfpathlineto{\pgfqpoint{2.090979in}{0.393162in}}%
\pgfpathlineto{\pgfqpoint{2.127371in}{0.366871in}}%
\pgfpathlineto{\pgfqpoint{2.163763in}{0.340581in}}%
\pgfpathlineto{\pgfqpoint{2.200154in}{0.314290in}}%
\pgfpathlineto{\pgfqpoint{2.236546in}{0.287999in}}%
\pgfpathlineto{\pgfqpoint{2.272938in}{0.261709in}}%
\pgfpathlineto{\pgfqpoint{2.309329in}{0.235418in}}%
\pgfpathlineto{\pgfqpoint{2.345721in}{0.209127in}}%
\pgfpathlineto{\pgfqpoint{2.379503in}{0.184722in}}%
\pgfusepath{stroke}%
\end{pgfscope}%
\begin{pgfscope}%
\pgfpathrectangle{\pgfqpoint{0.198611in}{0.198611in}}{\pgfqpoint{3.602778in}{2.602778in}} %
\pgfusepath{clip}%
\pgfsetrectcap%
\pgfsetroundjoin%
\pgfsetlinewidth{0.501875pt}%
\definecolor{currentstroke}{rgb}{0.501961,0.501961,0.501961}%
\pgfsetstrokecolor{currentstroke}%
\pgfsetdash{}{0pt}%
\pgfpathmoveto{\pgfqpoint{0.955194in}{1.213694in}}%
\pgfpathlineto{\pgfqpoint{2.000000in}{2.541111in}}%
\pgfusepath{stroke}%
\end{pgfscope}%
\begin{pgfscope}%
\pgfpathrectangle{\pgfqpoint{0.198611in}{0.198611in}}{\pgfqpoint{3.602778in}{2.602778in}} %
\pgfusepath{clip}%
\pgfsetrectcap%
\pgfsetroundjoin%
\pgfsetlinewidth{0.501875pt}%
\definecolor{currentstroke}{rgb}{0.501961,0.501961,0.501961}%
\pgfsetstrokecolor{currentstroke}%
\pgfsetdash{}{0pt}%
\pgfpathmoveto{\pgfqpoint{2.000000in}{0.458889in}}%
\pgfpathlineto{\pgfqpoint{3.441111in}{0.458889in}}%
\pgfusepath{stroke}%
\end{pgfscope}%
\begin{pgfscope}%
\pgfpathrectangle{\pgfqpoint{0.198611in}{0.198611in}}{\pgfqpoint{3.602778in}{2.602778in}} %
\pgfusepath{clip}%
\pgfsetrectcap%
\pgfsetroundjoin%
\pgfsetlinewidth{0.501875pt}%
\definecolor{currentstroke}{rgb}{0.501961,0.501961,0.501961}%
\pgfsetstrokecolor{currentstroke}%
\pgfsetdash{}{0pt}%
\pgfpathmoveto{\pgfqpoint{3.441111in}{0.458889in}}%
\pgfpathlineto{\pgfqpoint{3.441111in}{1.500000in}}%
\pgfusepath{stroke}%
\end{pgfscope}%
\begin{pgfscope}%
\pgftext[x=2.720556in,y=0.289708in,left,base]{\rmfamily\fontsize{10.000000}{12.000000}\selectfont d\(\displaystyle k^\prime\)}%
\end{pgfscope}%
\begin{pgfscope}%
\pgftext[x=3.477139in,y=0.914375in,left,base]{\rmfamily\fontsize{10.000000}{12.000000}\selectfont d\(\displaystyle \omega^\prime\)}%
\end{pgfscope}%
\begin{pgfscope}%
\pgftext[x=2.180139in,y=0.497931in,left,base]{\rmfamily\fontsize{10.000000}{12.000000}\selectfont \(\displaystyle \alpha\)}%
\end{pgfscope}%
\begin{pgfscope}%
\pgftext[x=0.703000in,y=1.460958in,left,base]{\rmfamily\fontsize{10.000000}{12.000000}\selectfont \(\displaystyle 2 \alpha\)}%
\end{pgfscope}%
\begin{pgfscope}%
\pgftext[x=1.189375in,y=2.085625in,left,base]{\rmfamily\fontsize{10.000000}{12.000000}\selectfont \(\displaystyle l\)}%
\end{pgfscope}%
\begin{pgfscope}%
\pgftext[x=1.549653in,y=1.825347in,left,base]{\rmfamily\fontsize{10.000000}{12.000000}\selectfont \(\displaystyle h\)}%
\end{pgfscope}%
\end{pgfpicture}%
\makeatother%
\endgroup%
}
        \caption{Die Kreuzungstelle bei $k_{0+}'$ von ganz nah angeschaut}
        \label{fig:schulgeometrie}
    \end{minipage}
\end{figure}

\begin{equation}
    \rwhat{\Delta_m^{* 2}} (\omega, k)
    = \int \theta(\omega') \delta(\omega^{\prime 2}-k^{\prime2}-m^2)\theta(\omega-\omega')
      \delta((\omega-\omega')^2-(k-k')^2-m^2) \d \omega' \d k'
\label{eq:mass_shell_convolution}
\end{equation}

An Abbildung \ref{fig:mass_shell_convolution} sehen wir schon, dass das Faltungsintegral nur dann ungleich null ist, wenn $(\omega, k)$ in der 2$m$-Massenschale liegen. Es ist also insbesondere $\omega > 0$.
Da $\Delta_m$ Poincare-invariant ist, sind $\Delta_m^2$ und $\rwhat{\Delta_m^{*2}}$ es auch. Es genügt also $\rwhat{\Delta_m^{*2}}$ für $k=0$ und positive $\omega$ zu berechnen. Alle anderen Werte holen wir uns dann aus der Poincare-Invarianz.

\todo{Wie erklärt man das besser, ohne an Anschaulichkeit oder Rigorosität zu verlieren}
Um nun das Integral über zwei sich schneidende lineare\footnote{Linear in dem Sinne, dass die Distribution entlang einer Linie getragen ist. Nicht das es eine lineare Distribution ist} $\delta$-Distributionen zu berechnen bedienen wir uns eines Physikertricks und stellen uns die $\delta$-Distribution als Grenzwert ($h \rightarrow 0$) einer $\frac{1}{h}$-hohen und $h$-breiten Rechtecksfunktion vor. Dann ist das Integral über die sich schneidenden Rechteckfunktionen proportional zu der Schnittfläche und damit zu $l \cdot h$ in Abb. \ref{fig:schulgeometrie}. Außerdem schneiden sich die beiden Hyperbeln für $\omega \rightarrow +\infty$ in einem rechten Winkel, das Faltungsintegral ergibt hier also 2.

Aus Abb. \ref{fig:schulgeometrie} lesen wir ab:

\begin{align}
    \tan(\alpha) &= \frac{\d \omega'}{\d k'}
    ~~\textrm{ und }~~
    \frac{h}{l} = \sin (2 \alpha) \nonumber \\
    \Rightarrow l &=
    \frac{h}{\sin\left(2 \arctan \left(\frac{\d \omega'}{\d k'}\right)\right)}
    = \frac{h \left(\left(\frac{\d \omega'}{\d k'}\right)^2+1\right)}
           {2 \frac{\d \omega'}{\d k'}}
    \label{eq:schulgeometrie2}
\end{align}

außerdem gilt

\begin{equation}
    \omega' = \sqrt{k{\prime 2} + m^2}
    ~~\Longrightarrow~~
    \frac{\d \omega'}{\d k'} = \frac{k'}{\sqrt{k^{\prime 2}+m^2}}
    \label{eq:schulgeometrie3}
\end{equation}

Wenn wir nun \eqref{eq:schulgeometrie2} und \eqref{eq:schulgeometrie3} sowie die vorhergehenden Gedanken kombinieren erhalten wir

\begin{dmath}
    \rwhat{\Delta_m}^{*2} (\omega, 0) = C \frac{
        \left.\left(\d \omega'/\d k'\right)^2\right|_{k'_0} +1
    }
    {
        \left. \d \omega' / \d k'\right|_{k'_0}
    }
    \Theta(\omega^2-(2m)^2)
    =
    C \frac{
        \sqrt{k_0^{\prime 2}+m^2}(2 k_0^{\prime 2}+m^2)
    }{
        2 k'_0 (k_0^2+m^2)
    }\Theta(\dots)
    =
    C \frac{\sqrt{\frac{1}{4} \omega^2 -m^2+m^2} (\omega^2-4m^2+m^2)}
    {\sqrt{\omega^2-4m^2}(\frac{1}{4}\omega^2-m^2+m^2)}
    \Theta(\dots)
    =
    C \frac{\omega^2-3m^2}{\omega \sqrt{\omega^2-4m^2}}\Theta(\dots)
    \stackrel{C=2}{=}
    2 \frac{\omega^2-3m^2}{\omega \sqrt{\omega^2-4m^2}}\Theta(\dots)
    \nonumber
\end{dmath}

\todo{\texttt{breqn}-Package ausprobieren!}

Jetzt erhalten wir $\rwhat{\Delta_m}^{*2}(\omega, k)$ noch aus der Poincare-Invarianz:

\begin{dmath}
    \rwhat{\Delta_m}^{*2}(\omega, k)
    \stackrel{(\omega,k) \sim (\sqrt{\omega^2-k^2},0)}{=}
    \rwhat{\Delta_m}^{*2}(\sqrt{\omega^2-k^2},0)
    = 2 \frac{\omega^2-k^2-3m^2}
              {\sqrt{\omega^2-k^2}\sqrt{\omega^2-k^2-4m^2}}
              \Theta(\omega^2-k^2-4m^2)
\end{dmath}

Es ist zu beachten, dass die Heaviside-Funktion genau bei der ersten Nullstelle der zweiten Wurzel im Nenner abschneidet und alle weiteren Nullstellen sowohl des Nenners als auch des Zählers außerhalb der $2m$-Massenschale und damit außerhalb des Trägers der Heaviside-Funktion liegen.


\begin{figure}
    \centering
    \begin{minipage}{0.55\textwidth}
        \centering
        \resizebox{\textwidth}{!}{%% Creator: Matplotlib, PGF backend
%%
%% To include the figure in your LaTeX document, write
%%   \input{<filename>.pgf}
%%
%% Make sure the required packages are loaded in your preamble
%%   \usepackage{pgf}
%%
%% Figures using additional raster images can only be included by \input if
%% they are in the same directory as the main LaTeX file. For loading figures
%% from other directories you can use the `import` package
%%   \usepackage{import}
%% and then include the figures with
%%   \import{<path to file>}{<filename>.pgf}
%%
%% Matplotlib used the following preamble
%%   \usepackage[utf8x]{inputenc}
%%   \usepackage[T1]{fontenc}
%%   \usepackage{amssymb}
%%
\begingroup%
\makeatletter%
\begin{pgfpicture}%
\pgfpathrectangle{\pgfpointorigin}{\pgfqpoint{4.000000in}{2.200000in}}%
\pgfusepath{use as bounding box, clip}%
\begin{pgfscope}%
\pgfsetbuttcap%
\pgfsetmiterjoin%
\definecolor{currentfill}{rgb}{1.000000,1.000000,1.000000}%
\pgfsetfillcolor{currentfill}%
\pgfsetlinewidth{0.000000pt}%
\definecolor{currentstroke}{rgb}{1.000000,1.000000,1.000000}%
\pgfsetstrokecolor{currentstroke}%
\pgfsetdash{}{0pt}%
\pgfpathmoveto{\pgfqpoint{0.000000in}{0.000000in}}%
\pgfpathlineto{\pgfqpoint{4.000000in}{0.000000in}}%
\pgfpathlineto{\pgfqpoint{4.000000in}{2.200000in}}%
\pgfpathlineto{\pgfqpoint{0.000000in}{2.200000in}}%
\pgfpathclose%
\pgfusepath{fill}%
\end{pgfscope}%
\begin{pgfscope}%
\pgfsetbuttcap%
\pgfsetmiterjoin%
\definecolor{currentfill}{rgb}{1.000000,1.000000,1.000000}%
\pgfsetfillcolor{currentfill}%
\pgfsetlinewidth{0.000000pt}%
\definecolor{currentstroke}{rgb}{0.000000,0.000000,0.000000}%
\pgfsetstrokecolor{currentstroke}%
\pgfsetstrokeopacity{0.000000}%
\pgfsetdash{}{0pt}%
\pgfpathmoveto{\pgfqpoint{0.198611in}{0.333208in}}%
\pgfpathlineto{\pgfqpoint{3.119722in}{0.333208in}}%
\pgfpathlineto{\pgfqpoint{3.119722in}{1.866792in}}%
\pgfpathlineto{\pgfqpoint{0.198611in}{1.866792in}}%
\pgfpathclose%
\pgfusepath{fill}%
\end{pgfscope}%
\begin{pgfscope}%
\pgfpathrectangle{\pgfqpoint{0.198611in}{0.333208in}}{\pgfqpoint{2.921111in}{1.533583in}} %
\pgfusepath{clip}%
\pgfsys@transformshift{0.198611in}{0.333208in}%
\pgftext[left,bottom]{\pgfimage[interpolate=true,width=2.921667in,height=1.535000in]{delta_m2-img0.png}}%
\end{pgfscope}%
\begin{pgfscope}%
\pgfpathrectangle{\pgfqpoint{0.198611in}{0.333208in}}{\pgfqpoint{2.921111in}{1.533583in}} %
\pgfusepath{clip}%
\pgfsetrectcap%
\pgfsetroundjoin%
\pgfsetlinewidth{0.501875pt}%
\definecolor{currentstroke}{rgb}{0.894118,0.101961,0.109804}%
\pgfsetstrokecolor{currentstroke}%
\pgfsetdash{}{0pt}%
\pgfpathmoveto{\pgfqpoint{0.208387in}{1.868458in}}%
\pgfpathlineto{\pgfqpoint{0.409353in}{1.669314in}}%
\pgfpathlineto{\pgfqpoint{0.573263in}{1.507381in}}%
\pgfpathlineto{\pgfqpoint{0.707903in}{1.374861in}}%
\pgfpathlineto{\pgfqpoint{0.819128in}{1.265885in}}%
\pgfpathlineto{\pgfqpoint{0.912791in}{1.174617in}}%
\pgfpathlineto{\pgfqpoint{0.994746in}{1.095284in}}%
\pgfpathlineto{\pgfqpoint{1.059139in}{1.033424in}}%
\pgfpathlineto{\pgfqpoint{1.117678in}{0.977674in}}%
\pgfpathlineto{\pgfqpoint{1.170364in}{0.928022in}}%
\pgfpathlineto{\pgfqpoint{1.217195in}{0.884431in}}%
\pgfpathlineto{\pgfqpoint{1.258172in}{0.846836in}}%
\pgfpathlineto{\pgfqpoint{1.293296in}{0.815128in}}%
\pgfpathlineto{\pgfqpoint{1.322566in}{0.789161in}}%
\pgfpathlineto{\pgfqpoint{1.351835in}{0.763705in}}%
\pgfpathlineto{\pgfqpoint{1.375251in}{0.743786in}}%
\pgfpathlineto{\pgfqpoint{1.398667in}{0.724343in}}%
\pgfpathlineto{\pgfqpoint{1.422082in}{0.705469in}}%
\pgfpathlineto{\pgfqpoint{1.439644in}{0.691756in}}%
\pgfpathlineto{\pgfqpoint{1.457206in}{0.678485in}}%
\pgfpathlineto{\pgfqpoint{1.474768in}{0.665725in}}%
\pgfpathlineto{\pgfqpoint{1.492330in}{0.653554in}}%
\pgfpathlineto{\pgfqpoint{1.504038in}{0.645812in}}%
\pgfpathlineto{\pgfqpoint{1.515745in}{0.638403in}}%
\pgfpathlineto{\pgfqpoint{1.527453in}{0.631358in}}%
\pgfpathlineto{\pgfqpoint{1.539161in}{0.624715in}}%
\pgfpathlineto{\pgfqpoint{1.550869in}{0.618510in}}%
\pgfpathlineto{\pgfqpoint{1.562577in}{0.612782in}}%
\pgfpathlineto{\pgfqpoint{1.574285in}{0.607573in}}%
\pgfpathlineto{\pgfqpoint{1.585993in}{0.602924in}}%
\pgfpathlineto{\pgfqpoint{1.597700in}{0.598875in}}%
\pgfpathlineto{\pgfqpoint{1.609408in}{0.595465in}}%
\pgfpathlineto{\pgfqpoint{1.621116in}{0.592729in}}%
\pgfpathlineto{\pgfqpoint{1.632824in}{0.590696in}}%
\pgfpathlineto{\pgfqpoint{1.644532in}{0.589391in}}%
\pgfpathlineto{\pgfqpoint{1.650386in}{0.589017in}}%
\pgfpathlineto{\pgfqpoint{1.656240in}{0.588829in}}%
\pgfpathlineto{\pgfqpoint{1.662094in}{0.588829in}}%
\pgfpathlineto{\pgfqpoint{1.667948in}{0.589017in}}%
\pgfpathlineto{\pgfqpoint{1.673801in}{0.589391in}}%
\pgfpathlineto{\pgfqpoint{1.685509in}{0.590696in}}%
\pgfpathlineto{\pgfqpoint{1.697217in}{0.592729in}}%
\pgfpathlineto{\pgfqpoint{1.708925in}{0.595465in}}%
\pgfpathlineto{\pgfqpoint{1.720633in}{0.598875in}}%
\pgfpathlineto{\pgfqpoint{1.732341in}{0.602924in}}%
\pgfpathlineto{\pgfqpoint{1.744049in}{0.607573in}}%
\pgfpathlineto{\pgfqpoint{1.755757in}{0.612782in}}%
\pgfpathlineto{\pgfqpoint{1.767464in}{0.618510in}}%
\pgfpathlineto{\pgfqpoint{1.779172in}{0.624715in}}%
\pgfpathlineto{\pgfqpoint{1.790880in}{0.631358in}}%
\pgfpathlineto{\pgfqpoint{1.802588in}{0.638403in}}%
\pgfpathlineto{\pgfqpoint{1.814296in}{0.645812in}}%
\pgfpathlineto{\pgfqpoint{1.826004in}{0.653554in}}%
\pgfpathlineto{\pgfqpoint{1.843565in}{0.665725in}}%
\pgfpathlineto{\pgfqpoint{1.861127in}{0.678485in}}%
\pgfpathlineto{\pgfqpoint{1.878689in}{0.691756in}}%
\pgfpathlineto{\pgfqpoint{1.896251in}{0.705469in}}%
\pgfpathlineto{\pgfqpoint{1.913813in}{0.719567in}}%
\pgfpathlineto{\pgfqpoint{1.937228in}{0.738877in}}%
\pgfpathlineto{\pgfqpoint{1.960644in}{0.758685in}}%
\pgfpathlineto{\pgfqpoint{1.989914in}{0.784026in}}%
\pgfpathlineto{\pgfqpoint{2.019183in}{0.809899in}}%
\pgfpathlineto{\pgfqpoint{2.054307in}{0.841515in}}%
\pgfpathlineto{\pgfqpoint{2.089431in}{0.873632in}}%
\pgfpathlineto{\pgfqpoint{2.130408in}{0.911607in}}%
\pgfpathlineto{\pgfqpoint{2.177239in}{0.955537in}}%
\pgfpathlineto{\pgfqpoint{2.229925in}{1.005483in}}%
\pgfpathlineto{\pgfqpoint{2.288464in}{1.061482in}}%
\pgfpathlineto{\pgfqpoint{2.358711in}{1.129212in}}%
\pgfpathlineto{\pgfqpoint{2.440666in}{1.208778in}}%
\pgfpathlineto{\pgfqpoint{2.534329in}{1.300239in}}%
\pgfpathlineto{\pgfqpoint{2.645554in}{1.409377in}}%
\pgfpathlineto{\pgfqpoint{2.780194in}{1.542033in}}%
\pgfpathlineto{\pgfqpoint{2.944104in}{1.704079in}}%
\pgfpathlineto{\pgfqpoint{3.109946in}{1.868458in}}%
\pgfpathlineto{\pgfqpoint{3.109946in}{1.868458in}}%
\pgfusepath{stroke}%
\end{pgfscope}%
\begin{pgfscope}%
\pgfpathrectangle{\pgfqpoint{0.198611in}{0.333208in}}{\pgfqpoint{2.921111in}{1.533583in}} %
\pgfusepath{clip}%
\pgfsetrectcap%
\pgfsetroundjoin%
\pgfsetlinewidth{0.200750pt}%
\definecolor{currentstroke}{rgb}{0.993248,0.906157,0.143936}%
\pgfsetstrokecolor{currentstroke}%
\pgfsetdash{}{0pt}%
\pgfpathmoveto{\pgfqpoint{0.243269in}{1.868458in}}%
\pgfpathlineto{\pgfqpoint{0.344959in}{1.770226in}}%
\pgfpathlineto{\pgfqpoint{0.438622in}{1.680228in}}%
\pgfpathlineto{\pgfqpoint{0.526431in}{1.596369in}}%
\pgfpathlineto{\pgfqpoint{0.602532in}{1.524182in}}%
\pgfpathlineto{\pgfqpoint{0.672779in}{1.458037in}}%
\pgfpathlineto{\pgfqpoint{0.737173in}{1.397901in}}%
\pgfpathlineto{\pgfqpoint{0.795712in}{1.343722in}}%
\pgfpathlineto{\pgfqpoint{0.848397in}{1.295434in}}%
\pgfpathlineto{\pgfqpoint{0.901083in}{1.247674in}}%
\pgfpathlineto{\pgfqpoint{0.947914in}{1.205740in}}%
\pgfpathlineto{\pgfqpoint{0.988892in}{1.169515in}}%
\pgfpathlineto{\pgfqpoint{1.029869in}{1.133795in}}%
\pgfpathlineto{\pgfqpoint{1.064993in}{1.103638in}}%
\pgfpathlineto{\pgfqpoint{1.100116in}{1.073966in}}%
\pgfpathlineto{\pgfqpoint{1.129386in}{1.049660in}}%
\pgfpathlineto{\pgfqpoint{1.158656in}{1.025782in}}%
\pgfpathlineto{\pgfqpoint{1.187925in}{1.002386in}}%
\pgfpathlineto{\pgfqpoint{1.217195in}{0.979530in}}%
\pgfpathlineto{\pgfqpoint{1.240611in}{0.961678in}}%
\pgfpathlineto{\pgfqpoint{1.264026in}{0.944253in}}%
\pgfpathlineto{\pgfqpoint{1.287442in}{0.927298in}}%
\pgfpathlineto{\pgfqpoint{1.310858in}{0.910860in}}%
\pgfpathlineto{\pgfqpoint{1.334274in}{0.894992in}}%
\pgfpathlineto{\pgfqpoint{1.351835in}{0.883498in}}%
\pgfpathlineto{\pgfqpoint{1.369397in}{0.872383in}}%
\pgfpathlineto{\pgfqpoint{1.386959in}{0.861674in}}%
\pgfpathlineto{\pgfqpoint{1.404521in}{0.851400in}}%
\pgfpathlineto{\pgfqpoint{1.422082in}{0.841593in}}%
\pgfpathlineto{\pgfqpoint{1.439644in}{0.832284in}}%
\pgfpathlineto{\pgfqpoint{1.457206in}{0.823506in}}%
\pgfpathlineto{\pgfqpoint{1.474768in}{0.815295in}}%
\pgfpathlineto{\pgfqpoint{1.486476in}{0.810153in}}%
\pgfpathlineto{\pgfqpoint{1.498184in}{0.805287in}}%
\pgfpathlineto{\pgfqpoint{1.509891in}{0.800710in}}%
\pgfpathlineto{\pgfqpoint{1.521599in}{0.796430in}}%
\pgfpathlineto{\pgfqpoint{1.533307in}{0.792458in}}%
\pgfpathlineto{\pgfqpoint{1.545015in}{0.788802in}}%
\pgfpathlineto{\pgfqpoint{1.556723in}{0.785474in}}%
\pgfpathlineto{\pgfqpoint{1.568431in}{0.782480in}}%
\pgfpathlineto{\pgfqpoint{1.580139in}{0.779829in}}%
\pgfpathlineto{\pgfqpoint{1.591846in}{0.777529in}}%
\pgfpathlineto{\pgfqpoint{1.603554in}{0.775586in}}%
\pgfpathlineto{\pgfqpoint{1.615262in}{0.774005in}}%
\pgfpathlineto{\pgfqpoint{1.626970in}{0.772792in}}%
\pgfpathlineto{\pgfqpoint{1.638678in}{0.771949in}}%
\pgfpathlineto{\pgfqpoint{1.650386in}{0.771481in}}%
\pgfpathlineto{\pgfqpoint{1.662094in}{0.771387in}}%
\pgfpathlineto{\pgfqpoint{1.673801in}{0.771668in}}%
\pgfpathlineto{\pgfqpoint{1.685509in}{0.772324in}}%
\pgfpathlineto{\pgfqpoint{1.697217in}{0.773352in}}%
\pgfpathlineto{\pgfqpoint{1.708925in}{0.774750in}}%
\pgfpathlineto{\pgfqpoint{1.720633in}{0.776512in}}%
\pgfpathlineto{\pgfqpoint{1.732341in}{0.778635in}}%
\pgfpathlineto{\pgfqpoint{1.744049in}{0.781111in}}%
\pgfpathlineto{\pgfqpoint{1.755757in}{0.783934in}}%
\pgfpathlineto{\pgfqpoint{1.767464in}{0.787097in}}%
\pgfpathlineto{\pgfqpoint{1.779172in}{0.790590in}}%
\pgfpathlineto{\pgfqpoint{1.790880in}{0.794405in}}%
\pgfpathlineto{\pgfqpoint{1.802588in}{0.798532in}}%
\pgfpathlineto{\pgfqpoint{1.814296in}{0.802962in}}%
\pgfpathlineto{\pgfqpoint{1.826004in}{0.807685in}}%
\pgfpathlineto{\pgfqpoint{1.837712in}{0.812690in}}%
\pgfpathlineto{\pgfqpoint{1.855273in}{0.820705in}}%
\pgfpathlineto{\pgfqpoint{1.872835in}{0.829297in}}%
\pgfpathlineto{\pgfqpoint{1.890397in}{0.838433in}}%
\pgfpathlineto{\pgfqpoint{1.907959in}{0.848078in}}%
\pgfpathlineto{\pgfqpoint{1.925520in}{0.858199in}}%
\pgfpathlineto{\pgfqpoint{1.943082in}{0.868767in}}%
\pgfpathlineto{\pgfqpoint{1.960644in}{0.879750in}}%
\pgfpathlineto{\pgfqpoint{1.978206in}{0.891120in}}%
\pgfpathlineto{\pgfqpoint{1.995768in}{0.902851in}}%
\pgfpathlineto{\pgfqpoint{2.019183in}{0.919011in}}%
\pgfpathlineto{\pgfqpoint{2.042599in}{0.935714in}}%
\pgfpathlineto{\pgfqpoint{2.066015in}{0.952910in}}%
\pgfpathlineto{\pgfqpoint{2.089431in}{0.970553in}}%
\pgfpathlineto{\pgfqpoint{2.112846in}{0.988603in}}%
\pgfpathlineto{\pgfqpoint{2.142116in}{1.011683in}}%
\pgfpathlineto{\pgfqpoint{2.171386in}{1.035279in}}%
\pgfpathlineto{\pgfqpoint{2.200655in}{1.059334in}}%
\pgfpathlineto{\pgfqpoint{2.235779in}{1.088737in}}%
\pgfpathlineto{\pgfqpoint{2.270902in}{1.118659in}}%
\pgfpathlineto{\pgfqpoint{2.306026in}{1.149037in}}%
\pgfpathlineto{\pgfqpoint{2.347003in}{1.184982in}}%
\pgfpathlineto{\pgfqpoint{2.387981in}{1.221403in}}%
\pgfpathlineto{\pgfqpoint{2.434812in}{1.263530in}}%
\pgfpathlineto{\pgfqpoint{2.487498in}{1.311476in}}%
\pgfpathlineto{\pgfqpoint{2.540183in}{1.359922in}}%
\pgfpathlineto{\pgfqpoint{2.598722in}{1.414250in}}%
\pgfpathlineto{\pgfqpoint{2.663116in}{1.474524in}}%
\pgfpathlineto{\pgfqpoint{2.733363in}{1.540795in}}%
\pgfpathlineto{\pgfqpoint{2.809464in}{1.613096in}}%
\pgfpathlineto{\pgfqpoint{2.897273in}{1.697063in}}%
\pgfpathlineto{\pgfqpoint{2.990936in}{1.787154in}}%
\pgfpathlineto{\pgfqpoint{3.075065in}{1.868458in}}%
\pgfpathlineto{\pgfqpoint{3.075065in}{1.868458in}}%
\pgfusepath{stroke}%
\end{pgfscope}%
\begin{pgfscope}%
\pgfpathrectangle{\pgfqpoint{0.198611in}{0.333208in}}{\pgfqpoint{2.921111in}{1.533583in}} %
\pgfusepath{clip}%
\pgfsetbuttcap%
\pgfsetroundjoin%
\pgfsetlinewidth{0.501875pt}%
\definecolor{currentstroke}{rgb}{0.501961,0.501961,0.501961}%
\pgfsetstrokecolor{currentstroke}%
\pgfsetdash{{1.850000pt}{0.800000pt}}{0.000000pt}%
\pgfpathmoveto{\pgfqpoint{1.584472in}{0.331542in}}%
\pgfpathlineto{\pgfqpoint{3.119722in}{1.866792in}}%
\pgfpathlineto{\pgfqpoint{3.119722in}{1.866792in}}%
\pgfusepath{stroke}%
\end{pgfscope}%
\begin{pgfscope}%
\pgfpathrectangle{\pgfqpoint{0.198611in}{0.333208in}}{\pgfqpoint{2.921111in}{1.533583in}} %
\pgfusepath{clip}%
\pgfsetbuttcap%
\pgfsetroundjoin%
\pgfsetlinewidth{0.501875pt}%
\definecolor{currentstroke}{rgb}{0.501961,0.501961,0.501961}%
\pgfsetstrokecolor{currentstroke}%
\pgfsetdash{{1.850000pt}{0.800000pt}}{0.000000pt}%
\pgfpathmoveto{\pgfqpoint{0.198611in}{1.866792in}}%
\pgfpathlineto{\pgfqpoint{1.733861in}{0.331542in}}%
\pgfpathlineto{\pgfqpoint{1.733861in}{0.331542in}}%
\pgfusepath{stroke}%
\end{pgfscope}%
\begin{pgfscope}%
\pgfsetrectcap%
\pgfsetmiterjoin%
\pgfsetlinewidth{0.501875pt}%
\definecolor{currentstroke}{rgb}{0.000000,0.000000,0.000000}%
\pgfsetstrokecolor{currentstroke}%
\pgfsetdash{}{0pt}%
\pgfpathmoveto{\pgfqpoint{1.659167in}{0.333208in}}%
\pgfpathlineto{\pgfqpoint{1.659167in}{1.866792in}}%
\pgfusepath{stroke}%
\end{pgfscope}%
\begin{pgfscope}%
\pgfsetrectcap%
\pgfsetmiterjoin%
\pgfsetlinewidth{0.501875pt}%
\definecolor{currentstroke}{rgb}{0.000000,0.000000,0.000000}%
\pgfsetstrokecolor{currentstroke}%
\pgfsetdash{}{0pt}%
\pgfpathmoveto{\pgfqpoint{0.198611in}{0.406236in}}%
\pgfpathlineto{\pgfqpoint{3.119722in}{0.406236in}}%
\pgfusepath{stroke}%
\end{pgfscope}%
\begin{pgfscope}%
\pgfsetroundcap%
\pgfsetroundjoin%
\pgfsetlinewidth{0.501875pt}%
\definecolor{currentstroke}{rgb}{0.000000,0.000000,0.000000}%
\pgfsetstrokecolor{currentstroke}%
\pgfsetdash{}{0pt}%
\pgfpathmoveto{\pgfqpoint{1.659167in}{1.872920in}}%
\pgfpathquadraticcurveto{\pgfqpoint{1.659167in}{1.873738in}}{\pgfqpoint{1.659167in}{1.866792in}}%
\pgfusepath{stroke}%
\end{pgfscope}%
\begin{pgfscope}%
\pgfsetroundcap%
\pgfsetroundjoin%
\pgfsetlinewidth{0.501875pt}%
\definecolor{currentstroke}{rgb}{0.000000,0.000000,0.000000}%
\pgfsetstrokecolor{currentstroke}%
\pgfsetdash{}{0pt}%
\pgfpathmoveto{\pgfqpoint{1.631389in}{1.817364in}}%
\pgfpathlineto{\pgfqpoint{1.659167in}{1.872920in}}%
\pgfpathlineto{\pgfqpoint{1.686944in}{1.817364in}}%
\pgfusepath{stroke}%
\end{pgfscope}%
\begin{pgfscope}%
\pgftext[x=1.659167in,y=1.936236in,,bottom]{\rmfamily\fontsize{10.000000}{12.000000}\selectfont \(\displaystyle \omega\)}%
\end{pgfscope}%
\begin{pgfscope}%
\pgfsetroundcap%
\pgfsetroundjoin%
\pgfsetlinewidth{0.501875pt}%
\definecolor{currentstroke}{rgb}{0.000000,0.000000,0.000000}%
\pgfsetstrokecolor{currentstroke}%
\pgfsetdash{}{0pt}%
\pgfpathmoveto{\pgfqpoint{3.125847in}{0.406236in}}%
\pgfpathquadraticcurveto{\pgfqpoint{3.126667in}{0.406236in}}{\pgfqpoint{3.119722in}{0.406236in}}%
\pgfusepath{stroke}%
\end{pgfscope}%
\begin{pgfscope}%
\pgfsetroundcap%
\pgfsetroundjoin%
\pgfsetlinewidth{0.501875pt}%
\definecolor{currentstroke}{rgb}{0.000000,0.000000,0.000000}%
\pgfsetstrokecolor{currentstroke}%
\pgfsetdash{}{0pt}%
\pgfpathmoveto{\pgfqpoint{3.070292in}{0.434014in}}%
\pgfpathlineto{\pgfqpoint{3.125847in}{0.406236in}}%
\pgfpathlineto{\pgfqpoint{3.070292in}{0.378458in}}%
\pgfusepath{stroke}%
\end{pgfscope}%
\begin{pgfscope}%
\pgftext[x=3.189167in,y=0.406236in,left,]{\rmfamily\fontsize{10.000000}{12.000000}\selectfont \(\displaystyle k\)}%
\end{pgfscope}%
\begin{pgfscope}%
\pgftext[x=2.298160in,y=0.953944in,left,base]{\rmfamily\fontsize{10.000000}{12.000000}\selectfont \(\displaystyle supp~(\hat\Delta_m(-\cdot))\)}%
\end{pgfscope}%
\begin{pgfscope}%
\pgftext[x=1.111458in,y=1.501653in,left,base]{\rmfamily\fontsize{10.000000}{12.000000}\selectfont \(\displaystyle \hat\Delta_m^{*2}(-\cdot)\)}%
\end{pgfscope}%
\begin{pgfscope}%
\pgfpathrectangle{\pgfqpoint{3.302292in}{0.435000in}}{\pgfqpoint{0.063333in}{1.330000in}} %
\pgfusepath{clip}%
\pgfsetbuttcap%
\pgfsetmiterjoin%
\definecolor{currentfill}{rgb}{1.000000,1.000000,1.000000}%
\pgfsetfillcolor{currentfill}%
\pgfsetlinewidth{0.010037pt}%
\definecolor{currentstroke}{rgb}{1.000000,1.000000,1.000000}%
\pgfsetstrokecolor{currentstroke}%
\pgfsetdash{}{0pt}%
\pgfpathmoveto{\pgfqpoint{3.302292in}{0.435000in}}%
\pgfpathlineto{\pgfqpoint{3.302292in}{0.439948in}}%
\pgfpathlineto{\pgfqpoint{3.302292in}{1.701667in}}%
\pgfpathlineto{\pgfqpoint{3.333958in}{1.765000in}}%
\pgfpathlineto{\pgfqpoint{3.333958in}{1.765000in}}%
\pgfpathlineto{\pgfqpoint{3.365625in}{1.701667in}}%
\pgfpathlineto{\pgfqpoint{3.365625in}{0.439948in}}%
\pgfpathlineto{\pgfqpoint{3.365625in}{0.435000in}}%
\pgfpathclose%
\pgfusepath{stroke,fill}%
\end{pgfscope}%
\begin{pgfscope}%
\pgfsys@transformshift{3.301667in}{0.435000in}%
\pgftext[left,bottom]{\pgfimage[interpolate=true,width=0.063333in,height=1.330000in]{delta_m2-img1.png}}%
\end{pgfscope}%
\begin{pgfscope}%
\pgfsetbuttcap%
\pgfsetroundjoin%
\definecolor{currentfill}{rgb}{0.000000,0.000000,0.000000}%
\pgfsetfillcolor{currentfill}%
\pgfsetlinewidth{0.803000pt}%
\definecolor{currentstroke}{rgb}{0.000000,0.000000,0.000000}%
\pgfsetstrokecolor{currentstroke}%
\pgfsetdash{}{0pt}%
\pgfsys@defobject{currentmarker}{\pgfqpoint{0.000000in}{0.000000in}}{\pgfqpoint{0.048611in}{0.000000in}}{%
\pgfpathmoveto{\pgfqpoint{0.000000in}{0.000000in}}%
\pgfpathlineto{\pgfqpoint{0.048611in}{0.000000in}}%
\pgfusepath{stroke,fill}%
}%
\begin{pgfscope}%
\pgfsys@transformshift{3.365625in}{0.435000in}%
\pgfsys@useobject{currentmarker}{}%
\end{pgfscope}%
\end{pgfscope}%
\begin{pgfscope}%
\pgftext[x=3.462847in,y=0.387172in,left,base]{\rmfamily\fontsize{10.000000}{12.000000}\selectfont \(\displaystyle 1.5\)}%
\end{pgfscope}%
\begin{pgfscope}%
\pgfsetbuttcap%
\pgfsetroundjoin%
\definecolor{currentfill}{rgb}{0.000000,0.000000,0.000000}%
\pgfsetfillcolor{currentfill}%
\pgfsetlinewidth{0.803000pt}%
\definecolor{currentstroke}{rgb}{0.000000,0.000000,0.000000}%
\pgfsetstrokecolor{currentstroke}%
\pgfsetdash{}{0pt}%
\pgfsys@defobject{currentmarker}{\pgfqpoint{0.000000in}{0.000000in}}{\pgfqpoint{0.048611in}{0.000000in}}{%
\pgfpathmoveto{\pgfqpoint{0.000000in}{0.000000in}}%
\pgfpathlineto{\pgfqpoint{0.048611in}{0.000000in}}%
\pgfusepath{stroke,fill}%
}%
\begin{pgfscope}%
\pgfsys@transformshift{3.365625in}{0.688333in}%
\pgfsys@useobject{currentmarker}{}%
\end{pgfscope}%
\end{pgfscope}%
\begin{pgfscope}%
\pgftext[x=3.462847in,y=0.640506in,left,base]{\rmfamily\fontsize{10.000000}{12.000000}\selectfont \(\displaystyle 2.0\)}%
\end{pgfscope}%
\begin{pgfscope}%
\pgfsetbuttcap%
\pgfsetroundjoin%
\definecolor{currentfill}{rgb}{0.000000,0.000000,0.000000}%
\pgfsetfillcolor{currentfill}%
\pgfsetlinewidth{0.803000pt}%
\definecolor{currentstroke}{rgb}{0.000000,0.000000,0.000000}%
\pgfsetstrokecolor{currentstroke}%
\pgfsetdash{}{0pt}%
\pgfsys@defobject{currentmarker}{\pgfqpoint{0.000000in}{0.000000in}}{\pgfqpoint{0.048611in}{0.000000in}}{%
\pgfpathmoveto{\pgfqpoint{0.000000in}{0.000000in}}%
\pgfpathlineto{\pgfqpoint{0.048611in}{0.000000in}}%
\pgfusepath{stroke,fill}%
}%
\begin{pgfscope}%
\pgfsys@transformshift{3.365625in}{0.941667in}%
\pgfsys@useobject{currentmarker}{}%
\end{pgfscope}%
\end{pgfscope}%
\begin{pgfscope}%
\pgftext[x=3.462847in,y=0.893839in,left,base]{\rmfamily\fontsize{10.000000}{12.000000}\selectfont \(\displaystyle 2.5\)}%
\end{pgfscope}%
\begin{pgfscope}%
\pgfsetbuttcap%
\pgfsetroundjoin%
\definecolor{currentfill}{rgb}{0.000000,0.000000,0.000000}%
\pgfsetfillcolor{currentfill}%
\pgfsetlinewidth{0.803000pt}%
\definecolor{currentstroke}{rgb}{0.000000,0.000000,0.000000}%
\pgfsetstrokecolor{currentstroke}%
\pgfsetdash{}{0pt}%
\pgfsys@defobject{currentmarker}{\pgfqpoint{0.000000in}{0.000000in}}{\pgfqpoint{0.048611in}{0.000000in}}{%
\pgfpathmoveto{\pgfqpoint{0.000000in}{0.000000in}}%
\pgfpathlineto{\pgfqpoint{0.048611in}{0.000000in}}%
\pgfusepath{stroke,fill}%
}%
\begin{pgfscope}%
\pgfsys@transformshift{3.365625in}{1.195000in}%
\pgfsys@useobject{currentmarker}{}%
\end{pgfscope}%
\end{pgfscope}%
\begin{pgfscope}%
\pgftext[x=3.462847in,y=1.147172in,left,base]{\rmfamily\fontsize{10.000000}{12.000000}\selectfont \(\displaystyle 3.0\)}%
\end{pgfscope}%
\begin{pgfscope}%
\pgfsetbuttcap%
\pgfsetroundjoin%
\definecolor{currentfill}{rgb}{0.000000,0.000000,0.000000}%
\pgfsetfillcolor{currentfill}%
\pgfsetlinewidth{0.803000pt}%
\definecolor{currentstroke}{rgb}{0.000000,0.000000,0.000000}%
\pgfsetstrokecolor{currentstroke}%
\pgfsetdash{}{0pt}%
\pgfsys@defobject{currentmarker}{\pgfqpoint{0.000000in}{0.000000in}}{\pgfqpoint{0.048611in}{0.000000in}}{%
\pgfpathmoveto{\pgfqpoint{0.000000in}{0.000000in}}%
\pgfpathlineto{\pgfqpoint{0.048611in}{0.000000in}}%
\pgfusepath{stroke,fill}%
}%
\begin{pgfscope}%
\pgfsys@transformshift{3.365625in}{1.448333in}%
\pgfsys@useobject{currentmarker}{}%
\end{pgfscope}%
\end{pgfscope}%
\begin{pgfscope}%
\pgftext[x=3.462847in,y=1.400506in,left,base]{\rmfamily\fontsize{10.000000}{12.000000}\selectfont \(\displaystyle 3.5\)}%
\end{pgfscope}%
\begin{pgfscope}%
\pgfsetbuttcap%
\pgfsetroundjoin%
\definecolor{currentfill}{rgb}{0.000000,0.000000,0.000000}%
\pgfsetfillcolor{currentfill}%
\pgfsetlinewidth{0.803000pt}%
\definecolor{currentstroke}{rgb}{0.000000,0.000000,0.000000}%
\pgfsetstrokecolor{currentstroke}%
\pgfsetdash{}{0pt}%
\pgfsys@defobject{currentmarker}{\pgfqpoint{0.000000in}{0.000000in}}{\pgfqpoint{0.048611in}{0.000000in}}{%
\pgfpathmoveto{\pgfqpoint{0.000000in}{0.000000in}}%
\pgfpathlineto{\pgfqpoint{0.048611in}{0.000000in}}%
\pgfusepath{stroke,fill}%
}%
\begin{pgfscope}%
\pgfsys@transformshift{3.365625in}{1.701667in}%
\pgfsys@useobject{currentmarker}{}%
\end{pgfscope}%
\end{pgfscope}%
\begin{pgfscope}%
\pgftext[x=3.462847in,y=1.653839in,left,base]{\rmfamily\fontsize{10.000000}{12.000000}\selectfont \(\displaystyle 4.0\)}%
\end{pgfscope}%
\begin{pgfscope}%
\pgfsetbuttcap%
\pgfsetmiterjoin%
\pgfsetlinewidth{0.501875pt}%
\definecolor{currentstroke}{rgb}{0.000000,0.000000,0.000000}%
\pgfsetstrokecolor{currentstroke}%
\pgfsetdash{}{0pt}%
\pgfpathmoveto{\pgfqpoint{3.302292in}{0.435000in}}%
\pgfpathlineto{\pgfqpoint{3.302292in}{0.439948in}}%
\pgfpathlineto{\pgfqpoint{3.302292in}{1.701667in}}%
\pgfpathlineto{\pgfqpoint{3.333958in}{1.765000in}}%
\pgfpathlineto{\pgfqpoint{3.333958in}{1.765000in}}%
\pgfpathlineto{\pgfqpoint{3.365625in}{1.701667in}}%
\pgfpathlineto{\pgfqpoint{3.365625in}{0.439948in}}%
\pgfpathlineto{\pgfqpoint{3.365625in}{0.435000in}}%
\pgfpathclose%
\pgfusepath{stroke}%
\end{pgfscope}%
\end{pgfpicture}%
\makeatother%
\endgroup%
} %
        \caption{Plot von $\hat{\Delta_m}^{*2}$ und $\hat{\Delta_m}$.
        Je weiter wir uns von der 2m-Massenschale wegbewegen, desto konstanter
        wird $\hat{\Delta_m}^{*2}$ und ist singulär genau auf der $2m$-Massenschale}
        \label{fig:delta_2m}
    \end{minipage}\hfill
    \begin{minipage}{0.45\textwidth}
        \centering
        \resizebox{\textwidth}{!}{%% Creator: Matplotlib, PGF backend
%%
%% To include the figure in your LaTeX document, write
%%   \input{<filename>.pgf}
%%
%% Make sure the required packages are loaded in your preamble
%%   \usepackage{pgf}
%%
%% Figures using additional raster images can only be included by \input if
%% they are in the same directory as the main LaTeX file. For loading figures
%% from other directories you can use the `import` package
%%   \usepackage{import}
%% and then include the figures with
%%   \import{<path to file>}{<filename>.pgf}
%%
%% Matplotlib used the following preamble
%%   \usepackage[utf8x]{inputenc}
%%   \usepackage[T1]{fontenc}
%%   \usepackage{amssymb}
%%
\begingroup%
\makeatletter%
\begin{pgfpicture}%
\pgfpathrectangle{\pgfpointorigin}{\pgfqpoint{4.000000in}{2.200000in}}%
\pgfusepath{use as bounding box, clip}%
\begin{pgfscope}%
\pgfsetbuttcap%
\pgfsetmiterjoin%
\definecolor{currentfill}{rgb}{1.000000,1.000000,1.000000}%
\pgfsetfillcolor{currentfill}%
\pgfsetlinewidth{0.000000pt}%
\definecolor{currentstroke}{rgb}{1.000000,1.000000,1.000000}%
\pgfsetstrokecolor{currentstroke}%
\pgfsetdash{}{0pt}%
\pgfpathmoveto{\pgfqpoint{0.000000in}{0.000000in}}%
\pgfpathlineto{\pgfqpoint{4.000000in}{0.000000in}}%
\pgfpathlineto{\pgfqpoint{4.000000in}{2.200000in}}%
\pgfpathlineto{\pgfqpoint{0.000000in}{2.200000in}}%
\pgfpathclose%
\pgfusepath{fill}%
\end{pgfscope}%
\begin{pgfscope}%
\pgfsetbuttcap%
\pgfsetmiterjoin%
\definecolor{currentfill}{rgb}{1.000000,1.000000,1.000000}%
\pgfsetfillcolor{currentfill}%
\pgfsetlinewidth{0.000000pt}%
\definecolor{currentstroke}{rgb}{0.000000,0.000000,0.000000}%
\pgfsetstrokecolor{currentstroke}%
\pgfsetstrokeopacity{0.000000}%
\pgfsetdash{}{0pt}%
\pgfpathmoveto{\pgfqpoint{0.198611in}{0.198611in}}%
\pgfpathlineto{\pgfqpoint{3.801389in}{0.198611in}}%
\pgfpathlineto{\pgfqpoint{3.801389in}{2.001389in}}%
\pgfpathlineto{\pgfqpoint{0.198611in}{2.001389in}}%
\pgfpathclose%
\pgfusepath{fill}%
\end{pgfscope}%
\begin{pgfscope}%
\pgfpathrectangle{\pgfqpoint{0.198611in}{0.198611in}}{\pgfqpoint{3.602778in}{1.802778in}} %
\pgfusepath{clip}%
\pgfsetbuttcap%
\pgfsetroundjoin%
\pgfsetlinewidth{0.501875pt}%
\definecolor{currentstroke}{rgb}{0.501961,0.501961,0.501961}%
\pgfsetstrokecolor{currentstroke}%
\pgfsetdash{{1.850000pt}{0.800000pt}}{0.000000pt}%
\pgfpathmoveto{\pgfqpoint{0.767471in}{0.184722in}}%
\pgfpathlineto{\pgfqpoint{0.767471in}{2.001389in}}%
\pgfusepath{stroke}%
\end{pgfscope}%
\begin{pgfscope}%
\pgfpathrectangle{\pgfqpoint{0.198611in}{0.198611in}}{\pgfqpoint{3.602778in}{1.802778in}} %
\pgfusepath{clip}%
\pgfsetrectcap%
\pgfsetroundjoin%
\pgfsetlinewidth{1.003750pt}%
\definecolor{currentstroke}{rgb}{0.894118,0.101961,0.109804}%
\pgfsetstrokecolor{currentstroke}%
\pgfsetdash{}{0pt}%
\pgfpathmoveto{\pgfqpoint{0.770354in}{2.015278in}}%
\pgfpathlineto{\pgfqpoint{0.776582in}{1.099358in}}%
\pgfpathlineto{\pgfqpoint{0.784174in}{0.905618in}}%
\pgfpathlineto{\pgfqpoint{0.791766in}{0.815405in}}%
\pgfpathlineto{\pgfqpoint{0.799359in}{0.761699in}}%
\pgfpathlineto{\pgfqpoint{0.806951in}{0.725675in}}%
\pgfpathlineto{\pgfqpoint{0.814544in}{0.699731in}}%
\pgfpathlineto{\pgfqpoint{0.822136in}{0.680141in}}%
\pgfpathlineto{\pgfqpoint{0.837321in}{0.652591in}}%
\pgfpathlineto{\pgfqpoint{0.852506in}{0.634292in}}%
\pgfpathlineto{\pgfqpoint{0.867690in}{0.621422in}}%
\pgfpathlineto{\pgfqpoint{0.882875in}{0.612017in}}%
\pgfpathlineto{\pgfqpoint{0.905652in}{0.602089in}}%
\pgfpathlineto{\pgfqpoint{0.928429in}{0.595392in}}%
\pgfpathlineto{\pgfqpoint{0.958799in}{0.589572in}}%
\pgfpathlineto{\pgfqpoint{0.996761in}{0.585335in}}%
\pgfpathlineto{\pgfqpoint{1.049908in}{0.582586in}}%
\pgfpathlineto{\pgfqpoint{1.125831in}{0.581888in}}%
\pgfpathlineto{\pgfqpoint{1.254902in}{0.584131in}}%
\pgfpathlineto{\pgfqpoint{2.112842in}{0.604087in}}%
\pgfpathlineto{\pgfqpoint{2.606347in}{0.610424in}}%
\pgfpathlineto{\pgfqpoint{3.266885in}{0.615635in}}%
\pgfpathlineto{\pgfqpoint{3.815278in}{0.618380in}}%
\pgfpathlineto{\pgfqpoint{3.815278in}{0.618380in}}%
\pgfusepath{stroke}%
\end{pgfscope}%
\begin{pgfscope}%
\pgfpathrectangle{\pgfqpoint{0.198611in}{0.198611in}}{\pgfqpoint{3.602778in}{1.802778in}} %
\pgfusepath{clip}%
\pgfsetrectcap%
\pgfsetroundjoin%
\pgfsetlinewidth{1.003750pt}%
\definecolor{currentstroke}{rgb}{0.894118,0.101961,0.109804}%
\pgfsetstrokecolor{currentstroke}%
\pgfsetdash{}{0pt}%
\pgfpathmoveto{\pgfqpoint{0.184722in}{0.284458in}}%
\pgfpathlineto{\pgfqpoint{0.430369in}{0.284458in}}%
\pgfpathlineto{\pgfqpoint{0.767471in}{0.284458in}}%
\pgfusepath{stroke}%
\end{pgfscope}%
\begin{pgfscope}%
\pgfpathrectangle{\pgfqpoint{0.198611in}{0.198611in}}{\pgfqpoint{3.602778in}{1.802778in}} %
\pgfusepath{clip}%
\pgfsetbuttcap%
\pgfsetroundjoin%
\pgfsetlinewidth{0.501875pt}%
\definecolor{currentstroke}{rgb}{0.501961,0.501961,0.501961}%
\pgfsetstrokecolor{currentstroke}%
\pgfsetdash{{1.850000pt}{0.800000pt}}{0.000000pt}%
\pgfpathmoveto{\pgfqpoint{0.184722in}{0.627844in}}%
\pgfpathlineto{\pgfqpoint{3.815278in}{0.627844in}}%
\pgfusepath{stroke}%
\end{pgfscope}%
\begin{pgfscope}%
\pgfsetrectcap%
\pgfsetmiterjoin%
\pgfsetlinewidth{0.501875pt}%
\definecolor{currentstroke}{rgb}{0.000000,0.000000,0.000000}%
\pgfsetstrokecolor{currentstroke}%
\pgfsetdash{}{0pt}%
\pgfpathmoveto{\pgfqpoint{0.388231in}{0.198611in}}%
\pgfpathlineto{\pgfqpoint{0.388231in}{2.001389in}}%
\pgfusepath{stroke}%
\end{pgfscope}%
\begin{pgfscope}%
\pgfsetrectcap%
\pgfsetmiterjoin%
\pgfsetlinewidth{0.501875pt}%
\definecolor{currentstroke}{rgb}{0.000000,0.000000,0.000000}%
\pgfsetstrokecolor{currentstroke}%
\pgfsetdash{}{0pt}%
\pgfpathmoveto{\pgfqpoint{0.198611in}{0.284458in}}%
\pgfpathlineto{\pgfqpoint{3.801389in}{0.284458in}}%
\pgfusepath{stroke}%
\end{pgfscope}%
\begin{pgfscope}%
\pgftext[x=0.843319in,y=0.335966in,left,base]{\rmfamily\fontsize{10.000000}{12.000000}\selectfont \(\displaystyle \omega = 2 m\)}%
\end{pgfscope}%
\begin{pgfscope}%
\pgftext[x=0.160687in,y=0.645013in,left,base]{\rmfamily\fontsize{10.000000}{12.000000}\selectfont \(\displaystyle \hat \Delta^{*2} = 2\)}%
\end{pgfscope}%
\begin{pgfscope}%
\pgftext[x=0.843319in,y=0.971230in,left,base]{\rmfamily\fontsize{10.000000}{12.000000}\selectfont \(\displaystyle \approx \frac{1}{\sqrt{\omega}}\)}%
\end{pgfscope}%
\begin{pgfscope}%
\pgftext[x=3.042909in,y=0.645013in,left,base]{\rmfamily\fontsize{10.000000}{12.000000}\selectfont \(\displaystyle \approx 2\)}%
\end{pgfscope}%
\begin{pgfscope}%
\pgfsetroundcap%
\pgfsetroundjoin%
\pgfsetlinewidth{0.501875pt}%
\definecolor{currentstroke}{rgb}{0.000000,0.000000,0.000000}%
\pgfsetstrokecolor{currentstroke}%
\pgfsetdash{}{0pt}%
\pgfpathmoveto{\pgfqpoint{0.388231in}{2.007506in}}%
\pgfpathquadraticcurveto{\pgfqpoint{0.388231in}{2.008330in}}{\pgfqpoint{0.388231in}{2.001389in}}%
\pgfusepath{stroke}%
\end{pgfscope}%
\begin{pgfscope}%
\pgfsetroundcap%
\pgfsetroundjoin%
\pgfsetlinewidth{0.501875pt}%
\definecolor{currentstroke}{rgb}{0.000000,0.000000,0.000000}%
\pgfsetstrokecolor{currentstroke}%
\pgfsetdash{}{0pt}%
\pgfpathmoveto{\pgfqpoint{0.360453in}{1.951951in}}%
\pgfpathlineto{\pgfqpoint{0.388231in}{2.007506in}}%
\pgfpathlineto{\pgfqpoint{0.416009in}{1.951951in}}%
\pgfusepath{stroke}%
\end{pgfscope}%
\begin{pgfscope}%
\pgftext[x=0.388231in,y=2.070833in,,bottom]{\rmfamily\fontsize{10.000000}{12.000000}\selectfont \(\displaystyle \hat\Delta^{*2} ~(\omega, 0)\)}%
\end{pgfscope}%
\begin{pgfscope}%
\pgfsetroundcap%
\pgfsetroundjoin%
\pgfsetlinewidth{0.501875pt}%
\definecolor{currentstroke}{rgb}{0.000000,0.000000,0.000000}%
\pgfsetstrokecolor{currentstroke}%
\pgfsetdash{}{0pt}%
\pgfpathmoveto{\pgfqpoint{3.807500in}{0.284458in}}%
\pgfpathquadraticcurveto{\pgfqpoint{3.808327in}{0.284458in}}{\pgfqpoint{3.801389in}{0.284458in}}%
\pgfusepath{stroke}%
\end{pgfscope}%
\begin{pgfscope}%
\pgfsetroundcap%
\pgfsetroundjoin%
\pgfsetlinewidth{0.501875pt}%
\definecolor{currentstroke}{rgb}{0.000000,0.000000,0.000000}%
\pgfsetstrokecolor{currentstroke}%
\pgfsetdash{}{0pt}%
\pgfpathmoveto{\pgfqpoint{3.751945in}{0.312235in}}%
\pgfpathlineto{\pgfqpoint{3.807500in}{0.284458in}}%
\pgfpathlineto{\pgfqpoint{3.751945in}{0.256680in}}%
\pgfusepath{stroke}%
\end{pgfscope}%
\begin{pgfscope}%
\pgftext[x=3.870833in,y=0.284458in,left,]{\rmfamily\fontsize{10.000000}{12.000000}\selectfont \(\displaystyle \omega\)}%
\end{pgfscope}%
\end{pgfpicture}%
\makeatother%
\endgroup%
}
        \caption{Plot von $\left.\hat{\Delta_m}^{*2}\right|_{k=0}$ um das asymptotische Verhalten für $\omega \rightarrow 0$ und $\omega \rightarrow \infty$ zu verdeutlichen}
        \label{fig:delta_2m_k0}
    \end{minipage}
\end{figure}

\subsection{\dots und nun zur Wellenfrontmenge} % (fold)
\label{sec:dots_und_nun_zur_wellenfrontmenge}

Mit diesen Ausdrücken für $\rwhat{\Delta_m}^{*2}$ können wir uns nun der Wellenfrontmenge widmen.

\subsubsection*{\texorpdfstring{Fall $|s|>1$}{Fall s>1}}
Genau wie im Fall $s \neq 1$ bei der massiven Zweipunktfunktion (vgl. \ref{sec:die_wellenfrontmenge_von_delta_m}) ist hier nichts zu tun, da für $a$ klein genug wieder

\begin{align*}
    supp (\hat\psi_{ast}) \cap supp (\rwhat{\Delta_m}^{*2}) = \varnothing
    \Rightarrow
    \left\langle\hat\psi_{ast}, \rwhat{\Delta_m}^{*2}\right\rangle = 0
\end{align*}

gilt.

\subsubsection*{\texorpdfstring{Fall $s<1$}{Fall s<1}}
Hier bedienen wir uns direkt bei \eqref{Eq:substitution2} und schreiben

\begin{dmath*}
    \left\langle \rwhat{\psi}_{ast},\rwhat{\Delta_m}^{*2}\right\rangle
    =
    2 a^{-\frac{3}{4}} \int \frac{
    \hat\psi_1(\omega)~\hat\psi_2(k) \left(
        \omega^2 a^{-2} - \omega^2\left(a^{-\frac{1}{2}}k+sa^{-1}\right)^2
            -3m^2
            \right)
    }
    {
        \sqrt{\omega^2 a^{-2}-\omega^2\left(a^{-\frac{1}{2}} k +s^{-1}\right)^2}
        \sqrt{\omega^2 a^{-2}-\omega^2\left(
            a^{-\frac{1}{2}} k +sa^{-1}\right)^2
            -4m^2
             }
    }
    \cdot
    \Theta \left(\omega^2-k^2-4m^2\right)
      e^{-i \omega \left(\frac{t'-sx'}{a}+k \frac{x'}{\sqrt{a}}\right)}
    \omega \d \omega \d k
%    =
    \kern -4em \underset{\Delta s := 1-s^2 > 0}{=} \kern -1.5em
     2 a^{-\frac{3}{4}} \int \frac{
        \hat\psi_1(\omega)~ \hat\psi_2(k) \cancel{a^{-2}} \left(
        \omega^2 \left(\Delta s - 2 a^{\frac{1}{2}} k s - ak^2
                \right) - 3a^2m^2
        \right)
     e^{\dots} \Theta(\dots) \cancel{\omega}
     }
     {
        \cancel{\omega} \cancel{a^{-2}}
        \sqrt{\Delta s -2a^{\frac{1}{2}}ks - ak^2}
            \sqrt{\Delta s \omega^2 -2a^{\frac{1}{2}} \omega^2 k s
                    - a\omega^2k^2-4 a^2 m^2}
     }
     \d \omega \d k
\end{dmath*}

Für hinreichend kleine $a$ können wir den Integrand nun majorisieren

\begin{dmath*}
    \left|
    2 \frac{
        \hat\psi_1(\omega) ~\hat\psi_2(k) \omega^2 \Delta s \Theta(\dots)}
    {
        \sqrt{\Delta s} \sqrt{\Delta s \omega^2}
    }
    \right|
    \geq
    \left|
    \frac{
        \hat\psi_1(\omega)~ \hat\psi_2(k) \left(
        \omega^2 \left(\Delta s - 2 a^{\frac{1}{2}} k s - ak^2
                \right) - 3a^2m^2
        \right)
         \Theta(\dots)
     }
     {
        \sqrt{\Delta s -2a^{\frac{1}{2}}ks - ak^2}
            \sqrt{\Delta s \omega^2 -2a^{\frac{1}{2}} \omega^2 k s
                    - a\omega^2k^2-4 a^2 m^2}
     }
    \right|
\end{dmath*}

und dürfen also Lebesgue verwenden und schreiben

\begin{dmath*}
    \lim_{a \rightarrow 0} \int \dots ~\d \omega \d k
    = \int \lim_{a \rightarrow 0} \dots ~\d \omega \d k
    = 2 a^{-\frac{3}{4}} \int
    \frac{\hat\psi_1(\omega) ~\hat\psi_2(k) ~\omega^{\cancel{2}}
        ~\cancel{\Delta s} ~\Theta(\dots)
    }
    {
        \cancel{\sqrt{\Delta s}}\cancel{\sqrt{\Delta s}}\cancel{\omega}
    }
    e^{-i \omega \left(\frac{t'-sx'}{a} + k \frac{x'}{\sqrt{a}}\right)}
    \d \omega \d k
    \underset{k\rightarrow \frac{k}{\omega}}{=}
    2 a^{-\frac{3}{4}} \int
    \hat\psi_1(\omega) ~\hat\psi_2\left(\tfrac{k}{\omega}\right)
    e^{-i\omega \frac{t'-sx'}{a} + ik \frac{x'}{\sqrt{a}}}
    \d \omega \d k \\
    =
    2 a^{-\frac{3}{4}} \psi \left(\frac{t'-sx'}{a}, \frac{x'}{a}\right)
\end{dmath*}

Und da Shearlets nach Proposition \ref{prop:shearlets_decay_rapidly} schnell abfallen erhalten wir schließlich

\begin{dmath*}
    \left\langle \psi_{ast},\Delta_m^{2}\right\rangle
    \sim O(a^k) ~\forall k \in \mathbb{N} \condition{falls $(t',x') \neq 0$}
    \sim O(a^{-\frac{3}{4}}) \condition{falls (t',x') = 0}
\end{dmath*}
% section dots_und_nun_zur_wellenfrontmenge (end)









% section die_wellenfrontmenge_von_delta_m_2_ (end)
