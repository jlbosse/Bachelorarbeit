%%%%%%%%%%%%%%%%%%%%%%%%%%%%%%%%%%%%%%%%%%%%%%%%%%%%%%%%%%%%%%%%%%%%%%%%%%%%%%%
% % Berechnen der Wellenfrontmenge von Delta_m
%%%%%%%%%%%%%%%%%%%%%%%%%%%%%%%%%%%%%%%%%%%%%%%%%%%%%%%%%%%%%%%%%%%%%%%%%%%%%%%
%
\section{\texorpdfstring{Die Wellenfrontmenge von $\Delta_m^2$}
    {Wellenfrontmenge von delta m}} % (fold)
\label{sec:die_wellenfrontmenge_von_delta_m_2_}

\subsection{\texorpdfstring{$\hat\Delta^{\ast 2}$ berechnen}
    {delta-ast-hat berechnen}}
Bevor wir die Wellenfrontmenge von $\Delta_m^2$ berechnen können, benötigen wir einen Ausdruck dafür, oder besser noch einen für die Fouriertransformierte davon. Gemäß dem Faltungssatz gilt $\rwhat{\Delta_m^2} = \rwhat \Delta_m * \rwhat \Delta_m = \rwhat\Delta_m^{*2}$. Wir müssen also die Faltung von $\rwhat \Delta_m$ mit sich selber ausrechnen.

\begin{figure}
    \centering
    \begin{minipage}{0.5\textwidth}
        \centering
        \resizebox{\textwidth}{!}{%% Creator: Matplotlib, PGF backend
%%
%% To include the figure in your LaTeX document, write
%%   \input{<filename>.pgf}
%%
%% Make sure the required packages are loaded in your preamble
%%   \usepackage{pgf}
%%
%% Figures using additional raster images can only be included by \input if
%% they are in the same directory as the main LaTeX file. For loading figures
%% from other directories you can use the `import` package
%%   \usepackage{import}
%% and then include the figures with
%%   \import{<path to file>}{<filename>.pgf}
%%
%% Matplotlib used the following preamble
%%   \usepackage[utf8x]{inputenc}
%%   \usepackage[T1]{fontenc}
%%   \usepackage{amssymb}
%%
\begingroup%
\makeatletter%
\begin{pgfpicture}%
\pgfpathrectangle{\pgfpointorigin}{\pgfqpoint{4.000000in}{3.000000in}}%
\pgfusepath{use as bounding box, clip}%
\begin{pgfscope}%
\pgfsetbuttcap%
\pgfsetmiterjoin%
\definecolor{currentfill}{rgb}{1.000000,1.000000,1.000000}%
\pgfsetfillcolor{currentfill}%
\pgfsetlinewidth{0.000000pt}%
\definecolor{currentstroke}{rgb}{1.000000,1.000000,1.000000}%
\pgfsetstrokecolor{currentstroke}%
\pgfsetdash{}{0pt}%
\pgfpathmoveto{\pgfqpoint{0.000000in}{0.000000in}}%
\pgfpathlineto{\pgfqpoint{4.000000in}{0.000000in}}%
\pgfpathlineto{\pgfqpoint{4.000000in}{3.000000in}}%
\pgfpathlineto{\pgfqpoint{0.000000in}{3.000000in}}%
\pgfpathclose%
\pgfusepath{fill}%
\end{pgfscope}%
\begin{pgfscope}%
\pgfsetbuttcap%
\pgfsetmiterjoin%
\definecolor{currentfill}{rgb}{1.000000,1.000000,1.000000}%
\pgfsetfillcolor{currentfill}%
\pgfsetlinewidth{0.000000pt}%
\definecolor{currentstroke}{rgb}{0.000000,0.000000,0.000000}%
\pgfsetstrokecolor{currentstroke}%
\pgfsetstrokeopacity{0.000000}%
\pgfsetdash{}{0pt}%
\pgfpathmoveto{\pgfqpoint{0.198611in}{0.198611in}}%
\pgfpathlineto{\pgfqpoint{3.801389in}{0.198611in}}%
\pgfpathlineto{\pgfqpoint{3.801389in}{2.801389in}}%
\pgfpathlineto{\pgfqpoint{0.198611in}{2.801389in}}%
\pgfpathclose%
\pgfusepath{fill}%
\end{pgfscope}%
\begin{pgfscope}%
\pgfsetbuttcap%
\pgfsetroundjoin%
\definecolor{currentfill}{rgb}{0.000000,0.000000,0.000000}%
\pgfsetfillcolor{currentfill}%
\pgfsetlinewidth{0.803000pt}%
\definecolor{currentstroke}{rgb}{0.000000,0.000000,0.000000}%
\pgfsetstrokecolor{currentstroke}%
\pgfsetdash{}{0pt}%
\pgfsys@defobject{currentmarker}{\pgfqpoint{0.000000in}{-0.048611in}}{\pgfqpoint{0.000000in}{0.000000in}}{%
\pgfpathmoveto{\pgfqpoint{0.000000in}{0.000000in}}%
\pgfpathlineto{\pgfqpoint{0.000000in}{-0.048611in}}%
\pgfusepath{stroke,fill}%
}%
\begin{pgfscope}%
\pgfsys@transformshift{1.219976in}{0.632407in}%
\pgfsys@useobject{currentmarker}{}%
\end{pgfscope}%
\end{pgfscope}%
\begin{pgfscope}%
\pgftext[x=1.219976in,y=0.535185in,,top]{\rmfamily\fontsize{10.000000}{12.000000}\selectfont \(\displaystyle k^\prime_{0-} = -\sqrt{\left(\frac{\omega}{2}\right)^2-m^2}\)}%
\end{pgfscope}%
\begin{pgfscope}%
\pgfsetbuttcap%
\pgfsetroundjoin%
\definecolor{currentfill}{rgb}{0.000000,0.000000,0.000000}%
\pgfsetfillcolor{currentfill}%
\pgfsetlinewidth{0.803000pt}%
\definecolor{currentstroke}{rgb}{0.000000,0.000000,0.000000}%
\pgfsetstrokecolor{currentstroke}%
\pgfsetdash{}{0pt}%
\pgfsys@defobject{currentmarker}{\pgfqpoint{0.000000in}{-0.048611in}}{\pgfqpoint{0.000000in}{0.000000in}}{%
\pgfpathmoveto{\pgfqpoint{0.000000in}{0.000000in}}%
\pgfpathlineto{\pgfqpoint{0.000000in}{-0.048611in}}%
\pgfusepath{stroke,fill}%
}%
\begin{pgfscope}%
\pgfsys@transformshift{2.780024in}{0.632407in}%
\pgfsys@useobject{currentmarker}{}%
\end{pgfscope}%
\end{pgfscope}%
\begin{pgfscope}%
\pgftext[x=2.780024in,y=0.535185in,,top]{\rmfamily\fontsize{10.000000}{12.000000}\selectfont \(\displaystyle k^\prime_{0+} = \sqrt{\left(\frac{\omega}{2}\right)^2-m^2}\)}%
\end{pgfscope}%
\begin{pgfscope}%
\pgfsetbuttcap%
\pgfsetroundjoin%
\definecolor{currentfill}{rgb}{0.000000,0.000000,0.000000}%
\pgfsetfillcolor{currentfill}%
\pgfsetlinewidth{0.803000pt}%
\definecolor{currentstroke}{rgb}{0.000000,0.000000,0.000000}%
\pgfsetstrokecolor{currentstroke}%
\pgfsetdash{}{0pt}%
\pgfsys@defobject{currentmarker}{\pgfqpoint{-0.048611in}{0.000000in}}{\pgfqpoint{0.000000in}{0.000000in}}{%
\pgfpathmoveto{\pgfqpoint{0.000000in}{0.000000in}}%
\pgfpathlineto{\pgfqpoint{-0.048611in}{0.000000in}}%
\pgfusepath{stroke,fill}%
}%
\begin{pgfscope}%
\pgfsys@transformshift{2.000000in}{1.066204in}%
\pgfsys@useobject{currentmarker}{}%
\end{pgfscope}%
\end{pgfscope}%
\begin{pgfscope}%
\pgftext[x=1.780831in,y=1.018376in,left,base]{\rmfamily\fontsize{10.000000}{12.000000}\selectfont \(\displaystyle m\)}%
\end{pgfscope}%
\begin{pgfscope}%
\pgfsetbuttcap%
\pgfsetroundjoin%
\definecolor{currentfill}{rgb}{0.000000,0.000000,0.000000}%
\pgfsetfillcolor{currentfill}%
\pgfsetlinewidth{0.803000pt}%
\definecolor{currentstroke}{rgb}{0.000000,0.000000,0.000000}%
\pgfsetstrokecolor{currentstroke}%
\pgfsetdash{}{0pt}%
\pgfsys@defobject{currentmarker}{\pgfqpoint{-0.048611in}{0.000000in}}{\pgfqpoint{0.000000in}{0.000000in}}{%
\pgfpathmoveto{\pgfqpoint{0.000000in}{0.000000in}}%
\pgfpathlineto{\pgfqpoint{-0.048611in}{0.000000in}}%
\pgfusepath{stroke,fill}%
}%
\begin{pgfscope}%
\pgfsys@transformshift{2.000000in}{1.500000in}%
\pgfsys@useobject{currentmarker}{}%
\end{pgfscope}%
\end{pgfscope}%
\begin{pgfscope}%
\pgftext[x=1.338511in,y=1.444321in,left,base]{\rmfamily\fontsize{10.000000}{12.000000}\selectfont \(\displaystyle \omega/2 = \omega^\prime_0\)}%
\end{pgfscope}%
\begin{pgfscope}%
\pgfsetbuttcap%
\pgfsetroundjoin%
\definecolor{currentfill}{rgb}{0.000000,0.000000,0.000000}%
\pgfsetfillcolor{currentfill}%
\pgfsetlinewidth{0.803000pt}%
\definecolor{currentstroke}{rgb}{0.000000,0.000000,0.000000}%
\pgfsetstrokecolor{currentstroke}%
\pgfsetdash{}{0pt}%
\pgfsys@defobject{currentmarker}{\pgfqpoint{-0.048611in}{0.000000in}}{\pgfqpoint{0.000000in}{0.000000in}}{%
\pgfpathmoveto{\pgfqpoint{0.000000in}{0.000000in}}%
\pgfpathlineto{\pgfqpoint{-0.048611in}{0.000000in}}%
\pgfusepath{stroke,fill}%
}%
\begin{pgfscope}%
\pgfsys@transformshift{2.000000in}{1.933796in}%
\pgfsys@useobject{currentmarker}{}%
\end{pgfscope}%
\end{pgfscope}%
\begin{pgfscope}%
\pgftext[x=1.519645in,y=1.885969in,left,base]{\rmfamily\fontsize{10.000000}{12.000000}\selectfont \(\displaystyle \omega - m\)}%
\end{pgfscope}%
\begin{pgfscope}%
\pgfsetbuttcap%
\pgfsetroundjoin%
\definecolor{currentfill}{rgb}{0.000000,0.000000,0.000000}%
\pgfsetfillcolor{currentfill}%
\pgfsetlinewidth{0.803000pt}%
\definecolor{currentstroke}{rgb}{0.000000,0.000000,0.000000}%
\pgfsetstrokecolor{currentstroke}%
\pgfsetdash{}{0pt}%
\pgfsys@defobject{currentmarker}{\pgfqpoint{-0.048611in}{0.000000in}}{\pgfqpoint{0.000000in}{0.000000in}}{%
\pgfpathmoveto{\pgfqpoint{0.000000in}{0.000000in}}%
\pgfpathlineto{\pgfqpoint{-0.048611in}{0.000000in}}%
\pgfusepath{stroke,fill}%
}%
\begin{pgfscope}%
\pgfsys@transformshift{2.000000in}{2.367593in}%
\pgfsys@useobject{currentmarker}{}%
\end{pgfscope}%
\end{pgfscope}%
\begin{pgfscope}%
\pgftext[x=1.811343in,y=2.319765in,left,base]{\rmfamily\fontsize{10.000000}{12.000000}\selectfont \(\displaystyle \omega\)}%
\end{pgfscope}%
\begin{pgfscope}%
\pgfpathrectangle{\pgfqpoint{0.198611in}{0.198611in}}{\pgfqpoint{3.602778in}{2.602778in}}%
\pgfusepath{clip}%
\pgfsetbuttcap%
\pgfsetroundjoin%
\pgfsetlinewidth{0.501875pt}%
\definecolor{currentstroke}{rgb}{0.501961,0.501961,0.501961}%
\pgfsetstrokecolor{currentstroke}%
\pgfsetdash{{1.850000pt}{0.800000pt}}{0.000000pt}%
\pgfpathmoveto{\pgfqpoint{1.219976in}{0.632407in}}%
\pgfpathlineto{\pgfqpoint{1.219976in}{1.500000in}}%
\pgfusepath{stroke}%
\end{pgfscope}%
\begin{pgfscope}%
\pgfpathrectangle{\pgfqpoint{0.198611in}{0.198611in}}{\pgfqpoint{3.602778in}{2.602778in}}%
\pgfusepath{clip}%
\pgfsetbuttcap%
\pgfsetroundjoin%
\pgfsetlinewidth{0.501875pt}%
\definecolor{currentstroke}{rgb}{0.501961,0.501961,0.501961}%
\pgfsetstrokecolor{currentstroke}%
\pgfsetdash{{1.850000pt}{0.800000pt}}{0.000000pt}%
\pgfpathmoveto{\pgfqpoint{2.780024in}{0.632407in}}%
\pgfpathlineto{\pgfqpoint{2.780024in}{1.500000in}}%
\pgfusepath{stroke}%
\end{pgfscope}%
\begin{pgfscope}%
\pgfpathrectangle{\pgfqpoint{0.198611in}{0.198611in}}{\pgfqpoint{3.602778in}{2.602778in}}%
\pgfusepath{clip}%
\pgfsetrectcap%
\pgfsetroundjoin%
\pgfsetlinewidth{0.501875pt}%
\definecolor{currentstroke}{rgb}{0.894118,0.101961,0.109804}%
\pgfsetstrokecolor{currentstroke}%
\pgfsetdash{}{0pt}%
\pgfpathmoveto{\pgfqpoint{0.184722in}{0.566020in}}%
\pgfpathlineto{\pgfqpoint{0.524413in}{0.881513in}}%
\pgfpathlineto{\pgfqpoint{0.768088in}{1.104150in}}%
\pgfpathlineto{\pgfqpoint{0.957613in}{1.273814in}}%
\pgfpathlineto{\pgfqpoint{1.120063in}{1.415436in}}%
\pgfpathlineto{\pgfqpoint{1.255438in}{1.529408in}}%
\pgfpathlineto{\pgfqpoint{1.363738in}{1.616727in}}%
\pgfpathlineto{\pgfqpoint{1.444963in}{1.679103in}}%
\pgfpathlineto{\pgfqpoint{1.526188in}{1.737927in}}%
\pgfpathlineto{\pgfqpoint{1.607413in}{1.792107in}}%
\pgfpathlineto{\pgfqpoint{1.661563in}{1.824956in}}%
\pgfpathlineto{\pgfqpoint{1.715713in}{1.854594in}}%
\pgfpathlineto{\pgfqpoint{1.769863in}{1.880437in}}%
\pgfpathlineto{\pgfqpoint{1.824013in}{1.901850in}}%
\pgfpathlineto{\pgfqpoint{1.878163in}{1.918201in}}%
\pgfpathlineto{\pgfqpoint{1.932313in}{1.928924in}}%
\pgfpathlineto{\pgfqpoint{1.986463in}{1.933600in}}%
\pgfpathlineto{\pgfqpoint{2.013537in}{1.933600in}}%
\pgfpathlineto{\pgfqpoint{2.040612in}{1.932036in}}%
\pgfpathlineto{\pgfqpoint{2.094762in}{1.924297in}}%
\pgfpathlineto{\pgfqpoint{2.148912in}{1.910696in}}%
\pgfpathlineto{\pgfqpoint{2.203062in}{1.891737in}}%
\pgfpathlineto{\pgfqpoint{2.257212in}{1.868029in}}%
\pgfpathlineto{\pgfqpoint{2.311362in}{1.840212in}}%
\pgfpathlineto{\pgfqpoint{2.365512in}{1.808899in}}%
\pgfpathlineto{\pgfqpoint{2.419662in}{1.774644in}}%
\pgfpathlineto{\pgfqpoint{2.500887in}{1.718774in}}%
\pgfpathlineto{\pgfqpoint{2.582112in}{1.658661in}}%
\pgfpathlineto{\pgfqpoint{2.690412in}{1.573580in}}%
\pgfpathlineto{\pgfqpoint{2.798712in}{1.484365in}}%
\pgfpathlineto{\pgfqpoint{2.934087in}{1.368722in}}%
\pgfpathlineto{\pgfqpoint{3.096537in}{1.225745in}}%
\pgfpathlineto{\pgfqpoint{3.313137in}{1.030397in}}%
\pgfpathlineto{\pgfqpoint{3.583887in}{0.781444in}}%
\pgfpathlineto{\pgfqpoint{3.815278in}{0.566020in}}%
\pgfpathlineto{\pgfqpoint{3.815278in}{0.566020in}}%
\pgfusepath{stroke}%
\end{pgfscope}%
\begin{pgfscope}%
\pgfpathrectangle{\pgfqpoint{0.198611in}{0.198611in}}{\pgfqpoint{3.602778in}{2.602778in}}%
\pgfusepath{clip}%
\pgfsetrectcap%
\pgfsetroundjoin%
\pgfsetlinewidth{0.501875pt}%
\definecolor{currentstroke}{rgb}{0.215686,0.494118,0.721569}%
\pgfsetstrokecolor{currentstroke}%
\pgfsetdash{}{0pt}%
\pgfpathmoveto{\pgfqpoint{0.184722in}{2.433980in}}%
\pgfpathlineto{\pgfqpoint{0.524413in}{2.118487in}}%
\pgfpathlineto{\pgfqpoint{0.768088in}{1.895850in}}%
\pgfpathlineto{\pgfqpoint{0.957613in}{1.726186in}}%
\pgfpathlineto{\pgfqpoint{1.120063in}{1.584564in}}%
\pgfpathlineto{\pgfqpoint{1.255438in}{1.470592in}}%
\pgfpathlineto{\pgfqpoint{1.363738in}{1.383273in}}%
\pgfpathlineto{\pgfqpoint{1.444963in}{1.320897in}}%
\pgfpathlineto{\pgfqpoint{1.526188in}{1.262073in}}%
\pgfpathlineto{\pgfqpoint{1.607413in}{1.207893in}}%
\pgfpathlineto{\pgfqpoint{1.661563in}{1.175044in}}%
\pgfpathlineto{\pgfqpoint{1.715713in}{1.145406in}}%
\pgfpathlineto{\pgfqpoint{1.769863in}{1.119563in}}%
\pgfpathlineto{\pgfqpoint{1.824013in}{1.098150in}}%
\pgfpathlineto{\pgfqpoint{1.878163in}{1.081799in}}%
\pgfpathlineto{\pgfqpoint{1.932313in}{1.071076in}}%
\pgfpathlineto{\pgfqpoint{1.986463in}{1.066400in}}%
\pgfpathlineto{\pgfqpoint{2.013537in}{1.066400in}}%
\pgfpathlineto{\pgfqpoint{2.040612in}{1.067964in}}%
\pgfpathlineto{\pgfqpoint{2.094762in}{1.075703in}}%
\pgfpathlineto{\pgfqpoint{2.148912in}{1.089304in}}%
\pgfpathlineto{\pgfqpoint{2.203062in}{1.108263in}}%
\pgfpathlineto{\pgfqpoint{2.257212in}{1.131971in}}%
\pgfpathlineto{\pgfqpoint{2.311362in}{1.159788in}}%
\pgfpathlineto{\pgfqpoint{2.365512in}{1.191101in}}%
\pgfpathlineto{\pgfqpoint{2.419662in}{1.225356in}}%
\pgfpathlineto{\pgfqpoint{2.500887in}{1.281226in}}%
\pgfpathlineto{\pgfqpoint{2.582112in}{1.341339in}}%
\pgfpathlineto{\pgfqpoint{2.690412in}{1.426420in}}%
\pgfpathlineto{\pgfqpoint{2.798712in}{1.515635in}}%
\pgfpathlineto{\pgfqpoint{2.934087in}{1.631278in}}%
\pgfpathlineto{\pgfqpoint{3.096537in}{1.774255in}}%
\pgfpathlineto{\pgfqpoint{3.313137in}{1.969603in}}%
\pgfpathlineto{\pgfqpoint{3.583887in}{2.218556in}}%
\pgfpathlineto{\pgfqpoint{3.815278in}{2.433980in}}%
\pgfpathlineto{\pgfqpoint{3.815278in}{2.433980in}}%
\pgfusepath{stroke}%
\end{pgfscope}%
\begin{pgfscope}%
\pgfpathrectangle{\pgfqpoint{0.198611in}{0.198611in}}{\pgfqpoint{3.602778in}{2.602778in}}%
\pgfusepath{clip}%
\pgfsetbuttcap%
\pgfsetroundjoin%
\pgfsetlinewidth{0.501875pt}%
\definecolor{currentstroke}{rgb}{0.501961,0.501961,0.501961}%
\pgfsetstrokecolor{currentstroke}%
\pgfsetdash{{1.850000pt}{0.800000pt}}{0.000000pt}%
\pgfpathmoveto{\pgfqpoint{1.535234in}{0.184722in}}%
\pgfpathlineto{\pgfqpoint{3.815278in}{2.380971in}}%
\pgfpathlineto{\pgfqpoint{3.815278in}{2.380971in}}%
\pgfusepath{stroke}%
\end{pgfscope}%
\begin{pgfscope}%
\pgfpathrectangle{\pgfqpoint{0.198611in}{0.198611in}}{\pgfqpoint{3.602778in}{2.602778in}}%
\pgfusepath{clip}%
\pgfsetbuttcap%
\pgfsetroundjoin%
\pgfsetlinewidth{0.501875pt}%
\definecolor{currentstroke}{rgb}{0.501961,0.501961,0.501961}%
\pgfsetstrokecolor{currentstroke}%
\pgfsetdash{{1.850000pt}{0.800000pt}}{0.000000pt}%
\pgfpathmoveto{\pgfqpoint{0.184722in}{2.380971in}}%
\pgfpathlineto{\pgfqpoint{2.464766in}{0.184722in}}%
\pgfpathlineto{\pgfqpoint{2.464766in}{0.184722in}}%
\pgfusepath{stroke}%
\end{pgfscope}%
\begin{pgfscope}%
\pgfpathrectangle{\pgfqpoint{0.198611in}{0.198611in}}{\pgfqpoint{3.602778in}{2.602778in}}%
\pgfusepath{clip}%
\pgfsetbuttcap%
\pgfsetroundjoin%
\pgfsetlinewidth{0.501875pt}%
\definecolor{currentstroke}{rgb}{0.501961,0.501961,0.501961}%
\pgfsetstrokecolor{currentstroke}%
\pgfsetdash{{1.850000pt}{0.800000pt}}{0.000000pt}%
\pgfpathmoveto{\pgfqpoint{2.464766in}{2.815278in}}%
\pgfpathlineto{\pgfqpoint{0.184722in}{0.619029in}}%
\pgfpathlineto{\pgfqpoint{0.184722in}{0.619029in}}%
\pgfusepath{stroke}%
\end{pgfscope}%
\begin{pgfscope}%
\pgfpathrectangle{\pgfqpoint{0.198611in}{0.198611in}}{\pgfqpoint{3.602778in}{2.602778in}}%
\pgfusepath{clip}%
\pgfsetbuttcap%
\pgfsetroundjoin%
\pgfsetlinewidth{0.501875pt}%
\definecolor{currentstroke}{rgb}{0.501961,0.501961,0.501961}%
\pgfsetstrokecolor{currentstroke}%
\pgfsetdash{{1.850000pt}{0.800000pt}}{0.000000pt}%
\pgfpathmoveto{\pgfqpoint{3.815278in}{0.619029in}}%
\pgfpathlineto{\pgfqpoint{1.535234in}{2.815278in}}%
\pgfpathlineto{\pgfqpoint{1.535234in}{2.815278in}}%
\pgfusepath{stroke}%
\end{pgfscope}%
\begin{pgfscope}%
\pgfsetrectcap%
\pgfsetmiterjoin%
\pgfsetlinewidth{0.501875pt}%
\definecolor{currentstroke}{rgb}{0.000000,0.000000,0.000000}%
\pgfsetstrokecolor{currentstroke}%
\pgfsetdash{}{0pt}%
\pgfpathmoveto{\pgfqpoint{2.000000in}{0.198611in}}%
\pgfpathlineto{\pgfqpoint{2.000000in}{2.801389in}}%
\pgfusepath{stroke}%
\end{pgfscope}%
\begin{pgfscope}%
\pgfsetrectcap%
\pgfsetmiterjoin%
\pgfsetlinewidth{0.501875pt}%
\definecolor{currentstroke}{rgb}{0.000000,0.000000,0.000000}%
\pgfsetstrokecolor{currentstroke}%
\pgfsetdash{}{0pt}%
\pgfpathmoveto{\pgfqpoint{0.198611in}{0.632407in}}%
\pgfpathlineto{\pgfqpoint{3.801389in}{0.632407in}}%
\pgfusepath{stroke}%
\end{pgfscope}%
\begin{pgfscope}%
\pgfsetroundcap%
\pgfsetroundjoin%
\pgfsetlinewidth{0.501875pt}%
\definecolor{currentstroke}{rgb}{0.000000,0.000000,0.000000}%
\pgfsetstrokecolor{currentstroke}%
\pgfsetdash{}{0pt}%
\pgfpathmoveto{\pgfqpoint{2.000000in}{2.807510in}}%
\pgfpathquadraticcurveto{\pgfqpoint{2.000000in}{2.808331in}}{\pgfqpoint{2.000000in}{2.801389in}}%
\pgfusepath{stroke}%
\end{pgfscope}%
\begin{pgfscope}%
\pgfsetroundcap%
\pgfsetroundjoin%
\pgfsetlinewidth{0.501875pt}%
\definecolor{currentstroke}{rgb}{0.000000,0.000000,0.000000}%
\pgfsetstrokecolor{currentstroke}%
\pgfsetdash{}{0pt}%
\pgfpathmoveto{\pgfqpoint{1.972222in}{2.751954in}}%
\pgfpathlineto{\pgfqpoint{2.000000in}{2.807510in}}%
\pgfpathlineto{\pgfqpoint{2.027778in}{2.751954in}}%
\pgfusepath{stroke}%
\end{pgfscope}%
\begin{pgfscope}%
\pgftext[x=2.000000in,y=2.870833in,,bottom]{\rmfamily\fontsize{10.000000}{12.000000}\selectfont \(\displaystyle \omega^{\prime}\)}%
\end{pgfscope}%
\begin{pgfscope}%
\pgfsetroundcap%
\pgfsetroundjoin%
\pgfsetlinewidth{0.501875pt}%
\definecolor{currentstroke}{rgb}{0.000000,0.000000,0.000000}%
\pgfsetstrokecolor{currentstroke}%
\pgfsetdash{}{0pt}%
\pgfpathmoveto{\pgfqpoint{3.807510in}{0.632407in}}%
\pgfpathquadraticcurveto{\pgfqpoint{3.808332in}{0.632407in}}{\pgfqpoint{3.801389in}{0.632407in}}%
\pgfusepath{stroke}%
\end{pgfscope}%
\begin{pgfscope}%
\pgfsetroundcap%
\pgfsetroundjoin%
\pgfsetlinewidth{0.501875pt}%
\definecolor{currentstroke}{rgb}{0.000000,0.000000,0.000000}%
\pgfsetstrokecolor{currentstroke}%
\pgfsetdash{}{0pt}%
\pgfpathmoveto{\pgfqpoint{3.751955in}{0.660185in}}%
\pgfpathlineto{\pgfqpoint{3.807510in}{0.632407in}}%
\pgfpathlineto{\pgfqpoint{3.751955in}{0.604630in}}%
\pgfusepath{stroke}%
\end{pgfscope}%
\begin{pgfscope}%
\pgftext[x=3.870833in,y=0.632407in,left,]{\rmfamily\fontsize{10.000000}{12.000000}\selectfont \(\displaystyle k^{\prime}\)}%
\end{pgfscope}%
\end{pgfpicture}%
\makeatother%
\endgroup%
} %
        \caption{Das zu berechnende Integral aus \eqref{eq:mass_shell_convolution} visualisiert}
        \label{fig:mass_shell_convolution}
    \end{minipage}\hfill
    \begin{minipage}{0.5\textwidth}
        \centering
        \resizebox{\textwidth}{!}{%% Creator: Matplotlib, PGF backend
%%
%% To include the figure in your LaTeX document, write
%%   \input{<filename>.pgf}
%%
%% Make sure the required packages are loaded in your preamble
%%   \usepackage{pgf}
%%
%% Figures using additional raster images can only be included by \input if
%% they are in the same directory as the main LaTeX file. For loading figures
%% from other directories you can use the `import` package
%%   \usepackage{import}
%% and then include the figures with
%%   \import{<path to file>}{<filename>.pgf}
%%
%% Matplotlib used the following preamble
%%   \usepackage[utf8x]{inputenc}
%%   \usepackage[T1]{fontenc}
%%   \usepackage{amssymb}
%%
\begingroup%
\makeatletter%
\begin{pgfpicture}%
\pgfpathrectangle{\pgfpointorigin}{\pgfqpoint{4.000000in}{3.000000in}}%
\pgfusepath{use as bounding box, clip}%
\begin{pgfscope}%
\pgfsetbuttcap%
\pgfsetmiterjoin%
\definecolor{currentfill}{rgb}{1.000000,1.000000,1.000000}%
\pgfsetfillcolor{currentfill}%
\pgfsetlinewidth{0.000000pt}%
\definecolor{currentstroke}{rgb}{1.000000,1.000000,1.000000}%
\pgfsetstrokecolor{currentstroke}%
\pgfsetdash{}{0pt}%
\pgfpathmoveto{\pgfqpoint{0.000000in}{0.000000in}}%
\pgfpathlineto{\pgfqpoint{4.000000in}{0.000000in}}%
\pgfpathlineto{\pgfqpoint{4.000000in}{3.000000in}}%
\pgfpathlineto{\pgfqpoint{0.000000in}{3.000000in}}%
\pgfpathclose%
\pgfusepath{fill}%
\end{pgfscope}%
\begin{pgfscope}%
\pgfsetbuttcap%
\pgfsetmiterjoin%
\definecolor{currentfill}{rgb}{1.000000,1.000000,1.000000}%
\pgfsetfillcolor{currentfill}%
\pgfsetlinewidth{0.000000pt}%
\definecolor{currentstroke}{rgb}{0.000000,0.000000,0.000000}%
\pgfsetstrokecolor{currentstroke}%
\pgfsetstrokeopacity{0.000000}%
\pgfsetdash{}{0pt}%
\pgfpathmoveto{\pgfqpoint{0.198611in}{0.198611in}}%
\pgfpathlineto{\pgfqpoint{3.801389in}{0.198611in}}%
\pgfpathlineto{\pgfqpoint{3.801389in}{2.801389in}}%
\pgfpathlineto{\pgfqpoint{0.198611in}{2.801389in}}%
\pgfpathclose%
\pgfusepath{fill}%
\end{pgfscope}%
\begin{pgfscope}%
\pgfpathrectangle{\pgfqpoint{0.198611in}{0.198611in}}{\pgfqpoint{3.602778in}{2.602778in}} %
\pgfusepath{clip}%
\pgfsetrectcap%
\pgfsetroundjoin%
\pgfsetlinewidth{1.003750pt}%
\definecolor{currentstroke}{rgb}{0.215686,0.494118,0.721569}%
\pgfsetstrokecolor{currentstroke}%
\pgfsetdash{}{0pt}%
\pgfpathmoveto{\pgfqpoint{0.198611in}{1.239722in}}%
\pgfpathlineto{\pgfqpoint{0.235003in}{1.266013in}}%
\pgfpathlineto{\pgfqpoint{0.271395in}{1.292304in}}%
\pgfpathlineto{\pgfqpoint{0.307786in}{1.318594in}}%
\pgfpathlineto{\pgfqpoint{0.344178in}{1.344885in}}%
\pgfpathlineto{\pgfqpoint{0.380570in}{1.371176in}}%
\pgfpathlineto{\pgfqpoint{0.416961in}{1.397466in}}%
\pgfpathlineto{\pgfqpoint{0.453353in}{1.423757in}}%
\pgfpathlineto{\pgfqpoint{0.489745in}{1.450048in}}%
\pgfpathlineto{\pgfqpoint{0.526136in}{1.476338in}}%
\pgfpathlineto{\pgfqpoint{0.562528in}{1.502629in}}%
\pgfpathlineto{\pgfqpoint{0.598920in}{1.528920in}}%
\pgfpathlineto{\pgfqpoint{0.635311in}{1.555210in}}%
\pgfpathlineto{\pgfqpoint{0.671703in}{1.581501in}}%
\pgfpathlineto{\pgfqpoint{0.708095in}{1.607792in}}%
\pgfpathlineto{\pgfqpoint{0.744487in}{1.634082in}}%
\pgfpathlineto{\pgfqpoint{0.780878in}{1.660373in}}%
\pgfpathlineto{\pgfqpoint{0.817270in}{1.686664in}}%
\pgfpathlineto{\pgfqpoint{0.853662in}{1.712955in}}%
\pgfpathlineto{\pgfqpoint{0.890053in}{1.739245in}}%
\pgfpathlineto{\pgfqpoint{0.926445in}{1.765536in}}%
\pgfpathlineto{\pgfqpoint{0.962837in}{1.791827in}}%
\pgfpathlineto{\pgfqpoint{0.999228in}{1.818117in}}%
\pgfpathlineto{\pgfqpoint{1.035620in}{1.844408in}}%
\pgfpathlineto{\pgfqpoint{1.072012in}{1.870699in}}%
\pgfpathlineto{\pgfqpoint{1.108403in}{1.896989in}}%
\pgfpathlineto{\pgfqpoint{1.144795in}{1.923280in}}%
\pgfpathlineto{\pgfqpoint{1.181187in}{1.949571in}}%
\pgfpathlineto{\pgfqpoint{1.217579in}{1.975861in}}%
\pgfpathlineto{\pgfqpoint{1.253970in}{2.002152in}}%
\pgfpathlineto{\pgfqpoint{1.290362in}{2.028443in}}%
\pgfpathlineto{\pgfqpoint{1.326754in}{2.054733in}}%
\pgfpathlineto{\pgfqpoint{1.363145in}{2.081024in}}%
\pgfpathlineto{\pgfqpoint{1.399537in}{2.107315in}}%
\pgfpathlineto{\pgfqpoint{1.435929in}{2.133605in}}%
\pgfpathlineto{\pgfqpoint{1.472320in}{2.159896in}}%
\pgfpathlineto{\pgfqpoint{1.508712in}{2.186187in}}%
\pgfpathlineto{\pgfqpoint{1.545104in}{2.212478in}}%
\pgfpathlineto{\pgfqpoint{1.581496in}{2.238768in}}%
\pgfpathlineto{\pgfqpoint{1.617887in}{2.265059in}}%
\pgfpathlineto{\pgfqpoint{1.654279in}{2.291350in}}%
\pgfpathlineto{\pgfqpoint{1.690671in}{2.317640in}}%
\pgfpathlineto{\pgfqpoint{1.727062in}{2.343931in}}%
\pgfpathlineto{\pgfqpoint{1.763454in}{2.370222in}}%
\pgfpathlineto{\pgfqpoint{1.799846in}{2.396512in}}%
\pgfpathlineto{\pgfqpoint{1.836237in}{2.422803in}}%
\pgfpathlineto{\pgfqpoint{1.872629in}{2.449094in}}%
\pgfpathlineto{\pgfqpoint{1.909021in}{2.475384in}}%
\pgfpathlineto{\pgfqpoint{1.945412in}{2.501675in}}%
\pgfpathlineto{\pgfqpoint{1.981804in}{2.527966in}}%
\pgfpathlineto{\pgfqpoint{2.018196in}{2.554256in}}%
\pgfpathlineto{\pgfqpoint{2.054588in}{2.580547in}}%
\pgfpathlineto{\pgfqpoint{2.090979in}{2.606838in}}%
\pgfpathlineto{\pgfqpoint{2.127371in}{2.633129in}}%
\pgfpathlineto{\pgfqpoint{2.163763in}{2.659419in}}%
\pgfpathlineto{\pgfqpoint{2.200154in}{2.685710in}}%
\pgfpathlineto{\pgfqpoint{2.236546in}{2.712001in}}%
\pgfpathlineto{\pgfqpoint{2.272938in}{2.738291in}}%
\pgfpathlineto{\pgfqpoint{2.309329in}{2.764582in}}%
\pgfpathlineto{\pgfqpoint{2.345721in}{2.790873in}}%
\pgfpathlineto{\pgfqpoint{2.379503in}{2.815278in}}%
\pgfusepath{stroke}%
\end{pgfscope}%
\begin{pgfscope}%
\pgfpathrectangle{\pgfqpoint{0.198611in}{0.198611in}}{\pgfqpoint{3.602778in}{2.602778in}} %
\pgfusepath{clip}%
\pgfsetrectcap%
\pgfsetroundjoin%
\pgfsetlinewidth{1.003750pt}%
\definecolor{currentstroke}{rgb}{0.215686,0.494118,0.721569}%
\pgfsetstrokecolor{currentstroke}%
\pgfsetdash{}{0pt}%
\pgfpathmoveto{\pgfqpoint{1.620497in}{0.184722in}}%
\pgfpathlineto{\pgfqpoint{1.654279in}{0.209127in}}%
\pgfpathlineto{\pgfqpoint{1.690671in}{0.235418in}}%
\pgfpathlineto{\pgfqpoint{1.727062in}{0.261709in}}%
\pgfpathlineto{\pgfqpoint{1.763454in}{0.287999in}}%
\pgfpathlineto{\pgfqpoint{1.799846in}{0.314290in}}%
\pgfpathlineto{\pgfqpoint{1.836237in}{0.340581in}}%
\pgfpathlineto{\pgfqpoint{1.872629in}{0.366871in}}%
\pgfpathlineto{\pgfqpoint{1.909021in}{0.393162in}}%
\pgfpathlineto{\pgfqpoint{1.945412in}{0.419453in}}%
\pgfpathlineto{\pgfqpoint{1.981804in}{0.445744in}}%
\pgfpathlineto{\pgfqpoint{2.018196in}{0.472034in}}%
\pgfpathlineto{\pgfqpoint{2.054588in}{0.498325in}}%
\pgfpathlineto{\pgfqpoint{2.090979in}{0.524616in}}%
\pgfpathlineto{\pgfqpoint{2.127371in}{0.550906in}}%
\pgfpathlineto{\pgfqpoint{2.163763in}{0.577197in}}%
\pgfpathlineto{\pgfqpoint{2.200154in}{0.603488in}}%
\pgfpathlineto{\pgfqpoint{2.236546in}{0.629778in}}%
\pgfpathlineto{\pgfqpoint{2.272938in}{0.656069in}}%
\pgfpathlineto{\pgfqpoint{2.309329in}{0.682360in}}%
\pgfpathlineto{\pgfqpoint{2.345721in}{0.708650in}}%
\pgfpathlineto{\pgfqpoint{2.382113in}{0.734941in}}%
\pgfpathlineto{\pgfqpoint{2.418504in}{0.761232in}}%
\pgfpathlineto{\pgfqpoint{2.454896in}{0.787522in}}%
\pgfpathlineto{\pgfqpoint{2.491288in}{0.813813in}}%
\pgfpathlineto{\pgfqpoint{2.527680in}{0.840104in}}%
\pgfpathlineto{\pgfqpoint{2.564071in}{0.866395in}}%
\pgfpathlineto{\pgfqpoint{2.600463in}{0.892685in}}%
\pgfpathlineto{\pgfqpoint{2.636855in}{0.918976in}}%
\pgfpathlineto{\pgfqpoint{2.673246in}{0.945267in}}%
\pgfpathlineto{\pgfqpoint{2.709638in}{0.971557in}}%
\pgfpathlineto{\pgfqpoint{2.746030in}{0.997848in}}%
\pgfpathlineto{\pgfqpoint{2.782421in}{1.024139in}}%
\pgfpathlineto{\pgfqpoint{2.818813in}{1.050429in}}%
\pgfpathlineto{\pgfqpoint{2.855205in}{1.076720in}}%
\pgfpathlineto{\pgfqpoint{2.891597in}{1.103011in}}%
\pgfpathlineto{\pgfqpoint{2.927988in}{1.129301in}}%
\pgfpathlineto{\pgfqpoint{2.964380in}{1.155592in}}%
\pgfpathlineto{\pgfqpoint{3.000772in}{1.181883in}}%
\pgfpathlineto{\pgfqpoint{3.037163in}{1.208173in}}%
\pgfpathlineto{\pgfqpoint{3.073555in}{1.234464in}}%
\pgfpathlineto{\pgfqpoint{3.109947in}{1.260755in}}%
\pgfpathlineto{\pgfqpoint{3.146338in}{1.287045in}}%
\pgfpathlineto{\pgfqpoint{3.182730in}{1.313336in}}%
\pgfpathlineto{\pgfqpoint{3.219122in}{1.339627in}}%
\pgfpathlineto{\pgfqpoint{3.255513in}{1.365918in}}%
\pgfpathlineto{\pgfqpoint{3.291905in}{1.392208in}}%
\pgfpathlineto{\pgfqpoint{3.328297in}{1.418499in}}%
\pgfpathlineto{\pgfqpoint{3.364689in}{1.444790in}}%
\pgfpathlineto{\pgfqpoint{3.401080in}{1.471080in}}%
\pgfpathlineto{\pgfqpoint{3.437472in}{1.497371in}}%
\pgfpathlineto{\pgfqpoint{3.473864in}{1.523662in}}%
\pgfpathlineto{\pgfqpoint{3.510255in}{1.549952in}}%
\pgfpathlineto{\pgfqpoint{3.546647in}{1.576243in}}%
\pgfpathlineto{\pgfqpoint{3.583039in}{1.602534in}}%
\pgfpathlineto{\pgfqpoint{3.619430in}{1.628824in}}%
\pgfpathlineto{\pgfqpoint{3.655822in}{1.655115in}}%
\pgfpathlineto{\pgfqpoint{3.692214in}{1.681406in}}%
\pgfpathlineto{\pgfqpoint{3.728605in}{1.707696in}}%
\pgfpathlineto{\pgfqpoint{3.764997in}{1.733987in}}%
\pgfpathlineto{\pgfqpoint{3.801389in}{1.760278in}}%
\pgfusepath{stroke}%
\end{pgfscope}%
\begin{pgfscope}%
\pgfpathrectangle{\pgfqpoint{0.198611in}{0.198611in}}{\pgfqpoint{3.602778in}{2.602778in}} %
\pgfusepath{clip}%
\pgfsetrectcap%
\pgfsetroundjoin%
\pgfsetlinewidth{1.003750pt}%
\definecolor{currentstroke}{rgb}{0.894118,0.101961,0.109804}%
\pgfsetstrokecolor{currentstroke}%
\pgfsetdash{}{0pt}%
\pgfpathmoveto{\pgfqpoint{1.620497in}{2.815278in}}%
\pgfpathlineto{\pgfqpoint{1.654279in}{2.790873in}}%
\pgfpathlineto{\pgfqpoint{1.690671in}{2.764582in}}%
\pgfpathlineto{\pgfqpoint{1.727062in}{2.738291in}}%
\pgfpathlineto{\pgfqpoint{1.763454in}{2.712001in}}%
\pgfpathlineto{\pgfqpoint{1.799846in}{2.685710in}}%
\pgfpathlineto{\pgfqpoint{1.836237in}{2.659419in}}%
\pgfpathlineto{\pgfqpoint{1.872629in}{2.633129in}}%
\pgfpathlineto{\pgfqpoint{1.909021in}{2.606838in}}%
\pgfpathlineto{\pgfqpoint{1.945412in}{2.580547in}}%
\pgfpathlineto{\pgfqpoint{1.981804in}{2.554256in}}%
\pgfpathlineto{\pgfqpoint{2.018196in}{2.527966in}}%
\pgfpathlineto{\pgfqpoint{2.054588in}{2.501675in}}%
\pgfpathlineto{\pgfqpoint{2.090979in}{2.475384in}}%
\pgfpathlineto{\pgfqpoint{2.127371in}{2.449094in}}%
\pgfpathlineto{\pgfqpoint{2.163763in}{2.422803in}}%
\pgfpathlineto{\pgfqpoint{2.200154in}{2.396512in}}%
\pgfpathlineto{\pgfqpoint{2.236546in}{2.370222in}}%
\pgfpathlineto{\pgfqpoint{2.272938in}{2.343931in}}%
\pgfpathlineto{\pgfqpoint{2.309329in}{2.317640in}}%
\pgfpathlineto{\pgfqpoint{2.345721in}{2.291350in}}%
\pgfpathlineto{\pgfqpoint{2.382113in}{2.265059in}}%
\pgfpathlineto{\pgfqpoint{2.418504in}{2.238768in}}%
\pgfpathlineto{\pgfqpoint{2.454896in}{2.212478in}}%
\pgfpathlineto{\pgfqpoint{2.491288in}{2.186187in}}%
\pgfpathlineto{\pgfqpoint{2.527680in}{2.159896in}}%
\pgfpathlineto{\pgfqpoint{2.564071in}{2.133605in}}%
\pgfpathlineto{\pgfqpoint{2.600463in}{2.107315in}}%
\pgfpathlineto{\pgfqpoint{2.636855in}{2.081024in}}%
\pgfpathlineto{\pgfqpoint{2.673246in}{2.054733in}}%
\pgfpathlineto{\pgfqpoint{2.709638in}{2.028443in}}%
\pgfpathlineto{\pgfqpoint{2.746030in}{2.002152in}}%
\pgfpathlineto{\pgfqpoint{2.782421in}{1.975861in}}%
\pgfpathlineto{\pgfqpoint{2.818813in}{1.949571in}}%
\pgfpathlineto{\pgfqpoint{2.855205in}{1.923280in}}%
\pgfpathlineto{\pgfqpoint{2.891597in}{1.896989in}}%
\pgfpathlineto{\pgfqpoint{2.927988in}{1.870699in}}%
\pgfpathlineto{\pgfqpoint{2.964380in}{1.844408in}}%
\pgfpathlineto{\pgfqpoint{3.000772in}{1.818117in}}%
\pgfpathlineto{\pgfqpoint{3.037163in}{1.791827in}}%
\pgfpathlineto{\pgfqpoint{3.073555in}{1.765536in}}%
\pgfpathlineto{\pgfqpoint{3.109947in}{1.739245in}}%
\pgfpathlineto{\pgfqpoint{3.146338in}{1.712955in}}%
\pgfpathlineto{\pgfqpoint{3.182730in}{1.686664in}}%
\pgfpathlineto{\pgfqpoint{3.219122in}{1.660373in}}%
\pgfpathlineto{\pgfqpoint{3.255513in}{1.634082in}}%
\pgfpathlineto{\pgfqpoint{3.291905in}{1.607792in}}%
\pgfpathlineto{\pgfqpoint{3.328297in}{1.581501in}}%
\pgfpathlineto{\pgfqpoint{3.364689in}{1.555210in}}%
\pgfpathlineto{\pgfqpoint{3.401080in}{1.528920in}}%
\pgfpathlineto{\pgfqpoint{3.437472in}{1.502629in}}%
\pgfpathlineto{\pgfqpoint{3.473864in}{1.476338in}}%
\pgfpathlineto{\pgfqpoint{3.510255in}{1.450048in}}%
\pgfpathlineto{\pgfqpoint{3.546647in}{1.423757in}}%
\pgfpathlineto{\pgfqpoint{3.583039in}{1.397466in}}%
\pgfpathlineto{\pgfqpoint{3.619430in}{1.371176in}}%
\pgfpathlineto{\pgfqpoint{3.655822in}{1.344885in}}%
\pgfpathlineto{\pgfqpoint{3.692214in}{1.318594in}}%
\pgfpathlineto{\pgfqpoint{3.728605in}{1.292304in}}%
\pgfpathlineto{\pgfqpoint{3.764997in}{1.266013in}}%
\pgfpathlineto{\pgfqpoint{3.801389in}{1.239722in}}%
\pgfusepath{stroke}%
\end{pgfscope}%
\begin{pgfscope}%
\pgfpathrectangle{\pgfqpoint{0.198611in}{0.198611in}}{\pgfqpoint{3.602778in}{2.602778in}} %
\pgfusepath{clip}%
\pgfsetrectcap%
\pgfsetroundjoin%
\pgfsetlinewidth{1.003750pt}%
\definecolor{currentstroke}{rgb}{0.894118,0.101961,0.109804}%
\pgfsetstrokecolor{currentstroke}%
\pgfsetdash{}{0pt}%
\pgfpathmoveto{\pgfqpoint{0.198611in}{1.760278in}}%
\pgfpathlineto{\pgfqpoint{0.235003in}{1.733987in}}%
\pgfpathlineto{\pgfqpoint{0.271395in}{1.707696in}}%
\pgfpathlineto{\pgfqpoint{0.307786in}{1.681406in}}%
\pgfpathlineto{\pgfqpoint{0.344178in}{1.655115in}}%
\pgfpathlineto{\pgfqpoint{0.380570in}{1.628824in}}%
\pgfpathlineto{\pgfqpoint{0.416961in}{1.602534in}}%
\pgfpathlineto{\pgfqpoint{0.453353in}{1.576243in}}%
\pgfpathlineto{\pgfqpoint{0.489745in}{1.549952in}}%
\pgfpathlineto{\pgfqpoint{0.526136in}{1.523662in}}%
\pgfpathlineto{\pgfqpoint{0.562528in}{1.497371in}}%
\pgfpathlineto{\pgfqpoint{0.598920in}{1.471080in}}%
\pgfpathlineto{\pgfqpoint{0.635311in}{1.444790in}}%
\pgfpathlineto{\pgfqpoint{0.671703in}{1.418499in}}%
\pgfpathlineto{\pgfqpoint{0.708095in}{1.392208in}}%
\pgfpathlineto{\pgfqpoint{0.744487in}{1.365918in}}%
\pgfpathlineto{\pgfqpoint{0.780878in}{1.339627in}}%
\pgfpathlineto{\pgfqpoint{0.817270in}{1.313336in}}%
\pgfpathlineto{\pgfqpoint{0.853662in}{1.287045in}}%
\pgfpathlineto{\pgfqpoint{0.890053in}{1.260755in}}%
\pgfpathlineto{\pgfqpoint{0.926445in}{1.234464in}}%
\pgfpathlineto{\pgfqpoint{0.962837in}{1.208173in}}%
\pgfpathlineto{\pgfqpoint{0.999228in}{1.181883in}}%
\pgfpathlineto{\pgfqpoint{1.035620in}{1.155592in}}%
\pgfpathlineto{\pgfqpoint{1.072012in}{1.129301in}}%
\pgfpathlineto{\pgfqpoint{1.108403in}{1.103011in}}%
\pgfpathlineto{\pgfqpoint{1.144795in}{1.076720in}}%
\pgfpathlineto{\pgfqpoint{1.181187in}{1.050429in}}%
\pgfpathlineto{\pgfqpoint{1.217579in}{1.024139in}}%
\pgfpathlineto{\pgfqpoint{1.253970in}{0.997848in}}%
\pgfpathlineto{\pgfqpoint{1.290362in}{0.971557in}}%
\pgfpathlineto{\pgfqpoint{1.326754in}{0.945267in}}%
\pgfpathlineto{\pgfqpoint{1.363145in}{0.918976in}}%
\pgfpathlineto{\pgfqpoint{1.399537in}{0.892685in}}%
\pgfpathlineto{\pgfqpoint{1.435929in}{0.866395in}}%
\pgfpathlineto{\pgfqpoint{1.472320in}{0.840104in}}%
\pgfpathlineto{\pgfqpoint{1.508712in}{0.813813in}}%
\pgfpathlineto{\pgfqpoint{1.545104in}{0.787522in}}%
\pgfpathlineto{\pgfqpoint{1.581496in}{0.761232in}}%
\pgfpathlineto{\pgfqpoint{1.617887in}{0.734941in}}%
\pgfpathlineto{\pgfqpoint{1.654279in}{0.708650in}}%
\pgfpathlineto{\pgfqpoint{1.690671in}{0.682360in}}%
\pgfpathlineto{\pgfqpoint{1.727062in}{0.656069in}}%
\pgfpathlineto{\pgfqpoint{1.763454in}{0.629778in}}%
\pgfpathlineto{\pgfqpoint{1.799846in}{0.603488in}}%
\pgfpathlineto{\pgfqpoint{1.836237in}{0.577197in}}%
\pgfpathlineto{\pgfqpoint{1.872629in}{0.550906in}}%
\pgfpathlineto{\pgfqpoint{1.909021in}{0.524616in}}%
\pgfpathlineto{\pgfqpoint{1.945412in}{0.498325in}}%
\pgfpathlineto{\pgfqpoint{1.981804in}{0.472034in}}%
\pgfpathlineto{\pgfqpoint{2.018196in}{0.445744in}}%
\pgfpathlineto{\pgfqpoint{2.054588in}{0.419453in}}%
\pgfpathlineto{\pgfqpoint{2.090979in}{0.393162in}}%
\pgfpathlineto{\pgfqpoint{2.127371in}{0.366871in}}%
\pgfpathlineto{\pgfqpoint{2.163763in}{0.340581in}}%
\pgfpathlineto{\pgfqpoint{2.200154in}{0.314290in}}%
\pgfpathlineto{\pgfqpoint{2.236546in}{0.287999in}}%
\pgfpathlineto{\pgfqpoint{2.272938in}{0.261709in}}%
\pgfpathlineto{\pgfqpoint{2.309329in}{0.235418in}}%
\pgfpathlineto{\pgfqpoint{2.345721in}{0.209127in}}%
\pgfpathlineto{\pgfqpoint{2.379503in}{0.184722in}}%
\pgfusepath{stroke}%
\end{pgfscope}%
\begin{pgfscope}%
\pgfpathrectangle{\pgfqpoint{0.198611in}{0.198611in}}{\pgfqpoint{3.602778in}{2.602778in}} %
\pgfusepath{clip}%
\pgfsetrectcap%
\pgfsetroundjoin%
\pgfsetlinewidth{0.501875pt}%
\definecolor{currentstroke}{rgb}{0.501961,0.501961,0.501961}%
\pgfsetstrokecolor{currentstroke}%
\pgfsetdash{}{0pt}%
\pgfpathmoveto{\pgfqpoint{0.955194in}{1.213694in}}%
\pgfpathlineto{\pgfqpoint{2.000000in}{2.541111in}}%
\pgfusepath{stroke}%
\end{pgfscope}%
\begin{pgfscope}%
\pgfpathrectangle{\pgfqpoint{0.198611in}{0.198611in}}{\pgfqpoint{3.602778in}{2.602778in}} %
\pgfusepath{clip}%
\pgfsetrectcap%
\pgfsetroundjoin%
\pgfsetlinewidth{0.501875pt}%
\definecolor{currentstroke}{rgb}{0.501961,0.501961,0.501961}%
\pgfsetstrokecolor{currentstroke}%
\pgfsetdash{}{0pt}%
\pgfpathmoveto{\pgfqpoint{2.000000in}{0.458889in}}%
\pgfpathlineto{\pgfqpoint{3.441111in}{0.458889in}}%
\pgfusepath{stroke}%
\end{pgfscope}%
\begin{pgfscope}%
\pgfpathrectangle{\pgfqpoint{0.198611in}{0.198611in}}{\pgfqpoint{3.602778in}{2.602778in}} %
\pgfusepath{clip}%
\pgfsetrectcap%
\pgfsetroundjoin%
\pgfsetlinewidth{0.501875pt}%
\definecolor{currentstroke}{rgb}{0.501961,0.501961,0.501961}%
\pgfsetstrokecolor{currentstroke}%
\pgfsetdash{}{0pt}%
\pgfpathmoveto{\pgfqpoint{3.441111in}{0.458889in}}%
\pgfpathlineto{\pgfqpoint{3.441111in}{1.500000in}}%
\pgfusepath{stroke}%
\end{pgfscope}%
\begin{pgfscope}%
\pgftext[x=2.720556in,y=0.289708in,left,base]{\rmfamily\fontsize{10.000000}{12.000000}\selectfont d\(\displaystyle k^\prime\)}%
\end{pgfscope}%
\begin{pgfscope}%
\pgftext[x=3.477139in,y=0.914375in,left,base]{\rmfamily\fontsize{10.000000}{12.000000}\selectfont d\(\displaystyle \omega^\prime\)}%
\end{pgfscope}%
\begin{pgfscope}%
\pgftext[x=2.180139in,y=0.497931in,left,base]{\rmfamily\fontsize{10.000000}{12.000000}\selectfont \(\displaystyle \alpha\)}%
\end{pgfscope}%
\begin{pgfscope}%
\pgftext[x=0.703000in,y=1.460958in,left,base]{\rmfamily\fontsize{10.000000}{12.000000}\selectfont \(\displaystyle 2 \alpha\)}%
\end{pgfscope}%
\begin{pgfscope}%
\pgftext[x=1.189375in,y=2.085625in,left,base]{\rmfamily\fontsize{10.000000}{12.000000}\selectfont \(\displaystyle l\)}%
\end{pgfscope}%
\begin{pgfscope}%
\pgftext[x=1.549653in,y=1.825347in,left,base]{\rmfamily\fontsize{10.000000}{12.000000}\selectfont \(\displaystyle h\)}%
\end{pgfscope}%
\end{pgfpicture}%
\makeatother%
\endgroup%
}
        \caption{Die Kreuzungstelle bei $k_{0+}'$ von ganz nah angeschaut}
        \label{fig:schulgeometrie}
    \end{minipage}
\end{figure}

\begin{equation}
    \rwhat{\Delta_m^{* 2}} (\omega, k)
    = \int \theta(\omega') \delta(\omega'^2-k'^2-m^2)\theta(\omega-\omega')
      \delta((\omega-\omega')^2-(k-k')^2-m^2) \d \omega' \d k'
\label{eq:mass_shell_convolution}
\end{equation}

An Abbildung \ref{fig:mass_shell_convolution} sehen wir schon, dass das Faltungsintegral nur dann ungleich null ist, wenn $(\omega, k)$ in der 2$m$-Massenschale liegen. Es ist also insbesondere $\omega > 0$.
Da $\Delta_m$ Poincare-invariant ist, sind $\Delta_m^2$ und $\rwhat{\Delta_m^{*2}}$ es auch. Es genügt also $\rwhat{\Delta_m^{*2}}$ für $k=0$ und positive $\omega$ zu berechnen. Alle anderen Werte holen wir uns dann aus der Poincare-Invarianz.

\todo{Wie erklärt man das besser, ohne an Anschaulichkeit oder Rigorosität zu verlieren}
Um nun das Integral über zwei sich schneidende lineare\footnote{Linear in dem Sinne, dass die Distribution entlang einer Linie getragen ist. Nicht das es eine lineare Distribution ist} $\delta$-Distributionen zu berechnen bedienen wir uns eines Physikertricks und stellen uns die $\delta$-Distribution als Grenzwert einer $\frac{1}{h}$-hohen und $h$ breiten Rechtecksfunktion vor.
% section die_wellenfrontmenge_von_delta_m_2_ (end)
